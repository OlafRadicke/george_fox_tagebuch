
% \picinclude{./060-069/p_s060.jpg} 
wurden gewonnen und haben noch jetzt große Versammlungen von
Freunden in der Nähe von Sedbergh, die ich damals zuerst zu-
sammen sammelte im Namen Jesu.
GS fand ein großer Jahrmarkt statt, an welchem man pflegte
Dienstboten zu dingen; ich verkündete den Tag dee Herrn. Nach-
dem ich dietz getan, ging ich auf den Platz des Turmhausetz, und
viele Leute kamen vom Jahrmarkt zu mir und eine Menge Priester
und »Fwmme«. Da verkündete ich die ewige Wahrheit dez
Herrn und das Wort dez Lebenö während mehrerer Stunden
und zeigte, daß der Herr gekommen sei, sein Volk selbst zu lehren
und es abzubringen von den Wegen dieser Welt und ihren Lehrern,
zu Ehristuö dem wahren Lehrer und wahren Weg. Jch machte
ihnen klar, wie ihre Lehrer denen gleich seien, die von jeher
von den Propheten, von Ehristuß und den Aposteln verdammt.
worden sind. Jch ermahnte alle von ihren mit Händen gemachten
Tempeln abzulassen und auf den Empfang dez Geistes zu warten,
damit sie erkennen könnten, daß sie der Tempel Gottes seien.
Nicht ein einziger von den Priestern hatte Macht, seinen Mund
auszutun gegen daß, waß ich verkündete; zuletzt sagte einer von
der Wache: ,,Warum geht ihr nicht in die ,,Kirche«? hier ist
kein geeigneter Platz zum Predigen«. Jch sagte ihm, ich leugne
ihre Kirche. Da erhob sich Francis Howgill,1) Prediger einer
Gemeinschaft. Er hatte mich nie vorher gesehen, aber er unternahm
es, diesem Hauptmann zu antworten und brachte ihn bald zum
Schweigen; und von mir sagte er: ,,dieser predigt gewaltig und
nicht wie die Schriftgelehrten (Matth. 7, 29)«. Jch erklärte da-
rauf den Leuten, daß dieser Boden hier nicht heiliger sei alö an
einem andern Ort und daß nicht dieses Haus die Kirche sei,
sondern die Gemeinde, deren Haupt Christutz ist. Bald nachher
kamen dann einige Priester zu mir und ich ermahnte sie, Buße zu
tun. Einer von ihnen sagte, ichsei verrückt, und wandte sich von
mir ab; aber manche wurden gewonnen an dem Tage und freuten
sich über die Verkündigung der Wahrheit und nahmen sie mit
Freuden auf. Einer unter ihnen, Hauptmann Ward, nahm die
Wahrheit in Liebe auf und lebte darin bis zu seinem Tode.. . .
Von da ging ich nach Unterbarrow, zu einem namenß Miles
Bateman ..... Am Morgen ging ich auß . . und als ich in
1) Franeiö Hotogill, später ein eisriger Qnäkerprediger (s.Wein-
garten a. a. O.)


% \picinclude{./060-069/p_s061.jpg} 
Christns in uns. Erkenntnis der Quäkerischen Weltmission usw. 61
der Nähe auf einem Hügel hin und her ging, sah ich einige
Reisende, welche um Unterstützung baten, und ich sah, daß sie
eß nötig hatten; aber man gab ihnen nichts und sagte ihnen, sie
seien—Strolche. GS betrübte mich solche Hartherzigkeit unter den
,,Frommen« zu sehen, und ale sie alle beim Frühstück saßen, lies
ich den Reisenden etwa eine Viertelmeile nach und gab ihnen
etwaß Geld. A16 nun einige von den andern aus dem Hause
kamen und sahen, daß ich eine Viertelmeile weg war, sagten sie, ich
hätte nicht so weit kommen können, wenn ich nicht Flügel hätte.
Daraufhin war es nahe daran, daß man die Versammlung ab-
sagte; denn man hatte eine so merkwürdige Meinung von mir
bekommen, daß viele nicht eine Versammlung mit mir haben
wollten. Jch sagte ihnen, ich sei jenen armen Reisenden nachge-
laufen, um ihnen etwaß Geld zu geben, weil mich die Hart-
herzigkeit, mit der man sie fortgeschickt, betrübt habe ....
Von da tzging ich nach Ulverstone und Swarthmore zu
Richter Fell; es kam auch einer, Priester Lampitt, der behauptete,
Eingebungen zu haben- Jch redete lange mit ihm, denn er sprach
von wichtigen Eingebungen und von Vollkommenheit und blen-
dete die Leute dadurch. Gr hätte mich gerne gewähren lassen,
aber ich konnte ihn nicht gewähren lassen, weil er so unlauter
war. Gr sagte, er sei mehr als Johanneß, und tat, alz ob er
alle Dinge wüßte. Jch sagte ihm, der Tod habe von Adam biz
Moses regiert (Röm. 5, 14); und weil er tot sei, kenne er Moses
nicht, denn Moseß habe daß Paradies- Gotteß gesehen; er aber
kenne weder Moseß noch die Propheten noch Johannes. Denn
die höckerichte und rauhe Natur war noch in ihm, und der Berg
der Sünde und deö Verderbens, und der Weg für den Herrn
war nicht bereitet in ihm (Jes. 40). Er bekannte, er sei in großer
Trübsal gewesen, beteuerte aber, nun könne er Psalmen fmgen
und alleö machen, maß; man von ihm verlange. Ich sagte ihm,
er gehöre zum Diebögesindel, aber Moses und die Propheten
und Christuö predigen, das- könne er nicht; dazu müßte er den
gleichen Geist haben wie jene. Margaret Fell war den ganzen
Tag nicht zu Hause gewesen; am Abend erzählten ihr ihre Kinder,
daß Priester Lampitt und ich gestritten hätten; die-J betrübte sie,
weil er dem gleichen Bekenntnis angehörte wie sie; aber er
oerbarg sein schmutzigeß Treiben vor ihnen.
Wir sprachen noch lange miteinander am Abend, und ich ver-


% \picinclude{./060-069/p_s062.jpg} 
kiindete ihr und ihrer Familie die Wahrheit. Am folgenden Tage
kam Lampitt wieder, und ich redete lange mit ihm, und Margaret
Fell, die ihn jetzt ganz durchschaute, war dabei. Eine Über-
zeugung der Wahrheit kam über sie und die Jhrigen. Als bald
darauf ein allgemeiner Bußtag abgehalten werden sollte, bat sie
mich, mit ihr ine Turmhaus von Ulverstone zu kommen, denn
sie hatte sich noch nicht gänzlich davon lo?-gemacht. Jch erwiderte
ihr: ,,Jch muß tun, wie mich der Herr heißt.« Jch verließ sie
und ging int? Freie und daß- Wort dee Herrn geschah also zu
mir: ,,Gehe ihnen nach inß Turmhau?-.« Als ich kam, sang
Lampitt gerade mit den Leuten; aber sein Geist war so unlanter,
und waß sie sangen, paßte so wenig für ihr Bedürfnis, daß, als-
sie fertig gesungen hatten, der Herr mich trieb, also zu ihnen zu
reden: ,,Der ist nicht ein Jude, der etz äußerlich ist, sondern der
ist ein Jude, der innerlich einer ist, in seinem Leben, daß er nicht
vor den Menschen, sondern vor Gott führt« (Röm. 2, 28.29).
Dann zeigte ich ihnen nach des Herrn weiterer Offenbarung, daß
Gott gekommen sei, sein Volk zu lehren (1.Joh.2, 27), (Joh.1-i,26).
durch seinen Geist und sie abzubringen von allen ihren früheren
Gebräuchen, ihren Bekenntnissen, Kirchen und Gotte?-diensten;
denn daß alleö seien nur Menschensatzungen; daß Leben und den
Geist, auz dem diese Satzungen entstanden, die hätten sie doch
nicht. Da rief Friedenörichter Sawrey: ,,Fort mit ihm!« Aber
Richter Fell?. Frau sagte zu den Beamten: ,,Laßt ihn gehen;
warum soll er nicht so gut reden wie ein anderer?« Auch Lampitt,
der Betrüger, sagte, man solle mich reden lassen. Aber alö ich eine
Zeitlang geredet hatte, ließ mich Friedenßrichter Sawrey hinaus
bringen durch die Konstabler; da redete ich auf dem Kirchhof
weiter . . . Jch ging nun nach Becliff . . . und andere Orte . . .
Bald darauf, alß Richter Fell nach Hause kam, ließ Margaret
Fell mich holen und ließ mir sagen, ich solle doch zu ihnen
kommen; ich fühlte die Freiheit vom Herrn, etz zu tun und ging
hin. Ich sah, daß die Priester und die ,,Frommen« und der
Frieden?-richter Sawrey, Richter Fell und Hauptmann Sande durch
ihre Lügen gegen die Wahrheit eingenommen hatten, aber alß ich
kam und mit ihnen redete, gelang ez mir, alle ihre Einwände zu
widerlegen, und ich überzeugte Hauptmann Sands an Hand der
Schrift so völlig, daß er ganz befestigt war in seiner Uberzeugung.
Nach einigem Hin- und Herreden war Richter Fell ebenfalltz zu-


% \picinclude{./060-069/p_s063.jpg} 
Chrisiuß in unß. Erkenntnis der Quitkerischen Weltmission usko. 63
frieden gestellt und gelangte dazu, durch daß, waß ihm der Geist
Gotteß eröffnet hatte, etwaß Höhereß zu erkennen, alß waß die
weltlichen Priester und Lehrer lehrten, und ging nicht mehr hin,
sie zu hören alle die Jahre biß zu seinem Tode; denn er wußte
nun, daß daß, waß ich lehrte, die Wahrheit sei, und daß Ehristuß
der Lehrer seineß Volkeß ist und sein Heiland . . . Während ich
in dieser Gegend war, kamen Richard Farnßworth und Jameß
Naylor, mich und die anderen zu sehen, und weil Richter Fell
nun darüber beruhigt war, daß eß die Wahrheit sei, die ich ver-
kündige, so erlaubte er mir, Versammlungen in seinem Hause zu
haben trotz aller Einwände; und eß wurde eine große Versamm-
lung eingerichtet, die fast vierzig Jahre, biß 1690, bestand, so
daß ein neueß Versammlungßhauß in der Nähe gebaut wurde . . .
Jch hörte von einer großen Versammlung, die in Uloerstone
stattfinden sollte, und ging darum dorthin und begab mich inß
Turmhauß, in der Furcht und der Kraft Gottes-. Alß der Priester
geendet hatte, redete ich daß Wort deß Herrn zu ihnen, daß wie ein
Hammer und ein Feuer unter ihnen wirkte (Jer. 23, 29). Lampitt,
der Priester des Orteß, war mit den meisten andern Priestern
uneinß gewesen vorher, nun aber taten sie sich alle zusammen
gegen die Wahrheit. Aber die mächtige Kraft deß Herrn war
über allem und tat sich so herrlich kund, daß Priester Bennett
sagte: »Die Kirche erbebt!« und sich fiirchtete und zitterte. Und
nachdem er einige unverständliche Worte geredet, eilte er hinauß,
auß Furcht, sie möchte über seinem Kopf zusammenstürzen. Viele
Priester versammelten sich, aber sie hatten noch keine Macht, Ver-
folgungen zu veranstalten.
Alß ich nun hier fertig war, ging ich wieder nach Swarth-
more, wohin vier oder siins Priester kamen; im Gespräch mit
ihnen fragte ich, ob einer unter ihnen sei, der sagen könne, daß
Wort deß Herm: ,,Gehe hin und rede zu den oder jenen«, sei je
einmal an ihn ergangen? Keiner wagte, eß von sich zu be-
haupten. Aber einer von ihnen wurde zornig und sagte, er
könne von Erfahrungen so gut berichten wie ich. Ich erwiderte
ihm, Erfahrungen seien allerdingß etwas, aber eine Botschaft
erhalten und damit außziehen, ein Wort vom Herm haben und
verkünden wie die Apostel und Propheten und wie ich, wenn
ich unter ihnen predige, daß sei noch etwaß andereß. Und ich
fragte sie darum noch einmal, ob einer unter ihnen sagen könne,


% \picinclude{./060-069/p_s064.jpg} 
er habe irgend einmal einen Vefehl unmittelbar vom Herrn
empfangen; aber eß konnte eß keiner. Da erklärte ich ihnen, daß
seien falsche Propheten und falsche Apostel und Antichristen,,die
die Worte der wahren Propheten und wahren Apostel und Christi
gebrauchen und die Erfahrungen anderer verwenden und selber
nie eine Stimme Gotteß oder Christi vernommen haben. Solche
wie sie könnten eben bloß die Erfahrungen un.d Worte anderer
vernehmen. Daß verwirrte sie sehr und stellte sie bloß. Ein
andermal im Gespräch mit einigen Priestern im Hause Richter
Fellö und in dessen Anwesenheit stellte ich die gleiche Frage und
siigte hinzu: daß könne eben jeder, der lesen könne, die Erfah-
rungen der Propheten rmd Apostel verkünden, die in der Schrift
aufgezeichnet seien. Hierauf bekannte ein alter Priester, Thomas
Taylor, dem Richter Fell ehrlich: er habe nie die Stimme Gotte-5
oder Christi vernommen, die ihn irgendwohin gesandt habe; er
rede von seinen eigenen Erfahrungen und den Erfahrungen der
Heiligen früherer Zeiten, und dies predige er. Solcheß bestärlte
Richter Fell in der Überzeugung, daß die Priester im Jrrtum
seien. Denn er hatte vorher, wie die meisten Leute damaltz,
geglaubt, sie seien von Gott gesandt.
Zu dieser Zeit wurde Thoma-8 Taylor 1) bekehrt und durch-
reiste init mir Westmorland ..... Die Priester wurden immer
aufgebrachter gegen unß und verfolgten un?-, wo sie nur konnten.
Jameß Naylor und Franciß Howgill wurden inß Gefängnis ge-
worfen . . . Aber dem Herrn sei Lob, die Wahrheit breitete sich
immer mehr aus. Denn um diese Zeit spürten sich John Aud-
land, John Camm, Edward Burroughi), Richard Hubberthorn 9)
1) Thomar Taylor hatte in Oxford studiert und war Puritanerprediget
geworden. Dann, weil er nicht mehr wollte ,,11m Lohn predigen--, schloß er
sich den Quakern an und wirkte eifrig als Prediger und durch Schristen.
2) Edward Burtough, ursprünglich Prediger der Epiekopalkirche, war
aus dieser ausgetreten und hatte sich den Preöbyterianetn angeschlossen; nach
einigen Untertedungen mit Fox bekehrte er sich sodann zum Quäkertnm, dessen
eifriges tätigeö Glied er blieb, bis er 1662 für seinen Glauben im Kerker, wohin
man ihn auz einer Versammlung gebracht hatte, starb.
3) Richard Hubberthorn, eine bescheidene, friedliche Natur, kriinklich und
mit einer schwachen Stimme, arbeitete dennoch Großez als Prediger. 1662 wurde
er ebenfalls aus einer Versammlung in den Kerker geschleppt, wo sein schwacher
Körper bald den Entbehrungen erlag; doch pries er noch auf seinem Toteubeit
die Güte Gottes.


% \picinclude{./060-069/p_s065.jpg} 
Christus in uns. Erkenntnis der Quüterischen Weltmission usw. 65
und andere mit der Kraft von oben ausgerüstet und traten auf
als Prediger und erwiesen sich alß treue Arbeiter, die umher-
zogen und daß Evangelium umsonst predigten, wodurch Tausende
bekehrt wurden und von nun an dem Herrn gehörten ....
Nachdem ich die Freunde in Westmorland besucht hatte, ging
ich wieder nach Uloerstone, wo Priester Lampitt war. Dieser
hatte selber gepredigt, man müsse sich von Gott lehren lassen;
und alle Menschen, Männer und Frauen, können dazu kommen,
daß Evangelium zu predigen. A18 ez sich aber darum handelte,
dieZ mit der Tat zu beweisen, so verfolgte er sowohl diese Lehre
als die Lehrer .... Als nun die Versammlungen ihren Anfang
nahmen und wir in einer Privatwohnung zusammenkamen, wurde
Lampitt sehr aufgebracht und sagte, wir verlassen den Tempel
und gehen in den Götzentempel Jerobeamö, so daß viele der
,,Frommen« sahen, wie wenig er mit dem, maß er gepredigt hatte,
Ernst machte. Nun legte man ihnen die Sache mit dem Götzen-
tempel Jerobeamz C1. Könige 13), auz und zeigte ihnen, daß
eher ihre Häuser, die sie Kirchen nennen, die Götzentempel Jero-
beamß seien und die alten Meßhäuser, die daß finstere Papsttum
eingesetzt und die jene, die sich Protestanten nennen und meinen,
sie seien ausgeklärter als- die Päpstlichen, auch noch festhalten,
obgleich doch Gott sie nie angeordnet. Und der Tempel, den
Gott in Jerusalem eingesetzt, dem habe Ehristuö. eine Ende ge-
macht, und die, welche ihn aufnahmen und an ihn glaubten, deren
Leiber wurden Tempel Gottes, in denen Christus- und der heilige
Geist wohnen (1. Cor. 6, 19T Diese versammelten sich ....
und kamen zusammen in ihren Wohnhäusern, die dann nicht
Tempel genannt wurden oder Kirchen, sondern ihre Leiber waren
i die Tempel und die Gläubigen die Kirche, deren Haupt Ehristutz
ist .... Daraus trieb ee mich ins Turmhauß zu gehen, two Priester
und ,,Fromme« und viel Volk versammelt waren. Jch stand in der
Nähe von Priester Lampitt, der daraus loß wütete in seiner
Predigt. Nachdem der Herr meinen Mund aufgetan, daß ich
reden sollte, kam Friedenßrichter John Sawrey zu mir und sagte,
wenn ich mich an das halten wolle, waz in der Schrift stehe,
so könne ich reden. Jch wunderte mich über diese Rede und sagte,
ich werde mich sicher an die Schrift halten und sie vorweisen,
um daß Gesagte zu begründen; denn ich hätte ihm und Lampitt
etwas zu sagen. Hierauf sagte er wieder, ich solle nicht reden,
George Fox. Z


% \picinclude{./060-069/p_s066.jpg} 
und widersprach sich damit selber, nachdem er ja eben gesagt
hatte, ich solle reden, wenn ich mich an die Schrift halten wolle.
Die Leute waren ruhig und hörten mir gerne zu, bis Friedens-
tichtet Sawrey, der Hauptanstifter der grausamen Verfolgung
im Norden, sie gegen mich aufhetzte, und sie ansingen, mich zu
stoßen, schlagen und quälen. Sie gerieten alsbald in Wut und
fielen über mich her, im Turmhau-8, und schlugen mich vor seinen
Augen zu Boden, stießen mich rmd traten mich mit Füßen. Der
Aufruhr war groß, so daß etliche über ihre Stühle fielen im
Gedränge. Schließlich kam Sawrey und befreite mich aus ihren
Händen und führte mich hinauö und übergab mich den Konstablern
und hieß sie mich peitschen und zur Stadt hinauß führen. Sie
führten mich etwa eine Meile weit, etliche hielten mich am Kragen,
etliche an den Armen und den Schultern und schleppten und
zerrten mich vorwärtö. Von den Freunden, die auf den Markt
und ins Turmhauß gekommen waren, um mich zu hören, wurden
viele auch zu Boden geworfen und dermaßen geschlagen, daß
manche von Blut überströmt waren. Richter Fellö Sohn, der
mir nachrannte, um zu sehen, maß mit Mir geschehe, warfen sie
in einen Wassergraben und einige schrien: ,,schlagt ihm die Zähne
aus dem Kopf.« A16 sie mich nun biz insz Moor hinauzgeschleppt
hatten, gefolgt von einem großen Haufen, gaben mir die Kon-
stabler mit ihren Weidenruten ein paar Schläge über den
Rücken und überließen mich dem Pöbel, der mit Stöcken, Hecken-
pfählen, Stechpalmen und Gichenzweigen versehen über mich
herfiel und mich auf Kopf und Glieder schlug, bis mir die
Sinne vergingen und ich auf den nassen Boden hinfiel. Als-
ich wieder zu mir kam und merkte, daß ich aus der nassen
Erde lag und die Leute mich umstanden, blieb ich einige
Zeit unbeweglich; und die Kraft des Herrn durchzuckte mich und
die ewige Erquickung erquickte mich, so daß ich wieder ausstehen
konnte in der stärkenden Kraft des Herrn; und die Arme auß-
streckend, sagte ich mit lauter Stimme: ,,Schlaget wieder, hier
sind meine Arme, mein Kopf und meine Wangen.« Einer auß
dem Haufen, ein ,,Frommer«, aber ein roher Kerl, schlug mir
mit seinem Stab genade auf die ausgestreckte Hand; meine Hand
wurde von diesem Schlag so zerquetscht und mein Arm so ge-
lähmt, daß ich ihn nicht wieder zurück ziehen konnte, und etliche
riefen: ,,Seine Hand ist für immer verstiimmelt; er wird sie nie


% \picinclude{./060-069/p_s067.jpg} 
Christus in uns. Erkenutniz der Quäkerischen Weltmission usw. 67
mehr gebrauchen können!« Aber ich betrachtete die Hand in der
Liebe zu Gott; denn ich stand zu allen, die mich verfolgt hatten,
in der Liebe Gottes; und nach einer Weile durchzuckte mich die
Liebe Gotte-3 und zuckte durch meinen Arm und meine Hand,
so daß ich augenblicklich die Kraft darin wieder spürte, vor aller
Augen. Daraufhin gerieten sie selber untereinander in Streit und
sagten mir, wenn ich ihnen Geld gebe, so wollten sie mich vor
den andern schützen. Aber der Herr trieb mich, ihnen das Wort
des Lebens zu verkünden, und ich zeigte ihnen, maß sie für ein
oerkehrtetz Christentum haben, und waß für Früchte die Predigten
ihrer Priester brächten; ich sagte ihnen, sie seien eher Juden oder
Heiden al-3 Christen. Darauf trieb mich der Herr, wieder durch
das Volk hindurch aus den Markt von Ulverstone zu gehen. Auf
dem Wege begegnete mir ein Soldat mit dem Schwert an der
Seite: ,,Herr,« sagte er, ,,ich sehe, daß Jhr ein Mann seid, und
es tut mir leid, daß Jhr so mißhandelt werdet;« und er bot mir
an, mir nach Kräften zu helfen. Aber ich sagte ihm: ,,Die Kraft
des Herrn ist über AlleZ,« und ging durch das Volk hindurch auf
den Markt, und keiner hatte Macht, mich anzurühren. Und als
auf dem Markte einige Freunde mißhandelt wurden, und ich jenen
Soldaten mit seinem nackten Schwerte mitten darunter sah, da
sprang ich hinzu, ergriff seinen Arm und befahl ihm, das Schwert
wieder einzustecken, wenn er eß mit mir halten wolle, und mit
mir auß dem Haufen herauö zu kommen; denn ich wolle nicht,
daß durch ihn ein Unheil geschehe. Einige Tage darauf wurde
dieser Soldat von sieben Männern ergriffen und durchgeprligelt,
weil er ez mit mir und den Freunden gehalten habe. ES war
in jenen Tagen die Art der Verfolger in diesen Gegenden,
ihrer 20 oder 40 aus einen einzigen loßzugehen. An Vielen Orten «
wurden die Freunde in der Weise überfallen, daß sie schier nicht
auf die Straße konnten; man warf ihnen Steine an und miß-
handelte sie. Al?. ich nach Swarthmore kam, kam ich gerade
dazu, wie die dortigen Freunde den Freunden die von Lampittö
Zuhörern mißhandelt worden waren die gebrochenen und ver-
letzten Glieder verbunden. Mein ganzer Körper war gelb, schwarz
und blau von den Schlägen, die ich an jenem Tage erhalten
hatte. Und die Priester singen wieder an zu prophezeihen, daß
wir in einem halben Jahr alle vernichtet sein würden.
Etwa zwei Wochen später ging ich auf die Jnsel Walney und
 


% \picinclude{./060-069/p_s068.jpg} 
James Naylor mit mir. An einem Morgen ging ich mit einem
Boot zu James Lancaster. Sowie ich ans Land stieg, stürzten vierzig
Männer mit Stöcken, Fischangeln und Knütteln hervor, überfielen
mich, schlugen und zerrten mich und versuchten, mich ins Wasser zurück
zu stoßen. Tlls sie mich beinahe in die See zurück geworfen hatten
und ich sah, daß sie mich umbringen wollten, lief ich mitten unter sie
zurück; aber sie fielen wieder über mich her und schlugen mich,
bis ich betäubt war. Als ich wieder zu mir kam, sah ich wie
James Lancasters Frau Steine nach meinem Gesicht wars, und
ihr Mann beugte sich über mich, um die Steine von mir abzu-
halten. Die Leute hatten James Lancasters Frau glauben
machen, ich hätte ihren Mann verhext, und hatten ihr ver-
sprochen, wenn sie ihnen melde, wann ich kommen werde, so
wollten sie mich töten. Und als bekannt wurde, daß ich komme,
hatten viele aus der Stadt sich aufgemacht, mit Knütteln und
Stöcken, um mich zu töten. Aber die Kraft des Herrn
schützte mich, daß sie mir nichts antun konnten. Zuletzt gelang
es mir, wieder aufzustehen; aber sie warfen mich sogleich wieder
ins Boot zurück. Als James Lancaster es sah, kam er sogleich
und brachte mich übers Wasser, daß ich vor ihnen sicher war;
aber so lange wir noch erreichbar waren auf dem Wasser, warfen
sie uns Steine nach. Als wir am anderen User ankamen, sahen
wir, wie sie James Naylor schlugen. Während ich noch drüben
gewesen war, hatte er sich abseits gehalten, so daß sie ihn erst sahen,
als ich fort war; da übersielen sie ihn und schrieen: ,,Tötet ihn!-«
Llls ich die Stadt, am anderen User, erreichte, kamen die
Leute mit Drefchflegeln und Stöcken, um mich zu verhindern, in
die Stadt zu kommen und schrieen: ,,Tötet ihn, tötet ihn! Schlagt
ihn auf den Kopf- bringt den Karren und führt ihn aus den
Kirchhoflss Nachdem sie mich mtßhandelt hatten, schleppten sie
mich aus der Stadt und ließen mich liegen. James Lancaster
ging nun zurück, um nach James Naylor zu sehen; als ich nun
allein da war, ging ich zu einem Wassergraben, und nachdem ich
mich gewaschen hatte, ging ich drei Meilen weit zu Thomas
Hutton, wo Lawson, der bekehrte Priester war. Als ich eintrat,
konnte ich fast nicht reden, so war ich zugerichtet; ich sagte ihnen nur,
wie ich James Naylor verlassen; da nahm jeder von ihnen ein
Pferd und holten ihn noch in jener Nacht zu sich. Als Margaret
Fell am nächsten Tag davon hörte, schickte sie ein Pferd und lleß


% \picinclude{./060-069/p_s069.jpg} 
Chrisiutz in unö. Erkenntnis der Quiikerischen Weltmission usw. 69
mich holen; aber ich war so verwundet, daß ich daß Schütteln
dez Pferdes nur mit großen Schmerzen ertragen konnte. Al-3 ich
nach Swarthmore kam, erließen die Friedenörichter Sawrey und
Thompson von Lancaster einen Verhastbefehl gegen mich; aber
als Richter Fell zurückkam, wurde er nicht auzgeführts Richter
Fell war nämlich die ganze Zeit meiner Mißhandlung nicht in
der Stadt gewesen. A13 er zurück kam, schickte er Verhaftbefehle
nach der Jnsel Walney, um alle jene Aufrührer festzunehmen,
worauf viele von ihnen entftohen; James Lanrasterß Frau be-
kehrte sich später zur Wahrheit und bereute, waö sie mir ange-
tan hatte, sowie auch andere jener grausamen Verfolger; aber
viele von ihnen traf daß Gericht Gottes-, und ez sind etliche unter
ihnen seither zu Grunde gegangen. Richter Fell verlangte einen
Bericht meiner Verfolgung; aber ich sagte ihm, sie hätten ja nicht
anderß handeln können in dem Geiste, in dem sie seien; es seien
die Früchte von dem, waö ihre Priester predigten, und beweise,
daß ihre Frömmigkeit und Religion falsch sei; er berichtete seiner
Frau, ich nehme die Sache leicht, wie einer, den sie nichts angehe;
und in der Tat hatte mich deö Herrn Kraft wieder geheilt ....
Jch ging mit Richter Fell zur Gerichtösitzung nach Lancaster;
er gestand mir unterwegs-, daß ihm noch nie eine solche Ange-
legenheit vorgekommen sei, und daß er nicht recht wisse, wie
er sich dabei verhalten solle. Ich sagte ihm, daß Pauluö, als-
er vor die Obersten der Schule trat und die Juden und Priester
viele falsche Anklagen gegen ihn vorbrachten, die ganze Zeit stille
schwieg. Dann, alß Festuö und Agrippa ihn hießen, für sichselber
reden, tat er ek- und reinigte sich von allen jenen falschen An-
schuldigungen; so sollte er ez mit mir machen. Vor dem Gericht
in Lancaster traten etwa vierzig Priester gegen mich aus; zuxihrem
Redner hatten sie einen namens Marshall gewählt und als
Zeugen einen jungen Priester und zwei Priestersöhne, die schon
vorher beschworen, daß ich eine Gotteölästerung ausgesprochen.
Die Richter hörten alleö an, maß die Priester und ihre Zeugen
gegen mich Vorbringen konnten .... aber die Zeugen waren
so verwirrt, daß sie sich bald alö falsche Zeugen verrieten ....
ES waren mehrere Leute anwesend, die auch in jener Ver-
sammlung gewesen waren, in der ich die Gotteßlästerung auöge-
sprochen haben sollte, alles Leute, die geachtet und angesehen waren
in dieser Gegend; sie erklärten, vor offenem Gerichtöhof, daß die

