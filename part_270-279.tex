% \picinclude{./270-279/p_s270.jpg} 
% Ein Teil des textes nach 260-269 verschoben.


Der erwähnte Vries an den König von Polen ist folgender:
»O König!
Wir wünschen dir Wohlergehen sowohl in diesem als dem
zukünftigen Leben. Und wir hoffen, daß wir unter deiner Herrschaft
unsre christliche Freiheit haben werden, Gott zu dienen und an-
zubeten. Denn wir haben den Grundsatz, nichtö zu tun, daß dem
König oder seinem Volke schaden kann. Wir sind Leute, die mit
einem guten Gewissen vor Gott wandeln wollen, durch seinen
heiligen Geist, und in demselben ihm dienen und ihn ehren wollen
und den Menschen gegenüber in allem, waß recht und billig ist,
indem wir ihnen tun, was- wir möchten, daß sie unß tun, im
Blick auf Jesus, den Anfänger und Vollender unsereö Glaubenz,
welcher Glaube unsre Herzen reinigt und uns Zugang zum
Vater verschafft, ohne welchen wir ihm nicht gefallen können,
und durch den alle Gerechten leben, wie die Schrift sagt (Hebr.
12, Röm. 5). Was wir von dir bitten, o König, ist, daß wir
Gewissen?-freiheit haben, Gott anzubeten und ihm zu dienen, und
ihn zu verehren, und in unseren Versammlungen miteinander zu
ihm zu beten im Namen Jesu, wie er gebietet, mit der Ver-
heißuug, mitten unter uns zu sein. Wir hoffen, der König müsse


% \picinclude{./270-279/p_s271.jpg} 
Rückkehr nach England. Kampf det Ordnungöbartei usw. 271
zugeben, daß solcher Dienst Gott gebühret und Christus. Und
wir geben dem Kaiser, waö ihm gebühret und bezahlen unsre
Abgaben und Steuern, wie unsre Nachbarn, je nach unsern Ver-
hältnissen. Wir haben nie in irgend einer Schrift dez neuen
Testamenteö gelesen, daß Ehristuö oder seine Jünger irgend je-
mand verbannten oder gefangen nahmen, der nicht ihren Glauben
hatte und sie nicht anhören wollte, oder daß sie befohlen hätten,
solches zu tun, sondern im Gegenteil solle man den Weizen und
die Spreu beisammen wachsen lassen, biß zur Ernte. Und die
Ernte ist daö Ende der Welt; dann wird Ehristuz seine Engel
senden, um die Spreu vom Weizen zu scheiden. Er tadelte die,
welche wollten Feuer vom Himmel regnen lassen, um die, so
Christus nicht aufnehmen wollten, zu vertilgen, und sagte ihnen,
sie wüßten nicht, welch Geisteß Kinder sie seien (Luc. 9). Gr
kam nicht, um die Leben der Menschen zu zerstören, sondern sie
zu erretteix
Wir bitten den König, daran zu denken, wie viel Verfolgung
gewesen ist um des Glaubens willen seit den Tagen der Apostel
unter den Christen. Ehristud sagte, daß die, welche ihn nicht be-
suchten, alö er gefangen war (Matth. 25), in die ewige Verdamm-
niö kommen: was wird dann erst auö denen werden, die ihn
gefangen nehmen in den Seinen, in denen er sich kund tut?
Noch ist das Ende der Welt nicht gekommen; wie will sich daß
Chrisientum am Tage dez Gerichts vor dem furchtbaren Gott
verantworten darüber, daß man sich untereinander um der Religion
willen verfolgte, unter dem Vorwand, die Spreu vom Weizen zu
scheiden, ehe daß Ende der Welt da war? Christuß befiehlt
den Menschen, sich unter einander zu lieben und die Feinde zu
lieben, daran solle man erkennen, daß sie seine Jünger seien
(Joh.1?-,35). O, daf; doch alle Christen in Einigkeit und Frieden
gelebt hätten, damit sie durch ihre Mäßigkeit und Selbstbe-
herrschung sowohl Türken als Heiden beschämt hätten! Lasset alle,
die Gott und Christum bekennen und nach dem herrlichen Evan-
gelium deck- Herrn Jesu leben, ihre Freiheit haben. EH ist unser
Wunsch- daß der Herr dee- Königö Herz geneigt mache gegen
alle zarten Gewissen, die den Herrn fürchten und sich scheuen,
ihm ungehorsam zu sein.
Wir bitten inständig den König, einige der edelmütigen
Kundgebungen verschiedener Könige und anderer zu lesen über


% \picinclude{./270-279/p_s272.jpg} 
272 Kapitel 11111.
die Gewissenöfreiheit, und besondere daö waö Stephanuß, der
König von Polen sagt, nämlich: »Et3 kommt mir nicht zu, die
Gewissen zu reformieren. Jch habe dietz immer Völlig Gott über-
lassen, weil es bei ihm steht; so halte ich'S jetzt und werde eö
in Zukunft halten. Ich lasse den Weizen wachsen bis zur Ernte,
denn ich weiß, daß die Zahl der Gläubigen klein ist. Jch bin,«
sagte er, wenn andere fortfuhren mit Verfolgungen, ,,der König
der Leute und nicht der Gen-issen.« Er war auch der Ansicht,
daß die Religion nicht solle mit Feuer und Schwert gepflanzt
werden .....
Ebenso wird Gewissenöfreiheit zugesagt bei König Jakob, in s einer
Rede im Parlament, 1609 . . . ferner durch König Karl, in seiner
Este-)- Zaastntn, dann durch den Prinzen von Oranien, im
Jahre 1579 ..... Ebenso bestätigen sie: Eraßmuß . . . Jrenäuz . . .
der Kaiser Konstantin .. . Augustin . . . Heinrich 17 .... Eusebiuö . . .
und andere ..... Und nun, o König, im Blick aus all diese
Zeugnisse über die Gewissen?-freiheit, von Kaisern, Königen und
andern, und auf die Freiheit, die Paulus in Rom hatte in den
Tagen deckt heidnischen Kaiser?-, bitten wir, daß wir in Danzig
auch die Freiheit haben möchten, in unsern Häusern zusammen
zu kommen; etz kann weder dem König noch der Stadt irgend
etwaß schaden, wenn wir zusammen kommen, um aus den Herrn
zu harten und zu ihm zu beten und im Geist und in der Wahr-
heit ihm zu dienen in unsern eigenen Wohnungen, da unsere
Grundsätze unß in keiner Weise veranlassen, jemand zu schaden,
sondern unsre Feinde zu lieben und für sie zu beten, selbst für die,
so unß Verfolgen. Darum, o König, und du, Stadt Danzig, be-
denket, würde ez euch nicht grausam scheinen, wenn man euch zu
einer Religion zwingen wollte, die euern Gewissen entgegen wäre?
Und wenn ihr eö grausam finden würdet, wenn man euch soleheß
tiite, so tut nicht den andern, waß ihr nicht wollt, daß sie euch
tun; das ist das königliche Gesetz, dem man zu gehorchen hat-
Solches wurde geschrieben in der Liebe zu deiner unsterblichen
Seele und zu deinem ewigen Heil.« G. F.
P. S. »Selig sind die Barmherzigen, denn sie sollen Barm-
herzigkeit erlangen (Matth. 5). Und gedenke, o König, der zwei
Apologien des Justinuß Martyr an den römischen Kaiser zur Ver-
teidigung der verfolgten Ehristen und jener denkwtirdigen Apold-
gie, von Tertullian geschrieben, über denselben Gegenstand, und die


% \picinclude{./270-279/p_s273.jpg} 
Rückkehr nach England. Kampf der Ordnungzpartei usw. 273
nicht für die christliche Religion gelten, sondern für alle Verfol-
gungen um der Religion willen ..... «
Jch blieb noch einige Wochen in London, das Parlament tagte
wieder, und mehrere Freunde versuchten, Linderung der Leiden für
die Freunde zu erlangen ..... Aber obgleich in beiden Häusern
verschiedene Parlaments-mitglieder den Freunden wohlwollten und
gerne etwas- zu ihrer Hilfe getan hätten, so waren sie durch die
Menge der Geschäfte daran verhindert, so daß die Leiden der
Freunde andauerten.
Waö aber namentlich viel zum Kummer und zur Prüfung
der Freunde beitrug war, daß etliche, die sich zur gleichen Wahr-
heit bekannten wie wir, abgewichen waren von der Ginfalt deß
Gvangeliumö und in fleischliche Freiheit geraten waren und ver-
suchten, andere nach sich zu ziehen; sie widersetzten sich der
Ordnung, die Gott durch seine Kraft aufgestellt und in seiner
Kirche eingeführt hatte; sie machten viel Lärm und Unruhe über
allerlei Vorschriften, wobei ez ihnen leicht wurde, manche mit
sich zu reißen, die freiere Neigungen hatten und einen breiteren
Weg gehen wollten, altz den der Wahrheit. So geschah eö,
daß etliche, die zwar einfältigen Gemüts, aber in der Wahrheit
noch Neulinge und von wenig Urteil.-skraft waren, dadurch ver-
führt wurden, da sie die Abgründe dez Satans nicht kannten.
Diesen zum Nutzen, um die Betrogenen aufzuklären und den
Schwachen das Verständnis über diese Dinge zu eröffnen, schrieb
ich folgendes: ,,Jhr alle, die ihr urteil?-loz manche Gebote ver-
werft, ihr könnt ebensogut die ganze Schrift verwerfen, die unö
gegeben wurde durch die Kraft de-3 Geisteß Gottes. Denn zeigt
sie nicht, sowohl im alten als im neuen Testament, wie man vor
Gott und den Menschen wandeln soll? Zeigt sie nicht, von der
allerersten Verheißung auf Ehristum in der Genesiö an, bis auf
die Zeiten der Propheten, was man glauben und worauf man
trauen soll? Hat nicht der Herr seinem Volke Gebote gegeben
zuerst durch die Väter und darnach durch die Propheten? Hat
nicht der Herr seinem Volke immer wieder geboten, wie es wandeln
solle, obgleich ez sich gegen die Propheten deß alten Bundes auflehnte,
wenn sie ihm den Weg vorschrieben, den es gehen solle, um Gott
zu gefallen und in seiner Gnade zu bleiben? Und hat nicht
Christus zu seiner Zeit die Leute gelehrt und geboten, wie sie
glauben und handeln sollten? Und haben nach ihm die Apostel
Gebkzt Fc:.  


% \picinclude{./270-279/p_s274.jpg} 
274 Kapitel 231111.
nicht vorgeschrieben, wie man dazu kommen kann, zu glauben und
das Evangelium und das Reich Gottes anzunehmen, indem
sie aus das hinwiesen, was zur Erkenntnis Gottes führen kann?
Und zeigten sie nicht, wie man im neuen Bunde wandeln soll
und auf welchem Wege man zur Heiligen Stadt gelangt? Und
haben die Apostel ihre Gebote nicht durch treue auserlesene Männer,
welche ihr Leben für Christus aufs Spiel setzten, den Kirchen
mitgeteilt und dieselben dadurch begründet? Jndem ihr nun die
Verordnungen, die von dem Geist Gottes eingegeben wurden,
verwerst, widerstrebt ihr damit dem Geist, der durch alle Heiligen
zu euch geredet hat. Gab es nicht von jeher, zur Zeit Moses und
der Propheten, zur Zeit Christi rmd in den Tagen der Apostel
etliche, welche dem Geist, der aus allen diesen zu ihnen redete,
widerstanden? Und ist es nicht so gewesen auch seit den Tagen der
Apostel? Wie viele haben sichsnicht erhoben seit dem Erscheinen
der Wahrheit, um sich der Ordnung, die aus dem Geist Gottes
ist, zu widersetzen! Diese alle sind aus demselben Geist, der von
Anfang an sich dem Geist Gottes widersetzte. Seht was für
Namen und Bezeichnungen der Geist Gottes diesem Geist des
Widerstands gab im alten und auch im neuen Testament, und
dasselbe gilt noch heute. Als Gott der Herr den alten Bund ge-
schlossen hatte, wurden etliche unter einander uneins und wider-
setzten sich, und diese waren schlimmer als der äußere Feind.
Und seht nur, was das für eine Sorte war, die sich in den Tagen
des neuen Bundes, zur Zeit des Evangeliums, Christus und den
Aposteln widersetzten, nachdem sie mit der Wahrheit in Be-
rührung gekommen waren. Seht, was das für eine Freiheit war,
für die sie stritten, und zu der es die, welche dem Kreuz Christi
widerstanden, brachten.
Und es ist der gleiche hochsahrende Geist, der auch jetzt nach
einer Freiheit verlangt, die der Geist und die Kraft nicht gewähren
können. Gr schreit gegen Zwang, und übt doch selbst Zwang; er
schreit nach Gewissenssreiheit und widersetzt sich doch der Gewissens-
freiheit; er schreit gegen die Vorschriften und macht selber solche
in Wort und Schrist. Dieser Geist, sein Ursprung, Anfang und
Ende, sind durch den Geist Gottes erkannt und gerichtet. Wenn
jener Geist ruft: »man darf nicht über die Gewissen richten! man
darf nicht in Glaubenssachen, nicht in Sachen der Religion, nicht
über die Geister richten!« so sage ich: ,,doch! Die, welche in


% \picinclude{./270-279/p_s275.jpg} 
Rückkehr nach England. Kampf der Ordnungspartei usw. 275
der reinen Kraft und dem reinen Geist sind, in dem die Apostel
waren, richten die Gewissen, ob es verhärtete Gewissen oder
empfängliche seien; sie richten den Glauben, ob er lebendig oder
tot sei; sie richten die Religion, ob sie eitel sei oder rein und
unbefleckt; sie richten die Geister und prüfen sie, ob sie von Gott
seien oder nicht; sie prüfen die Herzen, Ohren und Lippen,
mer rein ist und wer nicht; sie richten die Prediger, Apostel
und Propheten, ob sie von Christus kommen oder nicht; sie
richten die Streitigkeiten über äußere Dinge, sowohl in als
auch außer der Kirche; ja, das geringste Glied der Kirche hat
Macht, diese Dinge zu richten, weil es das einige rechte Maß
und rechte Gewicht besitzt, womit alle Dinge können gewogen
und gemessen werden ohne Ansehen der Person. Solches
Urteil geschieht, und solche Dinge werden getan durch die gleiche
Kraft und den gleichen Geist, den die Apostel hatten. Solche
können auch über Erwählung und Verwerfung richten, wer die
Wohnung behält und wer nicht, wer Jude ist und wer zur
Synagoge des Teufels gehört (Offb. 3), wer die Lehre Christi hat
und wer die des Teufels, wer aus der Kraft und dem Geiste Gottes
gebietet und wer aus einem losen Geist, der vom Joch Ehristi
ab, in Ziigellosigkeit, führt ..... Darum sollen alle ermahnt
werden, in der Kraft und dem Geist Christi zu bleiben, im Wort
des Lebens und der Weisheit Gottes, die über dem, das hier
unten ist, steht, in welcher sie ihre himmlische Urteilskraft
und Unterscheidungsgabe festhalten und das himmlische, geist-
liche Richten über jenes Richten stellen können, welches Gott
verunehret und in falsche Freiheit führt, von der Einigkeit des
himmlischen Geistes ab, welcher dem Bilde des Sohnes Gottes
gleich macht, und von seinem Evangelium, dieser Kraft Gottes,
und von seiner Wahrheit (von welcher der Teufel abgesallen ist)-
jn welcher alle einer Meinung, ein Herz und eine Seele sind, und
aus einem Geiste schöpfen werden, weil sie in einem Geist
getauft sind, und also auch in einem Leib, davon Christus das
Haupt ist, und somit die Briiderlichkeit und Einigkeit im Geist
halten, welche das Band des Friedens ist, dieses Friedens des
Fürsten aller Fürsten. Und die, welche am meisten gegen das
Richten schreien und sich davor fürchten, seien sie nun abtriinnige
,,Fromme« oder Gottlose, richten am allermeisten mit ihrem
tadelstichtigen, falschen Geist und Gericht, und doch können sie das
1Ss


% \picinclude{./270-279/p_s276.jpg} 
276 Kapitel Xlllll.
wahre Gericht des Geistes Gottes nicht ertragen noch vor dem-
selben bestehen. Das hat man von Anfang an merken können,
weil sie falsches Maß und falsches Gewicht harten; denn niemand
hat das rechte Maß und rechtes Gewicht, als wer im Licht,
der Kraft und dem Geist Christi bleibt. Nun kommt ein loser
Geist auf, der nach Freiheit schreit und gegen die Vorschriften,
und doch selber Wege vorschreibt in Wort und Schrift. Derselbe
Geist schreit gegen das Richten und will sich nicht richten lassen
und richtet doch selber durch einen falschen Geist. Solches ist
geschrieben als Protest gegen diesen Geist.«
London, 9. des 4. Monats 1678. G. F.
Von London ging ich nach Hertford ..... Hier kam wieder
eine Unruhe über mein Gemüt wegen jener unordentlichen und
ziigellosen Geister, die sich von uns lvsgetrennt hatten und suchten,
andere in eine falsche Freiheit mit- sich zu ziehen. Da ich ahnte,
wie viel Unheil und Schaden dadurch angerichtet werden konnte,
trieb., es mich, einige Zeilen an die Freunde zu schreiben, um sie
davor zu warnen:
,,Jhr Freunde alle,
Bleibet im süßen Leben des Lammes und über dem un-
ordentlichen, ausgeblasenen Geiste, der Zank und Hader anrichtet,
angeblich um des Gewissens willen, der aber doch in falsche Frei-
heit und Zügellosigkeit führt, die der Jugend gefährlich werden.
Die, welche dazu anstisten, sind schuld an dem Verderben, indem
sie in ihrer Leidenschaft statt des Gewissens einen widerspeustigen
Geist ausrühren, der den guten Geist in ihnen selber und in
allen andern Menschen ersticken wird- ’Dieser Geist darf nicht
auskommen, weder in ihnen noch in andern; denn sie verschließen
damit sich und andern das Himmelreich. Gin loser Geist, der
sich unter dem Vorwand der Gewissens-freiheit erhebt, ein wider-
spenstiger Wille, der sich mit Worten ohne Kraft zur Wahrheit
bekennt, alle solche Unordentlichkeit kann sich nur verstecken hinter
etwas, das zur ewigen Verdammnis führt. Darum bleibet
im sanften Geist Gottes in aller Demut, damit ihr in demselben
wisset, daß ihr alle Glieder untereinander seid und alle einen
Dienst habt in der Kirche Christi. Alle lebendigen Glieder kennen
einander im Geist nicht im Fleisch. Da herrschet nicht der Mann
über die Frau, wie Adam über Goa vor dem Fall, sondern Christus,
der geistige Mann, über seine geistigen Glieder, die in der


% \picinclude{./270-279/p_s277.jpg} 
Rückkehr nach England. Kampf der Otdnnugepariei usw. 277
himmlischen Liebe sind, die Gott in ihre Herzen gab, und die
allem Hader ein Ende macht.«
Hertford, 11. des 5. Monats 1678. G. F.
Von Hertsord ging ich nach .... Leieester .... wo ich
Freunde besuchte, die hier im Gefängnis waren, weil sie Zeugnis
für Jesus abgelegt hatten. Jch redete ihnen zu, dem Herm treu
zu bleiben und nicht müde zu werden, fsür ihn zu leiden. Und
nachdem ich von ihnen Abschied genommen hatte, sprach ich noch
mit dem Kerkermeister und bat ihn, sie gut zu behandeln, und
ihnen zuweilen zu erlauben, die Jhrigen zu besuchen. Dann hatte
ich eine Versammlung in Warwickshire .... und von da ging
ich nach Staffordshire, wo ich mehrere schöne und erleuchtende
Versammlungen hatte, sowohl um für die Wahrheit zu gewinnen,
als auch in ihr zu befestigen. Während ich in Stassordshire war,
trieb es mich, folgendes zu schreiben:
,,Liebe Freunde der Vierteljahresversammlungen und Monats-
Versammlungen allenthalben. Gs ist mein Wunsch, daß ihr alle
dnmach trachtet, in der Kraft und Wahrheit des Herrn einerlei
Sinn zu haben, nämlich den einen friedsamen, in dem kein Streit
und keine Feindschaft ist; und daß ihr auch in der Weisheit
Gottes sein möchtet, die rein, »keusch, friedsam und gelinde ist,
und sich sagen läßt« und die »von oben kommt, und über dem
das irdisch, menschlich und teuflisch ist« (Jak. 3) ist. Ju dieser
Weisheit möget ihr dazu kommen, ,,alles, was ihr tut, zu Gottes
Ehre zu tun« I1. Cor. 10,31). Und, liebe Freunde, wenn irgend
einmal irgend etwas geschieht, das könnte Streit, Zank oder
Zwietracht herbei führen, in euern monatlichen oder vierteljähr-
lichen Versammlungen, so soll es vor etwa ein halbes Dutzend
Leute gebracht werden, die es außerhalb eurer Versammlungen
besprechen und schlichten mögen, wie es am Anfang war, damit
alle eure monatlichen und oierteljährlichen Versammlungen friedlich
bleiben. Sie mögen dann der Versammlung vom Entscheid Mitteilung
machen, damit die Schwachen und Jungen unter euch nicht verletzt
werden, wenn sie von Zank und Hader in euern Versammlungen
hören, in denen kein Streit sein sollte, sondern alle sollten in
allen Dingen in gleicher Meinung vorgehen und beschließen, in
der Kraft Gottes, der Ordnung des Evangeliums; in diesem
Evangelium des Fricdene werdet ihr den Frieden in allen euren
Versammlungen bewahren. Wenn irgend zwei, Mann oder Frau,


% \picinclude{./270-279/p_s278.jpg} 
278 Kapitel 11111.
etwas gegen einander haben, sollen sie miteinander reden, und
es unter sich abmachen; wenn sie es auf die Weise nicht ent-
scheiden können, so sollen sie zwei oder drei auswählen, um es
zu entscheiden. Falls die es auch nicht entscheiden können, so soll
es der Kirche vorgelegt werden; etwa ein halbes Dutzend aus der
monatlichen oder oierteljährlichen Versammlung sollen es wissen, und
endgültig ohne Ansehen der Person richten. Laßt alle Vorurteile
abgelegt und begraben sein, auch alle Gngherzigkeit untereinander,
und ,,lasset die Liebe, die sich nicht blähet, nicht haßt, nicht das
Jhre sucht, sondern alles erträgt« (1. Cor. 13), in euren Ver-
sammlungen herrschen, denn das erbauet den Leib, dessen Haupt
Christus ist; und diese Liebe ist besser als tönendes Erz und
eine klingende Schelle. Diese Liebe duldet alles und ist freund-
lich; sie hält alles nieder, was sich rühmen, sich erheben,
blähen oder unziemlich benehmen möchte, oder sich leicht erbittern
läßt; sie ist Herr über alles, was nicht aus dem Geist ist, dessen
Früchte sind Liebe, Friede, Freundlichkeit; aus daß ihr »in diesem
heiligen Geist alle getauft sein möget zu einem Leib« (1. Cor. 12),
und aus einem Geist schöpfet, in welchem ihr Einigkeit habet,
und in welchem ist das Band des ,,Königs der Könige« und
der Frieden des ,,Herrn aller Herren« ........ ,,Die in
der Liebe bleiben, bleiben in Gott; denn Gott ist die Liebe«
(1. Joh. 4). Bleibet alle in dieser Liebe. Meine Liebe in Christo
Jesu, dem ewigen Samen, der über allem ist.«
Staffordshire, 20. des 6. Monats 1678. G. F.
Jch blieb nun im Norden während ungefähr eines Jahres,
da ich dort unter den Freunden für den Herrn zu wirken
und oiel damit zu tun hatte, Antworten zu schreiben aus Bücher,
die die Gegner geschrieben hatten ..... Da es mir schien, daß
viele, welche die Wahrheit aufgenommen hatten und Offenbarungen
darüber gehabt hatten, davon abgeirrt waren, weil sie nicht
demütig geblieben waren, so trieb es mich, folgendes Schreiben
ergehen zu lassen als Warnung und Ermahnung an alle, in der
Demut zu bleiben:
,,Meine lieben Freunde,
Die der Herr besucht hat in seinerjfreundlichen Barmherzigkeit
mit dem Ausgang aus der Höhe, und eure Hechen öffnete, daß
ihr euch seinem Namen beugt und ihn bekennt: bleibet niedrig in
euern Herzen und höret aus Christus, damit ihr in der Demut


% \picinclude{./270-279/p_s279.jpg} 
Rückkehr nach England. Kampf der Qrdnungspartei usw. 279
bleibet, die er euch lehrt, aus daß die Jüngern unter euch in
keiner Weise aufgeblasen oder überstiegen oder eingebildet werden
durch ihre Offenbarungen und des Gnadenstandes verlustig gehen,
indem sie sich zum Gigendünkel hinreißen lassen und darnach in
Verzweifllung geraten und so die Kraft Gottes mißbrauchen.
Es war das Bestreben der Apostel, daß niemand die Kraft
Gottes mißbrauchen möchte, sondern der Glaube sollte sich in
allen Dingen aus sie gründen, so daß alle, die andern die Wahr-
heit predigen, sich selber auch mit einschließen, und nicht andern
predigten und sich selber aus-nehmen. Darum kommt euch zu,
euch mit einzuschließen in dem, das ihr andern predigt, und demütig
zu bleiben darin; dann wird der Gott der Wahrheit den Demütigen
erhöhen in seine Wahrheit, Licht, Gnade, Kraft, Geist und
Weisheit, zu seiner Ehre. Dann behalten alle ihr Maß von
Gnade, Licht, Kraft und Geist Christi. Lasset niemand den Geist
dämpfen oder seine Regungen, ihn betrüben oder davon abweichen;
sondem lasset euch von ihm, der alle in seiner Hut hält, leiten;
er gibt das Verständnis, wie man dem heiligen, reinen Gott,
dem Schöpfer und Erhalter in Christus, dienen, ihn anbeten
und ihm gefallen soll; und wie man seinen heiligen Geist der
Einigkeit und des Friedens in seinem Volke hüten, auf ihn
warten und ihn pflegen soll. Der heilige Geist lehret ein sanftes,
demütiges, ruhiges und heiliges Gemüt aus den Samen, den
Christus in eines jedenHerz gegeben hat, wirken, und auch in denen,
die vom Geist, der Gnade, dem Licht und dem Evangelium abgefallen
sind, das Licht, die Gnade, den Geist und das Evangelium
finden. So können, durch heiligen Wandel und heiliges Predigen,
alle dazu gelangen, daran teil zu haben, damit Gott in allen
Dingen geehrt werde durch euch, wenn ihr zu seinem Preis
Früchte hervor bringet.«
Swarthmore, 30. des 10. Monats 1679. G. F ....
Darnach trieb mich der Herr, die Freunde in Surrey und
Sussex zu besuchen. Ich ging zu Wasser nach Kingston, wo ich
einige Tage blieb; denn der Herr gab mir ein, dem Sultan
und dem Gouverneur von Algier zu schreiben, um diese und
ihre Untertanen zu ermahnen, von ihrer Schlechtigkeit zu
lassen, den Herm zu fürchten und recht zu tun, sonst werde
das Gericht Gottes über sie herein brechen und sie ohne Gnade
zerstören .....

