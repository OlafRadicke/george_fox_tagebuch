
% \picinclude{./090-099/p_s090.jpg} 
90 Kapitel 711.
,,Geh, George, geh nur,« so fürchtete ich, daß, wenn ich nicht
ginge, man sage, ich sei meinen Eltern ungehorsam; so ging ich,
und die übrigen Priester wollten daß Volk abhalten, aber es
gelang ihnen nicht, denn da alle uns- hören wollten, wurden wir
ganz umringt. Jch fragte den Priester, maß er zu sagen habe?
er antwortete, wenn er nicht aus dem rechten Wege sei, so sollte
ich für ihn beten; und wenn ich nicht auf dem rechten Wege sei,
so wollte er für mich beten; und er wolle mir oorsagen, was ich
für ihn beten solle. Ich erwiderte ihm: ,,eS scheint, daß du nicht
einmal weißt, ob du auf dem rechten Wege bist; ich aber weiß,
daß ich auf dem rechten Wege bin, Jesus Christus, in welchem
du nicht bist, und du wolltest mir vorsagen, wie ich zu beten
habe, und verwirfst doch daß Common-Prayerbook so gut wie ich,
und ich verwerfe dein Geplapper ebenfalls. So du willst, daß ich
nach etwas Hergesagtem für dich bete, heißt daß nicht, die Lehre
der Apostel mißachten und ihr Beten im Geist, der die Worte
eingibt?« Hier fingen die Leute an zu lachen; mich aber trieb
ez, weiter zu ihm zu reden. Nachdem ich ihm gesagt, was
mir zu sagen oblag, und daß ich, so Gott wolle, über acht Tage wieder
in der Stadt sein werde, gingen wir fort. Die Priester machten,
daß sie fort kamen und viele wurden gewonnen, denn die Kraft dez
Herrn kam über alle. Wenn sie schon meinten an diesem Tage
der Wahrheit geschadet zu haben, war doch mancher gewonnen
worden, und viele, die schon früher gewonnen worden, wurden durch
daß, was an jenem Tage geschehen, bestärkt, und etz gab den
Priestern einen Stoß. Mein Vater, obgleich er ein Anhänger der
Priester war, war so befriedigt, daß er mit seinem Stock auf die
Erde schlug und sagte: »wahrlich, ich sehe, daß wer willenß ist,
bei der Wahrheit zu bleiben, dem wird sie durchhelsen« .....
Darauf zog ich wieder umher und hielt Versammlungen und
kam nach Swannington, wohin auch wieder Soldaten kamen;
aber die Versammlung war ruhig, die Macht Gotteö war
über allen, und die Soldaten störten mich nicht. Darauf ging
ich nach Leicester und Whetstone. Dahin kamen siebzehn Soldaten
auö Oberst Hackerß Regiment, mit ihrem Anführer, und führten
mich, gerade vor Beginn der Versammlung, hinweg, obgleich die
Freunde, die von allen möglichen Orten hergekommen waren,
schon anfingen sich zu versammeln. Ich sagte dem Vorgesetzten,
er solle wenigstenß die Freunde in Ruhe lassen, ich wolle für sie


% \picinclude{./090-099/p_s091.jpg} 
Kämpfe mit schwärmerischen Ranters und zehutengierigen Priestern usw. 91
alle haften; so nahmen sie denn mich und ließen die andern in
Ruhe, ausgenommen Alexander Parken!) der mit mir kam. Am
Abend brachten sie mich vor Oberst Hacker; sein Major, seine
Hauptleute und viele seiner Leute waren zugegen und wir gaben
auöfiihrlich Auskunft über die Priester und über die Versammlungen,
denn ez ging damals gerade daß Gerücht von einer Verschwörung
gegen Oliver Eromwell. Jch hatte lange Grörterungen über das
Licht Christi, daß einen jeden, der in die Welt kommt, erleuchtet
(Joh. 1, 9). Oberst Hacker fragte, ob e-3 dieses Licht auß Ehristuß
gewesen sei, daß den Judaö dazu geführt habe, seinen Herrn zu
verraten und sich darnach zu erhängen? Jch sagte ihm: ,,nein,
das war der Geist der Finsterniß, der Christuz und sein Licht
haßte.« Darauf sagte Hacker, ich solle nach Hause gehen und dort
bleiben, und nicht überall zu den Versammlungen gehen. Jch sagte
ihm, ich sei ein ganz harmloser Mensch und habe nichts mit Ver-
schwörungen zu tun, vielmehr verabscheue ich solcheß. Sein Sohn
Needham sagte: »Vater, dieser Mensch hat nun schon lange ge-
herrscht, eö ist Zeit, daß man ihn unschädlich mache.« q Jch fragte
ihn, ,,warum, was habe ich getan? oder wem habe ich je etwas
zu leide getan? ich bin in dieser Gegend geboren und aufge-
wachsen, wer kann mir irgend etwaß Böses nachsagen seit meiner
Kindheit?« Darauf fragte mich Oberst Hacker nochmals-, ob ich
nach Hause gehen wolle und dort bleiben? Ich antwortete ihm,
ich würde mich ja mit einem solchen Versprechen schuldig bekennen,
wenn ich nach Hause ginge und auö meinem Hause ein Gefängniß
machen wollte; und ginge ich dann doch zu den Versammlungen, so
würde ez heißen, ich sei dem Befehl ungehorsam. Jch erklärte
ihnen, ich gehe auf dee- Herrn Geheiß zu den Versammlungen,
darum könne ich mich ihren Vorschriften nicht fügen; aber wir
seien ein sriedlichez Volk. ,,Gut denn,« sagte Oberst Hacker, »ich will
euch zum Lord Protektor schicken, durch Hauptmann Drury, einen
aus seiner Leibgarde.« Die Nacht über wurde ich alß Gefangener
gehalten und am folgenden Morgen um sechö Uhr dem Haupt-
mann Drury übergeben. Ich wünschte vor dem Fortgehen noch
mit Oberst Hacker zu reden, er ließ mich vor sein Bett kommen und
drang sogleich wieder in mich, nach Hause zu gehen und keine
Versammlungen zu halten; ich erklärte ihm, ich könne mich dem
1) Alexander Parker, ein Mann von vornehmer Herkunft, reiste viel im
Dienst des Quäkertnmö und schrieb viele Bücher und Briefe zu seiner Verbreitung. NT


% \picinclude{./090-099/p_s092.jpg} 
92 Kapitel 711.
nicht fügen, sondern müsse meine Freiheit haben. ,,Dann,« sagte
er, ,,müßt ihr vor den Protektorcks Hierauf kniete ich an feinem
Bett nieder und betete zum Herm, ihm zu vergeben, denn er war
ein Pilatus, auch wenn er seine Hände gewaschen hätte; und ich
flehte zum Herrn, daß, wenn der Tag seiner Prüfung und Heim-
suchung komme, er sich dessen, was ich ihm gesagt, erinnern möge.
Gr war eben aufgehetzt von Priester Stephens und den andern
Priestern und ,,From1nen«, die darin ihre Bosheit ausließen, weil
sie mich durch ihr Argument nicht hatten überwinden können
und dem Geiste Gottes in mir nicht hatten widerstehen können;
darum hatten sie nun die Soldaten geschickt, um mich zu greifen.
Als später dieser Oberst Hacker im Gefängnis in London
war, wurde es ihm ein oder zwei Tage vor seiner Hinrichtung
in Erinnerung gebracht, wie er an den Unschuldigen gehandelt
hatte, und er gedachte daran und bekannte es Margaret Fell;
und es bedriickte ihn. Nun konnte sein Sohn, der damals
gesagt hatte, ich habe genug geherrscht, es sei Zeit, mich fort zu
schaffen, zusehen, wie sein Vater sottgeschasft wurde, als man ihn
erhängte in Tyburn.
Jch wurde nun von Hauptmann Drury als Gefangener von
Leieester fortgebracht. Als wir nach Harborough kamen, fragte
er mich, ob ich heimgehen wolle und 14 Tage dort bleiben? Er
versprach mir die Freiheit, wenn ich weder Versammlungen halten
noch zu solchen gehen wolle. Jch erwiderte ihm, ich könne nichts
dergleichen versprechen; er fragte und versuchte mich wiederholt
auf dem Wege in derselben Weise, und immer gab ich ihm die-
selbe Antwort. So brachte er mich nach London und quartierte
mich in Mermaid ein; unterwegs trieb es mich, die Leute zu J
warnen vor dem Tag des Herrn, der über sie kommen werde.,
Nachdem Hauptmann Drury mich untergebracht, verließ er mich
und ging zum Protektor, um Bericht über mich zu erstatten. Als
er zurückkam, sagte er, der Protektor verlange, daß ich kein
mörderisches Schwert gegen ihn oder die Regierung gebrauche,
und daß ich dies in beliebigen Worten schristlich erklären und mit
meiner Unterschrift versehen solle. Jch antwortete Hauptmann
Drury nur wenig; aber am nächsten Morgen trieb mich der Herr,«
ein Schreiben an den Protektor auszusetzen, in dem ich vor dem«
Angesicht Gottes des Herrn erklärte, daß ich das Tragen eines
mörderischen Schwertes oder irgend einer anderen äußeren Waffe


% \picinclude{./090-099/p_s093.jpg} 
Kämpfe mit schwiirmerischen Routers und zehntengierigen Priestern usw. 93
verabscheue, und daß ich von Gott gesandt sei, Zeugnis abzu-
legen gegen jegliche Gewaltttitigkeit und gegen die Werke der
Finsternis; und um die Leute von der Finsternis zum Licht zu
bringen und vom Kriegen und Streiten zum Evangelium des
Friedens. Nachdem ich geschrieben, was der Herr mir eingegeben
hatte, setzte ich meinen Namen darunter und übergab es Haupt-
mann Drury, damit er es Oliver Eromwell gebe, was er auch
tat. Rath einiger Zeit brachte mich Hauptmann Drury vor den
Protektor in Whitehall; es war an einem Morgen, ehe er ange-
kleidet war, und einer, namens Harvey, der sich auch eine Zeit
lang zu den Freunden gehalten hatte aber ungehorsam geworden
war, bediente ihn. Als ich eintrat, trieb es mich zu sagen:
,,Friede sei mit diesem Hause,« und ich ermahnte ihn, in der Furcht
Gottes zu bleiben, damit er Weisheit von ihm empfangen möge,
daß sie ihn leite; und daß er alle Dinge, die in seiner Hand
seien, zu Gottes Ehre regiere. Jch redete lange mit ihm über
die Wahrheit und über die Religion, er zeigte sich sehr verständig;
aber er sagte, wir zankten mit den Priestern, die er Diener Gottes
nannte. Jch entgegnete ihm, ich zanke nicht mit ihnen, sondern
sie mit mir und mit meinen Freunden. ,,Aber«, sagte ich, ,,wenn
wir die Propheten und Apostel anerkennen, so können wir solche
Lehrer, Propheten und Hirten, gegen welche die Propheten und
Christus auftraten, nicht gut heißen, sondem wir müssen auch
gegen sie auftreten, durch denselben Geist und dieselbe Krast.«
Ferner zeigte ich ihm, daß die Propheten, Christus und die
Apostel umsonst predigten und gegen die auftraten, welche es
nicht umsonst taten, sondern um schändlichen Gewinnes willen
und die um Geld wahrsagten und um Lohn lehrten (Micha 3, 11),
gierig und geizig waren und nie genug bekamen; und daß die,
welche den Geist Christi und der Apostel und Propheten haben,
auch jetzt noch gegen das alles austreten müssen, wie jene damals.
Während ich sprach, sagte er mehrmals, es sei sehr gut, es sei
wahr. Ich sagte ihm, daß alle, die sich Christen nennen, die
H Schrift haben, aber nicht alle die Kraft und den Geist, welche
die hatten, die die Schrift geschrieben, und dies sei der Grund,
warum sie nicht in der Gemeinschaft mit dem Vater und dem
Sohne seien, noch mit der Schrift, noch unter einander. Jch
redete noch über vieles andere mit ihm; da aber Leute herein
kamen, zog ich mich ein wenig zurück; als ich mich anschickte fort


% \picinclude{./090-099/p_s094.jpg} 
94 Kapitel VU.
zu gehen, faßte er mich bei der Hand, und sagte mit Tränen in
den Augen: ,,Komm wieder zu mir, denn wenn du und ich nur
eine Stunde im Tage beisammen wären, so würden wir einander
näher kommen«; und er fügte bei, er wünsche mir so wenig etwaß
Böseö als seiner eigenen Seele. Jch sagte ihm, wenn er etz tun
würde, so würde er damit seiner eigenen Seele schaden; und ich
bat ihn, aus die Stimme Gottes zu hören, auf daß er in seiner
Wei?-heit bleiben möge und ihm gehorchen; wenn er ez tue, so
werde er vor Hartherzigkeit bewahrt bleiben; wenn er aber
nicht auf Gottes Stimme höre, so werde sein Herz verhärtet
werden. E-r sagte, dietz sei wahr; daraus ging ich hinauß, und
Hauptmann Drury kam hinter mir drein und teilte mir mit, sein
Lord Protektor sage, ich sei frei und könne gehen, wohin ich
wolle. Darauf wurde ich in einen großen Saal geführt, wo
die Kammerherrn des Lord Protektor zu speisen pflegten; ich fragte,
warum ich hierher geführt werde? sie sagten, es geschehe auf
Befehl des Protektor, damit ich mit ihnen speise. Jch hieß sie,
dem Protektor sagen, daß ich nicht von seinem Brote esse, noch
von seinen Getränken trinke. Al?-’ er dies hörte, sagte er: ,,nun
sehe ich, daß ein Volk entstanden und heroorgetreten ist, welcheß
ich nicht zu gewinnen vermag, weder durch Gaben, noch durch
Ehren, noch Stellen, während mir dies bei allen anderen Sekten
und Menschen gelingt«, worauf man ihm entgegnete, daß wir
ja daß Eigene hingeben und darum kaum nach dem Seinigen
trachten würden ....
Jch begab mich nach London, wo wir große und mächtige
Versammlungen hatten. Der Zudrang war so groß, daß ich fast
nicht hinein konnte, und die Wahrheit breitete sich ungeheuer auß.
Thomaö Aldam und Robert Craven und viele Freunde kamen
nach London, um nach mir zu sehen; aber Alexander Parker
blieb bei mir.
Nach einiger Zeit ging ich wieder nach Whitehall und ez
trieb mich, den Tag de,8 Herrn unter ihnen zu verkünden und
daß der Herr gekommen sei, sein Volk selbst zu lehren, und ich
predigte sowohl den Osfizieren alö denen von der Garde Oliver?-.
Aber ein Priester widersprach, alß ich daß Wort des Herm
verkündete; denn Oliver hatte verschiedene Priester um sich, und
dieser war ein Neuigkeitökrämer, ein häßlicher Priester, ein hinter-
listiger, mißgünstiger Mann; ich sagte ihm, er solle Buße tun


% \picinclude{./090-099/p_s095.jpg} 
Kämpfe mit schwärmerischen Ranterö und zehntengierigen Priestern usw. 95
und er setzte in der darauf folgenden Woche in seine Zeitung,
ich sei in Whitehall gewesen und habe dort einem Diener Gotteß
gesagt, er solle Buße tun. Als ich wieder dorthin kam, traf ich
ihn wieder, und viele Leute schatten sich um unö. Ich bewiez
dem Priester, daß er in verschiedenen Dingen gelogen habe, und
er mußte schweigen. Gr schrieb in der Zeitung, ich habe silberne
Knöpfe; was: falsch war, denn sie waren bloß auß Blech. Ferner
schrieb er, ich lege den Leuten Bänder um die Arme, damit sie
mir folgen; daß war wieder gelogen, denn ich hatte in meinem
ganzen Leben nie Bänder getragen oder gebraucht. Drei Freunde
gingen hin, um den Priester zur Rede zu stellen und ihn zu
fragen, woher er diese Dinge habe; er sagte, eine Frau habe es
ihm gesagt; und wenn sie wieder kommen, so wolle er ihnen ihren
Namen sagen. Alz sie wieder kamen, sagte er, ez sei ein Mami
gewesen, aber er sage den Namen nicht, wenn sie wieder kommen,
wolle er ihn. dann sagen. Alß sie daß drittemal kamen, sagte
er ihn wieder nicht, behauptete aber, wenn ich erkläre, daß alleß
nicht wahr sei, so wolle er etz in die Zeitung setzen. Als darauf
die Freunde ihm diese Erklärung brachten, so wollte er sie doch
nicht aufnehmen, sondern wurde zornig. So handelte dieser
infame Lügenschmied, um der Wahrheit zu schaden und um die
Leute gegen die Freunde und die Wahrheit einzunehmen, wovon
ein außführlicher Bericht in einem Buche, daß bald darauf ge-
druckt wurde, kann ersehen werden. Diese liignerischen Priester
waren Jndependenten, wie die zu Leieester; aber deß Herrn
Kraft oernichtete alle ihre Lügen, und viele kamen dazu, die
Schlechtigkeit der Priester einzusehen. Der Herr dez Himmelö
brachte mich durch seine Kraft durch alles- hindurch, und seine
herrliche Kraft tat sich kund im Lande, so daß in dieser Zeit
viele Freunde getrieben wurden, umher zu ziehen, um das ewige
Evangelium zu verkünden, in allen Teilen detß Landes und auch
in Schottland; und die Herrlichkeit des Herrn erschien allen zu
seiner ewigen Ehre ..... ES fanden große Bekehrungen in
London statt und auch mehrere im Hause deZ Protektorß uf in
seiner Famlie; ich versuchte zu ihm zu gehen, aber ich be am
keinen Zutritt, die Wachen waren so unfreundlich.
Die Preßbyterianer, Jndependenten und Baptisten waren sehr
erzürnt, denn viele bekehrten sich zum Herrn Jesuß Christus und
hörten seine Lehre. Sie empfingen seine Kraft und fpürten sie


% \picinclude{./090-099/p_s096.jpg} 
96 Kapitel 7111.
in ihren Herzen, und das trieb sie, gegen die übrigen auf-
zutreten.
Kapitel Vlll.
Brief an den Papst. Die Studenten von Cambridge. Die Qniiter
in der Bibel. Wachsende Entfremdung von Cromwell.
Es kam über mich vom Herrn, ein kurzes Schreiben auszu-
setzen und zu verbreiten, als Grmahnung an den Papst und alle
Könige und Herrscher von ganz Europa:
,,Freunde,
Jhr Häupter und Obersten, ihr Könige und Fürsten alle,
verfolget nicht in Erbitterung und Eifer die Lämmer Christi; wendet
euch nicht ab, wenn Gottes Stimme, seine innige Liebe und Barm-
herzigkeit aus der Höhe euch ruft, auf daß nicht sein Ann und
seine Macht, die jetzt die Welt ergriffen haben, euch unversehens
erfassen. Sie kehrt sich gegen die Könige, und die Weisen werden
weichen müssen, und ihre Krone wird zu Staub werden; und sie
werden erniedrigt und dem Erdboden gleich gemacht werden. Der
Herr wird König sein und wird die Krone dem geben, der seinen
Willen tut. Die Zeit ist gekommen, daß Gott der Herr Himmels
und der Erde die Stolzen entlarven wird und ihren Ruhm stürzen.
Jhr, die ihr Christus bekennet und liebet doch eure Feinde nicht,
sondern nehmet im Gegenteil seine Freunde gefangen, ihr zeiget
damit, daß ihr nicht in dem Leben seid, das aus ihm kommt, ihr.
liebet Christus nicht, wenn ihr nicht seine Gebote haltet. Des
Herrn Zorn fängt an zu brennen, und sein Feuer verbreitet
sich, um die Böfewichter zu zerstören, und es wird kein Zweig
noch Reis übrig lassen. Die so ihren Wandel nicht mehr in
Gott haben, sind nicht mehr in jenem Geist, der die Schrift ein-
gegeben hat, und nicht mehr im Lichte, damit Christus sie alle
erleuchtet hat .... Darum seid schnell zuhören, schnell zu reden,
aber langsam zu verfolgen (Jak. 1, 19); denn der Herr führt nun
sein Volk aus den Wegen der Welt zu Christus dem wahren
Weg, und von allen weltlichen Kirchen zu der Kirche, die in ihm,
dem Vater Jesu Christi, ist, und von allen Lehrern der Welt,
um selber ihr Lehrer zu sein durch seinen Geist; von den irdischen
Bildnissen, zum Gbenbilde seiner selbst; und von den irdischen


% \picinclude{./090-099/p_s097.jpg} 
Vries an den Papst. Die Studenten von Cambridge usw. 97
Kreuzen aus Holz und Stein, zu der Kraft des Kreuzes Christi.
Denn alle diese Bilder und Kreuze sind ein Absall von Gott und
seiner Kraft und dem Kreuz Christi, welches nun die Welt richten
wird und alles niederwersen, was ihm entgegen ist; seine Macht
hat kein Ende.
Lasset solches die Könige von Frankreich und von Spanien
und den Papst wissen, damit sie alles prüfen und das Gute behalten;
1md sie sollen vor allem prüfen, ob sie nicht den Geist dämpsten
(1. Thess. 5, 19), denn der große Tag des Herrn ist über die
Bosheit und Gottlosigkeit und Ungerechtigkeit der Menschen
gekommen, und der Herr wird ,,durchs Feuer richten und durch
sein Schwert alles Fleisch« (Jes. 66, 18). Und die Wahrheit
und die Krone der Ehren und das Szepter der Gerechtigkeit
werden erhöht werden; und das Göttliche, das in einem jeden
ist, auch wenn er davon abgesallen ist, wird hiervon Zeugnis
geben. Christus ist als Licht in die Welt gekommen und erleuchtet
einen jeden, der in die Welt kommt, damit dadurch alle zum
Glauben kommen. Und wer das Licht, womit Christus ihn
erleuchtet, spürt, der spüret Christus in seinem Jnnern und das
Kreuz Christi, diese Kraft Gottes; der brauchet kein hölzernes oder
steinernes Kreuz, um an Christus und sein Kreuz gemahnt zu
werden; denn es ist selber die Kraft Gottes, welche sich ihm
innerlich kund tut.« G. F.
Ferner trieb es mich, einen Brief an den Protektor zu
schreiben, um ihn zu ermahnen, aus das große Werk zu achten,
das der Herr unter allen Völkern zu tun im Begriffe ist, und
aus das Beben, das sie alle erzittern macht, damit er auf der
Hut sei, daß er nicht mit seinem scharfen Verstand, seiner Geschick-
lichkeit und seiner Klugheit selbstische Nebenzwecke verfolge.
Es wurde zu der Zeit eine Verordnung zur Prüfung der
sogenannten Geistlichen erlassen, ob man sie bestätigen oder ihrer
Amter und Besoldungen entsetzen solle, und es trieb mich, den
betreffenden Vorgesetzten darum zu schreiben.
,,Freunde,
.... Christus zeigt seinen Jüngern und dem Volk, wie
man solche wie diese zu prüfen hat Sie werden von den Menschen
Herr genatmt. Sie sitzen aus den ersten Plätzen der Versamm-
lung; sie sind Hörer aber nicht Täter. Gr rief siebenmal Wehe!
über ste und verurteilte sie (Matth. 23) .... Gs gab in alten Zeiten
George Fox. 7


% \picinclude{./090-099/p_s098.jpg} 
98 Kapitel 7111.
ein Kornhauö, wo die Waisen, die Fremdlinge und die Witwen
hinkamen und zu essen bekamen, und die, welche ihre Zehnten
nicht inö Kornhauö brachten, gediehen nicht (Maleachi 3); hat
aber Ehristuö nicht allen Zehnten und Priestern und Tempeln
ein Ende gemacht? .... Sind je die Priester, die Zehnten nach
Menschensatzungen nahmen, gediehen? .... Warfen die Apostel
je jemanden in den Kerker wegen der Zehnten, wie ihr es jetzt tut?
Zum Beispiel: Ralph Hollingworth, Priester von Phillingham,
hat zu Lincoln einen armen Dachdecker namenß Thomaö Bromby
wegen einer kleinen Abgabe, nicht mehr als sechö Schilling, inß
Gefängniß geworfen, wo er nun schon seit achtunddreißig Tagen
ist; und der Priester ersuchte den Richter, daß man dem Mann
nicht erlaube, etwaS zu seinem Unterhalt im Gefängnis in der
Stadt zu verdienen. Jst dieses eine Empfehlung für euch, die ihr
die Aufgabe habt, die Priester zu wählen? . . .» Ehristutz hieß
seine Jünger, alß er sie au?-sandte, umsonst zu geben, wie sie
umsonst empfangen hatten; und in den Städten, durch die sie
zogen, mußten sie sehen, wer würdig war, und dort bleiben und
essen, waö man ihnen oorsetzte; und alß sie zu Christus zurück
kamen und er sie fragte, ob sie Mangel gelitten hätten, so sagten
sie: ,,nein«. Sie gingen nicht in die Stadt und fragten die
Leute, wie oiel sie im Jahre bekommen, wie dietz jetzt geschieht
von denen, die abgesallen sind. Der Apostel sagt, ,,habe ich
nicht zu essen und zu trinken?« aber er sagt nicht: ,,habe ich nicht
Osterpfriinde, Aufbesserungen und Geldsummencks .... ,,EZ soll
dem Ochsen, der da drischt, nicht daß Maul verbunden werden«
(5. Mos. 25, 4), aber sehet zu, ob ihr auch gedroschen habt und
ob daß- Korn in den Scheimen ist! Dies sagt einer, der eure
Seelen lieb hat und euer ewigeß Heil will.« G. F.
Nachdem ich einige Zeit in London gewesen und dort gewirkt
hatte, trieb etz mich nach Bedsordshire zu John Crook 1) zu gehen,
wo eine große Versammlung war und viele die Wahrheit annahmen.
John Crook sagte mir, daß am folgenden Tage mehrere Herren
der Umgegend mit ihm speisen werden, um mit ihm zu diskutieren.
Sie kamen und ich redete von der ewigen Wahrheit Gotteö zu
ihnen. Mehrere Freunde gingen an jenem Tage inz Turmhauß.
1) John Crook, früher ein angesehener Friedens-richter der Grafschaft
Bedford, wurde ein in vielen Verfolgungen standhafter Quäker.


% \picinclude{./090-099/p_s099.jpg} 
Brief an den Papst. Die Studenten von Cambridge usw. 99
Und in der Umgegend war auch eine Versammlung und es trieb
mich hin zu gehen, obwohl es mehrere Meilen weit weg war.
John Crook ging mit mir. Es war einer dort, Gritton, der
Baptist gewesen, aber jetzt höher hinaus wollte und sich ein Prüfer
der Geister nannte. Gr sagte den Leuten, wie viel Vermögen
sie haben, und behauptete, ihnen sagen zu können, wenn ihnen
etwas gestohlen oder verbrannt wurde, wer es getan. Dadurch
hatte er die Gunst vieler erworben. Dieser Mann redete gerade
laut, als ich kam. Er hieß Alexander Parker seine Hoffnung
begründen. Alexander erwiderte: ,,C-hristus ist unsere Hofstiung;«
weil diese Antwort nicht so schnell gegeben wurde, wie er sie
erwartete, so schrie er: »sein Mund ist gestopst!« Daraus richtete
er sich an mich, denn ich stand schweigend dabei, weil er vieles
sagte, das sich nicht mit der Schrift vertrug. Jch fragte ihn,
ob er sich aus die Schrift berufen könne? er sagte: ,,ja;« ich hieß
die Leute ihre Bibeln nehmen und die Stellen aussuchen, die
er angeben würde, aber er konnte es nicht. So war er beschämt
und ging fort und seine Anhänger wurden meistens gewonnen ....
John Erook blieb in der Kraft Gottes, aber er wurde seines
Amtes als Richter entsetzt ....
Jch ging nach Romney, wo die Leute von meinem Kommen
gehört, und es war darum eine sehr große Versammlung. Zu
dieser kam Samuel Fischer 1), ein großer Baptistenprediger. E-r
hatte eine Pfarrei sgehabt, die ihm etwa zweihundert Pfund im
Jahre eingebracht hatte und die er um des Gewissens willen auf-
gegeben hatte. Der Pfarrer der Baptisten war auch dabei und
viele ihrer Leute. Die Kraft des Herrn ward so mächtig kund,
daß viele ergriffen wurden ..... Als die Versammlung vorüber
war, sagte Samuel Fischers Frau: »so, nun laßt uns darüber
reden, was geistig und was fleischlich ist, damit wir die Lehre
des Geistes von der Lehre des Fleisches [unterscheiden könne«n.«
Samuel Fischer und manche andere traten für das Wort des
Lebens ein, das an diesem Tage ihnen war erklärt worden. Der
andere Pfarrer unds seine Anhänger redeten dagegen .....
Samuel Fischer nahm die Wahrheit an und wurde ein getreuer
Prediger; er predigte umsonst und arbeitete viel für den Herrn;
1) Samuel Fischer und John Stubbs gingen später u. a. nach Rom und
traten dort mutig gegen papistischen Aberglauben auf. Fischer starb 1665 im
Gefängnis in London an der Pest.
 

