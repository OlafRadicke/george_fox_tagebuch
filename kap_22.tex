% \picinclude{./250-259/p_s251.jpg} 
%%%%%%%%%%%%%%%%%%% Kapitel 22. %%%%%%%%%%%%%%%%%%%%%%%%%%%%%%

\chapter[Reise nach Holland und Deutschland]{Reise nach Holland und Deutschland}

\begin{center}
\textbf{Reise nach Holland. Einrichtung der kirchlichen Ordnung für
Holland und Deutschland. Briefwechsel mit Prinzessin Elisabeth.
Reise nach Deutschland bis Oldenburg. Briefe an verschiedene
Behörden von Holland und Deutschland.}
\end{center}


Als das Schiff bereit war, nahmen wir, die ausersehen waren,
nach Holland zu gehen, von den Freunden Abschied und gingen
an Bord, am Abend des 25. des 5. Monates 1677\index{Jahr!1677} [...]
Am Morgen des 28. kamen wir in Rotterdam\ort{Rotterdam} an. Wir
hatten eine gute Überfahrt gehabt, gelobt sei der Name des Herrn
immerdar [...]

Von Rotterdam gingen wir über Ouderkerk und Delft nach
Leyden, welches 18 engl. Meilen von Rotterdam ist, wir hatten
5 Stunden zu fahren, denn unser Boot wurde Von einem Pferd
gezogen, das am Ufer entlang ging. Tags darauf gingen wir
nach Haarlem\ort{Haarlem}, wo wir eine zahlreiche Versammlung hatten, und
von da in Begleitung mehrerer Freunde nach Amsterdam\index{Amsterdam}.

Am Tag nach unsrer Ankunft war die Vierteljahresversammlung 
in Amsterdam, zu welcher Freunde von Haarlem und Rotterdam\ort{Rotterdam}
kamen und mit ihnen diejenigen unserer Gefährten, die wir
in Rotterdam zurückgelassen hatten, nämlich Robert Barclay\person{Barclay, Robert},
George Keith\person{Keith, George} und seine Frau und andere. Die Versammlung, die
im Hause von Gertrud Dirick Nieson\person{Nieson, Gertrud Dirick} 
stattfand, war sehr zahlreich und segensreich, denn wir beide, 
William Penn\person{Penn, William} und ich, wurden
getrieben, uns über die Ordnungen des Evangeliums auszusprechen
und den Segen und Nutzen der jährlichen, vierteljährlichen und 
monatlichen Versammlungen für Männer und Frauen zu zeigen. Am
folgenden Tage hatten wir wieder eine Versammlung bei Gertrud
Diriek, mehr öffentlich\index{Öffentliche Versammlung} und sehr 
zahlreich; \textit{Fromme} aller Art
wohnten derselben bei, und der Weg des Lebens und Heils wurde
ihnen eingehend und lebendig dargelegt. Sie hörten sehr 
aufmerksam zu und niemand erhob irgend einen Widerspruch gegen
das, was verkündet wurde. Am Nachmittag hatten wir abermals
eine Versammlung daselbst, aber im engeren Kreise.\index{Mehere 
Versammlungen pro Tag} Am Tage
darauf hatten wir eine Versammlung nur für 
Freunde,\index{Geschlossene Geschäftsversammlung} in welcher
nach allgemeinem Übereinkommen Verschiedene Versammlungen 
eingerichtet wurden, monatliche, Vierteljährliche und jährliche, welche
in Amsterdam gehalten werden sollten für die Freunde aller
% \picinclude{./250-259/p_s252.jpg} 
Provinzen Hollands, sowie von Emden\ort{Emden}, der 
Pfalz\ort{Pfalz}, Hamburg\ort{Hamburg} Friedrichsstadt\ort{Friedrichsstadt}, 
Danzig\ort{Danzig} und andern Orten Deutschlands\ort{Deutschland} und der
umliegenden Länder. Darüber waren die Freunde sehr froh, und
die Wahrheit wurde dadurch gefördert [...]

An einem Ersten Tage hatten wir eine sehr zahlreiche Versammlung, 
zu welcher viele Leute der verschiedensten Richtungen
herbei strömten: Baptisten\index{Baptisten}, Socinianer\index{Socinianer}, 
Seeker\index{Seeker}, Brawnisten\index{Brawnisten} und
einige Studierende. George Keith, Robert Barclay, William Penn
und ich verkündeten die ewige Wahrheit unter ihnen, indem wir
ihnen den Zustand des gefallenen Menschen dartaten und ihnen
zeigten, wie Männer und Frauen in den Stand der Wiedergeburt
durch Jesum Christum gelangen können; das Geheimnis der Sündhaftigkeit 
und das Geheimnis der Gottseligkeit wurden deutlich
ausgelegt, und die Versammlung endete ruhig und befriedigend.
Tags darauf ließen Robert Barclay, George Keith und William
Penn mich und etliche andere Freunde in Amsterdam zurück und
gingen nach Deutschland\ort{Deutschland} weiter, wo sie Viele hundert Meilen weit
umherreisten und für den Herrn wirkten, Benjamin Furly\person{Furly, Benjamin} ging mit
ihnen als Dolmetscher [...]

Ich schrieb einen Brief an die Prinzessin Elisabeth,\person{Prinzessin Elisabeth}
welchen Isabel Yeomanz\person{Yeomanz, Isabel} ihr überbrachte, 
als George Keithö Frau und sie sie besuchten:

\brief{Prinzessin Elisabeth}{
  Prinzessin Elisabeth!\footnote{Prinzessin Elisabeth war die 
  älteste Tochter Friedrichs V., König von
  Böhmen; sie hatte ihre Residenz in Herwerden. Sie war eine musterhafte
  Christin und hielt große Stücke auf die Qnäker und ihre Grundsätze, wie ihre
  Briefe an verschiedene Personen am englischen Hof bezeugen.}

  \bigskip 

  Ich habe durch Freunde, die dich besucht haben, sowie aus
  einigen deiner Briefe, die ich gesehen, erfahren, das du dem Herrn
  und seiner heiligen Wahrheit zugetan bist; es ist ein Großes,
  wenn Leute deines Standes einen so empfänglichen Sinn haben
  den Herrn und seine köstliche Wahrheit, während so viele in
  Üppigkeit und den Vergnügen dieser Welt untergehen; zwar 
  bekennen sich alle äußerlich in irgend einer Weise zu Gott und
  Christus, ohne jedoch innerlich etwas Tieferes für ihn zu empfinden.
  Denn nicht viele Mächtige noch Weise dieser Welt 
  (1. Kor. 1\bibel{Kor. 1. 01@1. Kor. 1})
  sind imstande um Christi willen Narren zu werden, oder sich von
  ihrer Höhe herab in die Demut Christi zu erniedrigen, durch
  welche sie einen viel mächtigeren Stand und ein mächtigeres Königtum 
  % \picinclude{./250-259/p_s253.jpg} 
  erlangen durch den heiligen Geist, das ewige Licht und die
  ewige Kraft Gottes, und eine höhere Weisheit, die von oben ist,
  rein und voll Frieden, die Weisheit, die über allem dem Irdischen,
  Sinnlichen und Teuflischen steht, wodurch die Menschen einander
  umbringen, sogar um der Religion, der Kirche und der Gottesdienste 
  willen. Solches haben sie nicht von Gott noch von Christus.
  Die Weisheit von oben, durch welche alle Dinge geschaffen sind,
  und die mit der Gottesfurcht in den Herzen beginnt, erhält die
  Herzen, rein und diese Weisheit soll alle Kinder Gottes regieren,
  und durch sie sollen sie alle Dinge zur Ehre Gottes tun. Dieses ist
  \zitat{die Weisheit, die sich rechtfertigt durch ihre Kinder} 
  (Math. 11,19)\bibel{Math. 11:19}.
  Es ist mein Wunsch, das du in dieser Weisheit und Gottesfurcht
  bewahret bleibest zur Ehre Gottes. Denn der Herr ist gekommen,
  um sein Volk selbst zu lehren und ein Panier aufzurichten unter
  den Völkern, aus das die Völker herzueilen. Viele sind seit den
  Tagen der Apostel abgefallen vom göttlichen Licht Christi, das
  ihnen hätte die Erleuchtung von der Erkenntnis der Klarheit
  Gottes in dem Angesicht Jesu Christi geben sollen und vom heiligen
  Geist, der sie in alle Wahrheit geleitet hätte und vom heiligen
  Glauben, dessen Anfänger und Vollender Christus ist, welcher
  die Herzen reinigt und Sieg gibt über alles, was von Gott trennt,
  durch welchen Glauben man Zugang hat zu Gott und in welchem
  man Gott angenehm ist, dessen Geheimnis bewahret ist in reinen
  Herzen. Sie sind abgefallen vom Evangelium, das in den Tagen
  der Apostel gepredigt wurde und welches in Mann und Frau
  Leben und unsterbliches Wesen ans Licht bringt, und durch das
  die Menschen den Teufel, der sie verfinstert hat, überwinden
  sollten; dieses Evangelium sollte alle, die es aufnehmen, im Leben
  und unsterblichen Wesen bewahren. Sie haben aus die Menschen
  gesehen und nicht auf den Herrn, der sein Gesetz allen Kindern
  des neuen Bundes in die Herzen schrieb, in welchem Licht, Leben
  und Gnade ist und durch welche alle, vom Höchsten bis zum
  Niedrigsten, zur Erkenntnis Gottes kommen, so das die 
  Gotteserkenntnis die Erde bedecket wie das Wasser das Meer. Dieses
  Wirken des Herrn hebt nun wieder an, wie in den Tagen der
  Apostel; die Menschen sollen \zitat{eine Salbung haben von dem, der
  heilig ist und durch den sie alles wissen} 
  (1. Joh. 2,20\bibel{Joh. 1. 02:20@1. Joh. 2:20}) und 
  \zitat{bedürfen nicht, das sie jemand lehre, sondern die Salbung 
  lehret sie} (1. Joh. 2,27)\bibel{Joh. 1. 2:27@1. Joh. 2:27} 
  [...] Jetzt kommen die Menschen wieder
  % \picinclude{./250-259/p_s254.jpg} 
  zurück vom Abfall in das Licht und den Geist Christi, und
  empfangen den Glauben von ihm und nicht mehr von Menschen,
  und empfangen von ihm das Evangelium und die Salbung und
  das Wort [...]. Denn Gott ist in seinem Sohne Jesus Christus
  gekommen, sein Volk selbst zu lehren und es abzubringen von
  den Wegen dieser Welt, zu Christus, dem Weg, der Wahrheit
  und dem Leben, welcher ist der Weg zum Vater, und von allen
  Lehrern und Predigern der Welt zu ihm, dem wahren Lehrer
  und von allen weltlichen Gottesdiensten Gott zu dienen
  im Geist und in der Wahrheit, welche Art des Gottesdienstes
  Christus vor 1600 Jahren eingesetzt hat, als er den jüdischen\index{Juden}
  Gottesdienst\index{Gottesdienst} im Tempel zu Jerusalem\ort{Jerusalem} 
  zu nichte machte und den
  Gottesdienst auf jenem Berge beim Jakobsbrunnen, um die
  Menschen von allen weltlichen Religionen, die eingeführt wurden
  seit den Tagen der Apostel, zu der von Christus und den Aposteln
  ausgerichteten Religion zu bringen, die rein und unbefleckt vor
  Gott ist und von der Welt unbefleckt erhält (Jac. 1)\bibel{Jac. 1}, und von den
  weltlichen Kirchen\index{Weltliche Kirche} und Gemeinschaften, die seit den Tagen der
  Apostel entstanden sind, in die Kirche, die in Gott dem Vater
  unseres Herrn Jesus Christus ist, in die Einigkeit und Gemeinschaft
  des heiligen Geisteis\index{Heiligen Geist}, welcher tötet\index{Töten}, 
  beschneidet\index{Beschneidung} und tauft\index{Taufe}, um die
  Sünde und Verderbtheit, welche durch die Übertretung in den
  Männern und Frauen entstanden sind, auszutilgen. In diesem
  heiligen Geist ist heilige Gemeinschaft und Einigkeit, er ist das
  Unterpfand des Fürsten der Fürsten, des Königs aller Könige,
  er ist das Band des Friedens des Herrn aller Herren; diesen
  himmlischen Frieden sollen alle wahren Christen sich erhalten, mit
  geistlichen Waffen, nicht mit fleischlichen.\index{Geistliche Waffen}
  Es haben nun, meine Freundin, die heiligen Männer Gottes
  die Schriften geschrieben\index{Bibel}, wie sie vom heiligen Geist getrieben
  wurden, und die ganze Christenheit ist in Verwirrung wegen dieser
  Schriften, weil alle nicht von demselben heiligen Geist geleitet sind,
  wie die, welche die Schrift geschrieben; diesen heiligen Geist müssen
  sie in sich einziehen lassen und sich von ihm leiten lassen, wenn\index{Exegese}
  sie in alle Wahrheit kommen wollen und in den Trost Gottes
  und Christi. Denn niemand kann Jesus Herr nennen,\index{Christus bekennen} als allein
  durch den heiligen Geist und alle, die Christum ohne den heiligen
  Geist Herr nennen, missbrauchen seinen Namen. Alle aber, die
  seinen Namen nennen, müssen mit der Sünde\index{Sünde} brechen, dann nennen
  % \picinclude{./250-259/p_s255.jpg} 
  sie seinen Namen in Gerechtigkeit und Wahrheit. Achte doch darum 
  auf die Gnade und Wahrheit, die durch Jesum Christum
  in dein Herz gekommen sind, um dich zu lehren, wie du leben
  und was du verleugnen sollst, sie wollen dein Herz stärken und
  deine Rede angenehm machen, dir das Heil bringen und dir
  immerdar ein Lehrer sein. Durch sie wirst du Christus aufnehmen,
  und alle, die ihn aufnehmen, denen gibt er Macht, nicht nur dem
  Bösen zu widerstehen, sondern Gottes Kinder\index{Gottes Kinder} zu sein und als
  Kinder dann Erben eines Lebens, einer Welt und eines Reiches
  ohne Ende und ewiger Reichtümer und Schätze. Dieses in Eile,
  meine Liebe im Herrn Jesus Christus, welcher den Tod gekostet
  für einen jeden und der Schlange den Kopf zerbrach, der zwischen
  Gott und dem Menschen stehet, damit durch ihn der Mensch wieder
  zu Gott zurückkehre. Er ist der himmlische Fels und geistige
  Grund, auf den das Volk Gottes sich gründen muss. Gelobt sei
  der Herr immerdar.

  \bigskip 

  \begin{flushright}Amsterdam, 7. dez 6. Monats 1677. G. F.\end{flushright}

  \bigskip 

  PS. Überbringerin dieses ist eine meiner Schwiegertöchter,
  die mit Gertrud Tirick Nieson\person{Nieson, Gertrud Tirick} 
  und George Keiths Frau\person{Keiths, George Frau} dich 
  besuchen will.
}

Die Antwort der Prinzessin Elisabeth\person{Prinzessin Elisabeth}:

\brief{Quaker-Gemeinde}{
Lieber Freund!

  \bigskip 

Ich muss alle lieb haben, die den Herrn Jesus lieben und
denen es gegeben ist, nicht nur an ihn zu glauben, sondern auch
um seinetwillen zu leiden. Darum habe ich mich über euern
Brief wie auch über den Besuch eurer Freunde sehr gefreut, und
ich werde ihren und euren Rat befolgen nach dem Maß der
Salbung und des Lichts, das Gott mir schenken wird.
Eure euch stets liebende Freundin

  \bigskip 

\begin{flushright}
Hertford\ort{Hertford}, 30. August 1677\index{Jahr!1677} [...] Elisabeth\end{flushright}
}

Wir gingen über Altmaar, Horn, Harlingen, Leuwarden, Groningen 
nach Delfziel; dieses ist eine Stadt am Flusse Ems\ort{Ems (Fluss)}, über
den wir dann nach Emden\ort{Emden} gelangten, hier waren die Freunde
grausam verfolgt und oft auch verbannt worden.\index{Verbannung} Ich ging nach
einer Herberge und aß zu Mittag mit einigen Leuten, die englisch
konnten; wir unterhielten uns gut, und sie waren sehr empfänglich. 
[...] Von hier gingen wir über Leer, Stickhausen, Detern
nach Apre in Dänemark (?); unterwegs trafen wir mit dem
Grafen von Oldenburg\person{Grafen von Oldenburg} zusammen, 
der zum Friedensvertrag zu
% \picinclude{./250-259/p_s256.jpg} 
Lembach\index{Friedensvertrag zu Lembach} ging. Von da gingen wir nach 
Oldenburg\ort{Oldenburg} [...] und
dann weiter nach Delmenhorst.\ort{Delmenhorst} Von hier ging es zu Wagen nach
Bremen,\index{Bremen} einer stattlichen Stadt in Deutschland, hier gingen wir
an ein Wasser, genannt Overdeland,\ort{Overdeland} und nahmen ein Boot und
fuhren nach Fischerhude\ort{Fischerhude} [...]. Hier nahmen wir wieder einen
Wagen und fuhren durchs Land des Bischofs von Münster\index{Bischof von Münster} 
nach Klosterseven, [...] dann am anderen Tage nach Buxtehude.\index{Buxtehude}
Die Leute in dieser Gegend des Bischofs von Münster waren
sehr im Dunkeln. Während der Reise predigte ich die Wahrheit
unter ihnen und wies sie auf den großen, wichtigen Tag des
Herrn und ermahnte sie zur Nüchternheit, und das sie auf den
guten Geist Gottes in ihrem Innern achten sollten.

Von hier gingen wir so schnell wie möglich, teils zu Wasser,
teils zu Wagen, nach Hamburg.\ort{Hamburg} Hier kamen wir zeitig genug an,
um noch am gleichen Abend eine Versammlung abzuhalten, sie
war sehr schön und erhebend. Es wohnte ihr unter anderem ein
Baptistenprediger mit seiner Frau bei, und ein angesehener Mann
aus Schweden und seine Frau, und alles verlief ruhig, gelobt sei
der Herr. Aber es ist ein arger, verfinsterter Ort und die Leute
sind der Wahrheit wenig zugänglich.

Es lebte eine Frau hier, die zur Zeit John Perrots\person{Perrot, John} gegen
mich geredet hatte, obschon sie mich bis jetzt nie gesehen hatte;
das hatte sie seither immer beunruhigt und sie war nun froh,
Gelegenheit zu haben, ihren Fehler einzugestehen, sie tat es auch
sehr bereitwillig, und ich vergab ihr ebenso bereitwillig und
völlig.\index{Vergebung}

Wir reisten weiter [...] nach Friedrichsstatt;\ort{Friedrichsstatt} dort gingen
wir zu William Pauls,\person{Pauls, William} wo mehrere Freunde zu uns kamen,
denn es ist eine ziemliche Anzahl von Freunden in dieser Stadt.
Wir hatten am Abend eine schöne erbauliche Versammlung miteinander, 
die uns unsre Ermüdung vergessen lies, denn wir waren
sehr müde gewesen, nachdem wir zwei Tage ununterbrochen gereist 
waren, oft ganz durchnässt vom Regen in den offenen
Wagen. Aber der Herr fügte uns alles zum besten, und wir
freuten uns, die Freunde zu sehen, gelobt sei sein heiliger Name
immerdar.

Die Stadt steht unter der Herrschaft des Herzogs von 
Holstein,\person{Herzogs von Holstein} 
welcher die Freunde gerne aus Stadt und Land verbannt
hätte und darum den Behörden Befehl gab, solches zu tun; diese
% \picinclude{./250-259/p_s257.jpg} 
aber erklärten, eher wollten sie ihre Ämter niederlegen, denn die
Freunde seien in diese Stadt gekommen, um Gewissensfreiheit zu
haben. So erfreuen sich denn die Freunde dort ihrer Freiheit,
und sie stehen in gutem Ansehen zu Stadt und Land.

An einem Ersten Tage hatte ich eine Versammlung, zu der
viele Leute kamen, auch etliche Widerspenstige, aber der Herr bannte
sie alle, und der Same des Lebens breitete sich aus über allen.\index{Störung der Andacht}
Während meines dortigen Aufenthaltes hatte ich eine Unterredung
mit einem Juden\index{Juden}, einem Leviten, über das Kommen des Messias,
er wurde sehr zuschanden gemacht in dem, was er sagte, doch 
benahm er sich anständig und lud mich in sein Haus ein. Ich ging
hin und unterhielt mich dort mit einem andern Juden, der mir
ihren Talmud\index{Talmud} und andere jüdische Bücher zeigte, aber sie sind
sehr in der Finsternis und Verstehen ihre eigenen Propheten nicht.

Es war ein Baptistenlehrer\index{Baptisten} in dieser Stadt, welcher 
Schmähungen und Lügen über die Freunde ausgestreut hatte, darum
ging John Claus\index{Personenn!Claus, John} mit zwei Freunden zu 
ihm in seine Wohnung und reinigte die Freunde und die Wahrheit von seinen 
Verleumdungen\index{Verleumdungen} und machte ihn zuschanden, indem er die 
Schmähungen gegen ihn selbst wendete.

Ehe wir weiterreisten, hatte ich noch eine Versammlung, ausschließlich 
für Freunde, in der ich ihnen den Nutzen der 
Monatsversammlungen\index{Monatsversammlung(Aufgaben)} 
vorstellte, um nach den Armen zu sehen und Sorge
zu tragen, das die Vermählungen und andere kirchliche Angelegenheiten 
sich in guter Ordnung Vollzögen. Da solches mit dem
Zeugnis Gottes in ihren Gewissen übereinstimmte, beschlossen sie
bereitwillig, fürderhin Monatsversammlungen unter sich 
abzuhalten, damit alle, Männer wie Frauen, sich der äußern kirchlichen
Angelegenheiten annehmen möchten [...].

Da ich mich nach dieser Versammlung meiner Pflichten gegen
diese Gegend entledigt fühlte, verabschiedeten wir uns von den
Freunden und gingen wieder nach Hamburg\ort{Hamburg}.
Am Tage nach unserer Ankunft hatten wir eine sehr schöne,
friedliche Versammlung. Nach derselben hatte ich eine 
Unterredung mit einem Schweden, der zu den Großen seines Landes
gehörte. Er war aus demselben verbannt worden um seines
Glaubens willen und darum nach Hamburg gekommen und hatte
schon früher einer meiner Versammlungen beigewohnt. Als ich
fertig war mit ihm, hatte ich noch eine Unterredung mit einem
% \picinclude{./250-259/p_s258.jpg} 
Baptisten über die sogenannten Sakramente\index{Sakramente}. 
Beide male gelang es mir, ihnen die Wahrheit zu zeigen.

Als ich mich auch in Hamburg meiner Ausgabe entledigt
fühlte, gingen wir in einem Boot in eine Stadt, die zum Lande
des Herzogs von Lüneburg gehörte. Von da gingen wir wieder
nach Bremen\ort{Bremen} zurück, teils zu Wasser, teils zu Land. Unterwegs
hatte ich oft gute Gelegenheit, den Leuten die Wahrheit zu 
verkünden, besonders in einem Marktflecken, wo wir die Pferde
wechselten; ich verkündete ihnen den Tag des 
Herrn\index{Tag des Herrn}, der über alles 
Fleisch kommen werde, und ermahnte sie zur Rechtschaffenheit.
Ich sagte ihnen, das Gott gekommen sei, sein Volk selber zu
lehren, und das sie sich zu ihm kehren und aus das Lehren seines
Geistes in ihren Herzen hören sollten.

In Bremen gingen wir in eine Herberge, bis ein anderer
Wagen zur Weiterfahrt bereit war. Obgleich ich fühlte, das die
Kraft Gottes mit dieser Stadt war und die unsauberen Geister
nieder hielt, so litt doch mein Geist sehr um dieser Leute willen.
Als unser Wagen bereit war, fuhren wir nach Keby, wo wir die
Nacht zubrachten, und früh am nächsten Morgen weiter nach
Oldenburg\ort{Oldenburg}. Es war ein trauriger Anblick, wie diese große 
stattliche Stadt niedergebrannt war! Wir begaben uns in eine
Herberge, wo, obgleich es der Erste Tag war, die Soldaten beim
Trinken und Talerschieben waren, und in den wenigen übrig 
gebliebenen Häusern waren die Verkaufsläden offen, und die Leute
handelten miteinander. Es trieb mich, die Wahrheit unter ihnen
zu verkünden und sie vor dem Gericht Gottes zu warnen; sie
hörten mich ruhig an, aber dennoch lastete ihre Schlechtigkeit
schwer auf mir [...].

Am nächsten Tage gingen wir nach Leer, durch viel tiefes
Wasser, und am folgenden nach Emden\ort{Emden} [...] Tags darauf, zu
Schiff, nach Delfziel\ort{Delfziel}. In der Herberge kam ein Freund zu uns,
der hier lebte, weil er häufig aus Emden verbannt worden war;
er war Goldschmied und hatte Haus und Geschäft in Emden gehabt. 
Er war, trotz wiederholter Verbannung, immer wieder
dorthin gegangen, bis sie ihn schließlich bei Wasser und Brot ins
Gefängnis taten, ihm Hab und Gut nahmen, ihn mit Weib und
Kind verbannten und ihm nichts mehr zum Unterhalt oder
Aufenthalt ließen. Wir sprachen ihm Mut und Trost\index{Trost} im Herrn
zu und ermahnten ihn, dem ihm anvertrauten Zeugnis treu zu
% \picinclude{./250-259/p_s259.jpg} 
bleiben. Nachdem wir uns von ihm verabschiedet hatten, gingen
wir noch am gleichen Tage in einem Boot nach Groningen\ort{Groningen}, wo
wir Cornelius Andries\person{Andries, Cornelius} trafen, 
einen Freund, der auch viel unter
Gefängnis und Verbannung von Emden gelitten hatte.

Von da gingen wir [...] über Amsterdam [...] nach Harlingen\ort{Harlingen}.
Am Tage nach unserer Ankunft war hier die Monatsversammlung
für Männer und Frauen. Sie war gut besucht und sehr schön.
Es wurde beschlossen, das jeden Monat eine Versammlung stattfinden 
solle, sowohl für Männer als für Frauen, um für die
Angelegenheiten der Kirche zu sorgen.

Am Nachmittag hatten wir eine öffentliche Versammlung,
zu welcher Leute der verschiedensten Richtungen kamen: 
Socinianer\index{Socinianer}, Baptisten\index{Baptisten}, 
Lutheraner\index{Lutheraner} und andere, worunter auch ein Doktor der
Medizin und ein Priester. Nachdem ich ihnen eingehend die
Wahrheit verkündet hatte und den glücklichen Zustand beschrieben,
in welchem die Menschen gewesen, als sie sich noch von Gott
lehren ließen und im Paradies Gottes blieben, und andrerseits
das Elend und den Jammer, die über sie kamen, seit sie Gottes
Lehre verließen und aus die Schlange hörten, Gottes Gebote 
übertraten und aus dem Paradies vertrieben wurden, und ihnen dann
den Weg zeigte, aus dem sie wieder in jenen glücklichen Zustand
zurückkehren könnten -- da, gerade als ich aufhörte zu reden,
stand ein Priester, ein ernster alter Mann, auf, nahm seinen Hut
ab und sagte: \zitat{Ich bitte Gott, das er diese Lehre fördere und
bestätige, denn sie ist Wahrheit und ich habe nichts gegen sie}.
Er wäre gerne bis ans Ende dieser Versammlung geblieben, aber
da er selber an diesem Abend zu predigen hatte und die Zeit
seines Gottesdienstes gekommen war, so konnte er nicht länger
bleiben. Nachdem er dieses Zeugnis für die Wahrheit abgelegt
hatte, eilte er fort, um nachher wieder kommen zu können. Er
kam auch wieder, aber erst als die Versammlung schon zu Ende
war. Nach der Versammlung hatte ich noch eine andere nur mit
Freunden im Hause Hessel Jakobs\person{Jakobs, Hessel}, 
wohin auch der Doktor der
Medizin kam, um mit William Penn\person{Penn, William} zu reden, 
und dieser verkündete mit Erfolg die Wahrheit. Durch diesen Doktor schickte
mir jener Priester einen Gruß und lies mir sagen, er habe eine
halbe Stunde früher als gewöhnlich aufgehört zu predigen an
diesem Abend, damit er wieder in unsre Versammlung kommen
könnte, um noch mehr von dieser guten Lehre zu hören.
% \picinclude{./260-269/p_s260.jpg} 
Ich hörte später, das seine Hörer ihn wegen dessen, was er
in unsrer Versammlung gesagt, zu Rede stellten, und als er dazu
stand, ihn bei den andern Priestern der Stadt verklagten, die ihn
darüber zur Rechenschaft zogen, aber vom Ausgang der Sache
konnte ich nichts erfahren [...].

Am folgenden Tage gingen wir nach Amsterdam\ort{Amsterdam}, wo wir
etwas nach Mitternacht ankamen, und da die Tore geschlossen
waren, so blieben wir bis zum folgenden Tage auf dem Schiff
und gingen dann ins Haus von Gertrud Ditick\person{Ditick, Gertrud}; hier besuchten
uns viele Freunde, froh, das wir wohlbehalten wieder zurück
waren. Am folgenden Tage fühlte ich mich im Geist beunruhigt
wegen etlicher verführerischer Geister, die Uneinigkeit\index{Uneinigkeit} unter die
Freunde brachten, und weil ich merkte, das sie sich suchten in Gunst
zu bringen, so trieb es mich, einige Zeilen deswegen an die
Freunde zu schreiben: 

\brief{Quaker-Gemeinde}{
  Alle, die sich in die Gunst der 
  Leute einschmeicheln wollen, trachten, sich in Gunst zu 
  bringen statt Christus.
  Aber Freunde, euer friedsames Bleiben in der Wahrheit, die ewig
  ist und sich nicht verändert, wird alles, was nicht aus der 
  Wahrheit ist, überdauern, auch wenn es mit noch so viel Worten 
  austritt. Lasset denn die, welche so für J. S. und J. W. auftreten
  zu ihnen halten und sich von euch trennen, und ihr, die ihr 
  Zeugnis abgelegt gegen diesen Geist, beharret bei diesem Zeugnis, bis
  sie euch mit Anschuldigungen angreifen. Zanket nicht, lasset euch
  nicht ein mit etwas, das nicht in der Wahrheit steht, noch suchet
  lebendig zu erhalten, was sollte Gott zum Opfer gebracht werden,
  damit ihr nicht des Reiches verlustig geht.

  \begin{flushright}Amsterdam\ort{Amsterdam}, 14. des 7. Monats 
  1677\index{Jahr!1677}. G. F.\end{flushright}

}

An einem grosen Fasttage wohnte ich einer der Versammlungen 
der Freunde bei. Ich hatte eigentlich vorgehabt, nach
Haarlem\ort{Haarlem} zu gehen, aber ich wurde in meinem Geist gehalten, zu
bleiben. Wir hatten eine sehr große Versammlung, eine große
Menge Leute strömte herbei, worunter viele angesehene Personen.
Die Kraft des Herrn war über der Versammlung, und in den
Offenbarungen\index{Offenbarungen}, die ich während derselben 
hatte, trieb es mich,
darzutun, das niemand, mit allem Studieren und allem Verstand
oder mit dem Lesen der Geschichte, wenn er sie nach seinem eigenen
Willen lese, die Abstammung von Christus wisse, der nicht nach
dem Willen eines Menschen, sondern nach dem Willen Gottes
gezeuget sei. [...] Nachdem ich ihnen das ausführlich erklärt hatte,
% \picinclude{./260-269/p_s261.jpg} 
erklärte ich ihnen den Unterschied zwischen wahrem und falschem
Fasten\index{Fasten}. Ich zeigte ihnen, wie alle, ob sie sich 
nun Christen, Juden\index{Juden} oder Türken\index{Türken} nennen, 
nicht in der rechten Weise fasten, sondern sie
fasten, \zitat{das sie hadern und zanken und mit gottloser Faust schlagen}
(Jes. 58,4)\bibel{Jes. 58:04@Jes. 58:4}, sie erheben nicht 
reine Hände zu Gott. Und wenn sie
schon vor den Leuten tun, als ob sie fasteten, und \zitat{des Tags den
Kopf hängen wie ein Schilf, so ist es doch nicht das Fasten, das
Gott erwählt}(Jes. 58,5\bibel{Jes. 58:05@Jes. 58:5}). Darum sind 
ihre Gebeine vertrocknet,
und wenn sie den Herrn anrufen, so hört er sie nicht und \zitat{ihre
Besserung wächst nicht}(Jes. 58,8\bibel{Jes. 58:08@Jes. 58:8}), 
weil sie ihr eigenes Fasten
halten und nicht das des Herrn. Ich ermahnte sie, das Fasten
des Herrn zu halten, welches ein Fasten von der Ungerechtigkeit
und der Sünde sei, vom Streiten, Hadern und Unterdrücken, und
auch allen bösen Schein zu meiden. Die Leute, die Fasttage
hielten, wunderten sich sehr über diese Eröffnungen, und die 
Versammlung nahm ein schönes friedliches Ende.

Am folgenden Tage ging ich nach Haarlem\ort{Haarlem}, wo ich zuvor eine
Versammlung angesagt hatte. Peter Hendricke\person{Hendricke, Peter} 
und Gertrud Dirick Nieson\person{Nieson, Gertrud Dirick} 
gingen mit mir und wir hatten eine gesegnete Versammlung. 
Es waren verschiedene \textit{Fromme} dabei, auch ein
Priester der Lutheraner\index{Lutheraner}, der mehrere 
Stunden andächtig zuhörte, \index{Andachtsdauer}
während ich ihnen die Wahrheit verkündete, Gertrud verdolmetschte. 
Als die Versammlung zu Ende war, sagte der Priester,
er habe nichts darin gehört, das nicht nach dem Worte Gottes
gewesen wäre, und er wünschte uns, das der Segen Gottes mit
uns und unsren Versammlungen sein möge. Auch andere erklärten,
man habe ihnen noch nie zuvor die Dinge so verständlich 
auseinandergesetzt.

Wir brachten die Nacht im Hause eines Freundes, Dirirk
Klassen\person{Klassen, Dirirk}, zu, am folgenden Tage 
kehrten wir nach Amsterdam zu
Gertrud Dirick zurück; wir waren noch nicht lange da, als ein
berühmter Priester kam, der früher unter dem deutschen Kaiser
gestanden, mit einem anderen deutschen Priester, um mit mir zu
reden. Ich ergriff die Gelegenheit, um ihnen den Weg der
Wahrheit zu erklären, indem ich ihnen zeigte, wie sie dazu kommen
können, Gott und Christus und sein Evangelium und Gesetz zu
kennen; ich zeigte ihnen, das sie niemals durch Studieren\index{Studieren} und
durch Philosophie\index{Philosophie} dazu kommen können, sondern durch göttliche
Offenbarung,\index{Offenbarung} durch den Geist Gottes, der ihnen in der Stille des
% \picinclude{./260-269/p_s262.jpg} 
Herzens kund werde. Die Beiden waren empfänglich und gingen
befriedigt fort.

Am folgenden Ersten Tage war ich in einer Versammlung
der Freunde in Amsterdam; außer vielen Verschiedenen \textit{Frommen}
war auch ein Doktor aus Polen anwesend, der um seiner Religion
willen aus seiner Heimat verbannt war; während der 
Versammlung wurde er ergriffen vom Zeugnis der Wahrheit und kam
nachher zu mir und wünschte eine Unterredung mit mir, und
nachdem wir eine Zeitlang miteinander geredet, und ich ihm die
Dinge noch mehr erklärt hatte, ging er sehr empfänglich und in
Liebe zur Wahrheit fort.

Während ich in Amsterdam war, brachte ich viel Zeit mit
Schreiben für die Wahrheit zu. Ich schrieb von hier mehrere Briefe
an die Freunde in England, ebenso: \zitat{Eine Warnung an die
Bewohner der Stadt Oldenburg}\index[buch]{Eine Warnung an die
Bewohner der Stadt Oldenburg}\ort{Oldenburg}, die kürzlich abgebrannt war,
ferner: \zitat{Eine Warnung an die Bewohner der Stadt Hamburg}
\index[buch]{Eine Warnung an die Bewohner der Stadt Hamburg}\ort{Hamburg} [...]
Ferner schrieb ich einen Brief an die Gesandten, die zu 
Nymwegen über den Frieden verhandelten [...].

Ich schrieb auch an die Behörden und Priester von Emden\ort{Emden},
um ihnen zu zeigen, wie unchristlich es sei, die Freunde zu 
verfolgen. Mehrere andere Bücher schrieb ich, Antworten an Priester
und andere, in Hamburg\ort{Hamburg}, Danzig\ort{Danzig} und 
anderwärts, um die Freunde
und die Wahrheit von allen Beschuldigungen und Verleumdungen
zu reinigen [...]. Ferner 

\brief{Verfolger}{
Ein Brief über das wahre Fasten,
das wahre Beten, und die wahre Ehre, gegen die Verfolgungen
und für die wahre Freiheit in Christus Jesus, damit ihr in eurem
Halten von Tagen, Monaten, Zeiten und Festen Sorge tragen
möget, das der Apostel nicht umsonst an euch gearbeitet habe,
und ihr nicht von neuem \zitat{den dürftigen Satzungen dienet}
(Gal. 4,9\bibel{Gal. 04:09@Gal. 4:9}) und sie anderen auferlegt.
Wo haben je Christus oder seine Apostel den Gläubigen oder
den Christen befohlen, Feste oder Tage zu halten? Zeiget uns,
wo in den Schriften des Neuen Testaments, in den vier 
Evangelien, in den Briefen oder in der Offenbarung geschrieben steht,
das Christus oder die Apostel je befahlen, die Zeit, die man
Christfest(Weihnachten)\index{Weihnachten} nennt, zu feiern, oder den Tag von der 
Geburt Christi, oder die Zeit, die man Ostern\index{Ostern} 
nennt, oder den weisen Sonntag,
oder Petrus, Paulus, Lukas oder Markus, oder irgend eines
andern Heiligen Tag? [...]
}
% \picinclude{./260-269/p_s263.jpg} 

Es war des Apostels Arbeit, sie aus den Banden dieser
Satzungen zu befreien. Und als sie sich dann dem Halten der
Tage wieder zuwandten, fürchtete er, er habe umsonst an ihnen
gearbeitet; und er ermahnte sie, Gal. 5, 1\bibel{Gal. 05:01@Gal. 5:1}: 
\zitat{ So bestehet nun
in der Freiheit, damit euch Christus befreit hat, und fallet
nicht wieder in das knechtische Joch der Sünde.} Hiermit sagt
er, das sie einst in dem knechtischen Joch gefangen waren. Aber
ach, wie sehr die sogenannten Christen seit den Tagen der Apostel
wiederum in dieses Joch gekommen sind, indem sie wieder Fasten
und Tage hielten, das sieht man an ihrem Tun. Ja, zwingen
nicht sogar sowohl die Papisten\index{Papisten} wie die 
Protestanten\index{Protestanten} die Leute,
Tage, Monden und Jahreszeiten zu halten? [...]. Es war und
ist Christus, der die Menschen von diesen dürftigen Satzungen frei
macht, darum sollen die Erlösten fest stehen in der Freiheit, womit
Christus sie befreit hat [...]. Die so in diesen Satzungen stehen, und
andere dazu zwingen wollen, sind abgewichen von der Erkenntnis
Gottes und stehen nicht fest in der Freiheit\index{Freiheit}, mit der Christus befreiet.

Was das Beten anbelangt, so sehen wir nirgends, das Christus
oder die Apostel je jemanden zwangen\index{Zwang im Glauben}, mit ihnen zu beten oder
zu fasten. Sondern Christus zeigte, wie man beten solle und sich
von den Heuchlern unterscheiden. Seine Worte sind: \zitat{Wenn du
betest, so sollst du nicht sein wie die Heuchler, die da gerne stehen
in den Schulen und an den Ecken auf den Gassen, auf das sie 
von den Leuten gesehen werden [...]. Wenn du betest, so gehe
in dein Kämmerlein und schließe die Türe zu und bete zu deinem
Vater im Verborgenen, und dein Vater, der ins Verborgene sieht,
wird dirs vergelten öffentlich} 
(Matth. 6, 5\bibel{Matth. 06:05@Matth. 6:5}) .[...]. Und wir
tun nun, wie es die Apostel und Heiligen getan. Wir beten im
Verborgenen und öffentlich, je nachdem der Geist es uns eingibt, 
welcher unsrer Schwachheit hilft, wie er den Aposteln und
allen wahren Christen half; so beten wir für uns und für alle
Menschen, hoch und niedrig. [...] über das Fasten sagt
Christus: \zitat{Wenn ihr fastet, so sollt ihr nicht sauer sehen wie die
Heuchler, sie verstellen ihr Angesicht, auf das sie vor den Leuten
scheinen mit ihrem Fasten. Wenn du fastest, so salbe dein Haupt
und wasche dein Angesicht, und dein Vater, der ins Verborgene
sieht, wird dirs vergelten öffentlich.} (Matth. 6\bibel{Matth. 06@Matth. 6}). 
In Jesaia 58\bibel{Jesaia 58@Jesaia 58} heißt es: 

  \grosszitat{
  Rufe laut 
  und schone nicht, erhebe deine Stimme wie
  eine Posaune und verkündige meinem Volk ihr Übertreten und
  % \picinclude{./260-269/p_s264.jpg}
  dem Hause Jakobs ihre Sünde; sie suchen mich täglich und wollen
  meine Wege wissen, als ein Volk, das Gerechtigkeit schon getan,
  und das Recht ihres Gottes nicht verlassen hätte; sie fordern
  mich zum Recht und wollen mit ihrem Gott rechten. Warum
  fasten wir und du siehest es nicht an? warum tun wir unserm
  Leib wehe und du willst es nicht wissen? [...] Siehe, ihr fastet,
  das ihr hadert und zanket und schlaget mit der Faust ungöttlich.
  Fastet nicht also, wie ihr jetzt tut, das ein Geschrei von euch
  in der Höhe gehört wird. Sollte das ein Fasten sein, das ich
  erwählen soll, [...] wollt ihr das ein Fasten nennen und einen
  Tag dem Herrn angenehm? Das ist ein Fasten, das ich erwähle:
  \zitat{Las los, welche du mit Unrecht gebunden hast; las ledig, welche
  du beschwerest; gib frei, welchs du drängest; reis weg allerlei
  Last.} [...] 

  Das Fasten also, das der Herr verlangt, ist nicht,
  das man Lasten auferlege und die Banden der Sünde noch 
  befestige, sondern solche Bande zu lösen und zu sprengen.

  Und nun darüber, das wir den Hut nicht abnehmen\index{Hutabnehmen} vor den
  Leuten. Viele, die sich Christen nennen, haben Anstoß an uns 
  genommen, weil wir den Hut nicht abnahmen und uns nicht vor
  ihnen Verneigten. Wir finden nirgends, das Christus das geboten
  hat, sondern eher das Gegenteil. Christus sagt: \zitat{nehmet nicht
  Ehre von den Menschen;} und ferner sagt Christus: \zitat{wie könnet
  ihr glauben, die ihr Ehre von einander nehmet, und die Ehre
  die von Gott kommt, sucht ihr nicht} (Joh. 5)\bibel{Joh. 5}. Christus nennt
  es ein Kennzeichen der Ungläubigen, Ehre voneinander zu nehmen
  und die Ehre, die von Gott kommt, nicht zu suchen, und ist denn
  nicht das Abnehmen des Hutes und das Verneigen eine Ehre, die
  sich die Menschen untereinander erzeigen, nach welcher sie trachten
  und beleidigt sind, wenn sie ihnen nicht erzeigt wird? Haben sie
  nicht sogar etliche gebüßt, verfolgt und gefangen genommen, weil
  sie den Hut nicht abnahmen? Ja, verhöhnen nicht die Türken
  die Christen in ihrem Sprichwort, welches sagt, die Christen
  bringen einen großen Teil ihrer Zeit damit zu, ihre Hüte 
  abzunehmen und einander ihre kahlen Köpfe zu zeigen? Sollten nun
  die, welche den edlen Namen Christen tragen dürfen, nicht über
  den Türken stehen und über dem Trachten nach Menschenehre
  und dem Verfolgen solcher, die ihnen diese Ehre nicht erweisen
  wollen, wie überhaupt alle wahren, gläubigen Christen allein die
  Ehre suchen sollten, die von Gott kommt? Es heißt: \zitat{Wer an
  % \picinclude{./260-269/p_s265.jpg} 
  den Sohn Gottes glaubt, der hat das ewige Leben; wer aber
  nicht glaubt, der wird verdammt werden} (Joh. 3, 36\bibel{Joh. 3:36@Joh. 03:36}). 
  Ist nicht die Redensart der Türken\index{Türken}, das die Christen so viel Zeit
  darauf verwenden, ihre Hüte abzuziehen und einander ihre kahlen
  Köpfe zu zeigen, ein Vorwurf für die Christen? Habt ihr nicht
  viele gefangen genommen und bestraft, weil sie den Hut nicht vor
  euch abnehmen wollten und euch ihre kahlen Köpfe zeigen? Ja,
  in vielen eurer Städte und Staaten haben solche, die ihre Hüte
  nicht abnehmen und ihre kahlen Köpfe nicht zeigen, weder Freiheit
  noch Recht, obgleich sie treue Untertanen sind. Habt ihr nicht
  ein Gesetz gegen sie erlassen, das sie zwei Gulden bezahlen müssen,
  wenn sie es nicht tun? Und trachtet ihr nicht, sie dazu zu zwingen
  und bestraft sie, wenn sie es nicht tun, wie in Landsmeer in
  Vaterland? Ist denn das nicht trachten nach Menschenlehre?
  Taten nicht die Pharisäer und Juden also? [...]

  Ihr habet keinerlei Befehl von Christus oder einem seiner
  Apostel, irgend jemanden zu verfolgen\index{Verfolgung}, 
  zu bestrafen oder gefangen
  zu nehmen um seiner Religion willen.

  \begin{flushright} Harlingen\ort{Harlingen} in 
  Friesland\ort{Friesland}, 11.des 6.Monats 
  1677\index{Jahr!1677}. G. F.\end{flushright}
}

Bald darauf kamen William Penn\person{Penn, William} 
und George Keith\person{Keith, George} von
Deutschland nach Amsterdam\ort{Amsterdam} zurück 
und hatten einen Disput\index{Disput} mit
Galenus Abrahams\person{Abrahams, Galenus}, einem 
der bekanntesten Baptisten\index{Baptisten} in Holland.
Viele \textit{Fromme} waren zugegen; da sie nicht Zeit hatten, den
Disput zu beendigen, kamen sie am folgenden Tage noch einmal
zusammen, und da wurde der Baptist gänzlich geschlagen, und die
Wahrheit gewann Boden [...]. Als wir nun unsern Dienst in
Amsterdam getan, gingen wir, zu Wagen, nach Leyden\ort{Leyden}. Wir
kamen dort mit einem Deutschen zusammen, der teilweise bekehrt
wurde. Er sagte uns von einem hervorragenden Mann, der die
Wahrheit suche. Etliche fanden ihn auf und besuchten ihn und
sahen einen ernst gesinnten Mann in ihm. Ich redete auch mit
ihm, und er bekannte sich zur Wahrheit. William Penn und Benjamin
Furly\person{Furly, Benjamin} besuchten noch einen andern 
angesehenen Mann, der ein wenig
auserhalb von Leyden wohnte, von dem es hieß, er sei General
beim König von Dänemark gewesen. Er und seine Frau waren
sehr liebevoll mit ihnen und nahmen die Wahrheit mit Freuden auf.

Von Leyden gingen wir nach dem Haag\ort{Haag}, wo der Prinz von
Oranien\person{Prinz von Oranien} seinen Hof hielt, und 
wir besuchten einen von der Regierung
von Holland, mit dem wir eine ziemlich lange Unterredung hatten.
% \picinclude{./260-269/p_s266.jpg} 

[...] Von da gingen wir über Delft nach Rotterdam\ort{Rotterdam}, wo wir
einige Tage blieben und mehrere Versammlungen hatten. Hier
verfasste ich auch ein Buch an die Juden\index{Juden}, mit denen ich gerne,
als ich in Amsterdam war, mich unterredet hätte, aber sie wollten
nicht. Ich erhielt hier auch einige andere Bücher und Schriften,
die ich früher herausgegeben und die nun übersetzt waren.
