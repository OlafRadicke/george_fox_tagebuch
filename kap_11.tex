%%%%%%%%%%%%%%%%%%% Kapitel 11. %%%%%%%%%%%%%%%%%%%%%%%%%%%%%%

\chapter[Kampf gegen die Prädestinationslehre]{Kampf gegen die Prädestinationslehre}

\begin{center}
\textbf{Reise nach Schottland. Kampf gegen die 
Prädestinationslehre und Widerstand der schottischen 
Geistlichkeit.}
\end{center}

Ich hatte schon längere Zeit in meinem Jnnern einen Zug
verspürt, nach Schottland zu gehen, und hatte Oberst William
Oöburn in Schottland bitten lassen, mir entgegen zu kommen; und
so kamen er und einige andere von Schottland her zur Versamm-
lung nach Pardsey C-rag. A15 dieselbe zu Ende war, die, wie
er sagte, die allerherrlichste gewesen sei, die er je erlebt habe,
ging ich mit ihm und seinen Begleitern nach Schottland ....,
An einem ersten Tage hatten wir in Heads einesgroße Ver-
sammlung, der viele ,,Fromme« beiwohnten. Die Priester hatten
den Leuten Angst gemacht gehabt mit der Lehre von der Erwählung
und der Verwerfung; sie hatten ihnen gesagt, Gott habe die Meisten
für die Hölle bestimmt; sie könnten nun beten, predigen und fingen,
so viel sie wollten, es sei alles umsonst, wenn sie für die Hölle
bestimmt seien; Gott habe eine gewisse Anzahl für den Himmel
auöerlesen; die könnten tun, maß sie wollten, wie David der Ghe-
brecher und Pauluö der Verfolger, sie seien dennoch für den
Himmel bestimmte Gefäße. Es hänge also nicht vom Tun der
Menschen, sondern von der Bestimmung Gottets ab. Es trieb
mich, diesen Leuten die Verkehrtheit in der Lehre ihrer Priester
aufzudecken, und ich zeigte ihnen, wie sie die Schriftstellen aus
% \picinclude{./120-129/p_s129.jpg} 
dem Judasbrief und andere, auf die sie sich beriefen, verdreht
hatten. Ihre Behauptung, das gar nichts von dem Tun des
Menschen abhänge, widerlegte ich ihnen aus dem Judasbrief, wo
die Schuld deutlich zu sehen ist bei Kain, Korah und Bileam, von
denen es heißt, sie seien von Anbeginn zur Verdammung bestimmt
gewesen. Denn hatte nicht Gott Kain und Bileam gewarnt und
an Kain die Frage gerichtet: "`Wenn du recht tust, bist du dann
nicht angenehm vor Gott?"' Und hat der Herr Korah und seine
Rotte nicht aus Ägypten geführt und sie haben sich ihm und
seinem Gesetze und Moses trotzdem widersetzt! "`Man sieht deutlich,"'
sagte ich, "`daß Kain und Korah und Bileam schuldig waren, so
wie es alle sind, die ihre Wege gehen. Oder haben die etwa
keine Schuld, die sich Christen nennen und doch dem Evangelium
zuwider handeln, wie Korah wider das Gesetz, und die vom
Geist abirren, wie Bileam und Übles tun wie Kain? An ihnen
liegt die Schuld, und nicht an der Verwerfung und nicht an Gott.
Sagt nicht Christus: "'Gehet hin und predigt das Evangelium
aller Welt."` Er würde sie nicht in alle Welt geschickt haben, um
die Lehre vom Heil zu predigen, wenn die meisten Menschen für
die Hölle bestimmt wären. Und war nicht Christus ein Sühnopfer 
für der ganzen Welt Sünden? für die Verworfenen wie für
die Auserwählten? Er starb für alle Menschen, für die Guten
wie für die Bösen, wie der Apostel bezeugt, 2. Cor. 5, 15 und
Römer 5, 6.  Christus gebietet, das alle an das Licht glauben; 1
die, welche aber das Licht hassen, an das Christus zu glauben
befiehlt, die sind Verworfene. Und wieder heißt es: ,,In einem
Jeglichen zeigen sich die Gaben des Geistes zu gemeinem Nutzen"'!
Die aber den Geist beleidigen, unterdrücken und betrüben, die
sind Verworfene und Schuldige, sowie auch die, welche das Licht
hassen. Der Apostel sagt, Tit. 2, 11. 12: ,,Die heilsame Gnade
Gottes ist allen Menschen erschienen und lehret uns abzusagen
aller Fleischeslust und hoffartigem Wesen und züchtig, gerecht und
gottselig zu wandeln auf dieser Erde.« Und darum sind alle,
Männer wie Frauen, Schuldige, wenn sie gottlos leben und das,
was sie selig machen würde, verachten. Aber scheints sehen die
Priester die Schuld nicht bei denen, die nicht an Gott und nicht
an Christus, der sie erkauft hat, glauben und an sein Licht und
seine Gnade, die sie selig mach en könnte. Alle aber, die Christi Befehl
gehorchen und an sein Licht glauben, sind Auöerwählte und stehen
George Fox. 9

% \picinclude{./130-139/p_s130.jpg} 
in der Zucht der göttlichen Gnade, welche selig macht. Die aber,
die Gottes Gnade in Mutwillen kehren und sein Licht hassen
(Jud.), sind verworfen; darum ermahne ich alle, an das Licht zu
glauben wie Christus gebietet, und die Gnade, die sie umsonst
lehrt, anzunehmen; dann werden sie gewißlich selig, denn sie
genüget. Viele andere Schriftstellen über die Verwersung wurden
auch noch ausgelegt, und die Augen der Leute wurden geöffnet,
so daß eine Quelle des Lebens unter ihnen hervorsprudelte.
Solches kam den Ptiestetct bald zu Ohren; denn den Leuten,
welche durch ihre schrecklichen Lehren irre geführt worden waren,
gingen allmählich die Augen auf, und sie kamen in den- Bund
des Lichts. Die Kunde, daß ich nach Schottland gekommen sei,
verbreitete sich unter den Priestern. Und sie erhoben ein großes
Geschrei, daß jetzt alles aus sei; denn ich hätte schon in England
alle rechten Männer und Frauen abspenstig gemacht, und ihnen
bleibe dann, wie sie selber zugaben, der schlechtere Teil. Sie ver-
anstalteten darum große Zusammenkünfte von Priestern und
stellten eine ganze Reihe von Verdammungen zusammen, welche
in den Turmhäusern verlesen werden sollten, und die Leute sollten
,,Amen« dazu sagen. Ginige davon will ich hier mitteilen. Zuerst
hieß es: ,,Verflucht ist, wer sagt, ein jeder habe ein Licht in sich,
welches genüge, um ihn selig zu machen. Dazu sage ein jeder:
Amen.« .... Nun sagt aber Christus: ,,Glaubet an das Licht,
damit ihr Kinder des Lichtes werdet« (Joh. 12,36) und weiter:
,,wer da glaubt, der soll selig werden« (Mark. 16); und ,,wer
da glaubt kommt vom Tode ins Leben« .... Und der Apostel
sagt: ,,Jhr tut wohl, aus das Licht zu achten, das da scheinet
an einem dunklen Ort, bis der Tag anbreche und der Morgen-
stern ausgehe in euren Herzen« (2. Petr. 1, 19) ..... Was
den 2. Punkt anbelangt, wo es heißt: ,,Verflucht, wer sagt, der
Glaube sei ohne Sünde,« .... so ist er ja eine Gabe Gottes
und jede Gabe Gottes ist rein .... . Der Glaube, dessen
Ursprung Christus ist, ist köstlich, göttlich und ohne Sünde. Dies
ist der Glaube, der die Herrschaft über die Sünde gibt und den
Zugang zu Gott ..... Aber sie sind alle von diesem Glauben
abgefallen .....
Gs waren in Schottland zwei Kirchen der Jndependenten;
in der einen fanden viele Bekehrungen statt; aber der Pre-
diger der andern war sehr erbost über die Wahrheit und die


% \picinclude{./130-139/p_s131.jpg} 
Reise in Schottland. Kampf gegen die Prädestinationslehre usw. 131
Freunde. Sie hatten Alteste die sich oft bestrebten, ihre Gaben
an ihren Gemeindegliedern zu brauchen und sich ost recht empfänglich
zeigten; aber da ihr Prediger so viel gegen uns und gegen das
Licht redete, verdunkelte sich ihr Blick, das; sie ganz blind wurden
und ganz dürr und ihre Empsänglichkeit verloren. Er fuhr sort
gegen die Freunde und gegen das Licht, aus Christus zu
predigen und nannte dasselbe ein natürliches Licht. Eines Tages
beschimpfte er in seiner Predigt das Licht und da fiel er hin wie
tot in seinem Pult. Man trug ihn hinaus und legte ihn auf
einen Grabstein und flößte ihm ein starkes Getränk ein, das ihn u
wieder zum Leben brachte; und sie trugen ihn heim, aber er war
schwachsinnig geworden. Er riß sich die Kleider vom Leib, hüllte
sich in einen schottischen Plaid und ging aufs Land zu den
Milchmädchen. Nachdem er etwa zwei Wochen dort gewesen,
kehrte er zurück und stieg wieder auf die Kanzel. Nun erwarteten
die Leute große Grössnungen von ihm; statt dessen erzählte er,
wie ihm eines der Mädchen abgerahmte Milch, ein anderes
Buttermilch und wieder ein anderes gewöhnliche Milch gegeben
habe; man mußte ihn wie-der von der Kanzel herunter holen und
heim führen. Der, welcher mir dies alles berichtete, ist Andrew
Robinson, einer seiner eisrigsten Zuhörer, der aber später sich
bekehrte und die Wahrheit annahm. Gr sagte mir, daß er nie
etwas davon gehört habe, daß jener Prediger seinen Verstand
wieder bekommen habe. Daran möge ein jeder sehen, wie es
dem geht, der das Licht beschimpft,. das Licht, welches das Leben s
in Christus, dem Wort, ist; und es möge allen zur Warnung s
dienen, welche Übles reden gegen das Licht Christi ..... Z
Viele der schottischen Priester waren sehr in Aufregung über
die Verbreitung der Wahrheit, weil sie dadurch ihre Zuhörer
verloren; und viele von ihnen gingen darum nach Edinburg,
um beim Rate Oliver Cromwells eine Klage gegen mich vorzu-
bringen. Jnsolge dieser eingereichten Klage kam, als ich einmal
aus einer Versammlung zurückkam, ein Beamter und brachte mir
folgenden Befehl:
,,Donnerstag, 8. Oktober 1657, der Rat Seiner Hoheit in
Schottland.
K Es wird befohlen, daß George Fox nächsten Dienstag,
13. Oktober, vormittags, vor dem Rat erscheint-»
G. Downing, Ratsbeamter.«
gt


% \picinclude{./130-139/p_s132.jpg} 
Als er mir den Befehl übergab, fragte er mich, ob ich kommen
wolle oder nicht. Jch antwortete ihm nicht darauf, sondern
fragte, ob der Befehl auch nicht gefälscht sei? er erwiderte nein,
es sei ein richtiger Befehl vom Rat, und er sei als Bote damit
gesandt. Jch erschien also zur vorgeschriebenen Zeit und wurde in
einen großen Saal geführt, wo viele angesehene Leute versammelt
waren, die mich alle aufmerksam betrachteten; schließlich wurde
ich ins Ratszimmer geführt und unter der Türe nahm man mir
den Hut ab; ich fragte, warum das geschehe? wer denn drinnen
sei, daß ich den Hut abnehmen müsse? ich habe ihn ja sogar vor
dem Protektor nicht abgenommen. Aber der Hut wurde aufge-
hängt und ich wurde hineingeführt. Als ich schon eine ganze
Weile drinnen war, ohne daß jemand etwas zu mir sagte, trieb
mich der Herr zu sagen; ,,Friede sei mit euch! wartet in der
Furcht Gottes auf den Empfang seiner Weisheit von oben, durch
die alle Dinge geschaffen sind, daß sie euch in allem, was euch
fz zu tun übergeben ist, leite, damit ihr es tut zur Ehre Gottes-«.
t Sie fragten mich, weshalb ich nach Schottland gekommen sei? ich
sagte: um den Samen Gottes aufzusuchen, der solange in den
Banden des Bösen gelegen habe, damit alle, welche sich in diesem
Lande zur Schrift, den Worten Christi, der Apostel und der Pro-
pheten bekennen, zum Licht und Geist und zur Kraft kommen, in
denen jene, die solche Worte geäußert, gewesen sind; und daß sie
in diesem Geist die Schrift verstehen und Christus und Gott er-
kennen und mit ihm und unter einander in der rechten Gemein-
—s— schaft stehen möchten. s Sie fragten mich, ob ich irgend etwas Ge-
schäftliches hier zu besorgen habe? Jch verneinte; darauf fragten
sie weiter: wie lange ich im Lande bleiben wolle? Jch antwortete,
dies könne ich nicht sagen, wahrscheinlich nicht sehr lange; doch
da meine Freiheit dem Herrn gehöre, so müsse ich den Willen
dessen, der mich gesandt habe, tun. Darauf hieß man mich hin-
ausgehen. Bald daraus ließ man mich wieder herein kommen
und erklärte mir, ich müsse Schottland verlassen, von jetzt an in
7 Tagen. Jch fragte: warum? wa-3 ich getan habe? Sie sagten,
sie wollen nicht mit mir verhandeln. Darauf bat ich sie, zu hören,
was ich ihnen zu sagen habe; aber sie wollten nicht. Jch er-
innerte sie daran, daß Pharao, der doch ein Heide gewesen sei,
Moses und Aaron angehört habe, und Herodes hörte Johannes
den Täufer; sie sollten doch nicht schlechter sein als jene! Aber


% \picinclude{./130-139/p_s133.jpg} 
Reise in Schottland. Kampf gegen die Prädeftinationzlehre usw. 133
sie schrien: »hinautz! hinautz!«, woraus ich wieder hinautzgesührt
wurde. Jch kehrte in meine Wohnung zurück und fuhr fort, in
Gdinburg die Freunde zu besuchen und auszurichten im Herrn.
Ich schrieb darauf an den Rat, um ihm sein unchristlichetz Be-
nehmen gegen mich oorzuhalten .....
Nach einiger Zeit ging ich wieder nach Headtz, wo die
Freunde in großer Not gewesen waren; denn die Pretzbyteri-
aner-Priester hatten sie in den Bann getan und befohlen,
etz solle niemand von ihnen kaufen oder ihnen etwatz
verkaufen oder mit ihnen essen und trinken. So konnten sie weder
ihre Ware verkaufen, noch sich datz ihnen Nötige anschaffen, watz
viele in große Bedrängnitz brachte. Denn wenn einer von ihren
Nachbarn ihnen Brot oder andere Lebentzmittel verkauft hätte, so
hätte ihn der Priester derart bedroht, daß er schleunigst gekommen
wäre, die Sachen wieder zu holen. Aber Oberst Ashsield, welcher
der Friedentzrichter jener Gegend war, machte diesem Vorgehen
der Priester ein Ende. Später wurde er selber gewonnen und hielt
Versammlungen in seinem Hause, verkündete selber die Wahrheit
und lebte und starb in derselben .....
Die Wahrheit und die Kraft detz Herm breitete sich autz in
Schottland, und durch die Kraft und den Geist Gottetz wurden
viele zum Herrn Jesutz Ehristutz bekehrt, ihrem Heiland und Lehrer,
der sein Blut für sie vergossen hat; und etz ist seither ein grorßetz
Wachtztum und wird etz immer mehr sein in Schottland. Denn
altz zuerst die Hufe meinetz Pferdetz schottischen Boden berührten,
da fühlte ich, wie überall Funken detz Samentz von Gott um mich
herum aufsprühten, wie unzählige Feuerfunken. Nicht altz ob
nicht noch viel hartetz, schlechtetz Erdreich von Falschheit und
Heuchelei dort gewesen wäre und ein knorriger Boden, der zu-
erst noch durch Gottetz Wort fruchtbar gemacht werden muß
und gepfliigt mit dem Pflug detz Geistetz, ehe der Same Gottetz
geistliche, himmlische Früchte hervorbringen kann zu Gottetz Ehre.
Aber der Landmann muß in Geduld warten (Jar. 5, 7).

