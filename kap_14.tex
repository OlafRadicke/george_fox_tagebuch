
%%%%%%%%%%%%%%%%%%% Kapitel 14. %%%%%%%%%%%%%%%%%%%%%%%%%%%%%%
\chapter[Beginn neuer Quälerverfolgungen]{Beginn neuer Quälerverfolgungen}

\begin{center}
\textbf{Beginn neuer Quälerverfolgungen bei Anlass der Verschwörungen
der Fifthmonarchy-Leute. Des Quäkers John Perots Verirrungen.
Quäker misshandelt in Neu-England und Malta.}
\end{center}

\section{Erfolglose politische Lobby-Arbeit}

Ich sah nun, warum ich durch so schwere Prüfungen hatte
gehen müssen in Reading, denn die ewige Kraft des Herrn war
über alle gekommen, und sein gesegnetes Leben und Licht und
seine Wahrheit war dem Lande ausgegangen; wir hatten herrliche 
Versammlungen, und viele scharten sich um die Wahrheit.
Richard Hubberthorn\person{Richard Hubberthorn} war beim König 
gewesen, und dieser hatte
gesagt, es dürfe uns niemand behelligen, so lange wir friedlich
leben; er versprach uns dies bei seinem königlichen Wort mit der
Ermahnung, sein Versprechen nicht zu missbrauchen. 


Einige Freunde erhielten auch Zutritt im 
\textit{House of Lords}\index{House of Lords}, und man
gestattete ihnen, ihre Gründe gegen das 
Zehntenwesen\index{Kirchensteuer}\footnote{Kirchensteuer}, 
das Schwören, die Turmhäuser, die Gottesdienste und anderes 
auseinander zusetzen, und man hörte ihnen ziemlich lange zu. 
Etwa 700 Freunde,
die unter Richards und Olivers Regierung in die verschiedenen
% \picinclude{./150-159/p_s155.jpg} 
Gefängnisse des Landes gebracht worden waren, wegen Verstößen,
wie sie es nannten, setzte der König, als er kam, in Freiheit.
Man spürte, das die Regierung geneigt war, den Freunden ihre
Freiheit zu sichern, weil sie sah, das wir unter der vorherigen
Herrschaft so gut wie sie gelitten hatten. Sobald aber wirklich
etwas für uns geschehen sollte, so wurde es wieder vereitelt durch
irgend einen Schuft, der dergleichen tat, als ob er uns wohl
wolle. Es hieß, es sei schon ein Befehl ausgesetzt, der unsere
Freiheit bestätige, er müsse nur noch unterzeichnet werden; da
brach gerade jenes hässliche Attentat\index{Attentat} der 
Fisthmonarchy-Leute\index{Fisthmonarchy}
aus und brachte die Hauptstadt\ort{Hauptstadt} 
und das ganze Land in Aufruhr.


Es geschah am Abend eines Ersten Tages, und wir hatten eine
besonders herrliche Versammlung gehabt, in der die Wahrheit
des Herrn allen erschienen war. Da, bald nach Mitternacht wurde
die Trommel geschlagen und erklang der Ruf: \zitat{zu den Waffen,
zu den Waffen!} Ich stand auf und nahm am folgenden Morgen
ein Boot und fuhr nach Whitehall\ort{Whitehall}, und stieg 
dort aus und schritt
durchs Schloss. Sie betrachteten mich erstaunt, aber ich schritt
durch sie hindurch bis nach Pall-Mall\footnote{eine Straße in der 
City of Westminster in London.}\ort{Pall-Mall}, wo sich etliche Freunde
zu mir gesellten, obgleich es jetzt gefährlich war, über die Straßen
zu gehen; denn schon waren die Stadt und die Vorstädte unter
Waffen, und das Volk und die Soldaten waren sehr roh; sie 
misshandelten Henry Fell,\person{Fell, Henry} der zu einem 
Freunde gehen wollte, und
hätten ihn getötet, wenn nicht der Herzog von 
York\person{Herzog von York} dazu gekommen
wäre. Es geschah viel Unheil während dieser Woche, und am
nächsten Ersten Tage wurden viele Freunde auf dem Weg in die
Versammlung gefangen genommen.

\section{Verschleppung von Fox}

Ich blieb in Pall-Mall, weil ich dort der Versammlung beiwohnen 
wollte; doch in der Nacht des Siebenten Tages kamen
Soldaten und klopften an die Tür. Da die Mägde sie einließen,
so stürzten sie herauf und ergriffen mich. Und einer von ihnen,
der beim Parlament gedient, fühlte mir in die Tasche und fragte,
ob ich keine Pistolen bei mir habe. Ich fragte ihn, warum er
auch solche Frage an mich stelle; er wisse ja, das sich ein 
friedlicher Mann sei [...] Diese Soldaten nahmen mich mit und
brachten mich nach Whitehall [...] Dort waren die Soldaten
und das Volk sehr wild; aber ich predigte doch die Wahrheit
unter ihnen. Einige Große aber, als sie das hörten, sagten:
\zitat{Was, ihr lasst ihn noch predigen? Bringt ihn doch an einen
% \picinclude{./150-159/p_s156.jpg} 
Ort, wo er nicht mehr hetzen kann.} Das taten sie denn auch
und bewachten mich. Ich sagte ihnen, wenn sie schon meinen
Leib binden und einsperren können, so können sie doch das Wort
des Lebens nicht aufhalten. Einige kamen und fragten mich,
was ich sei? Ich erwiderte ihnen: \zitat{ein Prediger der 
Gerechtigkeit} (2. Petr. 2, 5\bibel{Petr. 2. 02:05@2. Petr. 2:5}). 
Nachdem ich etwa drei 
Stunden eingesperrt gewesen war, ging Esquire Marsch zum Lord 
Gerrard\person{Lord Gerrard}, und darauf wurde ich frei gelassen [...]


Während dieses Aufstandes\index{Aufstand} der Fifthmonarchy-Leute 
1660\index{Jahr!1660},
fanden arge Metzeleien statt, sowohl auf dem Lande als in der
Stadt, so das es für anständige Leute noch lange gefährlich war,
auszugehen. Man konnte kaum ohne Gefahr Einkäufe machen.
Auf dem Lande schleppten sie die Leute, Männer und Frauen,
aus den Häusern und Kranke rissen sie aus ihren Betten. Ja
einen Fieberkranken rissen sie aus dem Bett und schleppten ihn
ins Gefängnis; als er dort ankam, starb er, er hieß Thomas
Pachyn.\person{Pachyn, Thomas}

Margaret Fell\person{Fell, Margaret} ging zum König 
und berichtete ihm, wie es
zugehe zu Stadt und Land. Sie setzte ihm auseinander, das wir
harmlose, friedliche Leute seien; das wir aber unsere Versammlungen 
auch fernerhin halten würden, was immer wir auch zu
dulden haben würden; aber es sei seine Pflicht für Frieden zu
sorgen, damit nicht noch mehr unschuldig Blut vergossen werde.
Die Gefängnisse waren nachgerade überall angefüllt mit
Freunden und andern aus der Stadt und vom Lande; überall
waren Wachen aufgestellt zur Durchsuchung der Briefe, so das
niemand passieren konnte, ohne untersucht zu werden. Wir hörten
von vielen Tausenden von Freunden, die im Lande herum in den
Gefängnissen waren, und Margaret Fell überbrachte dem König
und dem Rat einen Bericht darüber. Als wir in der darauf 
folgenden Woche von einigen weiteren Tausenden hörten, die 
gefangen genommen worden, ging sie abermals hin, um es dem König
und dem Rat mitzuteilen. Man verwunderte sich, woher wir
diese Nachrichten hätten, da ein strenger Befehl ergangen war,
alle Briefe aufzufangen. Aber der Herr fügte es, das wir Kunde
erhielten trotz allen ihren Hindernissen.

\section{Das Friedenszeugnis wird erstellt}

Wir ließen eine Erklärung gegen das Bekriegen und das
Verschwören drucken und schickten einige Abzüge an den König
und den Rat; andere wurden in den Straßen verkauft [...]
% \picinclude{./150-159/p_s157.jpg} 

Diese Erklärung klärte etwas die Luft, die auf Stadt und
Land lastete; und der König erließ bald darauf einen Befehl,
das die Soldaten keine Haussuchungen ohne einen Konstable
vornehmen sollten; aber noch immer waren die Gefängnisse gefüllt
und viele Freunde gefangen, woran namentlich der Aufstand der
Fifthmonarchy-Leute\index{Fifthmonarchy} Schuld war. 
Als aber die Gefangenen
sollten hingerichtet werden, ließ man ihnen doch Gerechtigkeit 
widerfahren und erklärte uns öffentlich frei von jeglicher Teilnahme an
den Verschwörungen. Und auf wiederholtes Drängen erließ der
König den Befehl, die Freunde frei zu lassen ohne Loskaufung.
Aber es hatte viel Mühe und Arbeit gebraucht, um das zu
erreichen. Thomas Moor und Margaret Fell waren oft deswegen
beim König gewesen.

Es wurde während dieses Jahres viel Blut vergossen, denn
viele von den Räten des früheren Königs wurden gehenkt, ertränkt
oder gevierteilt. Unter diesen war auch Oberst 
Hacker\person{Oberst Hacker}, der mich
unter Oliver Cromwell\person{Cromwell, Oliver} von 
Leicester\ort{Leicester} nach London\ort{London} als Gefangener
schickte, wie oben berichtet worden. Es war ein trauriger Tag
der Vergeltung\index{Vergeltung} von Blut durch Blut [...]

\section{Spektakuläre Auftritte von Willam Sympson}

Es war eine unsichtbare Hand, die diesen Tag über das
heuchlerische Geschlecht gebracht hatte, das, kaum war es zur
Herrschaft gelangt, so hochmütig und über alle Maßen grausam
geworden war und das Volk Gottes verfolgt hatte.
Mehr als einmal waren diese \textit{Frommen} gewarnt worden
durch Worte, Schrift und Zeichen; aber sie wollten es nicht
glauben, bis es zu spät war. \index{Auftritte!spektakuläre} 
\index{Auftritte!Nackt} \index{Nackte!Auftritte}
Willam Sympson\person{Sympson, Willam} war öfter während
drei Jahren getrieben worden, unbekleidet und barfuß unter sie
zu treten in den Städten, Märkten und Ortschaften, vor Priester
und Große, um ihnen zu sagen: \zitat{so nackt wie er würden auch sie
einhergehen.} Und manchmal trieb es ihn, sich mit einem Sack
anzutun und sein Gesicht zu beschmieren und ihnen zu sagen:
\zitat{also werde der Herr ihnen ihre Frömmigkeit besudeln, wie er
besudelt sei.} Der arme Mensch hatte schwere Leiden erduldet,
sich mit Pferdepeitschen auf seinen bloßen Leib peitschen, sich mit
Steinen bewerfen und einsperren lassen während dieser Jahre, vor
der Rückkehr des Königs; aber sie wollten sich nicht warnen
lassen, sie erwiderten seine Liebe mit Grausamkeit. Nur der
Bürgermeister von Cambridge behandelte ihn großmütig, indem
er seinen Rock um ihn legte und ihn in sein Haus nahm.
% \picinclude{./150-159/p_s158.jpg} 

Ein anderer Freund, Robert Huntingdon\person{Huntingdon, Robert}, 
wurde getrieben ins Turmhaus zu Carlisle zu gehen, unter die 
Haupt-Presbtyterianer\index{Presbtyterianer}
und Independeten\index{Independeten} dort, in einem weißen Hemd als
Zeichen, das das Chorhemd wieder aufkommen werde, und mit
einem Halfter, um zu zeigen, daß auch für sie ein Halfter kommen
werde, was sich auch an einigen von den Verfolgern erfüllte.


Zu einem andern, Richard Sale\person{Sale, Richard}, der 
Konstabler in der Nähe
von Chester war, wurde ein Freund mit einem Pass geschickt;
die schändlichen \textit{Frommen} hatten ihn als Vagabunden 
festgenommen, weil er als Prediger reiste. Dieser Konstabler wurde
durch den ihm zugeschickten Freund bekehrt und gab ihm die
Freiheit; später wurde er selber ins Gefängnis geworfen. Danach
an einem Ersten Tage trieb es Richard 
Sale\person{Sale, Richard} ins Turmhaus zu
gehen und den verfolgungssüchtigen Priestern und ihren Genossen
eine Kerze zu bringen als Anspielung auf ihre Finsternis; aber
sie misshandelten ihn, und, so recht wie verstockte 
\textit{Fromme}, warfen 
sie ihn ins Gefängnis von Little-Gase\ort{Little-Gase}, 
und quälten ihn dort dermaßen,
das er bald darauf starb. 

Die Freunde wurden mehrfach getrieben, 
dieses Geschlecht auf allerlei Art zu mahnen, aber nicht
nur hörte man sie nicht, sondern sie wurden noch misshandelt und
\zitat{verschrobene Quäker} genannt! Aber Gott schickte sein Gericht
\index{Gottesstrafe} \index{Gottesgericht} über die 
verfolgungssüchtigen Priester und Behörden; als der
König zurück kam, wurden den meisten von ihnen ihre Stellen und
Einkünfte entzogen; die Räuber wurden beraubt und es war nun
an uns zu sagen: \zitat{wo sind jetzt die Verschrobenen?} Viele gaben
jetzt zu, das wir wahre Propheten\index{Wahre Propheten} seien, 
und behaupteten, wenn
wir nur gegen einzelne Priester geeifert hätten, so hätten sie sich
sogar über uns gefreut; weil wir aber gegen alle geeifert hatten,
so haben sie sich über uns verärgert. 

\section{Die zwei Seiten der Toleranz}

Aber sie sahen es jetzt ein,
das die Priester, die damals für die besten gehalten wurden, so
schlecht waren wie die übrigen. Ja, viele, die als die aller 
hervorragendsten gegolten hatten, hetzten die Behörden am 
allermeisten zu den Verfolgungen auf. Diese wurden aber, als der
König zurück kam, damit bestraft, das ihnen die Gewissensfreiheit
entzogen wurde, die sie vorher, als sie die Oberhand gehabt hatten,
den andern nicht gegönnt hatten. Einer, namens Hewes, von
Plymouth,\person{Priester Hewes von Plymouth} ein angesehener Priester 
zu Olivers Zeits hatte immer,
wenn irgendwo Gewissensfreiheit zugestanden wurde, gebetet, Gott
wolle es den Behörden ins Herz geben, das sie diese verdammte
% \picinclude{./150-159/p_s159.jpg} 
Toleranz abschaffen. \index{Toleranz} \index{Gewissensfreiheit} 
Und andere beteten gegen \zitat{die nicht zu
duldende Duldsamkeit}. Als nun nach des Königs Rückkehr
diesem Priester Hewes seine großen Einkünfte entzogen wurden,
weil er sich nicht dem Common-Prayer Buch unterwerfen wollte,
fragte ihn ein Freund, als er ihm in Plymouth begegnete: ob er
nun die Duldsamkeit noch verdammungswürdig finde und nicht
vielmehr froh darüber wäre? worauf der Priester nicht antwortete
und nur das Gesicht abwandte. Aber so hartnäckig auch diese
Leute früher gegen Duldsamkeit geeifert hatten, — jetzt kamen
viele von ihnen selber beim König um einen Ort, wo sie ihre
Versammlungen halten könnten, ein, und bezahlten sogar, damit
es ihnen bewilligt werde [...]

\section{Die blutige Verfolgung in Neu-England hat ein Ende}

Wir erhielten die Kunde, das ein Freund, der getrieben worden
war, gegen den Götzendienst der Papisten\index{Papisten} 
zu predigen, in Rom
im Gefängnis gestorben war, und man hatte den Verdacht, das
er heimlich im Gefängnis umgebracht worden war. 
John Perrot \person{Perrot, John}
war auch dort gefangen gewesen und kam nach seiner Freilassung
zu uns zurück; später aber wandte er und andere sich ab von
der Gemeinschaft der Freunde und der Wahrheit [...]

Ungefähr um die gleiche Zeit 1661\index{Jahr!1661} erhielten 
wir die Nachricht 
aus Neu-England\ort{Neu-England}, das die dortige Regierung ein Gesetz
erlassen hatte, das die Quäker aus jenen Kolonien verbannte bei
Todesstrafe im Fall der Rückkehr, und das mehrere Freunde, die
nach ihrer Verbannung zurückgekehrt waren, wirklich gehängt
worden waren \footnote{1661 verfolgten die 
Puritaner\index{Puritaner} und 
Independenten\index{Independenten}, die selber nach Amerika geflohen, um 
Religionsfreiheit zu haben, die Quäler aufs Grausamste.
Ein Edikt von 1658\index{Jahr!1658} bestimmte, das jeder 
Quäker, der zum dritten 
mal in den Kolonien gefunden werde, gehängt werde, und 
dieser Befehl wurde an zwei Quäkern und einer Quäkerin 
ausgeführt.}, und andere vom gleichen Schicksal bedroht im
Gefängnis seien. 

Während jene hingerichtet wurden, war ich im
Gefängnis zu Lancaster gewesen und hatte eine deutliche Wahrnehmung 
ihrer Leiden, als ob sie mich selber betroffen hätten,
und der Strick um meinen eigenen Hals gelegt würde, und wir
hatten doch damals noch nichts davon gehört. Sobald wir nun
davon erfuhren, ging Edward Burrough\person{Burrough, Edward} 
zum König und sagte
ihm, es sei in seinem Reich eine Ader offen, aus der unschuldiges
Blut fließe, das, wenn es nicht gestillt werde, alles überschwemmen
werde. Der König erwiderte: \zitat{Ich werde dieses Blut stillen.}
% \picinclude{./160-169/p_s160.jpg} 
Edward Burrough sagte: \zitat{So tue es eilends, denn wir können nicht
wissen, wie viele in Bälde noch hingerichtet werden.} Der König
sagte: \zitat{So bald ihr wollt,} und befahl einem der Anwesenden:
\zitat{holt den Sekretär, so will ich es sogleich tun.} Als der Sekretär
kam, wurde sofort ein Erlass zugesagt. 

Ein paar Tage darauf
ging Edward Burrough wieder zum König, um ihn zu bitten,
den Erlass abzuschicken; der König antwortete, er habe jetzt keine
Gelegenheit ein Schiff dorthin zu schicken; wenn wir es aber tun
wollten, so stehe uns das frei, so bald wir wollten. Darauf fragte
Edward den König, ob er allenfalls auch einen sogenannter:
\zitat{Quäker} mit seiner Sendung betrauen würde? Der König 
antwortete: \zitat{Ja, es kann gehen, wer will.} 


Hierauf nannte 
Edward ihm Samuel Chattok,\person{Chattok, Samuel} der aus 
Neu-England, seiner Heimat
verbannt worden war und nicht zurückkehren durfte, es sei denn
mit dieser Sendung. Dann ließ er Ralph 
Goldsmith\person{Goldsmith, Ralph} kommen,
den Besitzer eines guten Schiffes, und einigte sich mit ihm auf
300 Pfund, in Waren oder bar, und Abfahrt in zehn Tagen. Er
rüstete sich alsbald, unter Segel zu gehen, und, vom Winde 
begünstigt, kam er nach etwa sechs Wochen, am Morgen eines
Ersten Tages, in Boston in Neu-England\ort{Neu-England} an. 
Es reisten viele mit ihm, aus Alt- und Neu-England, 
Freunde, die der Herr
trieb, mitzugehen und aufzutreten gegen die blutigen Verfolger,
welche alle Übrigen an Grausamkeit übertrafen.


Als die Bewohner von Boston ein Schiff mit englischen
Farben in den Hafen von Boston fahren sahen, kamen sie
gleich aufs Schiff und fragten nach dem Kapitän, und Ralph
Goldsmith sagte ihnen, das er es sei. Sie fragten ihn, ob er
Briefe habe? Er sagte: \zitat{ja}. Sie fragten, ob er sie 
ausliefern wolle? er antwortete: \zitat{nein, heute nicht.} 
Darauf begaben sie sich ans Ufer und berichteten, es sei 
ein ganzes Schiff voll Quäker angekommen, und Samuel Shattock 
sei darunter, der nach den Gesetzen hingerichtet werden müsse, 
wenn er aus der Verbannung zurückkomme! denn sie wussten nichts 
von seiner Sendung. Den ganzen Tag wurden alle streng abgesperrt, 
und keiner von der Schiffsmannschaft durfte landen. 

Am folgenden Morgen begaben sich die Gesandten des Königs, 
Samuel Shattock und der Befehlshaber des Schiffs, Ralph 
Goldsmith ans Ufer, und nachdem sie die Männer, die sie ans 
Land geführt hatten, zurückgeschickt hatten,
gingen sie durch die Stadt zum Haus des Gouverneurs, John
% \picinclude{./160-169/p_s161.jpg} 
Endicott,\person{Endicott, Gouverneur John} 
und klopften. Der Gouverneur schickte jemand heraus,
um sie nach ihrem Begehren zu fragen. Sie ließen ihm sagen,
sie kämen vom König von England und werden ihre Botschaft
niemand übergeben, als dem Gouverneur selbst. Darauf wurden
sie vorgelassen. Der Gouverneur erschien und nachdem er ihre
Botschaft vernommen und ihren Auftrag, nahm er seinen Hut ab
und betrachtete sie. Dann verließ er sie und begab sich zum
Untergouverneur, und nach einer kurzen Unterredung mit diesem
kam er zu den Freunden zurück und sagte ihnen: \zitat{Wir
werden Seiner Majestät Befehl gehorchen.} Hierauf erhielten
die Reisenden die Erlaubnis zu landen, und rasch verbreitete sich
die Kunde von dem Vorgefallenen in der Stadt, und die Freunde
aus der Stadt vereinigten sich mit den Reisenden des Schiffes,
um Gott zu loben und zu danken, das er sie so wunderbar aus
den Zähnen derer, die sie umbringen wollten, befreite. Während
sie beisammen waren, kam ein Freund herein, der von ihrem
blutigen Gesetz zum Tode verurteilt worden war und lange Zeit
in Fesseln gelegen und auf seine Hinrichtung gewartet hatte. Da
wurde die Freude noch größer, und alle erhoben ihre Herzen in
inbrünstigem Loben Gottes, welcher würdig ist zu nehmem Preis,
Ruhm und Ehre; denn er allein kann frei machen und erretten
und helfen allen denen, die ihr Vertrauen auf ihn setzen [...]

\section{Wieder der höfischen Anrede}  

Vorher, als ich noch im Gefängnis zu Lancaster war, war
ein Buch von mir (\buchtitel{The Battledore}) veröffentlicht 
worden, das zeigen sollte, wie in allen Sprachen \zitat{du} 
und \zitat{dich} die eigentliche Anrede an eine einzelne 
Person sei und \zitat{ihr} nur an mehrere.\index{Sprache}\index{Anrede}
Ich hatte es an Beispielen aus der Schrift und aus Lehrbüchern
in etwa dreißig Sprachen nachgewiesen. J. 
Stubbs\person{Stubbs, J.n} und Benjamin 
Furly\person{Furly, Benjamin} hatten sich auf meine 
Veranlassung sehr Mühe gegeben,
das Material zu sammeln, und ich fügte dann noch einiges bei.
Als es fertig war, erhielten der König und die Räte, die Bischöfe
von Canterbury und London und die beiden Universitäten eine
Abschrift und es wurde viel gekauft. Der König sagte, es sei
richtig, das diese Völker so sprechen; und als man den Bischof
von Canterbury fragte, was er davon halte, so wusste er nicht,
was er sagen sollte; denn es wirkte so überzeugend auf die Leute,
das viele daraufhin sich kaum mehr ärgerten, wenn wir \zitat{du}
und \zitat{dir} zu ihnen sagten, während man uns das vorher sehr
übel genommen hatte [...]
% \picinclude{./160-169/p_s162.jpg} 

Da die Priester und Bischöfe gerade eifrig am Werk waren,
ihre Gottesdienste einzurichten und alle zu zwingen, daran teil
zu nehmen, trieb es mich, folgendes zu schreiben, um die Art der
wahren Gottesdienste, die Christus eingesetzt hat und die Gott
annimmt, zu zeigen:

\brief{Verfolger}{
  Der wahre Gottesdienst Christ geschieht im Geist und steht
  allen Menschen offen. Die im Geist und in der Wahrheit anbeten, 
  die sind Gott angenehm (Join). Es gibt dem Volk Odem
  und den Geist denen, die auf der Erde sind 
  (Jes. 42:5\bibel{Jes. 42:05@Jes. 42:5}), und er
  gibt ihnen eine unsterbliche Seele; sie sind sie Tempel, in denen
  er wohnen will (1. Kor. 3:16\bibel{Kor. 1. 03:16@1. Kor. 3:16}). 
  Die, welche äußerlich
  Juden\index{Juden} waren, mussten nach Jerusalem\ort{Jerusalem} 
  gehen, um anzubeten, so
  lange sie dort ihren äußeren Tempel\index{Tempel!äußerer} 
  hatten; [...] nun aber
  sollen alle \zitat{Gott im Geist und in der Wahrheit anbeten}. Dies
  ist ein Gottesdienst der Freiheit, denn \zitat{wo der Geist ist, 
  da ist Freiheit} (2. Kor. 3:17\bibel{Kor. 2. 03:17@2. Kor. 3:17}). 
  Die Früchte des Geistes werden
  offenbar werden; und man soll der Geist keine Schranken setzen,
  sondern in ihm wandeln und lebens, damit man seine Früchte
  hervorbringen kann. [...] Denn ,an ihren Früchten sollt ihr
  sie erkennen (Matth. 7:16\bibel{Matth. 07:16@Matth. 7:16}) [...] 
  \medskip 
  \begin{flushright}G. F. \end{flushright}
}


\section{Missionsbemühungen der Katholiken}

Viele Papisten\index{Papisten} und Jesuiten\index{Jesuiten} 
fingen damals an, den Freunden \index{Ökumene}
zu schmeicheln und zu sagen, so oft sie einen von ihnen sahen,
von allen Sekten haben die Quäkeren meisten Selbstverleugnung,
und es sei schade, das sie nicht in die heilige Mutterkirche 
zurückkehrten. In dieser Weise schwatzten sie den Leuten vor und 
behaupteten, sie würden gern mit den Freunden unterhandeln; aber
die Freunde verabscheuten es, sich mit ihnen einzulassen und
hielten es für gefährlich und sogar anstößig, weil es Jesuiten
waren. 

Als ich aber davon hörte, sagte ich: \zitat{lasst uns mit ihnen
unterhandeln, seien sie, wer sie wollen.} Somit wurde die Zeit
festgesetzt, zu der zwei, die wie Höflinge aussahen, kamen; sie
fragten nach unsern Namen, die wir ihnen nannten; wir aber
fragten nicht nach ihren Namen, denn wir wussten ja, das sie
Papisten waren und wir Quäker. Ich fragte sie das selbe, was
ich schon früher einen Jesuiten gefragt hatte, nämlich, ob die
Römische Kirche nicht abgefallen sei von der Kraft, dem Geist
und den Grundsätzen der apostolischen Zeiten? [...] Als sie
sahen, das wir es genau nahmen, wichen sie aus, indem sie sagten,
es sei eine Anmaßung zu behauptet, irgend jemand habe den
% \picinclude{./160-169/p_s163.jpg} 
Geist und die Kraft, den die Apostel hatten. Aber ich sagte, es
sei eine Anmaßung von ihnen, die Worte Christi, der Apostel und
Propheten zu benützen und die Leute glauben zu machen, sie seien
Nachfolger der Apostel und Propheten, da sie doch zugeben müssen,
sie haben nicht den Geist und die Kraft der Apostel. Ich zeigte
ihnen, wie verschieden ihr Tun und ihre Früchte von denen der
Apostel seien. Darauf erwiderte mir einer von ihnen: \glqq Ihr seid
eine Gesellschaft von Träumern.\grqq{} \glqq Nein,\grqq{} erwiderte ich, 
\glqq sondern ihr seid widerwärtige Träumer, die ihr euch als die 
Nachfolger der Apostel träumt, während ihr doch zugebt, das ihr nicht ihren
Geist und ihre Kraft habt. Und ist es nicht Befleckung des Fleischer?
zu sagen, es sei Anmaßung zu behaupten, man habe den Geist
und die Kraft der Apostel? Und wenn ihr nun zugebt, das ihr
nicht den Geist und die Kraft der Apostel habt,\grqq{} sagte ich, \glqq 
so ist es klar, das ihr von einem anderen Geist und einer anderen
Kraft geleitet werdet als die erste Kirche und die Apostel.\grqq{} Ich
erklärte ihnen, das es ein böser Geist sei, der sie leite und sie zu
dem Beten mit Rosenkränzen und zu Bildern geführt habe und
zum errichten von Klöstern \index{Kloster} und zum Töten um des Glaubens
willen. Ich wies sie darauf hin, wie solches Tun gesetzlich und
nicht nach dem Evangelium der Freiheit sei. Sie waren dieser
Reden bald überdrüssig und gingen fort, und wir Vernahmen,
das sie den Papisten rieten, nicht mit uns zu disputieren, noch
von unsern Büchern zu lesen; somit waren wir sie los. Aber
wir setzten uns mit allen andern Sekten auseinander, mit den
Presbyterianern\index{Presbyterianern}, 
den Independenten\index{Independenten}, 
den Seekers\index{Independenten}, 
den Baptisten\index{Baptisten},
den Episkopalen\index{Episkopalen}, 
den Socinianern\index{Socinianern}, 
den Brownisten\index{Brownisten}, 
den Lutheranern\index{Lutheranern},
den Calvinisten\index{Calvinisten}, 
den Arminianern\index{Arminianern}, 
den Fifthmonarchyleuten\index{Fifthmonarchyleuten}, 
den Feministen\index{Feministen}, 
den Rantern\index{Rantern}. 
Von diesen allen behauptete niemand,
den gleichen Geist und die gleiche Kraft wie die Apostel zu haben.
In diesem Geist und dieser Kraft verlieh uns also der Herr den
Sieg über sie alle. Was die Fisthmonarchyleute betrifft, so trieb
es mich, eine Schrift zu schreiben, um ihren Irrtum aufzudecken.
Sie erwarteten Christi persönliche Wiederkunft\index{Christi 
(persönliche) Wiederkunft} in äußerer Form
und Weise und setzten dazu das Jahr 1666 fest, und viele, wenn
es um diese Zeit donnerte und regnete, machten sich bereit, weil
sie meinten, nun komme Christus, um sein Reich aufzurichten,
und bildeten sich ein, sie müssten nun die Hure draußen in der
Welt töten (Offb. 17\bibel{Offb. 17}). Aber ich sagte ihnen, 
die Hure\index{Hure} sei lebendig
% \picinclude{./160-169/p_s164.jpg} 
in ihnen und noch nicht verzehrt vom Feuer Gottes und von
ihnen im Geist und der Kraft des Herrn vernichtet. Und ihre
Erwartungen, daß Christus äußerlich wiederkomme, um sein Reich
aufzurichten, sei wie das \glqq siehe hier, siehe da\grqq{} (Luc. 
17,28\bibel{Luk. 17:28)} der
Pharisäer. Aber Christus sei vor mehr als 1600 Jahren gekommen, 
um sein Reich auszurichten, wie Nebukadnezar\person{Nebukadnezar} 
geträumt und Daniel\person{Daniel} prophezeit, und 
habe die vier Reiche zertrümmert und
das große Volk mit dem goldenen Kopf und den Armen und
Beinen aus Silber, alles habe der Wind Gottes weggeblasen
wie im Sommer die Spreu beim Dreschen (Dan. 2, 32\bibel{Dan. 2:32)}).
Christus habe gesagt, als er auf Erden war: \glqq Mein Reich ist
nicht von dieser Welt.\grqq{} Wenn es von dieser Welt gewesen wäre,
so hätten seine Diener gekämpft (Joh. 18)\bibel{Joh. 18}; aber es war nicht
von dieser Welt, darum kämpften sie nicht. Alle diese 
Fifthmonarchyleute, die mit fleischlichen Waffen kämpfen, sind keine
Diener Christi, sondern Diener des Tieres und der Hure; Christus
sagt: \glqq Mir ist gegeben alle Gewalt im Himmel und auf Erden\grqq{}
(Matth. 28,18)\bibel{Matth. 28:18}, und sein Reich, das vor 
1600 Jahren aufgerichtet  wurde, herrschet noch. Und der Apostel sagt: 
\glqq Wir sehen Christus regieren, und er wird fort regieren, 
bis das alle Dinge ihm untertan sind\grqq{} (1. Kor. 15)\bibel{Kor. 1. 15@1. Kor. 15}.

In diesem Jahre, 1661\index{Jahr!1661}, trieb es viele Freunde übers Meer,
zu gehen, um die Wahrheit in fremden Ländern zu verkünden.
John Stubbs\person{Stubbs, John}, Henry Fell\person{Fell, Henry} 
und Richard Costrob\person{Costrob, Richard} trieb es nach
China\ort{China} und Priester Johannes Gegend zu gehen; aber kein Schiff
wollte sie nehmen. Mit vieler Mühe erhielten sie eine Vollmacht
vom König; aber die Ostindische Gesellschaft fand Mittel und
Wege, sie zu umgehen, und die Schiffsherren wollten sie nicht
nehmen. Sie begaben sich nun nach Holland, in der Hoffnung,
dort überfahren zu können; aber auch dort wollte sie niemand
nehmen. Nun nahm Henry Fell und John Stubbs ein Schiff,
das nach Alexandrien in Ägypten\ort{Ägypten} ging, in der Absicht, von dort
aus sich einer Karawane anzuschließen. Doch da kam Daniel
Baker\person{Baker, Daniel} und veranlasste Richard 
Costrop gegen seine innere 
Freiheit, mit ihm nach Smyrna\footnote{heutiger türkischer Name: 
Izmir}\ort{Izmir} zu gehen. Auf der Überfahrt wurde
Richard krank, da kümmerte sich Daniel Baker gar nicht um ihn
und er starb. Aber der hartherzige Mann verlor später seine Stelle.

John Stubbs und Henry Fell erreichten Alexandrien,\ort{Alexandrien} aber
sie waren kaum dort, als der englische Konsul sie schon verbannte;
% \picinclude{./160-169/p_s165.jpg} 
doch verbreiteten sie, ehe sie fort gingen, viele Bücher und Schriften,
um den Türken und Griechen den Weg der Wahrheit zu zeigen;
das Buch betitelt: \glqq{}Die Gewalt des Papstes gebrochen\grqq{}, gaben
sie einem alten Mönch, damit er es dem Papst bringe oder
schicke; als der Mönch es durchgelesen, legte er die Hand aufs
Herz und sagte: "`Was hier geschrieben steht, ist Wahrheit; wenn
ich es aber öffentlich bekennen würde, so würden sie mich verbrennen."'
John Stubbs und Henry Fell kehrten nach England zurück, weil es 
ihnen nicht erlaubt wurde, weiter zu gehen,
und kamen wieder nach London. Stubbs hatte eine Vision, das
die Engländer und Holländer, die sich verbündet hatten, sie nicht
überzuschiffen, sich untereinander entzweien werden, und so kam
es auch [...]

Wir hatten aber nicht nur Schweres von außen zu erdulden,
sondern auch unter uns durch John Perrot\person{Perrot, John} 
und seine Anhänger. Einem trügerischen Geiste nachgebend, 
suchte er unter den Freunden den schlechten, unziemlichen 
Brauch einzuführen, das man während
des allgemeinen Gebets den Hut aufbehalten solle.\ort{Hutstreit} Viele Freunde
hatten mit ihm und seinen Anhängern darüber gesprochen, und
ich hatte einigen deswegen geschrieben, aber er und andere taten
sich nur noch mehr gegen uns zusammen [...]

Eine der Sorgen, die die Freunde von außen trafen, war,
das man die Art, wie sie sich verheirateten, beanstandete. So
kam zum Beispiel folgender Fall vor das Gericht von Nottingham: \ort{Nottingham}
etwa zwei Jahre vorher hatten sich zwei aus der Gemeinschaft 
der Freunde geehelicht; da starb der Mann und hinterließ
der Frau, die guter Hoffnung war, einen Besitz an Land und Zinslehen. 
Als das Kind geboren war, erklärte es das Gericht als
Erbe\index{Erbschaft} seines Vaters, und es wurde als solcher anerkannt. Später
heiratete ein anderer Freund die Witwe. Daraufhin kam ein
naher Verwandter des ersten Mannes und verklagte den Freund,
der die Witwe geheiratet, und suchte ihm seinen Besitz zu entreißen
und das Kind seines Erbes zu berauben und alles an sich zu
bringen, als nächster Erbe des ersten Mannes. Um dies zu
begründen, suchte er die illegitime Geburt des Kindes zu beweisen
mit der Behauptung, die Ehe\index{Eheschließung} sei nicht nach dem Gesetz gewesen.
Bei den Verhandlungen gebrauchte der Kläger ungebührliche Ausdrücke 
gegen die Freunde und sagte, sie täten sich zusammen wie
das Vieh; und andere scheußliche Dinge. Nachdem die Anwälte
% \picinclude{./160-169/p_s166.jpg} 
beider Parteien gesprochen hatten, nahm der Richter die Sache
in die Hand und sagte, es sei eine Ehe im Paradies geschlossen
worden, als der Adam die Eva und die Eva den Adam genommen
hatte; es sei eben die Zustimmung der beiden Teile, maß eine
Ehe ausmache. Was die Quäker anbelange, so kenne er ihre
Ansichten nicht, aber er glaube nicht, das sie sich zusammentun
wie das unvernünftige Vieh, wie man von ihnen behaupte, sondern
wie Christen, und darum glaube er, die Ehe sei gesetzlich gewesen
und das Kind legitimer Erbe. Um das Gericht zu überzeugen,
brachte er einen anderen Fall: Ein Mann, der schwach und 
bettlägerig war, hatte in diesem Zustande den Wunsch, sich zu 
verehelichen und erklärte vor Zeugen, das er diese Frau zum
Weibe nehme und die Frau erklärte, das sie diesen zum Mann
nehme; diese Ehe wurde später angefochten, aber alle Bischöfe
erklärten damals die Ehe für gültig. Daraufhin entschied das
Gericht auch zu Gunsten des Quäkerkindeß, gegen den Mann,
der es um sein Erbe bringen wollte.

Um diese Zeit wurde der Supremats- und Huldigungseid
von den Freunden gefordert als eine Falle, denn man wusste,
das wir nicht schwören konnten, und es wurden in der Folge
viele gefangen gesetzt. Bei dieser Gelegenheit veröffentlichten
die Freunde die Schrift: "`Die Gründe und Ursachen, warum wir
nicht schwören"' und es trieb mich, derselben einige Linien 
beizufügen, damit man sie dem Magistrate gebe:

\brief{Magistrat}{
  Die Welt sagt: "`Küsse das Buch"'\footnote{Aus der Formel beim 
  Schwören des Eides.}; das Buch aber sagt:
  "`Küsse den Sohn, das er nicht zürne"' (Ps. 2:12)\bibel{Ps. 2:12}. 
  Der Sohn sagt: "`Bleibet bei Ja und Nein in euren Reden, denn maß darüber ist,
  das ist vom Ubel"' (Matth. 5:37)\bibel{Matth. 5:37}. Wiederum 
  sagt die Welt: "`Leget die Hand auf daß Buch"'; aber das Buch sagt: "`Was unsre Hände
  betastet haben vom Worte des Lebens"' (1. Joh. 1:1)\bibel{Joh. 1. 1:1@1. Joh. 1:1} [...] Und
  Gott sagt: "`Dies ist mein lieber Sohn, den sollt ihr hören"';
  er ist das Leben, die Wahrheit, das Licht und der Weg zu Gott."'
  \medskip 
  \begin{flushright}G. F.\end{flushright}
}

Weil so viele Freunde gefangen waren, verfassten Richard
Hubberthorn\person{Hubberthorn, Richard} und ich eine Schrift 
und ließen sie dem König überreichen,
damit er erfahre, wie wir von seinen Beamten behandelt wurden
sie lautete:

% \picinclude{./160-169/p_s167.jpg} 

\brief{König}{
  \begin{center}An den König:\end{center}

  \medskip 

  Freund,

  \medskip 

  Der du der Herrscher dieses Reiches bist! Hier ist eine Aufzählung 
  eines Teiles der Leiden, die das Volk Gottes, das man
  im Ärger Quäker nennt, zu erdulden hat. Unter dem Wechsel
  der Mächte, die deiner Regierung voran gingen, haben sie viel
  gelitten; 3170 wurden gefangen genommen um des Gewissens
  willen, und weil sie Zeugnis ablegten für die Wahrheit, die in
  Christus ist; und noch jetzt sind 73 Personen im Namen des 
  Commonwealth gefangen; 32 Personen starben im Gefängnis während
  der Zeit des Commonwealth und unter Oliver und Richard, in
  harter, grausamer Gefangenschaft, aus schmutzigem Stroh und in
  gräulichen Löchern. Und 3068 Personen sind seit deiner Rückkehr
  gefangen genommen worden durch solche, die sich damit bei dir
  einzuschmeicheln suchten. Zudem werden unsre Versammlungen
  täglich gestört durch Männer mit Waffen und Knütteln, obwohl
  mir friedlich zusammenkommen, nach der Art des Volkes Gottes
  der ersten Zeiten; unsre Freunde werden ins Wasser geworfen
  und werden blutig geschlagen; ja, es können gar nicht alle die
  Gräueltaten aufgezählt werden. Nun möchten wir gerne von dir
  erbitten, das du alle, die im Namen des Commonwealth und im
  Namen der beiden Protektoren und in deinem eigenen Namen
  um des Gewissens und der Wahrheit willen gefangen sind, frei
  gebest; haben sie doch nie die Hand erhoben gegen dich oder
  irgend sonst jemand, und das, wenn sich die Freunde friedlich
  versammeln, um Gott anzubeten, sie nicht mehr durch rohe 
  Bewaffnete gestört werden. Ein Hauptgrund dieser frühern 
  Gefangennahme war der, das wir den Protektoren und den verschiedenen
  Regierungen keine Eide leisten konnten; und nun tut man uns
  ins,Gefängnis, weil wir den Huldigungseid nicht leisten können.

  Wenn nun dir oder irgend einem Menschen gegenüber unser
  ja nicht ja und unser nein nicht nein sein sollte, dann lass uns
  dafür das leiden, was andere leiden müssen, wenn sie einen Eid
  brechen! Wir haben alle diese Jahre viel gelitten an unserm
  eigenen Leib und an unserer Habe, unter mancherlei Regierungen,
  weil wir nicht schwören, sondern Christi Gebot folgen, das sagt,
  "`ihr sollt überhaupt nicht schwören"'; dieses besiegeln wir mit Leib
  und Gut, mit unserm ja und nein, wie Christus es befiehlt.
  Bedenke das in der Weisheit, die aus Gott ist, damit du in
  % \picinclude{./160-169/p_s168.jpg} 
  derselben solchem Tun Einhalt gebietest, du, der du die Herrschaft
  hast und solches vermagst. Wir möchten, das alle, die jetzt im
  Gefängnis sind, frei werden und nicht wieder um der Wahrheit
  und des Gewissens willen gefangen genommen werden. Und wenn
  du untersuchst, ob sie unschuldig leiden, so las ihre Ankläger vor
  dich kommen, und wir wollen, wenn nötig, ausführlich Bericht
  über ihre Leiden erstatten.

  \medskip 

  \begin{flushright}G. F. und R. H.\end{flushright}

}

Zwei Freunde, beides Frauen, waren auf Malta\ort{Malta} bei der
Inquisition gefangen, Katharine Evans\person{Evans, Katharine} 
und Sarah Chevers\person{Chevers, Sarah}; da
es hieß, ein Lord D'Aubeny\person{Lord D'Aubeny}, 
ein römisch-katholischer Priester,
könne ihnen die Freiheit verschaffen, so ging ich zu ihm. Nachdem 
ich ihn über alles, was ihre Gefangennahme betraf, unterrichtet 
hatte, bat ich ihn, an die dortigen Behörden um ihre
Freilassung zu schreiben. Er versprach bereitwilligst, es zu tun
und das, wenn ich in einem Monat wieder komme, man mir
ihre Freisprechung mitteilen wolle. Als ich zur bestimmten Zeit
wieder hin kam, sagte er, sein Brief sei scheins nicht angekommen,
denn er habe keine Antwort erhalten, aber er versprach, nochmals
zu schreiben, und tat es auch, und sie wurden beide frei.
Mit diesem hohen Herrn redete ich viel über Religion, und
er gab zu, das Christus jeden, der in die Welt kommt, erleuchtet
mit seinem geistigen Licht, und das er den Tod für einen jeden
gekostet hat, und das die heilsame Gnade Gottes allen Menschen
erschienen ist und sie lehrt und ihnen das Heil bringt, wenn sie
ihr gehorchen. Ich fragte ihn darauf, wozu denn die Papisten\index{Papisten}
alle ihre Bilder und Reliquien brauchen, wenn sie an dieses Licht
glauben und die Gnade, die sie lehrt und ihnen das Heil bringt,
annehmen? Er antwortete, das seien nur Mittel, um das
Volk in Unterwürfigkeit zu erhalten. Er zeigte sich in dieser
Unterredung sehr weitherzig; ich hörte nie einen Papisten soviel
zugeben wie diesen [...]

Im gleichen Jahre, als ich in Cambridgeshire war, hörte ich,
das Edward Burrough gestorben war; und da ich wuste, wie
schwer und traurig dieser Verlust für die Freunde war, schrieb
ich folgende Zeilen zur Aufrichtung und Beruhigung ihrer Gemüter:


\brief{Quaker-Gemeinde}{
  Freunde,

  \medskip 

  Seid stille und ergeben und gefasst im Samen Gottes, der
  sich nicht ändert, damit ihr den lieben Edward Burrough unter
  % \picinclude{./160-169/p_s169.jpg} 
  euch spüren möget in diesem Samen, durch den er euch bei Gott,
  bei dem er jetzt ist, vertreten wird; durch diesen Samen könnet
  ihr ihn alle sehen und fühlen, denn in diesem ist Einigkeit und
  Leben; freuet euch seiner im unvergänglichen Leben, das 
  unsichtbar ist."' [...] 

  \medskip 
  \begin{flushright}G. F\end{flushright}.

}
