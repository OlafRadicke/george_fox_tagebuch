
%%%%%%%%%%%%%%%%%%% Kapitel 19. %%%%%%%%%%%%%%%%%%%%%%%%%%%%%%

\chapter[Nordamerika unter Engländern und Indianern.]{Nordamerika unter Engländern und Indianern.}

\begin{center}
\textbf{Arbeit in Nordamerika unter Engländern und Indianern.}
\end{center}


Wir schifften uns am 8. des 1. Monats 1671 ein; und da
wir schlechten Wind hatten, segelten wir eine ganze Woche hin
und her, ehe wir von Jamaika fort kamen. Es war eine 
schwierige und gefahrvolle Reise, besonders als wir den Golf von
Florida passierten, wo wir manche Schwierigkeiten durch Wind
und Sturm zu bestehen hatten. Aber der große Gott, welcher
Herr ist über Meer und Land, welcher aus den Flügeln des
Windes dahin fährt, bewahrte uns durch seine Kraft vor vielen
großen Gefahren, wenn bei dem Ungestüm des Wetters unser
Schiff oft nahe daran war umzuschlagen, und das Tauwerk 
größtenteils zerbrochen wurde. Wahrlich, wir merkten, das der Herr ein
Gott der Nähe ist, und hört auf das Flehen seines Volkes. Denn
als die Winde so stark und heftig tobten, und es so mächtig
stürmte, das die Schiffsleute sich nicht zu helfen wussten, und
das Schiff sich selbst überließen, da beteten wir zum Herrn, welcher
uns gnädig erhörte, Wind und Wellen stillte und uns günstiges
Wetter gab, so das wir uns unsrer Errettung freuen durften.
Gelobt und gepriesen sei der herrliche Name des Herrn, der
Macht hat über alles, dem Wind und Wellen gehorchen. [...]

Wir waren etwa sechs bis sieben Wochen unterwegs von
Jamaika nach Maryland [...] Dort trafen wir John Burnyeat,\person{Burnyeat, John}
der die Absicht hatte, sich bald nach England einzuschiffen, aber
als wir kamen, änderte er seinen Vorsatz und schloss sich uns an
zum Dienst für den Herrn [...] Er hatte eine Versammlung für
alle Freunde von Maryland veranstaltet, damit er sie alle 
miteinander sehe, um Abschied von ihnen zu nehmen; und nun
fügte es die Vorsehung Gottes\index{Vorsehung Gottes} so, 
das wir gerade zur rechten
Zeit landeten, um dieser Versammlung beizuwohnen; [...] Es
war eine sehr große Versammlung, die vier Tage dauerte [...]
% \picinclude{./220-229/p_s225.jpg} 
Nach der allgemeinen Versammlung fingen die Männer- und
Frauen-Versammlungen an [...] Hernach gingen wir nach einem
andern Orte, die Klippen genannt, wo eine andere große 
Versammlung stattfinden sollte. Wir gingen einen Teil des Weges
zu Land, den Rest zu Wasser; und da sich ein Sturm erhob, stieß
unser Boot auf und wäre fast zertrümmert worden, und das
Wasser drang herein. Ich schwitzte stark, da ich sehr warm aus
der Versammlung gekommen war, und nun wurde ich vom Wasser
ganz durchnässt; aber weil ich Glauben\index{Glaube} hatte in die göttliche Kraft,
wurde ich vor Schaden bewahrt, der Herr sei gepriesen [...]
Wir hatten hier auch eine Männer- und Frauen-Versammlung,
und in vielen dieser Versammlungen wurde die Angelegenheit der
Kirche geordnet.

Nach diesen Versammlungen trennten wir uns und verteilten
uns auf die verschiedenen Küsten, zum Dienst der Wahrheit.
James Lancaster\person{Lancaster, James} und 
John Eartwright\person{Eartwright, John} gingen zu Wasser nach
Neu-England\ort{Neuengland}; William Edmundson\person{Edmundson, William}
und drei andere Freunde schifften sich für Virginia\ort{Virginia} ein, 
wo die Dinge sehr in Unordnung geraten waren; John Burnyeat\person{Burnyeat, John}, 
Robert Widders\person{Widders, Robert}, George 
Pattison\person{Pattison, George}
und ich mit einigen andern Freunden gingen mit einem Boot
zur Ostküste\ort{Ostküste (Nordamerika)} und hatten dort am Ersten Tag eine 
Versammlung, wo viele die Wahrheit mit Freude aufnahmen und die
Freunde reichlich erquickt wurden. Es war eine große, selige
Versammlung, und es waren mehrere Personen von Rang aus der 
Gegend dabei, darunter zwei Friedensrichter. Es kam über
mich vom Herrn, dem \zitat{Kaiser} der Indianer\index{Indianer in 
der Andacht} und seinen \zitat{Königen}
sagen zu lassen, sie sollten zu dieser Versammlung kommen. Der
\zitat{Kaiser} kam und wohnte ihr bei, aber seine \zitat{Könige}, welche weiter
weg wohnten, konnten nicht zur rechten Zeit kommen; aber sie
kamen später nach, mit ihren Leuten. Ich hatte am Abend zwei
mal eine gute Zeit mit ihnen, und sie hörten das Wort des Herrn
gerne und bekannten sich dazu. Ich bat sie, das, was ich ihnen
sagte, dann auch ihrem Volke zu sagen und ihm zu verkünden,
das Gott jetzt die Hütte des Zeugnisses in der Wüste aufrichte
und das Panier und segensreiche Zeichen seiner Gerechtigkeit. Sie
benahmen sich sehr anständig und fragten, wann die nächste 
Versammlung sein werde, sie wollten dazu herkommen; aber sie
erzählten uns, sie hätten eine heftige Auseinandersetzung gehabt
mit ihren Räten, wegen ihres Kommens. Am folgenden Tage
% \picinclude{./220-229/p_s226.jpg} 
traten wir unsre Reise nach Neu-England an, eine schwierige
Reise, durch Wälder und Sümpfe und große Flüsse; dann mussten
wir die Wildnis passieren, die jetzt West-Jersy\ort{West-Jersy} genannt wird,
die aber damals nicht von Engländern bewohnt war; einen ganzen
Tag reisten wir, ohne Mann oder Frau, Haus oder Wohnort
zu treffen. Zuweilen nächtigten wir im Wald bei einem Feuer,
zuweilen in den Hütten und Häusern der Imdianer. Eines Abends
kamen wir in eine indianische Ortschaft und übernachteten beim
\zitat{König}, der ein sehr achtenswerter Mann war; er sowohl als sein
Weib nahmen uns sehr liebevoll auf, und seine Dienerschaft 
behandelte uns sehr ehrerbietig. Sie gaben uns Matten, um darauf
zu schlafen, aber zu essen hatten sie wenig, da sie an dem Tage
wenig gefangen hatten. In einer andern indianischen Ortschaft, in
der wir uns aufhielten, kam der \zitat{König} zu uns; er sprach ein
wenig englisch. Ich redete viel mit ihm und auch mit seinen
Leuten, und sie waren sehr lieb mit uns. Schließlich kamen
wir nach Middletown,\ort{Middletown} einer englischen Pflanzung 
in Ost-Jersy\ort{Ost-Jersy};
doch konnten wir nicht zu einer Versammlung bleiben, da es uns
trieb, rechtzeitig zur Halbjahresversammlung in der Oysterbay,\ort{Oysterbay}
auf Long-Island,\ort{Long-Island} zu sein [...] Ein Freund, 
Richard Hartshorn,\person{Hartshorn, Richard}
setzte uns in seinem eigenen Boote über nach Long-Island, und
am zweitfolgenden Morgen erreichten wir die Oysterbay, wo wir der
Halbjahresversammlung beiwohnten [...] Nachdem dann die
Freunde wieder nach Hause gegangen waren, blieben wir noch
einige Tage auf der Insel und warteten dann in der Oysterbay
auf günstigen Wind, um nach Rhode-Island\ort{Rhode-Island} zu gehen [...]
Dort kamen wir am 13. des 3. Monats an und wurden mit
Freuden von den Freunden aufgenommen. Am nächsten
Ersten Tage hatten wir eine große Versammlung, welcher der
Unterstatthalter und mehrere von der Behörde beiwohnten. Sie
wurden mächtig von der Wahrheit ergriffen. In der daraus
folgenden Woche sand hier die Jahresveriammlung für alle Freunde
von Neu-England und den übrigen angrenzenden Kolonien statt;
außer sehr vielen Freunden, die in diesen Gegenden lebten, kamen
noch John Stubbs\person{Stubbs, John} aus Barbados 
und James Lancaster\person{Lancaster, James} und
John Cartwrigth,\person{Cartwrigth, John} von 
verschiedenen Seiten, um ihr beizuwohnen.
Diese Versammlung dauerte sechs Tage; an den vier ersten Tagen
waren allgemeine, gottesdienstliche Versammlungen, zu welchen
eine große Menge Leute kamen; denn sie hatten keinen Priester
% \picinclude{./220-229/p_s227.jpg} 
in Rhode-Island und darum keinerlei bestimmte Form irgend
einer Art von Gottesdienst; und weil der Unterstatthalter mit
mehreren von der Regierung täglich\index{Tägliche Andacht} 
zu den Versammlungen kamen, wurden die Leute ermutigt, so 
das sie von allen Seiten herbei strömten. Wir hatten ein 
gesegnetes Wirken unter ihnen, und
die Wahrheit fand gute Aufnahme. Ich habe selten Leute von
dieser Art mit mehr Aufmerksamkeit, Fleiß und Liebe zuhören
sehen, als diese es im ganzen während der Vier Tage taten, was
auch andere Freunde beobachteten. Nachdem die öffentlichen
Versammlungen zu Ende waren, begann die Männerversammlung,
welche sehr zahlreich, köstlich und feierlich war; und tags daraus
war die Frauenversamnrlung, die ebenfalls sehr zahlreich und
feierlich war. Da diese beiden Versammlungen zur Ordnung
kirchlicher Angelegenheiten veranstaltet waren, so wurden viele
wichtige Dinge eröffnet und mitgeteilt, durch Rat\index{Rat}, 
Belehrung\index{Belehrung} und Unterweisung,\index{Unterweisung} 
für das Verhalten in den verschiedenen in Frage
kommenden Verrichtungen, damit alles rein, lieblich und kräftig
unter ihnen erhalten werde. Verschiedene Männer- und Frauen-
Versammlungen für andere Gegenden wurden in diesen beiden
Versammlungen beschlossen und eingerichtet, zur Fürsorge für die
Armen und andere kirchliche Angelegenheiten, und damit dafür
gesorgt werde, das alle, die die Wahrheit bekennen, auch nach
dem Evangelium wandeln. Als diese große Versammlung auf
Rhode-Island zu Ende war, wurde den Freunden der Abschied
etwas schwer; denn die herrliche Macht des Herrn, die über allen
war, und seine gesegnete Wahrheit und sein Leben, die sich über sie
ausgossen, hatten sie so untereinander verbunden und vereinigt,
das sie zwei Tage damit zubrachten, um sich unter einander und
von den Freunden auf der Insel zu verabschieden. Darauf gingen
sie fort, mächtig erfüllt von der Gegenwart und der Kraft Gottes,
freudigen Herzens, jeder in seine Heimat, nach den verschiedenen
Kolonien, in denen sie lebten. [...]

Während dieser Zeit fand eine Vermählung von zweien von
den Freunden statt, der wir beiwohnten. Es war im Hause
eines Freundes, der früher Gouverneur\index{Gouverneur} dieser Gegend gewesen war;
drei Friedensrichter und viele, die nicht zu uns 
gehörten\index{Fremde in der Andacht}, waren
zugegen; aber alle, diese sowohl als die Freunde, sagten, sie hätten
noch nie eine so andächtige Zusammenkunft gesehen bei einem
derartigen Anlass, noch eine so feierliche Vermählung und eine
% \picinclude{./220-229/p_s228.jpg} 
so vorzügliche Ordnung. Dermaßen durchdrang die Wahrheit
alle. Es ist hoffentlich für viele ein gutes Beispiel gewesen, denn
sie sind aus allen Teilen des Landes dazu gekommen. Ich hatte
innerlich viel durchzumachen wegen der Ranter\index{Renter}\index{Ranter} in dieser Gegend,
die eine Versammlung,\index{Versammlung!Störung} der ich nicht beigewohnt hatte, gestört
hatten. Ich zeigte darum eine Versammlung unter ihnen an, im
Glauben, das der Herr mir Macht über sie geben werde, was
er auch tat, zu seiner Ehre und Verherrlichung. Sein Name
sei gelobt ewiglich [...]

Darauf hatten wir eine Versammlung in Providence\index{Providence} [...]
ebenso in Narraganset\ort{Narraganset} [...] Von dort ging ich zu der Insel
Shelter.\ort{Shelter} [...] Dort hatten wir am Tage nach unsrer Ankunft,
einem Ersten Tage, eine Versammlung. In der gleichen Woche hatte
ich eine unter den Indianern,\index{Versammlung!mit Indianern} 
ihr \zitat{König} war zugegen und sein
Rat und einige hundert Indianer; sie setzten sich unter uns, ganz
wie die Freunde taten, und hörten aufmerksam zu, während ich
durch einen Dolmetscher, einen Indianer, der gut englisch sprach,
zu ihnen redete. Sie waren nach der Versammlung sehr lieb
und bekannten, das, was man ihnen gesagt habe, sei Wahrheit
gewesen [...].

Wir reisten nun umher und kamen schließlich nach Shrewsbury,\ort{Shrewsbury}
wo sich etwas ereignete, das damals eine wichtige Probe für uns
war. John Jay,\person{Jay, John} ein Freund aus Barbadoes, der mit uns von
Rhode-Island gekommen war und uns durch die Wälder von
Maryland begleiten wollte, bestieg ein Pferd, um es zu versuchen;
das Pferd sing an zu galoppieren und warf ihn ab, so das er
auf den Kopf fiel und den Hals brach, wie die Leute sagten.
Einige hoben ihn auf, in der Meinung, er sei tot, und trugen
ihn ein Stück weit und legten ihn unter einen Baum. Ich ging
sogleich zu ihm, und als ich ihn anrührte, hielt ich ihn auch für
tot. Während ich neben ihm stand, und ihn und die Seinen
beklagte, fuhr ich ihm durch die Haare, wobei sein Kopf sich hin
und her drehte, so schlaff war sein Hals. Nun nahm ich seinen
Kopf zwischen meine beiden Hände, und, indem ich meine Knie
gegen den Baum stemmte, hob ich seinen Kopf in die Höhe und
sah, das da nichts zerrissen oder gebrochen war; ich faste ihn
nun mit einer Hand unter dem Kinn und mit der andern hinten
am Kopf und bewegte seinen Kopf zwei oder dreimal mit aller
Kraft hin und her und renkte ihn ein. Ich bemerkte bald, wie
% \picinclude{./220-229/p_s229.jpg} 
sein Hals wieder anfing, Halt zu bekommen; darauf fing er an
zu röcheln und gleich darauf zu atmen. Die Leute waren entsetzt, 
aber ich hieß sie, guten Muts zu sein, Glauben zu haben
und ihn ins Haus zu tragen. Sie taten es und legten ihn neben
das Feuer. Ich hieß sie, ihm etwas Warmes zu trinken zu geben
und ihn zu Bett zu bringen. Nach einer Weile fing er an zu sprechen,
aber er wusste nicht, was mit ihm geschehen war. Am folgenden
Tage zogen wir weiter, und er mit uns, ziemlich wohl, etwa
16 Meilen zu einer Versammlung nach Middletown\ort{Middletown} durch Wälder
und Sümpfe und über einen Fluss, wo wir unsere Pferde hinüber
schwimmen ließen und selber aus einem hohlen Baumstamm hinüber
setzten. Er reiste noch viele hundert Meilen mit uns [...]

Wir hatten hier eine herrliche Versammlung [...] Nach der
selben gingen wir nach Middletown-Harbour, etwa fünf Meilen weit,
um am nächsten Tage unsre große Reise anzutreten, durch die
Wälder nach Maryland. Unsre Führer waren Indianer. Ich
beschloss, den Weg durch die Wälder auf der andern Seite der
Delawara-Bay\ort{Delawara-Bay} zu nehmen, um die Flüsse und Buchten so viel wie
möglich zu vermeiden. Wir machten uns am 9. des 7. Monats.
auf den Weg und kamen durch viele indianische Ortschaften und über
mehrere Flüsse und Sümpfe; als wir etwa 40 Meilen weit
geritten waren, machten wir ein Feuer für die Nacht und legten
uns daneben. Wenn wir zu Indianern kamen, verkündeten
wir ihnen den Tag des Herrn. Tags darauf reisten wir 50 Meilen.
Nachts fanden wir ein altes Haus, aus dem die Indianer die
Leute vertrieben hatten; wir machten ein Feuer und blieben dort
am Eingang der Delawara-Bay. Am folgenden Tage ließen wir
unsre Pferde etwa eine Meile weit über den Fluss schwimmen,
nach der Insel Upper-Dinidock, und darauf aufs Festland. Wir
selber hatten von den Indianern ein Kanu gemietet und uns
darin von ihnen übersetzen lassen [...]

Dann gingen wir nach Newcastle,\ort{Newcastle} jetzt Neu-Amsterdam\ort{Neu-Amsterdam}
genannt; [...] am 16. des 7. Monats zogen wir weiter [....]
Nach einer beschwerlichen Reise erreichten wir das Haus Robert
Harwoods\person{Harwoods, Robert} in Miles River\ort{Miles River} 
in Maryland\ort{Maryland} und wohnten am
folgenden Tage einer Versammlung bei. [...] Von da gings
nach dem Kentischen Ufer, [...] und dann zu Wasser, etwa zwanzig
Meilen weit, zu einer sehr großen Versammlung [...] Es war
eine gesegnete Versammlung und von großem Nutzen sowohl zur

% \picinclude{./230-239/p_s230.jpg} 
Bekehrung zur Wahrheit als auch zur Befestigung solcher, die
schon bekehrt waren [...] Der Herr sei gelobt, der seine 
Wahrheit sich ausbreiten lässt! Nach der Versammlung kam eine Frau
zu mir; ihr Mann war einer der Friedensrichter der Gegend
und ein Mitglied der Behörde und sagte mir, ihr Mann
sei krank und werde wahrscheinlich sterben; ich solle doch mit ihr
heim kommen, um ihn zu besuchen. Sie wohnte drei Meilen weit
weg, und da ich gerade erhitzt aus der Versammlung kam, so war
es- hart für mich, mit ihr zu gehen; doch im Bewusstsein meines
Berufes nahm ich ein Pferd, ging mit ihr, besuchte ihren Mann
und redete, was der Herr mir eingab. Der Mann wurde sehr
erquickt und erholte sich gänzlich durch die Kraft des Herrn und
kam später zu unsern Versammlungen. Ich kehrte wieder am
gleichen Abend zu den Freunden zurück; und am folgenden Tage
zogen wir von dort weiter, etwa 20 Meilen nach Tredhaven
Creek,\ort{Tredhaven Creek} von wo wir am 3. des 8. Monats 
zur allgemeinen Versammlung aller Freunde von Maryland gingen. Dieselbe dauerte
5 Tage. Die drei ersten Versammlungen waren für öffentlichen
Gottesdienst,\index{Öffentliche Versammlung} zu welchen 
aller Arten Leute kamen; die beiden
andern für Männer- und Frauen-Versammlungen. Zu den öffentlichen 
Versammlungen kamen viele Protestanten\index{Protestanten} von verschiedenen
Richtungen, und einige Papisten;\index{Papisten} es waren mehrere Personen
von der Obrigkeit und ihre Frauen darunter und andere Angesehene 
der Gegend [...] 

Als wir unsern Dienst in Maryland verrichtet hatten, und
da wir die Absicht hatten, nach Virginia\ort{Virginia} zu gehen, hielten wir
eine Versammlung in Paturent am 4. des 9. Monats, um uns
von den Freunden zu verabschieden [...]

Am 5. schifften wir uns ein für Virginia und erreichten nach
drei Tagen Nanceum.\ort{Nanceum} Danach eilten wir 
nach Carolina\ort{Carolina}; doch
hatten wir unterwegs mehrere schöne Versammlungen [...] Am
21. des 9. Monats nach einem beschwerlichen Weg durch die
Wälder, Sümpfe, Moräste, erreichten wir 
Bonners Creek\ort{Bonners Creek}; dort
brachten wir die Nacht am Feuer zu; eine Frau gab uns eine
Matte, um darauf zu schlafen; das war das erste Haus in 
Carolina, das wir erreichten.
Hier ließen wir unsre ermüdeten Pferde und fuhren in einem
Kanu den Fluss hinunter, nach Hugh 
Smiths\person{Smith, Hugh} Haus, wo Leute
aller Richtungen uns besuchten, und viele von ihnen nahmen uns
% \picinclude{./230-239/p_s231.jpg} 
freundlich auf; Freunde gab es in der Gegend nicht. Nathaniel
Batts war darunter, der frühere Gouverneur von Roan Dak\index{Roan Dak}. Er
war bekannt als Hauptmann Batts\person{Hauptmann Batts} 
und war ein rauher, heftiger
Mann gewesen. Er fragte mich nach einer Frau in Cumberland,
die, wie er gehört habe, durch unsre Gebete und Handauflegen
geheilt worden sei, nachdem sie lange krank und von den Ärzten
aufgegeben worden war, und er wollte wissen, ob es wahr sei.
Ich sagte ihm, wir rühmen uns solcher Dinge nicht, aber es seien
viele solche Dinge geschehen in der Kraft Christi. An der 
Connie~Dak~Bay\ort{Connie~Dak~Bay} empfing uns der Gouverneur 
liebevoll; aber ein dortiger
Gelehrter wollte durchaus mit uns disputieren. Und sein Widerstand 
war uns sehr nützlich, da er uns Gelegenheit gab, die Leute
über manches aufzuklären, das Licht und den Geist Gottes betreffend;
er wollte nicht gelten lassen, das sie in einem jeden seien, und 
versicherte, sie seien nicht in den Indianern.\index{Rassismus} 
Hierauf rief ich einen
Indianer herbei und fragte ihn, ob nicht, wenn er lüge oder
jemandem Böses tue, etwas in ihm sei, das ihn dafür strafe? Er
sagte, es sei so etwas in ihm, das ihn darüber strafe, und er
schäme\index{Scharm} sich, wenn er Unrecht getan oder etwas Unrechtes gesagt
habe. Also beschämten wir den Gelehrten vor dem Gouverneur
und dem Volk weil der gute Mann so weit gegangen war, das er
nicht einmal die Schrift gelten\index{Bibel, Bedeutung} lies [...].

Von hier ging ich zu den Indianern und redete durch einen
Dolmetscher zu ihnen; ich zeigte ihnen, das Gott alle Dinge in
sechs Tagen gemacht habe\index{Schöpfungslehre} und nur eine Frau für einen Mann
gemacht habe,\index{Monogamie} und das Gott die alte Welt vernichtete wegen ihrer
Schlechtigkeit. Darauf sprach ich ihnen von Christus und zeigte
ihnen, das er für alle Menschen gestorben sei,\index{Opfertod} für ihre Sünden
so gut wie für die der andern, und das, wenn sie Böses täten, er sie
verbrennen werde, wenn sie aber Gutes tun, sie nicht verbrannt
würden.\index{Hölle} Ihr junger \zitat{König} war unter ihnen und andere ihrer
Häuptlinge und sie schienen, was ich ihnen sagte, gut aufzunehmen.

Nachdem wir die nördlichen Gegenden von Carolina\ort{Nord Carolina} durchreist
und der Wahrheit hier einigermasen den Weg bereitet hatten,
gingen wir in der Richtung von Virginia zurück. Wir hatten
unterwegs mehrere Versammlungen, so auch eine sehr segensreiche
bei Hugh Shmith. Gelobt sei der Herr ewiglich! Die Leute waren
sehr empfänglich und unser Wirken war gesegnet unter ihnen. Es
war ein indianischer Häuptling dabei, der sehr lieb war und
% \picinclude{./230-239/p_s232.jpg} 
bekannte, was wir sagten, sei Wahrheit. Auch ein Indianerpriester,\index{Indianerpriester} 
ein \zitat{Pawaw}, wie sie sagen, war da, der ruhig mit den
andern da saß. Am 9. des 10. Monats gingen wir nach Bonners
Creek zurück, wo wir unsre Pferde gelassen, nachdem wir etwa
18 Tage in Nord-Carolina gewesen waren.

Da unsre Pferde nun ausgeruht hatten, gingen wir weiter
nach Virginia durch Wälder und Sümpfe, so weit wir an diesem
Tage kommen konnten. Am folgenden Tage hatten wir eine
schwierige Reise durch Sumpf und Moor und waren den ganzen
Tag sehr nass und schmutzig; des Nachts trockneten wir uns dann
an einem Feuer. Am nächsten Abend erreichten wir Sommertown\ort{Sommertown}.
Als wir zur Herberge kamen, sah uns die Frau des Hauses und sagte
ihrem Sohn, er solle die Hunde festbinden; sie hatten nämlich
in Virginia und Carolina große Hunde, um ihre Häuser zu 
bewachen, da sie so einsam im Walde leben; aber der Sohn sagte,
es sei nicht nötig, denn die Hunde machten sich nicht an diese
Art Leute. Als wir nun in das Haus kamen, sagte er, wir
seien wie die Kinder Israel, gegen die die Hunde nicht mucksten
(2. Mos. 11,7)\bibel{Mos. 02. 11:07@2. Mos. 11:7} [...]

Wir waren drei Wochen unterwegs durch Virginia [...] Als
wir das Werk, das uns hier obgelegen, beendet hatten, fuhren
wir am 30. des 10. Monats in einer offenen Schaluppe ab, um
nach Maryland zurückzukehren [...] Da ein starker Sturm sich
erhob, waren wir froh, vor Nacht das Ufer zu erreichen und
übernachteten in einem Hause in Willougby Point\index{Willougby Point}. [...] 
Am Morgen kehrten wir zu unserm Boot zurück und segelten so rasch
als möglich weiter. Aber am Abend erhob sich abermals ein
Sturm, so das wir wieder ans Ufer mussten, wo wir die Nacht
an einem Feuer zubrachten, um uns zu trocknen, und um uns
herum heulten die Wölfe [...] Am 3. des 11. Monats
war der Wind ziemlich günstig und wir benützen ihn, um so
schnell wie möglich fort zu kommen; am Abend erreichten
wir Milfordhaven\ort{Milfordhaven}. Darauf passierten wir 
den Rappahannockflus\ort{Rappahannock (Fluss)}, den 
Patomac\ort{Patomac}, und dann fuhren wir in der Richtung
des Paturentflusses\ort{ Paturent (Fluss)} und erreichten in der Frühe des Morgens
James Prestons\person{Preston, James} Haus. Wir waren sehr müde, gingen aber
doch am nächsten Tage, einem Ersten Tage, zu einer 
Versammlung in der Nähe [...] Die Kälte wurde um so 
empfindlicher und der Frost und Schnee so heftig, das es fast unerträglich
% \picinclude{./230-239/p_s233.jpg} 
war. Auch war es gefährlich umherzureisen [...] Am 27. des
11. Monats hatten wir eine köstliche Versammlung in einem
Tabakshaus. Am darauffolgenden Tage kehrten wir wieder zu
James Preston zurück; als wir dort anlangten, war sein Haus
in der vorhergehenden Nacht abgebrannt, so das wir die Nächte
bei sehr kaltem Wetter im Freien am Feuer zubringen mussten.
Wir machten die merkwürdige Beobachtung, das sich eines Tages
mitten in diesem kalten Wetter der Wind gegen Süden drehte
und es ganz unerträglich heiß wurde [...] Am 2. des 12. Monats
hatten wir eine herrliche Versammlung in Paturent\ort{Paturent}. Am 12.
reisten wir weiter in unserm Boot. Den Anamessy und den
Amoroca\ort{Amoroca} umgehend, kamen wir nach Manaoke. Dann passierten
wir den Wicocomaco, wo wir eine gesegnete Versammlung hatten;
dann gings zu Pferde etwa 24 Meilen weit durch Sümpfe und
Wälder zum Hause eines Richters, wo wir ebenfalls eine 
lebendige Versammlung hatten. Zu dieser kam eine Frau aus 
Anamessy, die viele Jahre schwermütig\index{Schwermütigkeit} 
gewesen war und oft während
zwei Monaten nichts redete und sich um nichts kümmerte; als ich
von ihr hörte, trieb mich der Herr, zur ihr zu gehen, und ihr zu
verkünden, das ihrem Hause Heil widerfahren sei. Nachdem ich
Worte des Lebens zu ihr geredet hatte und den Herrn für sie
angefleht, kam sie zurecht und zog mit uns umher zu den 
Versammlungen und ist seither gesund, dem Herrn sei Lob [...]

Nun hatten wir unsre Arbeit in dieser Gegend getan und
verließen Anamessy\ort{Anamessy}. Wir gingen zu Wasser etwa 50 Meilen,
nach dem Hungerflus [...] Danach etwa 40 Meilen nach dem
kleinen Choptankflus [...] Ehe wir dann weiter zogen, hatten
wir eine herrliche Versammlung, zu der viele Leute kamen; unter
anderem auch mehrere von den Behörden der Stadt, mit ihren
Frauen. Von den Indianern kam einer, den sie ihren \zitat{Kaiser}
nannten, ein Indianer-\zitat{König} und sein Dolmetscher, die alle sehr
andächtig zuhörten und sehr lieb waren. Es war eine grundlegende 
Versammlung. Dies war am 23. des 1. Monats. Am
24. gingen wir 10 Meilen weit zu Wasser nach der Indianerstadt, 
wo jener \zitat{Kaiser} wohnte. Ich hatte ihn von meinem Kommen
benachrichtigt und gebeten, seine \zitat{Könige} und Räte zu versammeln.
Am Morgen kam der \zitat{Kaiser} selber und führte mich in die Stadt;
und sie waren alle beisammen mit ihren Dolmetschern und ihren
Leuten, und die alte \zitat{Kaiserin} war auch da. Sie waren sehr ernst
% \picinclude{./230-239/p_s234.jpg} 
und ruhig und sehr aufmerksam, mehr als manche, die sich Christen
nennen. Es waren einige mit mir, die verdolmetschen konnten
und wir hatten eine schöne Versammlung mit ihnen, die großen
Nutzen schaffte, denn sie brachte die Wahrheit und die Freunde
bei ihnen in Ansehen. Gelobt sei der Herr! [...]

Nachdem wir die meisten Gegenden dieses Landes durchzogen
hatten und die meisten der Plantagen besucht und überall, wo wir
hinkamen, Alarm geblasen\index{Alarm blasen} hatten und Gottes Tag des Heils
den Leuten verkündet hatten, spürten wir im Geist, das wir
bald unsere Aufgabe in diesen Weltgegenden erfüllt hätten, und
wandten uns wieder Alt-England zu. Doch es verlangte uns,
und der Herr schenkte uns die innere Freiheit dazu, bis zur
nahen Generalversammlung\index{Generalversammlung} 
für Maryland\ort{Maryland} zu bleiben, damit wir
alle Freunde beisammen sehen könnten, ehe wir abreisten. Die
Zwischenzeit brachten wir damit zu, Freunde und Gemeindemitglieder
zu besuchen, Versammlungen um die Cliffs und Paturent\ort{Paturent} herum
beizuwohnen, und Antworten zu schreiben auf allerlei Kritteleien,
welche etliche Gegner der Wahrheit erhoben und verbreitet hatten,\index{Verteidigungsschrift}
um die Leute zu verhindern, die Wahrheit anzunehmen; so waren
wir also nicht müßig, sondern wirkten für den Herrn, bis zur
allgemeinen Provinzialversammlung, die am 17. des 3. Monats
begann und vier Tage dauerte. Am Ersten Tage hatten die
Männer und Frauen ihre geschäftlichen Versammlungen, in welchen
die Angelegenheiten der Kirche besorgt wurden, und viele Dinge
wurden ihnen geoffenbart zu ihrer Erbauung und Trost. Die
drei übrigen Tage wurden mit öffentlichen 
Versammlungen\index{Öffentliche Versammlung} zugebracht, 
wobei verschiedene angesehene Regierungsbeamte anwesend
waren und viele andere; sie waren alle im allgemeinen befriedigt
und manchen ging es zu Herzen. Denn es war eine wundervolle,
herrliche Versammlung, und die mächtige Gegenwart des Herrn
ward überall fühlbar und gesehen; gelobt und gepriesen werde
sein heiliger Name immerdar, welcher Herrschaft gibt über alles.
[...] Nach dieser Versammlung nahmen wir Abschied von den
Freunden [...] Am nächsten Tage, dem 21. des 3. Monats
1673\index{Jahr!1673} schifften wir uns ein für England und am 28. des 4. Monats
landeten wir im Hafen von Bristol\ort{Bristol}. Dieselbe treue Hand der
Vorsehung, die uns geleitet und glücklich hinüber gebracht hatte,
wachte bei unserer Rückkehr über uns und brachte uns glücklich
zurück. Lob und Dank sei seinem heiligen Namen innnerdar!
% \picinclude{./230-239/p_s235.jpg} 

Wir hatten während der Reise Viele köstliche Versammlungen
an Bord des Schiffes gehabt, gewöhnlich zwei in der Woche, in
denen die gesegnete Gegenwart des Herrn uns mächtig erquickte,
und oft goss sie sich aus über uns und machte die Anwesenden
empfänglich.
