%%%%%%%%%%%%%%%%%%% Kapitel 6. %%%%%%%%%%%%%%%%%%%%%%%%%%%%%%

\chapter[Falsche Offenbarungen]{Falsche Offenbarungen}

\begin{center}
\textbf{Fox der Hexerei verdächtigt. Falsche Offenbarungen 
bei Freunden. Gefangenschaft in Carlisle.}
\end{center}

[...] Von Lancaster ging ich zu Friedensrichter 
West\person{West, Friedensrichter}; Richard
Hubberthorn\person{Hubberthorn, Richard} begleitete mich. 
Da wir den Weg und die Gefahr
der Sandbänke nicht kannten, ritten wir über eine Stelle, über
die, wie wir nachher erfuhren, noch nie jemand zuvor geritten
war. Wir ließen unsre Pferde über sehr gefährliche Stellen
schwimmen. Als wir ankamen, fragte uns Friedensrichter West,
ob wir nicht zwei Männer hätten über die Sandbänke reiten
sehen. \zitat{Ich werde}, fügte er bei, \zitat{über 
kurzem ihre Kleider
% \picinclude{./070-079/p_s072.jpg} 
haben, denn sie sind sicher ertrunken, und ich bin der 
Leichenschauer}. Als wir ihm nun sagten, das wir diese Männer seien,
da wunderte er sich sehr und wollte kaum glauben, das wir nicht
ertrunken seien. Und die Priester und \textit{Fromen} benützten es, um
das Gerücht über mich zu verbreiten, ich könne nicht ertrinken und
man könne mich nicht bluten machen, also sei ich ein 
Zauberer.\person{Fox!Gerüchte über Zauberei}
Es war in der Tat oft vorgekommen, das ich kaum blutete, wenn
sie mich mit ihren Stöcken schlugen und meinen Leib arg 
misshandelten. Alle diese Verleumdungen kümmerten mich nicht um
meiner selbst willen\person{Fox!Verleumdungen}; nur 
um die Wahrheit war mir bange,
gegen die sie mit solchen Mitteln die Leute einzunehmen suchten;
denn ich dachte daran, wie ihre verräterischen Vorfahren den
Hausherrn Beelzebub genannt hatten 
(Matth. 10:25)\bibel{Matth. 10:25}, und so
konnten ja diese von dem Leben und der Kraft Gottes 
abgefallenen Christen mit seinem Samen nicht anders verfahren. Aber
die Kraft des Herrn erhob mich über ihre verläumderischen Zungen
und ihre blutige, mörderische Gesinnung; sie waren selber behext
und darum konnten sie nicht zu Gott und Christus kommen.

Von Friedensrichter West ging ich nach Swarthmore\ort{Swarthmore}, wo
die Kraft des Herrn die Verfolger niederhielt. Es trieb mich,
verschiedene Briefe von hier aus an die Magistrate, Priester und
\textit{Frommen} der Umgegend, die sich früher an den Verfolgungen
beteiligt hatten, zu schreiben [...] und hernach trieb es mich, an
die Leute in Uloerstone im allgemeinen einen Mahnbrief zu
schreiben [...]
Unter den eifrigsten Zuhörern und Nachfolgern des Priesters
Lampitt von Ullerstone war ein Adam Sands\person{Sands, Adam}, 
ein sehr schlechter,
verdorbener Mensch, der gerne die Wahrheit und ihre Anhänger
vernichtet hätte, wenn er gekonnt hätte. Es trieb mich, an diesen
also zu schreiben:

\brief{}{
    Adam Sands!

    \bigskip

    Ich wende mich an das Licht in deinem Gewissen, du Kind
    des Teufels\person{Fox!Beschimpfung}\index{Beschimpfung}
    \index{Beleidigung}, du Feind der Gerechtigkeit. Der Herr wird dich
    darniederwerfen, wenn du schon eine Zeitlang jetzt herrschest. Die
    Strafe Gottes\index{Strafe Gottes} muss dich treffen, der 
    du dich in deiner Bosheit gegen
    die reine Wahrheit Gottes verhärtest. Durch die reine Wahrheit
    Gottes, die du verfolgest und der du widerstrebst, wirst du 
    vernichtet werden; sie ist ewig und schließt auch 
    dich ein; du wirst
    in dem Lichte, das du verachtest, gesehen und in demselben 
    % \picinclude{./070-079/p_s073.jpg} 
    verdammt\index{Verdammung}, du in deinem tierischen Wesen 
    und dein Weib in seiner
    Heuchelei; euer Morden der Gerechtigkeit wird erkannt werden;
    das Licht in deinem Gewissen wird dir das, was ich dir hier
    schreibe, bezeugen und wird dich erkennen lassen, das du nicht
    aus Gott geboren bist\index{Wiedergeboren}, sondern das du 
    fern von der Wahrheit
    noch in einem tierischen Wesen\index{tierisches Wesen} bist. 
    Wenn je einmal deine Augen
    die aufgehen werden und du bereust, so wirst du sehen, das ich
    ein Freund deiner Seele bin und dein ewiges Heil will.

    \bigskip

    \begin{flushright}G. F.\end{flushright}
}

Dieser Adam Sands kam später elendiglich um .....
Ich ging nach Swarthmore zurück. Ich hatte grose Offen-
banmgen vom Herm, nicht nur uber göttliche Dinge, sondern
auch über äusere, die die Regierung betrafen. Eines Tages,
als ich im Gerichtssaal Richter Fell und Friedensrichter Benson
über die jüngsten Ereignisse sprechen hörte und oom Parlament,
dasidamals tagte, und das man das ,,lange Parlament« nannte,
trieb es mich, ihnen zu sagen, das, ehe zwei Wochen um seien,
das Parlament aufgelöst und der Redner von seinem Stuhl herunter
gerissen sein werde. Und als nach zwei Wochen Friedensrichter
Benson wieder kam, sagte er zu Richter Fell, jetzt sehe er, das
George Fox ein wahrer Prophet sei: Oliver Cromwell habe das
Parlament ausgelöst! (20. April 1653.)
Um diese Zeit fastete ich etwa 10 Tage lang, weil mein
Geist um der Wahrheit willen schwer heimgesucht war; denn
James Milner und Richard Näher hatten Einbildungen und viele
machten es ihnen nach. Dieser James Milner und einige seiner
Anhänger hatten zuerst wahre Offenbarungen; aber da sie in
Hochmut und Selbstiiberhebung gerieten, irrten sie von der
Wahrheit ab. Der Herr trieb mich, zu ihnen zu gehen und ihnen
Ihre Verirrungen vorzustellen; und sie kamen dazu, ihre Torheit
emzusehen, und gaben sie auf und kamen aus den Weg der
Wahrheit zurück. Darauf begab ich mich in eine Versammlung
M Arn-Side, der Richard Myer beiwohnte; er hatte lange einen
lshmen Arm gehabt. Der Herr trieb mich, ihm vor allen An-
wesenden zu sagen: ,,Stehe auf!« und er stand aus und streckte
seinen Arm, der so lange lahm gewesen war, aus und sagte:
»Wisset, alle ihr Leute, das ich heute geheilt worden bin.« Seine
Eltern wollten es kaum glauben, und als die Versammlung
vorbei war, nahmen sie ihn aus die Seite und zogen ihm sein


% \picinclude{./070-079/p_s074.jpg} 
Wams au?-; da sahen sie, das ez wahr sei. Er kam bald darauf
in eine Versammlung in Swarthmore und berichtete da, wie
der Herr ihn geheilt habe. Dornach befahl ihm der Herr, nach
York zu gehen in seinem Auftrag; aber er gehorchte dem Herrn
nicht; und der Herr schlug ihn abermals, das er etwa dreiviertel
Jahr daraus starb ....
Um diese Zeit wurde Anthony Pearson X), der ein Gegner der
Freunde gewesen war, gewonnen. Er kam nach Swarthmore,
und da ich gerade dort bei Oberst West war, holte man mich.
Oberst West sagte: ,,Geht, Fox, denn Jhr könnt dem Mann zu
grosem Nutzen gereichen«—. Also ging ich, und die Kraft des
Herrn ergriff ihn.
Um diese Zeit tat der Herr auch etlichen den Mund auf,
das sie den Priestern und dem Volk die Wahrheit verkündeten,
und viele wurden de-zwegen inö Gefängniö geworfen. Ich ging
nun nach Cumberland, wo Anthony Pearson, seine Frau und
mehrere Freunde mich nach Bootle begleiteten; Anthony Pearson
verlies uns dann, um zur Gerichtssitzung nach Carlisle zu gehen;
denn er war Frieden?-richter in drei Grafschasten. An einem
Ersten Tage ging ich ins Turmhauö von Vootle, und als der
Priester fertig war, sing ich an zu reden. Aber die Leute waren
sehr unverschämt und prügelten mich im Hofe. Einer gab mir
einen starken Schlag aus das Handgelenk, sodas man allgemein
glaubte, er hätte meine Hand in Stücke geschlagen. Der Kon-
stabler hätte gem den Frieden wieder hergestellt und einige, die
mich geschlagen, eingesteckt; aber ich lies es nicht zu. Nachdem
ich zu ihnen geredet, ging ich nach der Wohnung dez Joseph
Nicolson und der Konstabler begleitete mich, um mich vor der
Menge zu schützen.
Am Nachmittag hatte der Priester einen andern Priester
kommen lassen, einen sehr angesehenen Mann aus London. Ehe
ich inö Turmhautz eintrat, sas ich eine Weile auf dem Platz davor
und einige Freunde mit mir; aber die Freunde wurden getrieben, ins
Turmhauz zu gehen, und ich ging ihnen nach. Der Londoner
Priester brachte in seiner Predigt alle erdenklichen Schriftftellen
von falschen Propheten rmd Antichristen und wandte sie auf unö
an. Aber als er geendet, nahm ich alle die Schriftstellen noch
1) Friedenörichtev Pearson wurde ,,bekehrt, als er auf dem Richterstuhl sas«.


% \picinclude{./070-079/p_s075.jpg} 
Fox der Hexerei verdächtigt. Falsche Qssenbarungen usw. 75
einmal durch und kehrte sie gegen ihn. Darauf überfielen mich
die Anwesenden, aber der Konstabler befahl ihnen Ruhe. Nun
wurde der Priester zornig und erklärte, ich dürfe nicht an diesem
Ort reden. Ich erklärte ihm, er habe auch seine Stunde zum
Predigen gehabt, nun sei seine Zeit um, und nun dürfe ich so
gut die meine reden wie er, denn er sei auch nur ein Fremder
hier. Und ich öffnete ihnen die Schrift und zeigte ihnen, das
diese Stellen, die von falschen Propheten, Betrügern und Anti-
christen reden, sie und ihreögleichen betrefse und alle, die in ihren
Fusstapfen gehen und die gleichen Früchte hervorbringen wie sie;
und nicht uns-, denn man könne uns solche Dinge nicht nach-
sagen. Ich zeigte ihnen, wie sie nicht in den Fusstapfen der
wahren Propheten und Apostel seien und wies ihnen an den
Früchten, die sie hervorbringen, nach, das sie etz seien, von denen
die Schriftstellen handeln und nicht wir. Und ich verkündete ihnen
die Wahrheit und das Wort des Lebens und wies sie aus Christ-.13,
ihren Lehrer. Alletz war ruhig während ich redete; aber als ich
geeudet hatte und hinaus kam, waren die Priester in einer solchen
Wut, das ihr Mund gegen mich schäumte. Der Priester des
Orts redete auf dem Turmplatz zu den Leuten und sagte ihnen:
,,Dieser Mensch hat in Laneashire alle rechtfchassenen Männer
und Frauen für sich zu gewinnen gewust, und nun will er hier
da?-selbe tun.« Ich erwiderte ihm: ,,WaS bleibt dann für die
Priester übrig, auser solchen wie sie selber sind? Denn wenn es
die Rechtschafsenen sind, die sich zur Wahrheit bekehren und sie
aufnehmen und sich zu Christus bekehren, so sind es die Schlechten,
die dir und deinesgleichen folgen! Etliche suchten für ihren Priester
einzutreten, und für das Zehntenwesen; aber ich sagte ihnen, sie
täten besser, für Christus einzutreten, der den Zehntenpriestern
und dem Zehntenwesen ein Ende machte und der seine Jünger
aussandte mit der Weisung: ,,untsonst zu geben, wa-3 sie umsonst
empfangen hatten«. Und des-Z Herrn Macht kam über alle und brachte
sie zum Schweigen und hielt die Schreier zurück, das sie den Unfug,
den sie planten, nicht ausführen konnten. A13 ich zu Joseph
Nieolson zurück kam, entdeckte ich ein groses Loch in meinem
Rock, das von einem grosen Messerstich herrührte; aber ez war
nicht tiefer alö der Rock gegangen, denn der Herr hatte ihre
Ubeltat vereitelt .....
Darnach ging ich in ein Dorf, und eine grose Schar be-


% \picinclude{./070-079/p_s076.jpg} 
gleitete mich. Während ich in einem mit Leuten ganz gefüllten
Hauö das- Wort des Lebenö verkündete, gewahrte ich eine Frau,
die, wie ich gleich merkte, einen unsauberen Geist hatte. Der
Herr trieb mich, ernstlich mit ihr zu reden und ihr zu sagen, sie
sei unter dem Einflus eineö unsauberen Geisteö; hierauf verlies
sie das Zimmer. Weil ich fremd war an diesem Orte und die
äuseren Verhältnisse der Frau nicht kannte, wanderten sich die
Leute sehr und sagten mir nachher, ich hätte etwas Merkwiirdiges
entdeckt; denn diese Frau sei wirklich lange als eine schlechte
Person bekannt gewesen. Der Herr hatte mir die Gabe der
Unterscheidung gegeben, durch welche ich den Zustand und die
Verfassung der Leute ost erkannte und die Geister prüfen konnte
denn nicht lange vorher, alö ich in eine Versammlung ging, sah
ich auf dem Felde einige Frauen, bei denen ich einen unsauberen
Geist erkannte; und etz trieb mich, von meinem Wege ab zu ihnen
zu gehen und ihnen ihren Zustand aufzudecken. Ein andermal
kam eine in die Versammlung in Swarthmore, und etz trieb mich,
ernstlich mit ihr zu reden und zu sagen, sie stehe unter der Macht
eines bösen Geisteö; und die Leute sagten nachher, etz sei das
allgemein von ihr bekannt. Gin andermal kam eine andere Frau
und stand in einiger Entfernung von mir, und es trieb mich zu
ihr zu gehen und zu sagen: »Du bist eine Hure gewesen«; denn
ich erkannte den Zustand und das Leben dieser Frau; sie antwortete
mir, es gebe viele, die ihr ihre äusern Sünden nennen können,
aber ihre inwendigen habe ihr noch niemand sagen können; daraus
sagte ich ihr, ihr Herz tue nicht recht vor dem Herrn, rmd
aus dem inwendigen komme das auöwendige; diese Frau wurde
nachher von der Wahrheit dez Herrn überzeugt und schlos sich
den Freunden an .....
Wir gingen nun nach Earliöle ..... An einem Markttage
ging ich aus den Markt. Die Magistrate hatten Drohungen er-
gehen lassen und ihre Leute geschickt; und ihre Frauen hatten
gesagt, wenn ich komme, so reisen sie mir die Haare auö, und
die Schutzleute sollten mich nur sestnehmen. Aber ich ging dennoch
auf den Platz, im Gehorsam gegen den Herrn, und verkündete
ihnen dort, das der Tag des Herm über all ihr betrügerischeö
Tun und ihre betriigerische Ware komme; und sie sollten sich alle
abwenden von ihrem Beträgen und Überlisten und sich an Ja
und Nein halten und einander die Wahrheit sagen; dann komme


% \picinclude{./070-079/p_s077.jpg} 
Fox der Hexerei verdächtigt. Falsche Offenbarungen usw. 77
die Kraft und die Wahrheit des Herrn zu ihnen. Nachdem ich
ihnen so das Wort des Lebens verkündet hatte, in einem Ge-
dränge, das zu gros gewesen war, als das die Schutzleute und
die Weiber der Magistrate zu mir hätten gelangen können, zog
ich ruhig weiter. Viele Soldaten und andere kamen zu mir und
einige Baptisten, die heftige Streiter waren; unter diesen war
auch ein Helfer, ein böser Mann, der, als er die Kraft des Herrn
verspürte, aufschrie vor Zorn, worauf ich meine Augen auf ihn
heftete und ernstlich zu ihm redete in der Kraft des Herm; und
er schrie: ,,Durchbohre mich nicht so mit deinen Augen! wende
deine Augen ab von mir«.
Am folgenden Ersten Tage ging ich ins Turmhaus, und
nachdem der Priester geendigt hatte, predigte ich den Leuten
die Wahrheit und das Wort des Lebens. Der Priester entfernte
sich und man wollte mich aus dem Turmhaus jagen. Aber ich
verkündete den Weg des Herrn weiter unter ihnen und sagte:
,,ich komme, euch das Wort des Lebens und der Seligkeit zu ver-
künden«. Die Macht des Herrn tat sich mächtig kund unter
ihnen, so das sie zitterten und bebten, und meinten, das Turm-
haus schwanke, und einige meinten, es werde auf ihre Köpfe fallen;
die Weiber der Magistrate rasten und suchten mit aller Gewalt,
an mich heran zu kommen; aber die Soldaten und die Freunde
umringten mich. Zuletzt kam der ganze Pöbel der Stadt ins
Turmhaus, mit Stöcken und Steinen und schrie: ,,nieder mit
diesen rundköpfigen Schuften!« und warfen mir Steine an.
Hierauf schickte der Statthalter Soldaten ins Turmhaus, um
Ruhe zu schaffen unter den Leuten; mich nahmen sie freundlich
bei der Hand und hiesen mich mit ihnen kommen. Als wir
auf die Strase kamen, war die Stadt in Aufrrchr, und einige
dieser Soldaten kamen ins Gefängnis, weil sie sich meiner ange-
nommen hatten, gegen die Leute aus der Stadt. Gin Leutnant,
der belehrt worden war, nahm mich in sein Haus, wo eine Vap-
tistenversammlung war; auch Freunde kamen dazu, und wir hatten
eine sehr ruhige Versammlung; sie hörten das Wort des Lebens
gerne, und viele nahmen es auf. Am folgenden Tage, als die
Magistrate im Stadthaus versammelt waren, liesen sie mich vor
sie bringen. Ich war eben im Haus eines Baptisten; als ich
von dem Befehl hörte, ging ich nach dem Stadthaus hinauf, wo
viel Pöbel versammelt war, der allerlei falsche Dinge über mich


% \picinclude{./070-079/p_s078.jpg} 
auögesagt hatte. Ich hatte eine lange Unterredung mit den
Magistraten, worin ich auseinandersetzte, waz für Früchte die
Predigten ihrer Priester bringen, und wie wenig Christentum darin
sei; und ich sagte ihnen, das sie zwar als grose »Fromme«
gelten, — sie waren Presbhterianer und Jndependenten — aber
eben nicht im Besitz ihrer Frömmigkeit seien. Nach einem langen
Verhör verurteilten sie mich zum Gefängniö, als Gottes-lästerer,
Ketzer und Verführer, obgleich sie mich gerechter Weise keines
dieser Dinge beschuldigen konnten. EZ waren zwei Kerkermeister
im Kerker von Carliöle, ein oberer und ein unterer, die aus-
sahen wie zwei grose Värensührer. Als ich gebracht wurde,
führte mich der Oberkerkermeister in ein groses Zimmer und sagte
mir, ich könne hier haben, was ich wolle; aber ich erwiderte
ihm, er solle kein Geld von mir erwarten, denn ich werde weder
in einem seiner Betten schlafen, noch von seinen Speisen essen,
woraus er mich in ein anderes Gemach führte, wo ich nach einiger
Zeit etwas zum drauf liegen erhielt. Hier lag ich gefangen bis-
zur Zeit der Gerichtösitzung, wo ich, wie es- allgemein hies, er-
henkt werde. Der Oberscherifs Wilfrid Lawson, hetzte sie auf,
mich zu töten, und sagte, er wolle mich selbst bis zu meiner Hin-
richtung bewachen. Sie waren sehr streng und setzten drei Muske-
tiere zu meiner Wache, einen vor meine Türe, einen anderen
unten an die Treppe und einen dritten vor die Haustüre, und
sie liesen niemand zu mir, auser um mir das nötigste zu bringen.
Dez Nachts brachten sie Priester zu mir, oft erst um zehn Uhr,
die schrecklich roh und teuflisch waren. GS gab eine Rotte von
schottischen Priestern, Presbyterianer, zusammengesetzt auö Neid
tmd Bo?-heit, die nicht »geschickt waren, göttliche Dinge zu reden«
und sehr schmutzige Reden führten. Aber der Herr verlieh mir
durch seine Kraft die Herrschaft über sie alle, so das sie erkannten,
in welchem Geist sie waren und mas sie für Früchte brachten.
Auch angesehene sogenannte ,,Damen« (lmtjez) kamen, um den
Mann zu sehen, von dem es hies, er müsse sterben. Während
die Richter und Räte miteinander berieten, auf welche Art ich
sterben solle, vereitelte der Herr in Unerwarteter Weise ihren
Anschlag, indem der Anwalt einen Einwand verbrachte, der
alle ihre Absichten über den Haufen warf, so das sie keine
Macht mehr hatten, mich vor Gericht zu bringen .....
Nachdem die Richter die Stadt verlassen hatten, erhielt der


% \picinclude{./070-079/p_s079.jpg} 
Fox der Hexerei verdächtigt. Falsche Ossenbarungen usw. 79
Kerkermeister Befehl, mich in den untersten Kerker zu den Strasen-
räubern, Dieben und Mördern zu werfen, obgleich ich schon vor-
her in sehr strengem Gewahrsam gewesen war. Ich kam nun
an einen gräulichen, schmutzigen Ort, wo nicht einmal ein Abtritt
war, Amd Frauen und Männer in unziemlicher Weise zusammen-
gesperrt waren, und die Gefangenen waren voll Läuse, so das eine
Frau fast davon aufgesressen wurde; aber so schlecht auch der
Ort war, so kamen doch die Gefangenen alle dazu, mir zugetan
und ganz nachgiebig zu werden, und etliche wurden von der Wahr-
heit bekehrt, wie dies bei Zöllnern und Huren zu allen Zeiten
geschehen, so das sie jeden Priester, der ans Gitter kam, um mit
ihnen zu diöputieren, zu Schanden machen konnten. Der Kerker-
meister war sehr hart und der Unterkerkermeister roh gegen mich und
gegen die Freunde, die zu mir kamen. Gr schlug oft Freunde,
die nur ans Gitter kamen, um mich zu sehen, mit einem grosen
Knüttel. Ich konnte am Gitter hinaus steigen, um zuweilen etwas
Fleisch herein zu langen, was ihn schrecklich bös machte. Einmal
überkam ihn ein solcher Zorn, das er mich mit einem Kniittel
durchprügelte und dazu schrie: ,,komm vom Fenster weg!« obschon
ich gerade damals nicht dran war. Während er mich schlug,
—kam es in der Kraft des Herrn über mich, zu singen, mas ihn
noch wütender machte. Esr holte einen Geigenspieler und lies
ihn vor mir spielen, weil er meinte, mich damit zu verdriesen.
Aber während seinem Spiel kam eZ über mich, in Gottes ewiger
Kraft zu singen, und meine Stimme übertäubte den Lärm des
Geigerö, waö ihn so oerwirrte, das er das Spielen aufgab und
sich daoonmachte.
Richter Vensontz Frau fühlte sich getrieben, mich zu besuchen,
und kein anderes- Fleisch zu essen, al-3 von dem, daö man mir
an die Kerkertür brachte. Später wurde sie selbst in York ins
Gesängnis getan, während sie schwanger war, weil sie einem
Priester widersprochen hatte, und man gestattete ihr nicht, aus
dem Gefängniö zu gehen zur Zeit ihrer Niederkunft; so gebar
sie im Kerker ein Kind. Sie war eine gläubige, gottselige Frau,
und blieb etz biö zu ihrem Tode.
Während meiner Gefangenschaft im Kerker zu Carliöle ver-
breitete sich das Gerücht von meiner wahrscheinlicher! Hinrichttmg
überall hin. Als sie im Parlament — ich glaube es wurde das
kleine Parlament genannt — hörten, es sollte in Carliöle ein


% \picinclude{./080-089/p_s080.jpg} 
junger Mann um seines Glaubens willen hingerichtet werden,
schrieben sie deshalb an die Magistrate.
Ungefähr um die gleiche Zeit schrieb ich an die Behörden
von Earlisle, die mich ins Gefängnis geworfen und die die
Freunde auf Anstiften der zehntengierigen Priester verfolgten:
,,Freunde! Thomas Eraston und Cuthbert Stadholm,
Guer Tun ist in London bei den Gutgesinnten bekannt ge-
worden. Was habt ihr alles geleistet an Gesangennehmen,
Güterschändungen, Metzeleien und anderen Scheuslichkeiten in den
letzten paar Jahren! ganz menschenunwiirdig, wie wenn ihr
noch nie die Schrift gelesen und zu Herzen genommen hättet!
Jst das das Ziel der Religion Earlisles und seiner Kirche und
seiner Ehristlichkeit? ihr habt es zu schanden gemacht mit eurer
Blindheit, eurem tollen Treiben und eurem verkehrten Gtfern.
War es nicht immer die Art der blinden Leiter und der falschen
Propheten zu zanken (Jes. 56), mit denen, die ihnen den Mund
nicht füllen wollen? Seid ihr nicht die Lasttiere und Diener der
Priester gewesen? Wenn sie euch anspornen, das Schwert gegen
den Unschuldigen zu gebrauchen, so rennt ihr auf solche, die nach 3
den Befehlen der Schrift die Waffe nicht gebrauchen dürfen, loss
Und doch wollt ihr eure unheiligen Hände und gemeinen Lippen
zu Gott erheben, und gebet oor, zu fasten und seid doch voll
Hader und Zank (Jes. 58, 4). Brannte nie euer Herz in euch?
habt ihr nie über euren Zustand nachgedacht? Seid ihr ganz
der Lust des Teufels, dem Verfolgen, anheimgefallen? Wo ist
eure Feindes-liebe? (Matth. 5). Wo ist euer Beherbergen der
Fremdlinge? (Matth. 25, 35). Wie überwindet ihr Böses mit
Gutem? (Röm. 12, 21). Wo sind eure Lehrer, die ,,durch heil-
same Lehre die Widersprecher strafen?« (Tit. 1, 9) .... Leset die
Schrift und sehet, wie unähnlich ihr den Aposteln und Propheten
seid; und wie ihr denen gleichet, die die Propheten, die Apostel
und Christus verfolgten. Jhr gehet in ihren Fusstapfen und
kämpfet mit Fleisch und Blut, nicht mit den Fürsten der Welt,
die in der Finsternis dieser Welt herrschen, und mit den bösen
Geistern unter dem Himmel« (Gph. 6, 12). Jn keinem anderen
Lande geschehen solche Greuel, das man den Leuten ihr Gut
raubt, ihnen ihre Ochsen und Rinder nimmt, ihre Schafe, ihr
Getreide und ihr Hausgeräte und gibt es den Priestern, die doch
nichts für sie gearbeitet haben. Jhr seid eher Strasenräuber


% \picinclude{./080-089/p_s081.jpg} 
Fox der Hexerei verdächtigt. Falsche Qsfenbarungen usw. 81
als Diener Gottes gegen die Freunde; ihr verklagt sie bei euren
Gerichten und legt ihnen Busen auf, weil sie die Gebote Christi
nicht übertreten, also nicht schwören wollen« ..... G
Anthony Pearson and Gervase Benson dursten mich nicht im
Gesängniö besuchen, obwohl sie Frieden?-richter waren. Sie
schrieben darum an die Magistrate und Priester von Carlisle:
,,Wir bezeugen, das dieser George Fox, der von den Magi-
straten, von den Friedenörichtern, den Priestern und dem Volt ver-
folgt wird und gegenwärtig alt?. Gotteslästerer und Verführer
gefangen gesetzt ist, ein Prediger dez Wortes Gottes ist und das
ewige Evangelium verkündet; durch sein mächtiges Predigen hat
der grose Vater der Heiligen den Blinden die Augen geöffnet,
den Tauben die Ohren aufgetan, die Gefangenen erlöst und die
Toten auferweckt (Jes. 35, 5). Christus wird jetzt gepredigt unter
den Seinen, wie er war und ist; und weil er mm, in der Gestalt
seineö getreuen Dieners, wieder erscheint, sv verfolgen ihn die
Abgefallenen, Fürsten, Herrscher, Priester und Volk. Nicht als
ein Übeltäter leidet er von euch, ihr Magistrate, sondern weil er
nicht abgefallen ist und gegen das Treiben der Welt und das
Böse auftritt. EZ ist immer so gewesen, das, wo die oerderbte
Natur den Samen Gottes unterdrückte, die Verderbten suchen die,
in denen dieser Same ausging, gefangen zu nehmen .... Wie
Christuö daö, mas man einem der Geringsten erweist, als ihm
getan ansieht (Matth. 5, 25), also siehet er auch das, wa?. man
ihnen nicht tut, als ihm nicht getan an. Wenn ihr nun soweit
geht, das ihr nicht einmal anderen gestatten wollt, einen gefangenen
Bruder in seinen Leiden zu besuchen, so werdet ihr in den feurigen
Pfuhl, der mit Schwefel brennt, geworfen (Offb. 19, 20). Der
Herr ist gekommen, die Berge zu stürzen und zu Staub zu zer-
nialmen (Jes. 41, 15), und er wird rächen die Unterdrückung der
Gewissen seineö Volkes- an allen ungerechten Herrschern, Beamten
und Gesetzen. Er wird seinem Volke sein Gesetz geben nicht nach
dem, wat?. vor Augen ist, sondern nach Recht und Gerechtigkeit.
Man hat nun gesehen, wie eure Herzen voll Has sind gegen die
Wahrheit Gotteö, die er durch sein von der Welt oerachteteö und
zum Spott ,,Quäker« genannteö Volk verkünden läst. Jhr seid
ärger als die Heiden, die Pauluö inö Gefängnis warfen; denn
niemand hat damals seinen Freunden verboten, ihn zu besuchen,
Gkotge Fox. 6


% \picinclude{./080-089/p_s082.jpg}
darum treten sie gegen euch als Zeugen auf. M ist offenbar
geworden, das ihr denen gleich seid, die Christus töteten und die
Apostel gefangen nahmen unter dem gleichen Vorwand, nämlich
das sie den Jrrtum Wahrheit und die Diener Gottes Gottes-
lästerer nannten. Aber das Gericht, das über euch kommen wird,
ist schrecklich, ihr ungerechten Magistrate und Priester und ihr
alle, die ihr mit Worten die Wahrheit bekennet, und doch die
Kraft der Wahrheit und die, die in der Wahrheit sind und für die
Wahrheit einstehen, verfobget. Gehet in euch, dieweil es Zeit
ist, und bedenket, was Jesaias 17 geschrieben steht!«
Geroase Benson
Anthony Pearson.
Bald darauf kam die Macht des Herrn über die Richter
und sie setzten mich frei. Kurz vorher war Anthony Pearson
mit dem Gouverneur in meinen Kerker gekommen um zu sehen,
wie ich behandelt werde. Sie fanden den Ort so gräulich und
den Geruch so schlecht, das sie sich über die Magistrate entsetzten,
die solches von dem Kerkermeister geschehen liesen. Sie liesen
die Wärter in den Kerker kommen und sich für ihr Betragen
rechtfertigen. Den Unterkerkermeister, der so grob gewesen war,
sperrten sie darauf zu uns ins Gefängnis unter die Räuber.
Nachdem ich nun frei war, ging ich zu Thomas Bewley . . .
Dann ging ich auss Land und hatte viele grose Versammlungen . . .
und tausende bekehrten sich zum Herrn Jesus Christus-.
Dann ging ich nach Westmorland . . . Durham, Hexhain . . .
Gilsland . . . nach Eumberland ..... Hier überall, sowie in
Northumberland, Laneashire und Yorkshire fanden grose Be-
kehrungen statt, und was Gott gepflanzt hatte, wuchs und gedieh
unter dem Himmelsregen von oben und Gottes leuchtender Herr-
lichkeit, sodas sich vieler Mund öffnete zum Lobe Gottes; ja: ,,aus
dem Munde der Unmitndigen und Säuglinge richtete er sich eine
Macht zu« (Psalm 8, Z).


% \picinclude{./080-089/p_s083.jpg} 