% \picinclude{./170-179/p_s170.jpg} 
A
170 Kapitel Ill.
Gesetz. Darauf brachten sie die Verordnungen gegen Quäker und
andere. Jch sagte, das gehe ja gegen solche, welche die Unter-
tanen des Königs gefährden und Grundsätze haben, welche der
Qbrigkeit gefährlich seien; also gehe es nicht gegen uns, denn wir
hätten keine der Obrigkeit gefährlichen Grundsätze und unsere
Versammlungen seien friedliche. Sie behaupteten, ich sei ein
Feind des Königs. Jch antwortete: ,,Wir lieben jedermann und
sind niemands Feind; was mich betrifft, so bin ich ins Ge-
fängnis zu Derby gebracht worden, weil ich nicht wollte die
Waffen gegen den König nehmen, und nachher bin ich von Oberst
Hacker nach London gebracht worden als ein Mitoerschworener
für die Rückkehr König Karls und dort gefangen gewesen, bis
Oliver mir die Freiheit schenkte«. Sie fragten mich, ob ich während
des Aufstandes gefangen gewesen sei? Ich sagte: »Ja, ich war
damals gefangen und seither wieder und erhielt die Freiheit auf
des Königs Vesehl«. Ich erklärte ihnen die Verordnung und
machte sie aus die letzte Kundmachung des Königs aufmerksam und
brachte ihnen Beispiele von andern Friedensrichtern und was
das Oberhaus darüber gesagt hatte. Ich redete auch mit
ihnen über ihren Seelenzustand und ermahnte sie, in der Furcht
Gottes zu wandeln und gegen ihre gottesfürchtigen Mitmenschen
mild zu sein und auf Gottes Weisheit zu achten, durch welche alle
Dinge geschaffen seien, damit diese Weisheit ihnen zu teil werde
und sie leite, sodaß sie in derselben alles zu Gottes Ehre
regieren möchten. Sie verlangten, daß wir uns verpflichten
sollten, bei der nächsten Gerichtssitzung zu erscheinen, aber wir oer-
weigerten jegliche Verpflichtung auf Grund unserer Unschuld. Dar-
aus wollten sie uns versprechen machen, nie mehr hierher zu
kommen, aber wir ließen uns auch darauf nicht ein. Als sie sahen,
daß sie nichts erreichten, sagten sie, sie wollten uns zeigen, daß
sie gewillt seien, uns höflich zu behandeln; der Bürgermeister habe
nämlich die Güte, uns die Freiheit zu schenken. Ich erwiderte,
ihr hösliches Benehmen bekunde eine anständige Gesinnung, und
so gingen wir von dannen .....
Joseph Hellen und G. Vewley waren im Loo gewesen, um
Blanch Pope, eine Rantersrau, zu besuchen, angeblich um sie zu
bekehren; aber ehe sie sie wieder verließen, waren sie so verstrickt
in ihre Ansichten, daß sie fast im Begriffe schienen, eher ihre An-
hänger zu werden, besonders Joseph Hellen. Sie hatte sie unter


% \picinclude{./170-179/p_s171.jpg} 
Ein Gotteßgericht. Verhaftung wegen angeblicher Verschwörung usw. 171
anderm gefragt: ,,Wer machte den Teufel? war es- nicht Gott?«
Diese einfältige Frage verblüffte die Beiden so, daß sie nicht ant-
worten konnten. Sie legten mir nachher die Frage vor, ich ver-
neinte sie, ,,denn« sagte ich, ,,alleS waß Gott machte, war gut,
und der Teufel ist nicht gut; er hieß Schlange, ehe er Teufel und
Feind hieß, und darnach wurde er Teufel genannt. Später
wurde er Drache genannt, weil er ein Zerstörer war. Der Teufel
blieb nicht in der Wahrheit (Joh. 8,44) und als er die Wahrheit
verließ, wurde er der Teufel. Von den Juden hieß eß, alß sie
die Wahrheit verließen, sie seien vom Teufel, und man nannte
sie Schlangen (Matth. 23). Für den Teufel gibt ez keine Ver-
heißung, daß er je wieder zur Wahrheit zurückkehren werde, aber
für die Menschen, die von ihm verführt werden, steht die Ver-
heißung, daß der Same de-J Weibeß der Schlange den Kopf zer-
treten und ihre Macht zertrümmern werde (1. Mos. 3). Nachdem
diese Fragen auöführlich zur Beruhigung der Freunde erörtert
worden waren, sahen sich die Beiden, die den Geist der Runterz-
frau hatten aufkommen lassen, von der Wahrheit gerichtet; der
eine, Joseph Heilen, wandte sich ganz von unz ab und die Freunde
erkannten ihn nicht mehr alö zu ihnen gehörend; der andre da-
gegen, George Bewley, wurde wieder zurückgewonnen und wurde
später recht brauchbar .....
Ich hörte von einem Oberst Robinson in Cornwall, einem
bösen Menschen, der bei der Rückkehr dez Königs- zum Friedenß-
richter gemacht worden war, daß er die Freunde grausam ver-
folge nnd oiele von ihnen inS Gefängniß getan habe; alß er hörte,
daß ihnen durch die Gunst dez Kerkermeiftertz einige kleine Frei-
heiten zugestanden wurden und sie auögehen durften, um Weib
und Kinder zu sehen, erhob er dezwegen beim Gericht eine An-
klage gegen den Kerkermeister, und dieser mußte eine Buße von
20 Pfund bezahlen, und die Freunde wurden einige Zeit sehr knapp
gehalten. Nach der Gerichtßsitzung schickte dann Robinson zu
einem benachbarten Friedenörichter und ließ ihm sagen, er solle
ihm helfen, auf diese Fanatiker Jagd zu machen. An dem Tage,
als sie MM ihr Vorhaben au?-führen wollten, schickte, er feinen
Knecht mit den Pferden Vorauß und ging zu Fuß von seiner
Wohnung nach einer Farm, auf der er seine Kühe und seine
Milchwirtschaft hatte und wo seine Knechte und Mägde gerade
am Melken waren. Ab?. er kam, fragte er nach dem Stier; die


% \picinclude{./170-179/p_s172.jpg} 
172 Kapitel ZI?.
Mägde sagten, sie hätten ihn auf dem Felde eingesperrt, weil er
störrig sei bei den Kühen und sie am Melken hindere. Da ging
er ins Feld und begann nach seiner Gewohnheit seinen Stock gegen
den Stier zu schwingen, der Stier schnaubte nach ihm und holte
nach rückwärts aus, dann kehrte er sich und rannte wütend auf
ihn los und bohrte ihm die Hörner in die Seite, nahm ihn auf
die Hörner, schleuderte ihn über sich hinweg und riß ihm die Seite
aus bis zum Bauch, dann wiihlte er mit den Hörnern im Boden
und brüllte und leckte seines Herrn Blut aus. Als eine der
Mägde den Herrn schreien hörte, rannte sie ins Feld, packte den
Stier bei den Hörnern und riß ihn von ihrem Meister weg.
Der Stier stieß sie ganz sanft mit seinen Hörnern zur Seite, ohne
ihr weh zu tun, und ließ nicht ab, sein Opfer zu durchstechen und
sein Blut auszulecken. Nun rannte sie davon und holte ein paar
Männer, die in einiger Entfernung arbeiteten, um ihrem Meister
zu helfen. Aber es gelang ihnen erst den Stier wegzubringen, als
sie die Kettenhunde auf ihn hetzten, da rannte er wutschnaubend
davon. Als die Schwester Robinsons hörte, was geschehen, kam
sie heraus und sagte: »Ach, Bruder, welch schweres Gericht hat
dich betrosfen!« Er antwortete: »Ja wahrlich ein schweres Ge-
richt! laß den Stier töten und sein Fleisch den Armen geben.««
Sie brachten ihn nach Hause, aber er starb bald darauf. Der
Stier war so wild geworden, daß sie ihn erschießen mußten, denn
niemand konnte sich ihm nähern, um ihn zu töten. So gibt der
Herr ost Beweise seines gerechten Gerichts über die Verfolger
seines Volkes, aus daß man sich fürchte und sich in acht nehme. . .
Jch kam nach Swarthmore, wo man mir sagte, Oberst Kirby
habe seine Leute geschickt, um mich festzunehmen. Während der
Nacht, als ich in meinem Bett lag, trieb mich der Herr, am
nächsten Tage nach Kirbyhall zu Oberst Kirby zu gehen, fast zwei
Stunden weit, um mit ihm zu reden; ich ging denn auch ....
und sagte ihm, ich hätte gehört, er wolle etwas von mir, ob er
irgend etwas gegen mich habe? Er sagte vor allen Anwesenden,
daß er ein Gentleman sei und darum nichts gegen mich habe,
hingegen solle Mistreß Fell keine Versammlungen in ihrem Hause
haben, das sei gegen die Verordnungen. Ich erklärte ihm, diese
Verordnungen tressen nicht uns, sondern die, welche sich ver-
sammeln, um Komplotte und Verschwörungen zu machen; ....
die, welche sich bei Margaret Fell versammelten, seien friedliche


% \picinclude{./170-179/p_s173.jpg} 
Ein Gotteögericht. Verhaftung wegen angeblicher Verschwörung usw. 173
Leute. Nachdem wir längere Zeit miteinander geredet, gab er
mir die Hand und wiederholte, daß er nichtß gegen mich habe.
Sv kehrte ich nach Swarthmore zurück ..... Bald darauf
ging Oberst Kirby nach London in eine Privatsitzung der Richter
in Holkerhall, und dort wurde ein Verhaftbefehl gegen mich auf-
gesetzt . . . Jch hörte davon und hätte gut entwischen können, . . .
aber da das Gerücht ging von einer Verschwörung, so fürchtete
ich, sie würden, wenn ich mich davon machte, über die Freunde
herfallen, wenn ich aber bleibe, so würden sie mich nehmen, und
die Freunde könnten sich eher davon machen, und ich blieb also . , .
Am folgenden Tage kam ein Beamter mit Pistole und Schwert.
Jch sagte ihm, ich wisse, warum er komme, und sei dageblieben,
um mich festnehmen zu lassen; . . . ich verlangte, daß er mir den
Befehl zeige, aber er weigerte sich. So ging ich mit ihm, und
Margaret Fell begleitete unß nach Holkerhall ..... Dort wurde
mir unter anderem der Supremat?-eid vorgelegt; alß ich ihn nicht
schwören wollte, verlangten einige, daß ich inß Gefängniß von
Lancaster geschickt werde, andere wollten nur, daß ich verspreche
an der Gericht?-sttzung zu erscheinen, worauf ich entlassen wurde,
und ich kehrte also wieder mit Margaret Fell nach Swarthmore
zurück.
Am Gerichtßtage ging ich wie verabredet war, nach Lancaster . .
Der alte Richter Ratvlinson, der Vorsitzende, fragte mich, ob ich
um die Verschwörung wisse? Jch sagte, ich habe in Yorkshire
davon gehört. Gr fragte mich, ob ich etz den Behörden ange-
zeigt? Jch erwiderte, ich hätte ja Schriften gegen Verschwörungen
geschrieben ..... Sie legten mir den Suprematö- und Huldi-
gungßeid vor; ich sagte ihnen, daß ich nicht schwören könne, weil
Christuß und seine Apostel es- verboten hätten, und sie hätten ja
schon genugsam erfahren, wie ez bei solchen gehe, welche schwören,
ich aber habe noch nie in meinem Leben einen Eid geleistet. Hierauf
fragte mich Rawlinson, ob ich es für gesetzwidrig halte, zu
schwören? Diese Frage stellte er absichtlich, um mich zu fangen;
denn es war eine Verordnung gemacht worden, daß alle, die
sagen, ez sei gesetzwidrig zu schwören, verbannt oder hart bestraft
würden. Aber weil ich die Falle merkte, vermied ich sie und er-
klärte ihm, daß in den Tagen des Gesetzeß, bevor Christuö ge-
kommen sei, daß Gesetz den Juden geboten habe, zu schwören
(3. Mos. 19); Christuz aber, der in den Tagen dez Evangeliums


% \picinclude{./170-179/p_s174.jpg} 
174 Kapitel IV.
das Gesetz erfüllte, befehle, überhaupt nicht zu schwören (Matth. 5),
und der Apostel Jakobuö verbiete daß Schwören selbst denen, die
Juden waren und daß Gesetz Gotteß hatten. Nach vielem Hin-
und Herreden riefen sie den Gesangenwärter und verurteilten mich
zum Gefängniö. Ich trug die Schrift bei mir, die ich gegen
Verschwörungen geschrieben hatte, und bat, daß man sie vor dem
ganzen Gericht?-hofe vorlese oder lesen lasse, aber sie wollten nicht.
Als ich nun solchermaßen eingesperrt war, dafür, daß ich mich
geweigert hatte zu schwören, war mir daran gelegen, daß sie und
alle Leute wissen möchten, daß ich um der Lehre Christi willen
leide und darum, daß ich seine Gebote gehalten. Ich hörte später,
daß die Richter sagten, sie hätten besondere Befehle vom Oberst
Kirby gehabt, mich zu verfolgen, trotz seinem schönen Benehmen
und seiner anscheinenden Freundlichkeit damalß, als er vor allen
Anwesenden erklärt hatte, er habe nichts- gegen mich ....
Ich wurde biß zur Gerichtöverhandlung gefangen gehalten, und
da Richter Turner und Richter Twiszden gerade an der Reihe waren,
wurde ich vor Richter Twiötden gebracht, am 14. Tage desk-
Monatö, den man März nennt, im Iahre 1663. A15 ich vor-
geführt wurde, sagte ich: ,,Friede sei mit euch allen«. Der
Richter sah mich an und fragte: ,,Warum kommst du hier vor
Gericht mit dem Hut aus dem Kopf?« A13 der Kerkenneister mir
ihn hieraus wegnahm, sagte ich: ,,Da3 Hutabnehmen ist doch nicht
eine Ehre, die vor Gott gilt!« Daraus fragte mich der Richter:
,,Wollet ihr den Huldigungßeid leisten, George Fox?« Ich er-
widerte: ,,Jch habe nie in meinem Leben einen Eid geleistet, noch
mich zu irgend einem Vertrag oerpfiichtet«; darauf fragte er:
»Wollt ihr schwören oder nicht?« Ich erwiderte: »Ich bin ein
Christ, und Ehristuö befiehlt, nicht zu schwören, ebenso der Apostel
Iakobuß, und ob ich Gott oder Menschen gehorchen soll, darüber
urteile du selbst«. Er sagte: ,,Ich frage euch nochmals, ob ihr
schwören wollt oder nicht?« Ich antwortete abermaltz: ,,Ich bin
weder Türke, noch Jude, noch Heide, sondem ein Christ und
werde mich zum Christentum bekennen«. Und darauf fragte ich
ihn, ob er nicht wisse, daß die Christen der ersten Zeiten unter
den 10 Verfolgungen, sowie auch einige Märtyrer in den Tagen
der Königin Maria sich weigerten zu schwören, weil Christsuö
und die Apostel es verboten hätten; ferner sagte ich ihm, sie
hätten ja genugsam die Erfahrung gemacht, wie viele zuerst dem


% \picinclude{./170-179/p_s175.jpg} 
Ein Gotteögericht. Verhaftung wegen angeblicher Verschwörung usw. 175
König geschworen hatten und nachher gegen ihn; waz mich be-
treffe, so habe ich nie in meinem Leben einen Eid geleistet, und
meine Huldigung bestehe nicht im Leisten eineß Eideö, sondern
darin, daß ich Wahrheit und Treue halte,. denn, sagte ich, ich
ehre jedermann, wieoielmehr denn den König. Ehristuß aber,
der große Prophet und der König aller Könige und Heiland der
Welt, der große Richter der ganzen Erde, hat gesagt, daß man
nicht schwören soll, soll ich nun Christu?7 oder dir gehorchen?
Denn etz geschiehet au-H Gewissens:-’zartheit und auz Gehorsam gegen
Christi Gebote, daß ich nicht schwöre, und wir haben ja ein
König?-wort für zarte Gewissen. Daraus fragte ich den
Richter, ob er den König anerkenne: ,,Ja«, sagt er, ,,ich aner-
kenne den König«. ,,Warum«, fragte ich, ,,befolgst du denn dann
nicht seinen Erlaß von Breda und seine Versprechen, die er bei
seiner Rückkehr machte, daß niemand um der Religion willen ver-
folgt werde, solange er ruhig lebe? Wenn du den König aner-
kennst, warum verfolgst du mich, verhöhnst mich und treibst mich
dazu, einen Eid zu leisten, maß doch Sache des Glaubenß ist, und
siehst doch, daß weder du noch sonst jemand mich eineß unfried-
lichen Lebens zeihen kann''. Hierauf wurde er sehr gereizt und
sagte: ,,Kerl, wollt ihr schwören!« Jch sagte darauf, ich sei keiner
seiner ,,Kerl=Z«, sondern ein Christ und es stehe einem alten Richter nicht
an, hier zu sitzen und den Gefangenen Spottnamen zu geben,
weder seinen grauen Haaren noch seinem Amt. Darauf sagte er:
»J—ch bin auch ein Ehrist«. ,,So handle auch christlich«, sagte ich;
»Kerl«, sagte er, »willst du mir mit deinen Reden Angst machen?
Aber«, fügte er Verlegen hinzu, ,,jetzt brauche ich ja dieses- Wort
wieder!« und er bezwang sich. Ich sagte: ,,Jch rede in Liebe so
mit dir, weil eine solche Sprache dir als; Richter nicht ansteht.
Du solltest deinem Gefangenen das- Gesetz erklären, wenn er un-
wissend ist und einen oerkehrten Weg geht«. ,,Jch rede ebenfalls-
in Liebe mit dir«, sagte er: ,,aber«, erwiderte ich, ,,die Liebe ge-
braucht keine Spottnamen«. Daraus erhob er sich und sagte: »Jch
lasse mich nicht von dir einschüchtem, du sprichst so laut, deine
Stimme iibertäubt die meinige uud alle andern, ich müßte drei
oder oier Au?-rufer kommen lassen, um dich zu übertönen, du hast
gute Lungen«. Ich erwiderte: ,,Jch bin hier gefangen um Jesu
willen, um seinetwillen leide ich und stehe ich heute hier, und
wenn meine Stimme fünfmal so laut wäre, so würde ich sie er-


% \picinclude{./170-179/p_s176.jpg} 
176 Kapitel RV.
heben und erschallen lassen sür Ehrisiu?-, für dessen Sache ich
heute vor dem Richtstuhl stehe im Gehorsam gegen Christus,
welcher gebietet, nicht zu schwören, vor dessen Richtstuhl ihr alle
stehen und Rechenschaft ablegen müßt«. ,,So antworte mir nun
George For«, sagte er, »ob du den Eid leisten willst oder nicht«.
Ich erwiderte: ,,Jch frage dich nochmalß, ob ich Gott oder den
Menschen gehorchen soll? beurteile du daß selber. Wenn ich
überhaupt einen Eid leisten wollte, so wäre es dieser; aber ich
leugne überhaupt alle Eide, nicht nur den oder jenen, nach der
Lehre Christi, der seinen Nachfolgern gebot, überhaupt nicht zu
schwören. Wenn nun du oder sonst jemand von euch, oder eure
Prediger oder Priester mir beweisen wollen, daß Christud oder
seine Apostel irgend einmal, nachdem sie alleß Schwören verboten
hatten, ez den Christen wieder geboten, so will ich schwören«.
Ich sah, daß verschiedene Priester zugegen waren, aber nicht ein
einziger wollte reden. »Nun denn«, sagte der Richter, »ich bin
ein Diener des König;3 und der König hat mich nicht geschickt,
um mit dir zu dißputieren, sondern daß Gesetz an dir auözuüben;
legt ihm also den Huldigungßeid oor«. »Wenn du den König
lieb hast«, sagte ich, »warum hälst du dich nicht an daß, watz er
sagt? und an seine Erklärung, in der er unß Gewissenßsreiheit zu-
gesagt hatte? Ich bin ein Mann mit einem zarten Gewissen und
kann auß Gehorsam gegen Christi Gebot nicht schwören«. »Wenn
er also nicht schwören will'', sagte der Richter, »so führet ihn
in den Kerker«. Ich sagte, eß sei um Christi willen, daß ich
nicht schwören könne, ihm müsse ich gehorchen; aber der Herr
möge ihnen allen vergeben. So führte mich der Kerkermeister
hinweg, aber ich sühlte, daß deß Herrn mächtige Kraft über ihnen
allen war .....
Während ich nun hier im Kerker war, trieb setz mich, an
Richter Flemming, einen der hestigsten Verfolger der Freunde,
folgendermaßen zu schreiben: »O, Richter Flemming! Barmherzig-
keit, Milde und Güte zieret die Menschen und auch die Behörden.
O, hörest du nicht daß Schreien derer, die durch die Versolgungen
Witwen und Waisen geworden sind? Sind sie nicht wie
Schafe von Konstabler zu Konstabler getrieben worden, wie wenn
sie die größten Übeltäter und Bösewichter im Lande wären? GS
betrübt die Herzen oieler einsichtiger Leute, zu sehen, wie man
ihre ehrlichen Mitmenschen, die ein friedsameß, stilleß Leben ge-


% \picinclude{./170-179/p_s177.jpg} 
Ein Gottesgerirht. Verhaftung wegen angeblicher Verschwörung usw. 177
ssührt, behandelt hat. Wieder ist einer gestorben, den ihr ins Ge-
fängnis geworfen; er hat fünf Kinder hinterlassen, die nun ver-
waist sind. Solltest du nun nicht für diese vaterlosen Kinder
sorgen, sowie auch für die Weiber und Hinterlassenen der andern?
Jst es nicht deine Pflicht? Denke an Hiob, Kap. 29: »Gr war
ein Vater der Armen; er errettete den Armen, der da schrie und
die Waisen, die keinen Helfer hatten, er brach die Kinnbaeken des
Ungerechten und riß den Raub aus seinen Zähnen.« Und nun
vergleiche dein Leben mit dem seinen und hüte dich vor dem Tage
des Gerichts, welcher kommen wird, und vor dem Urteil Christi,
wenn ein jeder muß Rechenschaft ablegen und den Lohn empfangen
für seine Taten. Als-dann wird es heißen; o, wo sind die ver-
lorenen Tage! — Als John Stubbs vor dich gebracht wurde, der
ein Weib und vier kleine Kinder hatte, und mit seiner Hände
Arbeit nur den dürstigsten Lebensunterhalt verdiente, da riesest
du: fordert diesem Menschen den Eid ab! und als er dir vor-
stellte, daß er ein armer Mann sei, ließest du kein Mitleid auf-
kommen und wolltest ihn nicht hören, und nun ist er im Ge-
fängnis, weil er nicht schwören konnte, also nicht das Gebot
Christi und der Apostel übertreten konnte. Hoffentlich wirst du für
seine Familie sorgen, damit seine Kinder nicht Hungers sterben.
Jst denn das dem König gehuldigt, wenn man tut, wovon Christus
und die Apostel sagen, es sei Unrecht und führe in die Verdamm-
nis? Jhr würdet wohl auch Christus und die Apostel, welche
das Schwören verboten, ins Gefängnis geworfen haben, wenn sie
zu eurer Zeit gelebt hätten.
Denke auch an deinen armen Mitmenschen William Wilson,
der allgemein als ein fleißiger Mann bekannt war, und der sein
Weib und seine Kinder ehrlich durchbrachte, obgleich er nichts be-
saß, als was er durch seiner Hände Arbeit erwarb. Sogar auf
den Märkten wird über den Tod dieser Beiden geredet; man hört
das Schreien derer, die um der Gerechtigkeit willen Witwen und
Waisen geworden sind. Wenn John Stubbs und William Wil-
son geschworen hätten, so hätten sie damit ihre Freiheit wieder
erlangt, wenn sie auch daneben es mit den Marktschreiern und
Schnurranten gehalten hätten. O gehet in euch! es ist solches
nicht nach des Herrn Sinn. Und auch der König hat erklärt, es
solle gegen keinen seiner Untertanen, der friedlich lebe, eine Grau-
samkeit ausgeübt werden. Sodann sind einigen sehr rechtschafsenen
George Fc;. 12


% \picinclude{./170-179/p_s178.jpg} 
178 Kapitel K7.
Leuten Bußen auferlegt worden, obgleich sie selber nichts hatten,
und es eher am Platze gewesen wäre, ihnen etwas zu geben, als
ihnen noch etwas zu nehmen. Weil du weißt, daß sie um ihrer zarten
Gewissen willen keinen Eid schwören können, so stellst du ihnen da-
mit eine Falle. Wie denkst du, daß das Volk über ein derartiges
Tun redet? Sie sagen: Wir wissen, daß die Quäker sich an ihr
ja und nein halten, andere dagegen sehen wir schwören und wieder
abschwören! Jch weise dich an den Geist Gottes in deinem Ge-
wissen, Richter Fleming, der du so eifrig die Gefangennahme
des George Fox betriebeft und so böse warst über die, die ihn
nicht gefangen nahmen. Wo ist dein Erbarmen mit den armen,
oerwaiften Kindern? Hüte dich vor der Grausamkeit des Herodes,
der kein Mitleid kannte; Esau hat es also gemacht und nicht
Jakob! Thomas Walters von Bolton ist auch hier im Gefäng-
nis und wird darin festgehalten, weil er sich nach Christi Gebot
weigert zu schwören, und dabei hat er fünf kleine Kinder und seine
Frau ist ihrer Niederkunst nahe; du solltest dich doch seiner an-
nehmen und dafür sorgen, daß seine Frau und seine Kinder nicht
Mangel leiden, da sie durch deine Schuld verwaist dastehn.
Klingt dir das Schreien der Verwaisten nicht in den Ohren, und
siehest du das Blut derer, die durch dich umgekommen sind, nicht
vor dir? Es wird dich am Tage des Gerichts ein schweres Ur-
teil treffen, wie willst du dich verantworten, wenn du nach deinen
Werken gerichtet werden wirst und vor den Richterstuhl des
Allmächtigen treten mußt? .... Aber trotz alledem sagen wir
Quäker: der Herr vergebe dir und rechne dir diese Dinge nicht
an, wenn es sein heiliger Wille ist.« G. F.
Bald darnach starb Richter Flemings Weib, und hinterließ
ihm dreizehn oder vierzehn mutterlose Kinder .....
Einige Zeit vorher war Margaret Fell auch von Richter
Fleming als Gefangene nach Lancaster geschickt worden, und als
ste, an der Gerichtssitzung, den Eid nicht schwören wollte, wurde
sie weiter zum Gefängnis verurteilt. ....
Während ich im Gefängnis zu Lancaster war, hieß es, der
Türke werde über die Christenheit herfallen, und viele kamen in
große Angst. Eines Tages, als ich in meiner Zelle auf und
nieder ging, kam es über mich vom Herrn, daß ich sah, wie die
Krast des Herrn sich gegen den Türken kehrte, sodaß er wieder
umkehren mußte, und ich teilte einigen mit, was der Herr mich


% \picinclude{./170-179/p_s179.jpg} 
Ein Gotteögericht. Verhaftung wegen angeblicher Verschwörung usw. 179
hatte sehen lassen, und binnen eineß Monatö kam die Nachricht,
daß er geschlagen worden war.!)
Ein andermal alö ich in meiner Zelle auf- und niederging
und zum Herrn aufschaute, sah ich den Engel dee- Herrn, wie er
mit einem leuchtenden Schwert gen Süden wieß, und daß ganze
Schloß schien in Feuer zu stehn. Nicht lange darauf brach der
Krieg in Holland aus, 2) und dann eine große Seuche und dann daß
Feuer in London ; 9-) da war wahrlich daß Schwert des Herm gezogen.
Durch die lange Gefangenschaft an diesem ungesunden Orte
war ich sehr angegriffen in meiner Gesundheit, aber die Kraft
des Herm war stärker alß alles, sie half mir hindurch und hielt
mich aufrecht und half mir für den Herrn wirken, so viel der
Ort es erlaubte. Ich antwortete denn auch während dieser Zeit
auf mehrere Bücher, wie: ,,die Messe«, ,,da;3 Eommon Prayer
Buch«, »daß Direetorrs«, ,,daH Kirchenbekenntni:-tz««, welcheß die
vier mächtigsten Religionen 4) sind, die sich seit den Tagen der
Apostel erhoben.
Nach der Gerichtßoerhandlung war es- einigen der Richter
etwas ungemütlich, daß ich in Lancaster war, denn ich hatte sie
bei den Verhandlungen tüchtig geärgert, und sie bemühten sich
sehr darum, daß man mich anderöwohin bringe .... Etwa sechß
Wochen nach der Gerichtsverhandlung erhielten sie denn auch den
Befehl vom König und dem Rat, mich von Lancaster fortzubringen,
und zugleich kam ein Brief vom Earl von Anglesea, worin ee;
hieß, daß, wenn alle-Z, dessen man mich beschuldigt hatte, wahr
sei, so verdiene ich keinerlei Nachsicht noch Milde. Und doch war
daß Ärgste, waß sie gegen mich vorgebracht hatten daß-’, daß ich
einem Gebot Christi nicht ungehorsam sein konnte .....
Sie brachten mich nun nach Schloß Scarbro, wo sie mich in
ein Gemach führten und mir einen zur Wache setzten. Da ich seht
schwach war und öfters ohnmächtig wurde, so ließen sie mich
« manchmal mit der Wache an die frische Luft gehen; nach einiger
Zeit brachten sie mich in ein andereß Gelaß, daß offen war, so
1) 1664 Sieg der Abendländet (Deutschland und Frankreich) über die
Türken bei der Abtei St. Gotthardt.
2) 1665 Krieg zwischen England und den Niederlanden.
3) Oktober 1665 die große Pest in London, September 1666 der große
Brand in London.
4) Römische, Bischässliche, Preßbhterianer und Jndependenten.
12*

