% \picinclude{./170-179/p_s170.jpg} 
Gesetz. Darauf brachten sie die Verordnungen gegen Quäker und
andere. Ich sagte, das gehe ja gegen solche, welche die Untertanen 
des Königs gefährden und Grundsätze haben, welche der
Obrigkeit gefährlich seien; also gehe es nicht gegen uns, denn wir
hätten keine der Obrigkeit gefährlichen Grundsätze und unsere
Versammlungen seien friedliche. Sie behaupteten, ich sei ein
Feind des Königs. Ich antwortete: "`Wir lieben jedermann und
sind niemands Feind; was mich betrifft, so bin ich ins Gefängnis 
zu Derby gebracht worden, weil ich nicht wollte die
Waffen gegen den König nehmen, und nachher bin ich von Oberst
Hacker nach London gebracht worden als ein Mitverschworener
für die Rückkehr König Karls und dort gefangen gewesen, bis
Oliver mir die Freiheit schenkte"'. Sie fragten mich, ob ich während
des Aufstandes gefangen gewesen sei? Ich sagte: "`Ja, ich war
damals gefangen und seither wieder und erhielt die Freiheit auf
des Königs Befehl"'. Ich erklärte ihnen die Verordnung und
machte sie auf die letzte Kundmachung des Königs aufmerksam und
brachte ihnen Beispiele von andern Friedensrichtern und was
das Oberhaus darüber gesagt hatte. Ich redete auch mit
ihnen über ihren Seelenzustand und ermahnte sie, in der Furcht
Gottes zu wandeln und gegen ihre gottesfürchtigen Mitmenschen
mild zu sein und auf Gottes Weisheit zu achten, durch welche alle
Dinge geschaffen seien, damit diese Weisheit ihnen zu teil werde
und sie leite, so das sie in derselben alles zu Gottes Ehre
regieren möchten. Sie verlangten, das wir uns verpflichten
sollten, bei der nächsten Gerichtssitzung zu erscheinen, aber wir 
verweigerten jegliche Verpflichtung auf Grund unserer Unschuld. 
Daraus wollten sie uns versprechen machen, nie mehr hierher zu
kommen, aber wir ließen uns auch darauf nicht ein. Als sie sahen,
das sie nichts erreichten, sagten sie, sie wollten uns zeigen, das
sie gewillt seien, uns höflich zu behandeln; der Bürgermeister habe
nämlich die Güte, uns die Freiheit zu schenken. Ich erwiderte,
ihr höfliches Benehmen bekunde eine anständige Gesinnung, und
so gingen wir von dannen [...]

Joseph Hellen\index{Personen!Hellen, Joseph} und G. 
Bewley\index{Personen!Bewley, G.} waren im Loo gewesen, um
Blanch Pope\index{Personen!Pope, Blanch}, eine Ranterfrau\index{Ranter}, 
zu besuchen, angeblich um sie zu
bekehren; aber ehe sie sie wieder verließen, waren sie so verstrickt
in ihre Ansichten, das sie fast im Begriffe schienen, eher ihre Anhänger 
zu werden, besonders Joseph Hellen. Sie hatte sie unter
% \picinclude{./170-179/p_s171.jpg} 
andrem gefragt: "`Wer machte den Teufel? war es nicht Gott?"'\index{Teufel}
Diese einfältige Frage verblüffte die Beiden so, das sie nicht 
antworten konnten. Sie legten mir nachher die Frage vor, ich 
verneinte sie, "`denn"` sagte ich, "`alles was Gott machte, war gut,
und der Teufel ist nicht gut; er hieß Schlange, ehe er Teufel und
Feind hieß, und danach wurde er Teufel genannt. Später
wurde er Drache genannt, weil er ein Zerstörer war. Der Teufel
blieb nicht in der Wahrheit (Joh. 8,44) und als er die Wahrheit
verließ, wurde er der Teufel. Von den Juden hieß es, als sie
die Wahrheit verließen, sie seien vom Teufel, und man nannte
sie Schlangen (Matth. 23)\index{Bibel!Matth. 23}. Für den Teufel 
gibt es keine Verheißung, 
das er je wieder zur Wahrheit zurückkehren werde, aber
für die Menschen, die von ihm verführt werden, steht die Verheißung, 
das der Samen des Weibes der Schlange den Kopf zertreten 
und ihre Macht zertrümmern werde (1. Mos. 3)\index{Bibel!1. Mos. 3}. Nachdem
diese Fragen ausführlich zur Beruhigung der Freunde erörtert
worden waren, sahen sich die Beiden, die den Geist der Runtersfrau 
hatten aufkommen lassen, von der Wahrheit gerichtet; der
eine, Joseph Heilen, 
wandte sich ganz von uns ab und die Freunde
erkannten ihn nicht mehr als zu ihnen gehörend; der andre dagegen, 
George Bewley, wurde wieder zurückgewonnen und wurde
später recht brauchbar [...]

Ich hörte von einem Oberst Robinson in Cornwall, einem
bösen Menschen, der bei der Rückkehr des Königs zum 
Friedensrichter gemacht worden war, das er die Freunde grausam 
verfolge und viele von ihnen ins Gefängnis getan habe; als er hörte,
das ihnen durch die Gunst des Kerkermeisters einige kleine 
Freiheiten zugestanden wurden und sie ausgehen durften, um Weib
und Kinder zu sehen, erhob er deswegen beim Gericht eine Anklage 
gegen den Kerkermeister, und dieser musste eine Buße von
20 Pfund bezahlen, und die Freunde wurden einige Zeit sehr knapp
gehalten. Nach der Gerichtssitzung schickte dann Robinson zu
einem benachbarten Friedensrichter und ließ ihm sagen, er solle
ihm helfen, auf diese Fanatiker Jagd zu machen. An dem Tage,
als sie nun ihr Vorhaben ausführen wollten, schickte, er feinen
Knecht mit den Pferden Voraus und ging zu Fuß von seiner
Wohnung nach einer Farm, auf der er seine Kühe und seine
Milchwirtschaft hatte und wo seine Knechte und Mägde gerade
am Melken waren. Als er kam, fragte er nach dem Stier; die
% \picinclude{./170-179/p_s172.jpg} 
Mägde sagten, sie hätten ihn auf dem Felde eingesperrt, weil er
störrig sei bei den Kühen und sie am Melken hindere. Da ging
er ins Feld und begann nach seiner Gewohnheit seinen Stock gegen
den Stier zu schwingen, der Stier schnaubte nach ihm und holte
nach rückwärts aus, dann kehrte er sich und rannte wütend auf
ihn los und bohrte ihm die Hörner in die Seite, nahm ihn auf
die Hörner, schleuderte ihn über sich hinweg und riss ihm die Seite
auf bis zum Bauch, dann wühlte er mit den Hörnern im Boden
und brüllte und leckte seines Herrn Blut aus. Als eine der
Mägde den Herrn schreien hörte, rannte sie ins Feld, packte den
Stier bei den Hörnern und riss ihn von ihrem Meister weg.
Der Stier stieß sie ganz sanft mit seinen Hörnern zur Seite, ohne
ihr weh zu tun, und ließ nicht ab, sein Opfer zu durchstechen und
sein Blut auszulecken. Nun rannte sie davon und holte ein paar
Männer, die in einiger Entfernung arbeiteten, um ihrem Meister
zu helfen. Aber es gelang ihnen erst den Stier weg zubringen, als
sie die Kettenhunde auf ihn hetzten, da rannte er wutschnaubend
davon. Als die Schwester Robinsons hörte, was geschehen, kam
sie heraus und sagte: "`Ach, Bruder, welch schweres Gericht hat
dich betroffen!"' Er antwortete: "`Ja wahrlich ein schweres 
Gericht! lass den Stier töten und sein Fleisch den Armen geben."'
Sie brachten ihn nach Hause, aber er starb bald darauf. Der
Stier war so wild geworden, das sie ihn erschießen mussten, denn
niemand konnte sich ihm nähern, um ihn zu töten. So gibt der
Herr oft Beweise seines gerechten Gerichts über die Verfolger
seines Volkes, auf das man sich fürchte und sich in acht nehme. . .

Ich kam nach Swarthmore\index{Swarthmore}, wo man mir sagte, Oberst Kirby
habe seine Leute geschickt, um mich festzunehmen. Während der
Nacht, als ich in meinem Bett lag, trieb mich der Herr, am
nächsten Tage nach Kirbyhall zu Oberst 
Kirby\index{Personen!Oberst Kirby} zu gehen, fast zwei
Stunden weit, um mit ihm zu reden; ich ging denn auch [...]
und sagte ihm, ich hätte gehört, er wolle etwas von mir, ob er
irgend etwas gegen mich habe? Er sagte vor allen Anwesenden,
das er ein Gentleman sei und darum nichts gegen mich habe,
hingegen solle Mistres Fell keine Versammlungen in ihrem Hause
haben, das sei gegen die Verordnungen. Ich erklärte ihm, diese
Verordnungen treffen nicht uns, sondern die, welche sich 
versammeln, um Komplotte und Verschwörungen zu machen; [...]
die, welche sich bei Margaret Fell versammelten, seien friedliche
% \picinclude{./170-179/p_s173.jpg} 
Leute. Nachdem wir längere Zeit miteinander geredet, gab er
mir die Hand und wiederholte, das er nichts gegen mich habe.
So kehrte ich nach Swarthmore zurück [...] Bald darauf
ging Oberst Kirby nach London in eine Privatsitzung der Richter
in Holkerhall\indexname{Holkerhall}, und dort wurde ein 
Verhaftbefehl gegen mich aufgesetzt [...] Ich hörte davon 
und hätte gut entwischen können, [...]
aber da das Gerücht ging von einer Verschwörung, so fürchtete
ich, sie würden, wenn ich mich davon machte, über die Freunde
herfallen, wenn ich aber bleibe, so würden sie mich nehmen, und
die Freunde könnten sich eher davon machen, und ich blieb also [...]
Am folgenden Tage kam ein Beamter mit Pistole und Schwert.
Ich sagte ihm, ich wisse, warum er komme, und sei dageblieben,
um mich festnehmen zu lassen; [...] ich verlangte, das er mir den
Befehl zeige, aber er weigerte sich. So ging ich mit ihm, und
Margaret Fell\index{Personen!Fell, Margaret} begleitete uns nach 
Holkerhall [...] Dort wurde
mir unter anderem der Suprematseid vorgelegt; als ich ihn nicht
schwören wollte, verlangten einige, das ich ins Gefängnis von
Lancaster geschickt werde, andere wollten nur, das ich verspreche
an der Gerichtssitzung zu erscheinen, worauf ich entlassen wurde,
und ich kehrte also wieder mit Margaret Fell nach Swarthmore
zurück.

Am Gerichtstage ging ich wie verabredet war, nach Lancaster [...]
Der alte Richter Rawlinfon, der Vorsitzende, fragte mich, ob ich
um die Verschwörung wisse? Ich sagte, ich habe in Yorkshire
davon gehört. Er fragte mich, ob ich es den Behörden angezeigt? 
Ich erwiderte, ich hätte ja Schriften gegen Verschwörungen
geschrieben [...] Sie legten mir den Supremats- und Huldigungseid 
vor; ich sagte ihnen, das ich nicht schwören könne, weil
Christus und seine Apostel es verboten hätten, und sie hätten ja
schon genugsam erfahren, wie es bei solchen gehe, welche schwören,
ich aber habe noch nie in meinem Leben einen Eid geleistet. Hierauf
fragte mich Rawlinson, ob ich es für gesetzwidrig halte, zu
schwören? Diese Frage stellte er absichtlich, um mich zu fangen;
denn es war eine Verordnung gemacht worden, das alle, die
sagen, es sei gesetzwidrig zu schwören, verbannt oder hart bestraft
würden. Aber weil ich die Falle merkte, vermied ich sie und erklärte 
ihm, das in den Tagen des Gesetzes, bevor Christus gekommen sei, 
das Gesetz den Juden geboten habe, zu schwören
(3. Mos. 19)\index{Bibel!3. Mos. 19}; Christus aber, der in den 
Tagen des Evangeliums
% \picinclude{./170-179/p_s174.jpg} 
das Gesetz erfüllte, befehle, überhaupt nicht zu schwören 
(Matth. 5)\index{Bibel!Matth. 5},
und der Apostel Jakobus verbiete das Schwören selbst denen, die
Juden waren und das Gesetz Gottes hatten. Nach vielem Hin und 
Herreden riefen sie den Gefangenwärter und verurteilten mich
zum Gefängnis. Ich trug die Schrift bei mir, die ich gegen
Verschwörungen geschrieben hatte, und bat, das man sie vor dem
ganzen Gerichtshofe vorlese oder lesen lasse, aber sie wollten nicht.
Als ich nun solchermaßen eingesperrt war, dafür, das ich mich
geweigert hatte zu schwören, war mir daran gelegen, das sie und
alle Leute wissen möchten, das ich um der Lehre Christi willen
leide und darum, das ich seine Gebote gehalten. Ich hörte später,
das die Richter sagten, sie hätten besondere Befehle vom Oberst
Kirby gehabt, mich zu verfolgen, trotz seinem schönen Benehmen
und seiner anscheinenden Freundlichkeit damals, als er vor allen
Anwesenden erklärt hatte, er habe nichts gegen mich [...]

Ich wurde bis zur Gerichtsverhandlung gefangen gehalten, und
da Richter Turner und Richter Twistden gerade an der Reihe waren,
wurde ich vor Richter Twistden gebracht, am 14. Tage desk Monats, 
den man März nennt, im Jahre 1663\index{Jahr!1663}. A1s ich 
vorgeführt wurde, sagte ich: "`Friede sei mit euch allen"'. Der
Richter sah mich an und fragte: "`Warum kommst du hier vor
Gericht mit dem Hut aus dem Kopf?"' A1s der Kerkermeister mir
ihn hierauf weg nahm, sagte ich: "`Das Hutabnehmen ist doch nicht
eine Ehre, die vor Gott gilt!"' Daraus fragte mich der Richter:
"`Wollet ihr den Huldigungseid leisten, George Fox?"' Ich erwiderte: 
"`Ich habe nie in meinem Leben einen Eid geleistet, noch
mich zu irgend einem Vertrag verpfiichtet"'; darauf fragte er:
"`Wollt ihr schwören oder nicht?"' Ich erwiderte: "`Ich bin ein
Christ, und Christus befiehlt, nicht zu schwören, ebenso der Apostel
Jakobuß, und ob ich Gott oder Menschen gehorchen soll, darüber
urteile du selbst"'. Er sagte: "`Ich frage euch nochmals, ob ihr
schwören wollt oder nicht?"' Ich antwortete abermals: "`Ich bin
weder Türke, noch Jude, noch Heide, sondern ein Christ und
werde mich zum Christentum bekennen"'. Und darauf fragte ich
ihn, ob er nicht wisse, das die Christen der ersten Zeiten unter
den 10 Verfolgungen, sowie auch einige Märtyrer in den Tagen
der Königin Maria sich weigerten zu schwören, weil Christus
und die Apostel es verboten hätten; ferner sagte ich ihm, sie
hätten ja genügsam die Erfahrung gemacht, wie viele zuerst dem
% \picinclude{./170-179/p_s175.jpg} 
König geschworen hatten und nachher gegen ihn; was mich betreffe, 
so habe ich nie in meinem Leben einen Eid geleistet, und
meine Huldigung bestehe nicht im Leisten eines Eides, sondern
darin, das ich Wahrheit und Treue halte,. denn, sagte ich, ich
ehre jedermann, wie vielmehr denn den König. Christus aber,
der große Prophet und der König aller Könige und Heiland der
Welt, der große Richter der ganzen Erde, hat gesagt, das man
nicht schwören soll, soll ich nun Christus oder dir gehorchen?
Denn es geschiehet aus Gewissenszartheit und aus Gehorsam gegen
Christi Gebote, das ich nicht schwöre, und wir haben ja ein
Königswort für zarte Gewissen. Daraus fragte ich den
Richter, ob er den König anerkenne: "`Ja"', sagt er, "`ich 
anerkenne den König"'. "`Warum"', fragte ich, "`befolgst du denn dann
nicht seinen Erlass von Breda\index{Erlass von Breda} und 
seine Versprechen, die er bei
seiner Rückkehr machte, das niemand um der Religion willen 
vefolgt werde, solange er ruhig lebe? Wenn du den König anerkennst, 
warum verfolgst du mich, verhöhnst mich und treibst mich
dazu, einen Eid zu leisten, maß doch Sache des Glaubens ist, und
siehst doch, das weder du noch sonst jemand mich eines unfriedlichen 
Lebens zeihen kann"'. Hierauf wurde er sehr gereizt und
sagte: "`Kerl, wollt ihr schwören!"' Ich sagte darauf, ich sei keiner
seiner "`Kerls"', sondern ein Christ und es stehe einem alten Richter nicht
an, hier zu sitzen und den Gefangenen Spottnamen zu geben,
weder seinen grauen Haaren noch seinem Amt. Darauf sagte er:
"`Ich bin auch ein Christ"'. "`So handle auch christlich"', sagte ich;
"`Kerl"', sagte er, "`willst du mir mit deinen Reden Angst machen?
Aber"', fügte er Verlegen hinzu, "`jetzt brauche ich ja dieses Wort
wieder!"' und er bezwang sich. Ich sagte: "`Ich rede in Liebe so
mit dir, weil eine solche Sprache dir als; Richter nicht ansteht.
Du solltest deinem Gefangenen das Gesetz erklären, wenn er unwissend 
ist und einen verkehrten Weg geht"'. "`Ich rede ebenfalls in Liebe 
mit dir"', sagte er: "`aber"', erwiderte ich, "`die Liebe gebraucht 
keine Spottnamen"'. Darauf erhob er sich und sagte: "`Ich
lasse mich nicht von dir einschüchten, du sprichst so laut, deine
Stimme übertäubt die meinige und alle andern, ich müsste drei
oder vier Ausrufer kommen lassen, um dich zu übertönen, du hast
gute Lungen"'. Ich erwiderte: "`Ich bin hier gefangen um Jesu
Willen, um seinetwillen leide ich und stehe ich heute hier, und
wenn meine Stimme fünfmal so laut wäre, so würde ich sie erheben 
% \picinclude{./170-179/p_s176.jpg} 
und erschallen lassen für Christus, für dessen Sache ich
heute vor dem Richtstuhl stehe im Gehorsam gegen Christus,
welcher gebietet, nicht zu schwören, vor dessen Richtstuhl ihr alle
stehen und Rechenschaft ablegen müsst"'. "`So antworte mir nun
George For"', sagte er, "`ob du den Eid leisten willst oder nicht"'.
Ich erwiderte: "`Ich frage dich nochmals, ob ich Gott oder den
Menschen gehorchen soll? beurteile du das selber. Wenn ich
überhaupt einen Eid leisten wollte, so wäre es dieser; aber ich
leugne überhaupt alle Eide, nicht nur den oder jenen, nach der
Lehre Christi, der seinen Nachfolgern gebot, überhaupt nicht zu
schwören. Wenn nun du oder sonst jemand von euch, oder eure
Prediger oder Priester mir beweisen wollen, das Christus oder
seine Apostel irgend einmal, nachdem sie alles Schwören verboten
hatten, es den Christen wieder geboten, so will ich schwören"'.
Ich sah, das verschiedene Priester zugegen waren, aber nicht ein
einziger wollte reden. "`Nun denn"', sagte der Richter, "`ich bin
ein Diener des Königs und der König hat mich nicht geschickt,
um mit dir zu disputieren, sondern das Gesetz an dir auszuüben;
legt ihm also den Huldigungseid vor"'. "`Wenn du den König
lieb hast"', sagte ich, "`warum hälst du dich nicht an das, was er
sagt? und an seine Erklärung, in der er uns Gewissensfreiheit 
zugesagt hatte? Ich bin ein Mann mit einem zarten Gewissen und
kann aus Gehorsam gegen Christi Gebot nicht schwören"'. "`Wenn
er also nicht schwören will"', sagte der Richter, "`so führet ihn
in den Kerker"'. Ich sagte, es sei um Christi willen, das ich
nicht schwören könne, ihm müsse ich gehorchen; aber der Herr
möge ihnen allen vergeben. So führte mich der Kerkermeister
hinweg, aber ich fühlte, das der Herrn mächtige Kraft über ihnen
allen war [...]

Während ich nun hier im Kerker war, trieb es mich, an
Richter Flemming, einen der heftigsten Verfolger der Freunde,
folgendermaßen zu schreiben: 


\grosszitat{
  O, Richter Flemming! 
  \medskip 

  Barmherzigkeit, 
  Milde und Güte zieret die Menschen und auch die Behörden.
  O, hörest du nicht das Schreien derer, die durch die Verfolgungen
  Witwen und Waisen geworden sind? Sind sie nicht wie
  Schafe von Konstabler zu Konstabler getrieben worden, wie wenn
  sie die größten Übeltäter und Bösewichter im Lande wären? Es
  betrübt die Herzen vieler einsichtiger Leute, zu sehen, wie man
  ihre ehrlichen Mitmenschen, die ein friedsames, stilles Leben geführt,
  % \picinclude{./170-179/p_s177.jpg} 
  behandelt hat. Wieder ist einer gestorben, den ihr ins Gefängnis 
  geworfen; er hat fünf Kinder hinterlassen, die nun verwaist 
  sind. Solltest du nun nicht für diese vaterlosen Kinder
  sorgen, sowie auch für die Weiber und Hinterlassenen der andern?
  Ist es nicht deine Pflicht? Denke an Hiob, Kap. 29\index{Bibel!Hiob. 29}:
  "`Er war ein Vater der Armen; er errettete den Armen, der da schrie und
  die Waisen, die keinen Helfer hatten, er brach die Kinnbacken des
  Ungerechten und riss den Raub aus seinen Zähnen."' Und nun
  vergleiche dein Leben mit dem seinen und hüte dich vor dem Tage
  des Gerichts, welcher kommen wird, und vor dem Urteil Christi,
  wenn ein jeder muss Rechenschaft ablegen und den Lohn empfangen
  für seine Taten. Als dann wird es heißen; O, wo sind die verlorenen 
  Tage! — Als John Stubbs\index{Personen!Stubbs, John} vor dich gebracht wurde, der
  ein Weib und vier kleine Kinder hatte, und mit seiner Hände
  Arbeit nur den dürftigsten Lebensunterhalt verdiente, da riesest
  du: fordert diesem Menschen den Eid ab! und als er dir vorstellte, 
  dass er ein armer Mann sei, ließest du kein Mitleid aufkommen 
  und wolltest ihn nicht hören, und nun ist er im Gefängnis, 
  weil er nicht schwören konnte, also nicht das Gebot
  Christi und der Apostel übertreten konnte. Hoffentlich wirst du für
  seine Familie sorgen, damit seine Kinder nicht Hungers sterben.
  Ist denn das dem König gehuldigt, wenn man tut, wovon Christus
  und die Apostel sagen, es sei Unrecht und führe in die Verdammnis? 
  Ihr würdet wohl auch Christus und die Apostel, welche
  das Schwören verboten, ins Gefängnis geworfen haben, wenn sie
  zu eurer Zeit gelebt hätten.

  Denke auch an deinen armen Mitmenschen William Wilson\index{Personen!Wilson, William},
  der allgemein als ein fleißiger Mann bekannt war, und der sein
  Weib und seine Kinder ehrlich durchbrachte, obgleich er nichts besaß, 
  als was er durch seiner Hände Arbeit erwarb. Sogar auf
  den Märkten wird über den Tod dieser Beiden geredet; man hört
  das Schreien derer, die um der Gerechtigkeit willen Witwen und
  Waisen geworden sind. Wenn John Stubbs und William Wilson 
  geschworen hätten, so hätten sie damit ihre Freiheit wieder
  erlangt, wenn sie auch daneben es mit den Marktschreiern und
  Schnurranten gehalten hätten. O gehet in euch! es ist solches
  nicht nach des Herrn Sinn. Und auch der König hat erklärt, es
  solle gegen keinen seiner Untertanen, der friedlich lebe, eine 
  Grausamkeit ausgeübt werden. Sodann sind einigen sehr rechtschaffenen
  % \picinclude{./170-179/p_s178.jpg} 
  Leuten Bußen auferlegt worden, obgleich sie selber nichts hatten,
  und es eher am Platze gewesen wäre, ihnen etwas zu geben, als
  ihnen noch etwas zu nehmen. Weil du weißt, das sie um ihrer zarten
  Gewissen willen keinen Eid schwören können, so stellst du ihnen damit 
  eine Falle. Wie denkst du, das das Volk über ein derartiges
  Tun redet? Sie sagen: Wir wissen, das die Quäker sich an ihr
  \textit{ja} und \textit{nein} halten, andere dagegen sehen 
  wir schwören und wieder abschwören! 
  Ich weise dich an den Geist Gottes in deinem Gewissen, 
  Richter Fleming, der du so eifrig die Gefangennahme
  des George Fox betriebest und so böse warst über die, die ihn
  nicht gefangen nahmen. Wo ist dein Erbarmen mit den armen,
  verwaisten Kindern? Hüte dich vor der Grausamkeit des Herodes\index{Personen!Herodes},
  der kein Mitleid kannte; Esau hat es also gemacht und nicht
  Jakob! Thomas Walters\index{Personen!Walters, Thomas} 
  von Bolton ist auch hier im Gefängnis 
  und wird darin festgehalten, weil er sich nach Christi Gebot
  weigert zu schwören, und dabei hat er fünf kleine Kinder und seine
  Frau ist ihrer Niederkunft nahe; du solltest dich doch seiner 
  annehmen und dafür sorgen, das seine Frau und seine Kinder nicht
  Mangel leiden, da sie durch deine Schuld verwaist dastehen.
  Klingt dir das Schreien der Verwaisten nicht in den Ohren, und
  siehest du das Blut derer, die durch dich umgekommen sind, nicht
  vor dir? Es wird dich am Tage des Gerichts ein schweres Urteil 
  treffen, wie willst du dich verantworten, wenn du nach deinen
  Werken gerichtet werden wirst und vor den Richterstuhl des
  Allmächtigen treten musst? [...] Aber trotz alledem sagen wir
  Quäker: der Herr vergebe dir und rechne dir diese Dinge nicht
  an, wenn es sein heiliger Wille ist.
  \medskip 
  \begin{flushright}G. F.\end{flushright}
}

Bald danach starb Richter Flemings Weib, und hinterließ
ihm dreizehn oder vierzehn mutterlose Kinder [...]
Einige Zeit vorher war Margaret Fell\index{Personen!Fell, Margaret} auch von Richter
Fleming als Gefangene nach Lancaster geschickt worden, und als
sie, an der Gerichtssitzung, den Eid nicht schwören wollte, wurde
sie weiter zum Gefängnis verurteilt. [...]
Während ich im Gefängnis zu Lancaster war, hieß es, der
Türke werde über die Christenheit herfallen, und viele kamen in
große Angst. Eines Tages, als ich in meiner Zelle auf und
nieder ging, kam es über mich vom Herrn, das ich sah, wie die
Kraft des Herrn sich gegen den Türken kehrte, so das er wieder
umkehren musste, und ich teilte einigen mit, was der Herr mich
% \picinclude{./170-179/p_s179.jpg} 
hatte sehen lassen, und binnen eines Monats kam die Nachricht,
das er geschlagen worden war!\footnote{1664 Sieg der 
Abendländer (Deutschland und Frankreich) über die
Türken bei der Abtei St. Gotthardt.}\index{Jahr!1664}\index{Abtei St. Gotthardt}

Ein andermal als ich in meiner Zelle auf und niederging
und zum Herrn aufschaute, sah ich den Engel des Herrn, wie er
mit einem leuchtenden Schwert gen Süden wies, und das ganze
Schloss schien in Feuer zu stehen. Nicht lange darauf brach der
Krieg in Holland aus,\footnote{1665 Krieg zwischen England 
und den Niederlanden.}\index{Jahr!1665} und dann eine große Seuche und dann das
Feuer in London;\footnote{Oktober 1665 die große Pest in London, 
September 1666 der große Brand in London.} da war wahrlich 
das Schwert des Herrn gezogen.

Durch die lange Gefangenschaft an diesem ungesunden Orte
war ich sehr angegriffen in meiner Gesundheit, aber die Kraft
des Herrn war stärker als alles, sie half mir hindurch und hielt
mich aufrecht und half mir für den Herrn wirken, so viel der
Ort es erlaubte. Ich antwortete denn auch während dieser Zeit
auf mehrere Bücher, wie: "`die Messe"', "`das Common Prayer
Buch"', "`daß Directory"', "`das Kirchenbekenntnis"', welches die
vier mächtigsten Religionen\footnote{Römische, Bischöfliche, 
Preßbyterianer und Independenten.} sind, die sich seit den Tagen der
Apostel erhoben.

Nach der Gerichtsverhandlung war es einigen der Richter
etwas ungemütlich, das ich in Lancaster war, denn ich hatte sie
bei den Verhandlungen tüchtig geärgert, und sie bemühten sich
sehr darum, das man mich anderswohin bringe [...] Etwa sechs
Wochen nach der Gerichtsverhandlung erhielten sie denn auch den
Befehl vom König und dem Rat, mich von Lancaster fortzubringen,
und zugleich kam ein Brief vom Earl von 
Anglesea\index{Personen!Earl von Anglesea}, worin es
hieß, das, wenn alles dessen man mich beschuldigt hatte, wahr
sei, so verdiene ich keinerlei Nachsicht noch Milde. Und doch war
das Ärgste, was sie gegen mich vorgebracht hatten das, das ich
einem Gebot Christi nicht ungehorsam sein konnte [...]

Sie brachten mich nun nach Schloss Scarbro, wo sie mich in
ein Gemach führten und mir einen zur Wache setzten. Da ich seht
schwach war und öfters ohnmächtig wurde, so ließen sie mich
manchmal mit der Wache an die frische Luft gehen; nach einiger
Zeit brachten sie mich in ein anderes Eelaß\footnote{Das Wort 
ist i Original nicht zu entziffern}, das offen war, so