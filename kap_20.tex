
%%%%%%%%%%%%%%%%%%% Kapitel 20. %%%%%%%%%%%%%%%%%%%%%%%%%%%%%%

\chapter[Penn, Frauenversammlungen und Grundsätze]{William Penn, Frauenversammlungen und  über die 
Grundsätze der Quäker.}

\begin{center}
\textbf{Ankunft in Bristol. Zusammentreffen mit William Penn und
anderen. Verteidigung der Frauenversammlungen. Vorgeahnte
Gefangenschaft in Worcester. Brief an den König über die 
Grundsätze der Quäker. Krankheit. Befreiung. Während der
Gefangenschaft verfasste Schriften.}
\end{center}



Als wir ans Ufer kamen, begaben wir uns nach Shirehampton\ort{Shirehampton} [...] 
und ritten von da nach Bristol\ort{Bristol} [...]. Am Abend
schrieb ich von hier einen Brief an meine Frau:

\brief{Fell, Margaret}{
Liebes Herz!

\bigskip 

Heute Abend landeten wir in Bristol, Gott dem Herrn sei
Lob immerdar, der unser Schutzherr war und unser Schiff lenkte,
dem Gott der ganzen Erde, der Meere und Winde, der die Wolken
zu seinen Wagen macht [...] Robert Widders\person{Widders, Robert} und James
Lancaster\person{Lancaster, James} sind mit mir und sind gesund. Dem Herrn sei Lob
und Ehre, der uns durch so manche Gefahr geholfen, auf dem
Wasser, in den Stürmen, vor Seeräubern und andern Räubern,
in den Gefahren der Wildnis und unter den falschen Frommen.
Sein Ruhm ist über alles, Amen. Darum werdet das neue
Leben inne, und lebet ganz dem Herrn in demselben. Ich möchte,
so der Herr will, eine Zeit lang hier bleiben, vielleicht bis zum
Jahrmarkt. Genug diesmal. Meine Liebe allen Freunden.

\bigskip 

\begin{flushright}Bristol, 28. des 4. Monats 1673\index{Jahr!1673}. G. F.\end{flushright}
}

Zwischen diesem Tage und dem Jahrmarkt kam meine Frau
aus dem Norden zu mir nach Bristol, und ihr Schwiegersohn,
Thomas Lower\person{Lower, Thomas} und zwei ihrer Töchter kamen mit ihr. Ihr
anderer Schwiegersohn, John Raus\person{Raus, John}, William Penn\person{Penn, William}\footnote{
William Penn, Sohn des großen Admiral Penn, zeigte schon während
seiner Studienzeit in Christchurch College\index{Christchurch College} 
in Oxford ernst religiöse Bedürfnisse.
Als er nach Hause zurückgekehrt, sich den herrschenden Hofsitten nicht fügen wollte,
sagte sein Vater sich von ihm los. Er schloss sich nun den Quäkern an und
wurde ihr bedeutendstes Mitglied. Aus einem ihm von Karl II. überlassenen
und nach ihm Pennsylwanien genannten Landstrich am Delaware\ort{Delaware} gründete er
eine Freistatt für seine verfolgten Glaubensgenossen. (Näheres siehe Weingarten
a.a.O.S. 408.)
} und seine
% \picinclude{./230-239/p_s236.jpg} 
Frau und Gerrard Roberts\person{Roberts, Gerrard} kamen von London; und noch viele
andere Freunde aus verschiedenen Gegenden des Landes kamen
zum Jahrmarkt\index{Jahrmarkt}, und wir hatten herrliche Versammlungen, in
denen des Herrn Kraft über allen war. Nachdem ich meine
Arbeit für den Herrn in dieser Stadt getan, ging ich nach 
Gloucestershire\ort{Gloucestershire}, [...] und von da nach Wiltshire. 
In Slattenford in Wiltshire\ort{Wiltshire} hatten wir eine schöne Versammlung, obgleich wir
manchen Widerstand erfuhren von solchen, die sich den 
Frauenversammlungen\index{Gleichberechtigung} widersetzten; der Herr 
trieb mich, dieselben den
Freunden anzuempfehlen\index{Konflikt} zu Nutz und Frommen der Kirche Christi:
\zitat{Gläubige Frauen, welche zum Glauben an die Wahrheit berufen 
sind und desselben köstlichen Glaubens teilhaftig gemacht
sind, wie die Männer, und Miterben desselben ewigen Evangeliums 
des Lebens und Heils, sollen gleicherweise in den Besitz
und Stand der Ordnungen des Evangeliums kommen und somit
Mitgehilfen der Männer werden bei den Neugestaltungen im
Dienst der Wahrheit in den Angelegenheiten der Kirche, wie sie
es in den zeitlichen Dingen des täglichen Lebens sind, damit
die ganze Hausgemeinde Gottes, sowohl, Männer wie Frauen,
ihre Pflicht und Aufgabe im Hausstand Gottes kennen, einsehen
und ausüben möchten, damit besser für die Armen gesorgt werde,
die Jungen unterrichtet und in Gottes Wegen unterwiesen werden,
die Wankelmütigen und Liederlichen zurechtgewiesen und getadelt
werden in der Furcht Gottes, die Unbescholtenheit solcher, die sich
verheiraten wollen, genauer und strenger untersucht werde in der
Weisheit Gottes, und alle Glieder des geistigen Leibes, der Kirche,
einander bewachen und helfen in der Liebe.} Aber nachdem diese
Gegner sich lange in Hader und Zank ereifert hatten, warf der
Herr einen ihrer Führer darnieder, so das er sich demütigte und
das Unrecht einsah, das er tat, als er sich Gottes himmlischer
Macht widersetzte, er gestand den Freunden seinen Irrtum ein
und veröffentlichte später ein Schreiben, in welchem er erklärte,
er hätte sich eigensinnig widersetzt, trotz meiner häufigen 
Warnungen, bis das Feuer des Herrn in ihm entbrannt sei, und er
den Engel des Herrn mit gezogenem Schwert gesehen habe, im
% \picinclude{./230-239/p_s237.jpg} 
Begriff ihn zu töten [...]. Wir besuchten viele Freunde und
kamen schließlich nach Kingston\ort{Kingston}, wo ich mit 
meiner Frau\person{Fell, Margaret} und
ihrer Tochter Rachel\person{Fell, Rachel} zusammen traf. 
Ich hielt mich nicht lange
hier auf, sondern ging nach London, wo die Baptisten\index{Baptisten} mit einigen
Sozinianern\index{Sozinianern} sehr bösartig geworden waren und Viele Bücher gegen
uns gedruckt hatten, und ich hatte viele Arbeit in der Kraft des
Herrn, ehe ich aus dieser Stadt loskommen konnte [...].\index{Schmähschriften}

Darauf machte ich eine kurze Reise durch einige Gegenden
in Essex\ort{Essex} und kehrte nach London\ort{London} 
zurück, wohin ich mich innerlich
gezogen fühlte, denn ich hatte gehört, das man viele Freunde
vor die Richter gebracht hatte und etliche in London und an andern
Orten gefangen genommen hatte, weil sie ihre Schaufenster an
Festtagen\index{Festtagen} und an sogenannten 
Fasttagen\index{Fasttagen} geöffnet hatten, und
weil sie Zeugnis ablegten gegen alles Feiern solcher Tage. Die
Freunde mussten dies tun, weil sie ja wussten, das die wahren
Christen die Feste der Juden zur Zeit der Apostel nicht hielten,
und so konnten sie auch nicht die sogenannten Feste der Heiden
und Papisten\index{Papisten}, welche unter den sogenannten Christen seit den
Tagen der Apostel eingesetzt wurden, halten. Denn wir sind 
befreit worden vom Halten bestimmter Tage, durch Jesum Christum
und sind in den Tag gebracht worden, der aus der Höhe aufgegangen 
ist, und wir sind jetzt in Ihm, der Herr ist über den
jüdischen Sabbath\index{Jüdischen Sabbath}, und über das Wesen 
der jüdischen Zeichen [...

Bald darauf, als ich, in Adderbury\ort{Adderbury}, am Nachtessen saß, fühlte
ich, das ich gefangen genommen werden würde, ich sagte aber noch
niemand etwas [...]. Am andern Tage hatten wir eine Versammlung 
[...] Nach derselben saß ich mit einigen Freunden in einem
Zimmer im Gespräch, da kam Henry Parker\person{Parker, Henry}, 
ein Friedensrichter, ins Haus, und mit ihm Rowland 
Hains\person{Hains, Rowland}, ein Priester aus Hunniton
in Warwickshire\index{Warwickshire}, [...] und Henry Parker nahm mich gefangen und
Thomas Lower\person{Lower, Thomas} zur Gesellschaft 
mit, und obgleich er nichts gegen
uns Vorbringen konnte, schickte er uns beide ins Gefängnis von
Woreester\ort{Woreester}, durch einen merkwürdigen Verhaftbefehl [...]
Da ich derart zum Gefangenen gemacht worden war, ohne viel
Aussicht, vor der vierteljährlichen Gerichtssitzung frei zu werden,
veranlassten wir einige Freunde, meine Frau\person{Fell, Margaret} und ihre Töchter
nach dem Norden zu begleiten [...] Als ich dachte, das meine
Frau zu Hause angelangt sei, schrieb ich ihr aus der Gefangenschaft 
folgenden Brief:
% \picinclude{./230-239/p_s238.jpg} 


\brief{Fell, Margaret}{
  Liebes Herz!

  \bigskip 

  Du schienst ein wenig betrübt, als ich vom Gefängnis sprach
  und dann gefangen genommen wurde. Ergib dich in den Willen
  des Herrn. Denn als ich im Hause des John Rous in Kingston
  war, hatte ich ein Gesicht, das ich gefangen genommen würde,
  und als ich bei Bray Doily in Oxfordshire war, sah ich, während
  ich am Abendessen saß, das ich gefangen genommen und Leiden
  zu erdulden haben würde. Aber des Herrn Macht ist über allem, sein
  Name sei gepriesen ewiglich [...]

\bigskip 
\begin{flushright}G. F.\end{flushright}

}

Wir wurden erst am letzten Tage der Gerichtssitzung 
vorgenommen, am 21. des 11. Monats 1673\index{Jahr!1673} [...] Ich musste
Bericht über meine Reise geben. Friedensrichter Parker hatte,
um den Fall recht schwer scheinen zu lassen, eine große Geschichte
gemacht, es seien Leute von London, von Cornwall, aus dem
Norden und von Bristol im Hause gewesen, als man mich gefangen 
genommen habe; hieraus erklärte ich ihm, das das alles
gewissermaßen eine einzige Familie gewesen 
sei,\footnote{Der Konventikel-Akt\index{Konventikel-Akt} von 1664\index{Jahr!1664} sagt: 
\zitat{jede Prioatandacht von mehr
als fünf Personen außer der Familie, wobei nicht das Common-Prayer-Boot
zugrunde gelegt wird, wird mit dreimonalichem Gefängnis, zum dritten mal
mit Verbannung bestraft.}} indem niemand von
London dagewesen sei, als ich selber, niemand aus dem Norden als
meine Frau und meine Töchter, niemand von Cornwall als mein
Schwiegersohn.

Als ich fertig geredet, stand der Vorsitzende, Simpson, ein
einstiger Presbyterianer, auf und sagte: \zitat{Was Ihr da sagt, klingt
recht unschuldig}. Darauf flüsterten er und Parker eine Weile
miteinander und daraus stand er wieder auf und sagte: \zitat{M. Fox,
Ihr seid ein ausgezeichneter Mann und alles, was Ihr da sagt,
mag wahr sein, aber, um uns ganz zu befriedigen, wollt Ihr den
Huldigungseid leisten?}\index{Eid} Ich erwiderte ihm, sie hätten versprochen,
uns keine Falle zu stellen, dies sei aber einfach eine Falle, da sie
ja wissen, das wir nicht schwören können. Sie ließen dennoch
den Eid vorlesen. Ich erklärte ihnen daraus: \zitat{Ich habe nie in
meinem Leben einen Eid geleistet, aber ich bin immer der Regierung
gehorsam gewesen; ich war im Kerker von Derby sechs Monate,
weil ich die Waffen nicht gegen König Karl bei Woreester 
erheben wollte, und weil ich in die Versammlungen ging, wurde
ich nach Leicestershire gebracht, vor Oliver 
Cromwell\person{Cromwell, Oliver}, als einer der
% \picinclude{./230-239/p_s239.jpg} 
Mitverschworenen für die Rückkehr des Königs. Ihr wisst es ja
nach euren eigenen Gewissen, das wir, die ihr Quäker nennt
keinen Eid leisten können, weil Christus es verboten hat. Was
aber den Inhalt des Eides anbelangt, so kann ich sagen, und
sage es auch, das ich den König von England als den rechtmäßigen 
Erben und Nachfolger des englischen Reiches anerkenne\index{Monarchie}
und alle Verschwörer und Verschwörungen gegen ihn verabscheue,
und ich hege nur Liebe und Wohlwollen in meinem Herzen gegen
ihn und gegen alle Menschen und wünsche ihm und ihnen allen
nichts als Gutes. Der Herr weiß es, vor welchem ich als ein
unschuldiger Mann stehe. Was den Suprematseid anbelangt, so
verabscheue ich den Papst\index{Papst} und seine Macht und 
seine Religion\index{Katholizismus}
von ganzem Herzen.} [...] Aber sie ließen mich vom 
Gefangenwärter hinweg führen und brachen so ihr Versprechen dem Lande
gegenüber, denn sie hatten mir freies Reden zugestanden und mir
es doch nicht gewährt [...]

Während meiner Gefangenschaft kam es über mich, unsere
Anschauungen und Grundsätze dem König darzutun, nicht 
hauptsächlich um meiner eigenen Leiden willen, sondern damit er unsere
Anschauungen und unsere Gemeinschaft besser verstehe.

\brief{König}{
  An den König!

  \bigskip 

  Der Ausgangspunkt der Quäker ist der Geist von Christus,
  der für uns gestorben und um unsrer Gerechtigkeit willen auferweckt
  worden ist, durch welchen wir wissen, das wir sein sind. Er
  wohnet in uns mit seinem Geist und der Geist Christi macht uns
  frei von aller Ungerechtigkeit und Gottlosigkeit. Der Geist Christi
  machet, das wir allem gottlosen Wesen absagen, als da sind
  Lügen, Stehlen, Töten, Verbrechen, Hurerei und alle Arten von
  Unreinigkeit, Unzucht, Bosheit, Hass, Betrügerei, Schlemmen und
  alle Werke des Teufels.\index{Sündlosigkeit} Der Geist Christi führt uns dazu, für
  alle Menschen den Frieden und das Gute zu suchen und friedlich
  zu leben. Er machet, das wir uns aller Anschläge und 
  Verschwörungen gegen den König oder irgend sonst Jemand 
  enthalten. Er hält uns zurück von jenem bösen Tun und Treiben,
  gegen welches das Schwert der Obrigkeit sich richtet. Unser Wunsch
  und Bestreben ist, das alle, die Christus bekennen, auch im Geiste
  Christi wandeln, damit sie durch denselben des Fleisches Geschäfte
  töten möchten und mit dem Schwert des Geistes ihre Sünde und
  Bosheit ausrotten. Dann würden die Richter und Beamten nicht
  % \picinclude{./240-249/p_s240.jpg} 
  soviel damit zu tun haben, das Böse im Reich zu bestrafen, und
  die Könige und Fürsten brauchten keinen ihrer Untertanen zu
  fürchten, wenn alle im Geist Christi wandelten; denn die Früchte
  des Geistes sind Liebe, Gerechtigkeit, Gütigkeit und Mäßigkeit.
  Wenn alle, die sich als Anhänger Christi bekennen, auch in feinem
  Geiste wandeln und durch denselben Sünde und Bosheit in sich
  ertöten würden, so wäre dies eine große Erleichterung für die
  Obrigkeit und würde ihr viel Mühe ersparen, denn dann würden
  alle dazu geführt, andern zu tun, wie sie wollten, das man ihnen
  tue, und das Königliche Gesetz der Freiheit würde somit erfüllt [...]
  
  Wir können aus großer Gewissenhaftigkeit gegen die Gebote
  Christi und seiner Apostel nicht schwören, denn es wird uns 
  geboten, in Matth. 5\bibel{Matth. 5} und Jak. 5\bibel{Jak. 5} 
  bei ja und nein zu bleiben und überhaupt nicht zu schwören, 
  weder beim Himmel, noch bei der Erde,
  noch bei irgend sonst etwas, auf das wir nicht übles tun und in
  Verdammnis fallen. Christus sagte: \zitat{Ihr habt gehört, das zu
  den Alten gesagt ist, ihr sollt keinen falschen Eid tun und Gott
  euren Eid halten} (Math. 5). Es waren dies wahre und feierliche 
  Eide, und die, welche sie einst leisteten, hatten sie zu halten,
  aber Christus und die Apostel verbieten sie zur Zeit des 
  Evangeliums so gut wie die falschen und unnützen Eide. Wenn wir irgend
  einen Eid leisten könnten, so wäre es der Huldigungseid, weil
  wir wissen, das König Karl durch Gottes Macht nach England
  zurückkam und zum König von England gemacht und über unsere
  früheren Verfolger gesetzt wurde, und was die Oberherrschaft des
  Papstes anbelangt, so erkennen wir sie in keiner Weise an.\index{Papstum} Aber
  da Christus und seine Apostel uns geboten, nicht zu schwören,
  sondern bei ja und nein zu bleiben, so dürfen wir ihren Geboten
  nicht ungehorsam sein. Darum haben viele uns den Eid vorglegt 
  als Falle, damit wir ihnen zur Beute würden. Unsre
  Weigerung des Eides geschieht nicht aus Eigensinn und 
  Hartnäckigkeit oder Missachtung, sondern nur aus Gehorsam gegen
  die Gebote Christi und der Apostel, und wir sind bereit, wenn wir
  unser ja und nein brechen, die gleiche Strafe zu leiden, wie jemand
  der seinen Eid bricht. Wir bitten darum den König, solches zu
  bedenken, und wie lange wir schon leiden um dieser Sache willen.
  Dies ist von einem, der dem König allezeit Glück und alles
  Wohlergehen wünscht und allen seinen Untertanen, durch Jesus
  Christus.
  \bigskip 
  \begin{flushright}G. F.\end{flushright}

}
% \picinclude{./140-149/p_s140.jpg} 
% Hier ist ein Teil des Textet nach 230-239 verschoben worden
% wegen Formatierung.
% \picinclude{./240-249/p_s241.jpg} 
Um diese Zeit hatte ich einen Krankheitsanfall der mich sehr
angriff und schwächte, und der einige Zeit anhielt, so das einige
Freunde an meiner Wiederherstellung zweifelten. Es kam mir\index{Vision}
vor, ich wandle unter Gräbern und Leichen, aber die unsichtbare
Macht hielt mich innerlich aufrecht und gab mir erquickende Kraft,
selbst wenn ich so schwach war, das ich kaum sprechen konnte.
Einmal, als ich des Nachts auf meinem Bett lag in der Betrachtung 
der Herrlichkeit Gottes, die über alles ist, hörte ich
eine Stimme, das der Herr noch viel Arbeit an seinem Werke für
mich habe, ehe er mich zu sich nehmen könne.

Es wurden Maßregeln getroffen, um meine Freilassung zu
bewirken, wenigstens für so lange, bis ich mich erholt hätte, aber
es zeigte sich, das es schwierig war, sie zu erlangen, denn der
König wolle mich auf keinem andern Weg als dem der Begnadigung\index{Begnadigung}
freilassen, da man ihm gesagt habe, nach dem Gesetz
könne er es nicht tun, und durch Begnadigung wollte ich nicht
frei werden, denn das schien sich mir nicht mit meiner Unschuld
zu vereinbaren [...].

Nun ging meine Frau nach London und erzählte dem König
von meiner langen unverschuldeten Gefangenschaft und der Art
meiner Gefangennahme, wie die Richter mit mir verfahren waren,
indem sie mir den Eid als Falle vorlegten, worauf sie mich den
Strafen des Praemunire unterwarfen, und es stehe nun bei ihm,
ihren Wunsch zu erfüllen und mich frei zu sprechen. Der König
antwortete ihr freundlich und wies sie an den Kanzler; zu diesem
ging sie, konnte jedoch nicht erreichen, was sie wünschte, denn
dieser sagte, der König könne mich nicht anders als durch Begnadigung 
frei lassen, und ich hatte keine Freiheit, mich begnadigen
zu lassen, da ich wusste, das ich nichts Böses getan hatte. Wenn
ich mich hätte wollen begnadigen lassen, so hätte ich nicht solange
zu warten brauchen, denn der König war schon längst willens 
gewesen, mich zu begnadigen, und hatte zu 
Thomas Moore\person{Moore, Thomas} gesagt,
ich brauche mir keine Bedenken zu machen, mich begnadigen zu
lassen, denn es sei schon mancher, der so unschuldig gewesen wie
ein Kind, begnadigt worden, aber ich konnte mich nicht dazu 
verstehen. Lieber wollte ich mein ganzes Leben lang im Gefängnis
bleiben, als aus demselben befreit zu werden auf eine Art, die
irgendwie der Wahrheit zur Unehre gereichen konnte, und ich zog
es darum vor, das die Rechtmäßigkeit meiner Anklage vor Gericht
% \picinclude{./240-249/p_s242.jpg} 
geprüft werde\index{Revision} [...]. So wurde denn ein Befehl
nach Worcester geschickt, mich nochmals in Kings Bench zu
verhören, um meine Anklage zu prüfen. Ich reiste also am 4.
des 12. Monats ab, am 8. kamen wir in London an, und
am 11. wurde ich vor die vier Richter von Kings Bench\ort{Kings Bench} 
gebracht, wo Corbet meine Sache führte. Er brachte einen neuen
Verteidigungsgrund vor, nämlich man dürfe niemand wegen eines
Prämunire\index{Prämunire}\footnote{Prämunire der tarm. techn. einer 
bestimmten Art von Verbrechen, z. B. das Verweigern des Huldigungseides, 
das bestimmte Strafen wie Entziehung des Grundbesitzes, oft auch 
Gefängnis u. a. zur Folge hatte. Vgl. Stephen, Englisches Strafrecht.} 
einsperren. Daraus erwiderte Richter Hale\person{Richter Hale}: \zitat{Mr.
Corbet, diese Verteidigung hätten sie früher bringen sollen}, dieser
erwiderte: \zitat{Wir konnten keine Abschrift der Anklage bekommen.}
Der Richter sagte: \zitat{Das hätten sie uns sagen sollen, dann
hätten wir sie gezwungen, eine solche früher zu schicken.} [...]
Corbet blieb dabei, das man nicht wegen eines Praemunire
jemand gefangen nehmen könne [...] \zitat{Gut}, sagte einer der
Richter, \zitat{wir müssen Zeit haben in unsern Büchern nachzusehen,
und die Statuten nachzuprüfen}. Somit wurde das Verhör auf
den folgenden Tag Verschoben.

Am folgenden Tage ließen sie diesen Verteidigungsgrund
lieber fallen und fingen gleich mit den Irrtümern in der Anklage
an, und als dieselben eröffnet wurden, so waren ihrer so viele
und so große, das die Richter alle die Überzeugung hatten,
die Anklage sei leer und nichtig und man solle mich frei
lassen [...] Es waren an dem Tage mehrere angesehene Leute,
Lords und andere da, die den Huldigungseid ablegen mussten, und
einige meiner Gegner suchten die Richter zu bewegen, mir den
selben auch noch einmal vorzulegen, weil, wie sie sagten, es 
gefährlich sei, mich frei zu lassen. Aber Richter Hale sagte, er habe
allerdings schon dergleichen Gerüchte über mich gehört, aber noch
Viel mehr gute, und so erklärten er und die übrigen Richter mich
öffentlich frei [...].

Während meiner Gefangenschaft in Worcester hatte ich trotz
meiner häufigen Krankheit, und trotzdem ich so oft nach London
und wieder zurückgezerrt wurde, mehrere Bücher für den Druck
geschrieben. Eines derselben war betitelt: \index[buch]{Eine Warnung an
England}\index{Eine Warnung an England}. Ein anderes war: \zitat{An die 
Juden, um zu beweisen, das der Messias gekommen ist}\index{Juden}\index[buch]{An die 
Juden, um zu beweisen, das der Messias gekommen ist}.
% \picinclude{./240-249/p_s243.jpg} 
Ein anderes: \zitat{Von der Inspiration, Offenbarung und Weissagung}\index[buch]{Von 
der Inspiration, Offenbarung und Weissagung}. Ein anderes: 
\zitat{Gegen alles unnütze Disputieren.}\index[buch]{Gegen 
alles unnütze Disputieren} Ein anderes: \zitat{An alle Bischöfe und
Prediger, das sie sich nach der Schrift prüfen}\index[buch]{An alle Bischöfe und
Prediger, das sie sich nach der Schrift prüfen}. Ein anderes:
\zitat{An die, welche sagen, wir lieben nur uns selbst}\index[buch]{An 
die, welche sagen, wir lieben nur uns selbst}. Ein anderes
betitelt: \zitat{Unser Zeugnis von Christus}\index[buch]{Unser 
Zeugnis von Christus}. Und ein anderes kleines
Buch: \zitat{Vom Schwören}\index[buch]{Vom Schwören}, welches 
das erste war von den beiden,
die dem Parlament vorgelegt wurden. Außerdem schrieb ich noch
verschiedene Schriften und Briefe an Freunde, um sie zu 
ermutigen und zu stärken im Dienst für den Herrn. Denn etliche,
die sich als Bekenner der Wahrheit ausgegeben hatten, dann
aber einem Geist der Verführung Raum gegeben hatten und von
der Einigkeit und Bruderschaft des Evangeliums, in der die
Freunde stehen, abgewichen waren, hatten versucht, die Freunde
in ihrem Dienst zu entmutigen, besonders in der fleißigen und
wachsamen Fürsorge für die Ordnung der Angelegenheiten der
Kirche Christi [...]
