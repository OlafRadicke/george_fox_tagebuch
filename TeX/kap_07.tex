
% \picinclude{./080-089/p_s083.jpg} 

%%%%%%%%%%%%%%%%%%% Kapitel 7. %%%%%%%%%%%%%%%%%%%%%%%%%%%%%%

\chapter[Begegnung mit Oliver Cromwell]{Begegnung mit Oliver Cromwell}

\begin{center}
\textbf{Kämpfe mit schwärmerischen Ranters und 
zehntengierigen Priestern.
Fox in Wetstone verhaftet und vor Cromwell geschickt.}
\end{center}

\section{Quaker im geschäftlichen Umgang}

Die Priester und \textit{Frommen} traten aufs neue mit ihren
Prophezeiungen gegen uns auf. Schon lange hatten sie 
vorausgesagt, das wir binnen eines Monats vernichtet sein werden;
hernach verlängerten sie die Frist auf ein halbes Jahr; als aber
auch diese Zeit längst um war, und wir im Gegenteil an Zahl
zunahmen, streuten sie aus, wir werden einander gegenseitig 
verzehren. Es kam nämlich oft vor, das nach den Versammlungen
manche, die einen weiten Heimweg hatten, bei Freunden blieben,
es waren oft mehr Leute als Betten vorhanden, so das Viele auf
dem Heu übernachten mussten. Da wurden die \textit{Frommen} von
der Furcht Cains gepackt; sie hatten Angst, das, wenn wir 
einander zu Grunde gerichtet hätten, wir dann der Gemeinde zur
Last fallen und uns von ihr unterhalten lassen werden. Als sie
aber sahen, wie der Herr den Freunden Segen und Gedeihen
gab, wie dem Abraham, \zitat{beim Acker und beim Korb, beim 
Eingehen und beim Ausgehen, beim Aufstehen und beim Niederliegen}
(5. Mose 28\bibel{Mose 5. 28@5. Mose 28}), da erkannten sie 
die Ungerechtigkeit ihrer Prophezeiungen, und das man 
\zitat{umsonst flucht, wo der Herr segnet}
(4. Mose 23\bibel{Mose 4. 23@4. Mose 23}). Als nach den 
ersten Bekehrungen die Freundes
den Hut nicht vor den Leuten abnahmen, einer einzelnen Person
nicht mit \zitat{ihr}, sondern mit \zitat{du}\index{Anrede} und 
\zitat{dich} antworteten, sich nicht
verneigten und nicht bei der Begrüßung schmeichelhafte Worte
gebrauchten und nicht die Art und Weise der Welt mitmachten, da 
verloren viele von ihnen in ihren Geschäften die Kundschaft; man
scheute sich vor ihnen und wollte keine Geschäfte mit ihnen machen,
so das eine Zeit lang die Freunde kaum ihr Brot verdienten. Aber
als die Leute sahen, wie treu und ehrlich die Freunde waren,
und das ihr \zitat{ja — ja} und ihr \zitat{nein — nein} war; 
das sie Wort hielten \index{Zeugnis!Wahrhaftigkeit}
im Verkehr und niemanden hintergingen noch betrogen, und wie
der Herr ihnen Segen und Gedeihen gab; wie ein Kind, das sie
schickten, um einen Einkauf zu machen, gerade so gut bedient
wurde wie sie selbst, da predigte das Leben und der Wandel der
Freunde, und es traf das, was von Gott kam, in ihren Gewissen.
Nun wandelten sich die Dinge dermaßen, das man beständig
fragen hörte: \zitat{Wo ist ein Krämer, ein Tuchhändler, ein 
Schneider,
% \picinclude{./080-089/p_s084.jpg} 
ein Schuster, ein Handwerker, der Quäker ist?}
\index{Geschäftlicher Erfolg} Die Freunde
bekamen mehr Arbeit als manche andere Handwerker und 
beteiligten sich reger am geschäftlichen Verkehr. Nun schlugen die
gehässigen \textit{Frommen} einen anderen Ton an und fingen an zu
murren: \zitat{Wenn wir diese Quäker gewähren lassen, so werden sie
uns den Handel des ganzen Landes an sich reisen.} Also tat
der Herr an seinem Volke, und es ist mein ernstlichster Wunsch
das alle, die seine heilige Wahrheit bekennen, in der Erkenntnits
bewahrt und durch den Geist und die Kraft in der Treue erhalten
bleiben mögen, erstlich gegen Gott, im Gehorsam in allen Dingen,
und dann gegen die Menschen, in Rechtschaffenheit und Gerechtigkeit 
in allem Verkehr; damit Gott der Herr verherrlicht werde
durch einen Wandel in Wahrheit und Heiligkeit, Gerechtigkeit und
Gottseligkeit [...].

Die Priester in Newcastle, Kendal und anderen nördlichen
Gegenden waren sehr aufgebracht gegen uns. Einer, namens
Gilpin\person{Gilpin}, der manchmal zu uns nach Kendal gekommen war, war
bald von der Wahrheit abgefallen und aus allerlei einfältige
Gedanken gekommen, und die Priester gebrauchten nun das gegen
uns, wo sie nur konnten; aber die Kraft des Herrn warf sie
alle darnieder. Der Herr vernichtete zwei der Verfolgungsüchtigen
Richter von Carlisle und der dritte wurde einige Zeit darauf
seines Amts entsetzt und verließ die Stadt.


Um diese Zeit wurde den Soldaten der Eid, den sie Oliver
Cromwell\person{Cromwell, Oliver} schwören sollten, 
vorgelegt, und viele wurden entlassen,
weil sie im Gehorsam gegen Christus nicht schwören konnten.
Einer von diesen war John Stubbs\person{Stubbs, John}, 
der bekehrt worden war
während meiner Gefangenschaft in Carlisle, und ein guter Soldat
im Kampfe des Lammes und ein treuer Jünger Jesu geworden
ist. Er reiste Viel umher im Dienste des Herrn, in Holland,
Schottland\index{Schottland}, Italien\index{Italien}, 
Irland\index{Irland}, Ägypten\index{Ägypten}, 
Amerika\index{Amerika}. Und die Kraft
Gottes bewährte ihn vor den Händen der 
Papisten\index{Papisten}, obgleich er
oft in großer Gefahr vor der Inquisition\index{Inquisition} 
war. Andere unter
den Soldaten\index{Soldat} jedoch, die wohl ihrer 
Überzeugung nach bekehrt
worden waren, aber nicht zum Gehorsam gegen die Wahrheit 
gelangten, schwuren den Eid Cromwells: als diese später in 
Schottland waren, kamen sie in die Nähe einer Garnison; die dortige
Mannschaft glaubte es seien Feinde und töteten sie [...].


Der Herr trieb viele von denen, die er auserlesen hatte, in
% \picinclude{./080-089/p_s085.jpg} 
seinem Weinberg zu arbeiten, nach Süden zu gehen und sich im
Dienste des Evangeliums nach den südlichen und westlichen Teilen
des Landes zu verteilen; so gingen Francis 
Howgill\person{Howgill, Francis} und Edward
Burrough\person{Burrough, Edward} nach London\ort{London}, 
John Camm\person{Camm, John} und John 
Audland\person{Audland, John} nach
Bristol\ort{Bristol}, Richard 
Hubberthorn\person{Hubberthorn, Richard} und George 
Whitehead\person{Whitehead, George}\footnote{George Whitehead 
und Thomas Holmes, zwei eifrige Quäkerprediger.
(Näheres f. Weingarten a. a. D.)} gegen
Norwich\ort{Norwich}, Thomas Holmes\person{Holmes, Thomas} nach 
Wales\ort{Wales} und andere nach anderen
Richtungen; etwa sechzig Diener hatte der Herr ausersehen und
aus dem Norden in die Verschiedenen Teile des Landes gesandt.

\section{Ranter und falsch Propheten}

Um die Zeit fingen Rice Jones\person{Jones, Rice} 
von Nottingham\ort{Nottingham}, ein früherer
Baptist\index{Baptist} und jetzt Ranter\index{Ranter}, und 
seine Anhänger an, gegen mich
zu prophezeien\index{Prophezeiung}; sie sagten, ich hätte 
jetzt meinen Höhepunkt
erreicht und werde nun bald tief fallen [...] Aber seine und
der Seinen Weissagungen\index{Weissagung} erfüllte sich 
an ihnen selber; denn bald
darauf fielen sie ganz auseinander und viele von ihnen wurden
Freunde und blieben es; und durch des Herrn mächtige Macht und
Wahrheit vermehrten sich die Freunde [...] Rice Jones 
dagegen leistete den Eid\index{Eid} und war also dem Gebot Christi 
ungehorsam. Viele falsche Propheten\index{Propheten!falsche} 
haben sich gegen mich erhoben,
aber der Herr hat sie alle vernichtet und wird auch ferner alle
vernichten, die sich gegen seinen gesegneten Samen erheben [...].
In der Nähe von Kidsley-Park\ort{Kidsley-Park} stieß ich auf 
eine Schar Ranter\index{Ranter}; aber die Kraft des Herrn 
hielt sie drunten. Von da
ging ich in die Gegend des Peak zu Thomas 
Hammersley\person{Hammersley, Thomas},
wohin die Ranter dieser Gegend kamen und Viele angesehene
\textit{Fromme}. Die Ranter traten gegen mich auf und fingen an
zu schwören; als ich ihnen deswegen Vorstellungen machte, 
versuchten sie, Schriftstellen zu bringen und sagten, Abraham, Jakob
und Joseph haben geschworen und die Priester und Moses und
die Engel. Ich erwiderte: 

\grosszitat{
    ich gebe zu, das alle diese es taten,
    wie die Schrift es berichtet; Christus aber 
    sagt: \zitat{schwöret nicht!}
    Und Christus ist das Ende der Propheten und des alten 
    Priestertums und des Gesetzes Moses und regiert über das Haus Jakobs
    und Josephs, und er sagt: \zitat{ihr sollt nicht schwören.} Und als
    Gott den Erstgeborenen in die Welt sandte, sagte 
    er: \zitat{alle Engel
    sollen ihn anbeten} (Hebr. 1:6\bibel{Hebr. 01:06@Hebr. 1:6}), 
    also diesen Christus, der sagte, ihr
    sollt nicht schwören. Und was die Begründung anbelangt, welche
% \picinclude{./080-089/p_s086.jpg} 
    die Menschen für das Schwören geltend machen, um ihre 
    Streitigkeiten zu Ende zu bringen, so hat Christus, der gesagt hat,
    ihr sollt nicht schwören, den Teufel und seine Werke, deren eines
    eben das Streiten ist, vernichtet. Und Gott 
    sagt: \zitat{dies ist mein
    lieber Sohn, an dem ich Wohlgefallen habe, ihn sollt ihr hören.}
    (Mark. 9:7\bibel{Mark. 09:07@Mark. 9:7}). Also soll man den 
    Sohn hören, der das Schwören
    verbietet. Und der Apostel Jakobus, welcher den Sohn hörte,
    und ihm folgte und ihn verkündete, verbietet das Schwören,
    Jakobus 5:12\bibel{Jakobus 05:12@Jakobus 5:12}.
} 

Die Kraft des Herrn erfasste sie und sein
Sohn und seine Lehre beherrschten sie. Das Wort des Lebens
wurde reich und herrlich unter ihnen verkündet an dem Tage,
und viele wurden bekehrt.


Diesem Thomas Hammersley\person{Hammersley, Thomas}
wurde einmal gestattet, an
einem Geschworenengericht als Geschworener zu amtieren ohne
einen Eid\index{Eid} abzulegen; als er dann, als 
Vorsitzender, sein Gutachten abgab, erklärte der Richter, 
er sei nun doch schon seit Vielen
Jahren Richter, aber er habe noch nie ein so redliches Gutachten
gehört, als das von diesem Quäker! Es ließe sich noch viel
derartiges berichten, wenn die Zeit reichen würde. Die herrliche
Wahrheit des Herrn goss sich aus; ihr gebühret Preis und Ehre
ewiglich!


Aus der Durchreise durch Derbyshire\ort{Derbyshire} besuchte ich überall
Freunde, bis ich nach Swannington kam; hier war eine große
Versammlung, zu der Baptisten\index{Baptisten}, 
Ranter\index{Ranter} und viele andere \textit{Fromme}
kamen. Es hatte viele Zusammenstöße mit ihnen und den Priestern
der Stadt gegeben. Von überall her kamen Freunde zu dieser
Versammlung, so John Audland\person{Audland, John}, 
Francis Howgill\person{Howgill, Francis}, Edward 
Pyot\person{Pyot, Edward} von Bristol und Edward 
Vurrough\person{Vurrough, Edward} aus London und es wurden
viele bekehrt. Die Ranter\index{Ranter} machten Störungen 
\index{Versammlung!Störung} und benahmen sich
sehr unverschämt; aber schließlich kam die Macht des Herrn über
sie und sie unterlagen. Am darauf folgenden Tage kam Jacob
Bottomley\person{Bottomley, Jacob}, ein großer Ranter 
von Leieester; aber die Kraft des
Herrn überwältigte ihn. So auch einen Priester. Wir ließen
den Rantern sagen, sie sollten kommen
\index{Versammlung!Einladung von Gegnern} und es mit ihrem Gott
versuchen; sie kamen in Haufen und waren sehr wild und sangen
und pfiffen und tanzten\index{Tanz}; aber die Kraft des 
Herrn überwältigte
sie so, das viele von ihnen bekehrt wurden.



Von hier ging ich nach Twycros, wohin auch Ranter kamen
und vor mir sangen und tanzten; aber in der Furcht des Herrn
% \picinclude{./080-089/p_s087.jpg} 
trieb es mich, sie zu tadeln; und die Kraft des Herrn kam über sie,
so das einige von ihnen bekehrt wurden und den Geist Gottes
aufnahmen. Sie sind tüchtige Leute geworden, die rechtschaffen
in der Wahrheit Christi leben und wandeln. Ich ging zu Anthony
Brickleh\person{Brickleh, Anthony} in 
Warwickshire\ort{Warwickshire}, wo eine große 
Versammlung war;
mehrere Baptisten\index{Baptisten} und andere kamen und lärmten; 
aber die Kraft des Herrn kam über sie.

\section{Fox besucht seine Verwandten}

Hierauf ging ich nach Drayton\ort{Drayton} in 
Leicestershire\ort{Leicestershire}, um meine
Verwandten zu besuchen. Kaum war ich angekommen, so lies
der Priester Nathanael Stephens\person{Stephens, Nathanael}, 
der noch einen andern Priester
hatte kommen lassen und die Umgegend von meinem Kommen
benachrichtigt hatte, mich zu sich holen, denn sie konnten nichts
machen, ehe ich kam. Da ich drei Jahre meine Angehörigen
nicht gesehen hatte, so wusste ich nichts von ihren Absichten. Ich
ging nun auf den Platz des Turmhauses, wo die beiden
Priester waren, und wo sich eine Menge Leute versammelt hatten.
Als ich kam, wollten die Leute, das ich ins Turmhaus gehe; ich
fragte sie, was ich dort tun solle; sie erwiderten, Stephens
könne die Kälte nicht ertragen; ich sagte, er könne sie so gut
ertragen wie ich. Zuletzt begaben wir uns in einen großen Saal;
Richard Farnsworth\person{Farnsworth, Richard} war auch dabei; 
wir hatten einen großen
Disput mit den Priestern über ihren Wandel, und das sie so
sehr das Gegenteil von dem seien, was Christus und die Apostel
gewesen. Die Priester wollten wissen, wo die 
Zehnten\index{Zehnte}\index{Kirchensteuer} verboten
oder aufgehoben seien; ich wies es ihnen nach im 7. Kap. des
Hebräerbriefes\bibel{Hebr. 07@Hebr. 7}, wo nicht nur die 
Zehnten, sondern das ganze
Priestertum\index{Klerus}\index{Priestertum}, das Zehnten 
annahm, aufgehoben war und das
Gesetz, nach welchem das Priestertum eingesetzt und die Zehnten
erhoben wurden. Hierauf hetzten die Priester das Volk zur 
Frechheit und Roheit gegen uns auf. Ich hatte Stephens seit seiner
Kindheit gekannt und konnte ihnen darum aufdecken, was für
eine Art von Mensch er sei und was hinter seinen Predigten
stecke, und wie er, wie alle Priester, die Verheißungen aus den
alten Menschen, der sterben muss, bezog; dann zeigte ich ihnen,
das die Verheißungen vielmehr dem Samen galten, nicht den
vielen Samen, sondern dem einen Samen, Christus, der derselbe
ist in Mann und Weib\index{Das von Gott in Jedem}; denn 
alle müssen wiedergeboren\index{Wiedergeburt} werden,
ehe sie ins Reich Gottes\index{Reich Gottes} eingehen 
können. Er erwiderte mir
darauf, ich sollte nicht in der Weise richten\index{Richten}; 
ich entgegnete ihm,
% \picinclude{./080-089/p_s088.jpg} 
\zitat{der Geistliche richtet alles} 
(1. Cor. 2:15\bibel{Cor. 1. 02:15@1. Cor. 2:15}); er gab zu, 
das dies genau der Schrift gemäß sei; dann aber 
fuhr er fort: \zitat{ihr Nachbarn, das ist die Sache: 
George Fox ist zum Lichte der Sonne
gekommen und nun möchte er mein Sternenlicht auslöschen.} Ich
erwiderte: \zitat{ich will nicht das kleinste Maß von dem, was einer
von Gott hat, in jemand unterdrücken, noch viel weniger sein
Sternenlicht auslöschen, wenn es ein wirkliches Sternenlicht ist,
ein Licht vom Morgenstern.} Dann erklärte ich ihm, das, wenn
er etwas von Gott oder Christus empfangen habe, er umsonst
predigen müsse und nicht Zehnten nehmen von den Leuten für
seine Predigten, da er ja gesehen habe, wie Christus seinen
Jüngern befohlen habe, umsonst zu geben, wie sie es umsonst
empfangen hätten. Ich schärfte ihm also ein, nicht mehr für
Zehnten und Lohn zu predigen. Aber er sagte, dem werde er
sich nicht fügen. Die Leute fingen an, unverschämt zu werden,
und wir brachen darum auf. Dennoch waren etliche an dem
Tage der Wahrheit zugetan worden. Ehe ich fort ging, sagte
ich ihnen, das ich im Sinn habe, nächste Woche, so Gott wolle,
wieder in der Stadt zu sein. In der Zwischenzeit ging ich in
die Umgegend und hielt Versammlungen, und nach acht Tagen
kam ich wieder zurück. Der Priester hatte für diese Zeit 7 Priester
kommen lassen, um ihm zu helfen; und Stephens hatte in einem
Gottesdienst am Markttage in Adderston angezeigt, das an dem
und dem Tage ein Disput mit mir stattfinden werde. Ich wusste
nichts davon und hatte nur gesagt, ich werde über acht Tage
wieder in der Stadt sein. Die acht Priester hatten etliche hundert
Leute versammelt, meist aus der Umgegend und wollten, ich sollte
ins Turmhaus gehen; aber ich wollte nicht hingehen, sondern
ich ging auf einen Hügel und redete von dort zum Volk [...].


Es kamen einige Unverschämte und nahmen mich auf die
Arme und trugen mich unter die Türe des Turmhauses, in der
Absicht, mich mit Gewalt ins Turmhaus zu bringen; da aber
die Tür geschlossen war, purzelten sie alle übereinander; und ich
lag zu unterst. So bald ich konnte, kroch ich hervor und ging wieder
auf den Hügel; nun schleppten sie mich bis an die Mauer
des Turmhauses und setzten mich auf eine Art Steinbank; alle
Priester waren auch herbeigelaufen und standen mitten unter dem
Volk herum, und alle schrien: \zitat{Beweise, beweise!} Ich sagte,
ich hörte nicht auf ihre Stimmen, denn es seien die Stimmen von
% \picinclude{./080-089/p_s089.jpg} 
Mietlingen und Fremdlingen. Sie schrien wieder: \zitat{Beweise,
beweise!} Ich wies auf Johannes, wo sie sehen können, was
Christus zu ihresgleichen sage, nämlich: \zitat{Ich bin der gute Hirte,
der sein Leben gibt für seine Schafe, der Mietling aber flieht,
wenn der Wolf kommt.} Ich schlug ihnen vor, ihnen zu beweisen,
das sie solche Mietlinge seien; darauf rissen die Priester mich
wieder herunter und stiegen selber alle auf Steinbänke
an der Mauer des Turmhauses. Da fühlte ich, wie Gottes
mächtige Kraft über alle kam und sprach zu ihnen: \zitat{Wenn ihr
mir Gehör schenken wollt und mich ruhig anhören, so will ich
euch aus der Schrift zeigen, warum ich die acht Priester oder
Lehrer, die vor mir stehen, nicht anerkenne und überhaupt keine
Mietlingslehrer der Welt.} Priester und Volk erklärten sich bereit
zu hören. Da zeigte, ich ihnen aus den Propheten Jesaja,
Jeremia, Ezechiel, Micha, Maleachi und anderen, das sie in den
Fusstapfen derer wandeln, gegen die Gott seine Propheten 
gesandt hatte [...].


Dann als ich an das neue Testament kam, zeigte ich ihnen,
das sie wie die Hohenpriester und Schriftgelehrten seien, und
wie die Pharisäer, gegen die Christus \zitat{wehe!} schrie 
(Matth. 23\bibel{Matth. 23}).


In dem ich in dieser Weise ausführlich aus der Schrift bewiesen
hatte, warum sie den Pharisäern gleichen, [...] und sie vor allem
Volk unter die Pharisäer, falschen Propheten und Verführer
gerechnet hatte und gezeigt, wie ihresgleichen von den wahren
Propheten und Christus verdammt werden, wies ich sie aus das
Licht Jesu Christi hin, das einen jeden, der in die Welt kommt,
erleuchtet (Joh. 1:9\bibel{Joh. 01:09@Joh. 1:9}), und durch 
dieses Licht könnten sie erkennen,
ob das Gesagte wahr sei\index{Exegese}. Sie mochten nichts 
davon hören, das
ich sie auf das Göttliche in ihnen, auf das Licht Jesu Christi
hinwies\index{Inneres Licht}. Bis dahin waren sie alle 
ruhig gewesen, nun aber
rief einer der \textit{Frommen}: \zitat{Wirst du denn nie 
fertig, Fox?} \index{Predigen!langes}
Ich erwiderte, ich sei nun bald fertig; ich fuhr noch eine Weile
fort, bis ich fühlte, das ich an ihnen getan hatte, was ich musste
in der Kraft des Herrn [...]. 

Als ich fertig war, flüsterten
die Priester untereinander, und Priester Stephens kam zu mir
und verlangte, das mein Vater und mein Bruder und ich mit
ihm beseite kommen, damit er mit uns reden könne; und die
anderen Priester mussten das Volk davon abhalten uns 
nachzukommen. Ich ging sehr ungern mit ihm, aber da das Volk schrie:
% \picinclude{./090-099/p_s090.jpg} 
\zitat{Geh, George, geh nur,} so fürchtete ich, das, wenn ich nicht
ginge, man sage, ich sei meinen Eltern ungehorsam; so ging ich,
und die übrigen Priester wollten das Volk abhalten, aber es
gelang ihnen nicht, denn da alle uns hören wollten, wurden wir
ganz umringt. Ich fragte den Priester, was er zu sagen habe?
er antwortete, wenn er nicht auf dem rechten Wege sei, so sollte
ich für ihn beten; und wenn ich nicht auf dem rechten Wege sei,
so wollte er für mich beten; und er wolle mir vorsagen, was ich
für ihn beten solle\index{Beten}. Ich erwiderte ihm: 

\grosszitat{Es scheint, das du nicht
einmal weist, ob du auf dem rechten Wege bist; ich aber weiß,
das ich auf dem rechten Wege bin, Jesus Christus, in welchem
du nicht bist, und du wolltest mir vorsagen, wie ich zu beten
habe, und verwirfst doch das 
Common-Prayerbook\index{Common-Prayerbook} so gut wie ich,
und ich verwerfe dein Geplapper ebenfalls. So du willst, das ich
nach etwas Hergesagtem für dich bete, heißt das nicht, die Lehre
der Apostel missachten und ihr Beten im Geist, der die Worte
eingibt?} 

Hier fingen die Leute an zu lachen; mich aber trieb
es, weiter zu ihm zu reden. Nachdem ich ihm gesagt, was
mir zu sagen oblag, und das ich, so Gott wolle, über acht Tage wieder
in der Stadt sein werde, gingen wir fort. Die Priester machten,
das sie fort kamen und viele wurden gewonnen, denn die Kraft des
Herrn kam über alle. Wenn sie schon meinten an diesem Tage
der Wahrheit geschadet zu haben, war doch mancher gewonnen
worden, und viele, die schon früher gewonnen worden, wurden durch
das, was an jenem Tage geschehen, bestärkt, und es gab den
Priestern einen Stoß. Mein Vater, obgleich er ein Anhänger der
Priester war, war so befriedigt, das er mit seinem Stock auf die
Erde schlug und sagte: 
\zitat{wahrlich, ich sehe, das wer willens ist,
bei der Wahrheit zu bleiben, dem wird sie durchhelfen} [...].

\section{Verhaftung in Whetstone}


Darauf zog ich wieder umher und hielt Versammlungen und
kam nach Swannington, wohin auch wieder Soldaten kamen;
aber die Versammlung war ruhig, die Macht Gottes war
über allen, und die Soldaten störten mich nicht. Darauf ging
ich nach Leicester und Whetstone. Dahin kamen siebzehn Soldaten
aus Oberst Hackers Regiment, mit ihrem Anführer, und führten
mich, gerade vor Beginn der Versammlung, hinweg, obgleich die
Freunde, die von allen möglichen Orten hergekommen waren,
schon anfingen sich zu versammeln. Ich sagte dem Vorgesetzten,
er solle wenigstens die Freunde in Ruhe lassen, ich wolle für sie
% \picinclude{./090-099/p_s091.jpg} 
alle haften; so nahmen sie denn mich und ließen die andern in
Ruhe, ausgenommen Alexander 
Parken\person{Parken, Alexander}\footnote{Alexander Parker, 
ein Mann von vornehmer Herkunft, reiste viel im
Dienst des Quäkertums und schrieb viele Bücher und Briefe 
zu seiner Verbreitung.} der mit mir kam. Am
Abend brachten sie mich vor Oberst Hacker; sein Major, seine
Hauptleute und viele seiner Leute waren zugegen und wir gaben
ausführlich Auskunft über die Priester und über die Versammlungen,
denn es ging damals gerade das Gerücht von einer Verschwörung
gegen Oliver Cromwell. Ich hatte lange Erörterungen über das
Licht Christi, das einen jeden, der in die Welt kommt, erleuchtet
(Joh. 1:9\bibel{Joh. 01:09@Joh. 1:9}). Oberst Hacker fragte, 
ob es dieses Licht aus Christus
gewesen sei, das den Judas dazu geführt habe, seinen Herrn zu
verraten und sich darnach zu erhängen? Ich sagte ihm: \zitat{nein,
das war der Geist der Finsternis\index{Finsternis!Geist der}, 
der Christus und sein Licht
haste.} Darauf sagte Hacker, ich solle nach Hause gehen und dort
bleiben, und nicht überall zu den Versammlungen gehen. Ich sagte
ihm, ich sei ein ganz harmloser Mensch und habe nichts mit 
Verschwörungen zu tun, vielmehr verabscheue ich solches. Sein Sohn
Needham sagte: \zitat{Vater, dieser Mensch hat nun schon lange 
geherrscht, es ist Zeit, das man ihn unschädlich mache.} Ich fragte
ihn, \zitat{warum, was habe ich getan? oder wem habe ich je etwas
zu leide getan? ich bin in dieser Gegend geboren und 
aufgewachsen, wer kann mir irgend etwas Böses nachsagen seit meiner
Kindheit?} Darauf fragte mich Oberst Hacker nochmals, ob ich
nach Hause gehen wolle und dort bleiben? Ich antwortete ihm,
ich würde mich ja mit einem solchen Versprechen schuldig bekennen,
wenn ich nach Hause ginge und aus meinem Hause ein Gefängnis
machen wollte; und ginge ich dann doch zu den Versammlungen, so
würde es heißen, ich sei dem Befehl ungehorsam. Ich erklärte
ihnen, ich gehe auf des Herrn Geheiß zu den Versammlungen,
darum könne ich mich ihren Vorschriften nicht fügen; aber wir
seien ein friedliches Volk. \zitat{Gut denn,} sagte Oberst 
Hacker, \zitat{ich will
euch zum Lord Protektor schicken, durch Hauptmann Drury, einen
aus seiner Leibgarde.} Die Nacht über wurde ich als Gefangener
gehalten und am folgenden Morgen um sechs Uhr dem 
Hauptmann Drury\person{Drury, Hauptmann} übergeben. 
Ich wünschte vor dem Fortgehen noch
mit Oberst Hacker zu reden, er lies mich vor sein Bett kommen und
drang sogleich wieder in mich, nach Hause zu gehen und keine
Versammlungen zu halten; ich erklärte ihm, ich könne mich dem
% \picinclude{./090-099/p_s092.jpg} 
nicht fügen, sondern müsse meine Freiheit haben. \zitat{Dann,} sagte
er, \zitat{müsst ihr vor den Protektor}. Hierauf kniete ich an feinem
Bett nieder und betete\index{Beten!für andere} zum Herrn, 
ihm zu vergeben, denn er war
ein Pilatus, auch wenn er seine Hände gewaschen hätte; und ich
flehte zum Herrn, das, wenn der Tag seiner Prüfung und 
Heimsuchung komme, er sich dessen, was ich ihm gesagt, erinnern möge.
Er war eben aufgehetzt von Priester Stephens und den andern
Priestern und \textit{Frommen}, die darin ihre Bosheit ausließen, weil
sie mich durch ihr Argument nicht hatten überwinden können
und dem Geiste Gottes in mir nicht hatten widerstehen können;
darum hatten sie nun die Soldaten geschickt, um mich zu greifen.


Als später dieser Oberst Hacker im Gefängnis in London
war, wurde es ihm ein oder zwei Tage vor seiner Hinrichtung
in Erinnerung gebracht, wie er an den Unschuldigen gehandelt
hatte, und er gedachte daran und bekannte es Margaret 
Fell\person{Fell, Margaret};
und es bedrückte ihn. Nun konnte sein Sohn, der damals
gesagt hatte, ich habe genug geherrscht, es sei Zeit, mich fort zu
schaffen, zusehen, wie sein Vater fortgeschafft wurde, als man ihn
erhängte in Tyburn.

Ich wurde nun von Hauptmann Drury als Gefangener von
Leicester fortgebracht. Als wir nach Harborough kamen, fragte
er mich, ob ich heimgehen wolle und 14 Tage dort bleiben? Er
versprach mir die Freiheit, wenn ich weder Versammlungen halten
noch zu solchen gehen wolle. Ich erwiderte ihm, ich könne nichts
dergleichen versprechen; er fragte und versuchte mich wiederholt
auf dem Wege in derselben Weise, und immer gab ich ihm die
selbe Antwort. So brachte er mich nach London und umquartierte
mich in Mermaid ein; unterwegs trieb es mich, die Leute zu
warnen vor dem Tag des Herrn, der über sie kommen werde.
Nachdem Hauptmann Drury mich untergebracht, verließ er mich
und ging zum Protektor, um Bericht über mich zu erstatten. Als
er zurück kam, sagte er, der Protektor verlange, das ich kein
mörderisches Schwert gegen ihn oder die Regierung gebrauche,
und das ich dies in beliebigen Worten schriftlich erklären und mit
meiner Unterschrift versehen solle. Ich antwortete Hauptmann
Drury nur wenig; aber am nächsten Morgen trieb mich der Herr,
ein Schreiben an den Protektor aufzusetzen, in dem ich vor dem
Angesicht Gottes des Herrn erklärte, das ich das Tragen eines
mörderischen Schwertes oder irgend einer anderen 
äußeren Waffe\index{Waffe!äussere}
% \picinclude{./090-099/p_s093.jpg} 
verabscheue, und das ich von Gott gesandt sei, Zeugnis 
abzulegen gegen jegliche Gewalttätigkeit und gegen die Werke der
Finsternis; und um die Leute von der Finsternis zum Licht zu
bringen und vom Kriegen und Streiten zum Evangelium des
Friedens. Nachdem ich geschrieben, was der Herr mir eingegeben
hatte\index{Eingebung}, setzte ich meinen Namen darunter 
und übergab es Hauptmann Drury, damit er es Oliver Cromwell gebe, 
was er auch tat. 

\section{Hauptmann Drury}

Nach einiger Zeit brachte mich Hauptmann 
Drury\person{Drury, Hauptmann} vor den
Protektor in Whitehall\ort{Whitehall}; es war an einem 
Morgen, ehe er angekleidet war, und einer, namens Harvey, 
der sich auch eine Zeit lang zu den Freunden gehalten hatte 
aber ungehorsam geworden war, bediente ihn. Als ich eintrat, 
trieb es mich zu sagen: \zitat{Friede sei mit diesem Hause,} 
und ich ermahnte ihn, in der Furcht
Gottes zu bleiben, damit er Weisheit von ihm empfangen möge,
das sie ihn leite; und das er alle Dinge, die in seiner Hand
seien, zu Gottes Ehre regiere. Ich redete lange mit ihm über
die Wahrheit\index{Wahrheit} und über die Religion, er 
zeigte sich sehr verständig;
aber er sagte, wir zankten mit den Priestern, die er Diener Gottes
nannte. Ich entgegnete ihm, ich zanke nicht mit ihnen, sondern
sie mit mir und mit meinen Freunden. \zitat{Aber}, sagte 
ich, \zitat{wenn wir die Propheten und Apostel anerkennen, 
so können wir solche
Lehrer, Propheten und Hirten, gegen welche die Propheten und
Christus auftraten, nicht gut heißen, sondern wir müssen auch
gegen sie auftreten, durch denselben Geist und dieselbe Kraft.}
Ferner zeigte ich ihm, das die Propheten, Christus und die
Apostel umsonst predigten und gegen die auftraten, welche es
nicht umsonst taten, sondern um schändlichen Gewinnes willen
und die um Geld wahrsagten und um Lohn lehrten 
(Micha 3:11\bibel{Micha 03:11@Micha 3:11}),
gierig und geizig waren und nie genug bekamen; und das die,
welche den Geist Christi und der Apostel und Propheten haben,
auch jetzt noch gegen das alles auftreten müssen, wie jene damals.
Während ich sprach, sagte er mehrmals, es sei sehr gut, es sei
wahr. Ich sagte ihm, das alle, die sich Christen nennen, die
Schrift\index{Bibel} haben, aber nicht alle die Kraft und 
den Geist, welche
die hatten, die die Schrift geschrieben, und dies sei der Grund,
warum sie nicht in der Gemeinschaft mit dem Vater und dem
Sohne seien, noch mit der Schrift, noch unter einander. 


Ich redete noch über vieles andere mit ihm; da aber Leute herein
kamen, zog ich mich ein wenig zurück; als ich mich anschickte fort
% \picinclude{./090-099/p_s094.jpg} 
zu gehen, faste er mich bei der Hand, und sagte mit Tränen in
den Augen: \zitat{Komm wieder zu mir, denn wenn du und ich nur
eine Stunde im Tage beisammen wären, so würden wir einander
näher kommen}; und er fügte bei, er wünsche mir so wenig etwas
Böses als seiner eigenen Seele. Ich sagte ihm, wenn er es tun
würde, so würde er damit seiner eigenen Seele schaden; und ich
bat ihn, auf die Stimme Gottes zu hören, auf das er in seiner
Weisheit bleiben möge und ihm gehorchen; wenn er es tue, so
werde er vor Hartherzigkeit bewahrt bleiben; wenn er aber
nicht auf Gottes Stimme höre, so werde sein Herz verhärtet
werden. Er sagte, dies sei wahr; darauf ging ich hinaus, und
Hauptmann Drury kam hinter mir drein und teilte mir mit, sein
Lord Protektor sage, ich sei frei und könne gehen, wohin ich
wolle. 


Darauf wurde ich in einen großen Saal geführt, wo
die Kammerherrn des Lord Protektor zu speisen pflegten; ich fragte,
warum ich hierher geführt werde? sie sagten, es geschehe auf
Befehl des Protektor, damit ich mit ihnen speise. Ich hieß sie,
dem Protektor sagen, das ich nicht von seinem Brote esse, noch
von seinen Getränken trinke\index{Gasfreundschaft!ablehnen}. 
Als er dies hörte, sagte er: \zitat{nun
sehe ich, das ein Volk entstanden und hervorgetreten ist, welches
ich nicht zu gewinnen vermag, weder durch Gaben, noch durch
Ehren, noch Stellen, während mir dies bei allen anderen Sekten
und Menschen gelingt}\index{Vereinnahmung}, worauf man 
ihm entgegnete, das wir
ja das Eigene hingeben und darum kaum nach dem Seinigen
trachten würden [...].

\section{Verleumdung in der Zeitung}

Ich begab mich nach London\ort{London}, wo wir große und mächtige
Versammlungen hatten. Der Zudrang war so groß, das ich fast
nicht hinein konnte,\index{Versammlung!großer Andrang} 
und die Wahrheit breitete sich ungeheuer aus.
Thomas Aldam\person{Aldam, Thomas} und Robert 
Craven\person{Robert, Craven} und viele Freunde kamen
nach London, um nach mir zu sehen; aber 
Alexander Parker\person{Parker, Alexander} blieb bei mir.

Nach einiger Zeit ging ich wieder nach Whitehall und es
trieb mich, den Tag des Herrn unter ihnen zu verkünden und
das der Herr gekommen sei, sein Volk selbst zu lehren, und ich
predigte sowohl den Offizieren\index{Predigen!Soldaten} 
als denen von der Garde Olivers.
Aber ein Priester widersprach, als ich das Wort des Herrn
verkündete; denn Oliver hatte verschiedene Priester um sich, und
dieser war ein Neuigkeitskrämer, ein hässlicher Priester, ein 
hinterlistiger, missgünstiger Mann; ich sagte ihm, er solle Busse tun
% \picinclude{./090-099/p_s095.jpg} 
und er setzte in der darauf folgenden Woche in seine Zeitung,
ich sei in Whitehall gewesen und habe dort einem Diener Gottes
gesagt, er solle Busse tun. Als ich wieder dorthin kam, traf ich
ihn wieder, und viele Leute scharten sich um uns. Ich bewies
dem Priester, das er in verschiedenen Dingen gelogen habe, und
er musste schweigen\person{Fox!Verleumdung}. Er schrieb in der 
Zeitung\index{Zeitung}, ich habe silberne
Knöpfe; was: falsch war, denn sie waren bloß aus Blech. Ferner
schrieb er, ich lege den Leuten Bänder um die Arme, damit sie
mir folgen; das war wieder gelogen, denn ich hatte in meinem
ganzen Leben nie Bänder getragen oder gebraucht. Drei Freunde
gingen hin, um den Priester zur Rede zu stellen und ihn zu
fragen, woher er diese Dinge habe; er sagte, eine Frau habe es
ihm gesagt; und wenn sie wieder kommen, so wolle er ihnen ihren
Namen sagen. Als sie wieder kamen, sagte er, es sei ein Mann
gewesen, aber er sage den Namen nicht, wenn sie wieder kommen,
wolle er ihn den sagen. Als sie das dritte mal kamen, sagte
er ihn wieder nicht, behauptete aber, wenn ich erkläre, das alles
nicht wahr sei, so wolle er es in die Zeitung setzen. Als darauf
die Freunde ihm diese Erklärung brachten, so wollte er sie doch
nicht aufnehmen, sondern wurde zornig. 

So handelte dieser
infame Lügenschmied, um der Wahrheit zu schaden und um die
Leute gegen die Freunde und die Wahrheit einzunehmen, wovon
ein ausführlicher Bericht in einem Buche, das bald darauf 
gedruckt wurde, kann ersehen werden. Diese lügnerischen Priester
waren Independenten\index{Independenten}, wie die zu 
Leicester; aber des Herrn
Kraft vernichtete alle ihre Lügen, und viele kamen dazu, die
Schlechtigkeit der Priester einzusehen. Der Herr des Himmels
brachte mich durch seine Kraft durch alles hindurch, und seine
herrliche Kraft tat sich kund im Lande, so das in dieser Zeit
viele Freunde getrieben wurden, umher zu ziehen, um das ewige
Evangelium zu verkünden, in allen Teilen des Landes und auch
in Schottland; und die Herrlichkeit des Herrn erschien allen zu
seiner ewigen Ehre [...]. Es fanden große Bekehrungen in
London statt und auch mehrere im Hause des Protektors und in
seiner Familie; ich versuchte zu ihm zu gehen, aber ich bekam
keinen Zutritt, die Wachen waren so unfreundlich.


Die Presbyterianer\index{Presbyterianer}, Independenten 
und Baptisten\index{Baptisten} waren sehr
erzürnt, denn viele bekehrten sich zum Herrn Jesus Christus und
hörten seine Lehre. Sie empfingen seine Kraft und spürten sie
% \picinclude{./090-099/p_s096.jpg} 
in ihren Herzen, und das trieb sie, gegen die übrigen 
aufzutreten.
