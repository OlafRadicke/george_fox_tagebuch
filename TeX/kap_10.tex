%%%%%%%%%%%%%%%%%%% Kapitel 10. %%%%%%%%%%%%%%%%%%%%%%%%%%%%%%

\chapter[Warnung an die Kegelspieler. Naylors Fall.]{Warnung an die Kegelspieler. Naylors Fall.}

\begin{center}
\textbf{Warnung an die Kegelspieler. Naylors Fall. Disput mit Paul
Gwin. Besuch bei Cromwell. Herumreisen bei den gefangenen
Freunden. Reise in Wales.}
\end{center}

\section{Zurück in der Freiheit}

Als ich während meiner Gefangenschaft sah, wie sie sich in
Castle-Green\ort{Castle-Green} mit Kegelschieben\index{Spielen!Kegeln} 
vergnügten, hatte ich eine Schrift geschrieben, worin es hieß:

\brief{An die Kegler im Castle-Green}
{
    Es gehet das Wort des Herrn an euch, ihr eitlen Müsiggänger, 
    die ihr so dem Spiel, dem Vergnügen und solchen 
    einfältigen Übungen zugetan seid, das ihr bedenken möget, was ihr
    tut. Ist das der Zweck eures Daseins? Machte Gott alles zu
    eurem Vergnügen und eurem Gebrauch? Machte nicht Gott alle
    Dinge, damit er darin in Furcht und Anbetung, im Geist und in
    der Wahrheit, in Gerechtigkeit und Heiligkeit geehrt werde? Wie
    könnet ihr Gott dienen, solange ihr euren 
    Vergnügen\index{Vergnügen} nachgeht?
    Ihr könnet nicht Gott dienen und den weltlichen Vergnügen, dem
    Kegeln, Jagen und Trinken und dergleichen; wenn euer Herz bei
    derartigem ist, so will Gott eure Lippen nicht, fraget euch, ob das
    nicht wahr ist [...]. 

    \bigskip
    % \picinclude{./120-129/p_s121.jpg} 
    An die Kegler im Castle-Green\ort{Castle-Green}, geschrieben 
    im Kerker zu Launceston\ort{Launceston}.
}

Als wir nun frei waren, [...] zogen wir [...] über Launceston 
[...]. Okington [...] nach Exeter\ort{Exeter}, wo viele Freunde 
gefangen waren, unter anderem James 
Naylor\person{Naylor, James!Gefangenschaft}. 
Kurz ehe wir frei wurden, hatte James Naylor sich in phantastische 
Ideen verirrt und viele mit ihm, was eine große Verwirrung im Land
anrichtete. Er kam nach Bristol und stiftete dort Unruhe; von
da wollte er nach Launceston gehen, um mich zu besuchen; aber
unterwegs wurde er angehalten und in Exeter gefangen gesetzt,
sowie verschiedene Andere; einer davon, ein ehrlicher, gottseliger
Mensch, starb in der Gefangenschaft; sein Blut kommt auf seine
Verfolger.

\section{Verwirrung und Konflikte mit James Naylor}

Am Abend, als wir nach Exeter kamen, redete ich mit James
Naylor, denn ich sah, das er ganz in Irrtum geraten war, sowie
auch seine Genossen. Am folgenden Tag -- es war der Erste
Tag -- besuchten wir die Gefangenen und hatten im Gefängnis
eine Versammlung mit ihnen; aber James Naylor und einige
von ihnen konnten es in der Versammlung nicht aushalten. Es
kam ein Kavallerie-Korporal in die Versammlung; er wurde 
gewonnen und blieb ein sehr guter Freund.


Am folgenden Tag redete ich wieder mit James Naylor; er
machte herunter, was ich ihm sagte, und war verwirrt und 
verdreht, dennoch wollte er gerne kommen und mich küssen. Aber
ich sagte, \zitat{weil er sich der Kraft Gottes widersetze, 
so könne ich seine Freundlichkeitsbezeugungen nicht annehmen}; 
der Herr trieb mich, ihn zu verweisen und ihn unter die Kraft 
des Herrn zu stellen. So war nun, nachdem ich gegen die Welt 
gekämpft, unter den Freunden ein böser Geist erwacht, gegen 
den man kämpfen musste. Ich ermahnte ihn und seine Genossen. 
Als er nach London\ort{London} kam, wurde ihm sein Widerstand 
gegen Gottes Kraft in
mir und gegen die ihm durch mich verkündete Wahrheit zur
größten Last. Aber er kam dazu, seine Abirrung einzusehen und
zu verdammen,\index{Buße} und nach einiger Zeit kehrte er 
sich der Wahrheit
wieder zu, wie man in den gedruckten Berichten seiner Buße,
Verurteilung\footnote{Im Original steht 
\zitat{Ververurteilung} offebar ein Fehler.} und 
Wiedererhebung ausführlich sieht [...].

\section{Große Versammlungen im Freiem}

Von Exeter gingen wir [...] zu E. Pyot\person{Pyot, E.} 
in Bristol\ort{Bristol}. Am
Morgen des Ersten Tages ging ich zu der Versammlung in
Broadmead\ort{Broadmead}; sie war zahlreich und ruhig. 
Es wurde eine Versammlung angezeigt auf den Nachmittag 
im Garten. Es war
% \picinclude{./120-129/p_s122.jpg} 
ein ungebildeter, unverschämter Baptist in Bristol, namens Paul
Gwin,\person{Gwin, Paul} der schon früher in unseren 
Versammlungen große Störungen\index{Versammlung!Störung}
verursacht hatte, ermutigt und angetrieben durch den 
Bürgermeister, welcher, wie gesagt wurde, ihm sogar manchmal ein
Mittagessen gab, um ihn zu ermutigen. Er war von einer
solchen Pöbelmenge gefolgt, das die Zahl derer, die zu unserer
Versammlung im Freien kamen, oft auf 10.000 geschätzt 
wurde.\index{Versammlung!Große}
Als ich auf dem Wege nach dem Garten war, sagte man mir,
das Paul Gwin, der zänkische Baptist, zur Versammlung kommen
würde. Ich sagte, man solle sich nicht darum kümmern, es sei
mir einerlei, wer käme.\index{Versammlung!für alle offen} 


In Garten angekommen, stieg ich auf
einen Stein, auf den die Freunde zu stehen pflegten, wenn sie
sprachen;\index{Rede(-beitrag)}\index{Versammlung!Rede(-beitrag)}
\index{Versammlung!im Freiehen} 
und der Herr trieb mich, den Hut abzunehmen\index{Hut!abnehmen} und so
geraume Zeit zu stehen und mich von den Leuten ansehen zu
lassen; es waren einige 1000 Leute da. Als ich nun so schweigend
dastand, fing jener Baptist an, mein Haar zu 
tadeln,\person{Fox!lange Haare}\index{Haare!lange} aber ich
sagte nichts zu ihm. Da brach er in einen Wortschwall aus und
rief: \zitat{Ihr Weisen von Bristol, ich staune über euch, das ihr hier
steht, um Einen etwas sagen und behaupten zu hören, der es
nicht beweisen kann}. Da öffnete der Herr meinen Mund, (bis
dahin hatte ich noch nichts geredet,) und ich fragte die Leute, ob
sie mich je hätten reden hören oder je zuvor gesehen hätten? und
ich hieß sie, nicht zu vergessen, was für eine Sorte von Mensch
der sei, der so frech sage, das ich rede und behaupte, was ich
nicht beweisen könne, da doch weder er noch sie mich je zuvor
gesehen hätten. Darum sei es ein lügnerischer, böswilliger und
schlechter Geist, der aus ihm rede; er sei vom 
Teufel\index{Teufel} und nicht
von Gott. Ich gebot ihm, bei der Furcht und Kraft Gottes zu
schweigen, und die mächtige Kraft Gottes kam über ihn und alle
seine Anhänger. Darauf hatten wir eine herrliche, friedliche 


Versammlung, und das Wort des Leben; ward unter ihnen verkündet
und sie kehrten sich von der Finsternis zum Licht, zu Jesus
Christus, dem Heiland. Die Schrift wurde ihnen reichlich 
geöffnet und die menschlichen Überlieferungen, Zutaten, Mittel und
Lehren darin nachgewiesen; sie wurden auf das Licht Christi 
hingewiesen, durch das sie solches alles erkennen können, sowie auch
Christum selbst, damit er sie erlöse. Ich erklärte ihnen auch die
Zeichen und Sinnbilder von Christus in den Zeiten des Gesetzes
und zeigte ihnen, das Christus gekommen war, den Zeichen,
\index{Christologie}\index{Gesetzlichkeit}
% \picinclude{./120-129/p_s123.jpg} 
Zehnten und Eiden ein Ende zu machen und das Schwören 
abzuschaffen, und einsetzte, das man bei \zitat{ja} 
und \zitat{nein} bleibe,
und das man umsonst predige, denn er wolle nun sein Volk
selber lehren und sein herrlicher \zitat{Ausgang aus der Höhe sei nun
erschienen} (Luk. 1:78\bibel{Luk. 01:78@Luk. 1:78}). Mehrere 
Stunden verkündete ich\person{Fox!langes Predigen} das
Wort des Lebens unter ihnen und die ewige Kraft Gottes, durch
die sie möchten zu dem, der von Anfang war, zurückkehren und
mit ihm versöhnt werden (2 Cor. 5\bibel{Cor. 2 05@2 Cor. 5}). 
Und nachdem ich sie an den
Geist Gottes in ihrem Innern gewiesen, der sie in alle Wahrheit
leiten würde (Joh. 16\bibel{Joh. 16}), trieb es mich zu 
beten in der mächtigen
Kraft Gottes; des Herrn Kraft kam über alle. 


Als ich geendet,
fing jener Kerl aufs Neue an zu schwatzen. John 
Audland\person{Audland, John}
wurde getrieben, ihn zur Buse und Furcht Gottes zu vermahnen.
Da nun seine eigenen Leute und Anhänger sich seiner schämten,
ging er fort und hat nie mehr eine unserer Versammlungen 
gestört. Die Versammlung ging ruhig auseinander und des Herrn
Kraft und Herrlichkeit leuchtete über allen; es war ein gesegneter
Tag; die Ehre war des Herrn. Einige Zeit darauf ging dieser
Paul Gwin\person{Gwin, Paul} über Meer; Viele Jahre nachher 
begegnete ich ihm wieder in Barbadoes\ort{Barbadoes} [...].

\section{Aufenthalt in London}

Von Kingston\ort{Kingston} ritten wir nach London\ort{London}. 
Als wir in die Nähe des Hyde Park\ort{Hyde Park} kamen, 
sahen wir eine große Volksmenge, und bei
näherem Zusehen erblickten wir den Protektor,\person{Protektor} 
der in seinem
Wagen daherkam. Ich ritt an die Seite seines Wagens. Einige
seiner Leibgarde suchten mich wegzutreiben, aber er wehrte es
ihnen. So ritt ich neben ihm her und verkündete, was der Herr
mir eingab über seinen Zustand und über die Not der Freunde
im Lande;\person{Fox!reitend predigend} ich zeigte ihm, 
wie sehr diese Verfolgungen Christus
und seinen Aposteln und dem Christentum zuwider seien. Als wir
am Tor des James Park ankamen, verließ ich ihn; ehe wir uns
trennten, sagte er noch, ich solle zu ihm nach Hause kommen.


Am folgenden Tag kam eine Magd seiner Frau zu mir in meine
Wohnung und erzählte mir, ihr Herr sei zu ihr gekommen und
habe gesagt, er wolle ihr eine frohe Nachricht mitteilen. Als sie
ihn fragte, was für eine, sagte er ihr: George Fox sei in die
Stadt gekommen. Sie habe geantwortet, das sei in der Tat eine
gute Nachricht (denn sie hatte die Wahrheit angenommen), aber sie
habe es kaum glauben können, bis er ihr gesagt habe, wie ich ihn
getroffen und mit ihm von Hyde Park bis James Park geritten sei.


% \picinclude{./120-129/p_s124.jpg} 
Nach einiger Zeit gingen Edward Pyot\person{Pyot, Edward} 
und ich nach Whitehall,\ort{Whitehall} und als wir vor 
den Protektor kamen, war Dr. Owen,\person{Owen, Dr.} 
Vizekanzler von Oxford, bei ihm. Es trieb uns, 
Oliver Cromwell\person{Cromwell, Oliver} die
Not der Freunde vorzustellen.   wiesen ihn auf das Licht
Jesu Christi, das jeden, der in die Welt kommt, erleuchtet.
Er sagte, es sei ein natürliches Licht, aber wir bewiesen ihm das
Gegenteil und legten dar, wie es göttlich und geistig sei, da es
von Christus ausgehe, dem geistigen und himmlischen Menschen
und das eben das, was das \zitat{Leben in Christus} genannt werde,
das nenne man auch das \zitat{Licht in uns.}. Die Kraft des Herrn
ging auf in mir und trieb mich, ihn zu ermahnen, seine Krone
zu den Füßen Jesu niederzulegen.\footnote{Cromwell 
lehnte 1656\jahr{1656} den Königötitel ab.} Wiederholt redete ich mit
ihm in dieser Absicht. Zuletzt -- ich stand neben dem Tisch --
kam er und setzte sich auf die Tischecke neben mich, sagte, er wolle
so hoch sein wie ich und fuhr fort gegen das Licht Jesu Christi
zu sprechen und ging gleichgültig hinaus. Aber des Herren Macht
kam über ihn, so das, als er zu seiner Frau und zu andern
Leuten kam, er sagte: \zitat{nie habe ich sie in der Weise verlassen};
denn er war in sich selbst gerichtet [...].

\section{Begegnung mit einem alten Verfolger}

Als er fort war, trafen wir beim Hinausgehen mit vielen
angesehenen Leuten zusammen, und einer von ihnen fing an, uns
gegen das Licht und die Wahrheit zu reden, und ich verachtete
ihn deswegen.\person{Fox!Verchtung} Da sagte mir ein anderer, 
er sei der General-Major von 
Northamptonshire.\person{General-Major von Northamptonshire} 
\zitat{Was!}, sagte ich, \zitat{unser früherer
Verfolger, der so viele unsrer Freunde in die Gefangenschaft ge-
schickt hat und eine Schande ist für die Christenheit und die
Religion? Ich bin froh, das ich dich getroffen habe.} Und ich
redete nun ernstlich mit ihm über sein unchristliches Benehmen;
und er schlich hinweg, denn er hatte die Verfolgungen in 
Northhamptonshire\ort{Northhamptonshire} sehr grausam betrieben 
gehabt [...].

Von London ging ich nach Buckinghamshire,\ort{Buckinghamshire} 
[...] dann nach Huntingdonshire,\ort{Huntingdonshire} [...] 
Boston,\ort{Boston} [...] Edgehill,\ort{Edgehill} [...] 
Warwick\ort{Warwick} [...]
und wieder zurück nach London. Überall hatte ich mich beflissen,
das, was mir der Herr aufgetragen, zu erfüllen. Denn, nachdem
ich aus der Gefangenschaft in Launceston entlassen war, hatte mich
der Herr getrieben im Lande umher zu reisen, wo die Wahrheit
sich verbreitet und recht befestigt hatte, um noch allerlei 
% \picinclude{./120-129/p_s125.jpg} 
Einwände zu beseitigen, welche die bösen Priester und \textit{Frommen}
in den Gemütern gegen uns gepflanzt hatten [...]. Und in
dieser Absicht trieb es mich nun auch, allerlei Erklärungen ergehen
zu lassen, [...] unter anderm folgende: 

\grosszitat{
    Es wird den Quäkern
    oft vorgeworfen, das sie das sogenannte Sakrament von Brot und
    Wein bestreiten,\index{Sakrament}\index{Abendmahl} von dem es 
    heißt, man müsse es gebrauchen zum
    Gedächtnis Christi (Luc. 22:19\bibel{Luc. 22:19}) bis an der 
    Welt Ende. Wir
    hatten deswegen und wegen der verschiedenen Arten des 
    Sakraments-Gebrauchs im sogenannten Christentum viel Mühe mit den
    Priestern und \textit{Frommen}, denn manche nehmen es kniend,
    manche sitzend; aber keine von allen, die ich je gesehen, nehmen
    es, wie die Jünger es nahmen, nämlich in einem Zimmer nach
    dem Nachtessen, sondern die meisten nehmen es vor dem 
    Mittagessen und manche sagen, wenn der Priester Brot und Wein 
    gesegnet hat, \zitat{es ist der Leib Christi}. Christus aber 
    sagte nur, \zitat{tut es zu meinem Gedächtnis}. Er sagte 
    ihnen nicht, wie oft
    sie es tun müssten oder wie lang; auch gebot er ihnen nicht, es
    ihr Leben lang zu tun, noch das alle, die an ihn glauben, es
    tun sollten bis an der Welt Ende. Der Apostel Paulus, der erst
    nach Christi Tod bekehrt worden, sagt den Corinthern, er habe vom
    Herrn empfangen, was er ihnen in dieser Sache mitteile, und er
    führt Christi Worte in bezug aus den Kelch also an: \zitat{Dieses
    tut, so oft ihr trinket, zu meinem Gedächtnis}; und er selbst fügt
    bei: \zitat{denn so oft ihr das Brot esset und den Kelch trinket, so
    verkündet ihr des Herrn Tod, bis das er kommt}
    (1. Cor. 11:26\bibel{Cor. 1 11:26@1. Cor. 11:26}).
    Nach dem also, was der Apostel hier mitteilt, gebot weder Christus
    noch er, dies allezeit zu tun, sondern stellten es jedem frei [...].
    Die Juden pflegten einen Kelch zu gebrauchen und Brot zu
    brechen und an ihren Festen unter sich zu verteilen, wie man an
    den jüdischen Altertümern sieht; das Brechen des Brotes und das
    Trinken des Weines waren also jüdische Gebräuche, die nicht für
    immer zu bestehen brauchen. Sie tauften auch mit Wasser; 
    darum befremdete es sie nicht, als Johannes der Täufer auftrat
    mit seiner Wassertaufe [...]. Was aber Brot und Wein anbelangt, 
    so hatte Christus gesagt, das er das Brot des Lebens
    sei (Joh. 6:48\bibel{Joh. 06:48@Joh. 6:48}), das vom Himmel 
    kommt, und das er 
    kommen\index{Christi!Wiederkunft}\index{Wiederkunft Christi}
    wolle und in ihnen wohnen. Das betrachteten die Apostel nun
    als erfüllt und ermahnten die andern, nach dem zu trachten, das
    von oben kommt (Col. 3:2\bibel{Col. 03:02@Col. 3:2}). Ihr nun, 
    die ihr diesen äußern Wein
    % \picinclude{./120-129/p_s126.jpg} 
    trinket und dieses äußere Brot esset zum Gedächtnis des Todes
    Christi, kennet ihr gar nichts Besseres, um dem Tode Christi näher
    zu kommen? [...]


    Es muss freilich durch manchen Zustand hindurch gehen, ehe
    die Leute dazu gelangen, das, was von oben kommt, zu sehen
    und daran teilzunehmen. Zuerst kommt der Gebrauch des äusseren
    Brotes und Weines zum Gedächtnis Christi; das war zeitlich und
    nicht gezwungen, sondern freiwillig [...]. Zweitens kommt das
    Eingehen in seinen Tod, ein Leiden mit Christus, und das ist
    notwendig zum Heil und nicht zeitlich, sondern beständig; es muss
    ein tägliches Sterben sein. Drittens ein Begrabensein mit Christus;
    viertens ein Auferstehen mit Christus; fünftens nach 
    dem Auferstandensein mit Christus 
    (Röm. 6\bibel{Röm. 06@Röm. 6}) ein Trachten nach dem, was
    droben ist, ein Suchen nach dem Brote, das vom Himmel 
    herunter kommt, ein Essen davon und eine Gemeinschaft durch 
    dasselbe. [...] Die Gemeinschaft, die sich aus den Gebrauch von
    Brot, Wein, Wasser, Beschneidung, äußere Tempel und sichtbare
    Dinge gründet, wird ein Ende haben; die Gemeinschaft aber, die
    sich auf das Evangelium gründet, auf die Kraft Gottes, die war,
    ehe der Teufel\index{Teufel} gewesen, und die Leben und 
    unvergängliches Wesen
    ans Licht bringt, und durch welche die Leute über den Teufel
    sehen, der sie verfinstert, diese Gemeinschaft wird ewig 
    bestehen. [...]
}


Somit waren die Einwände dieser Priester und \textit{Frommen}
widerlegt [...] und die Wahrheit breitete sich in diesem Jahre
(1656\jahr{1656}) recht aus, und viele Tausende bekehrten sich 
zum Herrn,
so das selten weniger als tausend im Gefängnis waren um der
Wahrheit willen, etliche wegen des 
Zehntenwesens,\index{Kirchensteuer} etliche
weil sie ins Turmhaus gegangen waren, etliche wegen 
irgendwelcher sogenannten Missachtung, etliche wegen 
des Schwörend\index{Eid} oder weil sie ihre 
Hüte\index{Hut} nicht abgenommen [...].


Von London\ort{London} zogen wir wieder weiter im Lande umher [...]
nach Farnham [...] Bafmgstoke [...] Exeter [...] Bristol, [...] dann
nach Brecknock (Wales) und dann [...] wieder nach England 
zurück, nach Shrewsburh [...].

\section{Über das Fasten und dürre als strafe Gottes}

Es war um die Zeit eine große Trockenheit im Lande [...].
Als nun Oliver Cromwell\person{Cromwell, Oliver} ein 
Fasten\index{Fasten} proklamierte um Regen,
nahm man wahr, das im Norden, soweit sich die Wahrheit 
ausgebreitet hatte, erquickende Niederschläge waren, während sie im
% \picinclude{./120-129/p_s127.jpg} 
Süden vielerorts schier umkamen aus Mangel an Regen. Da
trieb es mich, eine Erwiderung auf die Proklamation des 
Protektors zu schreiben, worin ich ihm sagte, wenn er sich Gottes
Wahrheit zugewendet hätte, so hätte er Regen gehabt, die 
Trockenheit sei ein Zeichen für ihre Dürre und ihren Mangel an Wasser
des Lebens\index{Wasser des Lebens} [...].


Wir gingen wieder nach Wales\ort{Wales} und hatten mehrere 
Versammlungen, bis wir nach Tenby\ort{Tenby} kamen. Auf der Straße kam
mir ein Friedensrichter entgegen und forderte mich auf, zu ihm
in seine Wohnung zu kommen, was ich denn auch tat. Am
Ersten Tage kam der Stadtmajor und einige der Häupter der
Stadt und blieben während der ganzen Zeit der Versammlung.
John-ap-John\footnote{Anmerkung: Es ist nicht klar, ob sich 
hier bei den Namen um ein Fehler handelt.}\person{John-ap-John} 
aber verließ sie und ging ins Turmhaus, wo ihn
der Gouverneur gefangen nehmen lies. Es war eine herrliche
Versammlung. 

Am Morgen des zweiten Tages schickte der 
Gouverneur einen seiner Leute ins Haus des Friedensrichters, um mich
holen zu lassen, was dem Major und dem Friedensrichter, die
beide bei mir waren, sehr Leid tat. Sie gingen darum gleich
voraus zum Gouverneur, und nach einiger Zeit kam ich ihnen nach
mit dem Beamten. Als ich eintrat, sagte ich: \zitat{Friede sei diesem
Hause.} Und ehe noch der Gouverneur mich etwas fragen konnte,
fragte ich ihn, warum er meinen Freund John-ap-John ins 
Gefängnis getan habe? \zitat{Weil er seinen Hut in der Kirche 
auf behielt},\index{Hut!abnehmen} erwiderte der Gouverneur. 
Ich sagte darauf: \zitat{Hatte nicht
der Priester zwei Kappen auf dem Kopf. eine weise und eine
schwarze? Schneide meinem Freund den Rand an seinem Hut
weg, und dann hätte er nur eine solche Kappe, und der Rand
ist nur, um ihn vor dem Regen zu schützen.} \zitat{Das sind dumme
Sachen}, sagte der Gouverneur. \zitat{Warum wirfst du dann meinen
Freund ins Gefängnis wegen dummer Sachen?} sagte ich. 

\section{Erwähnung und Verwerfung}

Er fragte mich nun, ob ich die Erwähnung und Verwerfung annehme?
\zitat{Ja}, sagte ich, \zitat{und du bist in der Verwerfung}.
\index{Verwerfung}\index{Erwähnung}\index{Gnade} Das machte
ihn bös und er drohte mir, er wolle mich ins Gefängnis werfen,
bis ich es ihm beweise; ich sagte, ich könne ihm das gleich beweisen,
wenn er der Wahrheit Gehör schenke. \index{Verwerfung!Beweis für} 
Nun fragte ich ihn, ob denn
Wut und Zorn nicht Zeichen der Verwerfung seien? denn der,
welcher aus dem Fleisch geboren sei, verfolge den, der aus dem
Geist geboren sei. Christus und seine Jünger haben nie jemand
verfolgt oder gefangen genommen. Daraufhin bekannte er offen,
% \picinclude{./120-129/p_s128.jpg} 
das er zu leidenschaftlich und zornmütig sei. Ich sagte ihm, er
sei wie Esau, der Erstgeborene, und nicht wie Jakob [...]. Die
Kraft des Herrn kam so mächtig über ihn, das er sich zur 
Wahrheit bekannte, und der Friedensrichter kam und schüttelte mir
die Hand [...] Als ich weiter zog, trieb es mich, noch einmal
mit dem Gouverneur zu reden und er lud mich zum Essen ein
und gab meinem Freund die Freiheit. [...] 


Wir gingen nach
England zurück [...] nach Liverpool\ort{Liverpool} 8...] 
Manchester\ort{Manchester} [...] Lancaster\ort{Lancaster} 
[...] nach Swarthmore,\ort{Swarthmore} wo die Freunde 
sich sehr freuten,
mich wieder zu sehen. Ich blieb während zwei Ersten Tagen
dort und ging in mehrere Versammlungen. Die Freunde freuten
sich mit mir der Güte Gottes, die mir durch so viele Gefahren
hindurch geholfen. Ihm sei Preis ewiglich.
