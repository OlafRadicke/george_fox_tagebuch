
%%%%%%%%%%%%%%%%%%% Kapitel 24. %%%%%%%%%%%%%%%%%%%%%%%%%%%%%%

\chapter[Mahn- und Trostschreiben]{Mahn- und Trostschreiben}

\begin{center}
\textbf{Allerlei Mahn- und Trostschreiben.}
\end{center}


Ungefähr zur gleichen Zeit erhielt ich zwei sehr gehässige
Bücher, die gegen die Wahrheit und die Freunde gerichtet waren;
das eine war von einem sogenannten Doktor aus Bremen\ort{Bremen} in
Deutschland, das andere von einem Priester aus Danzig\ort{Danzig}. Beide
waren Voll arger Falschheit und verleumderischer Vorwürfe\index{Verleumdung}. Es
kam über mich, auf beide zu antworten, und um nicht durch
andere Geschäfte und Besuche gestört zu werden, ging ich nach
Kingston\ort{Kingston} an der Themse\ort{Themse}, wo ich eine Antwort auf jedes der
Bücher schrieb, sowie aus einige andere gehässige Schriften, die
geschrieben und verbreitet worden waren, um die Freunde falsch
darzustellen [...].

Die Sheriff für die Stadt sollten neu gewählt werden und
die, welche für die Wahl vorgeschlagen waren, wünschten die
Stimmen der Freunde zu erhalten; da schrieb ich einige Zeilen,
um zu erfahren, wessen Geistes sie wären, und wie sie sich zu der
wahren Freiheit stellten. Ich tat es in Form einer Frage 
folgendermaßen:


% \picinclude{./280-289/p_s282.jpg} 
\brief{Sheriff}{
    Gibt irgend einer, der hier in London möchte zum Sheriff
    gewählt werden, zu, das Christus, der vor den Toren Jerusalems
    gekreuzigt wurde, \zitat{das Licht der Welt ist, das jeden, der in die
    Welt kommt, erleuchtet} und sagt, \zitat{glaubet an das Licht, aus das
    ihr Kinder des Lichts seid} (Joh. 12,36\bibel{Joh. 12:36})? Widersetzt sich einer,
    das man die Leute verfolge um der Religion willen, und darum, das
    sie Gottes Gebot halten und ihn im Geist und in der Wahrheit
    anbeten? Denn Christus sagt: \zitat{Ich bin nicht von dieser Welt}
    (Joh.17\bibel{Joh.17}), noch ist \zitat{mein Reich von 
    dieser Welt} (Joh. 18\bibel{Joh.18}), darum
    hält er seine Religion nicht mit weltlichen Waffen aufrecht. Christus
    sagte: \zitat{Ihr sollt überhaupt nicht schwören}, und sein Apostel
    Jakobus sagt dasselbe, und nun wollet ihr uns zwingen, zu
    schwören und somit die Gebote Christi und seiner Apostel zu
    brechen, indem ihr uns Eide vorlegt? Christus sagt zu seinen
    Aposteln: \zitat{Umsonst habt ihr es empfangen, umsonst gebt es
    auch} (Matth. 10,8\bibel{Matth. 10:08@Matth. 10:8}). Werdet 
    ihr uns nicht zwingen, Zehnten
    und Abgaben zu zahlen an solche Lehrer, von denen wir wissen,
    das Gott sie nicht gesandt hat? Werden wir frei sein, Gott
    zu dienen und ihn anzubeten und seine und seines Sohnes Gebote
    zu halten, wenn wir euch freiwillig unsre Stimmen geben? Denn
    wir sind nicht willens, unsre Stimmen solchen zu geben, die uns
    gefangen nehmen und uns unsre Habe nehmen [...].

}

Ich schrieb auch, während ich in London war, so oft ich
zwischen den Versammlungen und andern öffentlichen 
Gottesdiensten\index{Versammlung!öffentliche} Zeit hatte, 
verschiedene Bücher und Schriften, von denen
einige gedruckt und andere im Manuskript verbreitet wurden.
Eines davon richtete sich an die Bischöfe und andere, welche 
Verfolgungen anzetteln, und bewies ihnen aus der heiligen Schrift,
das sie nicht nach derselben wandelten und nach dem \zitat{königlichen
Gesetz, das gebietet, seinen Nächsten zu lieben wie sich selbst}
(Jak. 2,8\bibel{Jak. 02:08@Jak. 2:8}), und andern zu tun, wie man möchte, das andere uns
tun. Eine andere Schrift war: \buchtitel{An die Menge derer, Protestanten
wie Papisten\index{Papisten}, die sich für Christen ausgeben, deren Gottesdienst
und Religion aber in äußern Formeln und Zeremonien besteht}
Ich wies sie mit Nachdruck auf die Worte des Apostels Paulus
hin, Galater 5,2-4\bibel{Galater 05:02@Galater 5:2-4}: \zitat{Ich, 
Paulus, sage euch, wo ihr euch beschneiden lässt, so ist euch 
Christus kein nütze. Ich sage aber
jedem, der sich beschneiden lässt, das er noch das ganze Gesetz
schuldig ist. Ihr habt Christus verloren, die ihr noch durch das
% \picinclude{./280-289/p_s283.jpg} 
Gesetz gerecht werden wollt und seid aus der Gnade gefallen}.
Eine andere Schrift war: \zitat{Jeder wende sich dem Geiste Gottes
zu, um durch denselben ein rechtes Verständnis zu bekommen und
fähig zu werden, zwischen Recht und Unrecht zu unterscheiden,
zwischen Wahrheit und Irrtum und nicht unter dem Vorwand,
Übeltäter zu bestrafen, sich selber Übles zu tun, indem man den
Gerechten verfolgt}\index{Glaubenskrieg}\index{Verfolgung} \buchtitel{Jeder 
wende sich dem Geiste Gottes
zu, um durch denselben ein rechtes Verständnis zu bekommen und
fähig zu werden, zwischen Recht und Unrecht zu unterscheiden,
zwischen Wahrheit und Irrtum und nicht unter dem Vorwand,
Übeltäter zu bestrafen, sich selber Übles zu tun, indem man den
Gerechten verfolgt}[...].

Eine andere Schrift schrieb ich über \buchtitel{Betrachtung, Ergötzen,
Übung, Streben und Forschen}; ich zeigte aus der Schrift der
Wahrheit, worüber die wahren Christen nachdenken sollten und
worin ihren Sinn üben, was ihr Ergötzen sein sollte, und was sie sich
zu tun bestreben sollten. Denn in diesen Dingen sind nicht nur
die Weltlichen und die leichtfertigen Leute in großem Irrtum,
sondern auch die großen \textit{Frommen}, sie freuen sich über die 
irdischen, vergnüglichen Dinge, während sie über himmlische Dinge
nachdenken und sich am Gesetz Gottes ergötzen und sich bestreben
sollten, immer ein reines Gewissen gegen Gott und Menschen zu
haben, gleich dem Apostel« [...].

Da die Leiden immer noch schwer und drückend auf den
Freunden lasteten, nicht nur in der Stadt, sondern auch in fast
allen Gegenden des Landes, setzte ich ein Schreiben auf, das dem
König eingereicht werden sollte. Ich brachte darin unsre 
Bekümmernisse vor und bat für die Fälle, die mir in seiner Macht
zu stehen schienen, um Abhilfe. Da ich aber keine Hilfe von
ihm erhielt, kam es über mich, einen Brief an die Freunde zu
schreiben, um sie in ihren Leiden zu ermutigen\index{Ermunterung}, 
damit sie in Geduld die vielen Prüfungen ertragen möchten, die über sie 
gebracht wurden, sowohl von Seiten der Behörden, als auch von
falschen Brüdern und Abtrünnigen\index{Abtrünnige}, deren böse 
Bücher und gemeine Verleumdungen die Rechtschaffenen betrübten [...].

Ich blieb den größten Teil des Winters in London im Dienst
der Wahrheit unter den Freunden, ausgenommen eine kurze Zeit,
die ich in Kingston zubrachte, im 10. Monat des Jahres, wo ich
ein Buch schrieb über: 
\buchtitel{Das Wesen der zeitlichen Geburt und
das der geistigen}, worin ich die Pflicht und Stellung eines Kindes,
Jünglings, Erwachsenen und Greises der Wahrheit gegenüber darlete [...].

An einem Ersten Tage kam es über mich, am Nachmittage
in die Versammlung in Devonshire House\ort{Devonshire House} zu gehen, 
und da ich
% \picinclude{./280-289/p_s284.jpg} 
erfuhr, das die Freunde dort am Morgen nicht eingelassen worden
waren, wie es an dem Tage bei den meisten Versammlungen in
der Stadt geschehen war, ging ich früher hin und ging in den
Hof, ehe die Soldaten kamen, um die Eingänge zu bewachen,
aber die Konstabler waren vor mir dort und standen im Torweg
mit ihren Stäben\index{Versammlung!trotz Verbot}. Ich bat 
sie, mich hinein zu lassen; sie sagten,
sie können und dürfen es nicht, man habe ihnen das Gegenteil
geboten, es tue ihnen Leid. Ich sagte, ich wolle nicht in sie
dringen; so stand ich neben ihnen und sie waren sehr höflich.
Ich stand, bis ich müde war, und dann gab mir einer einen
Stuhl, um mich zu setzen, und nach einer Weile fing die Kraft
des Herrn an, unter den Freunden lebendig zu werden, und
einer fing an zu reden. Die Konstabler verboten es gleich und
sagten, er solle nicht sprechen; und da er nicht aufhörte, wurden
sie zornig. Da legte ich sanft meine Hand. auf die des einen 
Konstablers und bat ihn, den Mann ruhig zu lassen, der Konstabler tat
es und war still, und der Mann redete nicht lang. Nachdem er
geendet, trieb es mich, aufzustehen und zu reden, und in meiner
Verkündigung sagte ich, sie brauchen nicht gegen uns vorzugehen
mit Stäben und Schwertern, denn wir seien friedliche Leute und
es sei nichts als Wohlwollen in unsern Herzen gegen den König
und die Behörden und gegen alle Menschen auf der ganzen Welt.
Wir gebrauchen die Religion nicht als Vorwand, um uns zu
Verschwörungen und Bündnissen oder Aufständen 
zu versammeln,\index{Religion und Gewalt}
sondern wir versammeln uns, um Gott im Geist und in der
Wahrheit anzubeten. Wir haben Christus zum Bischof, Priester
und Hirten (1. Petr. 2\bibel{Petr. 1. 02@1. Petr. 2}), 
er erlabt und leitet unsre Seelen, darum
können wir alle hier stille sitzen und unsres Lehrers- genießen und
uns seiner Lehre freuen, und ich befahl sie alle Christus, ihrem
Bischof und Hirten. Darauf setzte ich mich nieder, und nach einer
Weile trieb es mich, zu beten\index{Gebet}, und die Kraft des Herrn war über
allen, und die Versammelten, die Soldaten und die Konstabler,
nahmen ihre Hüte ab. Als die Versammlung zu Ende war, und
die Freunde anfingen hinaus zu gehen, nahm der Konstabler,
seinen Hut ab und bat den Herrn, das er uns segne, denn die
Kraft des Herrn war über ihm und allen andern und überwältigte sie [...].

Im ersten Monat des Jahres 1683 ging ich nach Kingston\ort{Kingston}
an der Themse [...]. Daraus nach Guildsord in Snrrey, und
% \picinclude{./280-289/p_s285.jpg} 
nachdem ich die Freunde dort besucht hatte, weiter nach 
Worminghurst\ort{Worminghurst} in Sussex\ort{Sussex}, wo ich 
eine sehr gesegnete Versammlung
mit den Freunden hatte, ohne jegliche Störung. Während ich
dort war, wurde James Claypole\person{Claypole, James} aus London, der mit seiner
Frau auch dort war, plötzlich krank\index{Krankheit}; es war ein so heftiger Anfall,
das er weder stehen noch liegen konnte und vor heftigen Schmerzen
schrie. Als ich es hörte, wurde ich sehr betrübt im Geist um ihn
und ging zu ihm. Nachdem ich einige Worte mit ihm gesprochen
hatte, um seinen Sinn nach innen zu richten, trieb es mich, ihm
meine Hand aufzulegen\index{Handauflegen}, und ich bat den Herrn, seine Krankheit
von ihm zu nehmen; während ich ihm meine Hand auslegte, kam
die Kraft des Herrn über ihn, und durch den Glauben an diese
Kraft wurde es ihm gleich leichter, und er fiel in einen Schlaf.
Als er erwachte, war er so wohl, das er am nächsten Tage
25 Meilen mit mir in einem Wagen fuhr, während er früher,
wie er mir sagte, gewöhnlich zwei Wochen, manchmal einen
Monat an einem solchen Anfall darniederlag. Aber der Herr
war für ihn angerufen worden und schenkte ihm durch seine Kraft
diesmal schnelle Besserung; sein heiliger Name sei dafür gelobt
und gepriesen [...]\index{Heilung}.
