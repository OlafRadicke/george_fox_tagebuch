

%%%%%%%%%%%%%%%%%%% Kapitel 18. %%%%%%%%%%%%%%%%%%%%%%%%%%%%%%

\chapter[Reise nach Amerika, Barbadoes und Jamaika.]{Reise nach Amerika, Barbadoes und Jamaika.}

\begin{center}
\textbf{Reise nach Amerika, Barbadoes und Jamaika.}
\end{center}


Wie schon erwähnt, hatte ich zwei Töchter meiner Frau
zum König geschickt, um ihre Freisprechung zu erwirken, und sie
hatten auch seinen diesbezüglichen Befehl dem Befehlshaber in
Laneashire gebracht; [...] aber der Sturm der Verfolgung war
gerade so mächtig geworden, das man Mittel fand, sie weiter
gefangen zu halten. Als nun aber die Verfolgungen etwas
nachließen, trieb es mich, Martha 
Fischer\person{Fischer, Martha} und eine andere Frau
aus dem Kreise der Freunde zu veranlassen, abermals zum König
zu gehen, um ihre Freilassung zu erbitten. Sie gingen im Glauben
an die Kraft des Herrn, welcher sie Gnade finden ließ vor dem
König, so das er einen besiegelten Freilassungsbefehl bewilligte,
nachdem sie fast zehn Jahre gefangen gewesen war, und ihre Gitter
mit Beschlag belegt, dergleichen kaum je in England war erhört
worden. Ich schickte die Freisprechang sofort zu ihr durch einen
% \picinclude{./210-219/p_s216.jpg} 
Freund, und zugleich schrieb ich ihr, wie sie den Frestassungsbefehl
müsse der Richter zukommen lassen und teilte ihr auch
mit, daß es über gekommen sei vom Herrn, übers Meer zu
gehen nach Amerika; sie möge darum, sobald es ihr möglich sei,
nach London eilen, da das Schiff sich schon zur Abreise rüste.
In der Zwischenzeit ging ich nach Kigston zu John Rous, bis
meine Frau kam, und dann rüstete ich mich zur Reise. Doch
weil die Jahresversammlung\index{Jahresversammlung} bald stattfand, so blieb ich noch bis
zu derselben [...] Dann, als unsera Schiff und die Freunde, die,
mich zu begleiten beabsichtigten bereit waren, ging ich, am 12. des
6. Monates 1671\index{Jahr!1671} nach Gravesend, und meine Frau und mehrere der
Freunde begleiteten mich über die Downs. Die Freunde,
die die Reise mit mir machten waren: 
\begin{itemize}
 \item Thomas Briggs,\person{Briggs, Thomas}
 \item William Edmumdson,\person{Edmumdson, William} 
 \item John Stubbs,\person{Stubbs, John} 
 \item Salomon Eccles,\person{Eccles, Salomon} 
 \item James Lancaster,\person{Lancaster, James} 
 \item John Cartwright,\person{Cartwright, John} 
 \item Robert Widders,\person{Widders, Robert} 
 \item George Pattison,\person{Pattison, George}
 \item John Hull,\person{Hull, John} 
 \item Elisabeth Hooton und\person{Hooton, Elisabeth} 
 \item Elisabeth Miers.\person{Miers, Elisabeth}
\end{itemize}

Unser Schiff eine Jacht und hieß "`Industrie"', der Kapitän hieß Thomas
Forster, und wir waren etwa 50 Passa [...]

Als wir etwa drei Wochen auf dem Wasser waren, bemerkten
wir etwa vier Seemeilen hinter uns ein Schiff. Unser Kapitän
sagte, es sein maurisches Piratenschiff, das uns verfolgen
scheine. "`komt"' sagte er "`wir wollen zum Abendessen gehen,
wenn es dunkel geworden ist, sie unsere Spur verlieren."'
Dies sagte er, um die Reisenden zu beruhigen, denn es fingen
schon einige an sich zu ängstigen. Die Freund jedoch waren
guten Mutes, weil sie Gott vertrauten, und keinerlei Furcht ihr
Seele bedrücke. Als die Sonne untergegangen war, sah ich von
meiner Kajüte aus, wie das Schiff auf uns zukam. Als es
dunkel wurde, änderten wir die Richtung um ihm auszuweichen;
aber es änderte die seine auch, In der Nacht
kamen der Kapitän und andere zu mir in Kajüte und fragten
mich, was sie tun sollten. Ich antwortete,  kein Schiffsmann
und fragte sie, Was sie für das Beste hielten? Sie sagten,
Es gäbe nur zwei Wege: entweder wir müssten das Schiff überholen
oder hin und herkreuzen und die gleiche Richtung einhalten
wie vorher. Ich fragte, wenn es Räuber seien, so werden sie
sicherlich auch hin- und herkrezen und was das Überholen anbelangt,
so sei daran gar nicht zu denken, da man ja sehe wie
viel schneller fahren als wir. Sie fragten mich wieder, was
% \picinclude{./210-219/p_s217.jpg} 
sie denn tun sollten: \zitat{denn,} sagten sie, \zitat{wenn die Schiffsleute 
damals den Rat des Paulus befolgt hätten, so wäre es ihnen
nicht so schlimm ergangen.} Ich erwiderte: \zitat{Ee ist eine 
Glaubensprüfung, und darum muss man auf den Herrn warten und auf
seinen Rat.} Während ich mich nun innerlich sammelte, zeigte
mir der Herr, das er mit seinem Leben und mit seiner Kraft
zwischen uns und dem Schiff, das uns verfolgte, stehe. Ich
teilte dies dem Kapitän und den anderen mit, und das es nun
das Beste sei, zu kreuzen und den rechten Kurz einzuschlagen.
Ich hieß sie auch alle Lichter auslöschen außer dem einen, das
sie beim Steuer brauchten, und den Reisenden sagen, sie sollten
sich still und ruhig verhalten. In der Nacht etwa um 11 Uhr,
kam die Wache und sagte, sie seien ganz nahe hinter uns. Das
beunruhigte einige der Reisenden. Ich richtete mich in meiner
Kajüte auf, und da der Mond noch nicht untergegangen war. sah
ich durch die Luke, das sie ganz nahe waren. Ich wollte aufstehen 
und hinausgehen; aber ich erinnerte mich der Worte des
Herrn, \zitat{das er mit seinem Leben und seiner Kraft zwischen uns
und ihnen stehe,} und legte mich wieder nieder. Der Kapitän
und einige der Schiffsleute kamen abermals und fragten, ob sie
nicht nach dieser oder jener Richtung steuern sollten? Ich sagte
ihnen, sie sollten machen, wie sie wollten. Da ging der Mond
vollends unter, ein neuer Wind erhob sich, und der Herr verbarg
uns vor ihnen; wir segelten rasch und sahen sie nicht mehr. Am
folgenden Tag, einem Ersten Tag, hatten wir eine öffentliche
Versammlung auf dem Schiffe, wie wir sie gewöhnlich während
der ganzen Reise an diesem Tage zu halten pflegten; und des
Herrn Gegenwart war mächtig unter uns. Und ich ermahnte die
Leute, an Gottes Barmherzigkeit zu denken, die sie errettet; denn
sie wären jetzt vielleicht alle in den Händen der Türken,\index{Türken} wenn
des Herrn Hand sie nicht errettet hätte. Etwa eine Woche
darauf suchten der Kapitän und einige der Schiffsleute den
Reisenden einzureden, es seien nicht türkische Seeräuber\index{Seeräuber} gewesen,
die uns verfolgten, sondern ein Kaufmannssschiff, das nach den
Kanarischen Inseln ging. Als ich das hörte, fragte ich sie,
warum sie denn dann solches zu mir gesagt hätten? warum sie
die Reisenden beunruhigt hätten? und warum sie, um ihnen
davon zu fahren, den Kurs geändert hätten? Sie sollten sich
hüten, Gottes Barmherzigkeit zu verachten. Später, als wir in
% \picinclude{./210-219/p_s218.jpg} 
Barbados waren, kam ein maurischer Kaufmann und erzählte den
Leuten, die Mannschaft eines maurischen Piratenschiffs habe auf
dem Meer ein ungeheures Jachtschiff gesehen, das größte, das
sie je gesehen hätten, sie hätten es verfolgt, und seien schon ganz
nahe gewesen, aber es sei ein Geist darin gewesen, so das sie es
nicht erobern konnten. Dies bestätigte uns in unserer Überzeugung,
das es ein maurisches Piratenschiff war, das uns verfolgte, und
das es der Herr gewesen, der uns befreit hatte.

Ich war nicht seekrank gewesen auf der Reise, wie so viele
der Freunde und andere Reisende; aber alle die Wunden und
Schläge, die ich früher erlitten, die Krankheiten, die ich mir durch
die Kälte und die Entbehrungen während meiner Gefangenschaften
zugezogen hatte, machten sich nun während der Reise wieder
geltend, so das mein Magen sehr angegriffen war, und ich heftige
Schmerzen in allen Gliedern hatte. Es fing an, nachdem ich
etwa einen Monat auf der See war; zuerst schwitzte ich stark,\index{Erkrankung}
und an Kopf und Leib zeigten sich überall Pusteln,\index{Pusteln} und meine
Hände und Füße wurden so geschwollen, das ich nur mit Mühe
und unter großen Schmerzen meine Strümpfe und Pantoffeln
anziehen konnte; auf einmal hörte das Schwitzen auf, und
als ich in das heiße Klima kam, wo die anderen tüchtig schwitzten,
konnte ich gar nicht schwitzen, sondern mein Körper war heiß
und trocken und brennend, und was vorher in Pusteln 
ausgebrochen war, schlug jetzt nach innen auf Herz und Magen, so
das ich sehr krank war und über alle Maßen schwach; dies
dauerte während der ganzen übrigen Zeit der Reise, während
der etwa vier Wochen, die wir noch auf dem Wasser waren.
Am frühen Morgen des 3. Tages des 8. Monats erblickten wir
die Insel Barbadoes,\ort{Barbadoes} aber es dauerte noch bis zwischen neun und
zehn des Abends, ehe wir in den Hafen der Carlisle-Bay 
einfuhren. Wir gingen sobald wie möglich ans Land, und ich 
begab mich mit einigen Freunden in das Haus eines Freundes,
eines Kaufmanns namens Richard Forstall,\person{Forstall, Richard} der etwa zehn
Minuten von der Landungsbrücke wohnte. Aber ich war so
krank und schwach, das ich sehr müde wurde von diesem kurzen
Gang, und Vollständig erschöpft ankam. Ich lag dort mehrere
Tage krank, und obgleich man mir mehrmals Mittel gab, um
mich schwitzen zu machen, so kam es doch nie zu einem rechten
Schweiß. Was sie mir gaben, vertrocknete eher meinen Körper
% \picinclude{./210-219/p_s219.jpg} 
noch mehr, und machte mich noch kränker, als ich sonst gewesen
wäre. Diese Schmerzen in allen Gliedern dauerten etwa drei
Wochen, und ich litt sehr, so das ich kaum je Ruhe finden
konnte, aber ich war ziemlich getrost und der Geist ward Herr
über alles. Auch hinderte mich meine Krankheit nicht am Dienst
für die Wahrheit, sondern sowohl auf der See als in Barbadoeß,
ehe ich herum reisen konnte, gab ich verschiedene Schriften
heraus, die ein Freund für mich schrieb, und von denen ich einige
mit der ersten Gelegenheit nach England schickte, um gedruckt
zu werden [...]

Weil ich so schwach war, das ich nicht an die verschiedenen
Versammlungen reisen konnte, nahmen sich die andern Freunde
des Werkes des Herrn an; schon am Tage nach unserer Ankunft
hatten sie eine große Versammlung an der Landungsbrücke, und
nach derselben noch mehrere in verschiedenen Teilen der Insel,
maß die Bevölkerung sehr in Aufregung brachte, so das viele zu
den Versammlungen kamen, worunter mehrere von hohem Rang;
denn sie hatten gehört, das ich auf der Insel angekommen sei
und erwarteten, mich bei den Versammlungen zu sehen, da sie
nicht wussten, das ich zu schwach war, um zu kommen. Meine
Schwachheit wich darum so lange nicht von mir, weil mein 
Gemüt zuerst sehr niedergedrückt\index{Nidergeschlagenheit} 
war von der Schmutzigkeit und
Ungerechtigkeit und Gemeinheit der Leute, maß wie eine schwere
Last auf mir lag. Aber nachdem ich etwa einen Monat auf der
Insel gewesen war, wurde es mir etwas leichter zu Mut, und
ich fühlte mich wieder etwas kräftiger, so das ich wieder umher
gehen konnte zu den Freunden [...]

Weil ich aber doch nicht gut viel umher reisen konnte, so
kamen die Freunde auf der Insel überein, die Männer- und
Frauen-Versammlungen zur Ordnung der kirchlichen Angelegenheiten 
im Hause Thomas Rous, bei dem ich wohnte, abzuhalten,
so das ich bei allen Versammlungen dabei war und recht für den
Herrn wirken konnte. Denn sie hatten in manchen Dingen Belehrung\index{Belehrung}
nötig, weil sich aus Mangel an Vorsicht und Wachsamkeit 
allerlei Unordnungen eingeschlichen hatten. Ich ermahnte sie,
besonders in der Männerversammlung, recht vorsichtig und 
wachsam in Bezug auf das Heiraten\index{Heirat} zu sein und die Freunde zu
verhindern, in die Verwandtschaft zu heiraten, sowie auch zu
hastig vorzugehen bei Wiederverheiratung nach dem Tode des

% \picinclude{./220-229/p_s220.jpg} 
Mannes oder der Frau. Ich ermahnte sie, das in solchen Fällen
dem verstorbenen Teile die geziemende Ehrerbietung sollte bezeugt
werden. Ich wies sie auch daraus hin, wie unziemlich es sei,
ihre Kinder so früh einander zu verheiraten, mit dreizehn und
vierzehn Jahren, und was für Schäden und Nachteile aus solchen
frühen Heiraten entstehen. Ich ermahnte sie ferner, ihre 
Fußböden gründlich zu reinigen,\index{Reinlichkeit} ihre Häuser rein zu halten und auch
außerhalb der Versammlungen einander nicht mit verleumderischen\index{Verleumdung}
Reden zu schaden. Ich ermahnte sie, genaue Verzeichnisse 
zu führen über Geburten, Heiraten und 
Beerdigungen,\index{Gemeindebuch}\index{Buch über Geburten, Heiraten und Beerdigungen}
in eigens dazu bestimmten Büchern; auch sollten sie ein 
besonderes Buch führen über die Bestrafungen\index{Buch über Bestrafungen}
solcher, die von der Wahrheit abweichen und einen unordentlichen Wandel führen,
und über Buße und Wiederaufnahme solcher, die wieder zurück
kommen. Ich empfahl ihnen an, für geeignete Begräbnisplätze\index{Begräbnis}
zu sorgen, die an etlichen Orten noch fehlten. Ich gab ihnen
auch einige Räte inbetreff der Vermächtnisse,\index{Erbschaft} welche Freunde zu
beliebigem Gebrauch hinterlassen hatten, und wie sie darüber
verfügen sollten, und über allerlei andere kirchliche Angelegenheiten.
Inbetreff der Schwarzen oder Neger\footnote{Das Wort \zitat{Neger} wurde
im Originaltext verwendet. Würde die Übersetzerin und der Autor vermutlich 
Heute nicht mehr benutzen},\index{Sklaverei} hieß ich versuchen, 
dieselben in der Furcht Gottes zu unterweisen, sowohl die gekauften
als die, welche in der Familie geboren wurden, damit alle dazu
kommen möchten, den Herrn zu kennen, so das jeder Hausvater
mit Josua sagen könne: \zitat{ich aber und mein Haus wollen dem
Herrn dienen}. Ich ermahnte sie auch ihre Aufseher dazu zu
bringen, mild und freundlich gegen die Neger zu sein, und sie
nicht grausam zu behandeln, wie viele es taten und noch tun,
und sie, wenn sie einige Jahre als Sklaven gedient, frei zulassen
Viele köstliche, herrliche Dinge wurden in diesen Versammlungen
offenbar, durch den Geist und die Kraft Gottes, zur Erbauung,
Ausrichtung und Stärkung der Freunde im Glauben und der
Heiligen Ordnung des Evangeliums [...]

Als es mir wieder besser ging, machten wir dem 
Gouverneur\person{Gouverneur von Barbados}
einen Besuch. Er empfing uns sehr höflich und behandelte 
uns sehr freundlich und hieß uns mit ihm zu Mittag essen. In der
gleichen Woche ging ich nach Bridgetown\ort{Bridgetown}; und da die Behörden,
die militärischen wie die andern, von meinem Besuch beim Gouverneur 
und seiner freundlichen Aufnahme gehört hatten, so kamen
aus allen Teilen des Landes viele Leute von hohem Rang,
% \picinclude{./220-229/p_s221.jpg} 
Richter, Friedensrichter, Oberste, Hauptleute zu dieser 
Versammlung [...]\index{Prominente Versammlungsbesucher} 

Von den Freunden, die mit mir gekommen waren,
gingen viele nach Jamaika und andere Orte, so das wenige mit
mir in Barbados blieben. Wir hatten viele große und schöne
Versammlungen [...] sie waren friedlich und nicht gestört von
Seiten der Regierung; jedoch gehässige Priester\index{Andachsstörung} 
[...] und Baptisten\index{Baptisten} [...] und Fromme brachten 
Schmähschriften\index{Schmähschriften} gegen uns [...]
diesen traten wir mit einer Schrift entgegen,\index{Verteidigungsschriften} 
die im Namen der sogenannten Quäker sollte verbreitet werden, um die Wahrheit
und die Freunde von solchen falschen Anschuldigungen zu reinigen.
[...] Es hieß darin unter anderem: [...] 

\brief{Verfolger}{ 
  eine Verleumdung,
  die sie gegen uns ausstreuten ist, das wir die Neger zu Aufständen\index{Sklavenbefreihung}
  anstiften, und gerade das Verabscheuen wir im Innersten; der
  Herr, der die Herzen prüft, weiß es, und kann uns das Zeugnis
  geben, das dies eine ganz abscheuliche Unwahrheit ist.  Wir haben
  in Bezug auf sie gesagt, man solle sie lehren, nüchtern und 
  rechtschaffen zu sein, Gott zu fürchten und ihre Herren und Herrinnen
  zu lieben und treu und fleißig ihren Herren zu dienen; dann
  würden ihre Herren und ihre Aufseher sie lieben und gütig und
  freundlich behandeln; auch sollten sie ihre Weiber nicht schlagen
  noch die Weiber ihre Männer, und die Männer sollten nicht
  mehrere Weiber haben; sie sollten nicht stehlen noch sich betrinken,
  nicht Ehebruch noch Unzucht treiben, nicht fluchen, nicht schwören,
  nicht lügen oder sich unter einander beschimpfen, denn es sei
  etwas in ihnen, das; ihnen sage, sie sollten diese und andere
  schlechte Dinge nicht tun. Wenn sie sie aber dennoch tun, so
  lehrten wir sie, das es nur zwei Wege\index{Dualismus} gibt, der eine, der zum
  Himmel führt, den die Gerechten gehen, und der andere, der
  zur Hölle führt, den die Gottlosen gehen und die Ehebrecher,
  Hurer, Mörder und Lügner. Zu den Einen wird der Herr sagen:
  Kommet, ihr Gerechten meines Vaters, erbet das Reich!\index{Reich Gottes} Zu
  den Andern aber wird er sagen: Gehet hin, ihr Verfluchten in
  das ewige Feuer! und so werden die Ungerechten in die ewige
  Pein gehen, die Gerechten aber in das ewige Leben (Matth. 25).\bibel{Matth. 25}

  Wisset, Freunde es ist keine Schande für einen Hausvater,
  die Seinen selber zu unterweisen oder jemand anders es für ihn
  tun heißen, vielmehr ist es eine wichtige Pflicht, die ihm zu
  tun auferlegt ist. Abraham und Josua haben es also gemacht.
  Vom ersteren heißt es Genesis 18,19,5\bibel{Genesis 18,19:5}: \zitat{er wird befehlen seinen
  % \picinclude{./220-229/p_s222.jpg} 
  Kindern und seinem Haus, das sie des Herrn Wege halten [...] },
  und der zweite sagt, Josua 24,15\bibel{Josua 24:15}: 
  \zitat{erwählet euch heute welchem
  ihr dienen wollt; ich aber und mein Haus wollen dem Herrn
  dienen}. Wir erklären, das wir es für unsere Pflicht halten,
  mit denen und für die zu beten, die unsrem Hause angehören,
  und sie zu lehren und zu ermahnen; denn es ist dies ein Befehl
  vom Herrn und der Ungehorsam hieregen wird sein Missfallen
  erregen, wie wir Jeremias 10, 25\bibel{Jeremias 10:25} 
  sehen können: \zitat{Schütte deinen
  Zorn über die Heiden, die dich nicht kennen und über die 
  Familien, die deinen Namen nicht anrufen}. Nun bilden die
  Reger, die Rothäuter\index{Rothäuter}, die Twanies\index{Twanies}, 
  die Indianer\index{Indianer} überall einen
  großen Teil der Familien hier auf dieser Insel, und es wird 
  Rechenschaft über sie gefordert werden von dem, der kommen wird zu
  richten die Lebendigen und die Toten am großen Tage des
  Gerichts: \zitat{da ein jeder empfangen wird seinen Lohn, nach dem
  er gehandelt hat, es sei gut oder böse}, wenn er \zitat{wird 
  geoffenbaret werden mit Feuerflammen, Rache zu geben über die so
  Gott nicht erkennen} [...] und \zitat{es werden in den letzten Tagen
  Spötter kommen, die nach den eigenen Lüsten wandeln} [....] \zitat{es
  wird aber des Herrn Tag kommen wie ein Dieb in der Nacht} [...]
  wie 2. Thess. 1,8\bibel{Thess. 2. 01:8@2. Thess. 1,8} und 
  2. Pet. 3\bibel{Pet. 2. 03@2. Pet. 3} zu sehen ist.
}


Die Veranlassung zu diesem Gerücht, das wir versuchten die
Neger aufzuhetzen, hatten unsre Gegner daraus geschöpft, das
wir Versammlungen mit und unter den Negern gehabt hatten,
denn sowohl ich als andere der Freunde hatten mehrere 
Versammlungen mit ihnen in verschiedenen Plantagen, in denen wir
sie zu Rechtschaffenheit, zur Keuschheit, zur Nüchternheit und zur
Frömmigkeit ermahnten, und zum Gehorsam gegen ihre Herrn
und Meister, also gerade das Gegenteil von dem, was unsre
übelwollenden Gegner böswillig gegen uns ausstreuten. [...]
Ehe ich die Insel verließ, schrieb ich folgenden Brief an
meine Frau:

\brief{Fell, Margaret}{
Mein liebes Herz,

\bigskip 

Welcher meine Liebe gehört, sowie allen Kindern im Samen
des Lebens, der sich nicht verändert, sondern größer ist als alles,
gelobt sei der Herr ewiglich. Ich habe unaussprechlich an Seele
und Leib zu erdulden gehabt; aber der Gott des Himmels sei
gelobt, seine Wahrheit geht über alles. Ich bin jetzt gesund, und
so der Herr will, gehe ich in einigen Tagen von Barbados nach
% \picinclude{./220-229/p_s223.jpg} 
Jamaika\ort{Jamaika} und gedenke nur kurze Zeit dort zu bleiben. Ich hoffe,
das ihr alle im Samen des Lebens\index{Samen des Lebens}, frei von aller Kümmernis,
bewahret bleibet. Die Freunde sind im allgemeinen wohl. Grüße
mir die Freunde, die nach mir fragen. Soviel diesmal. Meine
Liebe im Samen und Leben, die nicht wechseln.

\bigskip 

Barbados, 6. des 11. Monats 1671\index{Jahr!1671}. G. F.
}

Ich schiffte mich am 8. des 11. Monats 1671 in Barbados
für Jamaika ein [...] Wir hatten eine gute, rasche Überfahrt, [...]
und trafen in Jamaika James Lancaster\person{Lancaster, James}, 
John Eartwright\person{Eartwright, John} und
George Pattison\person{Pattison, George} wieder, die 
eifrig im Dienste der Wahrheit gearbeitet hatten, dem wir 
uns nun auch widmeten, wir reisten
aus der Insel hin und her; es ist ein recht schönes Land, doch
sind die Leute zum Teil recht verdorben und ausschweifend. Wir
wirkten viel. Es war eine große Belehrung und viele nahmen
die Wahrheit auf, worunter manche angesehene Leute. Wir
hatten viele Versammlungen hier, die zahlreich und ganz ruhig
waren. Die Leute begegneten uns sehr anständig und niemand
tat den Mund gegen uns auf. Ich war zweimal beim Gouverneur 
und den Behörden, die sehr freundlich gegen mich waren.

Etwa eine Woche nach meiner Ankunft in Jamaika schied
Elisabeth Hooton\person{Hooton, Elisabeth}, eine sehr 
alte Frau, die viel im Dienst der
Wahrheit umhergereist war und viel dafür gelitten, aus diesem
Leben. Sie war noch am Tage vor ihrem Tode gesund und
schied in Frieden, und gab noch im Sterben der Wahrheit die
Ehre. Nachdem wir etwa sieben Wochen in Jamaika gewesen,
und unter den dortigen Freunden etwas Ordnung geschaffen und
mehrere Versammlungen unter ihnen eingerichtet hatten, ließen
wir Solomon Eccles\person{Eccles, Solomon} dort, und 
schifften uns für Maryland\ort{Maryland} ein [...]

Ehe ich Jamaika verließ schrieb ich noch einmal einen Brief
an meine Frau:

\brief{Fell, Margaret}{
Mein liebes Herz,

\bigskip 

Dir und den Kindern meine Liebe in dem, das über allem
ist und sich nicht verändert, und allen Freunden die bei euch sind,
Ich bin nun etwa fünf Wochen in Jamaika gewesen. Den
Freunden geht es im ganzen gut und wir haben große Bekehrungen 
gehabt; aber es würde zu weit führen, über alles zu
schreiben; überall warten Leiden meiner, aber der gesegnete Same
ist über allem. Der Herr sei gelobt, welcher Herr ist über Land
% \picinclude{./220-229/p_s224.jpg} 
und Meer und alles was darinnen ist. Wir haben im Sinn,
etwa anfangs des nächsten Monats von hier abzureisen nach
Maryland zu, so der Herr will. Bleibet alle miteinander im
Samen des Herrn; in seiner Wahrheit bleibe ich in der Liebe zu
euch allen

\begin{flushright}Jamaika, 23. des 12. Monats 1671. G. F.\end{flushright}
}
