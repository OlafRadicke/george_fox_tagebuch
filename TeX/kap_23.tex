

%%%%%%%%%%%%%%%%%%% Kapitel 23. %%%%%%%%%%%%%%%%%%%%%%%%%%%%%%
\chapter[Rückkehr nach England]{Rückkehr nach England}

\begin{center}
\textbf{Rückkehr nach England. Kampf der Ordnungspartei gegen die
unbotmäßigen Quäker. Briefe über Toleranz an den König von
Polen, den Großmogul und andere.}
\end{center}


Da wir spürten, das wir unser Werk in Holland getan
hatten, nahmen wir Abschied von den Freunden in Rotterdam [...].
Am 21. des 8. Monate reisten wir nach England ab, William
Penn\person{Penn, William}, George 
Keith\person{Keith, George} und ich und Gertrud Dirick 
Nieson\person{Nieson, Gertrud Dirick} mit ihren
Kindern. Wir hatten eine lange und gefahrvolle Überfahrt [...].
aber der Herr, der den Winden gebieten kann und die stürmischen
Wellen des Meeres stille macht, das sie auf und nieder gehen,
wie es ihm gefällt, er behütete uns [...]. Am Abend des 23.
kamen wir in Harwich an. Am nächsten Morgen gingen William
Penn und George Keith mit mir nach Colchester\ort{Colchester} [...]. Dort blieben
wir bis zum Ersten Tag, da es mich verlangte, der Versammlung
der Freunde beizuwohnen. Es war eine überaus zahlreiche und
wirksame, denn als die Freunde von meiner Rückkehr hörten,
strömten sie von allen Seiten herbei vom Lande und auch aus
der Stadt, so das etwa tausend Menschen anwesend waren [...]\index{Versammlung!große}.
Am 9. des 9. Monats kam ich nach London\ort{London}, wo ich mit großer
Freude empfangen wurde.

Als ich einige Zeit in London war, schrieb ich folgenden Brief
an meine Frau\person{Fell, Margaret}:

\brief{Fell, Margaret}{
  Liebes Herz,

  \bigskip 

  Dir und den Kindern meine Liebe und allen andern Freunden
  in der Wahrheit, der Kraft und dem Samen des Herrn, der über
  allem ist. Dem Herrn sei Ehre und sein Name sei immerdar
  hochgelobt! Er hat mich durch allerlei Trübsal und Gefahr 
  hindurch geführt, in seiner ewigen Kraft; ich bin zweimal in der 
  Versammlung in Gracechurch Street\ort{Gracechurch Street}\footnote{Eine 
  Straße in London, in der die Quaker bis 1821 ein Meeting-Haus, 
  bis es abbrannte. Siehe her zu 
  \url{http://en.wikipedia.org/wiki/Gracechurch_Street}} 
  gewesen, und obgleich auch feindliche
  Geister zugegen waren, war doch alles ruhig; der Tau des Himmels
  fiel auf die Anwesenden, und die Herrlichkeit des Herrn schien
  % \picinclude{./260-269/p_s267.jpg} 
  über allen. Ich muss wohl oder übel täglich zu
  Versammlungen\index{Versammlung!unfreiwillige}
  gehen, in geschäftlichen Angelegenheiten und wegen allerlei 
  Drangsal, deren es viele gibt rings umher, und viele Freunde haben
  gegenwärtig darunter zu leiden, darum in Eile euch alle grüsend.

  \bigskip 

  London, 24. des 9. Monats 1677\index{Jahr!1677}. G. F.
}

Um diese Zeit erhielt ich Briefe aus Neu-England\ort{Neu-England}, welche
berichteten, wie die Behörden grausam und unchristlich gegen die
dortigen Freunde verfuhren, indem sie sie abscheulich misshandelten
und peitschten; sie peitschten viele Frauen unter den Freunden.
Eine Frau banden sie an einen Karren und schleppten sie 
halb entblößt durch die Straßen. Sie peitschten einige Schiffskapitäne,
die selber keine Freunde waren, nur weil sie Freunde hergebracht
hatten. Währenddem sie aber in dieser barbarischen Weise die
Freunde verfolgten, schlugen die Indianer\index{Indianer} sechzig ihrer Leute,
nahmen einen der Führer gefangen und zogen ihm bei lebendigen
Leib die Haut vom Kopf und trugen sie im Triumph davon.
Manche einsichtige Leute sagten: \zitat{Gottes 
Gericht\index{Gottes Gericht} ist über sie 
gekommen, weil sie die Quäker verfolgten.} Aber die verblendeten,
verfinsterten Priester sagten, es sei, weil sie sie nicht genug 
verfolgt hätten. Ich hatte große Mühe, für die fernen leidenden
Freunde Erleichterung zu schaffen, damit sie nicht unter die Rute
der Bösewichter kämen [...].

Ich blieb etwa einen Monat in London; darauf ging ich
nach Buckinghamshire\ort{Buckinghamshire} und besuchte die dortigen Freunde und
hatte mehrere Versammlungen. Ofters machten während derselben,
solche, die von der wahren Einigkeit der Freunde in der Wahrheit
abgewichen und in Zank, Zwiespalt und Auflehnung geraten\index{Streit unter Quäkern}
waren, große Störungen, besonders während der Männerversammlungen 
bei Thomas Ellwoods\person{Ellwood, Thomas} in Hunger Hill; ihr Anführer
kam von Wickham\ort{Wickham} und versuchte die Freunde zu stören und
an der weiteren Abhaltung der Versammlung zu hindern. Als
ich ihr Vorhaben merkte, ermahnte ich sie, ruhig und vernünftig
zu sein und die Versammlung nicht durch Unterbrechungen zu
stören; sondern, wenn sie mit dem Vorgehen der Freunde nicht
einverstanden seien und etwas dagegen einzuwenden hätten, dafür
eine Versammlung auf einen andern Tag zu veranstalten. Die\index{Konfliktbeseitibung}
Freunde boten ihnen an, an einem folgenden Tag eine Versammlung 
für sie abzuhalten, und schließlich wurde eine solche für die
darauf folgende Woche bei Thomas Ellwood festgesetzt. Die
% \picinclude{./260-269/p_s268.jpg} 
Freunde trafen sie dort, und die Versammlung fand in der
Scheune statt, weil so viele gekommen waren, das das Haus sie
nicht fassen konnte. Nachdem wir eine Zeitlang dagesessen hatten,
fingen sie an mit ihren Zänkereien. Die meisten ihrer Pfeile
waren gegen mich gerichtet;\index{Persönliche Angriffe} aber der Herr war mit mir und
stärkte mich, das ich in seiner Kraft die Pfeile der Bosheit und
Falschheit gegen sie selber zurück schleudern konnte. Ihre 
Entgegnungen wurden widerlegt, und manches wurde den Leuten
geoffenbart, und die Wahrheit wurde gefördert; viele, die zuvor
schwach gewesen, wurden gestärkt und gefestigt; etliche, die 
geschwankt und gezweifelt, wurden überzeugt und befestigt, und die
gläubigen Freunde wurden erquickt und ermuntert im Wachstum
des Lebens. Denn die Kraft wuchs unter uns, und das Leben
gedieh, und manch lebendiges Zeugnis wurde abgelegt gegen die
bösen, trennenden und spaltenden Geister, von denen jene Gegner
getrieben wurden, und die Versammlung endete zur Zufriedenheit
der Freunde. Ich übernachtete mit anderen Freunden bei
Thomas Ellwood; in der gleichen Woche hatte ich noch eine 
Versammlung mit den Gegnern in Wickham, wo sie abermals ihre
Bosheit zeigten und vor den Rechtgesinnten bloßgestellt\index{Bloßstellung} wurden [...].

Hierauf besuchte ich die Freunde in Henley in Oxfordshire\ort{Oxfordshire},
und dann gings durch Cosham nach Reading, wo ich eine große
Versammlung mit Freunden hatte. Am folgenden Tage in einer
Versammlung zur Besprechung über die Einrichtung einer 
Frauenversammlung\index{Frauenversammlung} gerieten etliche, 
die dem Geist der Uneinigkeit Raum
gegeben hatten, in Streit und waren eine Zeitlang widerspenstig,
bis die Wucht der Wahrheit sie bezwang. Darauf hatte ich
Versammlungen an verschiedenen Orten, und am 24. des 
11.~Monats, gerade zum Jahrmarkt, kam ich nach Bristol\ort{Bristol}.

Ich blieb während der ganzen Zeit des Jahrmarkts da und
noch einige Zeit nachher. Wir hatten viele schöne Versammlungen.
Aus allen Gegenden des Landes waren viele Freunde da, teils
in Geschäften, teils um Sachen der Wahrheit willen. Groß war
die Liebe und Einigkeit unter denjenigen Freunden, die der
Wahrheit treu blieben. Jedoch etliche, die von der heiligen
Einigkeit abgewichen waren und in Streit, Uneinigkeit und 
Feindseligkeit geraten, waren grob und beleidigend und benahmen sich
unchristlich gegen mich.\index{Streit unter Quakern}
\index{Quaker greifen Fox an} Aber die Kraft des Herrn war über
allen; weil sie mich in der himmlischen Geduld erhielt, welche
% \picinclude{./260-269/p_s269.jpg} 
kann Schmähungen um seines Namens willen ertragen, so fühlte
ich mich Herr über die groben und ungeregelten Widerspenstigen
und überließ sie dem Herrn, der meine Unschuld kannte und sich
meiner Sache annehmen würde. Je eisiger diese waren, um
mich zu schmähen und zu erniedrigen, desto mehr Liebe strömte
mir von den aufrichtigen, wahren, ehrlichen Freunden entgegen,
und etliche, die von den Gegnern verführt worden waren, trennten
sich von ihnen, als sie ihre Schlechtigkeit und Bosheit und ihr
grobes Benehmen sahen; sie haben alle Ursache, Gott für ihre
Errettung zu preisen [...].

Am 8. des 3. Monats 1678\index{Jahr!1678} kam ich nach London\ort{London}; 
das Parlament tagte gerade, und Freunde, die eine Klage über ihre Leiden 
eingereicht hatten, warteten nun auf die Erklärung, das das Gesetz
gegen päpstliche Rekusanten uns nicht treffe. Man wusste zwar
wohl, das wir nichts mit diesen zu tun hatten; aber dennoch
hatten einige böswillige Behörden davon gegen uns Gebrauch
gemacht, um uns in verschiedenen Gegenden zu verfolgen. Ich
schloss mich nun den Freunden, die sich in dieser Sache bemühten,
an, und es war Aussicht vorhanden, etwas zur Erleichterung der
Freunde aus diesem Wege zu erreichen, weil viele der 
Parlamentsmitglieder den Freunden geneigt und wohlgesinnt waren und
einsahen, das uns unsere Gegner oft falsch darstellten. Als ich
aber eines Morgens mit George Whitehead\person{Whitehead, George} 
zum Parlamentsgebäude kam, war das Parlament vertagt [...].

Etwa zwei Wochen nach meiner Ankunft in London fand
die Jahresversammlung\index{Jahresversammlung} statt [...] 
worüber ich meiner Frau
bald darauf in einem Brief berichtete: 

\brief{Fell, Margaret}{

  Liebes Herz,

  \bigskip 

  Dir meine Liebe in dem ewigen Samen des Lebens, welcher
  alles regieret. Große Versammlungen sind hier gewesen, und die
  Kraft des Herrn hat alle gepackt wie noch nie. Der Herr hat
  durch seine Kraft die Freunde herrlich untereinander verbunden,
  und seine glorreiche Gegenwart erschien unter ihnen. Und jetzt,
  da die Versammlungen vorüber sind, lobe man den Herrn in
  Ruhe und Frieden. Aus Holland\ort{Holland} vernehme ich, das dort alles
  gut geht. Es sind einige Freunde hingegangen, um der 
  Jahresversammlung in Amsterdam\ort{Amsterdam} beizuwohnen. 
  In Emden\ort{Emden} sind
  Freunde, die verbannt gewesen waren, wieder in die Stadt
  zurückgekehrt. In Danzig\ort{Danzig} waren Freunde im Gefängnis und die
  % \picinclude{./270-279/p_s270.jpg} 
  Behörden drohten ihnen mit noch härterer Gefangenschaft. Aber
  am darauf folgenden Tage machten die Lutheraner\index{Lutheraner} einen Aufstand
  und zerstörten das papistische Kloster, und so haben sie nun
  genug mit sich selber zu tun. Der König von Polen\person{König von Polen} hat meinen
  Brief erhalten und selbst gelesen, und Freunde haben ihn seither
  Hochdeutsch\index{Hochdeutsch} gedruckt. Durch Briefe von der halbjährlichen 
  Versammlung in Irland\ort{Irland} höre ich, das sie dort alle in der Liebe
  bleiben. In Barbados\ort{Barbados} haben die Freunde Ruhe, und ihre 
  Versammlungen verlaufen friedlich. Auch in Antigua\ort{Antigua} und Nevis\ort{Nevis}
  gedeiht die Wahrheit, und die Freunde haben geordnete und
  schöne Versammlungen. In Neu-England\ort{Neu-England} und an anderen Orten
  geht ebenfalls alles in Betreff der Freunde und der Wahrheit
  seinen guten Gang; an diesen Orten sind die Männer- und
  Frauen-Versammlungen geordnet, gelobt sei der Herr. So bleibet
  denn im Samen und in der Kraft Gottes, die über allem ist,
  durch welche wir Leben und Heil haben, denn der Herr regieret
  alles in seinem Reich und seiner Herrlichkeit, ihm sei Ehre 
  ewiglich, Amen. In Eile grüße ich euch alle und alle Freunde.

  \bigskip 
  \begin{flushright}
  London, 26. des 3. Monats 1678\index{Jahr!1678}. G. F.\end{flushright}

}

% \picinclude{./270-279/p_s270.jpg} 
% Ein Teil des Textes nach 260-269 verschoben.


Der erwähnte Brief an den König von Polen ist folgender:

\brief{König von Polen}{
    O König!

    \bigskip 

    Wir wünschen dir Wohlergehen sowohl in diesem als dem
    zukünftigen Leben. Und wir hoffen, das wir unter deiner Herrschaft
    unsre christliche Freiheit haben werden, Gott zu dienen und 
    anzubeten. Denn wir haben den Grundsatz, nichts zu tun, das dem
    König oder seinem Volke schaden kann. Wir sind Leute, die mit
    einem guten Gewissen vor Gott wandeln wollen, durch seinen
    heiligen Geist, und in demselben ihm dienen und ihn ehren wollen
    und den Menschen gegenüber in allem, was recht und billig ist,
    indem wir ihnen tun, was wir möchten, das sie uns tun, im
    Blick auf Jesus, den Anfänger und Vollender unseres Glaubens,
    welcher Glaube unsre Herzen reinigt und uns Zugang zum
    Vater verschafft, ohne welchen wir ihm nicht gefallen können,
    und durch den alle Gerechten leben, wie die Schrift sagt (Hebr.~12\bibel{Hebr. 12}, 
    Röm. 5\bibel{Röm. 05@Röm. 5}). Was wir von dir bitten, o König, ist, das wir
    Gewissensfreiheit haben, Gott anzubeten und ihm zu dienen, und
    ihn zu verehren, und in unseren Versammlungen miteinander zu
    ihm zu beten im Namen Jesu, wie er gebietet, mit der 
    Verheisuug, mitten unter uns zu sein. Wir hoffen, der König müsse
% \picinclude{./270-279/p_s271.jpg} 
    zugeben, das solcher Dienst Gott gebühret und Christus. Und
    wir geben dem Kaiser, was ihm gebühret und bezahlen unsre
    Abgaben und Steuern, wie unsre Nachbarn, je nach unsern 
    Verhältnissen. Wir haben nie in irgend einer Schrift des neuen
    Testamentes gelesen, das Christus oder seine Jünger irgend 
    jemand verbannten oder gefangen nahmen, der nicht ihren Glauben
    hatte und sie nicht anhören wollte, oder das sie befohlen hätten,
    solches zu tun, sondern im Gegenteil solle man den Weizen und
    die Spreu beisammen wachsen lassen, bis zur Ernte. Und die
    Ernte ist das Ende der Welt; dann wird Christus seine Engel
    senden, um die Spreu vom Weizen zu scheiden. Er tadelte die,
    welche wollten Feuer vom Himmel regnen lassen, um die, so
    Christus nicht aufnehmen wollten, zu vertilgen, und sagte ihnen,
    sie wüsten nicht, welch Geistes Kinder sie seien (Luc. 9\bibel{Luc. 09@Luc. 9}). 
    Er kam nicht, um die Leben der Menschen zu zerstören, sondern sie
    zu erretten.

    Wir bitten den König, daran zu denken, wie viel Verfolgung
    gewesen ist um des Glaubens willen seit den Tagen der Apostel
    unter den Christen. Christus sagte, das die, welche ihn nicht 
    besuchten, als er gefangen war (Matth. 25\bibel{Matth. 25}), 
    in die ewige Verdammnis\index{Verdammnis} kommen: was wird 
    dann erst aus denen werden, die ihn
    gefangen nehmen in den Seinen, in denen er sich kund tut?
    Noch ist das Ende der Welt nicht gekommen; wie will sich das
    Christentum am Tage des Gerichts vor dem furchtbaren Gott\index{Verfolgung}
    \index{Glaubenskrieg} verantworten darüber, das man sich untereinander um der Religion
    willen verfolgte, unter dem Vorwand, die Spreu vom Weizen zu
    scheiden, ehe das Ende der Welt da war? Christus befiehlt
    den Menschen, sich unter einander zu lieben und die Feinde zu
    lieben, daran solle man erkennen, das sie seine Jünger seien
    (Joh.13,35\bibel{Joh.13,35}). O, das doch alle Christen in Einigkeit und Frieden
    gelebt hätten, damit sie durch ihre Mäßigkeit und Selbstbeherrschung 
    sowohl Türken\index{Türken} als Heiden\index{Heiden} beschämt hätten! Lasset alle,
    die Gott und Christum bekennen und nach dem herrlichen Evangelium 
    des Herrn Jesu leben, ihre Freiheit haben. Es ist unser
    Wunsch das der Herr des Königs Herz geneigt mache gegen
    alle zarten Gewissen, die den Herrn fürchten und sich scheuen,
    ihm ungehorsam zu sein.

    Wir bitten inständig den König, einige der edelmütigen
    Kundgebungen verschiedener Könige und anderer zu lesen über
% \picinclude{./270-279/p_s272.jpg} 
    die Gewissensfreiheit, und besonders das was Stephanus, der
    König von Polen\person{König von Polen} sagt, nämlich: \zitat{Es 
    kommt mir nicht zu, die
    Gewissen zu reformieren. Ich habe dies immer Völlig Gott überlassen, 
    weil es bei ihm steht; so halte ich's jetzt und werde es
    in Zukunft halten. Ich lasse den Weizen wachsen bis zur Ernte,
    denn ich weiß, das die Zahl der Gläubigen klein ist. Ich bin,}
    sagte er, wenn andere fort fuhren mit Verfolgungen, \zitat{der König
    der Leute und nicht deren Gewissen.} Er war auch der Ansicht,
    das die Religion nicht solle mit Feuer und Schwert gepflanzt
    werden [...].

    Ebenso wird Gewissensfreiheit zugesagt bei König Jakob\person{König Jakob}, 
    in s einer Rede im Parlament, 1609\index{Jahr!1609} [...]. Ferner 
    durch König Karl\person{König Karl}, [...], dann durch den 
    Prinzen von Oranien\person{Prinz von Oranien}, im
    Jahre 1579\index{Jahr!1579} [...]. Ebenso bestätigen sie: Erasmus, [...] 
    Irenäus\person{Irenäus} [...], der Kaiser Konstantin\person{Kaiser Konstantin}
    [...] Augustin [...] Heinrich IV.\person{Heinrich IV.} [...] 
    Eusebius\person{Eusebius} [...] und andere [...] Und nun, o König, 
    im Blick auf all diese Zeugnisse über die Gewissensfreiheit, von Kaisern, 
    Königen und andern, und auf die Freiheit, die Paulus in Rom hatte in den
    Tagen des heidnischen Kaisers, bitten wir, das wir in Danzig\ort{Danzig}
    auch die Freiheit haben möchten, in unsern Häusern zusammen
    zu kommen; es kann weder dem König noch der Stadt irgend
    etwas schaden, wenn wir zusammen kommen, um auf den Herrn
    zu harten\index{Harren auf den HERREN} und zu ihm zu beten und im 
    Geist und in der Wahrheit ihm zu dienen in unsern eigenen Wohnungen, 
    da unsere Grundsätze uns in keiner Weise veranlassen, jemand zu schaden,
    sondern unsre Feinde zu lieben und für sie zu beten, selbst für die,
    so uns Verfolgen. Darum, o König, und du, Stadt Danzig, bedenket, 
    würde es euch nicht grausam scheinen, wenn man euch zu
    einer Religion zwingen wollte, die euern Gewissen entgegen wäre?
    Und wenn ihr es grausam finden würdet, wenn man euch solches
    täte, so tut nicht den andern, was ihr nicht wollt, das sie euch
    tun; das ist das königliche Gesetz, dem man zu gehorchen hat.
    Solches wurde geschrieben in der Liebe zu deiner unsterblichen
    Seele und zu deinem ewigen Heil. 

    \bigskip 

    \begin{flushright}G. F.\end{flushright}.

    \bigskip 

    P. S. Selig sind die Barmherzigen, denn sie sollen Barmherzigkeit 
    erlangen (Matth. 5\bibel{Matth. 05@Matth. 5}). Und gedenke, o König, der zwei
    Apologien des Justinus Martyr\person{Martyr, Justinus} an den römischen 
    Kaiser\person{Rümischer Kaiser} zur Verteidigung der verfolgten 
    Christen und jener denkwürdigen Apologie, von 
    Tertullian geschrieben, über denselben Gegenstand, und die
% \picinclude{./270-279/p_s273.jpg} 
    nicht für die christliche Religion gelten, sondern für alle Verfolgungen 
    um der Religion willen [...]. 

}

Ich blieb noch einige Wochen in London, das Parlament tagte
wieder, und mehrere Freunde versuchten, Linderung der Leiden für
die Freunde zu erlangen [...]. Aber obgleich in beiden Häusern
verschiedene Parlamentsmitglieder den Freunden wohlwollten und
gerne etwas zu ihrer Hilfe getan hätten, so waren sie durch die
Menge der Geschäfte daran verhindert, so das die Leiden der
Freunde andauerten.

Was aber namentlich viel zum Kummer und zur Prüfung
der Freunde beitrug war, das etliche, die sich zur gleichen 
Wahrheit bekannten wie wir, abgewichen waren von der Einfalt des
Evangeliums und in fleischliche Freiheit geraten waren und 
versuchten, andere nach sich zu ziehen; sie widersetzten sich der
Ordnung, die Gott durch seine Kraft aufgestellt und in seiner
Kirche eingeführt hatte; sie machten viel Lärm und Unruhe über
allerlei Vorschriften, wobei es ihnen leicht wurde, manche mit
sich zu reisen, die freiere Neigungen hatten und einen breiteren
Weg gehen wollten, als den der Wahrheit. So geschah es,
das etliche, die zwar einfältigen Gemüts, aber in der Wahrheit
noch Neulinge und von wenig Urteilskraft waren, dadurch 
verführt wurden, da sie die Abgründe des Satans nicht kannten.
Diesen zum Nutzen, um die Betrogenen aufzuklären und den
Schwachen das Verständnis über diese Dinge zu eröffnen, schrieb
ich folgendes: 

\brief{Quaker-Gemeinde}{
    Ihr alle, die ihr urteilslos manche Gebote\index{Gebote verwerfen} verwerft, 
    ihr könnt ebensogut die ganze Schrift verwerfen, die uns
    gegeben wurde durch die Kraft de-3 Geistes Gottes. Denn zeigt
    sie nicht, sowohl im alten als im neuen Testament, wie man vor
    Gott und den Menschen wandeln soll? Zeigt sie nicht, von der
    allerersten Verheisung auf Christum in der Genesis an, bis auf
    die Zeiten der Propheten, was man glauben und worauf man
    trauen soll? Hat nicht der Herr seinem Volke Gebote gegeben
    zuerst durch die Väter und darnach durch die Propheten? Hat
    nicht der Herr seinem Volke immer wieder geboten, wie es wandeln
    solle, obgleich es sich gegen die Propheten des alten Bundes auflehnte,
    wenn sie ihm den Weg vorschrieben, den es gehen solle, um Gott
    zu gefallen und in seiner Gnade zu bleiben? Und hat nicht
    Christus zu seiner Zeit die Leute gelehrt und geboten, wie sie
    glauben und handeln sollten? Und haben nach ihm die Apostel
    % \picinclude{./270-279/p_s274.jpg} 
    nicht vorgeschrieben, wie man dazu kommen kann, zu glauben und
    das Evangelium und das Reich Gottes anzunehmen, indem
    sie aus das hinwiesen, was zur Erkenntnis Gottes führen kann?
    Und zeigten sie nicht, wie man im neuen Bunde wandeln soll
    und auf welchem Wege man zur Heiligen Stadt gelangt? Und
    haben die Apostel ihre Gebote nicht durch treue auserlesene Männer,
    welche ihr Leben für Christus aufs Spiel setzten, den Kirchen
    mitgeteilt und dieselben dadurch begründet? Indem ihr nun die
    Verordnungen, die von dem Geist Gottes eingegeben wurden,
    verwerft, widerstrebt ihr damit dem Geist, der durch alle Heiligen
    zu euch geredet hat. Gab es nicht von jeher, zur Zeit Moses und
    der Propheten, zur Zeit Christi und in den Tagen der Apostel
    etliche, welche dem Geist, der aus allen diesen zu ihnen redete,
    widerstanden? Und ist es nicht so gewesen auch seit den Tagen der
    Apostel? Wie viele haben sich nicht erhoben seit dem Erscheinen
    der Wahrheit, um sich der Ordnung, die aus dem Geist Gottes
    ist, zu widersetzen! Diese alle sind aus demselben Geist, der von
    Anfang an sich dem Geist Gottes widersetzte. Seht was für
    Namen und Bezeichnungen der Geist Gottes diesem Geist des
    Widerstands gab im alten und auch im neuen Testament, und
    dasselbe gilt noch heute. Als Gott der Herr den alten Bund\index{Alter Bund}
    geschlossen hatte, wurden etliche unter einander uneins und 
    widersetzten sich, und diese waren schlimmer als der äußre Feind.
    Und seht nur, was das für eine Sorte war, die sich in den Tagen
    des neuen Bundes, zur Zeit des Evangeliums, Christus und den
    Aposteln widersetzten, nachdem sie mit der Wahrheit in Berührung 
    gekommen waren. Seht, was das für eine Freiheit war,
    für die sie stritten, und zu der es die, welche dem Kreuz Christi
    widerstanden, brachten.

    Und es ist der gleiche hochfahrende Geist, der auch jetzt nach
    einer Freiheit verlangt, die der Geist und die Kraft nicht gewähren
    können. Er schreit gegen Zwang, und übt doch selbst Zwang; er
    schreit nach Gewissensfreiheit\index{Gewissensfreiheit} und 
    widersetzt sich doch der Gewissensfreiheit; er schreit gegen 
    die Vorschriften\index{Vorschriften} und macht selber solche
    in Wort und Schrift. Dieser Geist, sein Ursprung, Anfang und
    Ende, sind durch den Geist Gottes erkannt und gerichtet. Wenn
    jener Geist ruft: \zitat{man darf nicht über die Gewissen richten! man
    darf nicht in Glaubenssachen, nicht in Sachen der Religion, nicht
    über die Geister richten!}\index{Richten}\index{Subjektivismus}
    \index{Relativismus} so sage ich: \zitat{doch! Die, welche in
    % \picinclude{./270-279/p_s275.jpg} 
    der reinen Kraft und dem reinen Geist sind, in dem die Apostel
    waren, richten die Gewissen, ob es verhärtete Gewissen oder
    empfängliche seien; sie richten den Glauben, ob er lebendig oder
    tot sei; sie richten die Religion, ob sie eitel sei oder rein und
    unbefleckt; sie richten die Geister und prüfen sie, ob sie von Gott
    seien oder nicht; sie prüfen die Herzen, Ohren und Lippen,
    mehr rein ist und wer nicht; sie richten die Prediger, Apostel
    und Propheten, ob sie von Christus kommen oder nicht; sie
    richten die Streitigkeiten über äußere Dinge, sowohl in als
    auch außer der Kirche; ja, das geringste Glied der Kirche hat
    Macht, diese Dinge zu richten, weil es das einige rechte Maß
    und rechte Gewicht besitzt, womit alle Dinge können gewogen
    und gemessen werden ohne Ansehen der Person.} Solches
    Urteil geschieht, und solche Dinge werden getan durch die gleiche
    Kraft und den gleichen Geist, den die Apostel hatten. Solche
    können auch über Erwählung\index{Erwählung} und Verwerfung
    \index{Verwerfung} richten, wer die
    Wohnung behält und wer nicht, wer Jude\index{Juden} ist und wer zur
    Synagoge\index{Synagoge} des Teufels\index{Teufel} gehört (
    Offb.~3\bibel{Offb. 03@Offb. 3}), wer die Lehre Christi hat
    und wer die des Teufels, wer aus der Kraft und dem Geiste Gottes
    gebietet und wer aus einem losen Geist, der vom Joch Christi
    ab, in Zügellosigkeit, führt [...]. Darum sollen alle ermahnt
    werden, in der Kraft und dem Geist Christi zu bleiben, im Wort
    des Lebens und der Weisheit Gottes, die über dem, das hier
    unten ist, steht, in welcher sie ihre himmlische Urteilskraft
    und Unterscheidungsgabe festhalten und das himmlische, geistliche 
    Richten über jenes Richten stellen können, welches Gott
    verunehret und in falsche Freiheit\index{Freiheit, falsche} 
    führt, von der Einigkeit des
    himmlischen Geistes ab, welcher dem Bilde des Sohnes Gottes
    gleich macht, und von seinem Evangelium, dieser Kraft Gottes,
    und von seiner Wahrheit (von welcher der Teufel abgefallen ist)
    In welcher alle einer Meinung, ein Herz und eine Seele sind, und
    aus einem Geiste schöpfen werden, weil sie in einem Geist
    getauft\index{Taufe} sind, und also auch in einem Leib, davon Christus das
    Haupt ist, und somit die Brüderlichkeit und Einigkeit im Geist
    halten, welche das Band des Friedens ist, dieses Friedens des
    Fürsten aller Fürsten. Und die, welche am meisten gegen das
    Richten schreien und sich davor fürchten, seien sie nun abtrünnige
    \textit{Fromme} oder Gottlose, richten am allermeisten mit ihrem
    tadelsüchtigen\index{Tadelsüchtig}, falschen Geist und Gericht, 
    und doch können sie das
    % \picinclude{./270-279/p_s276.jpg} 
    wahre Gericht des Geistes Gottes nicht ertragen noch vor dem
    selben bestehen. Das hat man von Anfang an merken können,
    weil sie falsches Maß und falsches Gewicht harten; denn niemand
    hat das rechte Maß und rechtes Gewicht, als wer im Licht,
    der Kraft und dem Geist Christi bleibt. Nun kommt ein loser
    Geist auf, der nach Freiheit schreit und gegen die Vorschriften,
    und doch selber Wege vorschreibt in Wort und Schrift. Derselbe
    Geist schreit gegen das Richten und will sich nicht richten lassen
    und richtet doch selber durch einen falschen Geist. Solches ist
    geschrieben als Protest gegen diesen Geist.

    \bigskip 
    \begin{flushright}
    London\ort{London}, 9. des 4. Monats 1678\index{Jahr!1678}. G. F.\end{flushright}
}

Von London ging ich nach Hertford\ort{Hertford} [...]. Hier kam wieder
eine Unruhe über mein Gemüt\index{Gemüt(Unruhe)} wegen jener unordentlichen und
zügellosen Geister, die sich von uns losgetrennt hatten und suchten,
andere in eine falsche Freiheit\index{Freiheit, falsche} mit sich zu ziehen. 
Da ich ahnte,\index{Spaltung}\index{Schisma}
wie viel Unheil und Schaden dadurch angerichtet werden konnte,
trieb es mich, einige Zeilen an die Freunde zu schreiben, um sie
davor zu warnen:

\brief{Quaker-Gemeinde}{
    Ihr Freunde alle,
    \bigskip 

    Bleibet im süßen Leben des Lammes und über dem unordentlichen, 
    ausgeblasenen Geiste, der Zank und Hader anrichtet,\index{Konflikt}
    angeblich um des Gewissens willen, der aber doch in falsche 
    Freiheit und Zügellosigkeit führt, die der Jugend gefährlich werden.
    Die, welche dazu anstiften, sind schuld an dem Verderben, indem
    sie in ihrer Leidenschaft statt des Gewissens einen widerspenstigen
    Geist aufrühren, der den guten Geist in ihnen selber und in
    allen andern Menschen ersticken wird Dieser Geist darf nicht
    auskommen, weder in ihnen noch in andern; denn sie verschließen
    damit sich und andern das Himmelreich. Ein loser Geist, der
    sich unter dem Vorwand der Gewissensfreiheit\index{Gewissensfreiheit} erhebt, ein 
    widerspenstiger Wille, der sich mit Worten ohne Kraft zur Wahrheit
    bekennt, alle solche Unordentlichkeit kann sich nur verstecken hinter
    etwas, das zur ewigen Verdammnis führt. Darum bleibet
    im sanften Geist Gottes in aller Demut, damit ihr in demselben
    wisset, das ihr alle Glieder untereinander seid und alle einen
    Dienst habt in der Kirche Christi. Alle lebendigen Glieder kennen
    einander im Geist nicht im Fleisch. Da herrschet nicht der Mann
    über die Frau, wie Adam über Eva vor dem Fall, sondern Christus,
    der geistige Mann, über seine geistigen Glieder, die in der
    % \picinclude{./270-279/p_s277.jpg} 
    himmlischen Liebe sind, die Gott in ihre Herzen gab, und die
    allem Hader ein Ende macht.
    \bigskip 

    \begin{flushright}
    Hertford\ort{Hertford}, 11. des 5. Monats 1678\index{Jahr!1678}. G. F.\end{flushright}
}


Von Hertford ging ich nach [...] Leicester\ort{Leicester} [...] wo ich
Freunde besuchte, die hier im Gefängnis\index{Gefängnisbesuch} waren, weil sie Zeugnis
für Jesus abgelegt hatten. Ich redete ihnen zu, dem Herrn treu
zu bleiben und nicht müde zu werden, für ihn zu leiden\index{Leid}. Und
nachdem ich von ihnen Abschied genommen hatte, sprach ich noch
mit dem Kerkermeister und bat ihn, sie gut zu behandeln, und
ihnen zuweilen zu erlauben, die Ihrigen zu besuchen. Dann hatte
ich eine Versammlung in Warwickshire\ort{Warwickshire} [...] und von da ging
ich nach Staffordshire\ort{Staffordshire}, wo ich mehrere schöne und erleuchtende
Versammlungen hatte, sowohl um für die Wahrheit zu gewinnen,
als auch in ihr zu befestigen. Während ich in Stassordshire war,
trieb es mich, folgendes zu schreiben:

\brief{Quaker-Gemeinde}{
    Liebe Freunde der Vierteljahresversammlungen und Monats-
    Versammlungen allenthalben. 
    \bigskip 
    Es ist mein Wunsch, das ihr alle
    danach trachtet, in der Kraft und Wahrheit des Herrn einerlei
    Sinn zu haben, nämlich den einen friedsamen, in dem kein Streit
    und keine Feindschaft ist; und das ihr auch in der Weisheit
    Gottes sein möchtet, die rein, \zitat{keusch, friedsam und gelinde ist,
    und sich sagen lässt} und die {von oben kommt, und über dem
    das irdisch, menschlich und teuflisch ist} (Jak. 3\bibel{Jak. 03@Jak. 3}) ist. In dieser
    Weisheit möget ihr dazu kommen, \zitat{alles, was ihr tut, zu Gottes
    Ehre zu tun} (1. Cor. 10,31\bibel{Cor. 1. 10:31@1. Cor. 10:31}). 
    Und, liebe Freunde, wenn irgend\index{Konfliktbewältigung}
    einmal irgend etwas geschieht, das könnte Streit, Zank oder
    Zwietracht herbei führen, in euern monatlichen oder vierteljährlichen 
    Versammlungen, so soll es vor etwa ein halbes Dutzend
    Leute gebracht werden, die es außerhalb eurer Versammlungen
    besprechen und schlichten mögen, wie es am Anfang war, damit
    alle eure monatlichen und vierteljährlichen Versammlungen friedlich
    bleiben. Sie mögen dann der Versammlung vom Entscheid Mitteilung
    machen, damit die Schwachen und Jungen unter euch nicht verletzt
    werden, wenn sie von Zank und Hader in euern Versammlungen
    hören, in denen kein Streit sein sollte, sondern alle sollten in
    allen Dingen in gleicher Meinung vorgehen und beschließen, in
    der Kraft Gottes, der Ordnung des Evangeliums; in diesem
    Evangelium des Friedens werdet ihr den Frieden in allen euren
    Versammlungen bewahren. Wenn irgend zwei, Mann oder Frau,
    % \picinclude{./270-279/p_s278.jpg} 
    etwas gegen einander haben, sollen sie miteinander reden, und
    es unter sich abmachen; wenn sie es auf die Weise nicht 
    entscheiden können, so sollen sie zwei oder drei auswählen, um es
    zu entscheiden. Falls die es auch nicht entscheiden können, so soll
    es der Kirche vorgelegt werden; etwa ein halbes Dutzend aus der
    monatlichen oder vierteljährlichen Versammlung sollen es wissen, und
    endgültig ohne Ansehen der Person richten. Last alle Vorurteile
    abgelegt und begraben sein, auch alle Engherzigkeit untereinander,
    und \zitat{lasset die Liebe, die sich nicht blähet, nicht hast, nicht das
    Ihre sucht, sondern alles erträgt} (1. Cor. 13\bibel{Cor. 1. 13@1. Cor. 13}), 
    in euren Versammlungen herrschen, denn das erbauet den Leib, dessen Haupt
    Christus ist; und diese Liebe ist besser als tönendes Erz und
    eine klingende Schelle. Diese Liebe duldet alles und ist 
    freundlich; sie hält alles nieder, was sich rühmen, sich erheben,
    blähen oder unziemlich benehmen möchte, oder sich leicht erbittern
    lässt; sie ist Herr über alles, was nicht aus dem Geist ist, dessen
    Früchte sind Liebe, Friede, Freundlichkeit; aus das ihr \zitat{in diesem
    heiligen Geist alle getauft\index{Taufe} sein möget zu einem 
    Leib} (1. Cor. 12\bibel{Cor. 1. 12@1. Cor. 12}),
    und aus einem Geist schöpfet, in welchem ihr Einigkeit habet,
    und in welchem ist das Band des \zitat{Königs der Könige} und
    der Frieden des \zitat{Herrn aller Herren} [...] \zitat{Die in
    der Liebe bleiben, bleiben in Gott; denn Gott ist die Liebe}
    (1. Joh. 4\bibel{Joh. 1. 04@1. Joh. 4}). Bleibet alle in 
    dieser Liebe. Meine Liebe in Christo
    Jesu, dem ewigen Samen, der über allem ist.
    \bigskip 

    \begin{flushright}
    Staffordshire, 20. des 6. Monats 1678. G. F.\end{flushright}
}


Ich blieb nun im Norden während ungefähr eines Jahres,
da ich dort unter den Freunden für den Herrn zu wirken
und viel damit zu tun hatte, Antworten zu schreiben auf Bücher,
die die Gegner geschrieben\index{Erwiederung} hatten [...] Da es mir schien, das
viele, welche die Wahrheit aufgenommen hatten und Offenbarungen
darüber gehabt hatten, davon abgeirrt waren, weil sie nicht
demütig\index{Demut} geblieben waren, so trieb es mich, folgendes Schreiben
ergehen zu lassen als Warnung und Ermahnung an alle, in der
Demut zu bleiben:

\brief{Quaker-Gemeinde}{
    Meine lieben Freunde,

    \bigskip

    Die der Herr besucht hat in seiner freundlichen Barmherzigkeit
    mit dem Ausgang aus der Höhe, und eure Herzen öffnete, das
    ihr euch seinem Namen beugt und ihn bekennt: bleibet niedrig in
    euern Herzen und höret auf Christus, damit ihr in der Demut
    % \picinclude{./270-279/p_s279.jpg} 
    bleibet, die er euch lehrt, auf das die Jüngern unter euch in
    keiner Weise aufgeblasen oder überstiegen oder eingebildet werden
    durch ihre Offenbarungen und des Gnadenstandes\index{Gnadenstand} verlustig gehen,
    indem sie sich zum Eigendünkel hinreisen lassen und darnach in
    Verzweiflung geraten und so die Kraft Gottes missbrauchen.
    Es war das Bestreben der Apostel, das niemand die Kraft
    Gottes missbrauchen möchte, sondern der Glaube sollte sich in
    allen Dingen auf sie gründen, so das alle, die andern die 
    Wahrheit predigen, sich selber auch mit einschließen, und nicht andern
    predigten\index{Predigen} und sich selber ausnehmen. Darum kommt euch zu,
    euch mit einzuschließen in dem, das ihr andern predigt, und demütig
    zu bleiben darin; dann wird der Gott der Wahrheit den Demütigen
    erhöhen in seine Wahrheit, Licht, Gnade\index{Gnade}, Kraft, Geist und
    Weisheit, zu seiner Ehre. Dann behalten alle ihr Maß von
    Gnade, Licht, Kraft und Geist Christi. Lasset niemand den Geist
    dämpfen oder seine Regungen, ihn betrüben oder davon abweichen;
    sondern lasset euch von ihm, der alle in seiner Hut hält, leiten;
    er gibt das Verständnis, wie man dem heiligen, reinen Gott,
    dem Schöpfer und Erhalter in Christus, dienen, ihn anbeten
    und ihm gefallen soll; und wie man seinen heiligen Geist der
    Einigkeit und des Friedens in seinem Volke hüten, auf ihn
    warten und ihn pflegen soll. Der heilige Geist lehret ein sanftes,
    demütiges, ruhiges und heiliges Gemüt aus den Samen, den
    Christus in eines jeden Herz gegeben hat, wirken, und auch in denen,
    die vom Geist, der Gnade, dem Licht und dem Evangelium abgefallen
    sind, das Licht, die Gnade, den Geist und das Evangelium
    finden. So können, durch heiligen Wandel und heiliges Predigen,
    alle dazu gelangen, daran teil zu haben, damit Gott in allen
    Dingen geehrt werde durch euch, wenn ihr zu seinem Preis
    Früchte hervor bringet.
    \bigskip
    \begin{flushright}
    Swarthmore\ort{Swarthmore}, 30. des 10. Monats 1679\index{Jahr!1679}. G. F.\end{flushright}
}


Danach trieb mich der Herr, die Freunde in Surrey\ort{Surrey} und
Sussex\ort{Sussex} zu besuchen. Ich ging zu Wasser nach Kingston\ort{Kingston}, wo ich
einige Tage blieb; denn der Herr gab mir ein, dem Sultan\person{Sultan}
und dem Gouverneur von Algier\person{Gouverneur von Algier} zu schreiben, um diese und
ihre Untertanen zu ermahnen, von ihrer Schlechtigkeit zu
lassen, den Herrn zu fürchten und recht zu tun, sonst werde
das Gericht Gottes\index{Gericht Gottes} über sie herein 
brechen und sie ohne Gnade zerstören [...].

% \picinclude{./280-289/p_s280.jpg} 
Von Kingston ging ich nach London\ort{London} [...] und dann nach
Hertford\ort{Hertford} [...]. Dort traf ich John 
Story\person{Story, John}\footnote{John Story, der Führer einer Partei, 
die gegen die Frauenversammlungen und andere Einrichtungen 
auftrat.}, und etliche seiner
Richtung; aber das Zeugnis der Wahrheit hielt sie nieder, so
das wir eine ruhige Versammlung hatten. Dies war an einem Ersten
Tage, und da am folgenden Tage die geschäftliche Männer- und
Frauen-Versammlung war, so blieb ich auch noch zu dieser, um
so mehr, als etliche eine Geringschätzung derselben hatten aufkommen
lassen. Darum trieb es mich, über den Nutzen dieser Versammlungen
zu reden und über ihren Segen für die Kirche Christi, wie der
Herr es mir eingab, und dies tat den Freunden einen guten
Dienst [...].

Ich blieb den größten Teil dieses Winters 1680—81\index{Jahr!1680-81} in London,
eifrig im Dienst des Herrn, sowohl in Versammlungen als auch
sonst; denn da es eine Zeit schwerer Prüfungen für die Freunde
war, so zog es mich, ihre Versammlungen noch mehr als sonst
zu besuchen, um sie zu ermutigen und zu ermahnen durch Wort
und Beispiel. Das Parlament\index{Parlament} tagte, und die Freunde warteten
gespannt, bis sie ihre Anliegen Vorbringen konnten; denn wir
hörten fast jeden Tag von neuen Leiden, die die Freunde in 
verschiedenen Teilen des Landes zu erdulden hatten. Ich brachte
viel Zeit damit zu, meinen leidenden Brüdern Linderung zu 
verschaffen; mit einigen andern Freunden, die von sich aus sich auch
dieser Sache annahmen, wartete ich manchen Tag im 
Parlamentshaus und benutzte jede Gelegenheit, Parlanientsmitglieder, die
unsre berechtigten Klagen anhören wollten,\index{Politik}\index{Lobbyismus} 
zu sprechen. Einige
waren auch sehr entgegenkommend und Versprachen, für uns zu
tun, was sie könnten. Aber weil das Parlament gerade mit
aller Macht daran war, das päpstliche Komplott zu entdecken,
und die dabei Beteiligten herauszufinden, so benützten unsre Gegner,
die wussten, das wir nicht schwören noch die Waffen gebrauchen
dürfen, die Gelegenheit, die Strafen die über die Päpstlichen 
verhängt wurden, auch uns zuzuwenden, obgleich ihnen ihr Gewissen
sagte, das wir nicht päpstlich waren, und sie aus Erfahrung
wussten, das wir uns nicht an Verschwörungen beteiligten. Um
nun unsre Unschuld darzutun und unsern Gegnern das Maul zu
stopfen, verfasste ich ein kurzes Schreiben und lies es im 
Parlament vorweisen:
% \picinclude{./280-289/p_s281.jpg} 

\brief{Parlament}{

Wir haben den Grundsatz, uns von allen Verschwörungen\index{Verschwörung}
gegen den König oder seine Untertanen fernzuhalten; denn wir
haben den Geist Christi und durch ihn die Gesinnung Christi,
und er \zitat{ist gekommen, die Menschen zu erretten und nicht sie zu
verderben} (Luk. 9, 56\bibel{Luk. 09:56@Luk. 9:56}). Wir wollen die Sicherheit des Königs
und aller seiner Untertanen, darum erklären wir hiermit, das
wir trachten werden, alle Verschwörungen gegen ihn, von denen
wir hören, entdecken zu helfen. Solches versprechen wir euch
aufrichtig; was aber das Schwören\index{Schwören} und Kriegen\index{Krieg} anbelangt, so
können wir das ums unserer Gewissen willen nicht tun, wie ihr
ja wisset, denn wir haben alle diese Jahre hindurch um dieser
Weigerung willen viel gelitten. Wir hoffen nun, da euch der
Herr hier zusammengeführt hat, ihr werdet uns von diesen Leiden
befreien und nicht Dinge von uns verlangen, um deretwillen wir
schon so viel und so lange gelitten haben; dadurch würdet ihr
unsre Bande noch härter machen, als sie schon sind, anstatt sie
uns zu erleichtern.
\bigskip 

\begin{flushright}
G. F.\end{flushright}
}
