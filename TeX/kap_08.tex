%%%%%%%%%%%%%%%%%%% Kapitel 8. %%%%%%%%%%%%%%%%%%%%%%%%%%%%%%

\chapter[Brief an den Papst.]{Brief an den Papst.}

\begin{center}
\textbf{Brief an den Papst. Die Studenten von Cambridge. Die Quaker
in der Bibel. Wachsende Entfremdung von Cromwell.}
\end{center}

Es kam über mich vom Herrn, ein kurzes Schreiben 
aufzusetzen und zu verbreiten, als Ermahnung an den 
Papst\person{Papst} und alle
Könige und Herrscher von ganz Europa:

\brief{an alle Könige und Herrscher von ganz Europa}
{
    Freunde,

    \bigskip

    Ihr Häupter und Obersten, ihr Könige und Fürsten alle,
    verfolget nicht in Erbitterung und Eifer die Lämmer Christi; wendet
    euch nicht ab, wenn Gottes Stimme, seine innige Liebe und 
    Barmherzigkeit aus der Höhe euch ruft, auf das nicht sein Arm und
    seine Macht, die jetzt die Welt ergriffen 
    haben\index{Endzeiterwartung}, euch unversehens
    erfassen. Sie kehrt sich gegen die Könige, und die Weisen werden
    weichen müssen, und ihre Krone wird zu Staub werden; und sie
    werden erniedrigt und dem Erdboden gleich gemacht werden. Der
    Herr wird König sein und wird die Krone dem geben, der seinen
    Willen tut. Die Zeit ist gekommen, das Gott der Herr Himmels
    und der Erde die Stolzen entlarven wird und ihren Ruhm stürzen.

    Ihr, die ihr Christus bekennet und liebet doch eure Feinde nicht,
    sondern nehmet im Gegenteil seine Freunde gefangen, ihr zeiget
    damit, das ihr nicht in dem Leben seid, das aus ihm kommt, ihr.
    liebet Christus nicht, wenn ihr nicht seine Gebote haltet. Des
    Herrn Zorn fängt an zu brennen, und sein Feuer verbreitet
    sich, um die Bösewichter zu zerstören, und es wird kein Zweig
    noch Reis übrig lassen. Die so ihren Wandel nicht mehr in
    Gott haben, sind nicht mehr in jenem Geist, der die Schrift 
    eingegeben hat, und nicht mehr im Lichte, damit Christus sie alle
    erleuchtet hat [...]. Darum seid schnell zuhören, schnell zu reden,
    aber langsam zu verfolgen (Jak. 1:19\bibel{Jak. 01:19@Jak. 1:19}); 
    denn der Herr führt nun
    sein Volk aus den Wegen der Welt zu Christus dem wahren
    Weg, und von allen weltlichen Kirchen zu der Kirche, die in ihm,
    dem Vater Jesu Christi, ist, und von allen Lehrern der Welt,
    um selber ihr Lehrer zu sein durch seinen Geist; von den irdischen
    Bildnissen, zum Ebenbilde seiner selbst; und von den irdischen
    % \picinclude{./090-099/p_s097.jpg} 
    Kreuzen aus Holz und Stein, zu der Kraft des Kreuzes Christi.
    Denn alle diese Bilder und Kreuze sind ein Abfall von Gott und
    seiner Kraft und dem Kreuz Christi, welches nun die Welt richten
    wird und alles niederwerfen, was ihm entgegen ist; seine Macht
    hat kein Ende.

    Lasset solches die Könige von Frankreich und von Spanien
    und den Papst wissen, damit sie alles prüfen und das Gute behalten;
    Und sie sollen vor allem prüfen, ob sie nicht den Geist dämpften
    (1. Thess. 5:19\bibel{Thess. 1. 05:19@1. Thess. 5:19}), 
    denn der grose Tag des Herrn ist über die
    Bosheit und Gottlosigkeit und Ungerechtigkeit der Menschen
    gekommen, und der Herr wird \zitat{durchs Feuer richten und durch
    sein Schwert alles Fleisch} (Jes. 66:18\bibel{Jes. 66:18}). 
    Und die Wahrheit
    und die Krone der Ehren und das Zepter der Gerechtigkeit
    werden erhöht werden; und das Göttliche, das in einem jeden
    ist, auch wenn er davon abgefallen ist, wird hiervon Zeugnis
    geben. Christus ist als Licht in die Welt gekommen und erleuchtet
    einen jeden, der in die Welt kommt, damit dadurch alle zum
    Glauben kommen. Und wer das Licht, womit Christus ihn
    erleuchtet, spürt, der spüret Christus in seinem Innern und das
    Kreuz Christi, diese Kraft Gottes; der brauchet kein hölzernes oder
    steinernes Kreuz,\index{Kreuz!Schmuck}\index{Kreuz!Symbol} 
    um an Christus und sein Kreuz gemahnt zu
    werden; denn es ist selber die Kraft Gottes, welche sich ihm
    innerlich kund tut.
    \bigskip
    \begin{flushright}G. F.\end{flushright}
}

Ferner trieb es mich, einen Brief an den Protektor zu
schreiben, um ihn zu ermahnen, aus das große Werk zu achten,
das der Herr unter allen Völkern zu tun im Begriffe ist, und
aus das Beben, das sie alle erzittern macht, damit er auf der
Hut sei, das er nicht mit seinem scharfen Verstand, seiner 
Geschicklichkeit und seiner Klugheit selbstische Nebenzwecke verfolge.
Es wurde zu der Zeit eine Verordnung zur Prüfung der
sogenannten Geistlichen erlassen, ob man sie bestätigen oder ihrer
Ämter und Besoldungen entsetzen solle, und es trieb mich, den
betreffenden Vorgesetzten darum zu schreiben.

\brief{An die vorgesetzten der Geistlichen}{
    Freunde,\index{Klerus}

    \bigskip

    [...] Christus zeigt seinen Jüngern und dem Volk, wie
    man solche wie diese zu prüfen hat Sie werden von den Menschen
    Herr genannt. Sie sitzen auf den ersten Plätzen der 
    Versammlung; sie sind Hörer aber nicht Täter. Er rief 
    siebenmal Wehe!
    über sie und verurteilte sie (Matth. 23\bibel{Matth. 23}) [...]. 
    Es gab in alten Zeiten
    % \picinclude{./090-099/p_s098.jpg} 
    ein Kornhaus, wo die Waisen, die Fremdlinge und die Witwen
    hin kamen und zu essen bekamen, und die, welche ihre Zehnten
    nicht ins Kornhaus brachten, gediehen nicht 
    (Maleachi 3\bibel{Maleachi 3}); hat
    aber Christus nicht allen Zehnten und Priestern und Tempeln
    ein Ende gemacht? [...] Sind je die Priester, die Zehnten nach
    Menschensatzungen nahmen, gediehen? [...] Warfen die Apostel
    je jemanden in den Kerker wegen der Zehnten\index{Kirchensteuer}, 
    wie ihr es jetzt tut?
    Zum Beispiel: Ralph Hollingworth\person{Hollingworth, Ralph}, 
    Priester von Phillingham\ort{Phillingham},
    hat zu Lincoln einen armen Dachdecker namens Thomas 
    Bromby\person{Bromby, Thomas (Dachdecker)}
    wegen einer kleinen Abgabe, nicht mehr als sechs Schilling, ins
    Gefängnis geworfen, wo er nun schon seit achtunddreißig Tagen
    ist; und der Priester ersuchte den Richter, das man dem Mann
    nicht erlaube, etwas zu seinem Unterhalt im Gefängnis in der
    Stadt zu verdienen. Ist dieses eine Empfehlung für euch, die ihr
    die Aufgabe habt, die Priester zu wählen? [...] Christus hieß
    seine Jünger, als er sie aussendete, umsonst zu geben, wie sie
    umsonst empfangen hatten; und in den Städten, durch die sie
    zogen, mussten sie sehen, wer würdig war, und dort bleiben und
    essen, was man ihnen vorsetzte; und als sie zu Christus zurück
    kamen und er sie fragte, ob sie Mangel gelitten hätten, so sagten
    sie: \zitat{nein}. Sie gingen nicht in die Stadt und fragten die
    Leute, wie viel sie im Jahre bekommen, wie dies jetzt geschieht
    von denen, die abgefallen sind. Der Apostel sagt, \zitat{habe ich
    nicht zu essen und zu trinken?} aber er sagt nicht: 
    \zitat{habe ich nicht
    Osterpfründe, Aufbesserungen und Geldsummen} [...] \zitat{Es soll
    dem Ochsen, der da drischt, nicht das Maul verbunden 
    werden} (5. Mos. 25:4\bibel{Mos. 5. 25:04@5. Mos. 25:4}), 
    aber sehet zu, ob ihr auch gedroschen habt und
    ob das Korn in den Scheunen ist! Dies sagt einer, der eure
    Seelen lieb hat und euer ewiges Heil will.

    \bigskip

    \begin{flushright}G. F.\end{flushright}

}

\section{Fox zeigt sich bibeltreu}

Nachdem ich einige Zeit in London\ort{London} gewesen und dort gewirkt
hatte, trieb es mich nach Bedsordshire zu John 
Crook\footnote{John Crook, früher ein angesehener Friedensrichter 
der Grafschaft Bedford, wurde ein in vielen Verfolgungen 
standhafter Quäker.}\person{Crook, John} zu gehen,
wo eine große Versammlung war und viele die Wahrheit annahmen.
John Crook sagte mir, das am folgenden Tage mehrere Herren
der Umgegend mit ihm speisen werden, um mit ihm zu diskutieren.
Sie kamen und ich redete von der ewigen Wahrheit Gottes zu
ihnen. Mehrere Freunde gingen an jenem Tage ins Turmhaus.
% \picinclude{./090-099/p_s099.jpg} 
Und in der Umgegend war auch eine Versammlung und es trieb
mich hin zu gehen, obwohl es mehrere Meilen weit weg war.
John Crook ging mit mir. Es war einer dort, 
Gritton\person{Gritton}, der Baptist\index{Baptisten} gewesen, 
aber jetzt höher hinaus wollte und sich ein Prüfer
der Geister nannte. Er sagte den Leuten, wie viel Vermögen
sie haben, und behauptete, ihnen sagen zu können, wenn ihnen
etwas gestohlen oder verbrannt wurde, wer es getan.\index{Hellsehen}
Dadurch hatte er die Gunst vieler erworben. Dieser Mann redete gerade
laut, als ich kam. Er hieß Alexander Parker\person{Parker, Alexander} 
seine Hoffnung
begründen. Alexander erwiderte: \zitat{Christus ist unsere Hofstiung;}
weil diese Antwort nicht so schnell gegeben wurde, wie er sie
erwartete, so schrie er: \zitat{sein Mund ist gestopft!} Daraus richtete
er sich an mich, denn ich stand schweigend dabei, weil er vieles
sagte, das sich nicht mit der Schrift vertrug.\index{Bibeltreu} 
Ich fragte ihn,
ob er sich aus die Schrift berufen könne? er sagte: \zitat{ja;} ich hieß
die Leute ihre Bibeln nehmen und die Stellen aussuchen, die
er angeben würde, aber er konnte es nicht. So war er beschämt
und ging fort und seine Anhänger wurden meistens gewonnen [...].
John Crook\person{Crook, John} blieb in der Kraft Gottes, aber 
er wurde seines Amtes als Richter entsetzt [...].

\section{Die Studenten von Cambridge}

Ich ging nach Romney\ort{Romney}, wo die Leute von meinem Kommen
gehört, und es war darum eine sehr große Versammlung. Zu
dieser kam Samuel Fischer,\footnote{Samuel Fischer und John 
Stubbs gingen später u. a. nach Rom und
traten dort mutig gegen papistischen Aberglauben auf. 
Fischer starb 1665 im Gefängnis in London an der Pest.}
\person{Fischer, Samuel}\person{Stubbs, John} ein 
großer Baptistenprediger\index{Baptisten}. Er
hatte eine Pfarrei gehabt, die ihm etwa zweihundert Pfund im
Jahre eingebracht hatte und die er um des Gewissens willen 
aufgegeben hatte. Der Pfarrer der Baptisten war auch dabei und
viele ihrer Leute. Die Kraft des Herrn ward so mächtig kund,
das viele ergriffen wurden [...]. Als die Versammlung vorüber
war, sagte Samuel Fischers Frau: \zitat{so, nun last uns darüber
reden, was geistig und was fleischlich ist, damit wir die Lehre
des Geistes von der Lehre des Fleisches unterscheiden können.}.
Samuel Fischer und manche andere traten für das Wort des
Lebens ein, das an diesem Tage ihnen war erklärt worden. Der
andere Pfarrer und seine Anhänger redeten dagegen [...].
Samuel Fischer nahm die Wahrheit an und wurde ein getreuer
Prediger; er predigte umsonst und arbeitete viel für den Herrn;
% \picinclude{./100-109/p_s100.jpg} 
denn es trieb ihn, das Wort des Lebens in Dunkirk 
und Holland\ort{Holland} zu verkünden und in einigen 
Teilen Italiens\index{Italien}, sogar in Rom\index{Rom}.
Doch der Herr bewahrte ihn und seinen Begleiter John Stubbs
vor der Inquisition [...].


An einem sechsten Wochentage hatte ich eine Versammlung
in Colchester\ort{Colchester}, zu der viel \textit{Fromme} 
und die Lehrer der Independent kamen. Als ich zu reden 
aufgehört hatte und meinen Platz verließ, fing einer 
der Independentenlehrer\index{Independent} an Lärm zu
machen; Amor Stoddart\person{Stoddart, Amor}, der dies 
hörte, sagte zu mir: \zitat{Steh noch einmal aus, George,} 
denn ich hatte eben fort gehen wollen.
Ich stand nun wieder auf, als ich die lärmenden Independenten
hörte; und bald kam die Macht des Herrn über ihn und über
alle und überwältigte sie [...].


Am nächsten Ersten Tage hatten wir eine große Versammlung
in Colchester\ort{Colchester} [...]. Von da gingen wir 
nach Ipswich\ort{Ipswich} [...] dann
nach Mendlesham\ort{Mendlesham}, wo wir ein große Versammlung 
hatten; dann gingen wir nach Norfolk\ort{Norfolk}, wo wir 
uns von Amor Stoddart\person{Stoddart, Amor} verabschiedeten, 
der uns später wieder treffen wollte [...]. Dann
zogen wir nach Yarmouth\ort{Yarmouth} [...] und 
Norwich\ort{Norwich}, [...] und von
dort nach Lynn\ort{Lynn} [...]. Von Lynn gingen wir nach 
Cambridge\ort{Cambridge}.


Als ich in diese Stadt kam, waren die 
Studenten\index{Studenten}, die von
meinem Kommen gehört hatten, in Aufregung und benahmen
sich sehr ungezogen; ich hielt mich auf meinem Pferde und ritt
mitten durch sie hindurch in der Kraft des Herrn; 
Amor Stoddart\person{Stoddart, Amor}
aber warfen sie vom Pferd, ehe er die Herberge erreichte. Als
wir in der Herberge waren, taten sie so wüst im Hof und in den
Straßen, das Fuhrleute und Kohlengräber nicht wüster hätten
tun können. Die Wirtsleute fragten uns, was wir zum Nacht
essen haben wollten; ich erwiderte: \zitat{wenn nicht Gottes Macht
größer wäre als diese rohen Studenten, so würden sie uns sicher
gerne in Stücke reisen und ein Nachtessen aus 
uns machen.}\index{Gott!Macht} Sie
wussten, das ich sehr gegen das Gewerbe des 
Predigens\index{Predigt} war, das
sie dort als Lehrjungen erlernen sollten; darum tobten sie gegen
mich, wie nur je die Handwerksleute der Diana gegen den Paulus
(Act. 19\bibel{Act. 19}). 


In der Nacht kam der 
Stadtbürgermeister, der es gut mit mir meinte, und 
holte mich zu sich heim. Als wir durch
die Straße gingen, war großer Lärm in der Stadt, aber man
erkannte mich nicht, weil es finster war. Man war auch über
den Bürgermeister zornig, so das er sich sehr fürchtete, mit mir
% \picinclude{./100-109/p_s101.jpg} 
über die Straße zu gehen. Nachher ließen wir dann die Freunde
holen und hatten eine schöne Versammlung in der Kraft des
Herrn, und ich blieb die ganze Nacht in der Stadt. Wir hatten
unsere Pferde für den nächsten Morgen um sechs Uhr bestellt
und ritten in Frieden zur Stadt hinaus; die Störenfriede wurden
somit enttäuscht, denn sie hatten geglaubt, ich würde länger da
bleiben, und hatten beabsichtigt, uns etwas anzutun; aber unsre
frühe Abreise vernichtete ihre bösen Anschläge\index{Anschlag} [...].

\section{Briefe an Spötter und den Protektor}


Es trieb mich, ein Schreiben zu senden an die, welche über
das Zittern und Beben (quake) spotteten:
\buch{Schreiben an die Spötter\index{Verteidigungsschrift}}
{
    Ein Wort vom Herrn an euch, die ihr über das Zittern\index{Zittern}
    und Beben spottet; und die ihr solche, welche zittern und beben,
    verhöhnt, schlagt, bedroht und Verwünschungen gegen sie 
    ausstosset. Ihr kennet alle die Apostel und Propheten nicht! [...]
    Moses, der ein Richter über Israel war, zitterte und bebte,
    als der Herr zu ihm sagte: \zitat{ich bin der Gott Abrahams, Isaaks
    und Jakobs} (2. Mose 3) [...] . Der König David zitterte;
    und sie verspotteten ihn (Ps. 38\bibel{Ps. 38}) [...] 
    Hiob zitterte, bebte;
    und sie verlachten ihn (Hiob 21\bibel{Hiob 21}) [...]. 
    Der Prophet Jeremia
    bebte; es schüttelte ihn, seine Glieder zitierten, und er taumelte
    hin und her wie ein trunkener Mann 
    (Jer. 23:9\bibel{Jer. 23:09@Jer. 23:9}), als er die
    Betrügerei der Priester und Propheten sah, die sich vom Herrn
    abgekehrt hatten [...]. Jesaia sagte: \zitat{Höret was der Herr
    sagt, ihr, die ihr erzittert bei seinem Wort;} und weiter sagte er:
    \zitat{Ich sehe an den Elenden und der zerbrochenen Geistes ist und
    der erzittert bei meinem Wort} 
    (Jes. 66:2\bibel{Jes. 66:02@Jes. 66:2}) [...]. Habakuk, der
    Prophet des Herrn zitterte [...]. Und Joel, der Prophet des
    Herrn sagte: \zitat{Blaset mit der Posaune zu Zion, erzittert alle 
    Einwohner im Lande} (Joel 2:1\bibel{Joel 02:01@Joel 2:1}) 
    [...]. Daniel, ein Diener des
    Allerhöchsten, zitterte, und er hatte keine Kraft mehr 
    (Dan. 10:16\bibel{Dan. 10:16});
    und er war gefangen, gehasst und verfolgt [...].
    Paulus, ein Apostel Jesu Christi durch den Willen Gottes, ein
    auserwähltes Rüstzeug des Herrn, das er seinen Namen trage in
    alle Lande, zitterte [...] und sagte, als er zu den Corinthern
    kam: \zitat{ich war bei euch in Schwachheit und Furcht und großem
    Zittern} (1. Corinth. 
    2:3\bibel{Corinth. 1. 02:03@1. Corinth. 2:3}) [...].
    Hütet euch darum, ihr Großen der Erde, die zu Verfolgen,
    welche man zum Spott Quäker (Zitterer) nennt, die aber in der
    Kraft Gottes sind damit sich die Hand des Herrn nicht gegen
    % \picinclude{./100-109/p_s102.jpg} 
    euch kehre und euch verderbe. Es ergeht das Wort des Herrn
    an euch: fürchtet euch und zittert und hütet euch! denn der Herr
    siehet den am, der erzittert bei seinen Wort 
    (Jes. 66:2\bibel{Jes. 66:02@Jes. 66:2}); ihr aber,
    die ihr von dieser Welt seid, verspottet, verlacht, verhöhnt, 
    verfolgt ihn und nehmt ihn gefangen. Daran könnt ihr sehen, das
    ihr den Propheten und Aposteln zuwider handelt, wenn ihr die hasset,
    die der Herr ansieht, während wir, die ihr im Spott Quäker
    nennt, sie achten. Wir ehren und preisen die Macht, die den
    Teufel erzittern macht, die Erde erbeben lässt und den Stolz und
    Hochmut niederschmettert, die Tiere auf den Feldern erzittern
    macht und die Erde wanken (Jes. 2:11\bibel{Jes. 02:11@Jes. 2:11}). 
    Diese Kraft ehren
    und verkünden wir; aber alle, die spotten und höhnen und
    peitschen und plagen, die verabscheuen\person{Fox!verabscheut} 
    wir; denn alle, die solche-?
    tun und es nicht bereuen, werden das Reich 
    Gottes\index{Reich Gottes} nicht ererben,
    sondern das Verderben (2. Tess. 1\bibel{Tess. 2. 01@2. Tess. 1}).
    Selig aber sind, die um der Gerechtigkeit willen verfolgt
    werden; sie werden ihren Lohn im Himmel haben 
    (Matth. 5:12\bibel{Matth. 05:12@Matth. 5:12}).

    \bigskip

    \begin{flushright}G. F.\end{flushright}
}

Im Jahre 1655\jahr{1655} wurde der 
Abschwörungseid\index{Eid} gefordert, wodurch
viele Freunde zu leiden hatten; und viele gingen zum Protektor, um
mit ihm darüber zu sprechen; aber er fing an, härter zu werden.
Durch die Art, in der die gehässigen Beamten den Eid als
Schlingen gebrauchten, um die Freunde darin zu fangen, weil sie
wussten, das sie nicht schwören durften, nahmen die Leiden der
Freunde immer mehr zu, und es trieb mich, 
dem Protektor\person{Protektor} folgendes zu schreiben:

\brief{Schreiben an den Protektor}
{
    Die Obrigkeit\index{Obrigkeit}\index{Staatsgewalt} soll 
    das Schwert, das den Übeltätern\index{Kriminalität} ein
    Schrecken sein soll, nicht umsonst tragen; wie die Obrigkeit, die
    das Schwert umsonst trägt, den Übeltätern kein Schrecken ist, so
    ist sie auch kein Zeichen des Ruhmes für den, der recht tut;
    Gott hat nun durch seine Macht ein Volk erwecket, welches
    die Priester, die Obrigkeit und das Volk in ihrem Ärger 
    \zitat{Quäker} nennen. Dieses schreit gegen die Trunksucht 
    \index{Alkohol} und das Schwören\index{Eid};
    die Trunkenbolde aber, denen das Schwert der Obrigkeit ein
    Schrecken sein sollte, gehen, wie wir sehen, frei umher; von denen
    jedoch, die gegen dieses Laster eisern, kommen viele ins Gefängnis,
    weil sie Zeugnis ablegen gegen den Stolz, die Unreinheit, gegen
    das betrügerische Handeln auf den Märkten, gegen Ausschweifung
    und Leichtfertigkeit, gegen das Spiel\index{spielen} mit 
    Kegeln, Würfeln und
% \picinclude{./100-109/p_s103.jpg} 
    Karten und andere eitle und sündliche Vergnügen [...].
    Das Schwert der Obrigkeit wird, wie wir sehen, vergeblich 
    getragen, während die Übeltäter frei sind, Böses zu tun; die aber,
    welche gegen das Böse eifern, werden dafür bestraft von der
    Obrigkeit, die ihr Schwert gegen den Herrn kehrt [...]. Es haben
    viele große Strafen erlitten, darum, das sie nicht schwören konnten
    sondern der Lehre von Christus gehorchten, welche sagt: 
    \zitat{ihr sollt überhaupt nicht schwören}; sie sind 
    ein Raub geworden (Jes. 12:22\bibel{Jes. 12:22}),
    weil sie das Gebot Christi hielten. Es werden viele ins Gefängnis
    geworfen, weil sie den Abschwörungseid nicht leisten 
    können, obgleich
    sie alles missbilligen, was man darin abschwört; und es werden
    viele Diener und Boten des Herrn ins Gefängnis geworfen, weil sie
    nicht schwören wollen, noch Christi Gebot übertreten. Darum bedenke
    du dich doch! ich wende mich an das, was von göttlichem Leben
    in dir ist, Viele sind auch im Kerker, weil sie den Priestern die
    Zehnten\index{Kirchensteuer} nicht bezahlen können; viele 
    hat man ihrer Habe beraubt und
    dreifache Abgaben von ihnen gefordert; viele werden gepeitscht und
    geschlagen in den Korrektionshäusern, ohne das dadurch ein Gesetz
    übertreten würde. Solche Dinge tut man in deinem Namen, damit
    man bei solchem Tun geschützt sei. Wenn gottesfürchtige Männer
    das Schwert trügen, wenn das Unrecht bestraft würde und 
    gottesfürchtige Männer angestellt würden, dann würden sie den 
    Übeltätern ein Schrecken sein und ein Ruhm denen, 
    die Recht tun, statt
    ihnen Leiden zu verursachen. Dann würde Gerechtigkeit in unserm
    Lande herrschen und die Rechtschaffenheit sich erheben und 
    ausbreiten, welche das Unrecht nicht zulässt, sondern es richtet. 
    Ich
    rede zu dem, was vom Geiste Gottes in dir ist, das du in dich
    gehen und für Gott regieren mögest, damit du dem Göttlichen,
    das in eines jeden Menschen Gewissen 
    ist\index{Innerers Licht}, folgen mögest, denn
    dieses macht, das man alle Menschen achtet in dem Herrn. Siehe
    doch zu, für wen du regierest, aus das du Kraft vom Herrn
    empfangen mögest, für ihn zu herrschen, und alles, was wider ihn
    ist, durch sein Licht verdammt\person{Fox!Verdammung} werde.
    Von einem, der deine Seele lieb hat, und dein ewiges Bestes
    wünscht.

    \bigskip

    \begin{flushright}G. F.\end{flushright}

}
