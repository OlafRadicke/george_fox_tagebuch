
% \picinclude{./020-029/p_s020.jpg} 
20 Kapitel 11.
Lohn zu verkürzen, sondern ihnen zu geben, was recht und billig
sei, und die Dienstboten ermahnte ich, ihre Pflicht zu tun und
ehrlich zu dienen; sie nahmen meine Mahnungen freundlich auf,
denn ich wurde vom Herm dazu getrieben.
Ferner trieb etz mich, an verschiedene Gerichtöhöfe und in ver-
schiedene Turmhäuser in Manöfield und an andern Orten zu gehen,
um alle zu ermahnen vom Unterdrücken und vom Schwören abzu-
lassen und sich von der Ungerechtigkeit zum Herrn zu bekehren und
recht zu tun. Jnßbesondere trieb es mich, nach einer Gerichtßver-
handlung in Manöfield zu einem zu gehen, der einer der schlech-
testen Menschen der dortigen Gegend war, und mit ihm zu reden;
er war ein Säufer und berüchtigte: Mädchenhändler; ich warnte ihn
beim allmächtigen Gott wegen s eines schlechten Wandels; als ich auß-
geredet hatte und ihn Verlassen wollte, lies er mir nach und sagte
mir, während ich mit ihm gesprochen habe, sei er so ergriffen worden,
daß ihn seine Kräfte ganz verließen. So wurde dieser Mann be-
kehrt, und er ließ ab von seiner Schlechtigkeit und blieb rechtschaffen
und nüchtern zum Erstaunen aller, die ihn vorher gekannt hatten.
Und das Werk des Herrn nahm zu und viele kamen von der Finster- ,
nie zum Licht, im Laufe dieser drei Jahre 1646, 1647 und 1648.
GS wurden in dieser Zeit mehrere Versammlungen für Freunde ein-
gerichtet, damit Gott sich kund tue durch sein Licht, seinen Geist
und seine Kraft; denn dee Herrn Kraft brach immer herrlicher hervor.
Nun war ich ini Geiste bei Idem stammenden Schwert vorbei
inö Paradieß Gotteö eingedrungen. Alle Dinge waren wie um-
gewandelt ftir mich und die ganze Schöpfung hatte einen andern
Geruch für mich, über alles waß Worte außdrücken können. Ich
wußte nur noch von Reinheit, Unschuld und Rechtschaffenheit, denn
ich war erneuert zum Ebenbild Gotteß (Col. 3, 10) durch Christus,
in den Zustand, in dem Adam vor dem Fall gewesen war. Die
ganze Schöpfung wurde mir offenbar und es- wurde mir gezeigt,
wie alle Dinge mit dem Namen genannt wurden, der ihrem
Wesen und ihren Kräften entsprach. Jch war unschliisstg, ob ich
nicht sollte Heilkunde treiben zum Nutzen der Menschheit, als ich
sah, wie die Natur und die Kräfte aller Dinge mir so geoffenbart
wurden vom Herrn. Aber alsbald wurde ich ergriffen im Geist
und erkannte einen andern, sicherem Zustand als die Sitndlosig-
keit Adams, den Zustand Jesu Christi, der nicht fallen konnte.
Und der Herr zeigte mir, daß die, so ihm treu bleiben im Licht


% \picinclude{./020-029/p_s021.jpg} 
Erste Versammlungen und Proteste. 21
und in der Kraft Christi, erhoben werden in den Zustand, darin
Adam vor dem Fall gewesen war, in welchem die bewundernß-
werten Werke der Schöpfung und ihre Kräfte erkannt werden
können durch die Offenbarung deß göttlichen Worteß der Weiß-
heit und der Kraft, durch welche sie gemacht waren. Der Herr
führte mich in große Dinge ein, und wunderbare Tiefen wurden
mir geoffenbart, die alleß iibertrafen, waß Worte beschreiben
können. Aber wer sich dem Geist Gotteß unierwirst und hinein-
wächst in daß Gbenbild und die Kraft deß Allmächtigen, der wird
daß Wort der Weißheit empfangen, daß alle Dinge offenbar macht,
und wird dazu gelangen, die verborgene Einheit in dem ewigen
Wesen zu erkennen.
So reiste ich umher im Dienste deß Herrn, wie mich der
Herr führte. Alß ich nach Nottingham kam, war Gotteß mächtige
Kraft mit den Freunden. Von da ging ich nach Elawson in
Leieestershire im Tale Veavor, und auch dort wirkte die Kraft
Gotteß in Verschiedenen Städten und Dörfern, in denen Freunde
beisammen waren. Während ich dort war, offenbarte mir der
Herr drei Dinge, die sich auf die drei großen Berufßarten in der
Welt — Heilkunde, sogenannte Gotteßgelehrtheit und Recht?--H
wissenschast bezogen. Er zeigte mir, daß die Ärzte nicht die
Wei?-heit Gotteß haben, durch die alle Kreatur geschaffen ist, und
daß sie darum ihre Kräfte nicht kennen, weil sie nicht im Worte der
Weiß-heit sind, durch daß alleß gemacht ist. Gr zeigte mir, daß
die Priester nicht den wahren Glauben haben, dessen Ursprung
Christus ist; den Glauben, der reinigt und den Sieg gibt und
durch des man Gott gefällt, welcheß Geheimniß deß Glaubenß
in reinem Gewissen ist (1. Tim. 3, 9). Gr zeigte mir ferner, daß
die Rechtßgelehrten nicht die wahre Villigkeit und Gerechtigkeit
besitzen und nicht daß Gesetz Gotteß haben, nach welchem schon
die erste Ubertretung und alle weiteren Sünden gerichtet worden
sind und welcheß dem Geiste Gotteß entspricht, den die Menschen
in sich betrüben und gegen den sie sündigen (Eph. 4, 30).
Und daß diese drei, die Ärzte, die Priester und die Rechtßgelehrten,
die Welt ohne Weißheit regieren, ohne Glauben, ohne Billigkeit,
ohne Recht und ohne daß Gesetz Gotteß; die einen, indem sie
vorgeben, den Leib zu heilen, die andern die Seele und die dritten
daß Eigentum der Leute zu schützen. Aber ich sah, daß sie alle
die Weißheit, den Glauben, die Gerechtigkeit und daß GesetzZGotteS


% \picinclude{./020-029/p_s022.jpg} 
22 Kapitel 11.
nicht hatten. Und als der Herr mir diese Dinge osfenbarte, fühlte
ich, daß seine Kraft sich über alle ergoß und daß sie durch die-
selbe alle umgewandelt werden könnten, wenn sie sie aufnehmen und
sich ihr beugen würden. Die Priester würden umgewandelt werden
und zum wahren Glauben kommen, welcher eine Gabe Gottes
ist. Die Rechtsgelehrten würden umgewandelt werden und zum
Gesetz Gottes (Jar. 2, 2) kommen, welches dem göttlichen im
Herzen entspricht und es möglich macht, seinen Nächsten wie sich
selbst zu lieben. Dieses Gesetz läßt den Menschen erkennen, daß
wenn er seinem Nächsten schadet, so schadet er sich selber, und
es lehret ihn, andern zu tun, wie er möchte, daß die andern ihm
tun. Die Ärzte können umgewandelt werden und zur Weisheit
Gottes kommen, durch die alle Dinge geschaffen sind, und so
eine rechte Erkenntnis über diese Dinge erlangen und ihre Kräfte
erkennen an den Namen, die die Weisheit, die sie gemacht, ihnen
gab ....
Der Herr offenbaite mir durch seine unsichtbare Kraft, daß
ein jeder erleuchtet werde durch das heilige Licht Christi (Joh. 1, 9).
Und ich erkannte, daß es in allen leuchtet, und daß alle, die
daran glaubten, aus der Verdammnis zum Licht des Lebens
kamen und Kinder des Lichts wurden (Joh. 12, 36). Aber die,
welche es haßten und nicht daran glaubten, die verdammte es, wie-
wohl sie schienen Christum zu bekennen.,« Solches sah ich in der
reinen Offenbarung des Lichts, ohne jegliche menschliche Hilfe;
auch wußte ich damals nicht, wo es in der Schrift zu sinden
war; doch später, als ich in der Schrift forschte, fand ich es.
Damals aber hatte ich jenes Licht und jenen Geist geschaut, welche
gewesen, ehe die Schrift gegeben worden war, und welche die
heiligen Männer Gottes getrieben hatten, die Schrift zu schreiben;
und ich erkannte, daß alle, welche Gott, Christus oder die Schrift
recht kennen wollen, zu diesem Geist gelangen müssen. Aber ich
merkte eine Trägheit und faule Schläfrigkeit in den Leuten, die
mich erstaunte; oftmals, wenn ich einschlafen wollte, schweifte
mein Geist über alles hinaus zu dem, der von Ewigkeit zu Ewig-
keit ist. Jch sah, daß der Tod über diesen schltisrigen und faulen
Zustand kommen mußte, und ich sagte den Leuten, sie müßten
dazu kommen, dieses schläfrige, träge Wesen zu töten und zu
kreuzigen durch die Kraft Gottes, damit ihre Herzen und Sinne
droben seien.


% \picinclude{./020-029/p_s023.jpg} 
Erste Versammlungen und Proteste. 23
Einmal alß ich durchs Feld wanderte, sagte der Herr zu mir:
,,Dein Name ist geschrieben im Lebenßbuche deß Lammeö, welcheö
gewesen vor der Erschaffung der Welt«. Alk- der Herr dietz sagte,
da glaubte ich e3 und erkannte es, kraft der neuen Geburt. Einige
Zeit darauf befahl mir der Herr, in die Welt hinaus zu gehen,
die wie eine dornige Wildniß war; und alß ich in der Kraft
Gottes mit dem Wort des Lebenö in die Welt hinauß kam, lehnte
sich die Welt dagegen auf und tobte wie die großen tobenden
Wogen der See; Priester wie ,,Fromme«, die Obrigkeit wie das
Volk, alle waren wie die See, als ich kam, den Tag deß Herrn
unter ihnen zu verkünden und ihnen Buße zu predigen ......
Als mich Gott und sein Sohn Jesuß Christuß außsandten
in die Welt, um sein ewigeö Evangelium und Reich zu predigen,
freute ich mich, daß ich den Befehl hatte, die Leute jenem innern
Licht, Geist und Gnade zuzuführen, durch die alle ihr Heil und
den Weg zu Gott erkennen können; ja, jenem heiligen Geist,
der in alle Wahrheit führt und von welchem ich bestimmt wußte,
daß er nie jemanden trtigt.
Durch diese göttliche Kraft und den Geist Gottes und daß
Licht Jesu sollte ich nun die Menschen von ihren eigenen Wegen
ab zu Christus?-, dem neuen, lebendigen Weg bringen; ab von
ihren Kirchen von Menschen gemacht, zur Kirche in Gott, zur
Gemeinde derHeiligen, die imHi1nmel angeschrieben ist (Gbr. 12, 23),
deren Haupt Ehristuß ist; ab von den Lehrern dieser Welt, die
von Menschen eingesetzt sind, damit sie von Ehristus:3 lernen, der
der Weg, die Wahrheit und daß Leben ist (Joh. 14, 6), von welchem
der Vater sagt: ,,dieS ist mein lisber Sohn, den höret« (Luc. 9, 35);
ab von allem weltlichen Gottezdienst, damit sie den Geist der Wahr-
heit in ihrem Jnnern erkennen und sich von demselben führen
lassen; daß sie in demselben den Vater der Geister anbeten, dem
solcheß anbeten angenehm ist; die, welche nicht in diesem Geiste
anbeten, wissen nicht, maß sie anbeten. Jch sollte die Menschen
abbringen von all den Gottesdiensten dieser Welt, welche eitel
sind, damit sie zu dem wahren Gotteßdienst kommen, welcher die
Witwen und Waisen in ihrer Trübsal tröstet (Jar. 1, 27) und be-
wahret von der Befleckung der Welt; dann gäbe es:) nicht so viele
Bettler, deren Anblick so ost mein Herz betrübt, weil er von so
viel Hartherzigkeit zeugt unter denen, die vorgeben, C-hristus3 zu
bekennen. Ich sollte sie von allen Gemeinschaften, Singereien


% \picinclude{./020-029/p_s024.jpg} 
24 Kapitel ll.
und Betereien dieser Welt abbringen, welche Formen ohne Kraft
sind, auf daß ihre Gemeinschaft im heiligen Geist sei, im ewigen
Geist Gotteö, und sie darin anbeten und singen, durch die Gnade,
die von Christus kommt; und so dem Herm in ihren Herzen
singen’und spielen, der seinen geliebten Sohn gesandt hat, um
ihr Retter zu sein; der seine himmlische Sonne über und in allen
scheinen läßt und seinen himmlischen Regen über Gerechte und
Ungerechte außgießt (Matth. 5), wie der äußere Regen über alle
fällt und die äußere Sonne fiir alle scheint; dietz ist Gotteö un-
außsprechliche Liebe zur Welt. Jch sollte die Leute von den
jüdischen Zeremonien abbringen und von den heidnischen Fabeln
und den menschlichen Einrichtungen und weltlichen Lehren, durch
welche die Leute hin und her von einer Sekte zur andern ge-
trieben werden, und von allen ihren bettelhaften Lehranstalten
und ihren Schulen und Hochschulen, in denen sie Prediger Christi
machen wollen, die aber wahrlich Prediger ihrer eigenen Machen-
schaft sind und nicht Christi; von allen ihren Bildern und Kreuzen
und Besprengen von Kindern; allen ihren sogenannten heiligen
Tagen und nichtigen Traditionen, die sie seit den Tagen der
Apostel eingerichtet haben und gegen welche die Kraft Gottes
sich richtet; vermöge dieser Kraft wurde ich getrieben, gegen
alleß daß aufzutreten und gegen alle, die nicht umsonst pre-
digten und doch solche waren, die umsonst vom Herrn empfangen
hatten. s
Ferner verbot mir der Herr, als er mich in die Welt hinauö
sandte, meinen Hut abzunehmen vor irgendjemand, hoch oder
niedrig; und ich hatte den Befehl, zu allen, Männern und Frauen,
,,Du« zu sagen, ohne irgend einen Unterschied zu machen zwischen
reich oder arm, groß oder klein; und ich sollte unterwegs- auf
meinen Reisen den Leuten nicht guten Morgen oder guten Abend
sagen, noch mich vor irgendjemand neigen oder daß Knie beugen.
Solcheö machte die Sekten und Gemeinschaften zornig. Aber die
Kraft des Herrn half mir durch alleß hindurch, zu seiner Ehre,
und viele kehrten sich in kurzer Zeit zu Gott, denn der große
Tag des- Herrn ging auf auß der Höhe und brach eilendö an,
und in seinem Lichte gingen vielen die Augen über ihren Zu-
stand auf.
Aber o, die Wut, in welcher damals- Priester, Obrigkeit,
»Fromme« und andere waren! Aber hauptsächlich die Priester


% \picinclude{./020-029/p_s025.jpg} 
Erste Versammlungen und Proteste. 25
und die ,,Frommen«; denn obgleich das- ,,Du« gegen eine ein-
zelne Person ihrer eigenen Grammatik und Formenlehre, sowie
auch der Bibel entsprach, so konnten sie sich doch nicht drein
finden, es zu hören; und maß die Hut-Ehre anbetraf, daß ich
den Hut nicht vor ihnen abnehmen konnte, das machte sie ganz
wütend ....
In jener Zeit fühlte ich mich, zu meiner schweren Prüfung,
auch berufen, in die Gerichtßhöfe zu gehen, um nach Gerechtigkeit
zu schreien und die Richter und Behörden in Wort und Schrift
zur Gerechtigkeit zu mahnen; ich mußte solche, die öffentliche Gast-
häuser hielten, ermahnen, den Leuten nicht mehr zu trinken zu
geben, als ihnen gut sei; ich mußte auftreten gegen ihre Feste
und Gelage, Spiele, Späße und Belustigungen aller Art, durch
die die Leute zur Eitelkeit und Liederlichkeit verleitet und von
der Gotteßsurcht abgebracht wurden; am häufigsten schändeten
sie Gott (Röm. 2, 23) in dieser Weise an den Tagen, die sie als-
heilige bezeichneten. Auch an Jahrmärkten und Märkten mußte
ich mich gegen ihr trügerischeö Handeln wenden, ihren Schwindel
und Betrug; ich mußte sie mahnen, die Wahrheit zu sagen, ihr
ja—ja und ihr neinsnein sein zu lassen, und andern zu tun, wie
sie wollten, daß man ihnen tue, alleß indem ich sie an den großen
Tag dez Herrn erinnerte, der über sie alle kommen werde. Auch
gegen allerlei Musizieren und gegen die Schwindler, die in den
Vuden ihr Wesen trieben, mußte ich auftreten, denn sie gefähr-
deten die Unschuld und reizten den Sinn der Leute zur Eitelkeit.
Jch mußte auch manchen schweren Gang zu Lehrern und Lehrerinnen
tun, um sie zu erinahnen, die Kinder in der Furcht deö Herm zu
erziehen, damit sie nicht in Eitelkeit, Leichisinn und Schlechtigkeit
aufwachsen. Ebenso mußte ich Lehrer und Lehrerinnen, sowie die
Väter und Mütter ermahnen, darauf zu achten, daß man die
Kinder und die Dienstboten daheim im Hanse zur Gotteßstircht an-
halte, damit sie Vorbilder der Tugend und Mäßigkeit werden.
Die irdische Gesinnung der Priester tat mir weh, und wenn
ich die Glocken läuten hörte, welche die Leute inö Turnthauß
rufen sollten, ging es mir durch Mark gund Bein, denn eS war
gerade wie eine Marktglocke, welche die Leute zusammenruft, daß I
der Priester seine Ware Izum Verkauf außbieteu kann. O, die
großen Geldsummen, die zusammenkamen durch ihr Handeln mit
Bibeln und durch ihr Predigen, vom höchsten Bischof biz zum


% \picinclude{./020-029/p_s026.jpg} 
26 Kapitel lll.
einfachsten Priester! Wa;-’ für ein Handel in der Welt kommt
diesem gleich! Und doch wurde die Schrift gegeben umsonst! Und
Christus hatte seinen Jüngern befohlen, umsonst zu predigen;
und die Propheten und Apostel verkündeten allen geizigen Miet-
lingen und allen, die für Geld iveiösagten, daß Gericht. Jch
aber wurde au?-gesandt, in diesem freien Geist daß Wort vom
Leben und der Versöhnung umsonst zu predigen, auf daß alle zu
Christus kommen, welcher umsonst gibt und in daß E-benbild
Gotteß erneuert, nach dem Mann und Weib geschaffen waren
vor dem Fall, auf daß sie himmlische Güter in Jesuß Ehristuß
haben möchten.
Kapitel lll.
Der Tumult in Nottingham. Wachsender Widerstand, bis zum
Gefängnis in Derby.
A16 ich einmal am Morgen einetz Ersten Tageß in der Nähe
von Nottingham von einem Hügel auß die Stadt überblickte, da
gewahrte ich daß riesige Turmhauß, und der Herr sagte zu mir:
,,Du mußt hingehen und gegen jene großen Götzen schreien und
gegen die, welche drinnen anbeten«. Jch sagte den ,,Freunden«,
die mit mir waren, nichtß davon, sondern ging mit ihnen hin in
die Versammlung, wo die mächtige Kraft der- Herrn mit uns
war; hier ließ ich sie und ging zum Turmhauß. Die Menge,
die ich hier sah, kam mir vor wie ein Brachfeld und der Priester
wie ein großer Erdklumpen, der oben aus seiner Kanzel stand.
Gr hatte zum Text die Worte des Petruß: »Wir haben ein festes-
prophetischeö Wort und ihr tut wohl, daß ihr daraus achtet, alß
auf ein Licht, das da scheinet an einem dunkeln Ort, biß der Tag
anbreche und der Morgenstern ausgehe in eueren Herzen«
(2. Petr. 1, 191. Er sagte den Leuten, nach dem, waß hier ge-
schrieben stehe, sollten sie alle Lehren, Bekenntnisse und Meinungen
prüfen. Da kam die Kraft dez Herrn so mächtig über mich und
war so stark in mir, daß ich nicht an mich halten konnte, sondern
rusen mußte: ,,O nein, nicht nach dem, was geschrieben stehet!«
und ich sagte ihnen, nach maß: nämlich nach dem heiligen Geist,
durch den die heiligen Männer Gotteö die Schrift geschrieben
haben. Durch diesen, sagte ich, müssen alle Lehren, Bekenntnisse
und Meinungen geprüft werden. Dieser Geist leitet in alle


% \picinclude{./020-029/p_s027.jpg} 
Der Tuniult in Nottingham. Wachsender Widerstand usw. 27
Wahrheit und zur Erkenntniß aller Wahrheit. Die Juden haben
die Schrift gehabt und widerstanden dem heiligen Geist doch und ver-
warfen Christuö, den schönen Morgenstern ; sie verfolgten Ehriftuö
und seine Apostel und wollten ihre Lehren nach der Schrift prüfen;
aber sie irrten in ihrem Urteil und prüften sie nicht richtig, weil
sie ohne den heiligen Geist pritften. Da ich nun so zu ihnen
redete, kamen die Wachen und führten mich weg und brachten
mich in einen wüsten, stinkenden Kerker; der Geruch stieg mir so
in die Nase und den Halß, daß etz- eine Qual war, aber die
Kraft deß Herrn schallte an dem Tage so in ihren Ohren, daß
sie ganz von dem Schall betäubt waren, und ihre Ohren wurden
noch eine zeitlang nicht frei davon, so waren sie im Turmhause
von der Kraft dez Herrn ergriffen worden. Am Abend brachten
sie mich vor die Behörden der Stadt; als ich vor sie trat, war
der Bürgermeister in verdrießlicher, mürrischer Laune, aber die
Kraft deß Herrn beschwichtigte ihn. Sie verhörten mich ausführ-
lich und ich berichtete ihnen, wie der Herr mich getrieben hatte
zu kommen. Nach einigem Hin- und Herreden schickten sie mich
in?-’ Gefängniß zurück. Aber bald darauf ließ mich der Oberscheriff,
John Neckleß, zu sich in sein Haus holen. Als ich eintrat, be-
gegnete mir sein Weib im Flur und sagte: ,,Unserm Hause ist
Heil widerfahren.« Sie reichte mir die Hand und war mächtig
ergriffen von der Kraft Gotte;-’, und ihr Mann und ihre Kinder
und Dienstboten wurden ganz umgewandelt, denn die Kraft des
Herrn war mächtig in ihnen. Jch wohnte bei ihnen und wir
hatten große Versammlungen in ihrem Hause; eß kamen auch
etliche angesehene Standeß-personen, und deß Herrn Kraft tat sich
mächtig kund unter ihnen; John Reckleß ließ dann einen andern
Scherifs holen und eine Frau, mit der sie in Geschäften zu tun
gehabt hatten, und erklärte in Anwesenheit des andern Scheriff,
daß sie beide diese Frau bei einem Handel geschädigt hätten und
sie entschädigen müßten. Er sagte es sehr freundlich, aber der
andere Scherifs leugnete, und die Frau sagte, sie wisse nichtö da-
von. Aber der gerechte Scheriff sagte, ez sei so, und der andere
wisse das ganz gut; nachdem er die Sache aufgedeckt nnd daß
Unrecht, daß sie getan, eingestanden hatte, entschädigte er die
Frau und ermahnte den andern ein gleiches zu tun; die Kraft
Gotteö war mit diesem guten Scherifs und wirkte eine große
Wandlung in ihm und er hatte große Offenbarungen. Alö er


% \picinclude{./020-029/p_s028.jpg} 
28 Kapitel 111.
am darauffolgenden Martttage in den Pantoffeln in seinem
Zimmer auf- und abging, sagte er: »Jch muß auf den Markt
gehen und den Leuten Buße predigen,« und er ging auf den Markt
und in mehrere Straßen und predigte den Leuten Buße; und auch
noch andere aus der Stadt trieb ez, zu den Behörden zu gehen
und die Leute zur Buße zu erinahnen. Die Räte wurden sehr
böse über mich und ließen mich auß dem Hause deö Scheriff
holen und verurteilten mich zum Gefängniß. Alk; die Gerichtß-
fitzung stattfand, fühlte einer sich getrieben, sich statt meiner an-
zubieten, ,,Leib um Leib, Leben um Leben«. Alö ich vor den
Richter gebracht werden sollte, ging es ziemlich lang, bis mich
der Diener, der mich hinbringen sollte, abholte, und alß ich kam,
hatte sich der Richter schon erhoben, worauß ich sah, daß er er-
zürnt war; er sagte, er wolle dem Jüngling schon einen Verweis?.
geben, wenn er vor ihn gebracht werde; ich war damals unter
dem Namen ,,J«itngling« eingesperrt. Jch wurde denn wieder
inö Gefängnis gebracht. Die Kraft detz Herrn war mächtig
unter den ,,Freunden«, aber das?2 Volk fing an, tätlich zu
werden, so daß der Schloßkommandant Soldaten hinau?-schickte,
um die Leute au?-einander zu treiben, worauf es ruhig wurde;
alle, Priester und Volk, erstaunten ob der herrlichen Kraft, welche
heroorbrach, und etliche der Priester wurden empfänglich gemacht
und einige von ihnen bekannten sich zur Krast Gottes-.
Nachdem ich auö dem Gefängniß von Nottingham, wo ich
einige Zeit gefangen gewesen war, entlassen worden, zog ich
umher, wie vorher im Dienst des- Herm. Als ich nach Man?-field
Woodhouse kam, war dort eine verrückte Frau; daß Haar hing
ihr wirr über die Ohren und der Arzt war gerade bei ihr. Er
war daran, ihr zu Ader zu lassen, nachdem man sie zuvor ge-
bunden hatte; viele Leute waren um sie und hielten sie mit Ge-
walt fest, aber man konnte ihr kein Blut entziehen. Ich befahl,
daß man sie frei mache und ruhig lasse, denn sie konnten dem
Geiste, der sie plagte, nicht beikommen; sie machten sie srei und
ez trieb mich, zu ihr zu reden und sie im Namen dez Herrn still
und ruhig sein zu heißen, und sie war etz; die Krast dez Herrn
beruhigte ihr Gemüt und sie genaö, und sie nahm die Wahrheit
auf und blieb darin biß zu ihrem Tod. Des Herrn Name wurde
oerherrlichet, ihm gebührt die Ehre aller seiner Werke ....
Während ich in Manßfield Woodhouse war, trieb es mich,


% \picinclude{./020-029/p_s029.jpg} 
Der Tumnli in Nottingham. Wachsender Widerstand usn-. 29
ins Turmhaus zu gehen, um den Leuten die Wahrheit zu ver-
künden, aber das Volk fiel in großem Zorn über mich her, sie
schlugen mich zu Boden und erstickten mich fast; ich war arg zer-
schlagen und zerquetscht von ihren Händen, Bibeln und Stöcken.
Dann schleppten sie mich hinaus, wie wohl ich kaum fähig war
zu stehen, und taten mich in den Stock, wo ich einige Stunden
saß. Sie brachten Hundepeitschen und Pserdepeitschen und drohten
mir damit. Dann mußte ich vor die Behörden im Hause eines
Adligen, wo viele angesehene Leute zugegen waren. Als diese
sahen, wie ich mißhandelt worden war, gaben sie mir nach
Vielen Drohungen die Freiheit. Aber der Pöbel trieb mich
zur Stadt hinaus zum Dank dafür, daß ich ihnen das Wort des
Lebens verkündet hatte. Jch war kaum imstande zu stehen und
zu gehen, so übel hatten ste mich zugerichtet. Mit großer An-
strengung ging ich etwa eine Meile weit vor die Stadt, wo ich
Leute traf, die mir etwas zur Grquickung gaben, denn ich war
innerlich ganz auseinander, aber die Kraft des Herm heilte mich
bald wieder. Gs waren aber an dem Tage etliche von der Wahr-
heit des Herrn überzeugt worden, worüber ich mich freute. . . .
An einem Ersten Tage kamen wir nach Bagworth und gingen
ins Turmhaus, wohin einige der Freunde gebracht worden waren;
das Volk schloß sie darin ein und sich selbst mitsamt ihrem
Priester. Als der Priester fertig geredet hatte, machten sie die
Türe auf und wir gingen auch hinein und hatten einen Gottes-
dienst mit ihnen, und hernach hatten wir eine Versammlung in
der Stadt, mit manchen angesehenen Leuten. Als ich weiter zog,
hörte ich von solchen, die in Coventry um ihres Glaubens willen
gefangen waren. Aber als ich unterwegs zu ihrem Gefängnis war,
geschah das Wort des Herrn zu mir: ,,Meine Liebe war immer
mit dir und du bist in meiner Liebe«. Und ich fühlte mich ge-
hoben in der Liebe Gottes und sehr gestärkt an meinem innern
Menschen. Als ich in den Kerker zu den Gefangenen kam, über-
kam mich eine große Finsternis; ich hielt stille, denn mein Geist
ruhte in der Liebe Gottes. Schließlich singen die Gefangenen
an zu prahlen, und lärmten und lästerten, worüber meine
Seele sehr betrübt wurde. Sie sagten, daß sie Gott seien, aber
wir konnten solches nicht ertragen. Als sie ruhig geworden
waren, stand ich auf und fragte sie, ob sie solches aus innerem
Trieb oder auf Grund der Schrift täten? Sie sagten: ,,auf

