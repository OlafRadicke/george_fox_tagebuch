% \picinclude{./230-239/p_s230.jpg} 
Bekehrung zur Wahrheit als auch zur Befestigung solcher, die
schon bekehrt waren [...] Der Herr sei gelobt, der seine 
Wahrheit sich ausbreiten lässt! Nach der Versammlung kam eine Frau
zu mir; ihr Mann war einer der Friedensrichter der Gegend
und ein Mitglied der Behörde und sagte mir, ihr Mann
sei krank und werde wahrscheinlich sterben; ich solle doch mit ihr
heim kommen, um ihn zu besuchen. Sie wohnte drei Meilen weit
weg, und da ich gerade erhitzt aus der Versammlung kam, so war
es- hart für mich, mit ihr zu gehen; doch im Bewusstsein meines
Berufes nahm ich ein Pferd, ging mit ihr, besuchte ihren Mann
und redete, was der Herr mir eingab. Der Mann wurde sehr
erquickt und erholte sich gänzlich durch die Kraft des Herrn und
kam später zu unsern Versammlungen. Ich kehrte wieder am
gleichen Abend zu den Freunden zurück; und am folgenden Tage
zogen wir von dort weiter, etwa 20 Meilen nach Tredhaven
Creek,\ort{Tredhaven Creek} von wo wir am 3. des 8. Monats 
zur allgemeinen Versammlung aller Freunde von Maryland gingen. Dieselbe dauerte
5 Tage. Die drei ersten Versammlungen waren für öffentlichen
Gottesdienst,\index{Öffentliche Versammlung} zu welchen 
aller Arten Leute kamen; die beiden
andern für Männer- und Frauen-Versammlungen. Zu den öffentlichen 
Versammlungen kamen viele Protestanten\index{Protestanten} von verschiedenen
Richtungen, und einige Papisten;\index{Papisten} es waren mehrere Personen
von der Obrigkeit und ihre Frauen darunter und andere Angesehene 
der Gegend [...] 

Als wir unsern Dienst in Maryland verrichtet hatten, und
da wir die Absicht hatten, nach Virginia\ort{Virginia} zu gehen, hielten wir
eine Versammlung in Paturent am 4. des 9. Monats, um uns
von den Freunden zu verabschieden [...]

Am 5. schifften wir uns ein für Virginia und erreichten nach
drei Tagen Nanceum.\ort{Nanceum} Danach eilten wir 
nach Carolina\ort{Carolina}; doch
hatten wir unterwegs mehrere schöne Versammlungen [...] Am
21. des 9. Monats nach einem beschwerlichen Weg durch die
Wälder, Sümpfe, Moräste, erreichten wir 
Bonners Creek\ort{Bonners Creek}; dort
brachten wir die Nacht am Feuer zu; eine Frau gab uns eine
Matte, um darauf zu schlafen; das war das erste Haus in 
Carolina, das wir erreichten.
Hier ließen wir unsre ermüdeten Pferde und fuhren in einem
Kanu den Fluss hinunter, nach Hugh 
Smiths\person{Smith, Hugh} Haus, wo Leute
aller Richtungen uns besuchten, und viele von ihnen nahmen uns
% \picinclude{./230-239/p_s231.jpg} 
freundlich auf; Freunde gab es in der Gegend nicht. Nathaniel
Batts war darunter, der frühere Gouverneur von Roan Dak\index{Roan Dak}. Er
war bekannt als Hauptmann Batts\person{Hauptmann Batts} 
und war ein rauher, heftiger
Mann gewesen. Er fragte mich nach einer Frau in Cumberland,
die, wie er gehört habe, durch unsre Gebete und Handauflegen
geheilt worden sei, nachdem sie lange krank und von den Ärzten
aufgegeben worden war, und er wollte wissen, ob es wahr sei.
Ich sagte ihm, wir rühmen uns solcher Dinge nicht, aber es seien
viele solche Dinge geschehen in der Kraft Christi. An der 
Connie~Dak~Bay\ort{Connie~Dak~Bay} empfing uns der Gouverneur 
liebevoll; aber ein dortiger
Gelehrter wollte durchaus mit uns disputieren. Und sein Widerstand 
war uns sehr nützlich, da er uns Gelegenheit gab, die Leute
über manches aufzuklären, das Licht und den Geist Gottes betreffend;
er wollte nicht gelten lassen, das sie in einem jeden seien, und 
versicherte, sie seien nicht in den Indianern.\index{Rassismus} 
Hierauf rief ich einen
Indianer herbei und fragte ihn, ob nicht, wenn er lüge oder
jemandem Böses tue, etwas in ihm sei, das ihn dafür strafe? Er
sagte, es sei so etwas in ihm, das ihn darüber strafe, und er
schäme\index{Scharm} sich, wenn er Unrecht getan oder etwas Unrechtes gesagt
habe. Also beschämten wir den Gelehrten vor dem Gouverneur
und dem Volk weil der gute Mann so weit gegangen war, das er
nicht einmal die Schrift gelten\index{Bibel, Bedeutung} lies [...].

Von hier ging ich zu den Indianern und redete durch einen
Dolmetscher zu ihnen; ich zeigte ihnen, das Gott alle Dinge in
sechs Tagen gemacht habe\index{Schöpfungslehre} und nur eine Frau für einen Mann
gemacht habe,\index{Monogamie} und das Gott die alte Welt vernichtete wegen ihrer
Schlechtigkeit. Darauf sprach ich ihnen von Christus und zeigte
ihnen, das er für alle Menschen gestorben sei,\index{Opfertod} für ihre Sünden
so gut wie für die der andern, und das, wenn sie Böses täten, er sie
verbrennen werde, wenn sie aber Gutes tun, sie nicht verbrannt
würden.\index{Hölle} Ihr junger \zitat{König} war unter ihnen und andere ihrer
Häuptlinge und sie schienen, was ich ihnen sagte, gut aufzunehmen.

Nachdem wir die nördlichen Gegenden von Carolina\ort{Nord Carolina} durchreist
und der Wahrheit hier einigermasen den Weg bereitet hatten,
gingen wir in der Richtung von Virginia zurück. Wir hatten
unterwegs mehrere Versammlungen, so auch eine sehr segensreiche
bei Hugh Shmith. Gelobt sei der Herr ewiglich! Die Leute waren
sehr empfänglich und unser Wirken war gesegnet unter ihnen. Es
war ein indianischer Häuptling dabei, der sehr lieb war und
% \picinclude{./230-239/p_s232.jpg} 
bekannte, was wir sagten, sei Wahrheit. Auch ein Indianerpriester,\index{Indianerpriester} 
ein \zitat{Pawaw}, wie sie sagen, war da, der ruhig mit den
andern da saß. Am 9. des 10. Monats gingen wir nach Bonners
Creek zurück, wo wir unsre Pferde gelassen, nachdem wir etwa
18 Tage in Nord-Carolina gewesen waren.

Da unsre Pferde nun ausgeruht hatten, gingen wir weiter
nach Virginia durch Wälder und Sümpfe, so weit wir an diesem
Tage kommen konnten. Am folgenden Tage hatten wir eine
schwierige Reise durch Sumpf und Moor und waren den ganzen
Tag sehr nass und schmutzig; des Nachts trockneten wir uns dann
an einem Feuer. Am nächsten Abend erreichten wir Sommertown\ort{Sommertown}.
Als wir zur Herberge kamen, sah uns die Frau des Hauses und sagte
ihrem Sohn, er solle die Hunde festbinden; sie hatten nämlich
in Virginia und Carolina große Hunde, um ihre Häuser zu 
bewachen, da sie so einsam im Walde leben; aber der Sohn sagte,
es sei nicht nötig, denn die Hunde machten sich nicht an diese
Art Leute. Als wir nun in das Haus kamen, sagte er, wir
seien wie die Kinder Israel, gegen die die Hunde nicht mucksten
(2. Mos. 11,7)\bibel{Mos. 02. 11:07@2. Mos. 11:7} [...]

Wir waren drei Wochen unterwegs durch Virginia [...] Als
wir das Werk, das uns hier obgelegen, beendet hatten, fuhren
wir am 30. des 10. Monats in einer offenen Schaluppe ab, um
nach Maryland zurückzukehren [...] Da ein starker Sturm sich
erhob, waren wir froh, vor Nacht das Ufer zu erreichen und
übernachteten in einem Hause in Willougby Point\index{Willougby Point}. [...] 
Am Morgen kehrten wir zu unserm Boot zurück und segelten so rasch
als möglich weiter. Aber am Abend erhob sich abermals ein
Sturm, so das wir wieder ans Ufer mussten, wo wir die Nacht
an einem Feuer zubrachten, um uns zu trocknen, und um uns
herum heulten die Wölfe [...] Am 3. des 11. Monats
war der Wind ziemlich günstig und wir benützen ihn, um so
schnell wie möglich fort zu kommen; am Abend erreichten
wir Milfordhaven\ort{Milfordhaven}. Darauf passierten wir 
den Rappahannockflus\ort{Rappahannock (Fluss)}, den 
Patomac\ort{Patomac}, und dann fuhren wir in der Richtung
des Paturentflusses\ort{ Paturent (Fluss)} und erreichten in der Frühe des Morgens
James Prestons\person{Preston, James} Haus. Wir waren sehr müde, gingen aber
doch am nächsten Tage, einem Ersten Tage, zu einer 
Versammlung in der Nähe [...] Die Kälte wurde um so 
empfindlicher und der Frost und Schnee so heftig, das es fast unerträglich
% \picinclude{./230-239/p_s233.jpg} 
war. Auch war es gefährlich umherzureisen [...] Am 27. des
11. Monats hatten wir eine köstliche Versammlung in einem
Tabakshaus. Am darauffolgenden Tage kehrten wir wieder zu
James Preston zurück; als wir dort anlangten, war sein Haus
in der vorhergehenden Nacht abgebrannt, so das wir die Nächte
bei sehr kaltem Wetter im Freien am Feuer zubringen mussten.
Wir machten die merkwürdige Beobachtung, das sich eines Tages
mitten in diesem kalten Wetter der Wind gegen Süden drehte
und es ganz unerträglich heiß wurde [...] Am 2. des 12. Monats
hatten wir eine herrliche Versammlung in Paturent\ort{Paturent}. Am 12.
reisten wir weiter in unserm Boot. Den Anamessy und den
Amoroca\ort{Amoroca} umgehend, kamen wir nach Manaoke. Dann passierten
wir den Wicocomaco, wo wir eine gesegnete Versammlung hatten;
dann gings zu Pferde etwa 24 Meilen weit durch Sümpfe und
Wälder zum Hause eines Richters, wo wir ebenfalls eine 
lebendige Versammlung hatten. Zu dieser kam eine Frau aus 
Anamessy, die viele Jahre schwermütig\index{Schwermütigkeit} 
gewesen war und oft während
zwei Monaten nichts redete und sich um nichts kümmerte; als ich
von ihr hörte, trieb mich der Herr, zur ihr zu gehen, und ihr zu
verkünden, das ihrem Hause Heil widerfahren sei. Nachdem ich
Worte des Lebens zu ihr geredet hatte und den Herrn für sie
angefleht, kam sie zurecht und zog mit uns umher zu den 
Versammlungen und ist seither gesund, dem Herrn sei Lob [...]

Nun hatten wir unsre Arbeit in dieser Gegend getan und
verließen Anamessy\ort{Anamessy}. Wir gingen zu Wasser etwa 50 Meilen,
nach dem Hungerflus [...] Danach etwa 40 Meilen nach dem
kleinen Choptankflus [...] Ehe wir dann weiter zogen, hatten
wir eine herrliche Versammlung, zu der viele Leute kamen; unter
anderem auch mehrere von den Behörden der Stadt, mit ihren
Frauen. Von den Indianern kam einer, den sie ihren \zitat{Kaiser}
nannten, ein Indianer-\zitat{König} und sein Dolmetscher, die alle sehr
andächtig zuhörten und sehr lieb waren. Es war eine grundlegende 
Versammlung. Dies war am 23. des 1. Monats. Am
24. gingen wir 10 Meilen weit zu Wasser nach der Indianerstadt, 
wo jener \zitat{Kaiser} wohnte. Ich hatte ihn von meinem Kommen
benachrichtigt und gebeten, seine \zitat{Könige} und Räte zu versammeln.
Am Morgen kam der \zitat{Kaiser} selber und führte mich in die Stadt;
und sie waren alle beisammen mit ihren Dolmetschern und ihren
Leuten, und die alte \zitat{Kaiserin} war auch da. Sie waren sehr ernst
% \picinclude{./230-239/p_s234.jpg} 
und ruhig und sehr aufmerksam, mehr als manche, die sich Christen
nennen. Es waren einige mit mir, die verdolmetschen konnten
und wir hatten eine schöne Versammlung mit ihnen, die großen
Nutzen schaffte, denn sie brachte die Wahrheit und die Freunde
bei ihnen in Ansehen. Gelobt sei der Herr! [...]

Nachdem wir die meisten Gegenden dieses Landes durchzogen
hatten und die meisten der Plantagen besucht und überall, wo wir
hinkamen, Alarm geblasen\index{Alarm blasen} hatten und Gottes Tag des Heils
den Leuten verkündet hatten, spürten wir im Geist, das wir
bald unsere Aufgabe in diesen Weltgegenden erfüllt hätten, und
wandten uns wieder Alt-England zu. Doch es verlangte uns,
und der Herr schenkte uns die innere Freiheit dazu, bis zur
nahen Generalversammlung\index{Generalversammlung} 
für Maryland\ort{Maryland} zu bleiben, damit wir
alle Freunde beisammen sehen könnten, ehe wir abreisten. Die
Zwischenzeit brachten wir damit zu, Freunde und Gemeindemitglieder
zu besuchen, Versammlungen um die Cliffs und Paturent\ort{Paturent} herum
beizuwohnen, und Antworten zu schreiben auf allerlei Kritteleien,
welche etliche Gegner der Wahrheit erhoben und verbreitet hatten,\index{Verteidigungsschrift}
um die Leute zu verhindern, die Wahrheit anzunehmen; so waren
wir also nicht müßig, sondern wirkten für den Herrn, bis zur
allgemeinen Provinzialversammlung, die am 17. des 3. Monats
begann und vier Tage dauerte. Am Ersten Tage hatten die
Männer und Frauen ihre geschäftlichen Versammlungen, in welchen
die Angelegenheiten der Kirche besorgt wurden, und viele Dinge
wurden ihnen geoffenbart zu ihrer Erbauung und Trost. Die
drei übrigen Tage wurden mit öffentlichen 
Versammlungen\index{Öffentliche Versammlung} zugebracht, 
wobei verschiedene angesehene Regierungsbeamte anwesend
waren und viele andere; sie waren alle im allgemeinen befriedigt
und manchen ging es zu Herzen. Denn es war eine wundervolle,
herrliche Versammlung, und die mächtige Gegenwart des Herrn
ward überall fühlbar und gesehen; gelobt und gepriesen werde
sein heiliger Name immerdar, welcher Herrschaft gibt über alles.
[...] Nach dieser Versammlung nahmen wir Abschied von den
Freunden [...] Am nächsten Tage, dem 21. des 3. Monats
1673\index{Jahr!1673} schifften wir uns ein für England und am 28. des 4. Monats
landeten wir im Hafen von Bristol\ort{Bristol}. Dieselbe treue Hand der
Vorsehung, die uns geleitet und glücklich hinüber gebracht hatte,
wachte bei unserer Rückkehr über uns und brachte uns glücklich
zurück. Lob und Dank sei seinem heiligen Namen innnerdar!
% \picinclude{./230-239/p_s235.jpg} 

Wir hatten während der Reise Viele köstliche Versammlungen
an Bord des Schiffes gehabt, gewöhnlich zwei in der Woche, in
denen die gesegnete Gegenwart des Herrn uns mächtig erquickte,
und oft goss sie sich aus über uns und machte die Anwesenden
empfänglich.

%%%%%%%%%%%%%%%%%%% Kapitel 20. %%%%%%%%%%%%%%%%%%%%%%%%%%%%%%

\chapter[Penn, Frauenversammlungen und Grundsätze]{William Penn, Frauenversammlungen und  über die 
Grundsätze der Quäker.}

\begin{center}
\textbf{Ankunft in Bristol. Zusammentreffen mit William Penn und
anderen. Verteidigung der Frauenversammlungen. Vorgeahnte
Gefangenschaft in Worcester. Brief an den König über die 
Grundsätze der Quäker. Krankheit. Befreiung. Während der
Gefangenschaft verfasste Schriften.}
\end{center}



Als wir ans Ufer kamen, begaben wir uns nach Shirehampton\ort{Shirehampton} [...] 
und ritten von da nach Bristol\ort{Bristol} [...]. Am Abend
schrieb ich von hier einen Brief an meine Frau:

\brief{Fell, Margaret}{
Liebes Herz!

\bigskip 

Heute Abend landeten wir in Bristol, Gott dem Herrn sei
Lob immerdar, der unser Schutzherr war und unser Schiff lenkte,
dem Gott der ganzen Erde, der Meere und Winde, der die Wolken
zu seinen Wagen macht [...] Robert Widders\person{Widders, Robert} und James
Lancaster\person{Lancaster, James} sind mit mir und sind gesund. Dem Herrn sei Lob
und Ehre, der uns durch so manche Gefahr geholfen, auf dem
Wasser, in den Stürmen, vor Seeräubern und andern Räubern,
in den Gefahren der Wildnis und unter den falschen Frommen.
Sein Ruhm ist über alles, Amen. Darum werdet das neue
Leben inne, und lebet ganz dem Herrn in demselben. Ich möchte,
so der Herr will, eine Zeit lang hier bleiben, vielleicht bis zum
Jahrmarkt. Genug diesmal. Meine Liebe allen Freunden.

\bigskip 

\begin{flushright}Bristol, 28. des 4. Monats 1673\index{Jahr!1673}. G. F.\end{flushright}
}

Zwischen diesem Tage und dem Jahrmarkt kam meine Frau
aus dem Norden zu mir nach Bristol, und ihr Schwiegersohn,
Thomas Lower\person{Lower, Thomas} und zwei ihrer Töchter kamen mit ihr. Ihr
anderer Schwiegersohn, John Raus\person{Raus, John}, William Penn\person{Penn, William}\footnote{
William Penn, Sohn des großen Admiral Penn, zeigte schon während
seiner Studienzeit in Christchurch College\index{Christchurch College} 
in Oxford ernst religiöse Bedürfnisse.
Als er nach Hause zurückgekehrt, sich den herrschenden Hofsitten nicht fügen wollte,
sagte sein Vater sich von ihm los. Er schloss sich nun den Quäkern an und
wurde ihr bedeutendstes Mitglied. Aus einem ihm von Karl II. überlassenen
und nach ihm Pennsylwanien genannten Landstrich am Delaware\ort{Delaware} gründete er
eine Freistatt für seine verfolgten Glaubensgenossen. (Näheres siehe Weingarten
a.a.O.S. 408.)
} und seine
% \picinclude{./230-239/p_s236.jpg} 
Frau und Gerrard Roberts\person{Roberts, Gerrard} kamen von London; und noch viele
andere Freunde aus verschiedenen Gegenden des Landes kamen
zum Jahrmarkt\index{Jahrmarkt}, und wir hatten herrliche Versammlungen, in
denen des Herrn Kraft über allen war. Nachdem ich meine
Arbeit für den Herrn in dieser Stadt getan, ging ich nach 
Gloucestershire\ort{Gloucestershire}, [...] und von da nach Wiltshire. 
In Slattenford in Wiltshire\ort{Wiltshire} hatten wir eine schöne Versammlung, obgleich wir
manchen Widerstand erfuhren von solchen, die sich den 
Frauenversammlungen\index{Gleichberechtigung} widersetzten; der Herr 
trieb mich, dieselben den
Freunden anzuempfehlen\index{Konflikt} zu Nutz und Frommen der Kirche Christi:
\zitat{Gläubige Frauen, welche zum Glauben an die Wahrheit berufen 
sind und desselben köstlichen Glaubens teilhaftig gemacht
sind, wie die Männer, und Miterben desselben ewigen Evangeliums 
des Lebens und Heils, sollen gleicherweise in den Besitz
und Stand der Ordnungen des Evangeliums kommen und somit
Mitgehilfen der Männer werden bei den Neugestaltungen im
Dienst der Wahrheit in den Angelegenheiten der Kirche, wie sie
es in den zeitlichen Dingen des täglichen Lebens sind, damit
die ganze Hausgemeinde Gottes, sowohl, Männer wie Frauen,
ihre Pflicht und Aufgabe im Hausstand Gottes kennen, einsehen
und ausüben möchten, damit besser für die Armen gesorgt werde,
die Jungen unterrichtet und in Gottes Wegen unterwiesen werden,
die Wankelmütigen und Liederlichen zurechtgewiesen und getadelt
werden in der Furcht Gottes, die Unbescholtenheit solcher, die sich
verheiraten wollen, genauer und strenger untersucht werde in der
Weisheit Gottes, und alle Glieder des geistigen Leibes, der Kirche,
einander bewachen und helfen in der Liebe.} Aber nachdem diese
Gegner sich lange in Hader und Zank ereifert hatten, warf der
Herr einen ihrer Führer darnieder, so das er sich demütigte und
das Unrecht einsah, das er tat, als er sich Gottes himmlischer
Macht widersetzte, er gestand den Freunden seinen Irrtum ein
und veröffentlichte später ein Schreiben, in welchem er erklärte,
er hätte sich eigensinnig widersetzt, trotz meiner häufigen 
Warnungen, bis das Feuer des Herrn in ihm entbrannt sei, und er
den Engel des Herrn mit gezogenem Schwert gesehen habe, im
% \picinclude{./230-239/p_s237.jpg} 
Begriff ihn zu töten [...]. Wir besuchten viele Freunde und
kamen schließlich nach Kingston\ort{Kingston}, wo ich mit 
meiner Frau\person{Fell, Margaret} und
ihrer Tochter Rachel\person{Fell, Rachel} zusammen traf. 
Ich hielt mich nicht lange
hier auf, sondern ging nach London, wo die Baptisten\index{Baptisten} mit einigen
Sozinianern\index{Sozinianern} sehr bösartig geworden waren und Viele Bücher gegen
uns gedruckt hatten, und ich hatte viele Arbeit in der Kraft des
Herrn, ehe ich aus dieser Stadt loskommen konnte [...].\index{Schmähschriften}

Darauf machte ich eine kurze Reise durch einige Gegenden
in Essex\ort{Essex} und kehrte nach London\ort{London} 
zurück, wohin ich mich innerlich
gezogen fühlte, denn ich hatte gehört, das man viele Freunde
vor die Richter gebracht hatte und etliche in London und an andern
Orten gefangen genommen hatte, weil sie ihre Schaufenster an
Festtagen\index{Festtagen} und an sogenannten 
Fasttagen\index{Fasttagen} geöffnet hatten, und
weil sie Zeugnis ablegten gegen alles Feiern solcher Tage. Die
Freunde mussten dies tun, weil sie ja wussten, das die wahren
Christen die Feste der Juden zur Zeit der Apostel nicht hielten,
und so konnten sie auch nicht die sogenannten Feste der Heiden
und Papisten\index{Papisten}, welche unter den sogenannten Christen seit den
Tagen der Apostel eingesetzt wurden, halten. Denn wir sind 
befreit worden vom Halten bestimmter Tage, durch Jesum Christum
und sind in den Tag gebracht worden, der aus der Höhe aufgegangen 
ist, und wir sind jetzt in Ihm, der Herr ist über den
jüdischen Sabbath\index{Jüdischen Sabbath}, und über das Wesen 
der jüdischen Zeichen [...

Bald darauf, als ich, in Adderbury\ort{Adderbury}, am Nachtessen saß, fühlte
ich, das ich gefangen genommen werden würde, ich sagte aber noch
niemand etwas [...]. Am andern Tage hatten wir eine Versammlung 
[...] Nach derselben saß ich mit einigen Freunden in einem
Zimmer im Gespräch, da kam Henry Parker\person{Parker, Henry}, 
ein Friedensrichter, ins Haus, und mit ihm Rowland 
Hains\person{Hains, Rowland}, ein Priester aus Hunniton
in Warwickshire\index{Warwickshire}, [...] und Henry Parker nahm mich gefangen und
Thomas Lower\person{Lower, Thomas} zur Gesellschaft 
mit, und obgleich er nichts gegen
uns Vorbringen konnte, schickte er uns beide ins Gefängnis von
Woreester\ort{Woreester}, durch einen merkwürdigen Verhaftbefehl [...]
Da ich derart zum Gefangenen gemacht worden war, ohne viel
Aussicht, vor der vierteljährlichen Gerichtssitzung frei zu werden,
veranlassten wir einige Freunde, meine Frau\person{Fell, Margaret} und ihre Töchter
nach dem Norden zu begleiten [...] Als ich dachte, das meine
Frau zu Hause angelangt sei, schrieb ich ihr aus der Gefangenschaft 
folgenden Brief:
% \picinclude{./230-239/p_s238.jpg} 


\brief{Fell, Margaret}{
  Liebes Herz!

  \bigskip 

  Du schienst ein wenig betrübt, als ich vom Gefängnis sprach
  und dann gefangen genommen wurde. Ergib dich in den Willen
  des Herrn. Denn als ich im Hause des John Rous in Kingston
  war, hatte ich ein Gesicht, das ich gefangen genommen würde,
  und als ich bei Bray Doily in Oxfordshire war, sah ich, während
  ich am Abendessen saß, das ich gefangen genommen und Leiden
  zu erdulden haben würde. Aber des Herrn Macht ist über allem, sein
  Name sei gepriesen ewiglich [...]

\bigskip 
\begin{flushright}G. F.\end{flushright}

}

Wir wurden erst am letzten Tage der Gerichtssitzung 
vorgenommen, am 21. des 11. Monats 1673\index{Jahr!1673} [...] Ich musste
Bericht über meine Reise geben. Friedensrichter Parker hatte,
um den Fall recht schwer scheinen zu lassen, eine große Geschichte
gemacht, es seien Leute von London, von Cornwall, aus dem
Norden und von Bristol im Hause gewesen, als man mich gefangen 
genommen habe; hieraus erklärte ich ihm, das das alles
gewissermaßen eine einzige Familie gewesen 
sei,\footnote{Der Konventikel-Akt\index{Konventikel-Akt} von 1664\index{Jahr!1664} sagt: 
\zitat{jede Prioatandacht von mehr
als fünf Personen außer der Familie, wobei nicht das Common-Prayer-Boot
zugrunde gelegt wird, wird mit dreimonalichem Gefängnis, zum dritten mal
mit Verbannung bestraft.}} indem niemand von
London dagewesen sei, als ich selber, niemand aus dem Norden als
meine Frau und meine Töchter, niemand von Cornwall als mein
Schwiegersohn.

Als ich fertig geredet, stand der Vorsitzende, Simpson, ein
einstiger Presbyterianer, auf und sagte: \zitat{Was Ihr da sagt, klingt
recht unschuldig}. Darauf flüsterten er und Parker eine Weile
miteinander und daraus stand er wieder auf und sagte: \zitat{M. Fox,
Ihr seid ein ausgezeichneter Mann und alles, was Ihr da sagt,
mag wahr sein, aber, um uns ganz zu befriedigen, wollt Ihr den
Huldigungseid leisten?}\index{Eid} Ich erwiderte ihm, sie hätten versprochen,
uns keine Falle zu stellen, dies sei aber einfach eine Falle, da sie
ja wissen, das wir nicht schwören können. Sie ließen dennoch
den Eid vorlesen. Ich erklärte ihnen daraus: \zitat{Ich habe nie in
meinem Leben einen Eid geleistet, aber ich bin immer der Regierung
gehorsam gewesen; ich war im Kerker von Derby sechs Monate,
weil ich die Waffen nicht gegen König Karl bei Woreester 
erheben wollte, und weil ich in die Versammlungen ging, wurde
ich nach Leicestershire gebracht, vor Oliver 
Cromwell\person{Cromwell, Oliver}, als einer der
% \picinclude{./230-239/p_s239.jpg} 
Mitverschworenen für die Rückkehr des Königs. Ihr wisst es ja
nach euren eigenen Gewissen, das wir, die ihr Quäker nennt
keinen Eid leisten können, weil Christus es verboten hat. Was
aber den Inhalt des Eides anbelangt, so kann ich sagen, und
sage es auch, das ich den König von England als den rechtmäßigen 
Erben und Nachfolger des englischen Reiches anerkenne\index{Monarchie}
und alle Verschwörer und Verschwörungen gegen ihn verabscheue,
und ich hege nur Liebe und Wohlwollen in meinem Herzen gegen
ihn und gegen alle Menschen und wünsche ihm und ihnen allen
nichts als Gutes. Der Herr weiß es, vor welchem ich als ein
unschuldiger Mann stehe. Was den Suprematseid anbelangt, so
verabscheue ich den Papst\index{Papst} und seine Macht und 
seine Religion\index{Katholizismus}
von ganzem Herzen.} [...] Aber sie ließen mich vom 
Gefangenwärter hinweg führen und brachen so ihr Versprechen dem Lande
gegenüber, denn sie hatten mir freies Reden zugestanden und mir
es doch nicht gewährt [...]

Während meiner Gefangenschaft kam es über mich, unsere
Anschauungen und Grundsätze dem König darzutun, nicht 
hauptsächlich um meiner eigenen Leiden willen, sondern damit er unsere
Anschauungen und unsere Gemeinschaft besser verstehe.

\brief{König}{
  An den König!

  \bigskip 

  Der Ausgangspunkt der Quäker ist der Geist von Christus,
  der für uns gestorben und um unsrer Gerechtigkeit willen auferweckt
  worden ist, durch welchen wir wissen, das wir sein sind. Er
  wohnet in uns mit seinem Geist und der Geist Christi macht uns
  frei von aller Ungerechtigkeit und Gottlosigkeit. Der Geist Christi
  machet, das wir allem gottlosen Wesen absagen, als da sind
  Lügen, Stehlen, Töten, Verbrechen, Hurerei und alle Arten von
  Unreinigkeit, Unzucht, Bosheit, Hass, Betrügerei, Schlemmen und
  alle Werke des Teufels.\index{Sündlosigkeit} Der Geist Christi führt uns dazu, für
  alle Menschen den Frieden und das Gute zu suchen und friedlich
  zu leben. Er machet, das wir uns aller Anschläge und 
  Verschwörungen gegen den König oder irgend sonst Jemand 
  enthalten. Er hält uns zurück von jenem bösen Tun und Treiben,
  gegen welches das Schwert der Obrigkeit sich richtet. Unser Wunsch
  und Bestreben ist, das alle, die Christus bekennen, auch im Geiste
  Christi wandeln, damit sie durch denselben des Fleisches Geschäfte
  töten möchten und mit dem Schwert des Geistes ihre Sünde und
  Bosheit ausrotten. Dann würden die Richter und Beamten nicht
  % \picinclude{./240-249/p_s240.jpg} 
  soviel damit zu tun haben, das Böse im Reich zu bestrafen, und
  die Könige und Fürsten brauchten keinen ihrer Untertanen zu
  fürchten, wenn alle im Geist Christi wandelten; denn die Früchte
  des Geistes sind Liebe, Gerechtigkeit, Gütigkeit und Mäßigkeit.
  Wenn alle, die sich als Anhänger Christi bekennen, auch in feinem
  Geiste wandeln und durch denselben Sünde und Bosheit in sich
  ertöten würden, so wäre dies eine große Erleichterung für die
  Obrigkeit und würde ihr viel Mühe ersparen, denn dann würden
  alle dazu geführt, andern zu tun, wie sie wollten, das man ihnen
  tue, und das Königliche Gesetz der Freiheit würde somit erfüllt [...]
  
  Wir können aus großer Gewissenhaftigkeit gegen die Gebote
  Christi und seiner Apostel nicht schwören, denn es wird uns 
  geboten, in Matth. 5\bibel{Matth. 5} und Jak. 5\bibel{Jak. 5} 
  bei ja und nein zu bleiben und überhaupt nicht zu schwören, 
  weder beim Himmel, noch bei der Erde,
  noch bei irgend sonst etwas, auf das wir nicht übles tun und in
  Verdammnis fallen. Christus sagte: \zitat{Ihr habt gehört, das zu
  den Alten gesagt ist, ihr sollt keinen falschen Eid tun und Gott
  euren Eid halten} (Math. 5). Es waren dies wahre und feierliche 
  Eide, und die, welche sie einst leisteten, hatten sie zu halten,
  aber Christus und die Apostel verbieten sie zur Zeit des 
  Evangeliums so gut wie die falschen und unnützen Eide. Wenn wir irgend
  einen Eid leisten könnten, so wäre es der Huldigungseid, weil
  wir wissen, das König Karl durch Gottes Macht nach England
  zurückkam und zum König von England gemacht und über unsere
  früheren Verfolger gesetzt wurde, und was die Oberherrschaft des
  Papstes anbelangt, so erkennen wir sie in keiner Weise an.\index{Papstum} Aber
  da Christus und seine Apostel uns geboten, nicht zu schwören,
  sondern bei ja und nein zu bleiben, so dürfen wir ihren Geboten
  nicht ungehorsam sein. Darum haben viele uns den Eid vorglegt 
  als Falle, damit wir ihnen zur Beute würden. Unsre
  Weigerung des Eides geschieht nicht aus Eigensinn und 
  Hartnäckigkeit oder Missachtung, sondern nur aus Gehorsam gegen
  die Gebote Christi und der Apostel, und wir sind bereit, wenn wir
  unser ja und nein brechen, die gleiche Strafe zu leiden, wie jemand
  der seinen Eid bricht. Wir bitten darum den König, solches zu
  bedenken, und wie lange wir schon leiden um dieser Sache willen.
  Dies ist von einem, der dem König allezeit Glück und alles
  Wohlergehen wünscht und allen seinen Untertanen, durch Jesus
  Christus.
  \bigskip 
  \begin{flushright}G. F.\end{flushright}

}
