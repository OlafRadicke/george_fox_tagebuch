% \picinclude{./230-239/p_s230.jpg} 
230 Kapitel Ill
Bekehrung zur Wahrheit alö auch zur Befestigung solcher, die
schon bekehrt waren ..... Der Herr sei gelobt, der seine Wahr-
heit sich aus?-breiten läßt! Nach der Versammlung kam eine Frau
zu mir; ihr Mann war einer der Frieden?-richter der Gegend
und ein Mitglied der Behörde und sagte mir, ihr Mann
sei krank und werde wahrscheinlich sterben; ich solle doch mit ihr
heim kommen, um ihn zu besuchen. Sie wohnte drei Meilen weit
weg, und da ich gerade erhitzt auß der Versammlung kam, so war
es- hart für mich, mit ihr zu gehen; doch im Bewußtsein meines
Berufes-J nahm ich ein Pferd, ging mit ihr, besuchte ihren Mann
und redete, maß der Herr mir eingab. Der Mann wurde sehr
erquickt und erholte sich gänzlich durch die Kraft dee-’ Herrn und
kam später zu unsern Versammlungen. Ich kehrte wieder am
gleichen Abend zu den Freunden zurück; und am folgenden Tage
zogen wir von dort weiter, etwa 20 Meilen nach Tredhaven-
Creek, von wo wir am 3. deß 8. Monats zur allgemeinen Ver-
sammlung aller Freunde von Maryland gingen. Dieselbe dauerte
5 Tage. Die drei ersten Versammlungen waren für öffentlichen
Gotteödienst, zu welchen aller Arten Leute kamen; die beiden
andern für Männer- und Frauen-Versammlungen. Zu den öffent-
lichen Versammlungen kamen viele Protestanten von verschiedenen
Richdmgen, und einige Papisten; es waren mehrere Personen
von der Obrigkeit und ihre Frauen darunter und andere Ange-
sehene der Gegend .....
Als wir unsern Dienst in Maryland verrichtet hatten, und
da wir die Absicht hatten, nach Virginia zu gehen, hielten wir
eine Versammlung in Patuxent am 4. deß 9. Monat-3, um un?-
von den Freunden zu verabschieden .....
Am 5. schissten wir unö ein für Virginia und erreichten nach
drei Tagen Naneeum. Darnach eilten wir nach Carolina; doch
hatten wir unterwegß mehrere schöne Versammlungen ..... Am
21. des 9i Monats- nach einem beschwerlichen Weg durch die
Wälder, Sümpfe, Moräste, erreichten wir Bonnertz Creek; dort
brachten wir die Nacht am Feuer zu; eine Frau gab unß eine
Matte, um darauf zu schlafen; daß war daß erste Haus in Caro-
lina, daß wir erreichten.
Hier ließen wir unsre ermüdeten Pferde und fuhren in einem
Canoe den Fluß hinunter, nach Hugh Smithö Hautz, wo Leute
aller Richtungen uns besuchten, und viele von ihnen nahmen unö


% \picinclude{./230-239/p_s231.jpg} 
Arbeit in Nordamerika unter Engländern und Indianern. 231
freundlich auf; Freunde gab es in der Gegend nicht. Nathaniel
Batts war darunter, der frühere Gouverneur von Roan Oak. Gr
war bekannt als Hauptmann Batts und war ein rauher, heftiger
Mann gewesen. Gr fragte mich nach einer Frau in Cumberland,
die, wie er gehört habe, durch unsre Gebete und Handauflegen
geheilt worden sei, nachdem sie lange krank und von den Arzten
aufgegeben worden war, und er wollte wissen, ob es wahr sei.
Jch sagte ihm, wir rühmen uns solcher Dinge nicht, aber es seien
viele solche Dinge geschehen in der Kraft Christi. gl An der Connie-
Oak-Bah empfing uns der Gouverneur liebevoll; aber ein dortiger
Gelehrter wollte durchaus mit uns disputieren. Und sein Wider-
stand war uns sehr nützlich, da er uns Gelegenheit gab, die Leute
über manches aufzuklären, das Licht und den Geist Gottes betreffend;
er wollte nicht gelten lassen, daß sie in einem jeden seien, und ver-
sicherte, sie seien nicht in den Indianern. Hierauf rief ich einen
Jndianer herbei und fragte ihn, ob nicht, wenn er lüge oder
jemandem Böses tue, etwas in ihm sei, das ihn dafür strafe? Cr
sagte, es sei so etwas in ihm, das ihn darüber strafe, und er
schäme sich, wenn er Unrecht getan oder etwas Unrechtes gesagt
habe. Also beschämten wir den Gelehrten vor dem Gouverneur
und dem Volk weil der gute Mann so weit gegangen war, daß er
nicht einmal die Schrift gelten ließ .... . sf
Von hier ging ich zu den Jndianern und redete durch einen
Dolmetscher zu ihnen; ich zeigte ihnen, daß Gott alle Dinge in
sechs Tagen gemacht habe und nur eine Frau für einen Mann
gemacht habe, und daß Gott die alte Welt vernichtete wegen ihrer
Schlechtigkeit. Darauf sprach ich ihnen von Christus und zeigte
ihnen, daß er für alle Menschen gestorben sei, für ihre Sünden
so gut wie für die der andern, und daß, wenn sie Böses täten, er sie
verbrennen werde, wenn sie aber Gutes tüten, sie nicht verbrannt
würden. Jhr junger ,,König« war unter ihnen und andere ihrer
Häuptlinge und sie schienen, was ich ihnen sagte, gut aufzunehmen.
Nachdem wir die nördlichen Gegenden von Carolina durchreist
und der Wahrheit hier einigermaßen den Weg bereitet hatten,
gingen wir in der Richtung von Virginia zurück. Wir hatten
unterwegs mehrere Versammlungen, so auch eine sehr segensreiche
bei Hugh Shmith. Gelobt sei der Herr ewiglich! Die Leute waren
sehr empsänglich und unser Wirken war gesegnet unter ihnen. Es
war ein indianischer Häuptling dabei, der sehr lieb war und


% \picinclude{./230-239/p_s232.jpg} 
232 Kapitel Icllc. F
bekannte, maß wir sagten, sei Wahrheit. Auch ein Jndianer-
priester, ein Pawaw, wie sie sagen, war da, der ruhig mit den
andern da saß. Am 9. deS 10. Monats gingen wir nach Bonnerß
Creek zurück, wo wir unsre Pferde gelassen, nachdem wir etwa
18 Tage in Nord-Carolina gewesen waren.
Da unsre Pferde nun außgeruht hatten, gingen wir weiter
nach Virginia durch Wälder und Sümpfe, so weit wir an diesem
Tage kommen konnten. Am folgenden Tage hatten wir eine
schwierige Reise durch Sumpf und Moor und waren den ganzen
Tag sehr naß und schmutzig; dez Nachtz trockneten wir uns dann
an einem Feuer. Am nächsten Abend erreichten wir Sommertown.
Alß wir zur Herberge kamen, sah unß die Frau des Hauseß- und sagte
ihrem Sohn, er solle die Hunde festbinden; sie hatten nämlich
in Virginia und Carolina große Hunde, um ihre Häuser zu be-
wachen, da sie so einsam im Walde leben; aber der Sohn sagte,
es sei nicht nötig, denn die Hunde machten sich nicht an diese
Art Leute. Alß wir nun in das Haus kamen, sagte er, wir
seien wie die Kinder Jörael, gegen die die Hunde nicht mucksten
(2. Mos. 11, 7) . . . .
Wir waren drei Wochen unterwegs durch Virginia .... Als
wir das Werk, das uns hier obgelegen, beendet hatten, fuhren
wir am 30. dez 10. Monats- in einer offenen Schaluppe ab, um
nach Maryland zurückzukehren ..... Da ein starker Sturm sich
erhob, waren wir sroh, vor Nacht das Ufer zu erreichen und
übernachteten in einem Hause in Willougby Point. .... Am
Morgen kehrten wir zu unserm Boot zurück und segelten so rasch
als möglich weiter. Aber am Mend erhob sich abermals ein
Sturm, so daß wir wieder anß User mußten, wo wir die Nacht
an einem Feuer zubrachten, um unß zu trocknen, und um uns
herum heulien die Wölfe ..... Am 3. deö tt. Monatß
war der Wind ziemlich günstig und wir benützen ihn, um so
schnell wie möglich fort zu kommen; am Abend erreichten
wir Milfordhaven. Darauf passierien wir den Rappahannock-
fluß, den Patomac, und dann fuhren wir in der Richtung
deö Patuxentflusseö und erreichten in der Frühe dez Morgens
Jameö Prestonß Hartz. Wir waren sehr müde, gingen aber
doch am nächsten Tage, einem Ersten Tage, zu einer Ver-
sammlung in der Nähe ..... Die Kälte wurde um so empfind-
licher und der Frost und Schnee so heftig, daß etz fast unerträglich


% \picinclude{./230-239/p_s233.jpg} 
Arbeit in Nordamerika unter Englündern und Indianern. 233
war. Auch war es gefährlich umherzureisen ..... Am 27. des
11. Monats hatten wir eine köstliche Versammlung in einem
Tabakshaus. Am darauffolgenden Tage kehrten wir wieder zu
James Preston zurück; als wir dort anlangten, war sein Haus
in der vorhergehenden Nacht abgebrannt, so daß wir die Nächte
bei sehr kaltem Wetter im Freien am Feuer zubringen mußten.
Wir machten die merkwürdige Beobachtung, daß sich eines Tages
mitten in diesem kalten Wetter der Wind gegen Süden drehte
und es ganz unerträglich heiß wurde ..... Am 2. des 12. Monats «
hatten wir eine herrliche Versammlung in Patuxent. Am 12.
reisten wir weiter in unserm Boot. DeuAnamess1) und den
Ainoroca umgehend, kamen wir nach Manaoke. Dann passierten
wir den Wicoeomaeo, wo wir eine gesegnete Versammlung hatten;
dann gings zu Pferde etwa 24 Meilen weit durch Sümpse und
Wälder zum Hause eines Richters, wo wir ebenfalls eine leben-
dige Versammlung hatten. Zu dieser kam eine Frau aus Ana-
messy, die viele Jahre schwermütig gewesen war und ost während
zwei Monaten nichts redete und sich um nichts kümmerte; als ich
von ihr hörte, trieb mich der Herr, zur ihr zu gehen, und ihr zu
verkünden, daß ihrem Hause Heil widerfahren sei. Nachdem ich
Worte des Lebens zu ihr geredet hatte und den Herrn für sie
angefleht, kam sie zurecht und zog mit uns umher zu den Ver-
sammlungen und ist seither gesund, dem Herrn sei Lob .....
Nun hatten wir unsre Arbeit in dieser Gegend getan und
verließen Anamessy. Wir gingen zu Wasser etwa 50 Meilen,
nach dem Hungerfluß ..... Danach etwa 40 Meilen nach dem
kleinen Choptankfluß .... Ehe wir dann weiter zogen, hatten
wir eine herrliche Versammlung, zu der viele Leute kamen; unter
anderem auch mehrere von den Behörden der Stadt, mit ihren
Frauen. Von den Jndianern kam einer, den sie ihren ,,Kaiser«
nannten, ein Jndianer-,,König« und sein Dolmetscher, die alle sehr
andächtig zuhörten und sehr lieb waren. Gs war eine grund-
legende Versammlung. Dies war am 23. des 1. Monats. Am
24. gingen wir 10 Meilen weit zu Wasser nach der Jndianer-
stadt, wo jener »Kaiser« wohnte. Ich hatte ihn von meinem Kommen
benachrichtigt und gebeten, seine ,,Könige« und Räte zu versammeln.
Am Morgen kam der ,,Kaiser« selber und führte mich in die Stadt;
und sie waren alle beisammen mit ihren Dolmetschern und ihren
Leuten, und die alte ,,Kaiserin« war auch da. Sie waren sehr ernst


% \picinclude{./230-239/p_s234.jpg} 
234 Kapitel 111.
und ruhig und sehr aufmerksam, mehr als manche, die sich Christen
nennen. Es waren einige mit mir, die verdvlmetschen konnten
und wir hatten eine schöne Versammlung mit ihnen, die großen
Nutzen schaffte, denn sie brachte die Wahrheit und die Freunde
bei ihnen in Ansehen. Gelobt sei der Herr! ....
Nachdem wir die meisten Gegenden dieses Landes durchzogen
hatten und die meisten der Plantagen besucht und überall, wo wir
hinkamen, Alarm geblasen hatten und Gottes Tag des Heils
den Leuten verkündet hatten, spürten wir im Geist, daß wir
bald unsere Ausgabe in diesen Weltgegenden erfüllt hätten, und
wandten uns wieder Alt-England zu. Doch es verlangte uns,
und der Herr schenkte uns die innere Freiheit dazu, bis zur
nahen Generalversammlung für Maryland zu bleiben, damit wir
alle Freunde beisammen sehen könnten, ehe wir abreisten. Die
Zwischenzeit brachten wirdamit zu, Freunde und Gemeindemitglieder
zu besuchen, Versammlungen um die Cliffs und Patuxent herum
beizuwohnen, und Antworten zu schreiben auf allerlei Ktitteleien,
welche etliche Gegner der Wahrheit erhoben und verbreitet hatten,
um die Leute zu verhindern, die Wahrheit anzunehmen; so waren
wir also nicht müßig, sondern wirkten für den Herrn, bis zur
allgemeinen Provinzialversammlung, die am 17. des 3. Monats
begann und vier Tage dauerte. Am Ersten Tage hatten die
Männer und Frauen ihre geschäftlich en Versammlungen, in welchen
die Angelegenheiten der Kirche besorgt wurden, und viele Dinge
wurden ihnen geoffenbart zu ihrer Erbauung und Trost. Die
drei übrigen Tage wurden mit öffentlichen Versammlungen zuge-
bracht, wobei verschiedene angesehene Regierungsbeamte anwesend
waren und viele andere; sie waren alle im allgemeinen befriedigt
und manchen ging es zu Herzen. Denn es war eine wundervolle,
herrliche Versammlung, und die mächtige Gegenwart des Herrn
ward überall fühlbar und gesehen; gelobt und gepriesen werde
sein heiliger Name immerdar, welcher Herrschaft gibt über alles.
.... Nach dieser Versammlung nahmen wir Abschied von den
Freunden ..... Am nächsten Tage, dem 21. des 3. Monats
1673 schifften wir uns ein für England und am 28. des 4. Monats
landeten wir im Hafen von Bristol. Dieselbe treue Hand der
Vorsehung, die uns geleitet und glücklich hinüber gebracht hatte,
wachte bei unserer Rückkehr über uns und brachte uns glücklich
zurück. Lob und Dank sei seinem heiligen Namen innnerdar!


% \picinclude{./230-239/p_s235.jpg} 
Ankunft in Bristol. Zusammentreffen mit William Penn usw. 235
Wir hatten während der Reise Viele köstliche Versammlungen
an Bord des Schiffes gehabt, gewöhnlich zwei in der Woche, in
denen die gesegnete Gegenwart des Herrn uns mächtig erquickte,
und ost goß sie sich aus über uns und machte die Anwesenden
empsänglich.
Kapitel II.
Antunst in Bristol. Zusammentreffen mit William Pen und
anderen. Verteidigung der Franetwersanunlnngen. Vorgeahnte
Gefangenschaft in Woreestet. Vries an den König über die Grund-
sätze der Quäker. Krankheit. Befreiung. Während der
Gefangenschaft versaßte Schriften.
Als wir ans Ufer kamen, begaben wir uns nach Shire-
hampton . . . und ritten von da nach Bristol .... Am Abend
schrieb ich von hier einen Brief an meine Frau:
,,Liebes Herz!
Heute Abend landeten wir in Bristol, Gott dem Herrn sei
Lob immerdar, der unser Schutzherr war und unser Schifs lenkte,
dem Gott der ganzen Erde, der Meere und Winde, der die Wolken
zu seinen Wagen macht ..... Robert Widders und James
Lancaster sind mit mir und sind gesund. Dem Herrn sei Lob
und Ehre, der uns durch so manche Gefahr geholfen, auf dem
Wasser, in den Stürmen, vor Seeräubern und andern Räubern,
in den Gefahren der Wildnis und unter den falschen Frommen.
Sein Ruhm ist über alles, Amen. Darum werdet das neue
Leben inne, und lebet ganz dem Herrn in demselben. Jch möchte,
so der Herr will, eine Zeitlang hierbleiben, vielleicht bis zum
Jahrmarkt. Genug diesmal. Meine Liebe allen Freunden.«
Bristol, 28. des 4. Monats 1673. G. F.
Zwischen diesem Tage und dem Jahrmarkt kam meine Frau
aus dem Norden zu mir nach Bristol, und ihr Schwiegersohn,
Thomas Lower und zwei ihrer Töchter kamen mit ihr. Jhr
anderer Schn-iegersohn, John Raus, William Penns) und seine
1) William Penn, Sohn des großen Admiral Penn, zeigte schon während
seiner Studienzeit in Christchurch College in Oxford ernst religiöse Bedürfnisse.
Als et, nach Hause zurückgekehrt, sich den herrschenden Hossitten nicht fügen wollte,
sagte sein Vater sich von ihm los. Er schloß sich nun den Quäkern an und


% \picinclude{./230-239/p_s236.jpg} 
236 Kapitel II.
Frau und Gerrard Robertß kamen von London; und noch viele
andere Freunde aus verschiedenen Gegenden des Landeö kamen
zum Jahrmarkt, und wir hatten herrliche Versammlungen, in
denen des Herrn Kraft über allen war. Nachdem ich meine
Arbeit für den Herrn in dieser Stadt getan, ging ich nach Gloucester-
shire, .... und von da nach Wiltshire. In Slattenford in
Wiltshire hatten wir eine schöne Versammlung, obgleich wir
manchen Widerstand erfuhren von solchen, die sich den Frauen-
Versammlungen widersetzten; der Herr trieb mich, dieselben den
Freunden anzuempsehlen zu Nutz und Frommen der Kirche Christi:
»Gläubige Frauen, welche zum Glauben an die Wahrheit be-
rufen sind und desselben köstlichen Glaubens teilhaftig gemacht
sind, wie sdie Männer, und Miterben desselben ewigen Evange-
liumö de-3 Lebens und Heil?-, sollen gleicherweise in den Besitz
und Stand der Ordnungen deß Evangeliums kommen und somit
Mitgehilfen der Männer werden bei den Neugestaltungen im
Dienst der Wahrheit in den Angelegenheiten der Kirche, wie sie
etz in den zeitlichen Dingen dez täglichen Lebens sind, damit
die ganze Haußsgemeinde Gotteß, sowohl ,Männer wie Frauen,
ihre Pflicht und Aufgabe im Haußstand Gotteß kennen, einsehen
und aus-üben möchten, damit besser für die Armen gesorgt werde,
die Jungen unterrichtet und in Gotteß-’ Wegen unteiwiesen werden,
die Wankelmittigen und Liederlichen zurechtgewiesen und getadelt
werden in der Furcht Gotteß, die Unbescholtenheit solcher, die sich
verheiraten wollen, genauer und strenger untersucht werde in der
Weißheit Gotteß, und alle Glieder des geistigen Leibeß, der Kirche,
einander bewachen und helfen in der Liebe.« Aber nachdem diese
Gegner sich lange in Hader und Zank ereisett hatten, warf der
Herr einen ihrer Führer darnieder, sodaß er sich [demtitigte und
daß Unrecht einsah, das er tat, als er sich Gotteß himmlischer
Macht widersetzte, er gestand den Freunden seinen;Jrrtum ein
und veröffentlichte später ein Schreiben, in swelchemIser erklärte,.
er hätte sich eigensinnig widersetzt, trotz meiner shäusigen War-
nungen, biß das Feuer des Herrn in ihm entbrannt sei, und er
den Engel dez Herrn mit gezogenem Schwert gesehen habe, im
wurde ihr bedeutendstes Mitglied. Aus einem ihm von Karl ll. überlassenen
und nach ihm Pennsyloanien genannten Landstrich am Delaware gründete er
eine Fteistatt für seine verfolgten Glaubens-genosseii. Müheree s. Weingarten
a. a. O. S. 408.)


% \picinclude{./230-239/p_s237.jpg} 
Ankunft in Bristol. Zusammentreffen mit William Penn usw. 237
Begriff ihn zu töten .... Wir besuchten viele Freunde und
kamen schließlich nach Kingston, wo ich mit meiner Frau und
ihrer Tochter Rachel zusammen traf. Jch hielt mich nicht lange
hier auf, sondern ging nach London, wo die Baptisten mit einigen
Sozinianern sehr bööartig geworden waren und Viele Bücher gegen
unß gedruckt hatten, und ich hatte oiele Arbeit in der Kraft dez
Herrn, ehe ich auß dieser Stadt loökommen komite .....
Darauf machte ich eine kurze Reise durch einige Gegenden
in Essex und kehrte nach London zurück, wohin ich mich innerlich
gezogen fühlte, denn ich hatte gehört, daß man viele Freunde
oor die Richter gebracht hatte und etliche in London und an andern
Orten gefangen genommen hatte, weil sie ihre Schaufenster an
Festtagen und an sogenannten Fasttagen geöffnet hatten, und
weil sie Zeugniß ablegten gegen alleß Feiern solcher Tage. Die
Freunde mußten dietz tun, weil sie ja wußten, daß die wahren
Christen die Feste der Juden zur Zeit der Apostel nicht hielten,
und so konnten sie auch nicht die sogenannten Feste der Heiden
und Papisten, welche unter den sogenannten Christen seit den
Tagen der Apostel eingesetzt wurden, halten. Denn wir sind be-
freit worden vom Halten bestimmter Tage, durch Jesum Christum
und sind in den Tag gebracht worden, der auß der Höhe aufge-
gangen ist, und wir sind jetzt in Jhm, der Herr ist über den
jüdischen Sabbath, und über daß Wesen der jüdischen Zeichen ....
Bald darauf, alß ich, in Adderbury, am Nachtessen saß, fiihlte
ich, daß ich gefangen genommen werden würde, ich sagte aber noch
niemand etwaß .... Am andern Tage hatten wir eine Versamm-
lung .... Nach derselben saß ich mit einigen Freunden in einem
Zimmer im Gespräch, da kam Henry Parker, ein Friedenörichter,
inß Hauß, und mit ihm Rowland Hainö, ein Priester aus Hunniton
in Warwickshire, . . . und Henry Parker nahm mich gefangen und
Thomaß Lower zur Gesellschaft mit, und obgleich er nichtö gegen
uns Vorbringen konnte, schickte er une beide inß Gefängniß von
Woreester, durch einen merkwürdigen Verhaftbefehl .....
Da ich derart zum Gefangenen gemacht worden war, ohne viel
Aussicht, vor der oierteljährlichen Gerichtösitzrmg frei zu werden,
veranlaßten wir einige Freunde, meine Frau und ihre Töchter
nach dem Norden zu begleiten ..... A16 ich dachte, daß meine
Frau zu Hause angelangt sei, schrieb ich ihr auö der Gefangen-
schaft folgenden Brief:


% \picinclude{./230-239/p_s238.jpg} 
238 Kapitel K);.
,,Liebes Herz!
Du schienst ein wenig betrübt, als ich vom Gefängnis sprach
und dann gefangen genommen wurde. Ergib dich in den Willen
des Herrn. Denn als ich im Hause des John Rous in Kingston
war, hatte ich ein Gesicht, daß ich gefangen genommen würde,
und als ich bei Bray Doily in Oxfordshire war, sah ich, während
ich am Abendessen saß, daß ich gefangen genommen und Leiden
zu erdulden haben würde. Aber des Herrn Macht ist über allem, sein
Name sei gepriesen ewiglich« ..... G. F.
Wir wurden erst am letzten Tage der Gerichtssitzung vorge-
nommen, am 21. des 11. Monats 1673 ..... Jch mußte
Bericht über meine Reise geben. Friedensrichter Parker hatte,
um den Fall recht schwer scheinen zu lassen, eine große Geschichte
gemacht, es seien Leute von London, von Cornwall, aus dem
Norden und von Bristol im Hause gewesen, als man mich ge-
fangen genommen habe; hieraus erklärte ich ihm, daß das alles
gewissermaßen eine einzige Familie gewesen sei, 1) indem niemand von
London dagewesen sei, als ich selber, niemand aus dem Norden als
meine Frau und meine Töchter, niemand von Cornwall als mein
Schwiegersohn.
Als ich fertig geredet, stand der Vorsitzende, Simpson, ein
einstiger Presbyterianer, auf und sagte: ,,Was Jhr da sagt, klingt
recht unschuldig«. Daraus flüsterten er und Parker eine Weile
miteinander und daraus stand er wieder auf und sagte: ,,M. Fox,
Jhr seid ein ausgezeichneter Mann und alles, was Jhr da sagt,
mag wahr sein, aber, um uns ganz zu befriedigen, wollt Ihr den
Huldigungseid leisten ?« Jch erwiderte ihm, sie hätten versprochen,
uns keine Falle zu stellen, dies sei aber einfach eine Falle, da sie
ja wissen, daß wir nicht schwören können. Sie ließen dennoch
den Eid vorlesen. Ich erklärte ihnen daraus: »Jch habe nie in
meinem Leben einen Eid geleistet, aber ich bin immer der Regierung
gehorsam gewesen; ich war im Kerker von Derby sechs Monate,
weil ich die Waffen nicht gegen König Karl bei Woreester er-
heben wollte, und weil ich in die Versammlungen ging, wurde
ich nach Leicestershire gebracht, vor Oliver Eromwell, als einer der
1) Der Konventikel-Akt von 1664 sagt: ,,jede Prioatandacht von mehr
als fünf Personen außer der Familie, wobei nicht das Common-Prayer-Boot
zugrunde gelegt wird, wird mit dreimonuttichem Gefängnis, zum dritten mal
mit Verbannung bestruft.«


% \picinclude{./230-239/p_s239.jpg} 
Ankunst in Bristol. Zusammentreffen mit William Penn usw. 239
Mitverschworenen für die Rückkehr des Königs. Jhr wißt es ja
nach euren eigenen Gewissen, daß wir, die ihr Quäker nennt
keinen Eid leisten können, weil Christus es verboten hat. Was
aber den Jnhalt des Eides anbelangt, so kann ich sagen, und
sage es auch, daß ich den König von England als den recht-
mäßigen Erben und Nachfolger des englischen Reiches anerkenne
und alle Verschwörer und Verschwörungen gegen ihn verabscheue,
und ich hege nur Liebe und Wohlwollen in meinem Herzen gegen
ihn und gegen alle Menschen und wünsche ihm und ihnen allen
nichts als Gutes. Der Herr weiß es, vor welchem ich als ein
unschuldiger Mann stehe. Was den Suprematseid anbelangt, so
verabscheue ich den Papst und seine Macht und seine Religion
von ganzem Herzen« .... Aber sie ließen mich vom Gefangen-
wärter hinweg führen und brachen so ihr Versprechen dem Lande
gegenüber, denn sie hatten mir freies Reden zugestanden und mir
es doch nicht gewährt .....
Während meiner Gefangenschaft kam es über mich, unsere
Anschauungen und Grundsätze dem König darzutun, nicht haupt-
sachlich um meiner eigenen Leiden willen, sondern damit er unsere
Anschauungen und unsere Gemeinschast besser verstehe.
,,An den König!
Der Ausgangspunkt der Quäker ist der Geist von Christus,
der für uns gestorben und um unsrer Gerechtigkeit willen auferweckt
worden ist, durch welchen wir wissen, daß wir sein sind. Gr
wohnet in uns mit seinem Geist und der Geist Christi macht uns
srei von aller Ungerechtigkeit und Gottlosigkeit. Der Geist Christi
machet, daß wir allem gottlosen Wesen absagen, als da sind
Lügen, Stehlen, Töten, Ghebrechen, Hurerei und alle Arten von
Unreinigkeit, Unzucht, Bosheit, Haß, Betriigerei, Schlemmen und
alle Werke des Teufels. Der Geist Christi führt uns dazu, für
alle Menschen den Frieden und das Gute zu suchen und friedlich
zu leben. Gr machet, daß wir uns aller Anschläge und Ver-
schwörungen gegen den König oder irgend sonst Jemand ent-
halten. Er hält uns zurück von jenem bösen Tun und Treiben,
gegen welches das Schwert der Obrigkeit sich richtet. Unser Wunsch
und Bestreben ist, daß alle, die Christus bekennen, auch im Geiste
Christi wandeln, damit sie durch denselben des Fleisches Geschäfte
töten möchten und mit dem Schwert des Geistes ihre Sünde und
Bosheit ausrotten. Dann würden die Richter und Beamten nicht

