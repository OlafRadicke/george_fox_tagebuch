% \picinclude{./110-119/p_s110.jpg} 
110 Kapitel 11.
Jch wiederholte, man solle etz vorlesen, damit alle urteilen könnten,
ob etwas Verführerischeß darin sei, in dem Falle wolle ich dafür
leiden. Schließlich laö ez der Angestellte mit lauter Stimme, daß
alle es hören konnten; alö er fertig war, sagte ich: »ja, es ist
mein Blatt, ich stehe dazu, und ihr müßt auch dazu stehen, wenn
ihr nicht die Schrift verleugnen wollt, denn ist es denn nicht, maß
die Schrift sagt, und Christuö und die Apostel, denen alle wahren
Christen gehorchen müßen?« Nun ließen sie den Gegenstand
fallen, und der Richter kam wieder auf unsere Hüte zurück und
hieß den Kerkermeister sie une abnehmen, dieser tat es; aber wir
setzten sie wieder auf ..... Der Richter hielt nun eine lange
Rede über den Lord Protektor, wie er ihn zum obersten Richter
in England gesetzt, und ihn hierhergeschickt, und dergleichen mehr.
Wir baten ihn, er solle uns Gerechtigkeit erzeigen nach unserer
ungerechten Gefangenschaft diese nenn Wochen; statt dessen aber
brachten sie eine Anklage vor, die sie gegen unz zusammengesetzt
hatten, so voll Lügen, daß ich meinte, sie richte sich gegen einen
Dieb: wir seien nur mit Wasfengewalt und nach großem Wider-
stand hierher gebracht worden! und doch waren wir, wie oben
gemeldet, gekommen. Ich sagte ihnen, daß sei falsch und wir
wiederholten unser Gesuch um Gerechtigkeit; die Gefangennahme,
sagte ich, sei ungerecht, denn ich sei auf der Reise von Major
Ceely festgenommen worden. Ntm redete Peter Eeely mit dem
Richter und sagte, auf mich zeigend: »Grlaubt mein Herr, dieser
Mann nahm mich bei Seite und sagte, er könne in einer Stunde
vierzigtaufend Mann stellen und da-? Land in Blut stürzen und
König Karl zurückbringen, und ich könne ihm dabei behilflich sein.
Jch wollte ihm auö dem Lande helfen, aber er wollte nicht
gehen; ich habe Zeugen, die?7 zu beschwören«, und er rief den
Zeugen auf. Aber der Richter war nicht gewillt, ihn anzuhören,
und so bat ich, man möchte meine Anklage, auf Grund deren
ich verhaftet sei, vorlesen; der Richter sagte: ,,nein, sie soll nicht
vorgelesen werden«, .... als ich sah, daß man sie nicht lesen
wollte, sagte ich zu einem meiner Mitgefangenen, »du hast eine
Abschrift davon, lies die vor.« ,,Kerkermeister«, sagte hieraus der
Richter, »führ ihn sort! wir wollen doch sehen, wer hier Meister
ist, er oder ich!« und so wurde ich hinweg geführt. A16 ich
wieder gerufen wurde, bestand ich wiederum darauf, daß mein
Verhastbesehl vorgelesen werde, denn davon hing meine Gefangen-


% \picinclude{./110-119/p_s111.jpg} 
Angriffe der Jndependenien und Ptesbyterianer usw. 111
schaft ab. Jch hieß abermalö meinen—Mitgesangenen ihn lesen,
und er tat ez:
,,Peter Ceely, einer der Friedenörichter der Grafschaft, an
den Kerkermeister von Seiner Hoheit Gefängniö zu Launeeston:
»Jch sende Euch hiermit durch den Überbringer dieser- die
Personen Edward Pyot von Bristol und George Fox auz Dray-
ton-in-the-Elay in Leicestershire, und William Salt von London, . .
die alt; Quäker bekannt sind und sich selber als.3 solche bekennen;
sie haben Verschiedene Blätter verbreitet, die den öffentlichen
Frieden gefährden, und können keinen gesetzlichen Grund für ihr
Erscheinen in dieser Gegend angeben, sie sind gänzlich unbekannt
in dieser Gegend, haben keinen Paß, weigern sich, irgendwelche
Beweise ihreö guten Wandels zu geben, die da; Gesetz verlangt,
und weigern den Abschwörungtzeid zu leisten. Wir befehlen euch
darum im Namen seiner Hoheit dez? Lord Protektor, diese Per-
sonen, . . . wenn sie kommen, in Gewahrsam zu bringen und da-
rin zu lassen, bis sie gesetzlich srei gelassen werden. Versäumet
nicht, solcheö zu tun, wo andere'- eö euch gefährlich werden könnte.
Auögegeben mit meiner Unterschrift und Siegel, St. Joes, den
18. Januar 1655. P. Eeely.«
Als dietz vorgelesen worden war, sagte ich zu den Richtern, . . .
mich an Major Ceely wendend: »Wo und wann habe ich dich
beiseite genommen? .... und wenn du mein Ankläger bist, wa-
rum sitzest du auf der Richterbank? du solltest herunter kommen
und mir ins Gesicht sehn. Übrigens möchte ich fragen, ob nicht
Major Ceely sich der- Verrat-3 schuldig machte, dessen er mich an-
klagt, durch sein langeö Schweigen? Kennt er seinen Platz als
Soldat wie alö Friedentzrichter? denn .... wenn ich ihn beiseite
genommen, um ihm zu sagen, ich könne oierzigtaus end Mann stellen,
und so weiter, .... so sehet ihr deutlich, daß er ja in dieser
Verschwörung beteiligt gewesen wäre, indem er mich .... aus
der Gegend forthaben wollte .... und den Verrat nicht früher
entdeckte. Aber ich leugne seine Autzsagen und bin unschuldig an
diesem teuflischen Plan.« Die Richter ließen nun die Sache fallen,
denn sie sahen daß, anstatt daß sie mich in eine Falle gelockt hatten,
ich selber ihnen eine gestellt hatte. Major Eeely behauptete nun,
ich habe ihm ins Gesicht geschlagen ..... Jch fragte ihn, ob er
sich alö Richter und Soldat nicht schäme, solcheö zu sagen ....
Schließlich, als die Richter sahen, daß diese Fallen nichts nutzten,


% \picinclude{./110-119/p_s112.jpg} 
112 Kapitel llc.
ließen sie uns wieder ins Gefängnis führen und forderten von
jedem zwanzig Goldstücke, weil wir den Hut ausbehalten .....
Als das Urteil so lautete, daß keine baldige Freilassung
zu erwarten war, hörten wir aus, dem Wörter wöchentlich
7 Schilling für unsere Pferde und 7 für uns selber zu geben;
daraufhin wurde er böse und ganz teuflisch und brachte uns nach
Doomsdale hinunter, einen greulichen, stinkenden Ort wohin die
Mörder nach der Verurteilung gebracht wurden. Der Ort war
sehr ungesund, so daß wenige, die sich hier aushalten mußten,
wieder gesund heraus kamen; es war kein Abtritt da, und der
Unrat der Gefangenen war seit Jahren nie hinausgeschasst worden.
Es war ein förmlicher Sumpf darin, stellenweise bis über die
Schuhe, von dem Unrat; und man erlaubte uns nicht, rein zu
machen oder uns Betten oder Stroh zum drauf liegen zu ver-
schaffen. Am Abend brachten uns einige Bekannte aus der Stadt
ein Licht und etwas Stroh, um draus zu liegen; wovon wir einiges
verbrannten, um den Gestank zu vertreiben. Die Diebe schliefen
gerade über uns und der Wörter in einem Zimmer daneben.
Scheints drang der Rauch ins Zimmer des Wärters; er geriet
in einen solchen Zorn, daß er die Nachtgeschirre der Diebe nahm,
und sie durch ein Loch gerade auf unsere Köpfe ausleerte; wir
waren so beschmiert davon, daß wir weder einander noch uns selber
anrühren konnten; und der Gestank war so arg, daß wir fast
darin erstickten. Vorher hatten wir den Gestank zu unsern Füßen
gehabt, jetzt hatten wir ihn auch aus den Köpfen und am Rücken;
und da unser Stroh von dem heruntergeworsenen Dreck beschmutzt
war, so verbreitete es einen greulichen Dunst. Zudem fluchte der
Wörter gräszlich über uns und nannte uns »hackengesichtige Hunde«;
und andere merkwürdige Namen, die wir noch nie gehört. Jn
diesem Zustand gingen wir fast zu Grunde während der Nacht,
denn wir konnten nicht einmal absitzen, alles war so voll Unrat.
Wir mußten lange in dem Zustand ausharren, bis uns gestattet
wurde, reinzumachen und uns andere Lebensmittel zu verschaffen als
das, was durchs Gitter kam. Einmal brachte einMädchen uns etwas
Essen; der Wärter arretierte es und führte es vor Gericht, weil
es ins Gefängnis eingedrungen sei, und es geriet in große Not;
dadurch wurden viele andere entmutigt, so daß es uns schwer
wurde, uns Wasser oder Lebensmittel zu verschaffen. Wir ließen
nun eine junge Frau aus Londen kommen, Anna Downer, damit


% \picinclude{./110-119/p_s113.jpg} 
Angriffe der Jndependenten und Presbhterianer usw. 113
sie uns das Essen kaufe und zubereite; sie war dazu bereit, denn
es war über sie gekommen, zu uns zu kommen in der Liebe
Gottes, und sie war sehr dienstfertig gegen uns .....
Die Gefangenen und einige andere verschrobene Leute be-
richteten von Gespenstern, die in Doomsdale umgingen, und von
den vielen, die hier gestorben seien, um einen damit angst zu
machen. Aber ich sagte, daß, wenn auch alle Geister und Teufel
der Hölle dort seien, ich darüber stehe, durch die Kraft Gottes,
und nichts dergleichen fürchte; denn Christus unser Priester werde
uns das Haus und die Mauern heiligen, er, der dem Teufel den
Kopf zerbrochen habe .....
Es war gtun bald die Zeit der allgemeinen vierteljährlichen
Getichtssiizuttg, und da der Kerkermeister sich immer noch schlecht
gegen uns benahm, setzten wir einen Bericht über unsere Leiden
auf und schickten ihn zur Gerichtsfitzung nach Bodmin. Als die
Richter ihn gelesen, gaben sie den Befehl, daß die Türen von
Doomsdale geöffnet werden sollten und man uns erlaube, rein
zu machen und unsere Nahrung in der Stadt zu kaufen. Wir
sandten eine Abschrift unseres Leidensberichts an den Protektor
und erzählten ihm, wie wir von Major Ceely verhaftet und ver-
urteilt worden waren, und wie uns der Kerkermeister mißhandelt
hatte. Der Protektor schickte einen Befehl an Hauptmann Fox,
den Befehlshaber von Schloß Pendennis, daß er untersuche, wie
es sich mit den Soldaten, die uns mißhandelten, verhalte .....
Solches war der Sache des Herrn sehr förderlich; denn
nachher konnten die Freunde in jedem Turmhaus oder Markt-
platz reden und es tat ihnen niemand etwas. Jch hörte, daß
Hugh Peters, einer der Kapläne des Protektor, diesem gesagt habe,
man könne dem George Fox keinen größern Dienst zur Ausbrei-
’ tung seiner Ansichten in Cornwall tun, als ihn in Cornwall ein-
zusperren. Und wirklich kam meine Gesangennahme in Cornwall
vom Herrn zur Förderung seiner Sache in dieser Gegend; denn
als es nach der Gerichtssitzung hieß, wir würden gefangen bleiben,
kamen Freunde aus allen Teilen des Landes, um uns zu besuchen.
Diese westlichen Gegenden waren damals sehr in Finsternis, aber
das Licht und die Kraft und die Wahrheit des Herrn brachen
nun hervor und leuchteten über allen, und viele bekehrten sich
von der Finsternis zum Licht und von der Macht des Satans
zu Gott. Ge trieb viele, in die Turmhäuser zu gehen, und viele
George Fox. 8


% \picinclude{./110-119/p_s114.jpg} 
114 Kapitel llc.
besuchten unö, denn wir durften nun umher gehen im Schloßhof,
und an den Ersten Tagen kamen oiele zu unß, denen wir daß
Wort des Lebenö brachten .....
Ju Cornwall, Deoonfhire, Dorsetshire und Somersetshire
fing die Wahrheit an mächtig zu sprießen, und viele bekehrten sich
zuzEhristuS; viele Freunde fühlten sich getrieben, die Wahrheit
in diesen Gegenden zu verkünden, waS die Priester und die
,,Frommen« sehr aufbrachte, so daß sie die Behörden anstifteten,
den Freunden Fallen zu stellen. Sie stellten Wachen auf den
Landstraßeu, unter dem Vorwand, alle verdächtigen Personen
abzufassen, sie ergriffen nun daraufhin Freunde, die vorbeikamen,
um unß im Gefängnis zu besuchen ..... Aber gerade daß,
waö sie taten, um der Wahrheit Einhalt zu tun, diente dazu, sie
auözubreiten; denn dadurch wurden die Freunde oft getrieben, zu
den Konstablern oder den Behörden, vor die sie gebracht wurden,
zu reden, was viel dazu beitrug, daß die Wahrheit sich in allen
Distrikten aus-breitete. Oft wenn Freunde in die Hände der
Wachen gerieten, ging es zwei oder drei Wochen, ehe sie wieder
frei wurden.
Alß Thomaß Rawlinson au-J dem Norden her kam, um unz
zu besuchen, ergriff ihn ein Konstabler in Devonshire und nahm
ihm nachts zwanzig Schilling aus der Tasche, und darauf wurde
er zu Exeter ins Gefängniß geworfen. Henry Pollexfen warfen
sie auch inß Gefängniz, weil er ein Jesuit sei ..... Viele
Freunde wurden von ihnen mißhandelt; ja Leute, die an ihrer
Arbeit waren, wurden von ihnen gepeitscht und ergriffen, und
ez waren doch solche darunter, die eine Einnahme von mehr
als achtzig und hundert Pfund im Jahr hatten; und zwar
geschah ihnen solcheö, wenn sie kaum vier oder fünf Meilen von
zu Hause weg waren. Unter dem Eindruck all des Bösen, daß «
mit dem Aufstellen der Wachen und dem Gefangennehmen der
Freunde beabsichtigt war, kam ez über mich folgendeß zu schreiben:
,,Eine Mahnung und Warnung an die Behörden.
»Jhr Mächte der Erde, Christus ist gekommen um zu regieren,
und er ist unter euch, und ihr kennetihn nicht; er erleuchtet einen
jeden unter euch, damit ihr alle an ihn glauben möchtet, an
das Licht; an den »der die Kelter allein tritt«. Darum prüfet
alle in diesem Lichte, ob ihr reif seid, denn die Kelter ist bereit.
(Offb. 14, 19) ....


% \picinclude{./110-119/p_s115.jpg} 
Angriffe der Jndependenten und Presbhterianet usw. 115
»Jht verkündet Gewissensfreiheit; und doch darf man seinen
Freunden keine Briefe bringen, oder seine Freunde oder die
Gefangenen besuchen, oder ihnen Bücher bringen, ohne daß ihr
Wachen ausstellt, um sie anzuhalten und zu greifen; und sogar
bewaffnet müssen diese sein gegen die guten Leute, die kaum
einen Stock mit sich tragen, und die ihr aus Groll Quäker nennt.
Und die, welche diese Wachen ausstellen, die verkünden Gewissens-
freiheit und nehmen solche gefangen, die ihr Gewissen gegen Gott
und gegen die Menschen rein erhalten wollen, die Gott im Geist
und in der Wahrheit anbeten, was die, welche nicht im Licht sind,
Ketzerei nennen! . . . Jst je solch ein Geschlecht gewesen, das
so wahnsinnig schlecht und verfolgungssüchtig war und Bewasfnete
ausstellte gegen die Wahrheit und sie verfolgte, wie Grafschasten
und Städte es jetzt tun? das klingt wie Sodom und Gomorrah!«
G. F.
Gs kam mir eine Abschrift eines von der Sitzung von E-xeter
ausgehenden Verhastbesehls in die Hände, der in starken Aus-
drücken verlangte, ,,alle Quäker zu verhasten«, und der die Wahr-
heit und die Freunde schlecht machte; da trieb es mich, eine
Antwort zu schreiben und zu verbreiten, um die Freunde und
die Wahrheit gegen solche Verleumdungen zu verteidigen, und
die Schlechtigkeit und Bosheit des Verleumdungsgeistes zu zeigen. . .
Wir blieben im Gefängnis bis zur nächsten Sitzung; viele
Freunde, Männer und Frauen, die von der Wache ergriffen
worden waren, waren ins Gefängnis gebracht worden. Viele
von ihnen wurden nach Eröffnung der Sitzung vor die Richter
gebracht und beschuldigt, sie hätten sich gesträubt zu kommen
und waren doch von den Gefängniswärtern gebracht worden.
Der Richter legte ihnen Bußen auf, weil sie den Hut nicht ab-
nehmen. Wir hingegen mußten nicht mehr oor den Richter.
Während dieser ganzen Zeit und während der Sitzungen war
Unser Wirken für den Herrn reich gesegnet, denn es kamen viele
zu uns, »Fromrne« und andere, mit uns zu reden. Elisabeth
Trelawm) von Plymouth, die Tochter eines Barons, wurde bekehrt,
worüber Priester und ,,Fromme« und viele angesehene Personen
außer sich waren und ihr Briefe deswegen schrieben. Da sie eine
weise und gottselige Frau war und denen, die ihr geschrieben,
nicht wollte etwas in die Hand geben, das sie dann hätten können
gegen sie gebrauchen, so schickte sie mir die Briefe; und ich schrieb
 


% \picinclude{./110-119/p_s116.jpg} 
116 Kapitel 1:.
ihr darüber, rmd sie beantwortete sie dann. Sie nahm zu in der
Kraft und der Weiöheit Gotrtesz, so daß sie zuletzt imstande war,
den msichtigsten Priestern und ,,Frommen« zu antworten; sie hatte
die Herrschaft über sie in der Wahrheit durch die Kraft Gotteö,
der sie treu blieb biz in den Tod.
Während ich hier in der Gefangenschaft war, prophezeiten
die Baptisten und Fifthmonarchyleute, in diesem Jahre werde
Christus- kommen und tausend Jahre auf Grden regieren. Sie
erwarteten dieseö Reich als ein äußereö, während er doch in die
Herzen der Menschen gekommen war, um darinnen zu regieren;
aber so wollten ihn diese ,,Frommen« nicht aufnehmen, darum
mißlang ihnen ihr Prophezeien. A, Ehristuö ist ja schon ge-
kommen tmd wohnet in den Herzen der Menschen und regieret
darin. Tausende, bei denen er anklopfte, haben ihm aufgetan;
und er ist bei ihnen eingekehrt und hat das Abendmahl mit ihnen
gehalten (Offb. 3, 20). Z Viele dieser Baptisten und Fisthmonarchh-
leute sind die ärgsten Feinde derer, die sich zu Ehristuß hielten,
geworden; aber er regieret in den Herzen seiner Heiligen.
Während der Gerichtßsitzung kamen mehrere der Richter zu
unö und waren ziemlich höflich und redeten vernünftig über
göttliche Dinge mit uns und bezeugten unß Teilnahme. Haupt-
mann Fox, der Gouverneur von Schloß Pendenniö, trat zu mir
mid sah mir inß Gesicht, sagte aber nichtß; aber alß er wieder
zu seinen Begleitern zurück kam, sagte er, er habe noch nie in
seinem Leben einen einfältigeren Menschen gesehen. Jch rief ihm
nach: ,,Wir wollen sehen, wer der Ginfältigere ist!« Aber er
ging seineß Wegeß, der hochmütige Tropf.
Thomas Lower 1) besuchte unß ebenfallß .... Er stellte uns
viele Fragen darüber, daß wir behaupteten, die Schrift sei nicht
das- Wort Gotteö, und über die Sakramente und andere?-, und
wir konnten über alleß Ausschluß geben. Jch redete auch noch
allein mit ihm, und er bekannte nachher, meine Worte hätten
ihn wie ein Blitzstrahl durchzuckt. Er habe noch nie Leute, wie
wir seien, getroffen, die seine innersten Gedanken errieten. Gr
wurde nachher bekehrt und ist ein Freund geblieben biß auf diesen
Tag .... und hat viel um der Wahrheit willen gelitten. R
EZ trieb mich zu dieser Zeit, die folgende Mahnung an die
Freunde, die Prediger waren, zu richten:
1) Thomas Lower, später Schwiegersohn von G. F.


% \picinclude{./110-119/p_s117.jpg} 
Angriffe der Jndependenten und Presbyteriuner usw. 117
,,Freunde!
Bleibet in der Kraft des Lebenö und der Weisheit und in
der Furcht des Herrn Himmels und der Erde, damit ihr in der
Weiöheit Gottes bewahret bleiben möget und seinen Gegnern ein
Schrecken werdet, indem ihr die Wahrheit verbreitet, Zeugen für
sie erwecket, die Betrügerei stürzet, von der Übertretung zum
Leben bringt, in den Bund deß Lichtß und den Frieden Gotteö.
Lasset alle Welt diese Stimme hören, durch Wort oder Schrift.
Schonet keinen Ort, noch Sprache, noch Feder; tuet das Werk
in Gehorsam gegen Gott; kämpfet tapfer für die Wahrheit auf
Erden und zertretet alles, waß ihr entgegen ist. Jhr habt die
Kraft, mißbraucht sie nicht ..... Regieret mit Ehristuß, dessen
Thron und Szepter nun ausgerichtet sind, und der herrscht bis an
die Enden der Erden ..... GH soll nun daß Heil außgehen
von Zion, zu richten den Berg Esauß, und daß Gesetz soll von Jeru-
salem au?-gehen (Obad), damit es redezu dem Göttlichen, daßin einem
Jeden ist, und alle Erfindungen und Erfinder überwältige. Alle
Fürsten der Welt sind Luft vor der Macht Gotteß, die ihr ge-
schmecket habt; darum lebet in ihr .....
Ftthret alle zur Anbetung Gotteö; pflüget den brachliegenden
Acker, dreschet daß Korn, damit der Same, der Weizen, in die
Scheunen gesammelt werden könne und Alle zum Ursprung,
. zu Ehristuß, kommens der war, ehe der Welt Grund gelegt ward.
Die Spreu ist durch die Übertretung unter den Weizen gekommen;
der, welcher ihn auödrescht, hat die llbertretung verlassen und
erkennt sie, und unterscheidet zwischen dem Wertoollen und dem
Unwerten, er kann den Weizen vom Unkraut unterscheiden und
ihn in den Speicher sammeln und bringet so die unsterbliche
Seele zu Gott, von dem sie kamsf. . . Die Prediger des Geisteß
müssen dem gefangenen Geist predigen, damit durch den Geist
Christi die Menschen zu Gott, dem Vater alleö Geistes, geführt
werden, ihm zu dienen und eins zu sein mit ihm, mit der Schrift
und untereinander. Dieß ist das Wort deß Herrn an euch alle.
. .   Seid ein Vorbild und Beispiel in allen Ländern, Ort-
schaften, Jnseln und Völkern, zu denen ihr kommt, damit euer
Wandel allen Menschen predige .... Darin werdet ihr dem
Herrn angenehm sein und ein Segen werden.
Schonet den Betrug nicht; greiset ihn mit dem Schwert an;
bekämpfet ihn; trachtet nicht nach Blut, weder in Wort noch


% \picinclude{./110-119/p_s118.jpg} 
118 L Kapitel 11.
Schrist ..... Verktindet allen den lebendigen Gott; denn alle
Lehre, Kirche und Gottesdienst, die durch menschlichen Willen und
Verstand eingesetzt sind, werden von der Kraft Gottes vernichtet.
.... Verkiindet den großen Tag des Feuerß und des Schwerteß,
den Tag deö Herrn, der im Geist und in der Wahrheit will
angebetet sein, und bleibt in der Kraft Gottes, damit die Bewohner
der Erde vor euch erzittern; und damit die Kraft und Herrlich-
keit deö Herm unter den Heiden Hund den Heuchlern gepriesen
werde, und ihr in der Weisheit und Furcht, im Leben, im Schrecken
und in der Herrlichkeit bewahrt bleibt zu seiner Ehre. EZ gehet
ein Ruf, daß man die Übertretung verlasse, und der Geist ruft:
,,kommet«. EZ ergehet jetzt ein Ruf, die falschen Gotteödienste
zu verlassen und dem wahren Gott zu dienen; ein Ruf zur
Buße .... damit die Gerechtigkeit hervorbreche; und sie wird
über die ganze Erde sich ausbreiten. Darum tut treu in der
Kraft dee- Herrn euer Werk, ihr, die ihr au?-erwählt seid .....
Gehorchet der Kraft, sie wird euch erretten auß der Hand der
Unoernünftigen und von der Welt. Durch sie werdet ihr daß
Reich haben, dasz kein Ende hat und in welchem Herrlichkeit und
Leben isi.« .... G. F.
Nach der Sitzung hatten wir manche Unterredungen mit dem
Scheriff und einigen Soldaten, die eine zum Tode verurteilte Frau
bis zur Hintichnmg überwachen mußten. Einer von ihnen sagte:
,,Ehristuö war einer der heftigsten Menschen, die je gelebt«; wir
verwiesen ihm dies. Ein andermal fragten wir den Kerkermeister,
was- bei den Gericht?-Verhandlungen vorkomme. Er antwortete:
,,O, nur Kleinigkeiten; nur etwa dreißig, die wegen Bastardschaft
verurteilt sind.« Wir wunderten untz sehr, daß solche, die doch
Christen zu sein meinten, derartigeö eine Kleinigkeit fanden.
Aber dieser Kerkermeister war selber ein sehr schlechter Mensch.
Jch ermahnte ihn oft zur Rechtschaffenheit, aber er behandelte
die Leute, die unz besuchen wollten, schlecht. Edward Pyot bekam
einen Käse von seiner Frau geschickt; der Kerkermeifter nahm ihn
ihm weg und brachte ihn dem Major, angeblich um ihn auf
verräterische Briefe hin zu durchsuchen; aber obwohl sie nichts
von Vriefett fanden, so behielten sie ihn doch. EZ hätte diesem
Kerkermeister ganz gut gehen können, wenn er sich anständig
betragen hätte, aber er suchte selber sein Verderben, welcheö auch
bald über ihn kam; denn im darauf folgenden Jahre wurde er


% \picinclude{./110-119/p_s119.jpg} 
Angriffe der Jndependenten und Presbyterianer usw. 119
von seiner Stelle abgesetzt und kam selber ins Gefängnis und
bettelte dort bei den Freunden. Und wegen irgend eines Ver-
gehens brachte ihn sein Kerkermeister nach Doomsdale und legte
ihn in Ketten und schlug ihn, und erinnerte ihn daran, wie er
jene guten Leute mißhandelt habe, die er ohne jeden Grund in
diesen greulichen Kerker getan, und daß er nun die verdiente
Strafe für seine Bosheit leiden müsse, und ihm nun mit dem
Maße gemessen werde, mit dem er gemessen habe. Es ging ihm
sehr schlecht und er starb in der Gefangenschaft, und sein Weib
und seine Kinder kamen ins Elend.
Während ich zu Launceston gefangen war, ging ein Freund
zu Oliver Cromwell und erbot sich, an meiner Statt in Dooms-
dale gefangen zu sein, wenn er es annehmen und mich dafür
in Freiheit setzen wolle. Dies erstaunte Eromwell dermaßen,
daß er zu seinen Räten sagte: ,,Welcher unter Euch würde so oiel
für mich tun, wenn ich in dieser Lage wäre?« Und obgleich er
das Anerbieten des Freundes nicht annahm, sondern sagte, er
könne es nicht tun, weil es gegen das Gesetz sei, so ergriff ihn
doch die Wahrheit mächtig. Einige Zeit darauf schickte er den
Generalmajor Desborough in der Absicht, uns frei zu lassen;
dieser kam und bot uns die Freilassung an unter der Bedingung,
daß wir oersprechen, heim zu gehen und nicht mehr zu predigen,
, aber wir wollten ihm nichts versprechen; daraufhin schlug er uns
vor zu versprechen, nach Haus zu gehen, ,,wenn der Herr es zu-
lasse«, worauf Edward Phot ihm einen abschlägigen Brief schrieb.
Als einige Zeit verstrichen war, seitdem dieses Schreiben
abgegeben worden war, schrieb ich ebenfalls an ihn, folgender-
maßen:
,,Freund,
Wir, die wir in der Kraft Gottes des Herrschers aller Dinge
sind, die wir seine Kraft kennen und in ihr wohnen, müssen ihr
auch gehorchen; und darum müssen wir uns frei halten von
allem, was Menschenwille befiehlt. Wenn es sich darum handelt,
etwas zu kaufen oder zu verkaufen, so mag es etwa angehen zu
sagen: wir wollen, so der Herr es zuläßt; aber da wir in der
Kraft Gottes stehen, unter keines Menschen Willen, so können wir
solches nicht mit Wahrhaftigkeit sagen, wo es sich um unsere
Befreiung aus der Gefangenschaft handelt .....
13. des 6. Monats 1656. G. F.

