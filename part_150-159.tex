% \picinclude{./150-159/p_s150.jpg} 
und den König Karl zurück zu bringen. Das war alles ganz
falsch und eine Lüge, die er selber erfunden, wie ihm auch bewiesen 
wurde. Ich harte nie so etwas zu ihm gesagt; ich hatte
mich nie an einer Verschwörung beteiligt; ich hatte nie einen Eid
geschworen, nie Kriegsübungen gemacht. Wie jenes falsche 
Anschuldigungen gewesen, so sind es jetzt die, die Major Porter 
vorgebracht [...] Ich bin kein Störer des Landfriedens, sondern
ich suche den Frieden aller Menschen [...] Und ebenso ist es
falsch, wenn Major Porter sagt, ich sei ein Feind des Königs,
denn ich liebe ihn und alle Menschen, wenn sie schon Feinde
Gottes sind und ihre eigenen Feinde und meine. Ich weiß, das
seine Rückkehr vom Herrn kommt, damit er viel begangenes Unrecht 
wieder gut mache. Ich hatte ein Gesicht davon, drei Jahre
ehe er zurück kam. Es ist eigentümlich, zu sagen, ich sei ein
Feind des Königs; ich habe keinerlei Grund es zu sein, trotzdem
ich allerdings viel verfolgt und eingesperrt gewesen bin während
der letzten 11 oder 12 Jahre, von den Gegnern sowohl des
jetzigen Königs als seines Vaters, also eben von der Partei, die
Porter zum Major gemacht und für die er die Waffen führte,
aber nicht durch die, die für den König war. Ich war nie ein
Feind des Königs, noch irgend eines andern auf Erden. Ich
habe die Liebe, die des Gesetzes Erfüllung ist, die nichts Böses
denkt, sondern sogar die Feinde liebt, und möchte, das der König
errettet würde und die Wahrheit erkennete und dazu käme, Gott
zu fürchten und die Weisheit von oben zu erlangen, durch die
alle Dinge gemacht sind, damit er in dieser Weisheit regierete zur
Ehre Gottes [...].

Weil ich nun hier gefangen bin, bis ein Befehl vom König
oder dem Parlament mich frei macht, so habe ich solches geschrieben, 
damit ihr und der König und das Parlament es
leset, und alles bedenket, ehe ihr etwas in der Sache tut; und
in der Weisheit Gottes untersucht, was für Absichten zugrunde 
liegen, damit ihr nicht etwas tut, womit ihr die Hand
des Herrn gegen euch wendet, wie viele Machthaber zuvor getan,
die dann gestürzt wurden von dem Gotte, den wir fürchten und
dem wir trauen und zu dem wir Tag und Nacht schreien, und
der uns gehöret hat und noch erhört und uns rächen wird. Viel
unschuldig Blut ist schon vergossen worden, und viele sind bis in
den Tod verfolgt worden durch die, die vor euch die Herrschaft
% \picinclude{./150-159/p_s151.jpg} 
hatten; und Gott hat sie ausgespien, weil sie sich gegen das
Recht kehrten. Darum prüfet, wie es um euch steht, solange eß
Tag ist, und nehmet dieseö auf als eine Warnung in Liebe
an euch.«
Von einem der unschuldig in Lancaster gefangen liegt, genannt
George Fox .... .
Bald darauf gab ich eine Schrift gegen daß Verfolgen herauß:
,,Die Papisten, die Common-Prayerleute, die Preöbyterianer,
Jndependenten und Baptisten verfolgen einander um ihrer eigenen
Erfindungen willen, ihren Messen, ihren Common-Prayer Vüch em,
ihrem »Directory« und Bekenntniö, dies sie aufgesetzt haben,
aber nicht zum Nutzen der Wahrheit; denn sie wissen nicht, weß
Geisteö Kind sie sind, wenn sie verfolgen und die Leben der Menschen
zu zerstören suchen um de?. Kirchendiensteöz und der Religion
willen, während Ehristuß sagte, er sei nicht gekommen, daS
Leben der Menschen zu zerstören, sondern etz zu retten (Luc. 9).
Wir können unß doch nicht solchen anvertrauen, die nicht wissen,
weß Geisteß Kind sie sind ..... Jhr möchtet gerne ein Gebot
haben, um zu zerstören, wie einst die Jünger wollten Feuer vom
Himmel regnen lassen, um die, welche Ehristuß nicht aufnehmen
wollten, zu zerstören ..... i Die, welche daß Leben der Menschen
zerstören, sind nicht Jünger Christi, des Heilandß, E . . wenn ihr
die Leben anderer zerstört und versolget und nicht Buße tut, werdet
ihr nicht auferstehen zum Leben mit Gott. Die aber, die wissen
weß Geisteß Kinder sie sind, die haben den untadeligen Eifer und
geben durch den Geist Gottes dem Herrn Leib, Seele und Geist,
die sein sind, daß er sie bewahre« ..... G. F.
ES trieb mich auch, an den König zu schreiben, um ihn zu
ermahnen, Barmherzigkeit zu üben gegen seine Feinde und der
Zügellosigkeit und Gottlosiigkeit, die bei seiner Rückkehr im Lande
aufgekommen war, zu steuern.
An den König:
,,O König Karl,
Du kamst nicht inß Land durch Schwert noch durch Sieg im
Kriege, sonden durch die Kraft dez Herrn; wenn du nun nicht
in derselben lebest, so wirst du nicht gedeihen. Wenn der Herr
dir Barmherzigkeit erzeigi hat und dir vergeben hat, und du
iibest nun nicht auch Barmherzigkeit und Vergebung, so wird der


% \picinclude{./150-159/p_s152.jpg} 
152 Kapitel 1111.
Herr deine Gebete nicht erhören, noch die Gebete derer, die für
dich beten. Wenn du nicht den Verfolgungen Einhalt gebietest
und nicht die Gesetze, welche das Verfolgen um des Glaubens
willen gestatten, abschasfst .... so wirst du so blind werden
wie deine Vorgänger; denn das Verfolgen hat immer die Ver-
folger blind gemacht. Solche aber stiirzet Gott durch seine Macht
nnd oersährt streng mit ihnen; aber den Bedrückten schickt er
Rettung. Wenn du das Schwertlumsonst trägst und lässest
Trunkenheit, Schwören, Spielen und dergleichen eitles Treiben
ungestrast, wie z. V. das Aufstellen der Maibäume mit dem Bild
der Krone oben draus und dergleichen, so wird das Land bald
sein wie Sodom und Goniorra und so schlecht, wie die alte Welt,
die den Herrn so betrübte, daß er sie untergehen ließ. So wird
er auch euch tun, wenn ihr solche Dinge nicht abschafft. Es hat
kaum je solche Freiheit, Unrecht zu tun, geherrscht wie jetzt, als
ob es nicht Gewalt noch Schwert der Obrigkeit mehr gäbe; und
solches ist weder der Regierung noch denen, die recht tun, zum
Nutzen. Wir beten für die, welche die Herrschaft haben, daß
wir ein ruhiges Leben unter ihnen führen können, in Frieden
und Gottseligkeit, und daß wir nicht durch sie in Gottlosigkeit
fallen. Höre und denke darüber nach und tue Gutes, so lange
du kannst und Macht hast. Sei barmherzig und oergib; dies
ist der Weg, aus dem du überwindest und das Reich Christi
erlangst.« G. F.
Gs ging lange, ehe der Scheriff einwilligte, mich nach London
iiberzuführen, es sei denn, daß ich die Kosten trage, was ich ver-
weigerte. Schließlich, als sie sahen, daß es nicht anders ging,
gab der Scheriff zu, daß ich mit einigen Freunden nach London
gehe, ohne andere Verpflichtungen als mein Versprechen, an
dem und dem Tage vor den Richtern in London zu erscheinen,
so der Herr es zulasse, woraus ich entlassen wurde .....
Etwa drei Wochen nach meiner Freilassung gelangte ich nach
London. Als swir nach Charing Eroß kamen, Zwar dort eine
ungeheure Menschenmenge versammelt, um zu sehen, wie die
Überreste einiger der früheren Richter des Königs verbrannt
wurden, die erhenkt, ertränkt und geoierteilt worden waren ....
Als wir den Richtern die gegen uns gerichtete Anklage ein-
gereicht hatten, und sie die Worte lasen: meine Freunde und ich
trachteten Blutvergießen im Lande anzurichten, schlugen sie mit


% \picinclude{./150-159/p_s153.jpg} 
Ein Gottesgericht. Eruiahnung zur Barmherzigkeit usw. 153
der Hand aus den Tisch; ich erklärte, daß ich der Mann sei,
gegen den diese Anklage gehe, aber ich sei an allem derartigen
so unschuldig, wie ein neugeboreneß Kind, und habe die Anklage
selber hierher gebracht, und meine Freunde seien ohne Wache mit
mir gekommen. Sie hatten biß jetzt meinen Hut noch nicht
beachtet, aber jetzt fiel er ihnen aus und sie fragten: ,,WaZ! ihr
steht hier im Hut'-J« Jch sagte ihnen, etz geschehe keineßnoegs
aus Mangel an Achtung vor ihnen. Darauf befahlen sie, daß
man mir ihn abnehme, und dann riefen sie den Marschall von
Kingß-Bench und sagten zu ihm: .,Jhr müßt diesen Mann in
Gewahrsam bringen, gebt ihm aber ein Zimmer und legt ihn nicht
unter die anderen Gefangenen.« Alß der Marschall erklärte, er
habe kein Zimmer frei, das er mir geben könnte, so fragten sie
mich: ,,Wollt ihr morgen um zehn in Westminsterhall vor Kings-
Bench erscheinen?« Jch antwortete: ,,Ja, wenn der Herr mir die
Kraft gibt.« Hierauf sagte Richter Foster zum anderen Richter:
,,Wenn er sagt ja und etz verspricht, so könnet ihr auf sein Wort
gehen.« Somit war ich entlassen. Am nächsten Tage erschien
ich zur bestimmten Stunde vor Kingß-Bench . . . . und alö ich
eintrat, trieb etz mich zu sagen: ,,Friede sei mit euch,« und die
Krait des Herrn kam über alle. Meine Anklage wurde öffentlich
verlesen. Die Leute waren ziemlich still und die Richter ruhig
und freundlich, die Gnade deß Herrn war mit ihnen. Aber alß
sie an die Stelle kamen, wo eß hieß, ich und meine Freunde
wollten Blutvergießen über daß Land bringen und einen neuen
Krieg anstiften, und daß ich ein Feind der Königz sei, da hoben
ste ihre Hände auf. Da erhob ich meinen Arm und sagte: ,,Jch
bin der Mann, gegen den sich diese Anklage richtet, aber ich bin
so unschuldig wie ein neugeborenes Kind in dieser Sache, und
habe nie den Gebrauch der Waffen gelernt. Und meint ihr, wenn
ich oder meine Freunde solche wiren, wie eß in der Anklage
heißt, so hätten wir unsere Anklage selber hierher gebracht?« ....
Sie fragten mich, maß man mit der Anklage tun solle? ich sagte:
»Jhr seid die Richter und könnt hoffentlich in dieser Sache
richten; tut also waö ihr wollt. ....
Sie sagten, sie wollten mich nicht verurteilen, denn sie hätten
nichts gegen mich. ES-quite Mcttsh 1) erhob sich und sagte, ez sei
1) Esquire Matjh, eine angesehenesszetsönlichkeii am Hofe Karl 11, war den
Ouäkern geneigt und bemühte sich ost sürsie und schühtc sie vor Verfolgungen.


% \picinclude{./150-159/p_s154.jpg} 

154 Kapitel R17.
des- Königz Wohlgefallen, daß ich freigesprochen werde, wenn sich
kein Kläger gegen mich erhebe. Sie fragten inich, ob ich eß dem
König und dem Rat überlassen wolle? Jch sagte: ,,Ja, gerne.«
Hierauf schickten sie des- Scherisss Bericht, der die Anklage enthielt,
dem König, damit er sehe, wessen man mich beschuldigte .....
Der König, nachdem er dietz gelesen und von der ganzen
Angelegenheit unterrichtet war, war von meiner Unschuld über-
zeugt und sandte einen Befehl, mich frei zu lassen, welcher lautete:
,,ES beliebt Seiner Majestät, zu befehlen, daß man dem Manne
George Fox, bißlang Gefangener im Kerker von Lancaster, die
volle Freiheit schenke ..... Und diese Kundgebung von Seiner
Majestät Belieben soll euch alß Befehl geniigen.«
Whitehall, 24. Oktober, 1660. Edward Nicholat-.
A13 ich nun mehr als zwanzig Wochen gefangen gewesen
war, war ich auf Befehl dez Königß rechtmäßig frei geworden;
die Macht deß Herrn hatte meine Unschuld herrlich kund getan.
Porter wagte nicht, die Anklage, die er fälschlich gegen mich
erhoben hatte, öffentlich zu berichtigen .....


\chapter[Beginn neuer Quälerverfolgungen]{Beginn neuer Quälerverfolgungen}

\begin{center}
\textbf{Beginn neuer Quälerverfolgungen bei Anlaß der Verschwörungen
der Fifthmonarchy-Leute. Des Quäkers John Perots Verirrnngen.
Qnäker misshandelt in Neu-England und Malta.}
\end{center}

% \section[Beginn neuer Quälerverfolgungen]{Beginn neuer 
% Quälerverfolgungen bei Anlaß der Verschwörungen
% der Fifthmonarchy-Leute. Des Quäkers John Perots Verirrnngen.
% Qnäker misshandelt in Neu-England und Malta.}



Jch sah nun, warum ich durch so schwere Prüfungen hatte
gehen müssen in Reading, denn die ewige Kraft detz Herrn war
über alle gekommen, und sein gesegneteß Leben und Licht und
seine Wahrheit war dem Lande ausgegangen; wir hatten herr-
liche Versammlungen, und oiele scharten sich um die Wahrheit.
Richard Hubberihorn war beim König gewesen, und dieser hatte
gesagt, eß dürfe unö niemand behelligen, so lange wir friedlich
leben; er versprach uns dietz bei seinem königlichen Wort mit der
Ermahnung, sein Versprechen nicht zu mißbrauchen. Einige
Freunde erhielten auch Zutritt im House of Lordö, und man
gestattete ihnen, ihre Gründe gegen daß Zehntenwesen, daß Schwören,
die Turmhäuser, die Gotteßdienste und anderes auseinanderzu-
setzen, und man hörte ihnen ziemlich lange zu. Etwa 700 Freunde,
die unter Richards und Olioerz Regierung in die verschiedenen


% \picinclude{./150-159/p_s155.jpg} 

Beginn neuer Qnäkerverfolgnugete bei Anlaß der Verschwörangen usw. 155
Gefängniss e des- Landetz- gebracht worden waren, wegen Verstößen,
wie sie etz nannten, setzte der König, alß er kam, in Freiheit.
Man spürte, daß die Regierung geneigt war, den Freunden ihre
Freiheit zu sichern, weil sie sah, daß wir unter der vorherigen
Herrschaft so gut wie sie gelitten hatten. Sobald aber wirklich
etwas für unz geschehen sollte, so wurde es- wieder oereitelt durch
irgend einen Schust, der dergleichen tat, alß ob er unö wohl
wolle. ES hieß, ez sei schon ein Befehl ausgesetzt, der unsere
Freiheit bestätige, er müsse nur noch unterzeichnet werden; da
brach gerade jeneß häßliche Attentat der Fisthmonarchy-Leute
auß und brachte die Hauptstadt und das ganze Land in Aufruhr.
ES geschah am Abend eineß Ersten Tage?-, und wir hatten eine
besonderß herrliche Versammlung gehabt, in der die Wahrheit
des Herrn allen erschienen war. Da, bald nach Mitternacht wurde
die Trommel geschlagen und erklang der Ruf: ,-zu den Waffen,
zu den Waffen!« Jch stand aus und nahm am folgenden Morgen
ein Boot und fuhr nach Whitehall, und stieg dort auö und schritt
durchß Schloß. Sie betrachteten mich erstaunt, aber ich schritt
durch sie hindurch biz nach Pall-Mall, wo sich etliche Freunde
zu mir gesellten, obgleich eö jetzt gefährlich war, über die Straßen
zu gehen; denn schon waren die Stadt und die Vorstädte unter
Waffen, und daß Volk und die Soldaten waren sehr roh; sie miß-
handelten Henry Fell, der zu einem Freunde gehen wollte, und
hätten ihn getötet, wenn nicht der Herzog von York dazu gekommen
wäre. EZ geschah viel Unheil während dieser Woche, und am
nächsten Grsten Tage wurden oiele Freunde auf dem Weg in die
Versammlung gefangen genommen.
Jch blieb in Pall-Mall, weil ich dort der Versammlung bei-
wohnen wollte; doch in der Nacht dez Siebenten Tages kamen
Soldaten und klopften an die Tür. Da die Mägde sie einließen,
so stürzten sie herauf und ergriffen mich. Und einer von ihnen,
der beim Parlament gedient, fiihlte mir in die Tasche und fragte,
ob ich keine Pistolen bei mir habe. Jch fragte ihn, warum er
auch solche Frage an mich stelle; er wisse ja, daß sich ein sried-
licher Mann sei ..... Diese Soldaten nahmen mich mit und
brachten mich nach Whitehall ..... Dort waren die Soldaten
und daß Volk sehr wild; aber ich predigte doch die Wahrheit
unter ihnen. Einige Große aber, als sie daß hörten, sagten:
»WaS, ihr laßt ihn noch predigen? Bringt ihn doch an einen


% \picinclude{./150-159/p_s156.jpg} 
156 Kapitel 117.
Ort, wo er nicht mehr hetzen kann.« Da-Z taten sie denn auch
und bewachten mich. Ich sagte ihnen, wenn sie schon meinen
Leib binden und einsperren können, so können sie doch das Wort
des Lebens nicht aufhalten. Einige kamen und fragten mich,
was ich sei? Ich erwiderte ihnen: »ein Prediger der Gerechtig-
keit« (2. Petr. 2, 5). Nachdem ich etwa drei Stunden eingesperrt
gewesen war, ging EK-quire Marsch zum Lord Gerrard, und darauf
wurde ich frei gelassen .....
Während dieseö Aufstandeö der Fifthmonarchy-Leute 1660,
fanden arge Metzeleien statt, sowohl auf dem Lande alö in der
Stadt, so daß es für anständige Leute noch lange gefährlich war,
außzugehen. Man konnte kaum ohne Gefahr Einkäufe machen.
Auf dem Lande schleppten sie die Leute, Männer und Frauen,
auß den Häusern und Kranke rissen sie auß ihren Betten. Ja
einen Fieberkranken rissen sie auß dem Bett nnd schleppten ihn
inß Gefängni?-; alß er dort ankam, starb er, er hieß Thomaß
Pachyn
Margaret Fell ging zum König und berichtete ihm, wie ez
zugehe zu Stadt und Land. Sie setzte ihm auseinander, daß wir
harmlose, friedliche Leute seien; daß wir aber unsere Versamm-
lungen auch sernerhin halten würden, was immer wir auch zu
dulden haben würden; aber ez sei seine Pflicht für Frieden zu
sorgen, damit nicht noch mehr unschuldig Blut vergossen werde.
Die Gefängnisse waren nachgerade überall angefüllt mit
Freunden und andern aus der Stadt und vom Lande; überall
waren Wachen ausgestellt zur Durchsuchung der Briefe, so daß
niemand passieren konnte, ohne untersucht zu werden. Wir hörten
von vielen Tausenden von Freunden, die im Lande herum in den
Gefängnissen waren, und Margaret Fell überbrachte dem König
und dem Rat einen Bericht darüber. Alö wir in der darauf-
folgenden Woche von einigen weiteren Tausenden hörien, die ge-
fangen genommen worden, ging sie abermals hin, um etz dem König
und dem Rat mitzuteilen. Man oerwunderte sich, woher wir
diese Nachrichten hätten, da ein strenger Befehl ergangen war,
alle Briefe aufzufangen. Aber der Herr fügte ez, daß wir Kunde
erhielten trotz allen ihren Hindernissen.
Wir ließen eine Erklärung gegen daß Bekriegen und daö
Vekschwören drucken und schickten einige Abzüge an den König
und den Rat; andere wurden in den Straßen verkauft ....


% \picinclude{./150-159/p_s157.jpg} 

Beginn neuer Quäketvetfolgungen bei Anlaß der Vetschwörungen usw. 157
Diese Grklärung klärte etwaö die Luft, die aus Stadt und
Land lastete; und der König erließ bald darauf einen Befehl,
daß die Soldaten keine Hau?-suchungen ohne einen Konstabler
vornehmen sollten; aber noch immer waren die Gefängnisse gefüllt
und viele Freunde gefangen, woran namentlich der Aufstand der
Fifthmonarchy=Leute Schuld war. Alö aber die Gefangenen
sollten hingerichtet werden, ließ man ihnen doch Gerechtigkeit wider-
fahren und erklärte uns öffentlich frei von jeglicher Teilnahme an
den Verschwörungen. Und auf wiederholteö Drängen erließ der
König den Befehl, die Freunde frei zu lassen ohne Loökaufung.
Aber es- hatte viel Mühe und Arbeit gebraucht, um das zu
erreichen. Thomas Moor und Margaret Fell waren oft de?-wegen
beim König gewesen.
GZ wurde während dieses Jahreö viel Blut vergossen, denn
viele von den Räten des früheren Königs wurden gehenkt, ertränkt
oder geoierteilt. Unter diesen war auch Oberst Hacker, der mich
unter Oliver Eromwell von Leicester nach London als Gefangener
schickte, wie oben berichtet worden. GS war ein trauriger Tag
der Vergeltung von Blut durch Blut ....
GZ war eine unsichtbare Hand, die diesen Tag über daß
heuchlerische Geschlecht gebracht hatte, daß, kaum war es zur
Herrschaft gelangt, so hochmiitig und über alle Maßen grausam
geworden war und da-3 Volk Gottes verfolgt hatte.
Mehr als einmal waren diese ,,Frommen« gewarnt worden
durch Worte, Schrift und Zeichen; aber sie wollten ez nicht
glauben, biz es zu spät war. Willam Symps on war öfter während
drei Jahren getrieben worden, unbekleidet und barfuß unter sie
zu treten in den Städten, Märkten und Ortschaften, vor Priester
und Große, um ihnen zu sagen: ,,so nackt wie er würden auch sie
einhergehen.« Und manchmal trieb ez ihn, sich mit einem Sack
anzutun und sein Gesicht zu beschmieren und ihnen zu sagen:
,,also werde der Herr ihnen ihre Frömmigkeit besudeln, wie er
besudelt sei.« Der arme Mensch hatte schwere Leiden erduldet,
sich mit Pserdepeitschen auf feinen bloßen Leib peitschen, sich mit
Steinen bewersen und einsperren lassen während dieser Jahre, vor
der Rückkehr des Königö; aber sie wollten sich nicht warnen
lassen, sie erwiderten seine Liebe mit Grausamkeit. Nur der
Bürgermeister von Cambridge behandelte ihn großmütig, indem
er seinen Rock um ihn legte und ihn in sein Haus nahm.


% \picinclude{./150-159/p_s158.jpg} 
158 Kapitel :117.
Ein anderer Freund, Robert Huntingdon, wurde getrieben
ins Turmhaus zu Earlisle zu gehen, unter die Haupt-Presbty
terianer und Jndependeten dort, in einem weißen Hemd als
Zeichen, daß das Chorhemd wieder aufkommen werde, und mit
einem Halfter, um zu zeigen, daß auch siir sie ein Halfter kommen
werde, was sich auch an einigen von den Versolgern erfüllte.
Zu einem andern, Richard Sale, der Konstabler in der Nähe
von Chester war, wurde ein Freund mit einem Paß geschickt;
die schändlichen ,,Frommen« hatten ihn als Vagabunden festge-
nommen, roeil er als Prediger reiste. Dieser Konstabler wurde
durch den ihm zugeschickten Freund bekehrt und gab ihm die
Freiheit; später wurde er selber ins Gefängnis geworfen. Darnach
an einem Ersten Tage trieb es Richard Sale ins Turmhaus zu
gehen und den versolgungssiichtigen Priestern und ihren Genossen
eine Kerze zu bringen als Anspielung auf ihre Finsternis; aber
sie mißhandelten ihn, und, so recht wie verstockte »Fromme«, warfen
sie ihnsins Gefängnis von Little-Gase, und quälten ihn dortdermaßen,
daß er bald darauf starb. Die Freunde wurden mehrfach ge-
trieben, dieses Geschlecht auf allerlei Art zu mahnen, aber nicht
nur hörte man sie nicht, sondern sie wurden noch mißhandelt und
»oerschrobene O.uäker« genannt! Aber Gott schickte sein Gericht
über die versolgungssüchtigen Priester und Behörden; als der
König zurttckkam, wurden den meisten von ihnen ihre Stellen und
Einkünfte entzogen; die Räuber wurden beraubt und es war nun
an uns zu sragen: ,,wo sind jetzt die Verschrobenen?« Viele gaben
jetzt zu, daß wir wahre Propheten seien, und behaupteten, wenn
wir nur gegen einzelne Priester geeifert hätten, so hätten sie sich
sogar über uns gefreut; weil wir aber gegen alle geeisert hatten,
so haben sie sich über uns geärgert. Aber sie sahen es jetzt ein,
daß die Priester, die damals für die besten gehalten wurden, so
schlecht waren wie die übrigen. Ja, viele, die als die allerher-
norragendsten gegolten hatten, hetzten die Behörden am aller-
meisten zu den Versolgungen aus. Diese wurden aber, als der
König zuriickkam, damit bestraft, daß ihnen die Gewissenssreiheit
entzogen wurde, die sie vorher, als sie die Oberhand gehabt hatten,
den andern nicht gegönnt hatten. Einer, namens Hewes, von
Plymouth, ein angesehener Priester zu Olioers Zeitz hatte immer,
wenn irgendwo Gewifsensfreiheit zugestanden wurde, gebetet, Gott
wolle es den Behörden ins Herz geben, daß sie diese verdammte


% \picinclude{./150-159/p_s159.jpg} 
Beginn neuer Quäterberfolgmigen bei Anlaß der Versehwörungen usw. 159
Toleranz abschaffen. Und andere beteten gegen ,,die nicht zu
duldende Duldsamkeit-«. Alf- nun nach deß Königß Rückkehr
diesem Priester Heweö seine großen Einkünfte entzogen wurden,
weil er sich nicht dem Common-Prayer Buch unterwerfen wollte,
fragte ihn ein Freund, alß er ihm in Plymouth begegnete: ob er
nun die Duldsanikeit noch verdammungswürdig finde und nicht
vielmehr froh darüber wäre? worauf der Priester nicht antwortete
und nur daß Gesicht abwandte. Aber so hartnäckig auch diese
Leute früher gegen Duldsamkeit geeifert hatten, — jetzt kamen
viele von ihnen selber beim König um einen Ort, wo sie ihre
Versammlungen halten könnten, ein, und bezahlten sogar, damit
es ihnen bewilligt werde .....
Wir erhielten die Kunde, daß ein Freund, der getrieben worden
war, gegen den Götzendienst der Papisten zu predigen, in Rom
im Gefängnis gestorben war, und man hatte den Verdacht, daß
er heimlich im Gefängnis umgebracht worden war. John Perrot
war auch dort gefangen gewesen und kam nach seiner Freilassung
zu uns zurück; später aber wandte er und andere sich ab von
der Gemeinschaft der Freunde und der Wahrheit .....
Ungefähr um die gleiche Zeit 1661 erhielten wir die Nach-
richt aus; Neu-England, daß die dortige Regierung ein Gesetz
erlassen hatte, daß die Quäker auö jenen Kolonien uerbannte bei
Todesstrafe im Fall der Rückkehr, und daß mehrere Freunde, die
nach ihrer Verbannung zurückgekehrt waren, wirklich gehängt
worden waren 1), und andere vom gleichen Schicksal bedroht im
Gefängnis seien. Während jene hingerichtet wurden, war ich im
Gefängnis zu Lancaster gewesen und hatte eine deutliche Wahr-
nehmung ihrer Leiden, alß ob sie mich selber betroffen hätten,
und der Strick um meinen eigenen Halß gelegt würde, und wir
hatten doch damalß noch nichttz davon gehört. Sobald wir nun
davon erfuhren, ging Edward Burrough zum König und sagte
ihm, es sei in seinem Reich eine Ader offen, au-S3 der unschuldigeö
Blut fließe, daß, wenn es nicht gestillt werde, alles überschwemmen
werde. Der König erwiderte: »Jch werde diese-3 Blut stillen.«
1) 1661 verfolgten die Puritaner und Jndependeuten, die selber nach
Amerika geflohen, um Religionsfreiheit zu haben, die Quäler aufs Gruusamste.
Ein Edirlt von 1658 bestimmte, dvß jeder Quäker, der zum driltenmal in den
Kolonien gefunden werde, gehängt werde, und dieser Befehl wurde an zwei
Quäketn und einer Quükerin ausgeführt.

