  % Erster Durchlauf
  % \picinclude{./150-159/p_s150.jpg} 
  und den König Karl\person{König Karl} zurück zu bringen. Das war alles ganz
  falsch und eine Lüge, die er selber erfunden, wie ihm auch bewiesen 
  wurde. Ich harte nie so etwas zu ihm gesagt; ich hatte
  mich nie an einer Verschwörung beteiligt; ich hatte nie einen Eid
  geschworen, nie Kriegsübungen\indexname{Kriegsübungen} gemacht. Wie jenes falsche 
  Anschuldigungen gewesen, so sind es jetzt die, die Major Porter 
  vorgebracht [...] Ich bin kein Störer des Landfriedens, sondern
  ich suche den Frieden aller Menschen [...] Und ebenso ist es
  falsch, wenn Major Porter\person{Major Porter} sagt, ich sei ein Feind des Königs,
  denn ich liebe ihn und alle Menschen, wenn sie schon Feinde
  Gottes sind und ihre eigenen Feinde und meine. Ich weiß, das
  seine Rückkehr vom Herrn kommt, damit er viel begangenes Unrecht 
  wieder gut mache. Ich hatte ein Gesicht davon, drei Jahre
  ehe er zurück kam. Es ist eigentümlich, zu sagen, ich sei ein
  Feind des Königs; ich habe keinerlei Grund es zu sein, trotzdem
  ich allerdings viel verfolgt und eingesperrt gewesen bin während
  der letzten 11 oder 12 Jahre, von den Gegnern sowohl des
  jetzigen Königs als seines Vaters, also eben von der Partei, die
  Porter zum Major gemacht und für die er die Waffen führte,
  aber nicht durch die, die für den König war. Ich war nie ein
  Feind des Königs, noch irgend eines andern auf Erden. Ich
  habe die Liebe, die des Gesetzes Erfüllung ist, die nichts Böses
  denkt, sondern sogar die Feinde liebt, und möchte, das der König
  errettet würde und die Wahrheit erkennete und dazu käme, Gott
  zu fürchten und die Weisheit von oben zu erlangen, durch die
  alle Dinge gemacht sind, damit er in dieser Weisheit regierete zur
  Ehre Gottes [...].

  \medskip 

  Weil ich nun hier gefangen bin, bis ein Befehl vom König
  oder dem Parlament mich frei macht, so habe ich solches geschrieben, 
  damit ihr und der König und das Parlament es
  leset, und alles bedenket, ehe ihr etwas in der Sache tut; und
  in der Weisheit Gottes untersucht, was für Absichten zugrunde 
  liegen, damit ihr nicht etwas tut, womit ihr die Hand
  des Herrn gegen euch wendet, wie viele Machthaber zuvor getan,
  die dann gestürzt wurden von dem Gotte, den wir fürchten und
  dem wir trauen und zu dem wir Tag und Nacht schreien, und
  der uns gehöret hat und noch erhört und uns rächen wird. Viel
  unschuldig Blut ist schon vergossen worden, und viele sind bis in
  den Tod verfolgt worden durch die, die vor euch die Herrschaft
  % \picinclude{./150-159/p_s151.jpg} 
  hatten; und Gott hat sie ausgespien, weil sie sich gegen das
  Recht kehrten. Darum prüfet, wie es um euch steht, solange es
  Tag ist, und nehmet dieses auf als eine Warnung in Liebe
  an euch.\grqq{}

  \medskip 

  Von einem der unschuldig in Lancaster gefangen liegt, genannt
  \begin{center}George Fox [...]\end{center}

\end{quote} \bigskip

Bald darauf gab ich eine Schrift gegen das Verfolgen heraus:

\grosszitat{Verfolger}{
  Die Papisten\index{Papisten}, die 
  Common-Prayerleute\index{Common-Prayerleute}, die 
  Presbyterianer\index{Presbyterianer}, Independenten\index{Independenten} 
  und Baptisten\index{Baptisten} verfolgen einander um ihrer eigenen
  Erfindungen willen, ihren Messen, ihren Common-Prayer Bücherm,
  ihrem \zitat{Directory} und Bekenntnis, dies sie aufgesetzt haben,
  aber nicht zum Nutzen der Wahrheit; denn sie wissen nicht, wes
  Geistes Kind sie sind, wenn sie verfolgen und die Leben der Menschen
  zu zerstören suchen um des Kirchendienstes und der Religion
  willen, während Christus sagte, er sei nicht gekommen, das
  Leben der Menschen zu zerstören, sondern es zu retten (Luc. 9\bibel{Luc. 09@Luc. 9}).
  Wir können uns doch nicht solchen anvertrauen, die nicht wissen,
  wes Geistes Kind sie sind [...] Ihr möchtet gerne ein Gebot
  haben, um zu zerstören, wie einst die Jünger wollten Feuer vom
  Himmel regnen lassen, um die, welche Christus nicht aufnehmen
  wollten, zu zerstören [...] Die, welche das Leben der Menschen
  zerstören, sind nicht Jünger Christi, des Heilands, [...] wenn ihr
  die Leben anderer zerstört und verfolget und nicht Buße tut, werdet
  ihr nicht auferstehen zum Leben mit Gott. Die aber, die wissen
  wes Geistes Kinder sie sind, die haben den untadeligen Eifer und
  geben durch den Geist Gottes dem Herrn Leib, Seele und Geist,
  die sein sind, das er sie bewahre. [...]. 

  \bigskip 

  \begin{flushright}G. F.\end{flushright}

}

Es trieb mich auch, an den König zu schreiben, um ihn zu
ermahnen, Barmherzigkeit zu üben gegen seine Feinde und der
Zügellosigkeit und Gottlosigkeit, die bei seiner Rückkehr im Lande
aufgekommen war, zu steuern.

\grosszitat{König Karl}{
  \begin{center}An den König:\end{center}
  \medskip 

  O König Karl,
  \medskip 

  Du kamst nicht ins Land durch Schwert noch durch Sieg im
  Kriege, sondern durch die Kraft des Herrn; wenn du nun nicht
  in derselben lebest, so wirst du nicht gedeihen. Wenn der Herr
  dir Barmherzigkeit erzeigt hat und dir vergeben hat, und du
  übest nun nicht auch Barmherzigkeit und Vergebung, so wird der
  % \picinclude{./150-159/p_s152.jpg} 
  Herr deine Gebete nicht erhören, noch die Gebete derer, die für
  dich beten. Wenn du nicht den Verfolgungen Einhalt gebietest
  und nicht die Gesetze, welche das Verfolgen um des Glaubens
  willen gestatten, abschaffst [...] so wirst du so blind werden
  wie deine Vorgänger; denn das Verfolgen hat immer die Verfolger 
  blind gemacht. Solche aber stürzet Gott durch seine Macht
  und verfährt streng mit ihnen; aber den Bedrückten schickt er
  Rettung. Wenn du das Schwert umsonst trägst und lässest
  Trunkenheit, Schwören, Spielen und dergleichen eitles Treiben
  ungestraft, wie z.B. das Aufstellen der Maibäume mit dem Bild
  der Krone oben drauf und dergleichen, so wird das Land bald
  sein wie Sodom und Gomorra und so schlecht, wie die alte Welt,
  die den Herrn so betrübte, das er sie untergehen ließ. So wird
  er auch euch tun, wenn ihr solche Dinge nicht abschafft. Es hat
  kaum je solche Freiheit, Unrecht zu tun, geherrscht wie jetzt, als
  ob es nicht Gewalt noch Schwert der Obrigkeit mehr gäbe; und
  solches ist weder der Regierung noch denen, die recht tun, zum
  Nutzen. Wir beten für die, welche die Herrschaft haben, das
  wir ein ruhiges Leben unter ihnen führen können, in Frieden
  und Gottseligkeit, und das wir nicht durch sie in Gottlosigkeit
  fallen. Höre und denke darüber nach und tue Gutes, so lange
  du kannst und Macht hast. Sei barmherzig und vergib; dies
  ist der Weg, aus dem du überwindest und das Reich Christi
  erlangst.

  \medskip 

  \begin{flushright}G. F.\end{flushright}
}

Es ging lange, ehe der Sheriff einwilligte, mich nach London\ort{London}
überzuführen, es sei denn, das ich die Kosten trage, was ich 
verweigerte. Schließlich, als sie sahen, das es nicht anders ging,
gab der Scheriff zu, das ich mit einigen Freunden nach London
gehe, ohne andere Verpflichtungen als mein Versprechen, an
dem und dem Tage vor den Richtern in London zu erscheinen,
so der Herr es zulasse, woraus ich entlassen wurde [...]
Etwa drei Wochen nach meiner Freilassung gelangte ich nach
London. Als wir nach Charing Cros\ort{Charing Cros} kamen, war dort eine
ungeheure Menschenmenge versammelt, um zu sehen, wie die
Überreste einiger der früheren Richter des Königs verbrannt
wurden, die erhängt, ertränkt und gevierteilt worden waren [...].

Als wir den Richtern die gegen uns gerichtete Anklage eingereicht 
hatten, und sie die Worte lasen: meine Freunde und ich
trachteten Blutvergießen im Lande anzurichten, schlugen sie mit
% \picinclude{./150-159/p_s153.jpg} 
der Hand auf den Tisch; ich erklärte, das ich der Mann sei,
gegen den diese Anklage gehe, aber ich sei an allem derartigen
so unschuldig, wie ein neugeborenes Kind, und habe die Anklage
selber hierher gebracht, und meine Freunde seien ohne Wache mit
mir gekommen. Sie hatten bis jetzt meinen Hut noch nicht
beachtet, aber jetzt fiel er ihnen auf und sie fragten: \zitat{Was! ihr
steht hier im Hut?} Ich sagte ihnen, es geschehe keineswegs
aus Mangel an Achtung vor ihnen. Darauf befahlen sie, das
man mir ihn abnehme, und dann riefen sie den Marschall von
Kings-Bench und sagten zu ihm: \zitat{Ihr müsst diesen Mann in
Gewahrsam bringen, gebt ihm aber ein Zimmer und legt ihn nicht
unter die anderen Gefangenen.} Als der Marschall erklärte, er
habe kein Zimmer frei, das er mir geben könnte, so fragten sie
mich: \zitat{Wollt ihr morgen um zehn in Westminsterhall vor 
Kings-Bench erscheinen?} Ich antwortete: \zitat{Ja, wenn der Herr mir die
Kraft gibt.} Hierauf sagte Richter Foster\person{Richter Foster} zum anderen Richter:
\zitat{Wenn er sagt ja und es verspricht, so könnet ihr auf sein Wort
gehen.} Somit war ich entlassen. Am nächsten Tage erschien
ich zur bestimmten Stunde vor Kings-Bench\ort{Kings-Bench} [...] und als ich
eintrat, trieb es mich zu sagen: \zitat{Friede sei mit euch,} und die
Kraft des Herrn kam über alle. Meine Anklage wurde öffentlich
verlesen. Die Leute waren ziemlich still und die Richter ruhig
und freundlich, die Gnade des Herrn war mit ihnen. Aber als
sie an die Stelle kamen, wo es hieß, ich und meine Freunde
wollten Blutvergießen über das Land bringen und einen neuen
Krieg anstiften, und das ich ein Feind der Königs sei, da hoben
sie ihre Hände auf. Da erhob ich meinen Arm und sagte: \zitat{Ich
bin der Mann, gegen den sich diese Anklage richtet, aber ich bin
so unschuldig wie ein neugeborenes Kind in dieser Sache, und
habe nie den Gebrauch der Waffen gelernt. Und meint ihr, wenn
ich oder meine Freunde solche wären, wie es in der Anklage
heißt, so hätten wir unsere Anklage selber hierher gebracht?} [...]
Sie fragten mich, was man mit der Anklage tun solle? ich sagte:
\zitat{Ihr seid die Richter und könnt hoffentlich in dieser Sache
richten; tut also was ihr wollt. [...]}

Sie sagten, sie wollten mich nicht verurteilen, denn sie hätten
nichts gegen mich. Esquire Marsh\footnote{Esquire Marsh, eine 
angesehenen Persönlichkeit am Hofe Karl II, war den
Ouäkern geneigt und bemühte sich oft für sie und schützte sie vor 
Verfolgungen.}\person{Marsh, Esquire} erhob sich und sagte, es sei
% \picinclude{./150-159/p_s154.jpg} 
des Königs Wohlgefallen, das ich freigesprochen werde, wenn sich
kein Kläger gegen mich erhebe. Sie fragten mich, ob ich es dem
König und dem Rat überlassen wolle? Ich sagte: \zitat{Ja, gerne.}
Hierauf schickten sie des Scheriffs Bericht, der die Anklage enthielt,
dem König, damit er sehe, wessen man mich beschuldigte [...]

Der König, nachdem er dies gelesen und von der ganzen
Angelegenheit unterrichtet war, war von meiner Unschuld überzeugt 
und sandte einen Befehl, mich frei zu lassen, welcher lautete:

\grosszitat{
  Es beliebt Seiner Majestät, zu befehlen, das man dem Manne
  George Fox, bislang Gefangener im Kerker von Lancaster, die
  volle Freiheit schenke [...] Und diese Kundgebung von Seiner
  Majestät Belieben soll euch als Befehl genügen.

  \begin{flushright}Whitehall, 24. Oktober,1660\index{Jahr!1660}. 
Edward Nicholat\person{Nicholat, Edward}\end{flushright}
}

Als ich nun mehr als zwanzig Wochen gefangen gewesen
war, war ich auf Befehl des Königs rechtmäßig frei geworden;
die Macht des Herrn hatte meine Unschuld herrlich kund getan.
Porter wagte nicht, die Anklage, die er fälschlich gegen mich
erhoben hatte, öffentlich zu berichtigen [...]

\chapter[Beginn neuer Quälerverfolgungen]{Beginn neuer Quälerverfolgungen}

\begin{center}
\textbf{Beginn neuer Quälerverfolgungen bei Anlas der Verschwörungen
der Fifthmonarchy-Leute. Des Quäkers John Perots Verirrungen.
Quäker misshandelt in Neu-England und Malta.}
\end{center}

Ich sah nun, warum ich durch so schwere Prüfungen hatte
gehen müssen in Reading, denn die ewige Kraft des Herrn war
über alle gekommen, und sein gesegnetes Leben und Licht und
seine Wahrheit war dem Lande ausgegangen; wir hatten herrliche 
Versammlungen, und viele scharten sich um die Wahrheit.
Richard Hubberthorn\person{Richard Hubberthorn} war beim König 
gewesen, und dieser hatte
gesagt, es dürfe uns niemand behelligen, so lange wir friedlich
leben; er versprach uns dies bei seinem königlichen Wort mit der
Ermahnung, sein Versprechen nicht zu missbrauchen. Einige
Freunde erhielten auch Zutritt im 
\textit{House of Lords}\index{House of Lords}, und man
gestattete ihnen, ihre Gründe gegen das 
Zehntenwesen\index{Kirchensteuer}\footnote{Kirchensteuer}, 
das Schwören, die Turmhäuser, die Gottesdienste und anderes 
auseinander zusetzen, und man hörte ihnen ziemlich lange zu. Etwa 700 Freunde,
die unter Richards und Olivers Regierung in die verschiedenen
% \picinclude{./150-159/p_s155.jpg} 
Gefängnisse des Landes gebracht worden waren, wegen Verstößen,
wie sie es nannten, setzte der König, als er kam, in Freiheit.
Man spürte, das die Regierung geneigt war, den Freunden ihre
Freiheit zu sichern, weil sie sah, das wir unter der vorherigen
Herrschaft so gut wie sie gelitten hatten. Sobald aber wirklich
etwas für uns geschehen sollte, so wurde es wieder vereitelt durch
irgend einen Schuft, der dergleichen tat, als ob er uns wohl
wolle. Es hieß, es sei schon ein Befehl ausgesetzt, der unsere
Freiheit bestätige, er müsse nur noch unterzeichnet werden; da
brach gerade jenes hässliche Attentat\index{Attentat} der 
Fisthmonarchy-Leute\index{Fisthmonarchy}
aus und brachte die Hauptstadt\ort{Hauptstadt} und das ganze Land in Aufruhr.
Es geschah am Abend eines Ersten Tages, und wir hatten eine
besonders herrliche Versammlung gehabt, in der die Wahrheit
des Herrn allen erschienen war. Da, bald nach Mitternacht wurde
die Trommel geschlagen und erklang der Ruf: \zitat{zu den Waffen,
zu den Waffen!} Ich stand auf und nahm am folgenden Morgen
ein Boot und fuhr nach Whitehall\ort{Whitehall}, und stieg dort aus und schritt
durchs Schloss. Sie betrachteten mich erstaunt, aber ich schritt
durch sie hindurch bis nach Pall-Mall\footnote{eine Straße in der 
City of Westminster in London.}\ort{Pall-Mall}, wo sich etliche Freunde
zu mir gesellten, obgleich es jetzt gefährlich war, über die Straßen
zu gehen; denn schon waren die Stadt und die Vorstädte unter
Waffen, und das Volk und die Soldaten waren sehr roh; sie 
misshandelten Henry Fell, der zu einem Freunde gehen wollte, und
hätten ihn getötet, wenn nicht der Herzog von York dazu gekommen
wäre. Es geschah viel Unheil während dieser Woche, und am
nächsten Ersten Tage wurden viele Freunde auf dem Weg in die
Versammlung gefangen genommen.

Ich blieb in Pall-Mall, weil ich dort der Versammlung beiwohnen 
wollte; doch in der Nacht des Siebenten Tages kamen
Soldaten und klopften an die Tür. Da die Mägde sie einließen,
so stürzten sie herauf und ergriffen mich. Und einer von ihnen,
der beim Parlament gedient, fühlte mir in die Tasche und fragte,
ob ich keine Pistolen bei mir habe. Ich fragte ihn, warum er
auch solche Frage an mich stelle; er wisse ja, das sich ein 
friedlicher Mann sei [...] Diese Soldaten nahmen mich mit und
brachten mich nach Whitehall [...] Dort waren die Soldaten
und das Volk sehr wild; aber ich predigte doch die Wahrheit
unter ihnen. Einige Große aber, als sie das hörten, sagten:
\zitat{Was, ihr lasst ihn noch predigen? Bringt ihn doch an einen
% \picinclude{./150-159/p_s156.jpg} 
Ort, wo er nicht mehr hetzen kann.} Das taten sie denn auch
und bewachten mich. Ich sagte ihnen, wenn sie schon meinen
Leib binden und einsperren können, so können sie doch das Wort
des Lebens nicht aufhalten. Einige kamen und fragten mich,
was ich sei? Ich erwiderte ihnen: \zitat{ein Prediger der 
Gerechtigkeit} (2. Petr. 2, 5\bibel{Petr. 2. 02:05@2. Petr. 2:5}). 
Nachdem ich etwa drei 
Stunden eingesperrt gewesen war, ging Esquire Marsch zum Lord 
Gerrard\person{Lord Gerrard}, und darauf wurde ich frei gelassen [...]

Während dieses Aufstandes\index{Aufstand} der Fifthmonarchy-Leute 
1660\index{Jahr!1660},
fanden arge Metzeleien statt, sowohl auf dem Lande als in der
Stadt, so das es für anständige Leute noch lange gefährlich war,
auszugehen. Man konnte kaum ohne Gefahr Einkäufe machen.
Auf dem Lande schleppten sie die Leute, Männer und Frauen,
aus den Häusern und Kranke rissen sie aus ihren Betten. Ja
einen Fieberkranken rissen sie aus dem Bett und schleppten ihn
ins Gefängnis; als er dort ankam, starb er, er hieß Thomas
Pachyn.\person{Pachyn, Thomas}

Margaret Fell\person{Fell, Margaret} ging zum König und berichtete ihm, wie es
zugehe zu Stadt und Land. Sie setzte ihm auseinander, das wir
harmlose, friedliche Leute seien; das wir aber unsere Versammlungen 
auch fernerhin halten würden, was immer wir auch zu
dulden haben würden; aber es sei seine Pflicht für Frieden zu
sorgen, damit nicht noch mehr unschuldig Blut vergossen werde.
Die Gefängnisse waren nachgerade überall angefüllt mit
Freunden und andern aus der Stadt und vom Lande; überall
waren Wachen ausgestellt zur Durchsuchung der Briefe, so das
niemand passieren konnte, ohne untersucht zu werden. Wir hörten
von vielen Tausenden von Freunden, die im Lande herum in den
Gefängnissen waren, und Margaret Fell überbrachte dem König
und dem Rat einen Bericht darüber. Als wir in der darauf 
folgenden Woche von einigen weiteren Tausenden hörten, die 
gefangen genommen worden, ging sie abermals hin, um es dem König
und dem Rat mitzuteilen. Man verwunderte sich, woher wir
diese Nachrichten hätten, da ein strenger Befehl ergangen war,
alle Briefe aufzufangen. Aber der Herr fügte es, das wir Kunde
erhielten trotz allen ihren Hindernissen.

Wir ließen eine Erklärung gegen das Bekriegen und das
Verschwören drucken und schickten einige Abzüge an den König
und den Rat; andere wurden in den Straßen verkauft [...]
% \picinclude{./150-159/p_s157.jpg} 

Diese Erklärung klärte etwas die Luft, die aus Stadt und
Land lastete; und der König erließ bald darauf einen Befehl,
das die Soldaten keine Haussuchungen ohne einen Konstabler
vornehmen sollten; aber noch immer waren die Gefängnisse gefüllt
und viele Freunde gefangen, woran namentlich der Aufstand der
Fifthmonarchy-Leute Schuld war. Als aber die Gefangenen
sollten hingerichtet werden, ließ man ihnen doch Gerechtigkeit 
widerfahren und erklärte uns öffentlich frei von jeglicher Teilnahme an
den Verschwörungen. Und auf wiederholtes Drängen erließ der
König den Befehl, die Freunde frei zu lassen ohne Loskaufung.
Aber es hatte viel Mühe und Arbeit gebraucht, um das zu
erreichen. Thomas Moor und Margaret Fell waren oft deswegen
beim König gewesen.

Es wurde während dieses Jahres viel Blut vergossen, denn
viele von den Räten des früheren Königs wurden gehenkt, ertränkt
oder gevierteilt. Unter diesen war auch Oberst Hacker, der mich
unter Oliver Cromwell\person{Cromwell, Oliver} von 
Leicester\ort{Leicester} nach London\ort{London} als Gefangener
schickte, wie oben berichtet worden. Es war ein trauriger Tag
der Vergeltung von Blut durch Blut [...]

Es war eine unsichtbare Hand, die diesen Tag über das
heuchlerische Geschlecht gebracht hatte, das, kaum war es zur
Herrschaft gelangt, so hochmütig und über alle Maßen grausam
geworden war und das Volk Gottes verfolgt hatte.
Mehr als einmal waren diese \zitat{Frommen} gewarnt worden
durch Worte, Schrift und Zeichen; aber sie wollten es nicht
glauben, bis es zu spät war. Willam Sympson war öfter während
drei Jahren getrieben worden, unbekleidet und barfuß unter sie
zu treten in den Städten, Märkten und Ortschaften, vor Priester
und Große, um ihnen zu sagen: \zitat{so nackt wie er würden auch sie
einhergehen.} Und manchmal trieb es ihn, sich mit einem Sack
anzutun und sein Gesicht zu beschmieren und ihnen zu sagen:
\zitat{also werde der Herr ihnen ihre Frömmigkeit besudeln, wie er
besudelt sei.} Der arme Mensch hatte schwere Leiden erduldet,
sich mit Pferdepeitschen auf seinen bloßen Leib peitschen, sich mit
Steinen bewerfen und einsperren lassen während dieser Jahre, vor
der Rückkehr des Königs; aber sie wollten sich nicht warnen
lassen, sie erwiderten seine Liebe mit Grausamkeit. Nur der
Bürgermeister von Cambridge behandelte ihn großmütig, indem
er seinen Rock um ihn legte und ihn in sein Haus nahm.
% \picinclude{./150-159/p_s158.jpg} 

Ein anderer Freund, Robert Huntingdon\person{Huntingdon, Robert}, wurde getrieben
ins Turmhaus zu Carlisle zu gehen, unter die 
Haupt-Presbtyterianer\index{Presbtyterianer}
und Independeten\index{Independeten} dort, in einem weißen Hemd als
Zeichen, das das Chorhemd wieder aufkommen werde, und mit
einem Halfter, um zu zeigen, daß auch für sie ein Halfter kommen
werde, was sich auch an einigen von den Verfolgern erfüllte.

Zu einem andern, Richard Sale\person{Sale, Richard}, der Konstabler in der Nähe
von Chester war, wurde ein Freund mit einem Pass geschickt;
die schändlichen \zitat{Frommen} hatten ihn als Vagabunden 
festgenommen, weil er als Prediger reiste. Dieser Konstabler wurde
durch den ihm zugeschickten Freund bekehrt und gab ihm die
Freiheit; später wurde er selber ins Gefängnis geworfen. Darnach
an einem Ersten Tage trieb es Richard Sale ins Turmhaus zu
gehen und den verfolgungssüchtigen Priestern und ihren Genossen
eine Kerze zu bringen als Anspielung auf ihre Finsternis; aber
sie misshandelten ihn, und, so recht wie verstockte \glqq Fromme\grqq{}, warfen
sie ihn ins Gefängnis von Little-Gase\ort{Little-Gase}, 
und quälten ihn dort dermaßen,
das er bald darauf starb. Die Freunde wurden mehrfach getrieben, 
dieses Geschlecht auf allerlei Art zu mahnen, aber nicht
nur hörte man sie nicht, sondern sie wurden noch misshandelt und
\zitat{verschrobene Quäker} genannt! Aber Gott schickte sein Gericht
über die verfolgungssüchtigen Priester und Behörden; als der
König zurück kam, wurden den meisten von ihnen ihre Stellen und
Einkünfte entzogen; die Räuber wurden beraubt und es war nun
an uns zu sagen: \zitat{wo sind jetzt die Verschrobenen?} Viele gaben
jetzt zu, das wir wahre Propheten\index{Wahre Propheten} seien, 
und behaupteten, wenn
wir nur gegen einzelne Priester geeifert hätten, so hätten sie sich
sogar über uns gefreut; weil wir aber gegen alle geeifert hatten,
so haben sie sich über uns geärgert. Aber sie sahen es jetzt ein,
das die Priester, die damals für die besten gehalten wurden, so
schlecht waren wie die übrigen. Ja, viele, die als die aller 
hervorragendsten gegolten hatten, hetzten die Behörden am 
allermeisten zu den Verfolgungen aus. Diese wurden aber, als der
König zurück kam, damit bestraft, das ihnen die Gewissensfreiheit
entzogen wurde, die sie vorher, als sie die Oberhand gehabt hatten,
den andern nicht gegönnt hatten. Einer, namens Hewes, von
Plymouth, ein angesehener Priester zu Olivers Zeits hatte immer,
wenn irgendwo Gewissensfreiheit zugestanden wurde, gebetet, Gott
wolle es den Behörden ins Herz geben, das sie diese verdammte
% \picinclude{./150-159/p_s159.jpg} 
Toleranz abschaffen. Und andere beteten gegen \zitat{die nicht zu
duldende Duldsamkeit}. Als nun nach des Königs Rückkehr
diesem Priester Hewes seine großen Einkünfte entzogen wurden,
weil er sich nicht dem Common-Prayer Buch unterwerfen wollte,
fragte ihn ein Freund, als er ihm in Plymouth begegnete: ob er
nun die Duldsamkeit noch verdammungswürdig finde und nicht
vielmehr froh darüber wäre? worauf der Priester nicht antwortete
und nur das Gesicht abwandte. Aber so hartnäckig auch diese
Leute früher gegen Duldsamkeit geeifert hatten, — jetzt kamen
viele von ihnen selber beim König um einen Ort, wo sie ihre
Versammlungen halten könnten, ein, und bezahlten sogar, damit
es ihnen bewilligt werde [...]

Wir erhielten die Kunde, das ein Freund, der getrieben worden
war, gegen den Götzendienst der Papisten\index{Papisten} zu predigen, in Rom
im Gefängnis gestorben war, und man hatte den Verdacht, das
er heimlich im Gefängnis umgebracht worden war. John Perrot \person{Perrot, John}
war auch dort gefangen gewesen und kam nach seiner Freilassung
zu uns zurück; später aber wandte er und andere sich ab von
der Gemeinschaft der Freunde und der Wahrheit [...]

Ungefähr um die gleiche Zeit 1661\index{Jahr!1661} erhielten wir die Nachricht 
aus Neu-England\ort{Neu-England}, das die dortige Regierung ein Gesetz
erlassen hatte, das die Quäker aus jenen Kolonien verbannte bei
Todesstrafe im Fall der Rückkehr, und das mehrere Freunde, die
nach ihrer Verbannung zurückgekehrt waren, wirklich gehängt
worden waren \footnote{1661 verfolgten die Puritaner\index{Puritaner} und 
Independenten\index{Independenten}, die selber nach Amerika geflohen, um 
Religionsfreiheit zu haben, die Quäler aufs Grausamste.
Ein Edikt von 1658\index{Jahr!1658} bestimmte, das jeder Quäker, der zum dritten 
mal in den Kolonien gefunden werde, gehängt werde, und 
dieser Befehl wurde an zwei Quäkern und einer Quäkerin 
ausgeführt.}, und andere vom gleichen Schicksal bedroht im
Gefängnis seien. Während jene hingerichtet wurden, war ich im
Gefängnis zu Lancaster gewesen und hatte eine deutliche Wahrnehmung 
ihrer Leiden, als ob sie mich selber betroffen hätten,
und der Strick um meinen eigenen Hals gelegt würde, und wir
hatten doch damals noch nichts davon gehört. Sobald wir nun
davon erfuhren, ging Edward Burrough\person{Burrough, Edward} zum König und sagte
ihm, es sei in seinem Reich eine Ader offen, aus der unschuldiges
Blut fließe, das, wenn es nicht gestillt werde, alles überschwemmen
werde. Der König erwiderte: \zitat{Ich werde dieses Blut stillen.}