\grosszitat{
    An den König von Polen.
    \bigskip
    An Johann den Dritten\person{Johann III.}, König 
    von Polen\person{König von Polen}, Großherzog von
    Litauen\ort{Litauen}, Russland\ort{Russland} und Preußen\ort{Preußen}, 
    Beschützer der Stadt Danzig, [...]
    wegen der heimgesuchten und unschuldigen Leute, die man im
    Groll Quäker nennt, die jetzt bei Wasser und Brot in oben
    genannter Stadt sind, in strenger Gefangenschaft, wo man ihren
    Frauen und Kindern kaum erlaubt, sie zu besuchen.
    \bigskip
    O König!
    \bigskip
    Die Behörden der Stadt Danzig\ort{Danzig} sagen, es sei dein Wille,
    das dieses unschuldige, heimgesuchte Volk solche Unterdrückung zu
    erleiden habe. Nun ist die Strafe nur darum über sie verhängt
    worden, weil sie zusammenkommen im Namen Jesu Christi ihres
    Erlösers und Heilands, der für ihre Sünden starb und zu ihrer
    Rechtfertigung von den Toten auferstanden ist, der ihr Prophet
    ist, welchen Gott erweckt hat, wie Moses
    Und nun, in diesen Tagen des neuen Evangeliums und des
    neuen Bundes, sollten alle auf ihn hören, die \zitat{gewesen wie die
    irrenden Schafe, nun aber sich bekehrt haben zum Hirten und
    Bischof ihrer Seelen. Er hat sein Leben gegeben für seine
    Schafe, und sie hören seine Stimme und folgen ihm und er führt
    sie auf seine Weide} (Joh. 10:9\bibel{Joh. 10:09@Joh. 10:9}).
    % \picinclude{./300-309/p_s300.jpg} 
    Ich hörte, O König, du bekennest dich öffentlich zum Christen-
    tum und zum mächtigen Namen Jesu Christi, deö Königö der
    Könige, deß Herrn der Herren, dem alle Gewalt im Himmel und
    auf E-rden gegeben ist, der alle Völker mit eisernem Szepter
    regiert. GZ scheint uns darum hart, o König, das; jemand, der
    offen Christuö bekennt, solche Strafen über ein harmloseß und
    unschuldigeß Volk verhängt, nur weil sie zarte Gewissen haben
    und zusammen kommen, um den ewigen Gott, der sie gemacht
    hat, im Geist und in der Wahrheit anzubeten, wie Christuß etz vor
    1600 Jahren eingesetzt hat nach Joh. 4, 23. 24.
    Jch bitte nun den König, darüber nachzudenken, ob Christus
    im neuen Testament je seinen Aposteln ein Gebot gegeben, sie
    sollten jemand ins- Gefängniö werfen bei Wasser und Brot, der
    sich nicht in allen Stücken ihrer Religion, ihrem Glauben und
    ihrer Art der Anbetung anschloß? Wo haben die Apostel nach
    der Himmelfahrt in der wahren Kirche solcheß getan,? Lehren
    nicht Christue und die Apostel, seine Nachfolger sollen die Feinde
    lieben, und bitten für die, welche sie hassen, verfolgen und ver-
    leumden? (Matth. 5.)
    Jst etz nicht eine Schande für das Christentum den Türken
    und andern gegenüber, daß ein Christ den andern verfolgt um
    der Glarfenölehre, der Art der Anbetung und der Religion
    willen? Sie können nicht beweisen, daß Christuz, den sie ihren
    Herrn und Meister nennen, je ein solches Gebot gegeben. Christu-3
    sagt, seine Nachsolger sollen sich unter einander lieben, daran
    werde man erkennen, daß sie seine Jünger seien (Joh. 13). Und
    hat nicht Christuö jene getadelt, die wollten Feuer vom Himmel
    regnen lassen, um alle zu verderben, die ihn nicht aufnehmen
    wollten?« (Luc. 9). Gr sagte zu ihnen: »Wisset ihr nicht, weö
    Geistes Kinder ihr seid? Wissen je die, welche die Menschen
    verfolgen oder töten, weil sie eine Religion nicht annehmen
    wollen, weß Geisteö Kinder sie sind? Wäre ez nicht gut,
    wenn alle durch den Geist Christi wüßten, weß Geistes
    Kinder sie sind? Denn der Apostel sagt, Römer 8, 9: ,,Wer
    Christi Geist nicht hat, der ist nicht sein«; und 2. Cor. 10, 4:
    ,,Die Waffen unsrer Riiterschaft sind nicht sleischlich sondern geist-
    lich«; und Epl). 6, 12: ,,Wir haben nicht mit Fleisch und Blut zu
    % \picinclude{./300-309/p_s301.jpg} 
    Kampf für die Ordnung im Quäkertum. Jakobyll. Amnesiie. 301
    kämpfen, sondern mit den bösen Geistern unter dem Himmel.«
    Daraus- ersehen wir, daß der Kampf der ersten Christen und
    ihre Waffen geistiger Art waren. Würde eß dem König und den
    Behörden von Danzig nicht gegen daß Gewissen gehen, wenn die
    Türken sie zu ihrer Religion zwingen würden? oder wenn die
    Behörden von Danzig zur Religion det-’ Königö von Polen ge-
    zwungen würden? oder würde etz der König von Polen nicht
    grausam und gegen sein Gewissen finden, wenn er zur Religion
    der Behörden von Danzig gezwungen würde? und im Fall sie
    sich derselben nicht unterwerfen wollten, von Weib und Kindern
    getrennt und auz dem Lande verbannt, oder bei Wasser und Brot
    inö Gefängniß geworfen würden ?
    Wir bitten darum den König und die Behörden in aller
    christlichen Demut, daß sie in dieser Angelegenheit nach dem
    königlichen Gesetz Gotteß vorgehen möchten, nämlich ,,andern zu
    tun, waö sie möchten, daß man ihnen tue« (Jak. 2, 8), ,,rmd
    ihren Nächsten zu lieben als- sich selbst« (Matth. 22, 39). Denn
    wir hoffen und glauben denn doch, daß sowohl der König von
    Polen und seine Leute, altz auch die Behörden von Danzig, die
    Schriften deß neuen Testamentß sowie deö alten, kennen; und wir
    bitten darum den König und seine Räte, darauf zu achten, daß
    sie nicht dem königlichen Gesetz Gotteß und dem herrlichen und
    ewigen Evangelium der Wahrheit entgegen, ein unschuldigeß Volk
    gefangen nehmen, nur weil etz zusammen kommt mit zarten Ge-
    wissen, um Gott seinem Schöpfer zu dienen und ihn anzubeten.
    Wir bitten den König in christlicher Liebe, dieseö alles ernst-
    lich und eingehend zu bedenken, und Befehl zu geben, daß die
    unschuldigen Gefangenen, unsere Freunde, die sogenannten Quäker,
    frei gelassen werden autz ihrer harten Gefangenschaft in Danzig,
    daß sie frei sein mögen, den lebendigen Gott im Geist und in der
    Wahrheit anzubeten und ihm zu dienen, und heimzugehen um
    ihr Handwerk weiter zu treiben und ihre Familien zu erhalten.
    Wir glauben, daß der König, wenn er solch ein edles, ruhm-
    oolleß, ja christlicheö Werk tut, nicht unbelohnt bleiben wird
    von dem großen Gott, dem wir dienen, der die Herzen der
    Könige und ihr Leben und die Länge ihrer Tage in seiner
    Hand hat.
    Von Einem, der möchte, daß der König und alle seine
    Räte in der Furcht de?. Herrn bewahrt bleiben mögen und
    % \picinclude{./300-309/p_s302.jpg} 
    sein Wort der Weisheit annehmen, durch welches alle Dinge ge-
    schaffen wurden. [...]
    \bigskip
    \begin{flushright}
    London, 10. des 3. Monats, den man pflegt Mai zu nennen, 1684.
    G. F.\end{flushright}
}


Jch schrieb in dieser Zeit noch manches andere im Dienst
der Wahrheit; etwas »über das Richten,« denn etliche, die von
der Wahrheit abgesallen waren, hatten eine solche Angst, von ihr
gerichtet zu werden, daß sie sich eifrig bemühten, gegen das Richten
zu schreien. Darum erließ ich ein Schreiben, um aus der Schrift
der Wahrheit zu beweisen, daß die Kirche Christi Macht hat, alle,
die vorgeben dazu zu gehören, nicht nur in Dingen dieser Welt,
sondern auch in religiösen Dingen, zu richten. ....
Bei den Verhören in den Gerichtssitzungen, im 2. Monat 1686
in Hicks Hall, wurden viele Freunde vorgenommen; ich war
täglich bei ihnen, um zu raten und zu helfen, damit nichts ver-
säumt und kein Vorteil unbenutzt bleibe; und gewöhnlich hatten
sie guten Erfolg. Bald daraus gefiel es dem König, nachdem
wir ihm immer wieder Klagen über unsere Leiden vorgelegt hatten,
zu befehlen, daß man: »alle die um des Gewissens willen gefangen
waren, sreilasfe, soweit er Macht habe es zu bestimmen.« Die
Türen der Gesängniss e taten sich denn auch aus, und Viele hundert
Frermde, von denen manche lange gefangen gewesen waren, er-
hielten die Freiheit. Viele oon ihnen kamen zur Jahresversamnn
lung, zur großen Freude der Freunde. ....
Jch brachte den größten Teil des Jahres 1686 in London
zu, außer wenn ich nach Bethnal-Green oder Enfield ging, oder
nach Chiswick, wo ein Freund eine Schule errichtet hatte, in der
Kinder von Freunden erzogen wurden.
.... Auch schrieb ich noch allerlei in diesem Jahr, unter anderm
eine Ermahnung » an die Freunde, in der Einigkeit der Wahrheit zu
bleiben, in welcher keine Trennung noch Gntzweiung ist.« ....
Bald daraus, als ich merkte, daß etliche Abtrünnige, welche
der Feind zur Trennung und Spaltung von den Freunden geführt
hatte, fortsuhren in ihrem Schreien und ihrem Widerstand gegen
unsre Monats-, Vierteljahres- und Jahresversammlungen, so trieb
es mich, einen kurzen Brief an die Freunde zu schreiben, um sie zu
erinnern, daß sie durch den Geist des Herrn in ihrem Innern
die Bestätigung und Besiegelung empfangen hatten, daß diese
Versammlungen vom Herrn seien und von ihm angenommen


% \picinclude{./300-309/p_s303.jpg} 
Wirken in London unter dem Zeichen der Toleranz. 303
werden, und daß sie darum nicht von den Gegnern erschüttert
werden können:
,,Meine lieben Freunde im Herrn Jesus Christus,
Jhr alle, die ihr in seinem heiligen Namen versammelt seid,
wisset, daß eure Versammlungen, die oierteljährlichen wie die
andern, durch die Kraft und den Geist Gottes eingesetzt sind.
Sie sind in euern Herzen von dieser Kraft und diesem Geist be-
zeugt, durch die Kraft und den Geist Gottes sind sie in euch gegründet
und ihr in ihnen. Gott der Herr hat es euch durch seinen Geist
besiegelt, daß eure Versammlungen nach seiner Ordnung und
Einberufung geschehen, und er hat sie anerkannt, indem er euch
mit seiner gesegneten Gegenwart in denselben begnadete; ihr
habt es reichlich erfahren, wie er euch mit seinem Leben, seiner
Weisheit und Kraft und mit himmlischen Gütern aus seinen
Schätzen und Quellen ausriistete, und aus denselben sind wiederum
viele Danksagungen und Lobpreisungen in euern Versammlungen
zu seinem heiligen Namen zurückgekehrt. Er hat euch eure Ver-
sammlungen durch seinen Geist besiegelt, und daß euer Zusammen-
kommen im Herrn geschehen ist, in Christus seinem Sohn und in
seinem Namen und nicht durch Menschen. Darum gebühret dem
Herrn, daß er durch sie und in ihnen gepriesen werde, der euch
und sie beschützt hat mit seinem mächtigen Arm gegen alle Gegner
und Feinde und ihre verleumderischen Zungen und Bücher. Denn
des Herrn Macht und Same regieret über alles; er erhält seine
Kinder zu seiner Ehre, als die so »Obmacht und Hoheit haben
am Ehrentage« (Ps. 110, 3) .... aus daß alle dem Herrn in
Jesus Christus dienen von Geschlecht zu Geschlecht.«
London, den 3. des 11. Monats 1686. G. F.

\chapter[Toleranz]{Toleranz}

\begin{center}
\textbf{Wirken in London unter dem Zeichen der Toleranz.}
\end{center}


Jch besuchte in London die Kranken und Betrübten und
schrieb Bücher und Schriften zur Ausbreitung der Wahrheit oder
zur Widerlegung von Jrrtümern. Da es eine Zeit allgemeiner
Freiheit war, so traten die Papisten mehr als vorher mit ihren
Gottesdiensten hervor, und da viele Wankelmütige hingingen, um


% \picinclude{./300-309/p_s304.jpg} 
ihnen zuzusehen, so war ein großes- Gerede von ihrem Beten zu
Heiligen, mit dem Rosenkranz und dergleichen; darum schrieb ich
eine kurze Schrift über das Beten:
,,JesuK3 Ehristuß, alß er seine Jünger über daß Beten be-
lehrte, sagte zu ihnen: ,,Wenn ihr betet, so sollt ihr sagen: Unser
Vater, der du bist in dem Himmel, geheiliget werde dein Name«
(Matth. 6,9). Christuß sagt nicht, sie sollten zu Maria, der
Mutter Jesu beten, noch zu Engeln, noch zu den Heiligen, die
tot sind. Christus- lehrte sie nicht, zu Toten oder für Tote
beten, ebensowenig lehrte Ehristuß oder seine Apostel die Gläubigen
mit Rosenkränzen beten, oder zur Orgel singen, sondern der
Apostel sagt, er wolle singen und beten im Geist, ,,denn der
Geist vertritt unö, und der Herr, der die Herzen forschet, weiß maß
deö Geiste-3 Sinn fei« (Röm. 8,26.27). Daß- Gesetz Gotteß hat
verboten, sich mit den Toten zu beraten; man soll sich mit Gott
beraten, der im neuen Bunde in den Tagen dez Evangeliumß
C-hristum gesandt, um ein Berater und Führer zu sein für alle,
die an sein Licht glauben. Man soll sich nicht um der Lebenden
willen an die Toten wenden, daß Gesetz Gottes- verbietet etz
(Jes. 8,19). Uber die Juden, die daß stillrinnende Wasser Siloah
verachteten, kamen die Fluten und Wasser von Babylon und
Assyrien und brachten sie in die Gefangenschaft (Jes. 8) und die,
welche die Wasser Christi verachten, werden von den Wassern
der Welt, welche im Argen liegt, überflutet. Die, welche Holz
und Steine befragten, waren verführt vom Geist der Hure, und
der Verirrung, daß sie wider ihren Gott Hurerei trieben (Hos. 4,12).
Und die sich an den Baal Peor hingen und von den Opfern der
toten Götzen aßen, reizten den Herrn, und die Plage brach
auf sie ein (Psalm 106,28). Hier könnt ihr sehen, daß die
Totenopfer verboten waren. Die Lebendigen wissen, daß sie
sterben werden, die Toten aber wissen nichtö, sie verdienen auch
nichtö, denn ihr Gedächtniß ist vergessen (Pred. 9,5). »Wehe den
abtrünnigen Kindern«, sagt der Herr, ,,die ohne mich ratschlagen
und ohne meinen Geist Schutz suchen, zu häufen eine Sünde über
die an.dere« (Jes. 30,1). G. F.
Die Jahres-versammlung, welche am 16. des 3. Monattz 1687
begann, war sehr zahlreich, weil die Freunde leichter von über-
allher dazu kommen konnten, da überall größere Freiheit und
Toleranz herrschte .....


% \picinclude{./300-309/p_s305.jpg} 
Wirken in London unter dem Zeichen der Toleranz 305
Von London ging ich nach Hertford, . . . dann nach Wal-
tham Abbey ..... und hierauf mit einigen Freunden zu William
Mead, in Essex. Hier blieb ich einige Wochen, war jedoch nicht
miißig, sondern besuchte viele Versammlungen in der Umgegend.
.... Zwischendurch schrieb ich vieles zur Ausbreitung der Wahr-
heit, und um den Leuten das Verständnis dafür zu öffnen. In
einem dieser Schreiben bewies ich aus der Schrift, daß die Leute
zuerst Buße tun müssen, ehe sie das Evangelium und den heiligen
Geist und das Reich Gottes aufnehmen oder getauft werden können.
Eine andere Schrist, die ich um diese Zeit schrieb, zeigte, wie
alle, die Gott angehören, auch ihm ähnlich werden sollten:
,,Gott ist gerecht und will, daß auch die Seinen gerecht seien.
Gott ist heilig, und die Seinen sollen auch heilig sein. Gott ist
Licht, und die Seinen sollen in diesem Licht wandeln. Gott ist
Geist, ewig und unoergänglich, und die Seinen sollen in diesem
Geist wandeln. Gott ist barmherzig, und die Seinen gsollen auch
barmherzig sein. Gott läßt seine Sonne scheinen über Gute und
Böse und regnen über Gerechte und Ungerechte (Matth. 5), und
die Seinen sollen auch also gesinnt sein. Gott ist die Lieche, und
die in der Liebe bleiben, bleiben in Gott« (1. Joh. 4). Die Liebe
tut dem Nächsten nichts zu leide, also ist die Liebe des Gesetzes
Erfüllung (Röm. 13,10). Der Apostel sagt: ,,Das ganze Gesetz
ist erfüllt in dem Einen: liebe deinen Nächsten wie dich selbst«
(Gal. 5,14). ,,Wie mich der Vater liebet, also liebe ich euch;
bleibet in meiner Liebe« (Joh. 15,9). Und so sollte es sein bei
allen, die zum Volke Gottes gehören«.
Gooses, den 6. Monat 1687. G. F.
Daß Gottes Volk diese Gesinnung haben sollte, werden die
meisten Menschen zugeben, die wenigsten jedoch wissen, wie sie
dazu gelangen können; darum schrieb ich nach den Offenbarungen
des Geistes der Wahrheit ein anderes kurzes Schreiben, um den
rechten Weg und die rechten Mittel zu zeigen, durch die man zu
Christus kommt und Gott ähnlich wird .....
Ehe ich weiterreiste, Versaßte ich noch ein anderes Schreibens-
in dem ich an vielen Beispielen der heiligen Schrift zeigte, daß
das Reich Gottes, von welchem die meisten als von etwas Fernem
reden und gänzlich in ein anderes Leben verlegen, zum Teil schon
in diesem Leben erreicht und erkannt wird, daß aber niemand
in dasselbe eingehen kann, als wer wiedergeboren ist .....  
George Fo:. 20


% \picinclude{./300-309/p_s306.jpg} 
Nachdem ich mehr als ein Vierteljahr aus dem Lande ge-
wesen war, kehrte ich nach London zurück, ein wenig wohler al-3
vorher, da mir die Landlust sehr gut getan hatte. Da es eine
Zeit großer Freiheit war und viel Gmpfänglichkeit unter den
Leuten, so wirkte ich viel für den Herrn in der Stadt, indem ich
sast täglich öffentlichen Versammlungen beiwohnte und häufig in
Anspruch genommen war durch Besuche bei kranken Freunden und
durch andere Arbeiten für die Kirche. Ich war etwa drei Monate
in London, dann waren meine Kräfte so erschöpft durch die un-
ausgesetzte Arbeit für den Herrn und mein Körper so elend aus
Mangel an frischer Luft, daß ich zu meinem Sohn Robert Rouß
in die Nähe von Kingston ging, wo ich einige Zeit blieb und die
Freunde in Kingston besuchte. Während ich dort war, kam ec-
über mich, ein Schreiben über die Juden abzusassen, in welchem
ich zeigte, wie sie durch ihre Widerspenstigkeit und ihren Unge-
horsam die heilige Stadt und daß heilige Land verloren hatten,
woran alle, die vorgeben, Christen zu sein, sehen können, waö
ihrer wartet, wenn sie Gott durch Ungehorsam versuchen ..... «

\chapter[Krankheit und Tod]{Krankheit und Tod}

\begin{center}
\textbf{Ahnung kommender Revolutionen. Christus König. Letzte
Arbeiten. Krankheit und Tod.}
\end{center}

Jm gz. Monat des Jahres 1688 war ich in London .....
Ich war noch nicht lange da, alö es mir sehr schwer nmz Herz
wurde, und ich vom Herrn ein Gesicht hatte von den großen
Unruhen und Trübsalen, Revolutionen und Umwälzungen, die bald
daraus sich vollzogen. Unter diesem Eindruck rmd getrieben vom
Geist dez Herrn, schrieb ich einen Generalbrief an die Freunde,
um sie vor dem nahenden Sturm zu warnen, auf daß sie alle
ihre Zuflucht beim Herrn suchen möchten:
»Lieben Freunde und Brüder allenthalben, die ihr den Herrn
Jesus Christnö angenommen habt, und denen er Macht gegeben
hat, seine Söhne und Töchter zu werden, .... zeiget euch, trotz
der Fluten und Stürme, die die Welt bewegen, als- die unschul-
digen sansten Lämmlein Christi, die in seiner friedsamen Wahrheit
wandeln und in dem Wort seiner Kraft, Weißheit und Geduld
bleiben; dieses Wort wird euch bewahren in den Tagen der


% \picinclude{./300-309/p_s307.jpg} 
Ahnung kommender Revolutionen. Christus3yKönig usw. 307
Heimsuchungen und Versuchungen, welche über die ganze Erde
kommen werden, um alle, die aus der Erde sind, zu prüfen- Denn
das Wort des Herrn war, ehe die Welt gewesen, und alle Dinge
sind durch dasselbe gemacht (Joh. 1); ez ist ein erprobteö Wort,
das zu allen Zeiten dem Volke Gottez Weißheit, Kraft und Ge-
duld verlieh. Darum bleibet und wandelt in Christus Jesuß,
der das Wort Gotteß genannt wird, und in seiner Kraft, die über
allem ist. Suchet, wa-J droben ist, wo Christuö sitzet zur Rechten
Gottes (Kol. 3,1), .... und suchet nicht, maß Oergänglich ist.
Gepriesen sei der Herr, der sich mit seinem ewigen Arm und ,
seiner Kraft ein einiges Volk zubereitet hat, und sich dasselbe treu
erhalten hat durch viel Triibsal, Prüfungen und Versuchungen,
seine Kraft und sein Same, Christus, ist über allem, und in ihm
habet ihr Leben und Frieden in Gott. Darum stehet alle fest in
ihm und sehet in ihm eure Erlösung, den Anfang und das Ende,
das Amen. Der Allmächtige bewahre und erhalte euch alle in
ihm, eurer Arche und eurem Llllerheiligsten; in ihm seid ihr sicher
vor allen Fluten und Stürmen, denn er war, ehe sie waren,
und wird sein, wenn sie nicht mehr sind.«
London, 17. des 8. Monats- 1688. G. F.
EZ kam um diese Zeit eine große Traurigkeit und Schwer-
mut über mich, wie dietz fast jedeömal vor einer großen Um-
wälzung und Änderung der Regierung gewesen war, und meine
Kräfte verließen mich, sodaß ich vor Schwäche fast zusammenbrach,
wenn ich durch die Straßen ging. Zuletzt konnte ich eine Zeit-
lang gar nicht mehr außgehn, so schwachllwar ich, biz ich fühlte,
daß die Kraft Gottes über allem ausging, und er mir die Gewißheit
gab, daß er seine Gläubigen durch alles hindurch bewahren werde.  
Da ich fortwährend leidend war, so ging ich mit meinem
Sohn Mead in sein Hauö nach Essex und blieb einige Wochen
dort. Ich schrieb Verschiedeneß während dieser Zeit, unter anderm
folgende Zeilen:
,,Während die Menschen hienieden sich um Throne streiten,
sitzt Christuö auf seinem Thron und seine heiligen Engel sind um
ihn. Er ist Anfang rmd Ende, der Erste und der Letzte, über
allem. Und der Herr wird Mittel und Wege schaffen für alle,
die aus reinem heiligen Geiste geboren und Kinder des himm-
lischen Jerusalem sind, damit sie heim kommen zu ihrer wahren
Mutter« ....
20*


% \picinclude{./300-309/p_s308.jpg} 
Ferner verfaßte ich eine Schrift, in der ich an Beispielen
aus der Schrift zeigte, daß viele heilige Männer Gottes, Pro-
pheten und Apostel Christi Handwerker und Ackerbauer ge-
wesen waren, damit man den Unterschied zwischen diesen und den
jetzigen Lehrern der Welt sehe:
,,Der gerechte Abel war ein Hirte und hütete die Schafe
(1. Mos. 4) .... Abraham war ein Ackersmann und hatte
große Herden von Schafen (1. Mos. 13) ..... Jakob war ein
Arkersmann (1. Mos. 30) ..... Moses hütete die Schafe
(2. Mos. 3) ..... David hütete die Schafe seines Vaters in
der Wildnis (1. Sam. 16) ..... Elisa war ein Ackersmann;
er wurde vom Pfluge abberufen, um das Volk Gottes zu lehren
(1. Kön. 19,19) ..... Zu Amos geschah das Wort des Herrn,
als er bei den Hirten war (Amos. 1,1) ..... Petrus und
Andreas berief Christus, als sie fischten (Matth. 4,18) . . . und
Matthäus sah er am Zoll sitzen und sagte: »Folge mir nach''
(Matth. 9,9) ..... Paulus war ein Zeltweber, und weil er
das gleiche Handwerk betrieb wie Aquila und Priseilla, so wohnte
er bei ihnen in Corinth und wob« (Act. 18,3).
Gooses, 1. Monat 1689. G. F.
Viel zu viel Zeit wurde damals zugebracht mit Anhören und
Verbreiten von Neuigkeiten und sonstigem Geschwätz ..... Um die
Nichtigkeit solches Treibens zu zeigen und davon abzumahnen,
schrieb ich folgendes:
,,Hienieden in diesem vergänglichen Leben sind alle Neuig-
keiten unsicher, nichts ist gewiß, aber im Reich Christi sind alle
Dinge beständig und sicher, und alle Neuigkeiten stets gut und
sicher. Denn Christus, dem alle Gewalt im Himmel und aus
Erden gegeben ist, regieret der Menschen Reiche, und er, welcher
der Erbherr ist über alle Heiden und über die Enden der Erde,
herrscht in seiner himmlischen Kraft und seinem Licht, er regieret
alle Völker mit seinem eisernen Szepter und schlägt sie in Stücke
wie die Gefäße eines Töpfers (Ps- 2,9, Offb. 2,27), wenn es
Gefäße zur Unehre und löchrichte Gefäße sind, die ein lebendiges
Wasser nicht halten können (Jer. 2,13), seine auserwählten Gefäße
der Ehre und Gnade jedoch behütet er. SeineMacht ist unerschütterlich
und verändert sich nicht, durch sie bewegt er Berge und Hügel
und macht Himmel und Erde erzittern. Löchrichte, schlechte Ge-
fäße, Hügel und Berge und die alten Himmel und Erden, müssen


% \picinclude{./300-309/p_s309.jpg} 
Ahnung kommender Revolutionen. Christus König usw. 309
alle erschüttert und zerschlagen werden, obgleich sie es nicht
merken, noch ihn sehen, der es tut. Seine Auserwiihlten und
Getreuen aber sehen es und kennen ihn und spüren seine Macht
welche nicht erschüttert werden kann und sich nicht verändert«.
5. des 1. Monats 1689. G. F.
Etwa um die Mitte des ersten Monats 1689 ging ich’;nach
London, das Parlament tagte gerade und beschäftigte sich mit
dem Jndulgenzgesetz. Obschon ich mich schwach fühlte und nicht
gut herumlaufen konnte, nahm mein Geist doch so lebhaft Anteil
an der Sache der Wahrheit und der Freunde, daß ich während
vieler Tage mit einigen Freunden den Verhandlungen beiwohnte
und mit den Mitgliedern verhandelte, damit die Sache nachdrück-
lich und wirksam ausfalle .....
So fuhr ich fort in allerlei Arbeit bis gegen das Ende des
zweiten Monats, wo ich dann, ermiidet von der vielen Arbeit,
die Stadt verließ und nach Southgate ging. ,... z Dort schrieb
ich unter anderm einen Brief an die Freunde in Danzig, die da-
mals unter schweren Verfolgungen litten. Joh errnutigte sie darin,
an ihrem Zeugnis festzuhalten und geduldig ihre Leiden zu er-
tragen. Zu gleicher Zeit schrieb ich aber auch an die Behörden
von Danzig und stellte ihnen das Unrecht dieser Verfolgungen vor:
,,Wir haben eure Verordnungen gesehen und euer Wut-
schnauben gegen jenes kleine Trüppchen, die Lämmlein Christi,
die unter eurer Gerichtsbarkeit in der Stadt Danzig leben. Zweie
habt ihr gefangen nehmen und verbannen lassen, und andern
drohtet ihr mit gleicher Strafe für den Fall, daß sie in die Stadt
zurückkommen. Ebenso verhängt ihr Strafen über solche, die
unsern Freunden ihre Häuser anbieten, um darin zu wohnen oder
um Versammlungen oder Gottesdienst darin zu halten. Wahrlich,
ich bin herzlich betrübt, sowohl über eure Behörden als über eure
Priester, die sich Christen nennen lassen und doch solche unmora-
lischen, unmenschlichen und unchristlichen Handlungen begehen,
weit entfernt vom königlichen Gesetz, welches befiehlt: ,,Tut andern
wie ihr wollt, daß man euch tut«. Witrdet ihr es moralisch,
menschlich, christlich und dem königlichen Gesetz entsprechend finden,
wenn der König von Polen, der eine andere Religion als ihr
hat, euch durch den Henker aus der Stadt verbannen ließe und
euch Seelenmörder nennen würde? Würdet ihr solches nach dem
Gesetz«Gottes finden, das befiehlt: ,,andern zu tun, wie man will,

