
% \picinclude{./130-139/p_s134.jpg} 
%%%%%%%%%%%%%%%%%%% Kapitel 12. %%%%%%%%%%%%%%%%%%%%%%%%%%%%%%

\chapter[Erste Jahresversammlung]{Erste Jahresversammlung}

\begin{center}
\textbf{Erste Jahresversammlung. Warnung an Cromwell vor 
der Königskrone. Trostbrief an dessen Tochter. Vorahnung 
vom Tode Cromwells und der kommenden Reaktion.}
\end{center}


Wir gingen zurück nach England ..... Die Priester
von Newcastel hatten verschiedene Bücher gegen uns geschrieben,
und ein Stadtitltester, Ledger, war uns und der Wahrheit sehr abge-
neigt. Gr sowie die Priester hatten behauptet, die Quäker könnten
nicht in einer Stadt leben, sondern schwirrten wie die Schmetterlinge
in den Hochtälern. Zu diesem Ledger und einigen andern Stadt-
ältesten ging ich, mit Anthony Pearson, mit der Bitte, eine Ver-
sammlung in Neweastel abhalten zu dürfen, nachdem sie so viel
gegen uns geschrieben haben, da wir nun ja in ihre große Stadt
gekommen seien! Aber sie wollten uns keine Versammlung ge-
statten, noch wollten sie mit sich reden lassen. Jch sagte: »Habt
ihr nicht die Freunde Schmetterlinge genannt und gesagt, wir
könnten nicht in Städten leben; nun sind wir in eure Stadt gekommen,
und ihr wollt uns nicht hören. Wer sind nun die Schmetterlinge?«
Ledger sing an, die Sabbathheiligung zu verteidigen. Aber ich
erwiderte ihm, an dem Tage, welcher der Sabbath sei, dem siebenten
Wochentag, hielten sie ja Märkte und Jahrmärkte; während der Tag,
an dem sich die, welche sich jetzt Christen nennen, versammeln, ja
der erste Wochentag sei. Da wir keine öffentliche Versammlung
unter ihnen halten konnten, so veranstaltete ich zu Gateshead
eine kleine im Kreise der Freunde und solcher, die sich zu ihnen
hielten, und dort wird seither eine Versammlung abgehalten im
Namen Jesu. Ms ich über den Marktplatz ging, erfaßte mich
die Kraft des Herrn, und ich ermahnte das Volk, an den Tag
des Herrn zu denken, der über sie kommen werde. Und nicht
lange daraus wurden alle jene Priester von Neweastel rmd ihre
Anhänger Vertrieben, bei der Rückkehr des Königs ..... Von
Heatshead gingen wir nach Durham; es war einer von London
dorthin gekommen, um eine Schule zu errichten, worin »Prediger
Christi'', wie sie sich ausdrückten, ausgebildet werden sollten. Jch
ging zu ihm, um mit ihm zu reden und ihm zu zeigen, daß das
Lehren von Griechisch und Latein und der sieben schönen Künste
nur ein Belehren des natürlichen Menschen sei und nicht das


% \picinclude{./130-139/p_s135.jpg} 
Erste Jahreöversammlung. Warnung an Cromwell usw. 135
Mittel, die Leute zu Predigern Christi zu machen. Die Sprach-
verschiedenheit komme von Babel, und den Griechen, deren Mutter-
sprache griechisch war, war daß Wort vom Kreuz Torheit, und
den Juden, deren Sprache hebräisch war, mar Christuß ein
Stein deß Anstoßeß (1. Cor. 1, 23). Die Römer, die lateinisch
redeten, verfolgten die Christen; und Pilatuß, der römische
Machthaber, schrieb in hebräischer, griechischer und lateinischer
Sprache eine Jnschrift über daß Kreuz Christi; daran, sagte ich,
könne man sehen, daß die Sprachen von Babel kommen, da die
Jnschrist über daß Kreuz in diesen Sprachen geschrieben war.
Johanneß, der daß Wort verkündete, welcheß im Anfang war,
sagt, daß daß Tier und die Hure Macht haben über die
Zungen und Sprachen, welche dem Wasser gleich seien (Offb. 17);
man könne also sehen, daß daß Tier und die Hure diese Macht
haben über die Sprachen, die von der Verwirrung zu Babel her-
rühren. Die Verfolger Christi haben sie dann höher gestellt alß
ihn, alß sie ihn kreuzigten; aber darnach ist er auferstanden, höher
alß alleß andere, er, der vor allen gewesen ist. ,,Gedenkst du
nun,« fragte ich den Mann, ,,Prediger Christi zu bilden, vermittelst
dieser verwirrten äußeren Sprachen, die auß Babel kommen und
dort gut geheißen und von den Verfolgern Christi gebraucht
wurden!« Gr mußte daß zum Teil zugeben; idarauf zeigten
wir ihm weiter, daß Christuß seine Prediger selber lehrte, ihnen
Gaben erteilte und sie hieß, den Herrn der Ernte zu bitten, daß
er Arbeiter sende. Und Petruß und Johanneß, die doch in Sachen
der Schulweißheit unwissend und ungelehrt waren, verkündeten
Christuß, daß Wort, daß am Anfang war, also auch vor Babel. *
Auch Pauluß hat daß Evangelium nicht durch irgend einen
Menschen empfangen, sondern durch Jesuß, welcher auch jetzt
derselbe ist, und so ist auch sein Evangelium unverändert.« —.—’ Der
Priester hörte auf unß und errichtete seine Schule nicht.
Wir zogen über Warwickshire, Northamptonshire und
Leicestershire, wo wir überall viele Freunde aussuchten, nach
Bedforshire, zum Hause John Crookß, wo eine allgemeine Jahreß-
versammlung für daß ganze Land abgehalten wurde; sie dauerte
drei Tage, und die Freunde strömten auß dem ganzen Land herzu,
so daß alle Herbergen und Wohnungen in der Umgegend über-
füllt waren. Und trotz einiger Störungen durch einige böse Leute,
die von ddr Wahrheit abgefallen waren, kam doch die Kraft deß


% \picinclude{./130-139/p_s136.jpg} 
Herrn über alle, so daß wir eine herrliche Versammlung hatten;
das ewige Evangelium wurde gepredigt, und viele nahmen ez
aus, und Leben und unsterblicheß Wesen ging auf in allen und
schien über allen .....
Jch suchte noch da und dort etliche Freunde aus und kam
vom Herrn geleitet nach London, (1658) .....
Jch war noch nicht lange dort, als ich hörte, daß ein Jesuit,
der mit einem Gesandten von Spanien hergekommen war, alle
Quäker aufgefordert hatte, zu einer Dißputation in das-3 Hauö des
Earl von Newport zu kommen. Die Freunde antworteten, daß
etliche kommen werden. Darauf ließ er uns sagen, er wünsche
mit 12 unserer gelehrtesten und weisesten Leuten zu reden; einige
Zeit darauf ließ er sagen, eö möchten nur 6 kommen, und schließlich
nur 3. Wir beeilten uns?-, so viel wir konnten, damit ez nicht am
Ende nach so vieler Prahlerei heiße, eß solle gar niemand kommen.
Alß wir hinkamen, hieß ich Nicolaö Bond und Edward Burrough
hinausgehen, um die Unterredung zu eröffnen; ich wollte eine
Zeitlang mich unten im Hofe aufhalten und dann nachkommen.
Jch riet ihnen, ihm die Frage zu stellen, ob die römische Kirche,
sowie sie jetzt sei, nicht von der wahren Kirche der ersten Zeiten,
von ihrem Leben und ihrer Lehre, ihrer Kraft und ihrem Geist
abgesallen sei. Diese Frage legten sie denn auch dem Jesuiten si,
vor. Er erwiderte, die römische Kirche sei jetzt noch in der Jung-
fräulichkeit und Reinheit der ersten Kirche. Da kam ich dazu.
Wir frugen ihn, ob der heilige Geist über sie wie über die Apostel
auögegossen worden sei? Er antwortete: ,,nein«! ,,Dann,« sagte
ich, ,,wenn nicht derselbe Geist über euch außgegossen worden ist
und dieselbe Kraft wie über die Apostel, so seid ihr vom Geist
und der Kraft der ersten Kirche abgesallen; weiter braucht dann
nicht viel beigefügt zu werden.« Dann fragte ich ihn, auf maß
für Schriftftellen sie sich beriesen bei Errichtung von Nonnen- und
Mönch?-klöstern und Abteien für alle ihre verschiedenen Orden?
und beim Beten mit Rosenkränzen und zu Bildern, und ihrem
Bekreuzen und ihren Verboten wegen allerlei Speisen und beim
Heiraten und bei ihrem Hinrichten um des Glaubens willen?
,,Wenn ihr,« sagte ich, ,,die Gebräuche der ersten Kirche habt, in
ihrer Reinheit und Jungfräulichkeit, so zeiget untz Schristworte,
welche beweisen, daß sie solches getan.« Wir hatten nämlich
vorher gegenseitig abgemacht, daß wir unsere Behauptungen aus-


% \picinclude{./130-139/p_s137.jpg} 
Erste Jahreeversamnsslung. Warnung an Cromwell usw. 137
der Schrift beweisen sollten. Er sprach nun von einem geschrie-
benen und einem ungeschriebenen Wort. Jch fragte ihn, maß er
daß ungeschriebene Wort nenne? Er sagte, daß geschriebene
Wort sei die Schrift, daß ungeschriebene daß mündliche Wort der
Apostel, also alle Überlieferungen, nach denen sie wandeln. Jch
hieß ihn, daß auß der Schrift beweisen. Er kam nun mit der
Stelle, wo der Apostel, 2. Thess. 2, 5, sagt: ,,Gedenket ihr nicht
daran, daß ich euch solcheß sagte, alß ich noch bei euch war?''
»Damit,'' sagte der Jesuit, »meint der Apostel die Klöster und
daß Hinrichten um deß Glaubenß willen und daß Beten mit
Rosenkränzen und zu Bildern und waß noch mehr der Gebräuche
der römischen Kirche sind. Daß ist gemeint mit dem ,,ungeschrie-
benen Wort'', daß der Apostel damalß gesagt und daß sich seither
durch Überlieferung biß auf unsere Zeit erhalten hat.'' Ich hieß
ihn nun, dieseß Wort noch einmal lesen, damit er sehe, wie er eß
verdreht hatte. Denn daß, wovon er dort den Thessalonichern
sagt, ,,daß er ihnen zuvor gesagt'', ist nicht ein »ungeschriebeneß
Wort'', sondern eineß, daß geschrieben ist, nämlich daß »der Mensch
der Sünde, daß Kind deß Verderbenß geoffenbart werde, ehe der Tag
Ehristi kommen werde'' (2. Thess. 2, 3). Er redete also durchauß
hy nicht von den Gebräurhen der römischen Kirche. Ähnlich redete
  der Apostel im dritten Kapitel dieseß Briefeß, von »etlichen argen
Menschen, Vorwitzigen, die nichtß treiben''; um deretwillen hatte
der Apostel mündlich gesagt: »wer nicht arbeiten will, soll auch nicht
essen''; an daß erinnerte er sie nun schriftlich. Diese Schriftstelle
bewieß also nichtß für ihre erfundene Überlieferungen; und eine andere s
Stelle konnte er nicht bringen. Jch sagte ihm darum: ,,dieß ist
eine neue Verirrung eurer Kirche in Überlieferungen und Erfin-
dungen, wie die Apostel und die Heiligen der ersten Kirche sie
niemalß kannten.''
Er ging nun zum Altarsakrament über; er fing beim Pass alamm
und den Schaubroten an und kam zuletzt auf die Worte Jesu:
»Dieß ist mein Leib'', und auf daß, waß der Apostel darüber an
die Corinther schrieb. Er verstand die Stelle so, daß, nachdem der
Priester Brot und Wein geweiht habe, sei daßselbe göttlich und
un-vergänglich, und wer eß genieße, genieße Christuß selber. Jch
folgte ihm bei seiner Aufzählung der Bibelstellen, biß er zu den
Worten Christi und der Apostel kam. Da zeigte ich ihm, wie
derselbe Apostel den Corinthern auch nach ihrem Genuß von Brot und


% \picinclude{./130-139/p_s138.jpg} 
Wem ltsagt habe, sie seien Verworfene, wenn Ehristuö nicht in
ihneniti; wenn aber Brot und Wein, die sie genossen, Ehristuö
ielber Wären, so müßte er ja nun in ihnen sein. übrigen?-, wenn
Bjwt und Wein Christi Leib und Blut wären, wie könnte er dann
semtzts Leib im Himmel haben? Und zu dem haben die Jünger
Christi Leib essen und sein Blut trinken müssen zu seinem Ge-
dlikhkilit sowohl beim Abendmahl als auch nachher, biz daß er
ksms WW ja klar beweise, daß daz Brot und der Wein nicht
iem wirklicher Leib war, denn wenn sie seinen wirklichen Leib ge-
tzfjfsen hätten, so wäre er ja schon gegenwärtig gewesen, und man
hatte es nicht zu seinem Gedächtnis- zu tun brauchen .....
W Vaz die Worte Christi betresse: dietz ist mein Leib, so nenne
lich Christuß auch selbst einen Weinstock (Joh. 1,5) und eine Tür
lJ1gh.1O); und die Schrift nennt ihn einen Felß; ob er darum
i Tuizefiich ein Weinstock, eine Tür, ein Fels sei? »Oh«« sagte der
itesutischdiese Worte muß man auSlegen!« ,,So muß man auch
die Worte: ,,dieS ist mein Leib«, aus-legen«, antwortete ich.
Nachdem ich ihm so den Mund gestopft, machte ich ihm fol-
gmhtsl Vorschlag: ,,Wegen deiner Behauptung, daß Brot und
WM! göttlich und unoergänglich und leibhaftig Christus- seien und
daß leder, der sie genieße, Christus genieße, laßt eine Zusammen-
khsnst veranstalten zwischen einigen von euch, die der Papst und
die Kali-inäle bestimmen sollen, und einigen von uns, und laß eine
Flasche Wein und einen Laib Brot bringen und beide in zwei
Teileteilen und den einen dieser Teile weihen. Dann oerwahret
ionsohi den geweihten alß den ungeweihten Teil an einem sichern
OU, sind laßt sie gut bewachen; und macht den Versuch, ob das
geweshte Brot und der geweihte Wein nicht gerade so schnell
schlecht werden, wie daß; ungeweihte Brot und der ungeweihte
Wem- . . . und so wird die Wahrheit über diese Punkte offen-
bar werden. Wenn daß geweihte Brot und der geweihte Wein
isch nicht verändern, sondern schmackhaft und gut bleiben, so
werden dadurch viele für eure Kirche gewonnen werden; ver-
andem sie sich aber, so müßt ihr nachgeben, euren Jrrtum fahren
Essen Und kein Blut mehr darum vergießen. E6 ist schon viel
Nzuiharum vergossen worden, zum Beispiel zur Zeit der Königin
Stßlsss'' Hierauf erwiderte der Jesuit: ,,Nehmt ein Stück neuen
dan? und schneidet ihn in zwei Hälften, und machet zwei Röcke
Us- und zieht den einen Rock dem König David an und den


% \picinclude{./130-139/p_s139.jpg} 
Erste Jahresversammlung. Warnung an Cromwell usw. 139
andern einem Bettler; beide Röcke werden sich gleichermaßen ab-
tragen«. »Jft das deine Antwort«, stragte ich; ,,ja«, antwortete
er; ,,dann«, entgegnete ich, ,,werden alle Anwesenden überzeugt
sein, das; euer geweihter Wein und euer geweihtes Brot nicht
Christus ist. Habet ihr den Leuten solange oorgeschwatzt, der
geweihte Wein und das geweihte Brot sei Christus, und nun
sagst du, sie verbrauchen sich so gut wie die andern? Jch sage
dir: Christus ist derselbe gestern und heute und verfällt nicht;
sondern er ist der Heiligen hinimliche Speise jetzt und zu allen
Zeiten«. Hieraus erwiderte er nichts mehr, sondern hörte gerne
auf, denn die Anwesenden sahen, daß er im Jrrtum war und sich
nicht verteidigen konnte. Ich fragte ihn weiter, warum seine
Kirche die Leute um des Glaubens willen töte und verfolge?
Gr erwiderte: ,,nicht die Kirche tue das, sondern die Obrigkeit«.
Jch fragte ihn, ob denn diese Obrigkeit sich nicht auch zu den
Gläubigen und Christen zähle? Gr bejahte es. ,,Nun also«,
sagte ich, »stnd sie denn dann nicht Glieder eurer Kirche?« Er
antwortete: »ja«. Daraus überließ ich es den Anwesenden, selber
zu urteilen, ob die römische Kirche nicht die Leute um des Glaubens
willen verfolge und töte. Hiermit trennten wir uns. Seine
Spitzfmdigkeiten erklärten sich durch seine Dummheit.
Gs lag vieles auf mir während der Zeit, da ich in London
war, denn es war eine Zeit großer Not. Es trieb mich, an
Oliver Cromwell zu schreiben, und ihm die Bedrängnis der
Freunde, sowohl in England als auch in Jrland vorzustellen.
Das Gerücht jverbreitete sich damals, man wolle C-romwell zum
König machen. Da trieb es mich, zu ihm zu gehen und ihn da-
vor zu warnen, wie auch vor mancher anderen Gefahr, die seinen
und seiner Nachkommen Untergang herbeiführen würden, wenn
er sie nicht meide. Gr schien alles, was ich ihm sagte, gut aus-
zunehmen, und dankte mir dafür; dennoch trieb es mich, ihm nach-
her noch ausführlicher darüber zu schreiben .....
Um diese Zeit erkrankte Lady Elaypole (die -Lieblings-tochter
Oliver Cromwells), und war sehr niedergeschlagen, und niemand
konnte sie trösten; als ich davon hörte, trieb es mich, ihr folgendes
zu schreiben: —
,,Freundin!
Sei stille und ruhig in deinem Innern, und frei von eigenem
Denken; dann wirst du das Walten Gottes erfahren, wie es

% \picinclude{./140-149/p_s140.jpg} 
deine Sinne auf den Herrn lenkt, aus welchem das Leben
kommt; und dann wirst du seine Kraft an dir spüren, die dich
stark macht gegen alle Stürme und Unwetter. So allein wirst
du Geduld erlangen, Unschuld, Reinheit, Ruhe, Festigkeit und
Frieden in Gott. Darum läßt dich der Herr ermahnen, dich
seinem Willen zu unterwerfen und Glauben zu haben, damit du
das, was dich bedrückt, überwindest. .   . Jch ermahne alle,
in der Furcht des Herrn zu bleiben, damit ihr die Geheimnisse
Gottes erfahren möget und seine Weisheit und unter dem Schatten
des Allmächtigen sitzet in allen Gefahren und Stürmen. Denn
der Herr ist nahe, und der Höchste regieret die Menschenkinder.
Des Herrn Wort ergehet an alle, daß sie in allen Versuchungen
und Verwirrungen, welche das Licht an den Tag bringt, sich nicht
bei diesen Versuchungen und Schlechtigkeiten aufhalten, sondern
auch das Licht sehen, welches sie ausdeckt und an den Tag
bringt; und in diesem Lichte könnet ihr über dies alles steigen
und die Kraft empfangen, ihm entgegen zu treten.s Das gleiche
Licht, welches euch die Sünde erkennen läßt, zeigt euch auch den
Bund mit Gott, welcher eure Sünden tilgt und euch Sieg gibt
über sie.   Wenn ihr die Versuchung und das Schlechte ansehet,
so wird es euch mitreißen; wenn ihr aber zum Licht ausblickt,
welches die Sünde aufdeckt, so wird es euch überwinden helfen.
Gs wird euch den Sieg geben und ihr werdet Gnade und Kraft
von oben erfahren; dies ist der erste Schritt zum Frieden. Jhr
werdet das Heil erlangen und werdet die Herrlichkeit sehen, die
war, ehe der Welt Grund gelegt war; und dadurch werdet ihr
den Samen Gottes erkennen lernen, der das E-rbe der Verhei-
ßungen Gottes ist ..... Also stärke dich der Herr im Namen
Jesu Ehristi.« G. F.
Als diese Zeilen Lady Elaypole vorgelesen wurden, sagte sie,
es habe für den Augenblick ihren Geist gestärkt. Später ver-
schafsten sich viele Freunde in England und Jrland Abdrücke
davon und lasen es anderen, die niedergedrückt waren, vor; und
es hat manchem zur Ausrichtung geholfen.
Um diese Zeit erschien ein Aufruf von Oliver Cromwell für
eine Kollekte zur Unterstützung der aus Polen vertriebenen pro-
testantischen Gemeinden und für 20 aus Böhmen oerbannte
Familien. Schon einige Zeit vorher war ein ähnlicher Ausruf
erlassen worden, der zu einem feierlichen Fast- und Bettag auf-


% \picinclude{./140-149/p_s141.jpg} 
Erste Jahresvetsammlung. Warnung an Eromwell usw. 141
forderte, damit eine Kollekte gemacht würde zum Besten der not-
leidenden Protestanten in den Tälern von Lucerne, Angrona und
an anderen Orten, welche der Herzog von Savoyen verfolgte.
ES trieb mich, dem Protektor und den obersten Behörden bei
dieser Gelegenheit zu schreiben, um ihnen die Art des wahren
Fasten-?-, das Gott gefällt und von ihm angenommen wird, dar-
zulegen und ihnen zum Bewußtsein zu bringen, wie Unrecht sie
tun und sich selbst verdammen, wenn sie die Papiften tadeln,
daß sie die Protestanten in andern Ländern verfolgen, während
sie zu gleicher Zeit ihre protestantischen Nachbarn und die
Freunde im eignen Land verfolgen .....
Jch begab mich nun nach Hampton Court, um mit dem
Protektor über die Not der Freunde zu reden. Jch traf ihn auf
einem Ritt im Hampton Court Park. Ehe ich ihn an der Spitze
seiner Leibgarde erreichte, spürte ich einen Hauch des Todes
ihm entgegen huschen; und als ich zu ihm kam, sah er aus wie
ein Toter. Nachdem ich ihm die Not der Freunde beschrieben
hatte und nach meinem inneren Trieb ihn gewarnt hatte, hieß
er mich, zu ihm heim kommen. So ging ich denn am folgenden
Tage wieder nach Hampton Eourt, um nochmals mit ihm zu
reden. Aber als ich kam, hieß es, er sei krank, und Harn-ey,
einer seiner Bedienten, teilte mir mit, die Arzte wollten nicht,
daß ich mit ihm spreche. So ging ich fort und habe ihn nachher
nie mehr gesehen.
Von hier ging ich zu Jsaak Pennington in Buckinghamshire,
wo ich eine Versammlung angezeigt hatte; und des Herrn Wahr-
heit und Kraft wurden herrlich offenbar unter uns. Nachdem
ich manche Freunde in dieser Gegend besucht hattes, ging ich
nach London und bald darauf nach Essex; kaum war ich dort,
so hörte ich, der Protektor sei gestorben, und sein Sohn Richard
sei zum Protektor gemacht worden; nun kehrte ich wieder nach
London zurück.
Noch vor dieser Zeit war das sogenannte Kirchenbekenntnis 1)
veröffentlicht worden, von dem es hieß, er sei in der Zeit von
11 Tagen gemacht worden. Jch verschaffte mir vor der Ver-
öffentlichung eine Abschrift, und schrieb eine Antwort dazu, und
überall wo nim dieses Buch über ihr Bekenntnis verkauft wurde,
1) Die Savoydeklaration der Jndepedenten von 1658.


% \picinclude{./140-149/p_s142.jpg} 
wurde auch meine Antwort verkauft. Dieses- ärgerte etliche der
Parlamentßmitglieder, sodaß mir einer von ihnen mitteilte, ich
müßte nach Smithfield; ich antwortete ihm: ich stehe über ihrem
Feuer und fürchte sie nicht! Und ich stellte ihm weiter vor, ob
denn alle die vielen Völker seit 1600 Jahren ohne Glauben ge-
wesen seien, daß die Priester jetzt kommen müssen und ihnen einen
machen? ,,Sagte nicht der Apostel, daß Jesuß der Anfänger und
Vollender dez Glaubens war (Hebr. 12, 2)? [Und wenn nun
Christus der Anfänger deö Glaubens der Apostel war und dez
GlaubenZ der ersten Kirche in den ersten Zeiten und des Glaubenß
der Märtyrer, sollten nicht alle Menschen zu ihm aussehen alk-
dem Anfänger und Vollender ihres Glaubentz, und nicht zu den
Priestern?« Wir hatten viel Not mit diesem Vekenntniß. ....
Ich ging nach Reading, wo ich während etwa zehn Wochen
viel unter schwerer Niedergeschlagenheit und Trübsal zu leiden
hatte. Denn ich sah, wie viel Uneinigkeit und Verworren-
heit unter den Völkern herrschte und wie die Mächte suchten, sich
gegenseitig aufzufressen. Und ich sah, wie die Unschuld vernichtet
und die Wahrheit verleugnet wurde. Heuchelei, Betrug und
Streit gewannen die Oberhand, sodaß man [überall bereit war,
sich gegenseitig da-8 Schwert durch die Brust zu sstoßen. Viele
waren empsänglich gewesen, als- sie noch niedrig gewesen waren;
nachdem sie aber empor gekommen waren, und Macht erlangt
hatten und geholfen, andere zu töten, wurden sie bald so
schlecht wie die übrigen, so daß wir ost mit ihnen in Streit ge-
rieten wegen unsrer Hüte und wegen des »Du«-sagenß. Sie
kehrten ihre schaugestellte Geduld und Mäßigkeit in Zorn und
Ungeberdigkeit, und viele von ihnen taten wie Wahnsinnige wegen
dieser Hutehre. Denn sie waren durch die Verfolgung der Un-
schuld verhärtet worden, und kreuzigten nun den Samen,
Christuß, in sich und in andern; biz sie schließlich ansingen, sich
untereinander zu beißen und auszuzehren, nachdem sie daß, waö
Gott in ihnen hatte ausgehen lassen, beleidigt und zerstört hatten.
Darum stürzte sie Gott bald und machte die Hohen niedrig, und
stellte den König über die, die so ost behauptet hatten, die Quäker
kommen zusammen, um die Rückkehr König Karlß zu beraten,
während doch die Freunde sich nie um die äußern Mächte und
Regierungen bekümmert hatten. Zuletzt hat Gott ihn dann zu-
rückgebracht, und viele, alz sie sahen, daß er doch kommen werde,


% \picinclude{./140-149/p_s143.jpg} 
Erste Jahresversammlnng. Warnung an Cromwell usw. 143
stimmten sür sein Kommen. So preiset nun Gott mit Herz und
Mund, der die Herrschaft hat über alles .... Jch ahnte die
Rückkehr des Königs voraus, und so taten manche andere. Jch
schrieb mehrere Male an Oliver um ihm zu sagen, daß, während
er das Volk Gottes verfolge, seine Feinde sich rüsten, ihn zu
stürzen. Als einige Voreilige unter uns Somerset House kaufen
wollten, um Versammlungen drin zu halten, verbot ichs ihnen;
denn ich sah die Rückkehr des Königs voraus. Sodann kam eine
Frau zu mir, welche eine Vorahnung von der Rückkehr des
Königs gehabt hatte, drei Jahre, ehe er wirklich kam; und er-
klärte mir, sie müsse hingehen und es ihm sagen. Jch riet ihr,
das dem Herrn zu überlassen und es für sich zu behalten; denn
wenn es entdeckt würde, in welcher Angelegenheit sie hingehe, so
würde man es als Verrat ansehen; sie beharcte daraus, sie müsse
zu ihm gehen und ihm sagen, daß er wieder nach England zu-
rückkehren werde. Da erkannte ich, daß ihre Vorahnung sich
erfüllen werde; denn es mußte ein schwerer Schlag die treffen,
die damals so große Macht hatten und so harte Verfolgungen
ausübten; sie hielten sich für heilig und nahmen doch den Freunden
ihre rechtmäßigen Besitzungen, weil sie nicht schwören wollten.
Oft wenn wir Oliver diese Dinge berichteten, wollte er sie nicht
glauben. Darum trieb es Thomas Aldam und Anthony Pearson,
in alle Kerker von ganz England zu gehen und Auszeichnrmgen
zu machen darüber, wie die Freunde von den Kerkermeistern
behandelt wurden, damit sie die Größe ihrer Leiden Oliver vor-
bringen könnten. Und als er dennoch keinen Befehl geben wollte,
sie srei zu lassen, trieb es Thomas Aldam seine Mütze vom Kopf
zu nehmen, sie vor Olioers Augen in Stücke zu zerreißen und zu
rufen: ,,Also soll auch deine Herrschaft von dir und deinem Hause
gerissen werden«. Eine Frau, die auch zu den Freunden ge-
hörte, trieb es, ins Parlament zu gehen, welches den Freunden
übel wollte, mit einem Wg in der Hand, den sie vor ihnen zer-
schlug und rief: ,,so sollt ihr in Stücke zerschlagen werden!« was
auch bald darauf geschah. Während meiner großen Niederge-
schlagenheit und inneren Prüfung, die ich um meines Landes willen
zu erdulden hatte, weil die große Heuchelei, Falschheit und Ver-
räberei mich schwer drückte, sah ich, daß Gott die, welche jetzt
unten waren, über die, welche jetzt oben waren, erhöhen werde,
und daß alle sich dem, das sie bekehren konnte, zrmeigen mußten,


% \picinclude{./140-149/p_s144.jpg} 
ehe sie Herr werden würden über den bösen Geist, nach innen
und nach außen. Denn nur der eine unsichtbare Geist kann und
wird die Heuchelei in den Menschen vernichten .....
Das ganze Land war in Zwiespalt und großer Aufregung;
die verschiedenen Parteien zankten sich beständig untereinander
und rotteten sich gegeneinander zusammen, weil jede ihre eigenen
Jnteressen durchsetzen wollte. Da ich in großer Sorge war, daß
die Jungen und Unersahrenen unter uns diesen Versuchungen er-
liegen werden, trieb es mich, allen diesen folgendes zu schreiben:
,,Jhr Freunde allenthalben! Hütet euch vor Komplotten
und Wühlereien, und vor dem Arm des Fleisches, denn alle
diese Machthaber sind gefallene Söhne Adams; sie richten der
Menschen Leben zugrunde, wie Hunde, Schweine und andere
Tiere sich zugrunde richten, sich beißen und zerreißen. Wie ent-
stand das Streiten und Töten anders als aus der Lust? Und
dies alles kommt vom gefallenen Adam her, nicht von demjenigen h
Adam aber, der nicht fiel, in welchem Leben und Frieden ist
(1. Cor. 15). Jhr seid zum Frieden berufen, darum jaget ihm
nach, und dieser Frieden ist in Christus und nicht in dem gefallenen
Adam. Alle, die jetzt vorgeben, für Christus zu kämpfen, betrügen
sich; denn sein Reich ist nicht von dieser Welt; darum kämpfen
seine Diener nicht. Die Streitenden gehören nicht zu seinem
Reich, denn sein Reich ist Frieden und Gerechtigkeit .... Jhr,
die ihr Erben seid des Evangeliums des Friedens, welches ge-
wesen, ehe der Satan war, lebet in diesem Evangelium, suchet den
Frieden und das Gute für alle, und lebet in Christus, der ge-
kommen ist, die Seele der Menschen vom gefallenen Adam zu
erlösen; das äußere Schwert der Juden, mit dem sie die Heiden
umbrachten, war ein Sinnbild des inwendigen Geistes Gottes,
der die inwendige heidnische Natur tötet. So lebet denn im
friedsamen Reich Jesu Christi, im Frieden Gottes und nicht in
den Lüften, aus denen der Krieg entsteht .... und suchet das
Wohl und Gedeihen für alle Menschen.« G. F.
Bald darauf ergriff George Booth in Cheshire die Waffen
und Laniberts) zog gegen ihn; daraufhin wollten etliche Hitzköpfe,
wie solche zuweilen unter uns waren, auch die Waffen ergreifen,
aber der Herr trieb mich, sie zu ermahnen und ste blieben ruhig,
1) Lambert war einer der bedeutendsten Generäle aus der Partei Cromwells.


% \picinclude{./140-149/p_s145.jpg} 
Ein Gottesgericht. Erntahnung zur Barmherzigkeit usw. 145
Zur Zeit des sogenannten Sicherheit?-aus-schusses forderte man uns
auf, die Waffen zu nehmen und manchem von unß wurden hohe
Stellen und Kommandoß angeboten, aber wir schlugen sie alle aus
und traten mündlich und schriftlich dagegen auf, indem wir er-
klärten, unsere Waffen und Rüstungen seien nicht fleischlich, sondern
geistlich, und damit keiner unter unz in diese Falle gehe, kam es
über mich vom Herrn, bei dieser Gelegenheit einige Zeilen der
Ermahnung an alle zu schreiben .....
Nachdem ich längere Zeit in London geroeilt hatte, zog ich
wieder in den Grafschaften umher, durch Essex und Suffolk nach
Norwich .... und von da durch Huntingdomshire und Eam-
bridgeshire wieder nach London, gerade alö General Monk 1) dort
eingezogen war und die Tore und Befestigungen der Stadt
fielen. Lange vorher hatte ich ein Gesicht gehabt, in welchem
ich die Stadt in Trümmer und die Tore eingestürzt gesehen
hatte, gerade so, wie ich sie nun mehrere Jahre später nach dem
Brande sehen sollte.