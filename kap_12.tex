
% \picinclude{./130-139/p_s134.jpg} 
%%%%%%%%%%%%%%%%%%% Kapitel 12. %%%%%%%%%%%%%%%%%%%%%%%%%%%%%%

\chapter[Erste Jahresversammlung]{Erste Jahresversammlung}

\begin{center}
\textbf{Erste Jahresversammlung. Warnung an Cromwell vor 
der Königskrone. Trostbrief an dessen Tochter. Vorahnung 
vom Tode Cromwells und der kommenden Reaktion.}
\end{center}

\section{Gerüchte über Quaker}

Wir gingen zurück nach England [...] Die Priester
von Newcastel hatten verschiedene Bücher gegen uns geschrieben,
und ein Stadtältester, Ledger\person{Ledger}, war uns und der 
Wahrheit sehr  abgeneigt. Er sowie die Priester hatten 
behauptet, die Quäker könnten
nicht in einer Stadt leben, sondern schwirrten wie die Schmetterlinge
in den Hochtälern. \index{Gerüchte!über Quaker} Zu diesem 
Ledger und einigen andern Stadtältesten ging ich, mit 
Anthony Pearson,\person{Pearson, Anthony} mit der Bitte, eine 
Versammlung in Newcastel\ort{Newcastel} abhalten zu dürfen, 
nachdem sie so viel
gegen uns geschrieben haben, da wir nun ja in ihre grose Stadt
gekommen seien! Aber sie wollten uns keine Versammlung gestatten, 
noch wollten sie mit sich reden lassen.\index{Versammlungsverbot} 
Ich sagte: \zitat{Habt
ihr nicht die Freunde Schmetterlinge genannt und gesagt, wir
könnten nicht in Städten leben; nun sind wir in eure Stadt gekommen,
und ihr wollt uns nicht hören. Wer sind nun die Schmetterlinge?}
Ledger fing an, die Sabbathheiligung \index{Sabbathheiligung} 
\index{Sonntag} zu verteidigen. Aber ich
erwiderte ihm, an dem Tage, welcher der Sabbath sei, dem siebenten
Wochentag, hielten sie ja Märkte und Jahrmärkte; während der Tag,
an dem sich die, welche sich jetzt Christen nennen, versammeln, ja
der erste Wochentag sei. 


Da wir keine öffentliche Versammlung
unter ihnen halten konnten, so veranstaltete 
ich zu Gateshead\ort{Gateshead}
eine kleine im Kreise der Freunde und solcher, die sich zu ihnen
hielten, und dort wird seither eine Versammlung abgehalten im
Namen Jesu. Als ich über den Marktplatz ging, erfasste mich
die Kraft des Herrn, und ich ermahnte das Volk, an den Tag
des Herrn zu denken, der über sie kommen 
werde.\index{Perdigt!spontan} Und nicht
lange darauf wurden alle jene Priester von Newcastel und ihre
Anhänger Vertrieben, bei der Rückkehr des Königs [...] 

\section{Fox kritisiert das Theologiestudium}

Von Heatshead gingen wir nach Durham; es war einer von London
dorthin gekommen, um eine Schule zu errichten, worin \zitat{Prediger
Christi}, wie sie sich ausdrückten, ausgebildet werden sollten. 
\index{Theologie!Studium} Ich
ging zu ihm, um mit ihm zu reden und ihm zu zeigen, das das
Lehren von Griechisch und Latein und der sieben schönen Künste
nur ein Belehren des natürlichen Menschen sei und nicht das
% \picinclude{./130-139/p_s135.jpg} 
Mittel, die Leute zu Predigern Christi zu machen. 
\index{Studium!Latein}\index{Studium!Griechisch}
\index{Studium!schönen Künste} \index{Studium!Hebräisch} 
\index{Sprache!Latein}\index{Sprache!Griechisch}
\index{Sprache!Hebräisch}Die Sprachverschiedenheit 
komme von Babel, und den Griechen, deren Muttersprache 
griechisch war, war das Wort vom Kreuz Torheit, und
den Juden, deren Sprache hebräisch war, war Christus ein
Stein des Anstoßes 
(1.~Cor.~1:23\bibel{Cor. 1. 01:23@1. Cor. 1:23}).
Die Römer, die lateinisch
redeten, verfolgten die Christen; und Pilatus, der römische
Machthaber, schrieb in hebräischer, griechischer und lateinischer
Sprache eine Inschrift über das Kreuz Christi; daran, sagte ich,
könne man sehen, das die Sprachen von Babel kommen, da die
Inschrift über das Kreuz in diesen Sprachen geschrieben war.
Johannes, der das Wort verkündete, welches im Anfang war,
sagt, das das Tier und die Hure Macht haben über die
Zungen und Sprachen, welche dem Wasser gleich seien 
(Offb. 17\bibel{Offb. 17});
man könne also sehen, das das Tier und die Hure diese Macht
haben über die Sprachen, die von der Verwirrung zu Babel 
herrühren. Die Verfolger Christi haben sie dann höher gestellt als
ihn, als sie ihn kreuzigten; aber danach ist er auferstanden, höher
als alles andere, er, der vor allen gewesen ist. \zitat{Gedenkst du
nun,} fragte ich den Mann, \zitat{Prediger Christi zu bilden, 
vermittelst dieser verwirrten äußeren Sprachen, die aus Babel 
kommen und dort gut geheißen und von den Verfolgern Christi gebraucht
wurden!} Er musste das zum Teil zugeben; hierauf zeigten
wir ihm weiter, das Christus seine Prediger selber lehrte, ihnen
Gaben erteilte und sie hieß, den Herrn der Ernte zu bitten, das
er Arbeiter sende. Und Petrus und Johannes, die doch in Sachen
der Schulweisheit\index{Schulweisheit} unwissend und 
ungelehrt waren, verkündeten
Christus, das Wort, das am Anfang war, also auch vor Babel.

\section{Fox bei der Jahresversammlung in Bedforshire}

Auch Paulus hat das Evangelium nicht durch irgend einen
Menschen empfangen, sondern durch Jesus, welcher auch jetzt
derselbe ist, und so ist auch sein Evangelium unverändert. Der
Priester hörte auf uns und errichtete seine Schule nicht.
Wir zogen über Warwickshire\ort{Warwickshire}, 
Northamptonshire\ort{Northamptonshire} und
Leicestershire\ort{Leicestershire}, wo wir überall viele Freunde 
aussuchten, nach Bedforshire\ort{Bedforshire}, zum Hause 
John Crooks,\person{Crooks, John} wo eine allgemeine 
Jahresversammlung\index{Jahresversammlung} für das ganze 
Land abgehalten wurde; sie dauerte
drei Tage, und die Freunde strömten aus dem ganzen Land herzu,
so das alle Herbergen und Wohnungen in der Umgegend überfüllt 
waren. Und trotz einiger Störungen durch einige böse Leute,
die von der Wahrheit abgefallen waren, kam doch die Kraft des
% \picinclude{./130-139/p_s136.jpg} 
Herrn über alle, so das wir eine herrliche Versammlung hatten;
das ewige Evangelium wurde gepredigt, und viele nahmen es
auf, und Leben und unsterbliches Wesen ging auf in allen und
schien über allen [...].

\section{Disput mit einen Jesuiten über die Eucharestie und Glaubenskriege}

Ich suchte noch da und dort etliche Freunde auf und kam
vom Herrn geleitet nach London,\ort{London} (1658\jahr{1658}) [...].
Ich war noch nicht lange dort, als ich hörte, das 
ein Jesuit,\index{Jesuiten}
der mit einem Gesandten von Spanien hergekommen war, alle
Quäker aufgefordert hatte, zu einer Disputation\index{Disput} 
in das Haus des Earl von Newport\person{Earl von Newport} 
zu kommen. Die Freunde antworteten, das
etliche kommen werden. Darauf lies er uns sagen, er wünsche
mit 12 unserer gelehrtesten und weisesten Leuten zu reden; einige
Zeit darauf lies er sagen, es möchten nur 6 kommen, und schließlich
nur 3. Wir beeilten uns, so viel wir konnten, damit es nicht am
Ende nach so vieler Prahlerei heiße, es solle gar niemand kommen.



Als wir hin kamen, hieß ich Nicolas Bond\person{Bond, Nicolas} 
und Edward Burrough\person{Burrough, Edward}
hinausgehen, um die Unterredung zu eröffnen; ich wollte eine
Zeitlang mich unten im Hofe aufhalten und dann nachkommen.
Ich riet ihnen, ihm die Frage zu stellen, ob die römische Kirche,
sowie sie jetzt sei, nicht von der wahren Kirche der ersten Zeiten,
von ihrem Leben und ihrer Lehre, ihrer Kraft und ihrem Geist
abgefallen sei. Diese Frage legten sie denn auch dem Jesuiten sie,
vor. Er erwiderte, die römische Kirche sei jetzt noch in der 
Jungfräulichkeit und Reinheit der ersten Kirche. Da kam ich dazu.
Wir frugen ihn, ob der heilige Geist über sie wie über die Apostel
ausgegossen worden sei? Er antwortete: \zitat{nein!} \zitat{Dann,} sagte
ich, \zitat{wenn nicht derselbe Geist über euch ausgegossen worden ist
und dieselbe Kraft wie über die Apostel, so seid ihr vom Geist
und der Kraft der ersten Kirche abgefallen; weiter braucht dann
nicht viel beigefügt zu werden.} Dann fragte ich ihn, auf was
für Schriftstellen sie sich beriefen bei Errichtung von Nonnen- und
Mönchs-klöstern und Abteien für alle ihre verschiedenen Orden?
und beim Beten mit Rosenkränzen und zu Bildern, und ihrem
Bekreuzen und ihren Verboten wegen allerlei Speisen und beim
Heiraten und bei ihrem Hinrichten um des Glaubens willen?
\zitat{Wenn ihr,} sagte ich, \zitat{die Gebräuche der 
ersten Kirche habt, in
ihrer Reinheit und Jungfräulichkeit, so zeiget uns Schriftworte,
welche beweisen, das sie solches getan.} Wir hatten nämlich
vorher gegenseitig abgemacht, das wir unsere Behauptungen aus-
% \picinclude{./130-139/p_s137.jpg} 
der Schrift beweisen sollten.\index{Absprachen} Er sprach nun 
von einem geschriebenen und einem ungeschriebenen Wort. Ich 
fragte ihn, was er
das ungeschriebene Wort nenne? Er sagte, das geschriebene
Wort sei die Schrift, das ungeschriebene das mündliche Wort der
Apostel, also alle Überlieferungen, nach denen sie wandeln. 
\index{Ungeschriebene Überlieferung} Ich
hieß ihn, das aus der Schrift beweisen. Er kam nun mit der
Stelle, wo der Apostel, 2. Thess. 2:5,
\bibel{Thess. 2. 02:05@2. Thess. 2:5} sagt: \zitat{Gedenket ihr nicht
daran, das ich euch solches sagte, als ich noch bei euch war?}.
\zitat{Damit,} sagte der Jesuit, \zitat{meint der Apostel 
die Klöster und
das Hinrichten um des Glaubens willen und das Beten mit
Rosenkränzen und zu Bildern und was noch mehr der Gebräuche
der römischen Kirche sind. Das ist gemeint mit dem ungeschriebenen 
Wort, das der Apostel damals gesagt und das sich seither
durch Überlieferung bis auf unsere Zeit erhalten hat.} Ich hieß
ihn nun, dieses Wort noch einmal lesen, damit er sehe, wie er es
verdreht hatte. Denn das, wovon er dort den Thessalonichern
sagt, \zitat{das er ihnen zuvor gesagt}, ist nicht ein 
\zitat{ungeschriebenes Wort}, sondern eines, das geschrieben 
ist, nämlich das \zitat{der Mensch
der Sünde, das Kind des Verderbens geoffenbart werde, ehe der Tag
Christi kommen werde} (Thess. 2. 02:03@2. Thess. 2:3). 
Er redete also durchaus nicht von den Gebräuchen der römischen Kirche.
\index{Römischen Kirche} \index{Katholische Kirche} Ähnlich redete
der Apostel im dritten Kapitel dieses Briefes, von \zitat{etlichen argen
Menschen, Vorwitzigen, die nichts treiben}; um deretwillen hatte
der Apostel mündlich gesagt: \zitat{wer nicht arbeiten will, 
soll auch nicht essen}; an das erinnerte er sie nun schriftlich. 
Diese Schriftstelle
bewies also nichts für ihre erfundene Überlieferungen; und eine andere
Stelle konnte er nicht bringen. Ich sagte ihm darum: \zitat{dies ist
eine neue Verirrung eurer Kirche in Überlieferungen und Erfindungen, 
wie die Apostel und die Heiligen der ersten Kirche sie
niemals kannten.}

\index{Abendmahl}\index{Eucharestie}
Er ging nun zum Altarsakrament über; er fing beim Passalamm
und den Schaubroten an und kam zuletzt auf die Worte Jesu:
\zitat{Dies ist mein Leib}, und auf das, was der Apostel darüber an
die Corinther schrieb. Er verstand die Stelle so, das, nachdem der
Priester Brot und Wein geweiht habe, sei dasselbe göttlich und
unvergänglich, und wer es genieße, genieße Christus selber. Ich
folgte ihm bei seiner Aufzählung der Bibelstellen, bis er zu den
Worten Christi und der Apostel kam. Da zeigte ich ihm, wie
derselbe Apostel den Corinthern auch nach ihrem Genus von Brot und
% \picinclude{./130-139/p_s138.jpg} 
Wem gesagt habe, sie seien Verworfene, wenn Christus nicht in
ihnen sei; wenn aber Brot und Wein, die sie genossen, Christus
Leiber Wären, so müsste er ja nun in ihnen sein. übrigens, wenn
Brot und Wein Christi Leib und Blut wären, wie könnte er dann
seinen Leib im Himmel haben? Und zu dem haben die Jünger
Christi Leib essen und sein Blut trinken müssen zu seinem 
Gedächtnis sowohl beim Abendmahl als auch nachher, bis das er
ja klar beweise, das dass Brot und der Wein nicht
sein wirklicher Leib war, denn wenn sie seinen wirklichen Leib 
gegessen hätten, so wäre er ja schon gegenwärtig gewesen, und man
hatte es nicht zu seinem Gedächtnis zu tun brauchen [...].


Was die Worte Christi betreffe: dies ist mein Leib, so nenne
sich Christus auch selbst einen Weinstock 
(Joh. 1:5\bibel{Joh. 01:05@Joh. 1:5}) und eine Tür
(Joh. 1O\bibel{Joh. 1O}); und die Schrift nennt ihn einen Fels; 
ob er darum
äußerlich ein Weinstock, eine Tür, ein Fels sei?  \zitat{Oh} sagte der
Jesuit \zitat{diese Worte mus man auslegen!} \zitat{So mus man auch
die Worte: \emph{die ist mein Leib}, auslegen}, antwortete ich.


Nachdem ich ihm so den Mund gestopft,\person{Fox!Kraftausdruck} 
machte ich ihm folgenden 
Vorschlag: \zitat{Wegen deiner Behauptung, das Brot und
Wein göttlich und unvergänglich und leibhaftig Christus seien und
das jeder, der sie genieße, Christus genieße, last eine 
Zusammenkunft veranstalten zwischen einigen von euch, die der Papst und
die Kardinäle bestimmen sollen, und einigen von uns, und las eine
Flasche Wein und einen Laib Brot bringen und beide in zwei
Teileteilen und den einen dieser Teile weihen. Dann verwahret
sowohl den geweihten als den ungeweihten Teil an einem sichern
Ort, und last sie gut bewachen; und macht den Versuch, ob das
geweihte Brot und der geweihte Wein nicht gerade so schnell
schlecht werden, wie das; ungeweihte Brot und der ungeweihte
Wein [... ] und so wird die Wahrheit über diese Punkte 
offenbar werden. Wenn das geweihte Brot und der geweihte Wein
sich nicht verändern, sondern schmackhaft und gut bleiben, so
werden dadurch viele für eure Kirche gewonnen werden; verändern 
sie sich aber, so müsst ihr nachgeben, euren Irrtum fahren
Essen Und kein Blut mehr darum vergießen. Es ist schon viel
Blut darum vergossen worden, zum Beispiel zur Zeit der Königin
Maria}\person{Königin Maria} Hierauf erwiderte der Jesuit: 
\zitat{Nehmt ein Stück neuen
Stoff und schneidet ihn in zwei Hälften, und machet zwei Röcke
daraus und zieht den einen Rock dem König David an und den
% \picinclude{./130-139/p_s139.jpg} 
andern einem Bettler; beide Röcke werden sich gleichermaßen 
abtragen}. \zitat{Ist das deine Antwort}, fragte ich; \zitat{ja}, 
antwortete er; \zitat{dann}, entgegnete ich, \zitat{werden 
alle Anwesenden überzeugt
sein, das; euer geweihter Wein und euer geweihtes Brot nicht
Christus ist. Habet ihr den Leuten solange 
vorgeschwatzt,\person{Fox!Kraftausdruck} der
geweihte Wein und das geweihte Brot sei Christus, und nun
sagst du, sie verbrauchen sich so gut wie die andern? Ich sage
dir: Christus ist derselbe gestern und heute und verfällt nicht;
sondern er ist der Heiligen himmlische Speise jetzt und zu allen
Zeiten}. Hieraus erwiderte er nichts mehr, sondern hörte gerne
auf, denn die Anwesenden sahen, das er im Irrtum war und sich
nicht verteidigen konnte. 


Ich fragte ihn weiter, warum seine \index{Glaubenskrieg}
Kirche die Leute um des Glaubens willen töte und verfolge?
Er erwiderte: \zitat{nicht die Kirche tue das, sondern die Obrigkeit}.
Ich fragte ihn, ob denn diese Obrigkeit sich nicht auch zu den
Gläubigen und Christen zähle? Er bejahte es. \zitat{Nun also},
sagte ich, \zitat{sind sie denn dann nicht Glieder eurer Kirche?}
\index{Ekklesiologie} Er
antwortete: \zitat{ja}. Daraus überließ ich es den Anwesenden, selber
zu urteilen, ob die römische Kirche nicht die Leute um des Glaubens
willen verfolge und töte. Hiermit trennten wir uns. Seine
Spitzfindigkeiten erklärten sich durch seine Dummheit.

\section{Trostbrief an Lady Elaypole}

Es lag vieles auf mir während der Zeit, da ich in London\ort{London}
war, denn es war eine Zeit großer Not. Es trieb mich, an
Oliver Cromwell\person{Cromwell, Oliver} zu schreiben, 
und ihm die Bedrängnis der Freunde, sowohl in England\ort{England} 
als auch in Irland\ort{Irland} vorzustellen.
Das Gerücht verbreitete sich damals, man wolle Cromwell zum
König machen. Da trieb es mich, zu ihm zu gehen und ihn davor 
zu warnen, wie auch vor mancher anderen Gefahr, die seinen
und seiner Nachkommen Untergang herbeiführen würden, wenn
er sie nicht meide. Er schien alles, was ich ihm sagte, gut 
aufzunehmen, und dankte mir dafür; dennoch trieb es mich, ihm 
nachher noch ausführlicher darüber zu schreiben [...].


Um diese Zeit erkrankte Lady Elaypole
\person{Elaypole, Tochter Cromwells} (die Lieblingstochter 
Oliver Cromwells), und 
war sehr niedergeschlagen, und niemand konnte sie trösten; als 
ich davon hörte, trieb es mich, ihr folgendes
zu schreiben:

\brief{An Lady Elaypole, Tochter Cromwells}{
Freundin!

\bigskip

Sei stille und ruhig in deinem Innern, und frei von eigenem
Denken; dann wirst du das Walten Gottes erfahren, wie es
% \picinclude{./140-149/p_s140.jpg} 
deine Sinne auf den Herrn lenkt, aus welchem das Leben
kommt; und dann wirst du seine Kraft an dir spüren, die dich
stark macht gegen alle Stürme und Unwetter. So allein wirst
du Geduld erlangen, Unschuld, Reinheit, Ruhe, Festigkeit und
Frieden in Gott. Darum lässt dich der Herr ermahnen, dich
seinem Willen zu unterwerfen und Glauben zu haben, damit du
das, was dich bedrückt, überwindest. [...]. 

\index{Inneres Licht}
Ich ermahne alle,
in der Furcht des Herrn zu bleiben, damit ihr die Geheimnisse
Gottes erfahren möget und seine Weisheit und unter dem Schatten
des Allmächtigen sitzet in allen Gefahren und Stürmen. Denn
der Herr ist nahe, und der Höchste regieret die Menschenkinder.
Des Herrn Wort ergehet an alle, das sie in allen Versuchungen
und Verwirrungen, welche das Licht an den Tag bringt, sich nicht
bei diesen Versuchungen und Schlechtigkeiten aufhalten, sondern
auch das Licht sehen, welches sie aufdeckt und an den Tag
bringt; und in diesem Lichte könnet ihr über dies alles steigen
und die Kraft empfangen, ihm entgegen zu treten. Das gleiche
Licht, welches euch die Sünde erkennen lässt, zeigt euch auch den
Bund mit Gott, welcher eure Sünden\index{Sünde} tilgt und euch Sieg gibt
über sie.   Wenn ihr die Versuchung und das Schlechte ansehet,
so wird es euch mitreisen; wenn ihr aber zum Licht ausblickt,
welches die Sünde aufdeckt, so wird es euch überwinden helfen.
Es wird euch den Sieg geben und ihr werdet Gnade und Kraft
von oben erfahren; dies ist der erste Schritt zum Frieden. Ihr
werdet das Heil erlangen und werdet die Herrlichkeit sehen, die
war, ehe der Welt Grund gelegt war; und dadurch werdet ihr
den Samen Gottes erkennen lernen, der das Erbe der 
Verheißungen Gottes ist [...]. 

\bigskip

Also stärke dich der Herr im Namen Jesu Christi.


\begin{flushright}G. F.\end{flushright}
}


Als diese Zeilen Lady Elaypole vorgelesen wurden, sagte sie,
es habe für den Augenblick ihren Geist gestärkt. Später 
verschafften sich viele Freunde in England und Irland Abdrücke
davon und lasen es anderen, die niedergedrückt waren, vor; und
es hat manchem zur Aufrichtung geholfen.

\section{Letzte Begegnung mit Oliver Cromwell}

Um diese Zeit erschien ein Aufruf von Oliver 
Cromwell\person{Cromwell, Oliver} für
eine Kollekte zur Unterstützung der aus Polen\ort{Polen} vertriebenen 
protestantischen Gemeinden \index{Protestanten} und 
für 20 aus Böhmen verbannte
Familien. Schon einige Zeit vorher war ein ähnlicher Ausruf
erlassen worden, der zu einem feierlichen Fast- und Bettag 
% \picinclude{./140-149/p_s141.jpg} 
aufforderte, damit eine Kollekte gemacht würde zum Besten der 
notleidenden Protestanten in den Tälern von 
Lucerne,\ort{Lucerne} Angrona\ort{Angrona} und
an anderen Orten, welche der Herzog von 
Savoyen\person{Herzog von Savoyen} verfolgte.
Es trieb mich, dem Protektor und den obersten Behörden bei
dieser Gelegenheit zu schreiben, um ihnen die Art des wahren
Fastens, das Gott gefällt und von ihm angenommen wird, 
darzulegen und ihnen zum Bewusstsein zu bringen, wie Unrecht sie
tun und sich selbst verdammen, wenn sie die 
Papisten\index{Papisten} tadeln,
das sie die Protestanten in andern Ländern 
verfolgen,\index{Vervolgung} während
sie zu gleicher Zeit ihre protestantischen Nachbarn und die
Freunde im eignen Land verfolgen [...].


Ich begab mich nun nach Hampton Court, um mit dem
Protektor über die Not der Freunde zu reden. Ich traf ihn auf
einem Ritt im Hampton Court Park.\ort{Hampton Court Park} 
Ehe ich ihn an der Spitze
seiner Leibgarde erreichte, spürte ich einen Hauch des Todes
ihm entgegen huschen; und als ich zu ihm kam, sah er aus wie
ein Toter. Nachdem ich ihm die Not der Freunde beschrieben
hatte und nach meinem inneren Trieb ihn gewarnt hatte, hieß
er mich, zu ihm heim kommen. So ging ich denn am folgenden
Tage wieder nach Hampton Court, um nochmals mit ihm zu
reden. Aber als ich kam, hieß es, er sei krank, und Harwey,
einer seiner Bedienten, teilte mir mit, die Ärzte wollten nicht,
das ich mit ihm spreche. So ging ich fort und habe ihn nachher
nie mehr gesehen.

\section{Das Kirchenbekenntnis}


Von hier ging ich zu Isaak Pennington\person{Pennington, Isaak} 
in Buckinghamshire, \ort{Buckinghamshire}
wo ich eine Versammlung angezeigt hatte; und des Herrn 
Wahrheit und Kraft wurden herrlich offenbar unter uns. Nachdem
ich manche Freunde in dieser Gegend besucht hatte, ging ich
nach London\ort{London} und bald darauf nach 
Essex;\ort{Essex} kaum war ich dort,
so hörte ich, der Protektor sei gestorben, und sein Sohn 
Richard\person{Cromwell, Richard}
sei zum Protektor gemacht worden; nun kehrte ich wieder nach
London zurück.


Noch vor dieser Zeit war das sogenannte 
Kirchenbekenntnis\footnote{Die Savoydeklaration der 
Indepedenten von 1658.}
veröffentlicht worden, von dem es hieß, er sei in der Zeit von
11 Tagen gemacht worden. Ich verschaffte mir vor der 
Veröffentlichung eine Abschrift, und schrieb eine Antwort dazu, und
überall wo nun dieses Buch über ihr Bekenntnis verkauft wurde,
% \picinclude{./140-149/p_s142.jpg} 
wurde auch meine Antwort verkauft. \index{Kontroverse} 
\index{Erwiederungsschrift}

Dieses ärgerte etliche der
Parlamentsmitglieder, so das mir einer von ihnen mitteilte, ich
müsste nach Smithfield;\ort{Smithfield} ich antwortete ihm: 
ich stehe über ihrem
Feuer und fürchte sie nicht! Und ich stellte ihm weiter vor, ob
denn alle die vielen Völker seit 1600 Jahren ohne Glauben 
gewesen seien, das die Priester jetzt kommen müssen und ihnen einen
machen? \zitat{Sagte nicht der Apostel, das Jesus der Anfänger und
Vollender des Glaubens war 
(Hebr. 12:2\bibel{Hebr. 12:02@Hebr. 12:2})? Und wenn nun
Christus der Anfänger des Glaubens der Apostel war und des
Glaubens der ersten Kirche\index{Kirche!erste} in den 
ersten Zeiten und des Glaubens
der Märtyrer\index{Märtyrer}, sollten nicht alle Menschen 
zu ihm aussehen als
dem Anfänger und Vollender ihres Glaubents, und nicht zu den
Priestern?} Wir hatten viel Not mit diesem Vekenntnis. [...]

\section{Ärger mit Emporkömmlingen}

Ich ging nach Reading,\ort{Reading} wo ich während etwa zehn Wochen
viel unter schwerer Niedergeschlagenheit und Trübsal zu leiden
hatte. Denn ich sah, wie viel Uneinigkeit und Verworrenheit 
unter den Völkern herrschte und wie die Mächte suchten, sich
gegenseitig aufzufressen. Und ich sah, wie die Unschuld vernichtet
und die Wahrheit verleugnet wurde. Heuchelei, Betrug und
Streit gewannen die Oberhand, so das man überall bereit war,
sich gegenseitig das Schwert durch die Brust zu stoßen. Viele
waren empfänglich gewesen, als sie noch niedrig gewesen waren;
nachdem sie aber empor gekommen waren, und Macht erlangt
hatten und geholfen, andere zu töten, wurden sie bald so
schlecht wie die übrigen, so das wir oft mit ihnen in Streit 
gerieten wegen unsrer Hüte\index{Hut} und wegen des 
\zitat{Du}~--~sagens. Sie
kehrten ihre schaugestellte Geduld und Mäßigkeit in Zorn und
Ungeberdigkeit, und viele von ihnen taten wie Wahnsinnige wegen
dieser Hutehre. Denn sie waren durch die Verfolgung der 
Unschuld verhärtet worden, und kreuzigten nun den Samen,
Christus,\index{Christus!innerlich kreuzigen} in sich und 
in andern; bis sie schließlich anfingen, sich
untereinander zu beißen und auszuzehren, nachdem sie das, waö
Gott in ihnen hatte ausgehen lassen, beleidigt und zerstört hatten.
Darum stürzte sie Gott bald und machte die Hohen niedrig, und
stellte den König über die, die so ost behauptet hatten, die Quäker
kommen zusammen, um die Rückkehr König Karls zu beraten,
während doch die Freunde sich nie um die äusern Mächte und
Regierungen bekümmert hatten. Zuletzt hat Gott ihn dann zu-
rückgebracht, und viele, alz sie sahen, das er doch kommen werde,


% \picinclude{./140-149/p_s143.jpg} 
Erste Jahresversammlnng. Warnung an Cromwell usw. 143
stimmten sür sein Kommen. So preiset nun Gott mit Herz und
Mund, der die Herrschaft hat über alles .... Jch ahnte die
Rückkehr des Königs voraus, und so taten manche andere. Jch
schrieb mehrere Male an Oliver um ihm zu sagen, das, während
er das Volk Gottes verfolge, seine Feinde sich rüsten, ihn zu
stürzen. Als einige Voreilige unter uns Somerset House kaufen
wollten, um Versammlungen drin zu halten, verbot ichs ihnen;
denn ich sah die Rückkehr des Königs voraus. Sodann kam eine
Frau zu mir, welche eine Vorahnung von der Rückkehr des
Königs gehabt hatte, drei Jahre, ehe er wirklich kam; und er-
klärte mir, sie müsse hingehen und es ihm sagen. Jch riet ihr,
das dem Herrn zu überlassen und es für sich zu behalten; denn
wenn es entdeckt würde, in welcher Angelegenheit sie hingehe, so
würde man es als Verrat ansehen; sie beharcte daraus, sie müsse
zu ihm gehen und ihm sagen, das er wieder nach England zu-
rückkehren werde. Da erkannte ich, das ihre Vorahnung sich
erfüllen werde; denn es muste ein schwerer Schlag die treffen,
die damals so grose Macht hatten und so harte Verfolgungen
ausübten; sie hielten sich für heilig und nahmen doch den Freunden
ihre rechtmäsigen Besitzungen, weil sie nicht schwören wollten.
Oft wenn wir Oliver diese Dinge berichteten, wollte er sie nicht
glauben. Darum trieb es Thomas Aldam und Anthony Pearson,
in alle Kerker von ganz England zu gehen und Auszeichnrmgen
zu machen darüber, wie die Freunde von den Kerkermeistern
behandelt wurden, damit sie die Gröse ihrer Leiden Oliver vor-
bringen könnten. Und als er dennoch keinen Befehl geben wollte,
sie srei zu lassen, trieb es Thomas Aldam seine Mütze vom Kopf
zu nehmen, sie vor Olioers Augen in Stücke zu zerreisen und zu
rufen: ,,Also soll auch deine Herrschaft von dir und deinem Hause
gerissen werden«. Eine Frau, die auch zu den Freunden ge-
hörte, trieb es, ins Parlament zu gehen, welches den Freunden
übel wollte, mit einem Wg in der Hand, den sie vor ihnen zer-
schlug und rief: ,,so sollt ihr in Stücke zerschlagen werden!« was
auch bald darauf geschah. Während meiner grosen Niederge-
schlagenheit und inneren Prüfung, die ich um meines Landes willen
zu erdulden hatte, weil die grose Heuchelei, Falschheit und Ver-
räberei mich schwer drückte, sah ich, das Gott die, welche jetzt
unten waren, über die, welche jetzt oben waren, erhöhen werde,
und das alle sich dem, das sie bekehren konnte, zrmeigen musten,


% \picinclude{./140-149/p_s144.jpg} 
ehe sie Herr werden würden über den bösen Geist, nach innen
und nach ausen. Denn nur der eine unsichtbare Geist kann und
wird die Heuchelei in den Menschen vernichten .....
Das ganze Land war in Zwiespalt und groser Aufregung;
die verschiedenen Parteien zankten sich beständig untereinander
und rotteten sich gegeneinander zusammen, weil jede ihre eigenen
Jnteressen durchsetzen wollte. Da ich in groser Sorge war, das
die Jungen und Unersahrenen unter uns diesen Versuchungen er-
liegen werden, trieb es mich, allen diesen folgendes zu schreiben:
,,Jhr Freunde allenthalben! Hütet euch vor Komplotten
und Wühlereien, und vor dem Arm des Fleisches, denn alle
diese Machthaber sind gefallene Söhne Adams; sie richten der
Menschen Leben zugrunde, wie Hunde, Schweine und andere
Tiere sich zugrunde richten, sich beisen und zerreisen. Wie ent-
stand das Streiten und Töten anders als aus der Lust? Und
dies alles kommt vom gefallenen Adam her, nicht von demjenigen h
Adam aber, der nicht fiel, in welchem Leben und Frieden ist
(1. Cor. 15). Jhr seid zum Frieden berufen, darum jaget ihm
nach, und dieser Frieden ist in Christus und nicht in dem gefallenen
Adam. Alle, die jetzt vorgeben, für Christus zu kämpfen, betrügen
sich; denn sein Reich ist nicht von dieser Welt; darum kämpfen
seine Diener nicht. Die Streitenden gehören nicht zu seinem
Reich, denn sein Reich ist Frieden und Gerechtigkeit .... Jhr,
die ihr Erben seid des Evangeliums des Friedens, welches ge-
wesen, ehe der Satan war, lebet in diesem Evangelium, suchet den
Frieden und das Gute für alle, und lebet in Christus, der ge-
kommen ist, die Seele der Menschen vom gefallenen Adam zu
erlösen; das äusere Schwert der Juden, mit dem sie die Heiden
umbrachten, war ein Sinnbild des inwendigen Geistes Gottes,
der die inwendige heidnische Natur tötet. So lebet denn im
friedsamen Reich Jesu Christi, im Frieden Gottes und nicht in
den Lüften, aus denen der Krieg entsteht .... und suchet das
Wohl und Gedeihen für alle Menschen.« G. F.
Bald darauf ergriff George Booth in Cheshire die Waffen
und Laniberts) zog gegen ihn; daraufhin wollten etliche Hitzköpfe,
wie solche zuweilen unter uns waren, auch die Waffen ergreifen,
aber der Herr trieb mich, sie zu ermahnen und ste blieben ruhig,
1) Lambert war einer der bedeutendsten Generäle aus der Partei Cromwells.


% \picinclude{./140-149/p_s145.jpg} 
Ein Gottesgericht. Erntahnung zur Barmherzigkeit usw. 145
Zur Zeit des sogenannten Sicherheit?-aus-schusses forderte man uns
auf, die Waffen zu nehmen und manchem von uns wurden hohe
Stellen und Kommandos angeboten, aber wir schlugen sie alle aus
und traten mündlich und schriftlich dagegen auf, indem wir er-
klärten, unsere Waffen und Rüstungen seien nicht fleischlich, sondern
geistlich, und damit keiner unter unz in diese Falle gehe, kam es
über mich vom Herrn, bei dieser Gelegenheit einige Zeilen der
Ermahnung an alle zu schreiben .....
Nachdem ich längere Zeit in London geroeilt hatte, zog ich
wieder in den Grafschaften umher, durch Essex und Suffolk nach
Norwich .... und von da durch Huntingdomshire und Eam-
bridgeshire wieder nach London, gerade alö General Monk 1) dort
eingezogen war und die Tore und Befestigungen der Stadt
fielen. Lange vorher hatte ich ein Gesicht gehabt, in welchem
ich die Stadt in Trümmer und die Tore eingestürzt gesehen
hatte, gerade so, wie ich sie nun mehrere Jahre später nach dem
Brande sehen sollte.