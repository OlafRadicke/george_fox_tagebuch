
% \picinclude{./080-089/p_s080.jpg} 
junger Mann um seines Glaubens willen hingerichtet werden,
schrieben sie deshalb an die Magistrate.
Ungefähr um die gleiche Zeit schrieb ich an die Behörden
von Earlisle, die mich ins Gefängnis geworfen und die die
Freunde auf Anstiften der zehntengierigen Priester verfolgten:
,,Freunde! Thomas Eraston und Cuthbert Stadholm,
Guer Tun ist in London bei den Gutgesinnten bekannt ge-
worden. Was habt ihr alles geleistet an Gesangennehmen,
Güterschändungen, Metzeleien und anderen Scheußlichkeiten in den
letzten paar Jahren! ganz menschenunwiirdig, wie wenn ihr
noch nie die Schrift gelesen und zu Herzen genommen hättet!
Jst das das Ziel der Religion Earlisles und seiner Kirche und
seiner Ehristlichkeit? ihr habt es zu schanden gemacht mit eurer
Blindheit, eurem tollen Treiben und eurem verkehrten Gtfern.
War es nicht immer die Art der blinden Leiter und der falschen
Propheten zu zanken (Jes. 56), mit denen, die ihnen den Mund
nicht füllen wollen? Seid ihr nicht die Lasttiere und Diener der
Priester gewesen? Wenn sie euch anspornen, das Schwert gegen
den Unschuldigen zu gebrauchen, so rennt ihr auf solche, die nach 3
den Befehlen der Schrift die Waffe nicht gebrauchen dürfen, loss
Und doch wollt ihr eure unheiligen Hände und gemeinen Lippen
zu Gott erheben, und gebet oor, zu fasten und seid doch voll
Hader und Zank (Jes. 58, 4). Brannte nie euer Herz in euch?
habt ihr nie über euren Zustand nachgedacht? Seid ihr ganz
der Lust des Teufels, dem Verfolgen, anheimgefallen? Wo ist
eure Feindes-liebe? (Matth. 5). Wo ist euer Beherbergen der
Fremdlinge? (Matth. 25, 35). Wie überwindet ihr Böses mit
Gutem? (Röm. 12, 21). Wo sind eure Lehrer, die ,,durch heil-
same Lehre die Widersprecher strafen?« (Tit. 1, 9) .... Leset die
Schrift und sehet, wie unähnlich ihr den Aposteln und Propheten
seid; und wie ihr denen gleichet, die die Propheten, die Apostel
und Christus verfolgten. Jhr gehet in ihren Fußstapfen und
kämpfet mit Fleisch und Blut, nicht mit den Fürsten der Welt,
die in der Finsternis dieser Welt herrschen, und mit den bösen
Geistern unter dem Himmel« (Gph. 6, 12). Jn keinem anderen
Lande geschehen solche Greuel, daß man den Leuten ihr Gut
raubt, ihnen ihre Ochsen und Rinder nimmt, ihre Schafe, ihr
Getreide und ihr Hausgeräte und gibt es den Priestern, die doch
nichts für sie gearbeitet haben. Jhr seid eher Straßenräuber


% \picinclude{./080-089/p_s081.jpg} 
Fox der Hexerei verdächtigt. Falsche Qsfenbarungen usw. 81
alß Diener Gottes gegen die Freunde; ihr verklagt sie bei euren
Gerichten und legt ihnen Bußen auf, weil sie die Gebote Christi
nicht übertreten, also nicht schwören wollen« ..... G
Anthony Pearson and Gervase Benson dursten mich nicht im
Gesängniö besuchen, obwohl sie Frieden?-richter waren. Sie
schrieben darum an die Magistrate und Priester von Carlißle:
,,Wir bezeugen, daß dieser George Fox, der von den Magi-
straten, von den Friedenörichtern, den Priestern und dem Volt ver-
folgt wird und gegenwärtig alt?. Gotteßlästerer und Verführer
gefangen gesetzt ist, ein Prediger dez Worteß Gottes ist und daß
ewige Evangelium verkündet; durch sein mächtiges Predigen hat
der große Vater der Heiligen den Blinden die Augen geöffnet,
den Tauben die Ohren aufgetan, die Gefangenen erlöst und die
Toten auferweckt (Jes. 35, 5). Christuß wird jetzt gepredigt unter
den Seinen, wie er war und ist; und weil er mm, in der Gestalt
seineö getreuen Dienerß, wieder erscheint, sv verfolgen ihn die
Abgefallenen, Fürsten, Herrscher, Priester und Volk. Nicht alß
ein Übeltäter leidet er von euch, ihr Magistrate, sondern weil er
nicht abgefallen ist und gegen daß Treiben der Welt und daß
Böse auftritt. EZ ist immer so gewesen, daß, wo die oerderbte
Natur den Samen Gottes unterdrückte, die Verderbten suchen die,
in denen dieser Same ausging, gefangen zu nehmen .... Wie
Christuö daö, maß man einem der Geringsten erweist, als ihm
getan ansieht (Matth. 5, 25), also siehet er auch daß, wa?. man
ihnen nicht tut, als ihm nicht getan an. Wenn ihr nun soweit
geht, daß ihr nicht einmal anderen gestatten wollt, einen gefangenen
Bruder in seinen Leiden zu besuchen, so werdet ihr in den feurigen
Pfuhl, der mit Schwefel brennt, geworfen (Offb. 19, 20). Der
Herr ist gekommen, die Berge zu stürzen und zu Staub zu zer-
nialmen (Jes. 41, 15), und er wird rächen die Unterdrückung der
Gewissen seineö Volkes- an allen ungerechten Herrschern, Beamten
und Gesetzen. Er wird seinem Volke sein Gesetz geben nicht nach
dem, wat?. vor Augen ist, sondern nach Recht und Gerechtigkeit.
Man hat nun gesehen, wie eure Herzen voll Haß sind gegen die
Wahrheit Gotteö, die er durch sein von der Welt oerachteteö und
zum Spott ,,Quäker« genannteö Volk verkünden läßt. Jhr seid
ärger als die Heiden, die Pauluö inö Gefängniß warfen; denn
niemand hat damals seinen Freunden verboten, ihn zu besuchen,
Gkotge Fox. 6


% \picinclude{./080-089/p_s082.jpg}
darum treten sie gegen euch als Zeugen auf. M ist offenbar
geworden, daß ihr denen gleich seid, die Christus töteten und die
Apostel gefangen nahmen unter dem gleichen Vorwand, nämlich
daß sie den Jrrtum Wahrheit und die Diener Gottes Gottes-
lästerer nannten. Aber das Gericht, das über euch kommen wird,
ist schrecklich, ihr ungerechten Magistrate und Priester und ihr
alle, die ihr mit Worten die Wahrheit bekennet, und doch die
Kraft der Wahrheit und die, die in der Wahrheit sind und für die
Wahrheit einstehen, verfobget. Gehet in euch, dieweil es Zeit
ist, und bedenket, was Jesaias 17 geschrieben steht!«
Geroase Benson
Anthony Pearson.
Bald darauf kam die Macht des Herrn über die Richter
und sie setzten mich frei. Kurz vorher war Anthony Pearson
mit dem Gouverneur in meinen Kerker gekommen um zu sehen,
wie ich behandelt werde. Sie fanden den Ort so gräulich und
den Geruch so schlecht, daß sie sich über die Magistrate entsetzten,
die solches von dem Kerkermeister geschehen ließen. Sie ließen
die Wärter in den Kerker kommen und sich für ihr Betragen
rechtfertigen. Den Unterkerkermeister, der so grob gewesen war,
sperrten sie darauf zu uns ins Gefängnis unter die Räuber.
Nachdem ich nun frei war, ging ich zu Thomas Bewley . . .
Dann ging ich auss Land und hatte viele große Versammlungen . . .
und tausende bekehrten sich zum Herrn Jesus Christus-.
Dann ging ich nach Westmorland . . . Durham, Hexhain . . .
Gilsland . . . nach Eumberland ..... Hier überall, sowie in
Northumberland, Laneashire und Yorkshire fanden große Be-
kehrungen statt, und was Gott gepflanzt hatte, wuchs und gedieh
unter dem Himmelsregen von oben und Gottes leuchtender Herr-
lichkeit, sodaß sich vieler Mund öffnete zum Lobe Gottes; ja: ,,aus
dem Munde der Unmitndigen und Säuglinge richtete er sich eine
Macht zu« (Psalm 8, Z).


% \picinclude{./080-089/p_s083.jpg} 

%%%%%%%%%%%%%%%%%%% Kapitel 7. %%%%%%%%%%%%%%%%%%%%%%%%%%%%%%

\chapter[Begegnung mit Oliver Cromwell]{Begegnung mit Oliver Cromwell}

\begin{center}
\textbf{Kämpfe mit schwärmerischen Ranters und 
zehntengierigen Priestern.
Fox in Wetstone verhaftet und vor Cromwell geschickt.}
\end{center}


Die Priester und Frommen traten aufs neue mit ihren
Prophezeihungen gegen uns auf. Schon lange hatten sie vor-
ausgesagt, daß wir binnen eines Monats vernichtet sein werden;
hernach verlängerten sie die Frist auf ein halbes Jahr; als aber
auch diese Zeit längst um war, und wir im Gegenteil an Zahl
zunahmen, streuten sie aus, wir werden einander gegenseitig ver-
zehren. Es kam nämlich ost vor, daß nach den Versammlungen
manche, die einen weiten Heimweg hatten, bei Freunden blieben,
es waren ost mehr Leute als Betten vorhanden, so daß Viele auf
dem Heu übernachten mußten. Da wurden die »Frommen« von
der Furcht Eains gepackt; sie hatten Angst, daß, wenn wir ein-
ander zu Grunde gerichtet hätten, wir dann der Gemeinde zur
Last fallen und uns von ihr unterhalten lassen werden. Llls sie
aber sahen, wie der Herr den Freunden Segen und Gedeihen
gab, wie dem Abraham, ,,beim Acker und beim Korb, beim Gin-
gehen und beim Ausgehen, beim Aufstehen und beim Niederliegen«
(5. Mose 28), da erkannten sie die Ungerechtigkeit ihrer Prophe-
zeihungen, und daß man ,,umsonst flucht, wo der Herr segnet«
(4. Mose, 23). Als nach den ersten Bekehrungen die Freundes
den Hut nicht vor den Leuten abnahmen, einer einzelnen Person
nicht mit ihr, sondern mit ,,du« und ,,dich« antworteten, sich nicht
verneigten und nicht bei der Begrüßung schmeichelhafte Worte
gebrauchten und nicht die Art und Weise der Welt mitmachten, da ver-
loren viele von ihnen in ihren Geschäften die Kundschaft; man
scheute sich vor ihnen und wollte keine Geschäfte mit ihnen machen,
so daß eine Zeitlang die Freunde kaum ihr Brot verdienten. Aber
als die Leute sahen, wie treu und ehrlich die Freunde waren,
und daß ihr ja — ja und ihr nein — nein war; daß sie Wort hielten
im Verkehr und niemanden hintergingen noch betrogen, und wie
der Herr ihnen Segen und Gedeihen gab; wie ein Kind, das sie
schickten, um einen Einkauf zu machen, gerade so gut bedient
wurde wie sie selbst, da predigte das Leben und der Wandel der
Fretmde, und es traf das, was von Gott kam, in ihren Gewissen.
Nun wandelten sich die Dinge dermaßen, daß man beständig
fragen hörte: ,,Wo ist ein Krämer, ein Tuchhändler, ein Schneider,
 


% \picinclude{./080-089/p_s084.jpg} 
ein Schuster, ein Handwerker, der Quäker ist?« Die Freunde
bekamen mehr Arbeit als manche andere Handwerker und betei-
ligten sich reger am geschäftlichen Verkehr. Nun schlugen die
gehässigen »Frommen« einen anderen Ton an und fingen an zu
murren: ,,Wenn wir diese Quäker gewähren lassen, so werden sie
unß den Handel deß ganzen Landeß an sich reißen.« Also tat
der Herr an seinem Volke, und es ist mein ernstlichster Wunsch-
daß alle, die seine heilige Wahrheit bekennen, in der E-rkenntnitz
bewahrt und durch den Geist und die Kraft in der Treue erhalten
bleiben mögen, erstlich gegen Gott, im Gehorsam in allen Dingen,
und dann gegen die Menschen, in Rechtschaffenheit und Gerech-
tigkeit in allem Verkehr; damit Gott der Herr verherrlicht werde
durch einen Wandel in Wahrheit und Heiligkeit, Gerechtigkeit und
Gottseligkeit .....
Die Priester in Newcastle, Kendal und anderen nördlichen
Gegenden waren sehr aufgebracht gegen unß. Einer, uamenß
Gilpin, der manchmal zu unß nach Kendal gekommen war, war
bald von der Wahrheit abgefallen und aus allerlei einfältige
Gedanken gekommen, und die Priester gebrauchten nun daß gegen
unö, wo sie nur konnten; aber die Kraft des- Herrn wars sie
alle darnieder. Der Herr vernichtete zwei der Verfolgung?-süchtigen
Richter von Carliöle und der dritte wurde einige Zeit darauf
seineß Amts entsetzt und verließ die Stadt.
Um diese Zeit wurde den Soldaten der Eid, den sie Oliwer
Eromwell schwören sollten, vorgelegt, und viele wurden entlassen,
weil sie im Gehorsam gegen Christus nicht schwören konnten.
Einer von diesen war John Stubbö, der bekehrt worden war
während meiner Gefangenschaft in Carliöle, und ein guter Soldat
im Kampfe de-3 Lamnieö und ein treuer Jünger Jesu geworden
ist. Er reiste Viel umher im Dienste dez Herm, in Holland,
Schottland, Jtalien, J-rland, Ägypten, Amerika. Und die Kraft
Gotteö bewährte ihn vor den Händen der Papisten, obgleich er
oft in großer Gefahr vor der Jnquisition war. Andere unter
den Soldaten jedoch, die wohl ihrer überzeugung nach bekehrt
worden waren, aber nicht zum Gehorsam gegen die Wahrheit
gelangten, schwuren den Eid Cromwellö: als diese später in Schott-
land waren, kamen sie in die Nähe einer Garnison; die dortige
Mannschaft glaubte eß seien Feinde und töteten sie ....
Der Herr trieb viele von denen, die er auöerlesen hatte, in


% \picinclude{./080-089/p_s085.jpg} 
Kämpfe mit schwärmerischen Ramcrs und zehntengierigeu Priestern usw. 85
seinem Weinberg zu arbeiten, nach Süden zu gehen und sich im
Dienste des Evangeliums nach den südlichen und westlichen Teilen
des Landes zu verteilen; so gingen Francis Howgill und Edward
Burrough nach London, John Camm und John Audland nach
Bristol, Richard Hubberthorn und George Whiteheads) gegen
Norwich, Thomas Holmes 1) nach Wales und andere nach anderen
Richtungen; etwa sechzig Diener hatte der Herr ausersehen und
aus dem Norden in die Verschiedenen Teile des Landes gesandt.
Um die Zeit singen Rice Jones oon Nottingham, ein früherer
Baptist und jetzt Ranter, und seine Anhänger an, gegen mich
zu prophezeien; sie sagten, ich hätte jetzt meinen Höhepunkt
erreicht und werde nun bald tief fallen .... Aber seine und
der Seinen Weissagungen erflillte sich an ihnen selber; denn bald
daraus fielen sie ganz auseinander und viele von ihnen wurden
Freunde und blieben es; und durch des Herm mächtige Macht und
Wahrheit vermehrten sich die Freunde .... Rice Jones da-
gegen leistete den Eid und war also dem Gebot Christi unge-
horsam. Viele falsche Propheten haben sich gegen mich erhoben,
aber der Herr hat sie alle vernichtet und wird auch ferner alle
vernichten, die sich gegen seinen gesegneten Samen erheben ....
Ju der Nähe von Kidsley-Park stieß ich aus eine Schar
Ranter; aber die Kraft des Herrn hielt sie drunten. Von da
ging ich in die Gegend des Peak zu Thomas Hammersley,
wohin die Ranter dieser Gegend kamen und Viele angesehene
,,Fromme«. Die Ranter traten gegen mich auf und fingen an
zu schwören; als ich ihnen deswegen Vorstellungen machte, oer-
suchten sie, Schriftstellen zu bringen und sagten, Abraham, Jakob
und Joseph haben geschworen und die Priester und Moses und
die Engel. Jch erwiderte: ,,ich gebe zu, daß alle diese es taten,
wie die Schrift es berichtet; Christus aber sagt: ,,schwöret nicht!«
Und Christus ist das Ende der Propheten und des alten Priester-
tums und des Gesetzes Moses und regiert über das Haus Jakobs
und Josephs, und er sagt: ,,ihr sollt nicht schwören.« Und als
Gott den Erstgeborenen in die Welt sandte, sagte er: »alle Engel
sollen ihn anbeten« (Gbr. 1, 6), also diesen Christus, der sagte, ihr
sollt nicht schwören. Und was die Begründung anbelangt, welche
1) George Whitehead und Thomas Holmes, zwei eisrige Quäkerprediger.
(Näheres s. Weingarten a. a. O.)


% \picinclude{./080-089/p_s086.jpg} 
die Menschen für daß Schwören geltend machen, um ihre Strei-
tigkeiten zu Ende zu bringen, so hat Ehristu-5, der gesagt hat,
ihr sollt nicht schwören, den Teufel und seine Werke, deren eines
eben daö Streiten ist, vernichtet. Und Gott sagt: »dieS ist mein
lieber Sohn, an dem ich Wohlgesallen habe, ihn sollt ihr hören.«
(Mark. 9, 7). Also soll man den Sohn hören, der daß Schwören
verbietet. Und der Apostel Jakobus, welcher den Sohn hörte,
und ihm folgte und ihn verkündete, verbietet daß Schwören,
Jakobuz 5, 12.« Die Kraft des Herm erfaßte sie und sein
Sohn und seine Lehre beherrschten sie. Das Wort des Lebenß
wurde reich und herrlich unter ihnen verkündet an dem Tage,
und viele wurden bekehrt.
Diesem Thomas Hammersley wurde einmal gestattet, an
einem Geschworenengericht als- Geschworener zu amtieren ohne
einen Eid abzulegen; als- er dann, als- Vorsitzender, sein Gut-
achten abgab, erklärte der Richter, er sei nun doch schon seit Vielen
Jahren Richter, aber er habe noch nie ein so redlicheß Gutachten
gehört, als daß von diesem Quäker! ES ließe sich noch viel
derartiges berichten, wenn die Zeit reichen würde. Die herrliche
Wahrheit des Herrn goß sich auß; ihr gebühret Preiß und Ehre
ewiglich!
Aus der Durchreise durch Derbyshire besuchte ich überall
Freunde, bis ich nach Swannington kam; hier war eine große
Versammlung, zu der Baptisten, Ranter und viele andere ,,Fromme«
kamen. ES hatte viele Zusammenstöße mit ihnen und den Priestern
der Stadt gegeben. Von überallher kamen Freunde zu dieser
Versammlung, so John Audland, Franciö Howgill, Edward Pyot
von Bristol und Edward Vurrough aus London und es wurden
viele bekehrt. Die Ranter machten Störungen und benahmen sich
sehr unverschämt; aber schließlich kam die Macht dez Herm über
sie und sie unterlagen. Am darauffolgenden Tage kam Jacob
Bottomley, ein großer Ranter von Leieester; aber die Kraft des
Herrn überwältigte ihn. So auch einen Priester. Wir ließen
den Rantern sagen, sie sollten kommen und ez mit ihrem Gott
versuchen; sie kamen in Haufen und waren sehr wild und sangen
und pfifsen und tanzten; aber die Kraft dez Herrn überwältigte
sie so, daß viele von ihnen bekehrt wurden.
Von hier ging ich nach Twycroß, wohin auch Ranter kamen
und vor mir sangen und tanzten; aber in der Furcht deß Herrn


% \picinclude{./080-089/p_s087.jpg} 
Kämpfe mit schwärmerischen Ranters und zehntengierigen Priestern usw. 87
trieb es mich, sie zu tadeln; und die Kraft des Herrn kam über sie,
so daß einige von ihnen bekehrt wurden und den Geist Gottes
aufnahmen. Sie sind tüchtige Leute geworden, die rechtschaffen
in der Wahrheit Christi leben und wandeln. Jch ging zu Anthony
Brickleh in Warwickshire, wo eine große Versammlung war;
mehrere Baptisten und andere kamen und lärmten; aber die Kraft
des Herrn kam über sie.
Hierauf ging ich nach Drayton in Leicestershire, um meine
Verwandten zu besuchen. Kaum war ich angekommen, so ließ
der Priester Nathanael Stephens, der noch einen andern Priester
hatte kommen lassen und die Umgegend von meinem Kommen
benachrichtigt hatte, mich zu sich holen, denn sie konnten nichts
machen, ehe ich kam. Da ich drei Jahre meine Angehörigen
nicht gesehen hatte, so wußte ich nichts von ihren Absichten. Jch
ging nun aus den Platz des Turmhauses, wo die beiden
Priester waren, und wo sich eine Menge Leute versammelt hatten.
Als ich kam, wollten die Leute, daß ich ins Turmhaus gehe; ich
fragte sie, was ich dort tun solle; sie erwiderten, Stephens
könne die Kälte nicht ertragen; ich sagte, er könne sie so gut
ertragen wie ich. Zuletzt begaben wir uns in einen großen Saal;
Richard Farnsworth war auch dabei; wir hatten einen großen
Disput mit den Priestern über ihren Wandel, und daß sie so
sehr das Gegenteil von dem seien, was Christus und die Apostel
gewesen. Die Priester wollten wissen, wo die Zehnten verboten
oder aufgehoben seien; ich wies es ihnen nach im 7. Kap. des
Hebräerbrieses, wo nicht nur die Zehnten, sondern das ganze
Priestertum, das Zehnten annahm, aufgehoben war und das
Gesetz, nach welchem das Priestertum eingesetzt und die Zehnten
erhoben wurden. Hierauf hetzten die Priester das Volk zur Frech-
heit und Roheit gegen uns auf. Jeh hatte Stephens seit seiner
Kindheit gekannt und konnte ihnen darum aufdecken, was für
eine Art von Mensch er sei und was hinter seinen Predigten
stecke, und wie er, wie alle Priester, die Verheißungen aus den
alten Menschen, der sterben muß, bezog; dann zeigte ich ihnen,
daß die Verheißungen vielmehr dem Samen galten, nicht den
vielen Samen, sondern dem einen Samen, Christus, der derselbe
ist in Mann und Weib; denn alle müssen wiedergeboren werden,
ehe sie ins Reich Gottes eingehen können. Er erwiderte mir
daraus, ich sollte nicht in der Weise richten; ich entgegnete ihm,


% \picinclude{./080-089/p_s088.jpg} 
,,der Geistliche richtet alle?-« (1. Cor. 2, 15); er gab zu, daß dietz
genau der Schrift gemäß sei; dann aber fuhr er fort: ,,ihr Nach-
barn, das ist die Sache: George Fox ist zum Lichte der Sonne
gekommen und nun möchte er mein Sternenlicht au8löschen.« Jch
erwiderte: ,,ich will nicht das- kleinste Maß von dem, was einer
von Gott hat, in jemand unterdrücken, noch viel weniger sein
Sternenlicht auölöschen, wenn eS ein wirkliches Sternenlicht ist,
ein Licht vom Mvrgenstern.« Dann erklärte ich ihm, daß, wenn
er etwaß von Gott oder Ehristuß empfangen habe, er umsonst
predigen müsse und nicht Zehnten nehmen von den Leuten sür
seine Predigten, da er ja gesehen habe, wie Christu;3 seinen
Jüngern befohlen habe, umsonst zu geben, wie sie es- umsonst
empfangen hätten. Jch schärste ihm also ein, nicht mehr für
Zehnten und Lohn zu predigen. Aber er sagte, dem werde er
sich nicht fügen. Die Leute fingen an, unverschämt zu werden,
und wir brachen darum auf. Dennoch waren etliche an dem
Tage der Wahrheit zugetan worden. Ghe ich fort ging, sagte
ich ihnen, daß ich im Sinn habe, nächste Woche, so Gott wolle,
wieder in der Stadt zu sein. Jn der Zwischenzeit ging ich in
die Umgegend und hielt Versammlungen, und nach acht Tagen
kam ich wieder zurück. Der Priester hatte für diese Zeit 7 Priester
kommen lassen, um ihm zu helfen; und Stephenß hatte in einem
Gottesdienst am Markttage in Adderßton angezeigt, daß an dem
und dem Tage ein Diöput mit mir stattfinden werde. Jch wußte
nichts davon und hatte nur gesagt, ich werde über acht Tage
wieder in der Stadt sein. Die acht Priester hatten etliche hundert
Leute versammelt, meist aua der Umgegend und wollten, ich sollte
ins Turmhauö gehen; aber ich wollte nicht hingehen, sondem
ich ging aus einen Hügel und redete von dort zum Volk .....
GS kamen einige Unverschämte und nahmen mich aus die
Arme und trugen mich unter die Türe dee Turmhauseß, in der
Absicht, mich mit Gewalt inß Turmhauß zu bringen; da aber
die Tür geschlossen war, purzelten sie alle übereinander; und ich
lag zu unterst. So bald ich konnte, kroch ich hervor und ging wieder
auf den Hügel; nun schleppten sie mich biö an die Mauer
dez Turmhauseö und setzten mich auf eine Art Steinbank; alle
Priester waren auch herbeigelaufen und standen mitten unter dem
Volk herum, und alle schrieen: »Beweise, beweise!« Jch sagte,
ich hörte nicht auf ihre Stimmen, denn eß seien die Stimmen von


% \picinclude{./080-089/p_s089.jpg} 
Kämpfe mit schwiirmetischen Ramerz und zehntengierigen Priestern usw. 89
Mietlingen und Fremdlingen. Sie schrieen wieder: ,,Beweise,
beweise!« Ich wies aus Johannes, wo sie sehen können, wac-
Christuß zu ihreßgleichen sage, nämlich: »Jch bin der gute Hirte,
der sein Leben gibt für seine Schafe, der Mietling aber flieht,
wenn der Wolf kommt.« Jch schlug ihnen vor, ihnen zu beweisen,
daß sie solche Mietlinge seien; darauf rissen die Priester mich
wieder herunter und stiegen selber alle auf Steinbänke
an der Mauer des Turmhauseß. Da fühlte ich, wie Gottes
mächtige Kraft über alle kam und sprach zu ihnen: ,,Wenn ihr
mir Gehör schenken wollt und mich ruhig anhören, so will ich
euch (muß der Schrift zeigen, warum ich die acht Priester oder
Lehrer, die vor mir stehen, nicht anerkenne und überhaupt keine
Mietlingßlehrer der Welt.« Priester und Volk erklärten sich bereit
zu hören. Da zeigte, ich ihnen auß den Propheten Jesaja,
Jeremia, Ezechiel, Micha, Maleachi und anderen, daß sie in den
Fußstapfen derer wandeln, gegen die Gott seine Propheten ge-
sandt hatte ....
Dann als ich an das neue Testament kam, zeigte ich ihnen,
daß sie wie die Hohenpriester und Schriftgelehrten seien, und
wie die Pharisäer, gegen die Christuö wehe! schrie (Matth. 23).
Jsndem ich in dieser Weise außführlich aus der Schrift bewiesen
hatte, warum sie den Pharisäern gleichen, . . . und sie vor allem
Volk unter die Pharisäer, falschen Propheten und Verführer
gerechnet hatte und gezeigt, wie ihresgleichen von den wahren
Propheten und Christus verdammt werden, wieß ich sie aus daß
Licht Jesu Christi hin, daß einen jeden, der in die Welt kommt,
erleuchtet (Joh. 1, 9), und durch dieseß Licht könnten sie erkennen,
, ob daß Gesagte wahr sei. Sie mochten nichtß davon hören, daß
ich sie aus das- Göttliche in ihnen, auf daß Licht Jesu Christi
hinwieö. Bis dahin waren sie alle ruhig gewesen, nun aber
rief einer der »From1nen«: ,,Wirst du denn nie fertig, Fox?«
Ich erwiderte, ich sei nun bald fertig; ich fuhr noch eine Weile
fort, biß ich fühlte, daß ich an ihnen getan hatte, maß ich mußte
in der Kraft dez Herrn ...... A13 ich fertig war, flüsterten
die Priester untereinander, und Priester Stephenö kam zu mir
und verlangte, daß mein Vater und mein Bruder und ich mit
ihm beseite kommen, damit er mit uns reden könne; und die
anderen Priester mußten daß Volk davon abhalten unß nachzu-
kommen. Jch ging sehr ungern mit ihm, aber da daö Volk schrie:

