%%%%%%%%%%%%%%%%%%% Kapitel 2. %%%%%%%%%%%%%%%%%%%%%%%%%%%%%%
\chapter[Erste Versammlungen]{Erste Versammlungen}

\begin{center}
\textbf{Erste Versammlungen und Proteste.}
\end{center}

Die Macht des Herrn  hatte nun, im Jahre 1648\jahr{1648}, schon vielen
die Herzen geöffnet, das sie das Wort des Lebens und der 
Versöhnung aufnahmen. Als ich nun einmal im Hause eines Freundes,
in Nottinghamshire, saß, erkannte ich, das ein großes Krachen
durch die ganze Erde gehen musste und ein großer Rauch 
aufsteigen, überall wo es krachte, und darnach würde ein großes
Beben entstehen: es war die Erde in der Menschen Herzen, die
erbeben musste, bevor der Same Gottes aus der Erde 
hervorgehen konnte. Und so geschah es: die Macht des Herrn  fing an,
sie erbeben zu machen, und wir fingen an, große Versammlungen
zu haben, und man spürte die mächtige Kraft und das Wirken
Gottes unter den Leuten, zu ihrer und der Priester Erstaunen [...].

Ich ging nach Mansfield\ort{Mansfield}, wo eine große Versammlung von
\zitat{Frommen} und andern Leuten stattfand; da trieb es mich zu
beten, und die Kraft des- Herrn  war so mächtig, das es schien,
als ob das ganze Haus erbebte. Als ich geendet, sagten etliche
der \zitat{Frommen}, es sei gerade wie in den Tagen der Apostel, da
sich \zitat{das Haus bewegte, in dem sie versammelt waren} 
(Act. 2:2\bibel{Act. 02:02@Act. 2:2}).
Nachdem ich gebetet, wollte einer der \zitat{Frommen} beten, aber
dadurch kam eine Trübung und etwas totes über sie und die
andern \zitat{Frommen} wurden betrübt über ihn und sagten, es sei
eine Versuchung über ihn gekommen; darauf kam er zu mir und
bat mich, ich solle wieder beten, aber ich konnte nicht auf 
eineis Menschen Geheis beten\index{Beten!nach Aufforderung}.
Bald darauf war abermals eine Versammlung von \zitat{Frommen}
% \picinclude{./010-019/p_s017.jpg} 
und ein Hauptmann namens Stoddard wohnte ihr bei. Sie
redeten über das Blut Christi\index{Blut Christi}, 
und während sie darüber sprachen,
sah ich durch die unmittelbare Offenbarung des unsichtbaren
Geistes das Blut Christi. Und ich schrie auf und rief: \zitat{Seht
ihr nicht das Blut Christi? Seht in eure Herzen, wie es eure
Herzen und Gewissen besprengt, das sie, los von den toten
Werken, dem lebendigen Gott dienen} (Hebr. 9\bibel{Hebr. 9}). 
Denn ich sah es, das Blut des neuen Testamentes, 
wie es ins Herz kam. Das
erschreckte die \zitat{Frommen}; sie wollten das Blut nur 
auswendig, nicht inwendig haben. Aber 
Hauptmann Stoddard\person{Hauptmann Stoddard} war ergriffen
und sagte: \zitat{Last den Jüngling reden, hört ihn an}, als er sah,
wie sie mich mit vielen Worten zu besiegen suchten.

Es waren auch eine Anzahl Priester da, die für gottselig
galten; einer von ihnen hieß Kellett, und etliche, die empfänglichen
Gemütes waren, gingen hin, um sie zu hören. Es trieb mich,
ihnen nachzugehen, um sie zu ermahnen, auf die Lehre Gottes in
ihrem Inneren zu hören. Damals war der Priester Kellett gegen
das Priesteramt; später jedoch nahm er selbst ein solches an und
wurde ein Verfolger.

Nachdem ich etliche Arbeit getan hatte in dieser Gegend,
ging ich durch Derbshire in meine 
Heimat Leicestershire\ort{Leicestershire}, und
es wurden mehrere, die empfänglich waren, gewonnen. Als ich
von dort weg zog, begegnete ich einer großen Zusammenkunft
von \zitat{Frommen}, die im Freien beteten und die Schrift 
auslegten. Sie reichten mir die Bibel und ich öffnete sie beim 5. Kap.
des Matth.\bibel{Matth. 5}, wo Christus das; 
Gesetz auslegt; und ich erklärte
ihnen den inneren Zustand und den äußeren Zustand worüber sie
in heftigen Streit gerieten und so auseinander gingen; aber die
Kraft des Herrn  nahm überhand.

Darauf hörte ich von einer großen Versammlung, die in
Leicester stattfinden würde; es sollte eine Disputation 
geben, die die Presbyterianer\index{Presbyterianer}, 
Independenten\index{Independenten}, Baptisten\index{Baptisten} 
und Common-Payen\index{Common-Payen}
Leute gleicherweise angehen sollte. Die Versammlung war in einem
Turmhause, und der Herr trieb mich, dorthin zu gehen und
zugegen zu sein. Ich hörte ihren Verhandlungen und 
Beweisführungen zu. Einige saßen in Kirchenstühlen und der Priester
war auf der Kanzel; es war eine große Menge versammelt.
Zuletzt tat eine Frau eine Frage über die Stelle bei Petrus:
\zitat{Wiedergeboren aus ewiglichem Samen, aus dem lebendigen Wort
% \picinclude{./010-019/p_s018.jpg} 
Gottes, das ewiglich bleibet} 
(1. Petr. 1\bibel{Petr. 1. 01@1. Petr. 1}). Der Priester sagte ihr:
\zitat{Ich erlaube keiner Frau in der Kirche zu reden,}
\index{Frauenrecht} obgleich er
vorher allen die Freiheit erteilt hatte, zu reden. Da wurde ich
von der Kraft des Herrn  übermannt wie in einer Verzückung,
und ich erhob mich und fragte den Priester: \zitat{Nennst du dies
hier, dieses Turmhaus, eine \textit{Kirche}? oder nennst du diese
bunte Menge eine Kirche?}\index{Ekklesiologie} Denn er 
hätte der Frau auf ihre
Frage antworten sollen, nachdem er vorher allen die Freiheit
erteilt hatte, zu reden. Anstatt mir zu antworten, fragte er mich:
was eine Kirche sei. Ich sagte: \zitat{Die Kirche ist der Pfeiler und
Grund der Wahrheit, aus lebendigen Steinen gemacht, aus
lebendigen Gliedern (1. Petr. 2\bibel{Petr. 1. 02@1. Petr. 2}), 
eine geistige Hausgemeinde,
deren Haupt Christus ist; aber er ist nicht das Haupt einer bunten
Menge oder eines alten Hauses aus Kalk, Steinen und Holz.}
Diese Worte brachten alles aus Rand und Band; der Priester
kam aus seiner Kanzel, andere aus ihren Stühlen, und die 
Verhandlungen waren gestört. Ich ging in eine große Herberge und
disputierte dort mit Priestern und \textit{Frommen} aller Richtungen;
und alle waren furchtbar hitzig.\index{Konflikt!theologisch} 
Aber ich bestand auf der wahren
Kirche und ihrem wahren Haupt, trotz ihnen allen, bis sie 
nachgaben und auseinanderstoben. Einer schien sehr geneigt und kam
eine Zeit lang, in der Absicht, sich mir anzuschließen; aber bald
kehrte er sich ganz gegen mich und schloss sich einem Priester an,
trat für die Kindertaufe ein, obgleich er vorher selber ein
Baptist gewesen war, und verließ mich. Aber es wurden an dem
Tage etliche gewonnen; auch die Frau, welche die Frage getan
hatte, wurde gewonnen, samt den Ihrigen; und des Herrn  Kraft
und Herrlichkeit leuchtete über allen.

Hierauf kehrte ich zurück nach Nottinghamshire und ging
ins \textit{Bale of Beavor}. Unterwegs predigte ich den Leuten Buße
und es wurden viele gewonnen, im \textit{Bale of Beavor} und in den
Städten; denn ich blieb einige Wochen dort. Eines Morgens,
als ich am Feuer saß, kam eine große Wolke über mich, und eine
große Versuchung\index{Versuchung} überkam mich; aber ich 
blieb ganz ruhig. Und \index{Vision}
ich hörte eine Stimme zu mir sagen: \zitat{Alle Dinge gehen aus der
Natur hervor}; und die Elemente und die Sterne kamen über
mich, so das ich ganz davon eingehüllt war. Aber die andern
im Hause merkten nichts von all dem, weil ich ganz still und
ruhig war. Und weil ich still und ruhig war und wartete, so
% \picinclude{./010-019/p_s019.jpg} 
stieg eine lebendige Hoffnung in mir auf, und ich Vernahm deutlich
eine Stimme, welche sagte: \zitat{Es gibt einen lebendigen Gott, der
alle Dinge geschaffen hat}; und sogleich verschwand die Wolke
und auch die Versuchung, und Leben breitete sich über alles; mein
Herz ward fröhlich und ich pries den lebendigen Gott. Einige
Zeit darauf traf ich etliche, die behaupteten, es gebe keinen Gott,
sondern alle Dinge gehen aus der Natur hervor.
\index{Atheisten} Ich hatte einen
langen Disput mit ihnen und brachte sie herum, so das mehrere
zugaben, es gebe einen lebendigen Gott. Da sah ich, das es gut
gewesen war, das ich jene Prüfung durchgemacht hatte. Wir
hatten große Versammlungen in jenen Gegenden, denn die Kraft
des Herrn  brach hervor in diesem Teil des Landes. Als ich nach
Nottinghamshire zurück kam, traf ich eine Schar von verworrenen
Baptisten\index{Baptisten} und Anden; die Kraft des Herrn  
wirkte mächtig und
gewann viele unter ihnen. Darauf ging ich in die Umgegend von
Mansfield, wo die Kraft des Herrn  herrlich kund ward, in der
Stadt Mansfield\ort{Mansfield} und auch in anderen Städten. 
In Derbshire
wirkte sie in herrlicher Weise. In Eton in der Nähe von Derby
war eine Versammlung von Freunden; die Kraft des Herrn  tat sich
darin so mächtig kund, das viele gewaltig erschüttert wurden, und
vieler Mund wurde aufgetan durch die Kraft des Herrn . Viele wurden
vom Herrn  getrieben in die Turmhäuser zu gehen, zu den Priestern
und zum Volk, um ihnen die ewige Wahrheit zu verkünden.

Einmal als ich in Mansfield war, fand eine Sitzung der
Richter wegen des Dingens von Dienstboten statt. Es trieb
mich hinzugehen und den Richtern zu sagen, sie sollten die
Dienstboten nicht am Lohn verkürzen. Ich kam in die Nähe
der Herberge, in der die Sitzung abgehalten wurde; aber
als ich dort eine Musikantenbande traf, ging ich nicht hinein,
sondern gedachte am folgenden Morgen wieder zu kommen, hoffend,
sie dann in ernster Stimmung zu treffen, um mit ihnen zu 
verhandeln; denn es schien mir jetzt nicht die geeignete Zeit. Aber
als ich am Morgen kam, war alles fort; da wurde mir ganz
schwarz vor den Augen, so das ich fast nichts mehr sah; ich fragte
den Wirt, wo die Richter an dem Tage Sitzung haben würden;
er sagte mir, in einer etwa acht Meilen entfernten Stadt. Nun
fing ich wieder an zu sehen und lief dorthin, so schnell ich konnte;
als ich zu dem Haus kam, in dem sie und ihre zahlreiche 
Dienerschaft waren, mahnte ich die Richter, die Dienstboten nicht am
% \picinclude{./020-029/p_s020.jpg} 
Lohn zu verkürzen, sondern ihnen zu geben, was recht und billig
sei, und die Dienstboten ermahnte ich, ihre Pflicht zu tun und
ehrlich zu dienen; sie nahmen meine Mahnungen freundlich auf,
denn ich wurde vom Herrn dazu getrieben.

Ferner trieb es mich, an verschiedene Gerichtshöfe und in 
verschiedene Turmhäuser in Mansfield und an andern Orten zu gehen,
um alle zu ermahnen vom Unterdrücken und vom Schwören 
abzulassen und sich von der Ungerechtigkeit zum Herrn zu bekehren und
recht zu tun. Insbesondere trieb es mich, nach einer 
Gerichtsverhandlung in Mansfield zu einem zu gehen, der einer der 
schlechtesten Menschen der dortigen Gegend war, und mit ihm zu reden;
er war ein Säufer\index{Säufer} und berüchtigte: Mädchenhändler; 
ich warnte ihn
beim allmächtigen Gott wegen seines schlechten Wandels; als ich 
ausgeredet hatte und ihn Verlassen wollte, lies er mir nach und sagte
mir, während ich mit ihm gesprochen habe, sei er so ergriffen worden,
das ihn seine Kräfte ganz verließen. So wurde dieser Mann bekehrt, 
und er lies ab von seiner Schlechtigkeit und blieb rechtschaffen
und nüchtern zum Erstaunen aller, die ihn vorher gekannt hatten.
Und das Werk des Herrn nahm zu und viele kamen von der Finsternis,
zum Licht, im Laufe dieser drei Jahre 1646\jahr{1646}, 
1647\jahr{1647} und 1648\jahr{1648}.
Es wurden in dieser Zeit mehrere Versammlungen für Freunde 
eingerichtet, damit Gott sich kund tue durch sein Licht, seinen Geist
und seine Kraft; denn des Herrn  Kraft brach immer herrlicher hervor.

Nun war ich im Geiste bei dem flammenden Schwert vorbei
ins Paradies Gottes eingedrungen. Alle Dinge waren wie 
umgewandelt für mich und die ganze Schöpfung hatte einen andern
Geruch für mich, über alles was Worte ausdrücken können. Ich
wusste nur noch von Reinheit, Unschuld und Rechtschaffenheit, denn
ich war erneuert zum Ebenbild Gottes 
(Col. 3:10\bibel{Col. 03:10@Col. 3:10}) durch Christus,
in den Zustand, in dem Adam vor dem Fall gewesen 
war\index{Erbsünde}\index{Sündlosigkeit}. Die
ganze Schöpfung wurde mir offenbar und es wurde mir gezeigt,
wie alle Dinge mit dem Namen genannt wurden, der ihrem
Wesen und ihren Kräften entsprach. Ich war unschlüssig, ob ich
nicht sollte Heilkunde\index{Heilkunde} treiben zum Nutzen 
der Menschheit, als ich
sah, wie die Natur und die Kräfte aller Dinge mir so geoffenbart
wurden vom Herrn. Aber alsbald wurde ich ergriffen im Geist
und erkannte einen andern, sicherem Zustand als die Sündlosigkeit 
Adams, den Zustand Jesu Christi, der nicht fallen konnte.
Und der Herr zeigte mir, das die, so ihm treu bleiben im Licht
% \picinclude{./020-029/p_s021.jpg} 
und in der Kraft Christi, erhoben werden in den Zustand, darin
Adam vor dem Fall gewesen war, in welchem die bewundernswerten 
Werke der Schöpfung und ihre Kräfte erkannt werden
können durch die Offenbarung des göttlichen Wortes der 
Weisheit und der Kraft, durch welche sie gemacht waren. Der Herr
führte mich in große Dinge ein, und wunderbare Tiefen wurden
mir geoffenbart, die alles übertrafen, was Worte beschreiben
können. Aber wer sich dem Geist Gottes unterwirft und 
hineinwächst in das Ebenbild und die Kraft des Allmächtigen, der wird
das Wort der Weisheit empfangen, das alle Dinge offenbar macht,
und wird dazu gelangen, die verborgene Einheit in dem ewigen
Wesen zu erkennen.
So reiste ich umher im Dienste des Herrn , wie mich der
Herr führte. Als ich nach Nottingham kam, war Gottes mächtige
Kraft mit den Freunden. Von da ging ich nach Elawson in
Leieestershire im Tale Veavor, und auch dort wirkte die Kraft
Gottes in Verschiedenen Städten und Dörfern, in denen Freunde
beisammen waren. Während ich dort war, offenbarte mir der
Herr drei Dinge, die sich auf die drei großen Berufsarten in der
Welt — Heilkunde, sogenannte Gottesgelehrtheit und Recht?--H
wissenschast bezogen. Er zeigte mir, das die Ärzte nicht die
Wei?-heit Gottes haben, durch die alle Kreatur geschaffen ist, und
das sie darum ihre Kräfte nicht kennen, weil sie nicht im Worte der
Weis-heit sind, durch das alles gemacht ist. Gr zeigte mir, das
die Priester nicht den wahren Glauben haben, dessen Ursprung
Christus ist; den Glauben, der reinigt und den Sieg gibt und
durch des man Gott gefällt, welches Geheimnis des Glaubens
in reinem Gewissen ist (1. Tim. 3, 9). Gr zeigte mir ferner, das
die Rechtsgelehrten nicht die wahre Villigkeit und Gerechtigkeit
besitzen und nicht das Geses Gottes haben, nach welchem schon
die erste Ubertretung und alle weiteren Sünden gerichtet worden
sind und welches dem Geiste Gottes entspricht, den die Menschen
in sich betrüben und gegen den sie sündigen (Eph. 4, 30).
Und das diese drei, die Ärzte, die Priester und die Rechtsgelehrten,
die Welt ohne Weisheit regieren, ohne Glauben, ohne Billigkeit,
ohne Recht und ohne das Geses Gottes; die einen, indem sie
vorgeben, den Leib zu heilen, die andern die Seele und die dritten
das Eigentum der Leute zu schützen. Aber ich sah, das sie alle
die Weisheit, den Glauben, die Gerechtigkeit und das GesetzZGotteS


% \picinclude{./020-029/p_s022.jpg} 
nicht hatten. Und als der Herr mir diese Dinge osfenbarte, fühlte
ich, das seine Kraft sich über alle ergos und das sie durch die-
selbe alle umgewandelt werden könnten, wenn sie sie aufnehmen und
sich ihr beugen würden. Die Priester würden umgewandelt werden
und zum wahren Glauben kommen, welcher eine Gabe Gottes
ist. Die Rechtsgelehrten würden umgewandelt werden und zum
Geses Gottes (Jar. 2, 2) kommen, welches dem göttlichen im
Herzen entspricht und es möglich macht, seinen Nächsten wie sich
selbst zu lieben. Dieses Geses läst den Menschen erkennen, das
wenn er seinem Nächsten schadet, so schadet er sich selber, und
es lehret ihn, andern zu tun, wie er möchte, das die andern ihm
tun. Die Ärzte können umgewandelt werden und zur Weisheit
Gottes kommen, durch die alle Dinge geschaffen sind, und so
eine rechte Erkenntnis über diese Dinge erlangen und ihre Kräfte
erkennen an den Namen, die die Weisheit, die sie gemacht, ihnen
gab ....
Der Herr offenbaite mir durch seine unsichtbare Kraft, das
ein jeder erleuchtet werde durch das heilige Licht Christi (Joh. 1, 9).
Und ich erkannte, das es in allen leuchtet, und das alle, die
daran glaubten, aus der Verdammnis zum Licht des Lebens
kamen und Kinder des Lichts wurden (Joh. 12, 36). Aber die,
welche es hasten und nicht daran glaubten, die verdammte es, wie-
wohl sie schienen Christum zu bekennen.,« Solches sah ich in der
reinen Offenbarung des Lichts, ohne jegliche menschliche Hilfe;
auch wuste ich damals nicht, wo es in der Schrift zu sinden
war; doch später, als ich in der Schrift forschte, fand ich es.
Damals aber hatte ich jenes Licht und jenen Geist geschaut, welche
gewesen, ehe die Schrift gegeben worden war, und welche die
heiligen Männer Gottes getrieben hatten, die Schrift zu schreiben;
und ich erkannte, das alle, welche Gott, Christus oder die Schrift
recht kennen wollen, zu diesem Geist gelangen müssen. Aber ich
merkte eine Trägheit und faule Schläfrigkeit in den Leuten, die
mich erstaunte; oftmals, wenn ich einschlafen wollte, schweifte
mein Geist über alles hinaus zu dem, der von Ewigkeit zu Ewig-
keit ist. Ich sah, das der Tod über diesen schltisrigen und faulen
Zustand kommen musste, und ich sagte den Leuten, sie müsten
dazu kommen, dieses schläfrige, träge Wesen zu töten und zu
kreuzigen durch die Kraft Gottes, damit ihre Herzen und Sinne
droben seien.


% \picinclude{./020-029/p_s023.jpg} 
Erste Versammlungen und Proteste. 23
Einmal als ich durchs Feld wanderte, sagte der Herr zu mir:
,,Dein Name ist geschrieben im Lebensbuche des Lammeö, welcheö
gewesen vor der Erschaffung der Welt«. Alk- der Herr dies sagte,
da glaubte ich e3 und erkannte es, kraft der neuen Geburt. Einige
Zeit darauf befahl mir der Herr, in die Welt hinaus zu gehen,
die wie eine dornige Wildnis war; und als ich in der Kraft
Gottes mit dem Wort des Lebenö in die Welt hinaus kam, lehnte
sich die Welt dagegen auf und tobte wie die großen tobenden
Wogen der See; Priester wie ,,Fromme«, die Obrigkeit wie das
Volk, alle waren wie die See, als ich kam, den Tag des Herrn 
unter ihnen zu verkünden und ihnen Buse zu predigen ......
Als mich Gott und sein Sohn Jesus Christus aussandten
in die Welt, um sein ewigeö Evangelium und Reich zu predigen,
freute ich mich, das ich den Befehl hatte, die Leute jenem innern
Licht, Geist und Gnade zuzuführen, durch die alle ihr Heil und
den Weg zu Gott erkennen können; ja, jenem heiligen Geist,
der in alle Wahrheit führt und von welchem ich bestimmt wuste,
das er nie jemanden trtigt.
Durch diese göttliche Kraft und den Geist Gottes und das
Licht Jesu sollte ich nun die Menschen von ihren eigenen Wegen
ab zu Christus?-, dem neuen, lebendigen Weg bringen; ab von
ihren Kirchen von Menschen gemacht, zur Kirche in Gott, zur
Gemeinde derHeiligen, die imHi1nmel angeschrieben ist (Gbr. 12, 23),
deren Haupt Ehristus ist; ab von den Lehrern dieser Welt, die
von Menschen eingesetzt sind, damit sie von Ehristus:3 lernen, der
der Weg, die Wahrheit und das Leben ist (Joh. 14, 6), von welchem
der Vater sagt: ,,dieS ist mein lisber Sohn, den höret« (Luc. 9, 35);
ab von allem weltlichen Gottezdienst, damit sie den Geist der Wahr-
heit in ihrem Inneren erkennen und sich von demselben führen
lassen; das sie in demselben den Vater der Geister anbeten, dem
solches anbeten angenehm ist; die, welche nicht in diesem Geiste
anbeten, wissen nicht, mas sie anbeten. Ich sollte die Menschen
abbringen von all den Gottesdiensten dieser Welt, welche eitel
sind, damit sie zu dem wahren Gottesdienst kommen, welcher die
Witwen und Waisen in ihrer Trübsal tröstet (Jar. 1, 27) und be-
wahret von der Befleckung der Welt; dann gäbe es:) nicht so viele
Bettler, deren Anblick so ost mein Herz betrübt, weil er von so
viel Hartherzigkeit zeugt unter denen, die vorgeben, C-hristus3 zu
bekennen. Ich sollte sie von allen Gemeinschaften, Singereien


% \picinclude{./020-029/p_s024.jpg} 
und Betereien dieser Welt abbringen, welche Formen ohne Kraft
sind, auf das ihre Gemeinschaft im heiligen Geist sei, im ewigen
Geist Gotteö, und sie darin anbeten und singen, durch die Gnade,
die von Christus kommt; und so dem Herrn in ihren Herzen
singen’und spielen, der seinen geliebten Sohn gesandt hat, um
ihr Retter zu sein; der seine himmlische Sonne über und in allen
scheinen läst und seinen himmlischen Regen über Gerechte und
Ungerechte ausgiest (Matth. 5), wie der äusere Regen über alle
fällt und die äusere Sonne fiir alle scheint; dies ist Gotteö un-
aussprechliche Liebe zur Welt. Ich sollte die Leute von den
jüdischen Zeremonien abbringen und von den heidnischen Fabeln
und den menschlichen Einrichtungen und weltlichen Lehren, durch
welche die Leute hin und her von einer Sekte zur andern ge-
trieben werden, und von allen ihren bettelhaften Lehranstalten
und ihren Schulen und Hochschulen, in denen sie Prediger Christi
machen wollen, die aber wahrlich Prediger ihrer eigenen Machen-
schaft sind und nicht Christi; von allen ihren Bildern und Kreuzen
und Besprengen von Kindern; allen ihren sogenannten heiligen
Tagen und nichtigen Traditionen, die sie seit den Tagen der
Apostel eingerichtet haben und gegen welche die Kraft Gottes
sich richtet; vermöge dieser Kraft wurde ich getrieben, gegen
alles das aufzutreten und gegen alle, die nicht umsonst pre-
digten und doch solche waren, die umsonst vom Herrn  empfangen
hatten. s
Ferner verbot mir der Herr, als er mich in die Welt hinauö
sandte, meinen Hut abzunehmen vor irgendjemand, hoch oder
niedrig; und ich hatte den Befehl, zu allen, Männern und Frauen,
,,Du« zu sagen, ohne irgend einen Unterschied zu machen zwischen
reich oder arm, gros oder klein; und ich sollte unterwegs- auf
meinen Reisen den Leuten nicht guten Morgen oder guten Abend
sagen, noch mich vor irgendjemand neigen oder das Knie beugen.
Solcheö machte die Sekten und Gemeinschaften zornig. Aber die
Kraft des Herrn  half mir durch alles hindurch, zu seiner Ehre,
und viele kehrten sich in kurzer Zeit zu Gott, denn der große
Tag des- Herrn  ging auf aus der Höhe und brach eilendö an,
und in seinem Lichte gingen vielen die Augen über ihren Zu-
stand auf.
Aber o, die Wut, in welcher damals- Priester, Obrigkeit,
»Fromme« und andere waren! Aber hauptsächlich die Priester


% \picinclude{./020-029/p_s025.jpg} 
Erste Versammlungen und Proteste. 25
und die ,,Frommen«; denn obgleich das- ,,Du« gegen eine ein-
zelne Person ihrer eigenen Grammatik und Formenlehre, sowie
auch der Bibel entsprach, so konnten sie sich doch nicht drein
finden, es zu hören; und mas die Hut-Ehre anbetraf, das ich
den Hut nicht vor ihnen abnehmen konnte, das machte sie ganz
wütend ....
In jener Zeit fühlte ich mich, zu meiner schweren Prüfung,
auch berufen, in die Gerichtshöfe zu gehen, um nach Gerechtigkeit
zu schreien und die Richter und Behörden in Wort und Schrift
zur Gerechtigkeit zu mahnen; ich musste solche, die öffentliche Gast-
häuser hielten, ermahnen, den Leuten nicht mehr zu trinken zu
geben, als ihnen gut sei; ich musste auftreten gegen ihre Feste
und Gelage, Spiele, Späse und Belustigungen aller Art, durch
die die Leute zur Eitelkeit und Liederlichkeit verleitet und von
der Gottessurcht abgebracht wurden; am häufigsten schändeten
sie Gott (Röm. 2, 23) in dieser Weise an den Tagen, die sie als-
heilige bezeichneten. Auch an Jahrmärkten und Märkten musste
ich mich gegen ihr trügerischeö Handeln wenden, ihren Schwindel
und Betrug; ich musste sie mahnen, die Wahrheit zu sagen, ihr
ja—ja und ihr neinsnein sein zu lassen, und andern zu tun, wie
sie wollten, das man ihnen tue, alles indem ich sie an den großen
Tag des Herrn  erinnerte, der über sie alle kommen werde. Auch
gegen allerlei Musizieren und gegen die Schwindler, die in den
Vuden ihr Wesen trieben, musste ich auftreten, denn sie gefähr-
deten die Unschuld und reizten den Sinn der Leute zur Eitelkeit.
Ich musste auch manchen schweren Gang zu Lehrern und Lehrerinnen
tun, um sie zu erinahnen, die Kinder in der Furcht deö Herrn zu
erziehen, damit sie nicht in Eitelkeit, Leichisinn und Schlechtigkeit
aufwachsen. Ebenso musste ich Lehrer und Lehrerinnen, sowie die
Väter und Mütter ermahnen, darauf zu achten, das man die
Kinder und die Dienstboten daheim im Hanse zur Gottesstircht an-
halte, damit sie Vorbilder der Tugend und Mäsigkeit werden.
Die irdische Gesinnung der Priester tat mir weh, und wenn
ich die Glocken läuten hörte, welche die Leute inö Turnthaus
rufen sollten, ging es mir durch Mark gund Bein, denn eS war
gerade wie eine Marktglocke, welche die Leute zusammenruft, das I
der Priester seine Ware Izum Verkauf ausbieteu kann. O, die
großen Geldsummen, die zusammenkamen durch ihr Handeln mit
Bibeln und durch ihr Predigen, vom höchsten Bischof biz zum


% \picinclude{./020-029/p_s026.jpg} 
einfachsten Priester! Wa;-’ für ein Handel in der Welt kommt
diesem gleich! Und doch wurde die Schrift gegeben umsonst! Und
Christus hatte seinen Jüngern befohlen, umsonst zu predigen;
und die Propheten und Apostel verkündeten allen geizigen Miet-
lingen und allen, die für Geld iveiösagten, das Gericht. Jch
aber wurde au?-gesandt, in diesem freien Geist das Wort vom
Leben und der Versöhnung umsonst zu predigen, auf das alle zu
Christus kommen, welcher umsonst gibt und in das E-benbild
Gottes erneuert, nach dem Mann und Weib geschaffen waren
vor dem Fall, auf das sie himmlische Güter in Jesus Ehristus
haben möchten.