%%%%%%%%%%%%%%%%%%% Kapitel 2. %%%%%%%%%%%%%%%%%%%%%%%%%%%%%%
\chapter[Erste Versammlungen]{Erste Versammlungen}

\begin{center}
\textbf{Erste Versammlungen und Proteste.}
\end{center}

Die Macht dez Herrn hatte nun, im Jahre 1648, schon vielen
die Herzen geöffnet, daß ste daß Wort des Lebenß und der Ver:
söhnung aufnahmen. A15 ich nun einmal im Hause eineß Freunde?-,
in Nottinghamshire, saß, erkannte ich, daß ein großeß Krachen
durch die ganze Erde gehen mußte und ein großer Rauch auf-
steigen, überall wo es krachte, und darnach würde ein großes
Beben entstehen: es war die Erde in der Menschen Herzen, die
erbeben mußte, bevor der Same Gotteß au-es der Erde hervor-
gehen konnte. Und so geschah eö: die Macht dez Herrn fing an,
sie erbeben zu machen, und wir fingen an, große Versammlungen
zu haben, und man spürte die mächtige Kraft und daß Wirken
Gottes unter den Leuten, zu ihrer und der Priester Erstaunen ....
Jch ging nach Manöfield, wo eine große Versammlung von
,,Frommen« und andern Leuten stattfand; da trieb etz mich zu
beten, und die Kraft des- Herrn war so mächtig, daß etz schien,
als ob das- ganze Hauö erbebte. Alö ich geendet, sagten etliche
der Frommen, es sei gerade wie in den Tagen der Apostel, da
sich ,,daT- Hauß bewegte, in dem sie versammelt waren« (Act. 2, 2).
Nachdem ich gebetet, wollte einer der ,,Frommen« beten, aber
dadurch kam eine Trübung und etwas toteß über sie und die
andern ,,Frommen« wurden betrübt über ihn und sagten, ez sei
eine Versuchung über ihn gekommen; darauf kam er zu mir und
bat mich, ich solle wieder beten, aber ich konnte nicht auf eineiz
Menschen Geheiß beten.
Bald darauf war abermalß eine Versammlung von ,,Frommen«


% \picinclude{./010-019/p_s017.jpg} 
Erste Versammlungen und Proteste. 17
und ein Hauptmann namenßt Stoddard wohnte ihr bei. Sie
redeten über das Blut Christi, und während sie darüber sprachen,
sah ich durch die unmittelbare Offenbarung des unsichtbaren
Geistes das Blut Christi. Und ich schrie auf und rief: »Seht
ihr nicht das Blut Christi? Seht in eure Herzen, wie ee eure
Herzen und Gewissen besprengt, daß sie, loß von den toten
Werken, dem lebendigen Gott dienen« (Gbr. 9). Denn ich sah
ez, das Blut dez neuen Testamenteß, wie ez ins Herz kam. Daß
erschreckte die »Frommen«; sie wollten daß Blut nur aus?-wendig,
nicht inwendig haben. Aber Hauptmann Stoddard war ergriffen
und sagte: ,,Laßt den Jüngling reden, hört ihn an«, alß er sah,
wie sie mich mit vielen Worten zu besiegen suchten.
ES waren auch eine Anzahl Priester da, die ster gottselig
galten; einer von ihnen hieß Kellett, und etliche, die empfänglichen
Gemüteö waren, gingen hin, um sie zu hören. GS trieb mich,
ihnen nachzugehen, um sie zu ermahnen, auf die Lehre Gotteß in
ihrem Jnnern zu hören. Damals war der Priester Kellett gegen
das Priesteramt; später jedoch nahm er selbst ein solchetz an und
wurde ein Verfolger.
Nachdem ich etliche Arbeit getan hatte in dieser Gegend,
ging ich durch Derbshire in meine Heimat Leicestershire, und
ez wurden mehrere, die empsänglich waren, gewonnen. A15 ich
von dort wegzog, begegnete ich einer großen Zusammenkunft
von »Frommen«, die im Freien beteten und die Schrift auß-
legten. Sie reichten mir die Bibel und ich öffnete sie beim 5. Kap.
des Jtzzatth., wo Ehristuß daß; Gesetz auölegt; und ich erklärte
ihnen en inneren Zustand und den äußeren Zustand worüber sie
in heftigen Streit gerieten und so auszeinandergingen; aber die
Kraft des Herrn nahm überhand.
Darauf hörte ich von einer großen Versammlung, die in
Leicester stattfinden würde; eß sollte eine Die-putation geben, die die
Preßbhterianer, Jndependenten, Baptisten und Common-Payen
Leute gleicherweise angehen sollte. Die Versammlung war in einem
Turmhause, und der Herr trieb mich, dorthin zu gehen und
zugegen zu sein. Jch hörte ihren Verhandlungen und Beweis-
führungen zu. Einige saßen in Kirchenstühlen und der Priester
war aus der Kanzel; es war eine große Menge versammelt.
Zuletzt tat eine Frau eine Frage über die Stelle bei Petrus:
»Wiedergeboren auß ewiglichem Samen, auö dem lebendigen Wort
George Ft--. 2



% \picinclude{./010-019/p_s018.jpg} 
Gottes-, das-8 ewiglich bleibet (1. Petr. 1). Der Priester sagte ihr:
,,Jch erlaube keiner Frau in der Kirche zu reden,« obgleich er
vorher allen die Freiheit erteilt hatte, zu reden. Da wurde ich-
von der Krast dez Herrn übermannt wie in einer Verziickung,
und ich erhob mich und fragte den Priester: »Nennst du dieß
hier, dieseö Turmhauz, eine ,,Kirche«? oder nennst du diese
bunte Menge eine Kirche?« Denn er hätte der Frau auf ihre
Frage antworten sollen, nachdem er vorher allen die Freiheit
erteilt hatte, zu reden. Anstatt mir zu antworten, sragte er mich:
maß eine Kirche sei. Ich sagte: »Die Kirche ist der Pfeiler und
Grund der Wahrheit, auß lebendigen Steinen gemacht, aus
lebendigen Gliedern (1. Petr. 2), eine geistige Haußgemeinde,
deren Haupt Christus- ift; aber er ist nicht daS Haupt einer bunten
Menge oder eines alten Hauseß auß Kalk, Steinen und Holz.«
Diese Worte brachten alleß auß Rand und Band; der Priester
kam auß seiner Kanzel, andere auö ihren Stühlen, und die Ver-
handlungen waren gestört. Ich ging in eine große Herberge und
dißputierte dort mit Priestern und ,,Frommen« aller Richtungen;
und alle waren furchtbar hitzig. Aber ich bestand auf der wahren
Kirche und ihrem wahren Haupt, trotz ihnen allen, bi-3 sie nach-
gaben und auöeinanderstoben. Einer schien sehr geneigt und kam
eine Zeit lang, in der Absicht, sich mir anzuschließen; aber bald
kehrte er sich ganz gegen mich und schloß sich einem Priester an,
trat für die Kindertanse ein, obgleich er vorher selber ein
Baptist gewesen war, und verließ mich. Aber etz wurden an dem
Tage etliche gewonnen; auch die Frau, welche die Frage getan
hatte, wurde gewonnen, samt den Jhrigen; und deö Herrn Kraft
und Herrlichkeit leuchtete über allen.
Hierauf kehrte ich zurück nach Nottinghamshire und ging
ink- Vale of Beavor. Unterwegß predigte ich den Leuten Buße
und ez wurden viele gewonnen, im Vale of Beavor und in den
Städten; denn ich blieb einige Wochen dort. Eineß Morgen?-,
alß ich am Feuer saß, kam eine große Wolke über mich, und eine
große Versuchung überkam mich; aber ich blieb ganz ruhig. Und
ich hörte eine Stimme zu mir sagen: »Alle Dinge gehen auß der
Natur heroor«; und die Elemente und die Sterne kamen über
mich, so daß ich ganz davon eingehiillt war. Aber die andern
im Hause merkten nichtß von all dem, weil ich ganz still und
ruhig war. Und weil ich still und ruhig war und wartete, so


% \picinclude{./010-019/p_s019.jpg} 

Erste Versammlungen und Proteste. 19
stieg eine lebendige Hoffnung in mir auf, und ich Vernahm deutlich
eine Stimme, welche sagte: ,,EZ gibt einen lebendigen Gott, der
alle Dinge geschaffen hat«; und sogleich verschwand die Wolke
und auch die Versuchung, und Leben breitete sich über alles; mein
Herz ward fröhlich und ich prieö den lebendigen Gott. Einige
Zeit darauf traf ich etliche, die behaupteten, ez gebe keinen Gott,
sondern alle Dinge gehen aus-3 der Natur hervor. Ich hatte einen
langen Di?-put mit ihnen und brachte sie herum, so daß mehrere
zugaben, es gebe einen lebendigen Gott. Da sah ich, daß etz gut
gewesen war, daß ich jene Prüfung durchgemacht hatte. Wir
hatten große Versammlungen in jenen Gegenden, denn die Kraft
deß Herm brach hervor in diesem Teil deß Landeö. A13 ich nach
Nottinghamshire zurück kam, traf ich eine Schar von verworrenen
Baptisten und andem; die Kraft des Herrn wirkte mächtig und
gewann viele unter ihnen. Darauf ging ich in die Umgegend von
Manöfield, wo die Kraft deß Herrn herrlich kund ward, in der
Stadt Manßfield und auch in anderen Städten. Jn Derbshire
wirkte sie in herrlicher Weise. Jn Eton in der Nähe von Derby
war eine Versammlung von Freunden; die Kraft dez Herrn tat sich
darin so mächtig kund, daß viele gewaltig erschüttert wurden, und
vieler Mund wurde aufgetan durch die Kraft dez Herrn. Viele wurden
vom Herrn getrieben in die Turmhäuser zu gehen, zu den Priestern
und zum Volk, um ihnen die ewige Wahrheit zu verkünden.
Einmal als- ich in Man?-field war, fand eine Sitzung der
Richter wegen dez Dingenö von Dienstboten statt. ES trieb
mich hinzugehen und den Richtern zu sagen, sie sollten die
Dienstboten richt am Lohn verkürzen. Jch kam in die Nähe
der Herberge, in der die Sitzung abgehalten wurde; aber
alL ich dort eine Musikantenbande traf, ging ich nicht hinein,
sondern gedachte am folgenden Morgen wieder zu kommen, hofsend,
sie dann in ernster Stimmung zu treffen, um mit ihnen zu ver-
handeln; denn ez schien mir jetzt nicht die geeignete Zeit. Aber
alö ich am Morgen kam, war alleö fort; da wurde mir ganz
schwarz vor den Augen, so daß ich fast nichtß mehr sah; ich fragte
den Wirt, wo die Richter an dem Tage Sitzung haben würden;
er sagte mir, in einer etwa acht Meilen entfernten Stadt. Nun
fing ich wieder an zu sehen und lief dorthin, so schnell ich konnte;
altz ich zu dem Hauö kam, in dem sie und ihre zahlreiche Diener-
schaft waren, mahnte ich die Richter, die Dienstboten nicht am



% \picinclude{./020-029/p_s020.jpg} 
Lohn zu verkürzen, sondern ihnen zu geben, was recht und billig
sei, und die Dienstboten ermahnte ich, ihre Pflicht zu tun und
ehrlich zu dienen; sie nahmen meine Mahnungen freundlich auf,
denn ich wurde vom Herm dazu getrieben.
Ferner trieb etz mich, an verschiedene Gerichtöhöfe und in ver-
schiedene Turmhäuser in Manöfield und an andern Orten zu gehen,
um alle zu ermahnen vom Unterdrücken und vom Schwören abzu-
lassen und sich von der Ungerechtigkeit zum Herrn zu bekehren und
recht zu tun. Jnßbesondere trieb es mich, nach einer Gerichtßver-
handlung in Manöfield zu einem zu gehen, der einer der schlech-
testen Menschen der dortigen Gegend war, und mit ihm zu reden;
er war ein Säufer und berüchtigte: Mädchenhändler; ich warnte ihn
beim allmächtigen Gott wegen s eines schlechten Wandels; als ich auß-
geredet hatte und ihn Verlassen wollte, lies er mir nach und sagte
mir, während ich mit ihm gesprochen habe, sei er so ergriffen worden,
daß ihn seine Kräfte ganz verließen. So wurde dieser Mann be-
kehrt, und er ließ ab von seiner Schlechtigkeit und blieb rechtschaffen
und nüchtern zum Erstaunen aller, die ihn vorher gekannt hatten.
Und das Werk des Herrn nahm zu und viele kamen von der Finster- ,
nie zum Licht, im Laufe dieser drei Jahre 1646, 1647 und 1648.
GS wurden in dieser Zeit mehrere Versammlungen für Freunde ein-
gerichtet, damit Gott sich kund tue durch sein Licht, seinen Geist
und seine Kraft; denn dee Herrn Kraft brach immer herrlicher hervor.
Nun war ich ini Geiste bei Idem stammenden Schwert vorbei
inö Paradieß Gotteö eingedrungen. Alle Dinge waren wie um-
gewandelt ftir mich und die ganze Schöpfung hatte einen andern
Geruch für mich, über alles waß Worte außdrücken können. Ich
wußte nur noch von Reinheit, Unschuld und Rechtschaffenheit, denn
ich war erneuert zum Ebenbild Gotteß (Col. 3, 10) durch Christus,
in den Zustand, in dem Adam vor dem Fall gewesen war. Die
ganze Schöpfung wurde mir offenbar und es- wurde mir gezeigt,
wie alle Dinge mit dem Namen genannt wurden, der ihrem
Wesen und ihren Kräften entsprach. Jch war unschliisstg, ob ich
nicht sollte Heilkunde treiben zum Nutzen der Menschheit, als ich
sah, wie die Natur und die Kräfte aller Dinge mir so geoffenbart
wurden vom Herrn. Aber alsbald wurde ich ergriffen im Geist
und erkannte einen andern, sicherem Zustand als die Sitndlosig-
keit Adams, den Zustand Jesu Christi, der nicht fallen konnte.
Und der Herr zeigte mir, daß die, so ihm treu bleiben im Licht


% \picinclude{./020-029/p_s021.jpg} 
Erste Versammlungen und Proteste. 21
und in der Kraft Christi, erhoben werden in den Zustand, darin
Adam vor dem Fall gewesen war, in welchem die bewundernß-
werten Werke der Schöpfung und ihre Kräfte erkannt werden
können durch die Offenbarung deß göttlichen Worteß der Weiß-
heit und der Kraft, durch welche sie gemacht waren. Der Herr
führte mich in große Dinge ein, und wunderbare Tiefen wurden
mir geoffenbart, die alleß iibertrafen, waß Worte beschreiben
können. Aber wer sich dem Geist Gotteß unierwirst und hinein-
wächst in daß Gbenbild und die Kraft deß Allmächtigen, der wird
daß Wort der Weißheit empfangen, daß alle Dinge offenbar macht,
und wird dazu gelangen, die verborgene Einheit in dem ewigen
Wesen zu erkennen.
So reiste ich umher im Dienste deß Herrn, wie mich der
Herr führte. Alß ich nach Nottingham kam, war Gotteß mächtige
Kraft mit den Freunden. Von da ging ich nach Elawson in
Leieestershire im Tale Veavor, und auch dort wirkte die Kraft
Gotteß in Verschiedenen Städten und Dörfern, in denen Freunde
beisammen waren. Während ich dort war, offenbarte mir der
Herr drei Dinge, die sich auf die drei großen Berufßarten in der
Welt — Heilkunde, sogenannte Gotteßgelehrtheit und Recht?--H
wissenschast bezogen. Er zeigte mir, daß die Ärzte nicht die
Wei?-heit Gotteß haben, durch die alle Kreatur geschaffen ist, und
daß sie darum ihre Kräfte nicht kennen, weil sie nicht im Worte der
Weiß-heit sind, durch daß alleß gemacht ist. Gr zeigte mir, daß
die Priester nicht den wahren Glauben haben, dessen Ursprung
Christus ist; den Glauben, der reinigt und den Sieg gibt und
durch des man Gott gefällt, welcheß Geheimniß deß Glaubenß
in reinem Gewissen ist (1. Tim. 3, 9). Gr zeigte mir ferner, daß
die Rechtßgelehrten nicht die wahre Villigkeit und Gerechtigkeit
besitzen und nicht daß Gesetz Gotteß haben, nach welchem schon
die erste Ubertretung und alle weiteren Sünden gerichtet worden
sind und welcheß dem Geiste Gotteß entspricht, den die Menschen
in sich betrüben und gegen den sie sündigen (Eph. 4, 30).
Und daß diese drei, die Ärzte, die Priester und die Rechtßgelehrten,
die Welt ohne Weißheit regieren, ohne Glauben, ohne Billigkeit,
ohne Recht und ohne daß Gesetz Gotteß; die einen, indem sie
vorgeben, den Leib zu heilen, die andern die Seele und die dritten
daß Eigentum der Leute zu schützen. Aber ich sah, daß sie alle
die Weißheit, den Glauben, die Gerechtigkeit und daß GesetzZGotteS


% \picinclude{./020-029/p_s022.jpg} 
nicht hatten. Und als der Herr mir diese Dinge osfenbarte, fühlte
ich, daß seine Kraft sich über alle ergoß und daß sie durch die-
selbe alle umgewandelt werden könnten, wenn sie sie aufnehmen und
sich ihr beugen würden. Die Priester würden umgewandelt werden
und zum wahren Glauben kommen, welcher eine Gabe Gottes
ist. Die Rechtsgelehrten würden umgewandelt werden und zum
Gesetz Gottes (Jar. 2, 2) kommen, welches dem göttlichen im
Herzen entspricht und es möglich macht, seinen Nächsten wie sich
selbst zu lieben. Dieses Gesetz läßt den Menschen erkennen, daß
wenn er seinem Nächsten schadet, so schadet er sich selber, und
es lehret ihn, andern zu tun, wie er möchte, daß die andern ihm
tun. Die Ärzte können umgewandelt werden und zur Weisheit
Gottes kommen, durch die alle Dinge geschaffen sind, und so
eine rechte Erkenntnis über diese Dinge erlangen und ihre Kräfte
erkennen an den Namen, die die Weisheit, die sie gemacht, ihnen
gab ....
Der Herr offenbaite mir durch seine unsichtbare Kraft, daß
ein jeder erleuchtet werde durch das heilige Licht Christi (Joh. 1, 9).
Und ich erkannte, daß es in allen leuchtet, und daß alle, die
daran glaubten, aus der Verdammnis zum Licht des Lebens
kamen und Kinder des Lichts wurden (Joh. 12, 36). Aber die,
welche es haßten und nicht daran glaubten, die verdammte es, wie-
wohl sie schienen Christum zu bekennen.,« Solches sah ich in der
reinen Offenbarung des Lichts, ohne jegliche menschliche Hilfe;
auch wußte ich damals nicht, wo es in der Schrift zu sinden
war; doch später, als ich in der Schrift forschte, fand ich es.
Damals aber hatte ich jenes Licht und jenen Geist geschaut, welche
gewesen, ehe die Schrift gegeben worden war, und welche die
heiligen Männer Gottes getrieben hatten, die Schrift zu schreiben;
und ich erkannte, daß alle, welche Gott, Christus oder die Schrift
recht kennen wollen, zu diesem Geist gelangen müssen. Aber ich
merkte eine Trägheit und faule Schläfrigkeit in den Leuten, die
mich erstaunte; oftmals, wenn ich einschlafen wollte, schweifte
mein Geist über alles hinaus zu dem, der von Ewigkeit zu Ewig-
keit ist. Jch sah, daß der Tod über diesen schltisrigen und faulen
Zustand kommen mußte, und ich sagte den Leuten, sie müßten
dazu kommen, dieses schläfrige, träge Wesen zu töten und zu
kreuzigen durch die Kraft Gottes, damit ihre Herzen und Sinne
droben seien.


% \picinclude{./020-029/p_s023.jpg} 
Erste Versammlungen und Proteste. 23
Einmal alß ich durchs Feld wanderte, sagte der Herr zu mir:
,,Dein Name ist geschrieben im Lebenßbuche deß Lammeö, welcheö
gewesen vor der Erschaffung der Welt«. Alk- der Herr dietz sagte,
da glaubte ich e3 und erkannte es, kraft der neuen Geburt. Einige
Zeit darauf befahl mir der Herr, in die Welt hinaus zu gehen,
die wie eine dornige Wildniß war; und alß ich in der Kraft
Gottes mit dem Wort des Lebenö in die Welt hinauß kam, lehnte
sich die Welt dagegen auf und tobte wie die großen tobenden
Wogen der See; Priester wie ,,Fromme«, die Obrigkeit wie das
Volk, alle waren wie die See, als ich kam, den Tag deß Herrn
unter ihnen zu verkünden und ihnen Buße zu predigen ......
Als mich Gott und sein Sohn Jesuß Christuß außsandten
in die Welt, um sein ewigeö Evangelium und Reich zu predigen,
freute ich mich, daß ich den Befehl hatte, die Leute jenem innern
Licht, Geist und Gnade zuzuführen, durch die alle ihr Heil und
den Weg zu Gott erkennen können; ja, jenem heiligen Geist,
der in alle Wahrheit führt und von welchem ich bestimmt wußte,
daß er nie jemanden trtigt.
Durch diese göttliche Kraft und den Geist Gottes und daß
Licht Jesu sollte ich nun die Menschen von ihren eigenen Wegen
ab zu Christus?-, dem neuen, lebendigen Weg bringen; ab von
ihren Kirchen von Menschen gemacht, zur Kirche in Gott, zur
Gemeinde derHeiligen, die imHi1nmel angeschrieben ist (Gbr. 12, 23),
deren Haupt Ehristuß ist; ab von den Lehrern dieser Welt, die
von Menschen eingesetzt sind, damit sie von Ehristus:3 lernen, der
der Weg, die Wahrheit und daß Leben ist (Joh. 14, 6), von welchem
der Vater sagt: ,,dieS ist mein lisber Sohn, den höret« (Luc. 9, 35);
ab von allem weltlichen Gottezdienst, damit sie den Geist der Wahr-
heit in ihrem Jnnern erkennen und sich von demselben führen
lassen; daß sie in demselben den Vater der Geister anbeten, dem
solcheß anbeten angenehm ist; die, welche nicht in diesem Geiste
anbeten, wissen nicht, maß sie anbeten. Jch sollte die Menschen
abbringen von all den Gottesdiensten dieser Welt, welche eitel
sind, damit sie zu dem wahren Gotteßdienst kommen, welcher die
Witwen und Waisen in ihrer Trübsal tröstet (Jar. 1, 27) und be-
wahret von der Befleckung der Welt; dann gäbe es:) nicht so viele
Bettler, deren Anblick so ost mein Herz betrübt, weil er von so
viel Hartherzigkeit zeugt unter denen, die vorgeben, C-hristus3 zu
bekennen. Ich sollte sie von allen Gemeinschaften, Singereien


% \picinclude{./020-029/p_s024.jpg} 
und Betereien dieser Welt abbringen, welche Formen ohne Kraft
sind, auf daß ihre Gemeinschaft im heiligen Geist sei, im ewigen
Geist Gotteö, und sie darin anbeten und singen, durch die Gnade,
die von Christus kommt; und so dem Herm in ihren Herzen
singen’und spielen, der seinen geliebten Sohn gesandt hat, um
ihr Retter zu sein; der seine himmlische Sonne über und in allen
scheinen läßt und seinen himmlischen Regen über Gerechte und
Ungerechte außgießt (Matth. 5), wie der äußere Regen über alle
fällt und die äußere Sonne fiir alle scheint; dietz ist Gotteö un-
außsprechliche Liebe zur Welt. Jch sollte die Leute von den
jüdischen Zeremonien abbringen und von den heidnischen Fabeln
und den menschlichen Einrichtungen und weltlichen Lehren, durch
welche die Leute hin und her von einer Sekte zur andern ge-
trieben werden, und von allen ihren bettelhaften Lehranstalten
und ihren Schulen und Hochschulen, in denen sie Prediger Christi
machen wollen, die aber wahrlich Prediger ihrer eigenen Machen-
schaft sind und nicht Christi; von allen ihren Bildern und Kreuzen
und Besprengen von Kindern; allen ihren sogenannten heiligen
Tagen und nichtigen Traditionen, die sie seit den Tagen der
Apostel eingerichtet haben und gegen welche die Kraft Gottes
sich richtet; vermöge dieser Kraft wurde ich getrieben, gegen
alleß daß aufzutreten und gegen alle, die nicht umsonst pre-
digten und doch solche waren, die umsonst vom Herrn empfangen
hatten. s
Ferner verbot mir der Herr, als er mich in die Welt hinauö
sandte, meinen Hut abzunehmen vor irgendjemand, hoch oder
niedrig; und ich hatte den Befehl, zu allen, Männern und Frauen,
,,Du« zu sagen, ohne irgend einen Unterschied zu machen zwischen
reich oder arm, groß oder klein; und ich sollte unterwegs- auf
meinen Reisen den Leuten nicht guten Morgen oder guten Abend
sagen, noch mich vor irgendjemand neigen oder daß Knie beugen.
Solcheö machte die Sekten und Gemeinschaften zornig. Aber die
Kraft des Herrn half mir durch alleß hindurch, zu seiner Ehre,
und viele kehrten sich in kurzer Zeit zu Gott, denn der große
Tag des- Herrn ging auf auß der Höhe und brach eilendö an,
und in seinem Lichte gingen vielen die Augen über ihren Zu-
stand auf.
Aber o, die Wut, in welcher damals- Priester, Obrigkeit,
»Fromme« und andere waren! Aber hauptsächlich die Priester


% \picinclude{./020-029/p_s025.jpg} 
Erste Versammlungen und Proteste. 25
und die ,,Frommen«; denn obgleich das- ,,Du« gegen eine ein-
zelne Person ihrer eigenen Grammatik und Formenlehre, sowie
auch der Bibel entsprach, so konnten sie sich doch nicht drein
finden, es zu hören; und maß die Hut-Ehre anbetraf, daß ich
den Hut nicht vor ihnen abnehmen konnte, das machte sie ganz
wütend ....
In jener Zeit fühlte ich mich, zu meiner schweren Prüfung,
auch berufen, in die Gerichtßhöfe zu gehen, um nach Gerechtigkeit
zu schreien und die Richter und Behörden in Wort und Schrift
zur Gerechtigkeit zu mahnen; ich mußte solche, die öffentliche Gast-
häuser hielten, ermahnen, den Leuten nicht mehr zu trinken zu
geben, als ihnen gut sei; ich mußte auftreten gegen ihre Feste
und Gelage, Spiele, Späße und Belustigungen aller Art, durch
die die Leute zur Eitelkeit und Liederlichkeit verleitet und von
der Gotteßsurcht abgebracht wurden; am häufigsten schändeten
sie Gott (Röm. 2, 23) in dieser Weise an den Tagen, die sie als-
heilige bezeichneten. Auch an Jahrmärkten und Märkten mußte
ich mich gegen ihr trügerischeö Handeln wenden, ihren Schwindel
und Betrug; ich mußte sie mahnen, die Wahrheit zu sagen, ihr
ja—ja und ihr neinsnein sein zu lassen, und andern zu tun, wie
sie wollten, daß man ihnen tue, alleß indem ich sie an den großen
Tag dez Herrn erinnerte, der über sie alle kommen werde. Auch
gegen allerlei Musizieren und gegen die Schwindler, die in den
Vuden ihr Wesen trieben, mußte ich auftreten, denn sie gefähr-
deten die Unschuld und reizten den Sinn der Leute zur Eitelkeit.
Jch mußte auch manchen schweren Gang zu Lehrern und Lehrerinnen
tun, um sie zu erinahnen, die Kinder in der Furcht deö Herm zu
erziehen, damit sie nicht in Eitelkeit, Leichisinn und Schlechtigkeit
aufwachsen. Ebenso mußte ich Lehrer und Lehrerinnen, sowie die
Väter und Mütter ermahnen, darauf zu achten, daß man die
Kinder und die Dienstboten daheim im Hanse zur Gotteßstircht an-
halte, damit sie Vorbilder der Tugend und Mäßigkeit werden.
Die irdische Gesinnung der Priester tat mir weh, und wenn
ich die Glocken läuten hörte, welche die Leute inö Turnthauß
rufen sollten, ging es mir durch Mark gund Bein, denn eS war
gerade wie eine Marktglocke, welche die Leute zusammenruft, daß I
der Priester seine Ware Izum Verkauf außbieteu kann. O, die
großen Geldsummen, die zusammenkamen durch ihr Handeln mit
Bibeln und durch ihr Predigen, vom höchsten Bischof biz zum


% \picinclude{./020-029/p_s026.jpg} 
einfachsten Priester! Wa;-’ für ein Handel in der Welt kommt
diesem gleich! Und doch wurde die Schrift gegeben umsonst! Und
Christus hatte seinen Jüngern befohlen, umsonst zu predigen;
und die Propheten und Apostel verkündeten allen geizigen Miet-
lingen und allen, die für Geld iveiösagten, daß Gericht. Jch
aber wurde au?-gesandt, in diesem freien Geist daß Wort vom
Leben und der Versöhnung umsonst zu predigen, auf daß alle zu
Christus kommen, welcher umsonst gibt und in daß E-benbild
Gotteß erneuert, nach dem Mann und Weib geschaffen waren
vor dem Fall, auf daß sie himmlische Güter in Jesuß Ehristuß
haben möchten.