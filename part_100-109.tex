% \picinclude{./100-109/p_s100.jpg} 
denn es trieb ihn, das Wort des Lebens in Dunkirk und Holland
zu verkünden und in einigen Teilen Italiens, sogar in Rom.
Doch der Herr bewahrte ihn und seinen Begleiter John Stubbs
vor der Inquisition ....
An einem sechsten Wochentage hatte ich eine Versammlung
in Colchester, zu der viel ,,Fromme« und die Lehrer der Jndepen-
denken kamen. Als ich zu reden aufgehört hatte und meinen
. Platz verließ, fing einer der Jndependentenlehrer an Lärm zu
machen; Amor Stoddart, der dies hörte, sagte zu mir: ,,Steh
noch einmal aus, George,« denn ich hatte eben fort gehen wollen.
Jch stand nun wieder auf, als ich die lärmenden Jndependenten
hörte; und bald kam die Macht des Herrn über ihn und über
alle und überwältigte sie ....
Am nächsten Ersten Tage hatten wir eine große Versammlung
in Colchester .... Von da gingen wir nach Jpswich .... dann
nach Mendlesham, wo wir ein große Versammlung hatten; dann
gingen wir nach Norfolk, wo wir uns von Amor Stoddart ver-
abschiedeten, der uns später wieder treffen wollte ..... Dann
zogen wir nach Yarmouth .... und Norwich, .... rmd von
dort nach Lynn ..... Von Lynn gingen wir nach Ecmbtidge.
Als ich in diese Stadt kam, waren die Studenten, die von
meinem Kommen gehört hatten, in Aufregung und benahmen
sich sehr ungezogen; ich hielt mich auf meinem Pferde und ritt
mitten durch sie hindurch in der Kraft des Herm; Amor Stoddart
aber warfen sie vom Pferd, ehe er die Herberge erreichte. Als
wkr in der Herberge waren, taten sie so wüst im Hof und inden
Straßen, daß Fuhrleute und Kohlengräber nicht witster hätten
tun können. Die Wirtsleute fragten uns, was wir zum Nacht-
essen haben wollten; ich erwiderte: »wenn nicht Gottes Macht
größer wäre als diese rohen Studenten, so würden sie uns sicher
gerne in Stücke reißen rmd ein Nachtessen aus uns machen.« Sie
wußten, daß ich sehr gegen das Gewerbe des Predigens war, das
sie dort als Lehrjungen erlernen sollten; darum tobten sie gegen
-mich, wie nur je die Handwerksleute der Diana gegen den Paulus
(Act. 19). Ju der Nacht kam der Stadtbürgermeister, der es gut
mit mir meinte, und holte mich zu sich heim. Als wir durch
die Straße gingen, war großer Lärm in der Stadt, aber man
erkannte mich nicht, weil es finster war. Man war auch über
den Bürgermeister zornig, so daß er sich sehr fürchtete, mit mir


% \picinclude{./100-109/p_s101.jpg} 
Brief an den Papst. Die Studenten von Cambridge usw. 101
über die Straße zu gehen. Nachher ließen wir dann die Freunde
holen und hatten eine schöne Versammlung in der Kraft des
Herrn, und ich blieb die ganze Nacht in der Stadt. Wir hatten
unsere Pferde für den nächsten Morgen um sechs Uhr bestellt
und ritten in Frieden zur Stadt hinaus; die Störenfriede wurden
somit enttäuscht, denn sie hatten geglaubt, ich würde länger da
bleiben, und hatten beabsichtigt, uns etwas anzutun; aber unsre
frühe Abreise oernichtete ihre bösen Anschläge .....
GH trieb mich, ein Schreiben zu senden an die, welche über
I das Zittern und Beben (quake) spotteten:
,,Gin Wort vom Herrn an euch, die ihr über das Zittern
und Beben spottet; und die ihr solche, welche zittern und beben,
uerhöhnt, schlagt, bedroht und Verwünschungen gegen sie aus-
stoszet. Jhr kennet alle die Apostel und Propheten nicht! ....
Moses, der ein Richter über Jsrael war, zitterte und bebte,
als der Herr zu ihm sagte: »ich bin der Gott Abrahams, Jsaaks
und Jakobs-« (2. Mose 3) .... . Der König David zitterte;
und sie verspotteten ihn (Ps. 38). . . . . Hiob zitterte, bebte;
und sie oerlachten ihn (Hiob 21) ..... Der Prophet Jeremia
bebte; es schüttelte ihn, seine Glieder zitierten, und er taumelte
hin und her wie ein trunkener Mann (Jer. 23, 9), als er die
Betrügerei der Priester und Propheten sah, die sich vom Herrn
abgekehrt hatten ..... Jesaia sagte: ,,Höret was der Herr
sagt, ihr, die ihr erzittert bei seinem Wort;« und weiter sagte er:
,,Jch sehe an den Elenden und der zerbrochenen Geistes ist und
der erzittert bei meinem Wort« (Jes. 66, 2) .... Habakuk, der
Prophet des Herrn zitterte ..... Und Joel, der Prophet des
Herrn sagte: ,,Blaset mit der Posaune zu Zion, erzittert alle Ein-
wohner im Lande« (Joel 2, 1) ..... Daniel, ein Diener des
Allerhöchsten, zitterte, und er hatte keine Kraft mehr (Dan. 10, 16);
und er war gefangen, gehaßt und verfolgt .....
Paulus, ein Apostel Jesu Christi durch den Willen Gottes, ein
auserwähltes Rüstzeug des Herrn, daß er seinen Namen trage in
alle Lande, zitterte .... und sagte, als er zu den Eorinthern
kam: ,,ich war bei euch in Schwachheit und Furcht und großem
Zit—tetn« (1. Eorinth. 2, 3) .....
Hütet euch darum, ihr Großen der Erde, die zu Verfolgen,
welche man zum Spott Quäker (Zitterer) nennt, die aber in der
Kraft Gottes sind- damit sich die Hand des Herrn nicht gegen


% \picinclude{./100-109/p_s102.jpg} 
euch kehre und euch oerderbe. ES ergeht daß Wort dez Herrn
an euch: fürchtet euch und zittert und hütet euch! denn der Herr
siehet den cm, der erzittert bei seinen Wort (Jes. 66, 2); ihr aber,
die ihr von dieser Welt seid, oerspottet, oerlacht, oerhöhnt, ver-
folgt ihn und nehmt ihn gefangen. Daran könnt ihr sehen, daß
ihr den Propheten und Aposteln zuwider handelt, wenn ihr die hasfet,
die der Herr ansieht, während wir, die ihr im Spott Quäker
nennt, sie achten. Wir ehren und preisen die Macht, die den
Teufel erzittern macht, die Erde erbeben läßt und den Stolz und
Hochmut niederschmettert, die Tiere auf den Feldern erzittern
macht und die Erde wanken (Jes. 2, 11). Diese Kraft ehren
und verkünden wir; aber alle, die spotten und höhnen und
peitschen und plagen, die verabscheuen wir; denn alle, die solche-?
tun und es nicht bereuen, werden daß Reich Gotteö nicht ererben,
sondern daß Verderben (2. Tess. 1).
Selig aber sind, die um der Gerechtigkeit willen verfolgt
werden; sie werden ihren Lohn im Himmel haben (Matth. 5, 12).« . .
G. F.
JmJahre 1655 wurde der Abschwörungzeid gefordert, wodurch
viele Freunde zu leiden hatten; und viele gingen zum Protektor, um
mit ihm darüber zu sprechen; aber er fing an, härter zu werden.
Durch die Art, in der die gehässigen Beamten den Eid alö
Schlingen gebrauchten, um die Freunde darin zu fangen, weil sie
wußten, daß sie nicht schwören durften, nahmen die Leiden der
Freunde immer mehr zu, und es trieb mich, dem Protektor folgendes-
zu schreiben:
,,Die Obrigkeit soll daß Schwert, das den Übeltätern ein
Schrecken sein soll, nicht umsonst tragen; wie die Obrigkeit, die
daß Schwert umsonst trägt, den Übeltätern kein Schrecken ist, so
ist sie auch kein Zeichen deö Ruhmes für den, der recht tut;
Gott hat nun durch seine Macht ein Volk erwecket, welches
die Priester, die Obrigkeit und daß Volk in ihrem Ärger ,,Quäker«
nennen. Dieseö schreit gegen die Trunksucht und das Schwören;
die Trunkenbolde aber, denen das Schwert der Obrigkeit ein
Schrecken sein sollte, gehen, wie wir sehen, frei umher; von denen
jedoch, die gegen dieseö Laster eisern, kommen viele ins Gefängniz,
weil sie Zeugnis ablegen gegen den Stolz, die Unreinheit, gegen
daß betrügerische Handeln auf den Märkten, gegen Au?-schweifung
und Leichtfertigkeit, gegen das Spiel mit Kegeln, Würfeln und


% \picinclude{./100-109/p_s103.jpg} 
Brief an den Papst. Die Studenten von Cambridge usw. 103
Karten und andere eitle und stindliche Vergnügen .......
Das Schwert der Obrigkeit wird, wie wir sehen, vergeblich ge-
tragen, während die Ubeltäter frei sind, Böses zu tun; die aber,
welche gegen das Böse eifern, werden dafür bestraft von der
Obrigkeit, die ihr Schwert gegen den Herrn kehrt ..... Gs haben
viele große Strafen erlitten, darum, daß sie nicht schwören konnten
sondern der Lehre von Christus gehorchten, welche sagt: ,,ihr sollt
überhaupt nicht schwören«; sie sind ein Raub geworden (Jes.-12, 22),
weil sie das Gebot Christi hielten. Gs werden viele ins Gefängnis
geworfen, weil sie den Abschwörungseid nicht leisten können, obgleich
sie alles mißbilligen, was man darin abschwört; und es werden
viele Diener und Boten des Herrn ins Gefängnis geworfen, weil sie
nicht schwören wollen, noch Christi Gebot tibertreten. Darum bedenke
du dich doch! ich wende mich an das, was von göttlichem Leben
in dir isti, Viele sind auch im Kerker, weil sie den Priestern die
Zehnten nicht bezahlen können; viele hat man ihrer Habe beraubtund
dreifache Mgaben von ihnen gefordert; viele werden gepeitscht und
geschlagen in den Korrektionshäusern, ohne daß dadurch ein Gesetz
übertreten würde. Solche Dinge tut man in deinem Namen, damit
man bei solchem Tun geschützt sei. Wenn gottessürchtige Männer
das Schwert trügen, wenn das Unrecht bestraft würde und gottes-
fürchtige Männer angestellt würden, dann würden sie den Übel-
tätern ein Schrecken sein und ein Ruhm denen, die Recht tun, statt
ihnen Leiden zu verursachen. Dann würde Gerechtigkeit in unserm
Lande herrschen und die Rechtschafsenheit sich erheben und aus-
breiten, welche das Unrecht nicht zuläßt, sondern es richtet. Jch
rede zu dem, was vom Geiste Gottes in dir ist, daß du in dich
gehen und siir Gott regieren mögest, damit du dem Göttlichen,
das in eines jeden Menschen Gewissen ist, folgen mögest, denn
dieses macht, daß man alle Menschen achtet in dem Herrn. Siehe
doch zu, für wen du regierest, aus daß du Kraft vom Herrn
empfangen mögest, für ihn zu herrschen, und alles, was wider ihn
ist, durch sein Licht verdammt.werde.
s Von einem, der deine Seele lieb hat, und dein ewiges Bestes
wünscht.« G. F. . .


% \picinclude{./100-109/p_s104.jpg} 

%%%%%%%%%%%%%%%%%%% Kapitel 9. %%%%%%%%%%%%%%%%%%%%%%%%%%%%%%

\chapter[Angriffe der Independenten und Presbyterianer.]{Angriffe der Independenten und Presbyterianer.}

\begin{center}
\textbf{Angriffe der Independenten und Presbyterianer. Ahnungen,
Heilungen, Bekehrungen. Dispute über Taufe und Erwählung.
Gefangennahme auf Grund angeblicher Verschwörungen. Wirken
während der Gefangenschaft.}
\end{center}


Nachdem ich meine Arbeit in London getan, ging ich nach
Bedfordshire und Northamptonshire. Jsn Wellingborough hatte
ich eine große Versammlung ..... Die ,,Frommen« waren in
großer Aufregung hier, deim die bösen Priester der Presbyteri-
aner und Jndependentens hatten fälschlich ausgestreut, wir triigen
Flaschen mit uns herum, aus denen wir den Leuten zu trinken
geben, damit sie uns nachfolgen; aber die Kraft und der Geist
Gottes ließen die Freunde über diesen falschen Gerüchten stehn . . .
Von Wellingborough ging ich nach Leieestershire, wo Oberst
Hacker drohte, wenn ich hierher käme, würde er mich wieder ge-
fangen nehmen lassen; aber als ich nach Whetstone kam, wo er
mich das letzte Mal in der Versammlung hatte festnehmen lassen,
war alles ruhig. Oberst Hackers Frau kam in die Versamm-
lung und wurde bekehrt, .... es waren auch zwei Friedens-
richter in dieser Versammlung, aus Wales, namens Walter
Jenkin und Peter Price, die beide später treue Diener des
Herrn wurden.
Von da gingen wir nach Sileby . . . und dann nach Drayton,
meiner Heimat, wo früher so viele Priester und »Fromme« gegen
mich aufgetreten waren, jetzt aber rührte sich keiner. Jch fragte
einen meiner Verwandten, wo alle Priester und »Frommen«
seien? Man sagte mir, der Priester von Nun-Eaton sei ge-
storben, und nun bewerben sich acht oder neun um seine Stelle.
,,Sie werden dich diesmal in Ruhe lasfen«, sagten sie zu mir,
,,denn wie die Krähen sich um ein totes Schaf schaaren, so tun die
Priester, wenn eine Psründe frei ist«. Das waren von ihren
eigenen Zuhörern, die so redeten! ....
Als ich nach Derbshire kam, kam James Naylor zu mir
und sagte mir, sieben oder acht Priester hätten ihn zu einer
Unterredung aufgefordert. Ich war nun seinetwegen sehr be-
kümmert in meinem Geist, und der Herr gebot mir ihm zu sagen,
er solle der Mfforderung folgen; denn der Herr der Allmächtige
wolle bei ihm sein und ihm durch feine Kraft den Sieg geben.


% \picinclude{./100-109/p_s105.jpg} 
Angriffe der Jndependenten und Presbyterianer usw. 105
Und der Herr tat es, sodaß die Leute merkten, daß die Priester
geschlagen waren; und sie riefen: »ein Nagler (Naylor-Nagler)
hat sie alle zu Grunde gerichtet!« Er kam nach dem Disput zu
mir voll Dank gegen Gott .....
Nun zogen wir durch Worrestershire; ich hatte in Birming-
ham eine Versammlung ..... Dann kamen wir nach Wor-
cester ..... Von da nach Tewkesbury .... Dann nach War-
, wirk, wo ich im Hause einer Witwe eine Versammlung hatte ....
Nach derselben, als ich gerade fort gehn wollte, . . . kam ein
Gerichtsdiener herein und sragte: ,,Wen hören die Leute zu so
später Stunde?« Er verhastete John Crook, Amor Stoddart,
Gerrard Roberts und mich, erlaubte uns jedoch in unsere Her-
berge zu gehen; nur sollten wir am Morgen wiederkommen ....
Aber am nächsten Morgen hieß es, wir können unsrer Wege
gehen ..... Run gingen wir weiter nach Eoventry, ....
dann durch Leicestershire nach Swannington und Baldock. Hier
fragte ich, ob keinerlei Art von besonderem Bekenntnis vertreten
«sei? Es hieß, es gebe einige Baptisten und eine kranke Vap-
tistenfrau. John Rush ging mit mir zu ihr ..... Als wir zu
ihr kamen, waren viele fromme Leute bei ihr. Mau sagte mir,
diese Frau gehöre nicht mehr diesem Leben an; wenn ich ihr aber
etwas über das zukünftige sagen könne, so solle ich es tun. Der
Herr trieb mich, zu ihr zu reden, und sie erholte sich wieder,
zum Erstaunen der Stadt und des ganzen Landes. Diese Bap-
tistensrau und ihr Mann wurden gewonnen, und viele Hunderte
von Leuten haben sich seither in ihrem Hause versammelt .....
Wir gingen nun über Market Street .... und St. Albans
nach London .... Nachdem ich mich einige Zeit in London
aufgehalten hatte und die dortigen Freunde in ihren Versamm-
lungen besucht hatte, verließ ich die Stadt, wo ich James Raylor
zurückließ. Als ich mich von ihm trennte, fiel mein Blick aus
ihn und eine Angst besiel mich seinetwegen, aber ich ging doch
weg und ritt nach Nyegate in Surrey .....
Von da gingen wir nach Dorchester und stiegen in einer
Herberge, die einem Baptisten gehörte, ab; wir baten die in der
Stadt wohnenden Baptisten, uns ihr Versammlungshaus zu
überlassen, damit wir in demselben Versammlungen halten könnten,
aber sie oerweigerten es; wir ließen sie fragen, warum sie es
verweigerten; dadurch ward die Sache in der Stadt ruchbar.


% \picinclude{./100-109/p_s106.jpg} 
Wir ließen ihnen nun sagen, daß sie und alle, die Gott fürchteten,
in unsere Herberge kommen könnten, wenn sie wollten. Sie
waren in großer Aufregung, und viele ihrer Lehrer und andere
von ihren Leuten kamen in unsere Herberge und schlugen mit den
Bibeln aus die Tische. Jch fragte sie, worüber sie denn so auf-
gebracht seien, ob sie gegen die Bibel so aufgebracht seien? Da
fingen sie an mit Auseinandersetzungen über ihre Wassertaufe.
Ich fragte sie, ob sie behaupten könnten, von Gott gesandt zu
sein, die Leute zu tausen, wie Johannes (Joh. 1,6) und ob sie
den gleichen Geist haben wie die Apostel? Sie sagten: nein.
Daraus fragte ich sie, wie vielerlei Kräfte es denn gebe! ob es
noch andere gebe als die Krast Gottes und die des Teufels?
Sie sagten, es gebe keine andere außer diesen beiden; darauf
sagte ich: ,,Wenn ihr nicht die Kraft Gottes habt, welche auch
die Apostel hatten, dann handelt ihr in der Macht des Teufels-.«
Viele der Anwesenden, die nüchterne verständige Leute waren,
sagten: ,,die Baptisten treten den Rückzug an!« Viele angesehene
Leute wurden an dem Abend gewonnen, und wir hatten einen köst-
lichen Gottesdienst und des Herrn Kraft war über allen. Am
folgenden Morgen, als wir fortgingen, schüttelten die Baptisten in
ihrer Wut hinter uns her den Staub von ihren Füßen. ,,So«
sagte ich, ,,ihr tut solches in der Macht der Finsternis? dann tun
wir es auch gegen euch, aber in der Kraft Gottes-«.
Wir verließen Dorchester und gingen nach Weymouth ....
Gs war ein Kaoalleriehauptmann in der Stadt, der mich zu sich
kommen ließ und mich gerne länger gehalten hätte; aber ich
durfte nicht länger bleiben. Gr und ein Diener ritten etwa
sieben Meilen mit mir; Gdward Phot war auch dabei. Dieser
Hauptmann war der behäbigste, sröhlichste, leutseligste und lach-
lustigste Mensch, der mir je begegnete, sodaß es mich einige Male
trieb, ihm in der gewaltigen Kraft des Herrn zuzusprechen, aber
es war ihm so zur Gewohnheit geworden, daß er immer wieder
über alles, was er sah, lachte. Aber ich ermahnte ihn immer
wieder, ernsthaft zu werden und gottessürchtig. Wir brachten die
Nacht in einem Wirtshause zu; am Morgen trieb es mich, noch
einmal mit ihm zu reden, ehe wir uns trennten. Als ich ihn
das nächste Mal sah, teilte er mir mit, daß die Kraft des Herrn
ihn so übernommen habe, während ich damals mit ihm redete
beim Mschied, daß er ganz ernsthaft geworden sei, ehe er heim


% \picinclude{./100-109/p_s107.jpg} 
Angriffe der Jndependenten und Preßbyterianer usw. 107
kam und sein Lachen gelassen habe; er bekehrte sich später und
wurde ernsthaft und gut und starb in der Wahrheit .....
Wir kamen nach Kingszbridge, wo wir in unsrer Herberge
nach den Grnstgesinnten in der Stadt fragten. Sie schickten unß
zu Nieolaz Tripe und seiner Frau und wir gingen dorthin. Sie
ließen den Priester holen, mit dem wir unß längere Zeit unterredeten,
aber da er unterlag, verließ er unö bald. Nieolaö Tripe und
seine Frau wurden gewonnen; und seitdem kommen in jener Gegend
häufig Freunde zusammen. Al?7 wir am Abend in unsere
Herberge kamen und viele dort antrafen, welche tranken, trieb
mich der Herr zu ihnen zu gehen und sie auf das- Licht hinzu-
weisen, welches Christus ihnen allen angezündet habe, durch das-
sie ihr böseß Tun erkennen könnten, ihre bösen Reden und auch
Jesus Christus ihren Heiland. Dem Wirt wurde es unbehaglich,
weil er sah, daß ich seine Leute vom Trinken abhielt, und sowie
ich die letzten Worte geredt, nahm er ein Licht und sagte: ,,Kommt,
hier ist ein Licht, mit dem ihr in euer Zimmer gehen könnt«. Am
nächsten Morgen, als er abgekühlt war, stellte ich ihm vor, wie
unziemlich er sich benommen hatte und ermahnte ihn beim Abschied,
an den Tag de-:2 Herrn zu denken ..... Wir zogen durch
Penryn nach Helston .... und von da nach Market-Jew, wo
wir in eine Herberge gingen ..... Am nächsten Morgen ver-
sammelten sich die Behörden und schickten ihre Konstabler, um
uns vor sie zu holen. Wir fragten sie nach dem Verhaftbesehl;
sie sagten, sie hätten keinen; .... ez kamen auch mehrere
andere höhere Beamte, und wir stellten ihnen Vor, waß daß für
ein schmählicheö Betragen sei, Reisende in ihrer Herberge zu be-
helligen ..... Ghe wir die Stadt verließen, oersaßte ich noch
ein Schreiben an die sieben Gemeinden in Lands-End. GZ hieß
darin zum Schluß: Nützet eure Zeit, dieweil sie euch gegeben
ist; denn jetzt ist ,,eure angenehme Zeit, jetzt ist euer Tag deß
HeilZ« (2. Cor. 6,2). In einem jeden von Euch ist ein Licht?
von Ehristuß, daß euch zeigt, daß ihr nicht lügen, nicht unrecht
tun, nicht schwören, nicht fluchen, nicht stehlen, noch Gotteß Namen
mißbrauchen sollt. Wenn ihr dieseö Licht lieb habt und ihmysl
folgt, so wird es euch zu Christuß führen, welcher der Weg zum
Vater ist, dem Vater dez Licht?-, bei welchem nichts Ungöttlicheß
ist. Wenn ihr dieseö Licht hasset, so wird e3 euch zum Ver-
derben werden; wenn ihr ez aber liebt, so bringt es euch ab von


% \picinclude{./100-109/p_s108.jpg} 
ßden Lehrern der Welt, damit ihr von Christus lernt, und be-
iwahrt euch vor dem Unrecht der Welt und allen ihren Ver-
führern.« G. F.
Dieses Schreiben trug ein Freund, der mich begleitete, bei
sich; als wir nun etwa drei Meilen von Market-Jew gegen
Westen weiter gegangen waren, begegnete er einen Manu, dem
er eine Abschrift davon gab. Gs stellte sich heraus, daß dieser
Mann ein Diener vom Gefolge des Peter Eeelh war, des
Obersten der Armee und Friedensrichters jener Gegend. Der Mann
ritt nun voraus und zeigte das Schreiben dem Major Eeely. Als
wirnach St. Jves kamen, verlor Edward Pyots Pferd ein Hufeisen
und wir hielten an, um es wieder beschlagen zu lassen. Während-
dessen ging ich zum Meeresstrand hinunter. Als ich zuriickkam,
fand ich die Stadt in Mfruhr, und sie schleppten eben Edward
Pyot und einen andern vor Major Eeely. Jch solgte ihnen ins
Richthaus, obgleich mich niemand dazu zwang. .... Man
legte uns den Abschwörungseid vor, worauf ich meine Hand in
die Tasche steckte ..... Major Eeely hatte einen albernen
Priester bei sich, der uns viele nichtssagende Fragen stellte; unter
anderm verlangte er, ich solle mein Haar, das damals ziemlich
lang war, schneiden lassen, aber ich mochte es nicht schneiden
lassen, obgleich viele sich oft daran stießen; ich sagte ihnen, ich
sei ja nicht stolz darauf, und ich lasse es ja nicht selber wachsen.
Zuletzt übergab man uns einer Wache, die so grob gegen uns
war, wie der Richter selber; dessen ungeachtet oerkündeten wir
die Wahrheit unter den Leuten. Am solgenden Morgen sandte
man uns unter Bewachung mehrerer Berittener, die mit Schwertern
und Pistolen bewaffnet waren, nach Redruth ..... und von da
wurden wir nach Launeeston gebracht .....
Gs waren noch 9 Wochen, bis wir vor Gericht erscheinen
mußten, wozu dann viel Volk herbeiströmte, um das Verhör der
Quäker zu hören. Hauptmann Bradden war in Launeeston mit
seiner Reiterei, und seine Leute geleiteten uns durch die Volks-
menge, welche die Straßen füllte, und es war kein Geringes, uns
hindurchzubringen; auch an allen Fenstern und Türen standen
Leute, die uns sehen wollten. Jm Gerichtshof angekommen,
warteten wir eine Weile, den Hut auf dem Kopfe; niemand be-
kümmerte sich um uns, zuletzt trieb es mich zu sagen: ,,Friede
sei mit E—uch.« Da sragte Richter Glynne, damals Ober-


% \picinclude{./100-109/p_s109.jpg} 
Angriffe der Jndependeuten und Presbyterianer usw. 109
Richter von England, den Wörter: »was sind das sür Leute, die
ihr in den Gerichtshof gebracht habt?« »Gefangene, Herr,« ant-
wortete dieser. ,,Warum nehmt ihr eure Hüte nicht ab?« fragte
uns der Richter; wir antworteten nichts. ,,Nehmt eure Hüte
ab!« wiederholte der Richter, wir sagten wieder nichts; der
Richter sagte: ,,der Rat befiehlt euch, die Hüte abzunehmen.«
Nun redete ich und sagte: ,,Wann hat je ein Richter, König oder
sonst eine obrigkeitliche Person von Moses bis Daniel, bei den
Juden, dem Volke Gottes, oder bei den Heiden, je befohlen, daß
man den Hut abnehme, wenn man vor Gericht erscheint? und
wenn das Gesetz von England irgend etwas derartiges besiehlt,
so zeiget uns dieses Gesetz irgendwo geschrieben oder gedruckt.«
Da wurdesder Richter sehr zornig und sagte: ,,ich trage mein
Gesetzbuch nicht aus dem Rücken!« »So nenne mir irgend ein
Buch, welches Statuten darüber enthält, daß ich es lesen kann,«
sagte ich. Da gebot der Richter: ,,führt den Kerl weg! ich will
ihn züchtigen«, und sie führten uns fort, zu den Dieben hinunter.
Doch gleich darauf rief er den Gesangenwärter wieder, und
gebot ihm, uns wieder zu bringen. Dann sagte er: ,,hatten sie
denn etwa Hüte zur Zeit des Moses und Daniels? antwortet
mir! nicht wahr, nun habe ich euch erwischt!« Jch erwiderte:
»Du kannst Daniel 3 lesen, daß die drei Männer auf Besehl des
Nebukadnezar »in Rock, Hosen und Hut« in den Feuerofen ge-
worsen wurden«. Dieses einfache Beispiel machte ihn verstummen,
sodaß er, weil er nichts mehr zu sagen wußte, ries: »flihret sie
wieder sort!« so wurden wir denn wieder zu den Dieben hin-
untergebracht. Ain Nachmittag wurden wir wieder vor Gericht
gebracht ..... Als wir dort warteten, bis wir an die Reihe
kamen, und ich die Menge derer, die hier schwörten, sah, betrübte
es mich, daß so viele, die sich für Christen ausgaben, so offen
dem Gebot Christi ungehorsam waren, und der Herr trieb mich,
ein Blatt auszuteilen gegen das Schwören, welches ich bei mir
trug .....
Dieses Blatt machte die Runde bei den Gerichtspersonen, und
sie gaben es zuletzt dem Richter, und als wir nun vor ihn ge-
rufen wurden, fragte er mich, ob dieses verführerische Blatt mein
sei? Jch antwortete: wenn sie es vor dem ganzen Hofe vorlesen
wollen, so höre ich, ob es mein sei, und dann wolle ich auch da-
zu stehen. Er wollte, daß ich es nehme und für mich durchlese.

