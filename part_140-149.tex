% \picinclude{./140-149/p_s140.jpg} 
140 Kapitel 111.
deine Sinne auf den Herrn lenkt, aus welchem das Leben
kommt; und dann wirst du seine Kraft an dir spüren, die dich
stark macht gegen alle Stürme und Unwetter. So allein wirst
du Geduld erlangen, Unschuld, Reinheit, Ruhe, Festigkeit und
Frieden in Gott. Darum läßt dich der Herr ermahnen, dich
seinem Willen zu unterwerfen und Glauben zu haben, damit du
das, was dich bedrückt, überwindest. .   . Jch ermahne alle,
in der Furcht des Herrn zu bleiben, damit ihr die Geheimnisse
Gottes erfahren möget und seine Weisheit und unter dem Schatten
des Allmächtigen sitzet in allen Gefahren und Stürmen. Denn
der Herr ist nahe, und der Höchste regieret die Menschenkinder.
Des Herrn Wort ergehet an alle, daß sie in allen Versuchungen
und Verwirrungen, welche das Licht an den Tag bringt, sich nicht
bei diesen Versuchungen und Schlechtigkeiten aufhalten, sondern
auch das Licht sehen, welches sie ausdeckt und an den Tag
bringt; und in diesem Lichte könnet ihr über dies alles steigen
und die Kraft empfangen, ihm entgegen zu treten.s Das gleiche
Licht, welches euch die Sünde erkennen läßt, zeigt euch auch den
Bund mit Gott, welcher eure Sünden tilgt und euch Sieg gibt
über sie.   Wenn ihr die Versuchung und das Schlechte ansehet,
so wird es euch mitreißen; wenn ihr aber zum Licht ausblickt,
welches die Sünde aufdeckt, so wird es euch überwinden helfen.
Gs wird euch den Sieg geben und ihr werdet Gnade und Kraft
von oben erfahren; dies ist der erste Schritt zum Frieden. Jhr
werdet das Heil erlangen und werdet die Herrlichkeit sehen, die
war, ehe der Welt Grund gelegt war; und dadurch werdet ihr
den Samen Gottes erkennen lernen, der das E-rbe der Verhei-
ßungen Gottes ist ..... Also stärke dich der Herr im Namen
Jesu Ehristi.« G. F.
Als diese Zeilen Lady Elaypole vorgelesen wurden, sagte sie,
es habe für den Augenblick ihren Geist gestärkt. Später ver-
schafsten sich viele Freunde in England und Jrland Abdrücke
davon und lasen es anderen, die niedergedrückt waren, vor; und
es hat manchem zur Ausrichtung geholfen.
Um diese Zeit erschien ein Aufruf von Oliver Cromwell für
eine Kollekte zur Unterstützung der aus Polen vertriebenen pro-
testantischen Gemeinden und für 20 aus Böhmen oerbannte
Familien. Schon einige Zeit vorher war ein ähnlicher Ausruf
erlassen worden, der zu einem feierlichen Fast- und Bettag auf-


% \picinclude{./140-149/p_s141.jpg} 
Erste Jahresvetsammlung. Warnung an Eromwell usw. 141
forderte, damit eine Kollekte gemacht würde zum Besten der not-
leidenden Protestanten in den Tälern von Lucerne, Angrona und
an anderen Orten, welche der Herzog von Savoyen verfolgte.
ES trieb mich, dem Protektor und den obersten Behörden bei
dieser Gelegenheit zu schreiben, um ihnen die Art des wahren
Fasten-?-, das Gott gefällt und von ihm angenommen wird, dar-
zulegen und ihnen zum Bewußtsein zu bringen, wie Unrecht sie
tun und sich selbst verdammen, wenn sie die Papiften tadeln,
daß sie die Protestanten in andern Ländern verfolgen, während
sie zu gleicher Zeit ihre protestantischen Nachbarn und die
Freunde im eignen Land verfolgen .....
Jch begab mich nun nach Hampton Court, um mit dem
Protektor über die Not der Freunde zu reden. Jch traf ihn auf
einem Ritt im Hampton Court Park. Ehe ich ihn an der Spitze
seiner Leibgarde erreichte, spürte ich einen Hauch des Todes
ihm entgegen huschen; und als ich zu ihm kam, sah er aus wie
ein Toter. Nachdem ich ihm die Not der Freunde beschrieben
hatte und nach meinem inneren Trieb ihn gewarnt hatte, hieß
er mich, zu ihm heim kommen. So ging ich denn am folgenden
Tage wieder nach Hampton Eourt, um nochmals mit ihm zu
reden. Aber als ich kam, hieß es, er sei krank, und Harn-ey,
einer seiner Bedienten, teilte mir mit, die Arzte wollten nicht,
daß ich mit ihm spreche. So ging ich fort und habe ihn nachher
nie mehr gesehen.
Von hier ging ich zu Jsaak Pennington in Buckinghamshire,
wo ich eine Versammlung angezeigt hatte; und des Herrn Wahr-
heit und Kraft wurden herrlich offenbar unter uns. Nachdem
ich manche Freunde in dieser Gegend besucht hattes, ging ich
nach London und bald darauf nach Essex; kaum war ich dort,
so hörte ich, der Protektor sei gestorben, und sein Sohn Richard
sei zum Protektor gemacht worden; nun kehrte ich wieder nach
London zurück.
Noch vor dieser Zeit war das sogenannte Kirchenbekenntnis 1)
veröffentlicht worden, von dem es hieß, er sei in der Zeit von
11 Tagen gemacht worden. Jch verschaffte mir vor der Ver-
öffentlichung eine Abschrift, und schrieb eine Antwort dazu, und
überall wo nim dieses Buch über ihr Bekenntnis verkauft wurde,
1) Die Savoydeklaration der Jndepedenten von 1658.


% \picinclude{./140-149/p_s142.jpg} 
142 Kapitel lll.
wurde auch meine Antwort verkauft. Dieses- ärgerte etliche der
Parlamentßmitglieder, sodaß mir einer von ihnen mitteilte, ich
müßte nach Smithfield; ich antwortete ihm: ich stehe über ihrem
Feuer und fürchte sie nicht! Und ich stellte ihm weiter vor, ob
denn alle die vielen Völker seit 1600 Jahren ohne Glauben ge-
wesen seien, daß die Priester jetzt kommen müssen und ihnen einen
machen? ,,Sagte nicht der Apostel, daß Jesuß der Anfänger und
Vollender dez Glaubens war (Hebr. 12, 2)? [Und wenn nun
Christus der Anfänger deö Glaubens der Apostel war und dez
GlaubenZ der ersten Kirche in den ersten Zeiten und des Glaubenß
der Märtyrer, sollten nicht alle Menschen zu ihm aussehen alk-
dem Anfänger und Vollender ihres Glaubentz, und nicht zu den
Priestern?« Wir hatten viel Not mit diesem Vekenntniß. ....
Ich ging nach Reading, wo ich während etwa zehn Wochen
viel unter schwerer Niedergeschlagenheit und Trübsal zu leiden
hatte. Denn ich sah, wie viel Uneinigkeit und Verworren-
heit unter den Völkern herrschte und wie die Mächte suchten, sich
gegenseitig aufzufressen. Und ich sah, wie die Unschuld vernichtet
und die Wahrheit verleugnet wurde. Heuchelei, Betrug und
Streit gewannen die Oberhand, sodaß man [überall bereit war,
sich gegenseitig da-8 Schwert durch die Brust zu sstoßen. Viele
waren empsänglich gewesen, als- sie noch niedrig gewesen waren;
nachdem sie aber empor gekommen waren, und Macht erlangt
hatten und geholfen, andere zu töten, wurden sie bald so
schlecht wie die übrigen, so daß wir ost mit ihnen in Streit ge-
rieten wegen unsrer Hüte und wegen des »Du«-sagenß. Sie
kehrten ihre schaugestellte Geduld und Mäßigkeit in Zorn und
Ungeberdigkeit, und viele von ihnen taten wie Wahnsinnige wegen
dieser Hutehre. Denn sie waren durch die Verfolgung der Un-
schuld verhärtet worden, und kreuzigten nun den Samen,
Christuß, in sich und in andern; biz sie schließlich ansingen, sich
untereinander zu beißen und auszuzehren, nachdem sie daß, waö
Gott in ihnen hatte ausgehen lassen, beleidigt und zerstört hatten.
Darum stürzte sie Gott bald und machte die Hohen niedrig, und
stellte den König über die, die so ost behauptet hatten, die Quäker
kommen zusammen, um die Rückkehr König Karlß zu beraten,
während doch die Freunde sich nie um die äußern Mächte und
Regierungen bekümmert hatten. Zuletzt hat Gott ihn dann zu-
rückgebracht, und viele, alz sie sahen, daß er doch kommen werde,


% \picinclude{./140-149/p_s143.jpg} 
Erste Jahresversammlnng. Warnung an Cromwell usw. 143
stimmten sür sein Kommen. So preiset nun Gott mit Herz und
Mund, der die Herrschaft hat über alles .... Jch ahnte die
Rückkehr des Königs voraus, und so taten manche andere. Jch
schrieb mehrere Male an Oliver um ihm zu sagen, daß, während
er das Volk Gottes verfolge, seine Feinde sich rüsten, ihn zu
stürzen. Als einige Voreilige unter uns Somerset House kaufen
wollten, um Versammlungen drin zu halten, verbot ichs ihnen;
denn ich sah die Rückkehr des Königs voraus. Sodann kam eine
Frau zu mir, welche eine Vorahnung von der Rückkehr des
Königs gehabt hatte, drei Jahre, ehe er wirklich kam; und er-
klärte mir, sie müsse hingehen und es ihm sagen. Jch riet ihr,
das dem Herrn zu überlassen und es für sich zu behalten; denn
wenn es entdeckt würde, in welcher Angelegenheit sie hingehe, so
würde man es als Verrat ansehen; sie beharcte daraus, sie müsse
zu ihm gehen und ihm sagen, daß er wieder nach England zu-
rückkehren werde. Da erkannte ich, daß ihre Vorahnung sich
erfüllen werde; denn es mußte ein schwerer Schlag die treffen,
die damals so große Macht hatten und so harte Verfolgungen
ausübten; sie hielten sich für heilig und nahmen doch den Freunden
ihre rechtmäßigen Besitzungen, weil sie nicht schwören wollten.
Oft wenn wir Oliver diese Dinge berichteten, wollte er sie nicht
glauben. Darum trieb es Thomas Aldam und Anthony Pearson,
in alle Kerker von ganz England zu gehen und Auszeichnrmgen
zu machen darüber, wie die Freunde von den Kerkermeistern
behandelt wurden, damit sie die Größe ihrer Leiden Oliver vor-
bringen könnten. Und als er dennoch keinen Befehl geben wollte,
sie srei zu lassen, trieb es Thomas Aldam seine Mütze vom Kopf
zu nehmen, sie vor Olioers Augen in Stücke zu zerreißen und zu
rufen: ,,Also soll auch deine Herrschaft von dir und deinem Hause
gerissen werden«. Eine Frau, die auch zu den Freunden ge-
hörte, trieb es, ins Parlament zu gehen, welches den Freunden
übel wollte, mit einem Wg in der Hand, den sie vor ihnen zer-
schlug und rief: ,,so sollt ihr in Stücke zerschlagen werden!« was
auch bald darauf geschah. Während meiner großen Niederge-
schlagenheit und inneren Prüfung, die ich um meines Landes willen
zu erdulden hatte, weil die große Heuchelei, Falschheit und Ver-
räberei mich schwer drückte, sah ich, daß Gott die, welche jetzt
unten waren, über die, welche jetzt oben waren, erhöhen werde,
und daß alle sich dem, das sie bekehren konnte, zrmeigen mußten,


% \picinclude{./140-149/p_s144.jpg} 
144 Kapitel Ill.
ehe sie Herr werden würden über den bösen Geist, nach innen
und nach außen. Denn nur der eine unsichtbare Geist kann und
wird die Heuchelei in den Menschen vernichten .....
Das ganze Land war in Zwiespalt und großer Aufregung;
die verschiedenen Parteien zankten sich beständig untereinander
und rotteten sich gegeneinander zusammen, weil jede ihre eigenen
Jnteressen durchsetzen wollte. Da ich in großer Sorge war, daß
die Jungen und Unersahrenen unter uns diesen Versuchungen er-
liegen werden, trieb es mich, allen diesen folgendes zu schreiben:
,,Jhr Freunde allenthalben! Hütet euch vor Komplotten
und Wühlereien, und vor dem Arm des Fleisches, denn alle
diese Machthaber sind gefallene Söhne Adams; sie richten der
Menschen Leben zugrunde, wie Hunde, Schweine und andere
Tiere sich zugrunde richten, sich beißen und zerreißen. Wie ent-
stand das Streiten und Töten anders als aus der Lust? Und
dies alles kommt vom gefallenen Adam her, nicht von demjenigen h
Adam aber, der nicht fiel, in welchem Leben und Frieden ist
(1. Cor. 15). Jhr seid zum Frieden berufen, darum jaget ihm
nach, und dieser Frieden ist in Christus und nicht in dem gefallenen
Adam. Alle, die jetzt vorgeben, für Christus zu kämpfen, betrügen
sich; denn sein Reich ist nicht von dieser Welt; darum kämpfen
seine Diener nicht. Die Streitenden gehören nicht zu seinem
Reich, denn sein Reich ist Frieden und Gerechtigkeit .... Jhr,
die ihr Erben seid des Evangeliums des Friedens, welches ge-
wesen, ehe der Satan war, lebet in diesem Evangelium, suchet den
Frieden und das Gute für alle, und lebet in Christus, der ge-
kommen ist, die Seele der Menschen vom gefallenen Adam zu
erlösen; das äußere Schwert der Juden, mit dem sie die Heiden
umbrachten, war ein Sinnbild des inwendigen Geistes Gottes,
der die inwendige heidnische Natur tötet. So lebet denn im
friedsamen Reich Jesu Christi, im Frieden Gottes und nicht in
den Lüften, aus denen der Krieg entsteht .... und suchet das
Wohl und Gedeihen für alle Menschen.« G. F.
Bald darauf ergriff George Booth in Cheshire die Waffen
und Laniberts) zog gegen ihn; daraufhin wollten etliche Hitzköpfe,
wie solche zuweilen unter uns waren, auch die Waffen ergreifen,
aber der Herr trieb mich, sie zu ermahnen und ste blieben ruhig,
1) Lambert war einer der bedeutendsten Generäle aus der Partei Cromwells.


% \picinclude{./140-149/p_s145.jpg} 
Ein Gottesgericht. Erntahnung zur Barmherzigkeit usw. 145
Zur Zeit des sogenannten Sicherheit?-aus-schusses forderte man uns
auf, die Waffen zu nehmen und manchem von unß wurden hohe
Stellen und Kommandoß angeboten, aber wir schlugen sie alle aus
und traten mündlich und schriftlich dagegen auf, indem wir er-
klärten, unsere Waffen und Rüstungen seien nicht fleischlich, sondern
geistlich, und damit keiner unter unz in diese Falle gehe, kam es
über mich vom Herrn, bei dieser Gelegenheit einige Zeilen der
Ermahnung an alle zu schreiben .....
Nachdem ich längere Zeit in London geroeilt hatte, zog ich
wieder in den Grafschaften umher, durch Essex und Suffolk nach
Norwich .... und von da durch Huntingdomshire und Eam-
bridgeshire wieder nach London, gerade alö General Monk 1) dort
eingezogen war und die Tore und Befestigungen der Stadt
fielen. Lange vorher hatte ich ein Gesicht gehabt, in welchem
ich die Stadt in Trümmer und die Tore eingestürzt gesehen
hatte, gerade so, wie ich sie nun mehrere Jahre später nach dem
Brande sehen sollte.
Kapitel Hill.
EinGoiteSgerirht. Ermahnnng zur Barmherzigkeit beiSchisfbtüchen.
Qnäkersreundlirher Erlaß des General Monk. Fox als Kbnigsseind
gefangen und schließlich aus Befehl Karls ll befreit.
A18 ich nun meine Arbeit in London getan, ging ich nach
Surrey und Sussex, .... dann nach Hampshire, Dorsetshire,
Ringwood und Poole, wo ich überall Freunde besuchte und große
Versammlungen unter ihnen hatte.
Jn Dorchester hatten wir eine große Abendoersammlung in
unserer Herberge, bei der viele Soldaten zugegen waren, die alle
ziemlich anständig waren. Aber da erschienen die Wachen und
Schutzleute der Stadt unter dem Vorwand, sie müßten einen gi-
schorenen Jesuiten suchen und verlangten, daß alle ihre Hüte ab-
nähmen, oder sie würden sie abnehmen, um die Tonsur dez Jesuiten
zu finden. So nahmen sie mir den Hut ab und untersuchten
mich genau, denn mich hatten sie im Verdacht; aber alß sie keine
kahle oder geschorene Stelle fanden, gingen sie beschämt fort. Und
1) General Monk, 1660 Genetalleutnant der Republik und nachher eifrig
bemüht für die Rückkehr Karl 11.
Ges:-ae Fox. 10


% \picinclude{./140-149/p_s146.jpg} 
146 Kapitel Jill.
die Soldaten und andere Leute ärgerten sich sehr über sie. Aber
es förderte die Sache Gotteö und alleö diente zum Guten, denn
es machte den Leuten Eindruck, und nachdem die Beamten fort
waren, hatten wir eine schöne Versammlung, und viele wurden
zum Herm Jesuö bekehrt, ihrem Lehrer, der sie erkauft hat und
sie versöhnen will mit Gott. Von da gingen wir nach Somer-
setshire, wo die Preßbhterianer und andere ,,Fromme« sehr böse
waren und oft die Versammlungen der Freunde störten. Einmal
hatten sie einen sehr schlechten Menschen dazu veranlaßt, in eine
Versammlung der Quäker zu gehen und eine Bärenhaut anzu-
ziehen und Unsinn zu treiben. Er setzte sich gerade dem Freund,
der redete, gegenüber mit seiner Bärenhaut über dem Rücken
und streckte die Zunge herauß und machte eine große Unruhe.
Aber ein s chwerez Gericht kam über ihn, und seine Strafe schlummerte
nicht; alö er au;-’ der Versammlung heim ging, kam er an einer
Stierhetze vorüber und blieb stehen, um zuzusehen; alß er aber
nahe bei dem Stier war, stieß dieser dem Mann das Horn in
den Hals, sodaß seine Zunge heraus hing, gerade wie er es vor-
her in der Versammlung gemacht hatte. Und der Stier stieß
sein Horn durch den Kopf des Manneß hindurch, und schwang
ihn schrecklich in der Luft herum. So kam er, der dem Volke s
Gotteö hatte Schaden zufügen wollen, selber zu Schaden, und
es wäre gut, wenn solche Beispiele der Rache Gotteö andere lehren
würden, sich zu hüten .....
Wir gingen durch Somersetshire, Plymouth und Devonshire
nach Cornwall ..... Während ich hier war, geschahen große
Schiffbrüche in der Nähe von Landß=End. Nun war eß Brauch
in jener Gegend, daß bei einem solchen Anlaß reich und arm
hinaus ging, um so viel wie möglich von den Überreften an
sich zu bringen, unbekümmert um die Rettung der Menschen. An
einigen Orten nannten sie sogar einen Schiffbruch eine Gottes-
gnade. GS betrtibte mich, von solchem unchristlicheni Treiben zu
hören und zu sehen, wie tief diese Leute unter den Heiden von
Melite stehen, die Paulus aufnahmen, ihm ein Feuer machten
und freundlich waren gegen ihn und die anderen Schiffbrüchigen
(Act. 28). Darum trieb eö mich, ein Schreiben an alle Gemeinden,
Priester und Behörden zu senden, um sie wegen ihreß habsüchtigen
Treibenö zu tadeln und sie zu ermahnen, so ost sie könnten, mit
Eifer behilflich zu sein, wenn ez gelte, Menschenleben zu retten


% \picinclude{./140-149/p_s147.jpg} 
Ein Gottesgericht. Ermahnung zur Barmherzigkeit usw. 1427
und Schiffe und Waren zu schützen; auch sollten sie dach be-
denken, wie grausam es ihnen vorkommen würde, wenn sie selber
Schiffbruch litten und die Leute suchen würden, ihnen soviel wie
möglich zu rauben, ohne sich um ihre Rettung zu kümmern ....
Diese Schrist hatte viel Erfolg beim Volk; die Freunde bemühten
sich um die Rettung der Schiffbrüchigen und den Schutz der
Schiffe und der Habe, ja, Freunde haben Schiffbrüchige, die halb
tot und Verhungert waren, bei sich aufgenommen und sie gepflegt
rmd unterstützt; alle wahren Christen sollten so handeln .....
Die Soldaten, die unter dem Befehl des General Monk
standen, waren damals oft sehr grob und störten die Versamm-
lungen der Freunde an manchen Orten. Als man sich darüber
beim General Monk beklagte, erließ er folgenden Befehl, worauf
es etwas besser wurde:
St. James, 9. März 1659.
,,Jch will, daß alle Offiziere und Soldaten sich hüten, die
friedlichen Versammlungen der Quäker zu stören, da sie nichts
tun, das dem Parlament oder dem Commonwealth von Gng-
land zuwider ist«. . George Monk .....
Wir gingen . . . über Oldeston . . . Nailsworth . . . Dray-
ton . . . Lancaster nach Swarthmore. Jch war noch nicht lange
dort, als Henry Porter, ein Friedensrichter, einen Verhastbefehl
sandte, um mich zu greifen. Jch hatte dies oorausgefühlt, und
so kam denn auch, während ich mit Richard Richardson und
Margaret Fell zusammen im Zimmer saß, ihre Dienerschast und
meldete, es seien einige da, die durchsuchten das Haus, angeb-
lich um zu sehen, ob Waffen darin seien. Es kam über mich, zu
ihnen hinaus zu gehen, und als ich an einem von ihnen vorüber
ging, redete ich ihn an, woraus sie mich nach meinem Namen
fragten; ich sagte ihn ohne weiteres, worauf sie mich ergriffen
und sagten, ich sei gerade der Mann, den sie suchten. Und sie
führten mich fort nach Uloerstone ..... Von da brachten sie
mich nach Lancaster ..... Als ich dorthin kam, war das Volk
sehr aufgeregt; ich blieb stehen und sah sie fest an, und sie
schrien: ,,seht diese Augen!« Nach einer Weile redete ich mit
ihnen, und da waren sie ziemlich ruhig. Gin junger Mann nahm
mich mit in seine Wohnung, und nach einiger Zeit kam ein Be-
amter und brachte mich zu Major Porter, der den Befehl gegen
mich erlassen hatte; es waren noch ein paar andere bei ihm. Als
10*


% \picinclude{./140-149/p_s148.jpg} 
148 Kapitel Illl.
ich hereinkam, sagte ich: ,,Friede sei mit euch«. Porter fragte
mich, warum ich in dieser unruhigen Zeit hierher komme? Jch
erwiderte: ,,Um meine Mitmenschen zu besuchen«. ,-Ihr habt
überall herum große Versammlungen«, sagte er; ich erwiderte
ihm, diese Versammlungen seien aber im ganzen Lande als fried-
liche und ruhige bekannt, und wir seien ein friedlicheö Volk. Gr
sagte: »Jhr seht aber den Teufel den Leuten im Gesicht ge-
schrieben«. Jch erwiderte: ,,Wenn— ich einen Trunkenbold oder
einen Schwörer oder einen Grobian sehe, so kann ich doch nicht
sagen, ich sehe den Geist Gotteö in ihm.« Und ich fragte ihn, ob.
er den Geist Gottes sehen könne? Gr sagte, wir treten gegen .
ihre Prediger auf. Ich antwortete: ,,Alß wir noch wie Sauluß waren
und unter den Priestern saßen, da hat man untz nicht schädliche
Männer genannt (Act. 24,5) oder Sektenmacher; aber alß wir
anfingen, Gott zu leben, so wurden wir schädliche Leute genannt
wie Paulus-«. Gr sagte, wir könnten recht gut reden, er wolle
lieber nicht mit unö diöputieren, aber greifen wolle er unz lassen. Jch
fragte ihn, warum und auf wessen Befehl er einen Verhastbesehl
gegen mich ergehen lasse, und beklagte mich über die Behandlung
der Beamten bei meiner Gefangenschaft und auf dem Wege
hierher. Gr hörte nicht aus mich, sondern sagte, er habe einen
Befehl, aber er wolle mich ihn nicht sehen lassen, denn er
wolle die Geheimnisse dez Königs- nicht preisgeben; und über-
dies brauche ein Gefangener nicht zu wissen, warum er ver-
haftet sei. Jch sagte ihm, daß sei unoernünftig, wie der Gefan-
gene sich denn dami verteidigen solle? er solle mir eine Abschrift
geben. Gr sagte, ee sei einmal ein Richter bestrast worden, weil
er einem Gefangenen den Verhastbesehl gezeigt habe .... und
er sagte mir, ich sei ein Friedenstörer im Land. Jch sagte, ich
sei im Gegenteil ein Segen für da-Z Land durch die Kraft und
die Wahrheit des Herrn, und der Geist Gottez in den Gewissen
gebe Zeugniß hiervon. Dann beschuldigte er mich, ich sei ein
Feind des König?. und beabsichtige einen neuen Krieg anzustiften
und neues Blutvergießen über das Land zu bringen. Ich erklärte
ihm, ich habe nie die Gebräuche des Kriegeß gelernt und sei in
diesen Dingen so unwissend wie ein Kind. Da kam der Schreiber
mit dem außgefertigten Verhastbefehl und der Kerkermeister wurde
gerufen, und er mußte mich ins Loch tun und niemand durfte
mich besuchen. Dort sollte ich nun gefangen bleiben, biz mich der


% \picinclude{./140-149/p_s149.jpg} 
Ein Gottesgeticht. Ermahuung zur Barmherzigkeit usw. 149
König oder daß Parlament frei sprechen würde ..... Jch ließ
nun Thomaß Eumminß und Thomas Green bitten, zum Gefangen-
wärter zu gehen und ihn um eine Abschrist des Verhaftbefehlß zu
bitten, damit ich wisse, warum ich verurteilt sei. Sie gingen hin;
der Wärter sagte, er könne ihnen die Abschrift nicht geben, weil
einmal einer bestraft worden, der dies getan, aber sie könnten sie
durchlesen. Soviel sie sich nachher erinnerten, lautete die An-
klage also: daß sich- im Verdacht stehe, ein Störer dez Land-
friedenö zu sein, ein Feind des-’ Königö und eine Hauptstütze der
Quäker-Sekte, und daß ich, zusammen mit andern dieser Fana-
tiker, kürzlich versucht habe, Aufstände in dieser Gegend anzu-
stiften und daß Land in Blut zu tauchen. Darum müsse der
Kerkermeister mich in sicherem Gewahrsam behalten, biz ich auf
Befehl des Königs; oder Parlamenteß befreit würde.
Alß mir nun meine Anklage in der Hauptsache bekannt war,
schrieb ich eine kurze Erwiderung, um meine Unschuld zu zeigen;
sie lautete:

\bigskip \begin{quote}

  Ich bin Gefangener in Lancaster, durch Friedenßrichter
  Porter verhaftet. Jch kann keine Abschrift der Anklage erhalten;
  doch erfahre ich, daß sie Behauptungen enthält, die durchauö un-
  richtig sind, z. B. daß ich im Verdacht stehe, ein Friedenstörer
  zu sein und ein Feind deö Königß, und daß ich versuche, mit
  andern zusammen Aussiände anzustiften und Blutoergießen überß
  Land zu bringen; das ist gänzlich falsch und ich bestreite etz. GS
  trifft mich keinerlei Verdacht, ein Friedenstörer zu sein; denn
  ich bin über jeden dieser Punkte schon früher oerhört worden, im
  ganzen Land herum. In den Tagen Cromwellß bin ich gefangen
  genommen worden, weil etz hieß, ich habe die Waffen gegen ihn
  ergriffen, waö falsch war, denn ich habe überhaupt nie Waffen
  getragen; dennoch wurde ich alß Gefangener nach London ge-
  bracht und vor ihn geführt; dort bewies ich ihm meine Umschuld
  und daß ich ja überhaupt gegen daß Gebrauchen irgend einer
  fleischlichen Waffe sei, da meine Waffen geistliche seien, solche,
  die die Ursachen deö Kriegeß hinwegnehmen und zum Frieden
  führen. Daraufhin sprach mich Oliver frei. Darnach wurde ich
  gefangen durch Major Ceely in Cornwall mid ins Gefängniß
  gebracht; er behauptete vor Gericht, ich hätte ihn beiseite ge-
  nommen und ihm gesagt, ich könne in Zeit einer Stunde vierzig-
  tausend Mann stellen, um daß Land in Blutoergießen zu stürzen

