% \picinclude{./290-299/p_s290.jpg} 
290 Kapitel XIV.
einen hervorragenden Lehrer der Mennoniten oder Baptisten.
Jch war bei ihm gewesen, als ich vor sieben Jahren in Holland
gewesen war, und William Penn und George Keith Dispute
mit ihm gehabt hatten. Gr war damals sehr hochmütig und
mißtrauisch gewesen, sodaß er nicht von mir angeriihrt noch
angesehen werden wollte, sondern mich hieß, meine Augen oon
ihm abwenden, da sie, wie er behauptete, ihn stechen. Jetzt aber
war er sehr empfänglich und geneigt und bekannte sich bis zu
einem gewissen Grad zur Wahrheit; auch seine Frau und Tochter
waren sehr empsänglich und freundlich, und wir trennten uns
voller Liebe .....
Wir hatten eine Versammlung in Amsterdam, bei welcher
viele außer den Freunden anwesend waren, unter anderm der
große Baptistenlehrer Galenus, der dem Zeugnis der Wahrheit
sehr aufmerksam zuhörte und nachher zu mir kam und mir sehr
liebevoll die Hand gab .....
Wir gingen noch nach Alkmar, Sardam, einer großen Stadt
von Schissbauern, .... dann nach Harlem, wo wir im Haus
eines Freundes eine große Versammlung hatten, . .. daraus nach
Rotterdam, wo wir zu zwei Versammlungen blieben, .... und
am 16. des 5. Monats nach Briel, um uns wieder für England
einzuschifsen .....
Den folgenden Winter brachte ich in London zu, nur zwei
oder dreimal ging ich mit meiner Frau, die bei mir in der
Stadt war, zu ihrem Sohn nach Kingston. Und obgleich ich
sehr elend war, war ich beständig an der Arbeit, entweder in
öffentlichen Versammlungen, wenn es mir möglich war, sie zu
ertragen, oder in privaten Angelegenheiten unter den Freunden
und mit Besuchen bei solchen, die um der Wahrheit willen litten,
in der Gefangenschaft oder durch Verlust ihrer Habe. Ich schrieb
auch allerlei in dieser Zeit, einiges für den Druck, anderes zu
privatem Gebrauch, so Briefe an den König von Dänemark und
den Herzog von Holstein wegen der Freunde, die in ihren Ländern
zu leiden hatten. Einer dieser Briefe lautet:
,,An den Herzog von Holstein, den ich in der Liebe Gottes
bitte, dieses durchzulesen, das ihm in Liebe gesandt wird.
Jch hörte, daß, als Elisabeth Hendricks nach Friedrichstadt
kam, um die Leute, die ihr Quäker nennt, zu besuchen, dir durch
einige übelmeinende Personen hinterbracht wurde, es sei ein Ärger-


% \picinclude{./290-299/p_s291.jpg} 
Zweite Reise nach Holland. Brief an den Herzog von Holstein usw. 291
niz für die christliche Religion, wenn einer Frau gestattet werde, in
einer öffentlichen religiösen Versammlung zu reden. Daraufhin hast
du den Behörden von Friedrichstadt den Befehl gegeben, dafür
zu sorgen, daß besagte Leute die Stadt verlassen, oder sie fortzu-
schicken. Da aber jene Behörden Arminianer waren, und alß ein
in Holland versolgtes Volk vor etwa sech-Zzig Jahren hierher ge-
kommen waren, schickten sie die Antwort: »Wir sind nicht willenß,
andere um der Religion willen zu verfolgen, denn alö wir selber
die Verfolgten waren, so sahen wir die Verfolgung alö anti-
christlich an.« Daraufhin hat das- Volk Gottes, zum Spott Quäker
genannt, an dich, o Herzog, von Friedrichstadt aus geschrieben,
und seither haben sie ihre Freiheit nnd friedlichen Versammlungen
gehabt und konnten frei und ohne Hinderniö die letzten beinahe
zwanzig Jahre in Friedrichstadt Gott dienen und ihn anbeten;
diese Freiheit betrachten sie als- eine grosze Gunst von dir.
Und in der Tat, o Herzog, der du dich zum Christentum be-
kennst, daö sich nach dem großen, mächtigen Namen Jesu Christi,
des Könige der Könige, deö Herrn aller Herren, nennt und sich
aus die heiligen Schriften der Wahrheit de-3 alten und neuen
Testamentö gründet, gebrauchst du nicht in deinem Gottesdienst
ost Worte von Frauen aus dem alten und neuen Testament? Der
Apostel sagt: ,,Gure Weiber sollen schweigen in der Gemeinde«
(1. Cor. 14), und er gestattet den Weibern nicht, daß sie lehren,
sondern sie sollen stille sein, und wenn sie etwaö lernen wollen,
,,so sollen sie daheim ihre Männer fragen, eß- stehe den Weibern
übel an, in der Gemeinde zu reden«; und 1. Tim. 2, 11.12: »Ein
Weib lerne in der Stille mit aller Untertänigkeit, und ist ihr nicht
gestattet, daß sie lehre, noch daß sie des Mannes Herr sei, sondern
daß sie sei stille«. Der Herzog kann aber sehen, waß daß für
Frauen waren, die stille sein sollten und untertänig, denen das-
Gesetz befiehlt, stille zu sein: es sind die unordentlichen Frauen!
Denn im gleichen Kapitel gebietet der Apostel denWeibern, »sich nicht
mit Zöpfen oder Gold oder Perlen oder köstlichem Gewand zu
schmücken«. Diese Dinge hat der Apostel verboten, und solche
Weiber, welche sich also schmücken, die sollen ,,in der Stille lernen
und untertänig sein« und nicht über die Männer herrschen, denn
solchen stehet eß übel, in der Gemeinde zu reden. Sind ez aber
nicht gerade solche Weiber, die Gold, Silber, Perlen und köstliche
Kleider tragen und sich mit Haarflechten schmücken, die in euern
19*


% \picinclude{./290-299/p_s292.jpg} 
292 Kapitel XIV.
Kirchen reden, wenn eure Priester sie die Psalmen singen lassen?
oder reden sie etwa nicht, wenn sie Psalmen fingen? Solcheß- be-
denke, oHerzog! Und dennoch sagst du, eure Weiber sollen schweigen
in der Kirche und nicht reden, wenn sie aber Psalmen singen in
der Kirche, sind sie dann still? Und während der Apostel den
oben genannten Weibern verbietet, in der Kirche zu sprechen, so
ermutigt er an anderer Stelle die guten und heiligen Frauen, die
,,Lehrer in guten Dingen zu sein«, wie in Tit. 2,3. Der Apostel
sagt, ,,ich bitte dich, mein treuer Geselle, stehe jenen Frauen, die
mit mir für datz Evangelium gekämpft haben, bei und den andern
meinen Gehilfen, welcher Namen sind im Buche des LebenZ«
(Phil. 4,3). Hier anerkennt er diese heiligen Frauen, die mit ihm
für daß Evangelium gekämpft haben, und ermutigt sie und ver-
bietet ihnen nicht, zu reden. Ebenso anempfiehlt er Phöbe der
Kirche von Rom und nennt sie: i,,eine Dienerin der Kirche zu
Kenchreae« (Röm. 16,1), sendet durch sie seinen Brief von Eorinth
an die Kirche von Rom und bittet, sie aufzunehmen in dem
Herrn, wie ez den Heiligen geziemt, und sie in allem, waß sie
brauche, zu unterstützen, sdenn sie sei vielsen eine Stütze gewesen,
so auch ihm selber. Und ferner sagt er: ,,Grüßet Aquila und
Priöcilla, meine Helfer in Jesus Ehristuß-« (Röm. 16,3). Hier
kann der Herzog sehen, daß dieseö gute, heilige Frauen waren,
denen der Apostel nicht verbot zu reden, sondern eö gebot.
Und Aquila und Prizcilla legten ,,Apollo den Weg Gottez noch
fleißiger au?-« (Act. 18,26), hier war also Priseilla Lehrer so gut
wie Aquila, und solchen heiligen Frauen verbietet der Apostel
daß Reden nicht. Ebenso verbot er den vier Töchtern dez Phi-
lippuö, welche Jungfrauen waren, nicht, zu wei?-sagen. Und die
Weiber dürfen in der Kirche beten und wei?-sagen (1. Cor. 11,5).
Die Apostel zeigten den Juden die Erfüllung der Weißsagungen
des- Propheten Joel: ,,Jn den letzten Tagen will ich meinen
Geist außgießen über alleß Fleisch, und eure Söhne und Töchter,
Knechte und Mägde sollen weiösagen durch- den Geist GotteZ«.
Der Apostel ermahnt also die Töchter und Mägde so gut wie die
Söhne, zu wei?-sagen, und wenn sie weißsagen, ,,so sollen sie zur
Gemeinde oder zum Volk reden« (Joel. 2,28, Act. 2,18). Sangen
nicht Mirjam, die Prophetin, und alle Frauen mit ihr, dem Herrn,
alß er die Kinder Jzraelß von Pharao errettet hatte?s Prieö sie
nicht den Herrn und prophezeite in der Versammlung der Kinder


% \picinclude{./290-299/p_s293.jpg} 
Zweite Reise nach Holland. Vries an den Herzog von Holstein usw. 293
JS-rael?-? und geschah daß nicht vor der Gemeinde? (2.sMos. 15,21).
Moseß und Aaron verboten ihr nicht, zu wei-Jsagen und zu reden,
sondern Moseö sagte: ,,Wollte Gott, daß daß ganze Volk des Herrn
weiß-sagte«! (4. Mos. 11,29). Und ,,daZ Volk des- Herrn« sind
Frauen so gut wie Männer. Deborah war Richterin und Prophetin,
und gebraucht ihr nicht die Worte Deborahß und Mirjamß in euren
Gottezdiensten? Denket an Deborahß lange Rede oder Gesang!
Barak verbot eß ihrnicht, noch irgend einer der jüdischen Priester, und
hielt sie nicht diese Rede in der Versammlung oder Kirche J3-
raelß? (Rich. 5). Jm Buch Ruth sind gute Reden jener treff-
lichen Frauen, die man auch nicht verboten hat. Hanna betete
im Tempel vor Eli, und der Herr erhörte ihr Gebet. Sieh doch,
was; für eine Rede Hanna hielt, und wie sie Gott prieß vor Eli-?
Ohren, und er verbot ez ihr nicht (1. Sam. 2,1-10). Josia
schickte seinen Priester mit mehreren andern zur Prophetin Hulda,
. die in Jerusalem wohnete, um sie um Rat zu fragen (2. Könige
22,14). Hier haben also der König sowie seine Priester
den Rat der Prophetin nicht verachtet, und sie weiß-sagte i
in der Versammlung vor den Jßraeliten, wie man in diesen
Kapiteln sehen kann.
Und sieh bei Lnkaß den göttlichen Gesang der Elisabeth an
Maria und den göttlichen langen Gesang Mariaß, wo Maria sagte:
,,der Herr hat die Niedrigkeit seiner Magd angesehen.« Und braucht
ihr nicht in euern Gotteßdiensten und Gebetea die Worte der
Maria und Elisabeth, auß Lukaö 1, 41—55 und verbietet dennoch
den Frauen in euern Kirchen zu reden? Auch reden alle möglichen
Frauen in euern Kirchen, wenn sie singen und Amen sagen. Jn
Lukas 2 war die Prophetin Hanna, eine Witwe von 84 Jahren,
die nicht aus- dem Tempel ging, sondern Gott diente mit Fasten
und Beten Tag und Nacht. Hatte sie nicht von Ehristuö ge-
zeugt im Tempel und dem Herrn Danksagung gebracht, und zu
allen, die da warteten auf die Erlösung zu Jerusalem, von
Christum geredet? (Luk. 2, 36—38). Solchen heiligen Frauen
war es nicht verboten, in der Kirche zu reden, weder im Gesetz
noch im Evangelium. Waren etz nicht Maria Magdalenazund
andere Frauen, die zuerst die Auferstehung Christi den Aposteln
oerkündeten? Daß Weib war es gewesen (nämlich Eva), Idie
zuerst sündigte, und so waren etz auch Frauen, die zuerst die  
erstehung Christi verkündeten; denn Christus sagte zu Maria, ,,Gehe


% \picinclude{./290-299/p_s294.jpg} 
294 Kapitel HIV.
zu meinen Brüdern und sage ihnen, ich gehe hin zu meinem
Vater und zu euerm Vater, zu meinem Gott und zu euerm Gott''
(Joh. 20, 17). Und Luk. 24, 10 waren es Maria Magdalena und
Johanna und Maria die Mutter des Jakobus und andere Frauen,
die mit ihnen waren, die den Aposteln sagten, daß Christus von
den Toten auferstanden sei: ,,und es deuchten sie ihre Worte eben,
als wären es Märlein, und glaubten ihnen nicht.'' Und Vers 22:
»Auch haben uns erschreckt etliche Weiber der Unsern.'' Hier
kann man sehen, daß die Reden der Frauen von der Auferstehung
Christi die Apostel erstaunten. Christus sandte die Frauen, seine
Auferstehung zu predigen; es ist also keine Schande für Frauen,
Christus zu predigen, und wenn Christus sie schickt, so dürfen sie
nicht schweigen. Der Apostel sagt: ,,Jede Zunge soll bekennen''
(Röm. 14, 11); und Phil. 2, 11: ,,Jede Zunge soll bekennen, daß
Christus der Herr ist, zur Ehre Gottes des Vaters.'' Das zeigt
klar, daß die Frauen so gut wie die Männer Christus bekennen
sollen, wenn jede Zunge ihn bekennen soll. Und der Apostel sagt,
»da ist weder Mann noch Weib, sondern sie sind alle eins in
Christo Jesu'' (Gal. 3, 28). Und wenn es heißt, die Frauen
sollen daheim ihre Männer sragen, so weiß ja der Herzog wohl,
daß Jungfrauen keine Männer haben noch die Witwen. Hanna,
die Prophetin, war eine Witwe. Und da ja Christus der »Eine
Mann ist'' (2. Cor. 11, 2), so iiiüfsen ihn die Männer daheim
sragen so gut wie die Frauen, ehe sie lehren. Und gesetzt der
Fall, das Weib eines Türken sei eine Christin, oder das Weib
eines Papisten eine L-utheranerin oder Calvinistin, müssen sie
dann auch ihre Männer daheim fragen und von ihnen lernen, ehe
sie Christus in der Versammlung des Herrn bekennen? Dann
würden sie den Rat erhalten, Türken oder Papisten zu werden.
Jch bitte den Herzog, diese Dinge zu bedenken. Jch bitte
ihn, aus Gottes Gnade und Wahrheit in seinem Herzen, die aus
Jesus kommen, zu merken, damit er durch diesen Geist der
Gnade und Wahrheit dazu kommen möge, Gott im Geist und in
der Wahrheit anzubeten und ihm zu dienen, ihm, dem lebendigen,
ewigen Gott, der ihn gemacht hat, und Frieden zu haben in
Christus, den Frieden, den die Welt nicht nehmen kann. Jch
wünsche ihm Glück, Frieden und Wohlergehen in dieser Welt und
immerdar Ruhe und Freude im Jenseits. Amen.''
London, 26. des 8. Monats 1684. G. F.


% \picinclude{./290-299/p_s295.jpg} 
Zweite Reise nach Holland. Brief an den Herzog von Holstein usw. 295
Jch verließ London für einige Zeit, und besuchte die Freunde
in South-Street und an andern Orten, und hielt Versammlungen,
.. . . Darauf kehrte ich im 3. Monat nach London zurück.
A13 die Jahreöversammlung nahte, war mir angst, etz möchte
den Freunden, die vom Lande kommen wollten, etwas zustoßen
unterwegß, da das Land in großer Aufregung war, weil es hieß«
der Herzog von Monmouth lande im Westen. Aber es gefiel
dem Herm nach seiner großen Güte, die Freunde zu bewahren, und
wir durften unö in Ruhe und Frieden versammeln, und er war unter
uns in unsern Versammlungen mit seiner lebendigen, erquickenden
Gegenwart; sein heiliger Name sei gepriesen ewiglich. Jn Anbe-
tracht der Unruhen, die daß Land erregten, trieb etz mich, am
Schlusse der Versammlungen, einige Zeilen an die Freunde zu
schreiben, um alle zu warnen, sich vor dem Geist dieser Welt, in
welchem Unruhe ist, zu hüten, und in der friedsamen Wahrheit
zu bleiben:
»Liebe Freunde und Brüder,
die der Herr berufen und erwählt hat in Jesu Christo,
euerm Leben und Heil, in welchem ihr alle Ruhe und Frieden
in Gott habt. Gott der Herr hat euch, durch seine mächtige
Kraft, die über allen ist, biz auf diesen Tag bewahrt, daß ihr
ihm sollt ein aus-erwählteö, heiligeö Volk sein, damit ihr durch
seinen ewigen Geist und seine Kraft alle vor der Welt bewahrt
bleiben möget, denn ,,in der Welt habt ihr Angst« (Joh. 16, 33).
GH ist jetzt der große Tag dez allmächtigen Gotteö; er erschüttert
den Himmel und die Erde der irdischen Religionen; ,,ihre Elemente
werden vor Hitze zerschmelzen« (2. Petr. 3, 12) und »Sonne und
Mond werden den Schein verlieren, und die Sterne werden vom
Himmel sallen« (Matth. 24, 29), wie bei den Juden vor dem
Grscheinen Christi. Darum, liebe Brüder, bleibet im Samen,
in Jesuö Chtistue, im Grund und Felsen, der nicht wanken
kann; stehet im Licht und dem Geist Jesu Christi, damit ihr wie
Fixsterne am Firmament Gotteß seid. Jn seiner Kraft und seinem
Licht werdet ihr über alle jene ,,herumirrenden Sterne« sehen,
über »die Wolken ohne Wasser, über die Bäume ohne Früchte«
(Jud.). Was wanken kann, wird jetzt wanken, also alle die ab-
geirrt sind vom Firmament der Kraft Gotteß.
Liebe Freunde und Brüder, die ihr erlöst seid vom Tod und
Fall Adamß, durch Christo?-, den zweiten Adam, in ihm habt


% \picinclude{./290-299/p_s296.jpg} 
296 Kapitel TRV.
ihr Leben, Ruhe und Frieden! denn Ehristuö sagt: ,,in mir habt
ihr Frieden, aber in der Welt habt ihr Angst«. Und der
Apostel sagt: ,,die glauben, gehen in die Ruhe ein« (Hebr. 4, 3)
nämlich zu Ehristuß, der die Welt überwunden hat und der
Schlange den Kopf zertritt und den Teufel und seine Werke zer-
stört und die Zeichen und Wei?-sagungen dez alten Tesiament;3
und der Propheten erfüllt, welcher der Erste und Letzte, Anfang
und Ende ist, die ewige Ruhe. Darum bleibet in Christus eurer
Ruhe, ein jeder, der ihn ausgenommen hat.
Und nun, liebe Freunde und Brüder, maß auch für Unruhen,
Unordnungen, Gewalttaten, Streitigkeiten und Zänkereien in der
Welt entstehen, haltet euch fern von denselben, und bleibet in
der Krast und der Wahrheit, die über allem ist; dann werdet ihr
den Frieden und daß Beste eineß jeden suchen. Lebet in der Liebe,
die Gott in eure Herzen aus-gegossen hat durch Jesuß Christus-.
Jn ihr kann nichtß euch scheiden von Gott und Ehristuß, weder
Trübsal noch Verfolgung, weder Hohes noch Tiefeö, und nichts
kann eure himmlische Jtingerschast im Licht, dem Geist und dem
Evangelium Christi hindern oder zerstören, noch eure heilige
Gemeinschaft im heiligen Geist, welcher vom Vater und vom
Sohn auzgehet, und euch in alle Wahrheit leitet. In diesem
heiligen Geist habt ihr Gemfinschast mit dem Vater und dem Sohn
und untereinander. Gr ist etz, der die Kirche Christi, den Leib,
zusammenhält und mit Christuö, dem himmlischen und geistlichen
Haupt, verbindet. Er ist dat; Band des Friedenß für alle leben-
digen Glieder seiner ganzen Kirche, darin sie Ruhe und ewigen
Frieden in Christuö und Gott haben. Ihm sei Ehre und Preis?
immerdar.
Liebe Freunde, versäumet nicht, euch untereinander zu ver-
sammeln, die ihr verbunden seid im Namen Jesu eureö Propheten,
den Gott im neuen Testament hat erstehen lassen, damit man
ihn in allen Dingen anhöre. Waß er euch eröffnet, kann niemand
verschließen und was er verschließt, kann niemand öffnen; er ist
euer Priester durch die Kraft, über alle Himmel erhaben in ein
ewigeß Leben; durch ihn seid ihr zum königlichen Priestertum be-
rufen, Gott geistliche Opfer zu bringen. Er ist der Bischof eurer
Seelen (1. Petr. 2), daß er über euch wache, damit ihr nicht von
Gott abweichet; er ist der gute Hirte, der sein Leben gelassen hat


% \picinclude{./290-299/p_s297.jpg} 
Zweite Reise noch Holland. Brief an den Herzog oon,Holsteiu usw. 297
für seine Schafe, und sie hören seine Simme und folgen ihm und
er gibt ihnen ewigeß Leben (Joh. 11).
Und nun, liebe Freunde und Brüder, bleibet in Ehristuß,
dem Weinstock, damit ihr Früchte bringen möget zu Gotteß Ehre.
Und wie ihr nun Ehristuß aufgenommen habt, so wandelt in ihm,
der nicht von dieser sündigen Welt ist, damit ihr bewahret bleibt
vor dem eiteln Tun und Treiben dieser Welt, welcheß deß Fleischeß
Lust und der Augen Lust und hofsürtigeß Wesen befriedigten, die
nicht auß dem Vater sind, sondern von der Welt, die oergehet
(1.Joh.2). Ein jeder, der sich an daß hält, waß nicht vom
Vater ist, oder solcheß begünstigt, wendet den Sinn ab von Gott
dem Vater und dem Herrn Jesuß C-hristuß. Darum lasset
C-hristnß in euern Herzen regieren, damit euch Herz, Sinn,
Seele und Geist bewahrt bleiben mögen vor den Eitelkeiten dieser
Welt, in Worten und Werken, und ihr ein außerwählteß Volk
seid, fleißig zu guten Werken, und dem Herm dieuet durch Jesuß
Ehristnß, zu Gotteß Ehre und Lob ..... «
London, 11. deß 4. Monatß 1685. G. F.
Alß die Jahreßversammlung vorüber war, ging ich ein wenig
vor die Stadt hinauß, da ich sehr ermüdet war von der Hitze,
dem Gedränge in den Versammlungen und der unaußgesetzten
Arbeit. Zuerst ging ich nach South-Street, wo ich einige Tage
blieb. Eß kam so recht daß Gefühl über mich vom Wachßtum
und Zunehmen der Eitelkeit, Hofsart und Außschreitung in der
Kleidung, und daß nicht nur unter den Weltleuten, sondern
auch viel zu viel unter etlichen, die sich zu unß hielten und schienen,
die Wahrheit zu bekennen. Unter dem Eindruck, wie verderblich
dieß sei, schrieb ich eine Schrift, wo eß unter anderm heißt:
»Jch laß von einem weisen Philosophen, der, alß er eine Frau
mit bloßem Halß und Nacken antraf, sie fragte: ,,Frau, wollt ihr
dieseß Fleisch oerkaufen?« und alß sie oerneinte, sagte er; ,,Dann
bitte schließt den Laden'', womit er ihre entblößte Brust meinte.
Also sogar unter den Heiden wurden solche alß schlechte Personen
angesehen, alß nicht ehrbare Leute. Darum sollten die, welche
vorgeben, daß wahre Christentum zu kennen, sich solcher Dinge
schämen. Eß gibt sogar von einem Papisten ein Buch gegen daß
Entblößen von Brust und Nacken, und ein andereß von Richard
Baxter, dem Preßbyterianer. Wenn nun solche, dielnur ein äußer-
licheß Bekenntniß haben, so reden, wievielmehr sollten die, welche


% \picinclude{./290-299/p_s298.jpg} 
298 Kapitel QKV1.
im Besitz der Wahrheit und des wahren Christentums sind, sich
solcher Dinge schämen. Bitte, leset das dritte Kapitel des Jesaia,
wie dieser heilige Prophet betrübt war über die hossärtigen Kleider
der närrischen Weiber, und wie er vom Herrn gesandt war, sie
darum zu tadeln. .... «
South-Street, 24.des 4. Monats i685. G. F.
K
Kapitel 20171.
Kampf sür die Ordnung im Quäkertum. Jakob ll. Amnestie.
Nachdem ich einige Wochen in South=Street gewesen war
und dort manche Versammlungen für die Freunde gehalten hatte,
kehrte ich nach London zurück. Hier half ich unter anderm den
Freunden ein Zeugnis aufsetzen, um sie von dem Verdachte zu
reinigen, sie hätten sich am letzten Aufstand im Westen oder an
irgend andern Ausständen oder Vers chw örungen gegen die Regierung
beteiligt. Und dieses Zeugnis wurde dann dem obersten Richter
eingereicht, der im Begriff war, nach dem Westen zu gehen, um
die Gefangenen zu soerhören.
Jch blieb einige Zeit in London und arbeitete im Dienst der
Wahrheit. Dann ging ich für etwa eine Woche wieder aufs
Land, weil meine Gesundheit unter dem Mangel an frischer Lust
sehr litt .... und kehrte dkgnn wieder in die Stadt zurück, wo
ich während zwei Monaten ie Versammlungen besuchte und mein
Möglichstes tat, um für die Freunde, die in allen Teilen des
Landes viel zu leiden hatten, Erleichterungen zu erwirken. Auch
schrieb ich mehrere Schriften zur Förderung der Wahrheit. Die
eine handelte von der Ordnung in der Kirche Gottes, der sich
etliche unter den Freunden stark widerseizt hatten. Sie lautete:
,,Überall in der Welt besteht fiir Familie, Gesellschaft oder Stadt
irgend eine Ordnung. Jin alten Testament war es die Ordnung
Arens und Melchisedeks (Hebr. 7,1 1) nnd darnach die Ordnung Jesu
Christi, und er verachtete diese Ordnung nicht. Gott ist ein Gott
der Ordnung in seiner ganzen Schöpfung, so auch in seiner Kirche.
Und alle, die an das Licht glauben, an das Leben in Christus,
durch das man vom Tod ins Leben eingeht, sind in der Ordnung
des heiligen Geistes, und im Licht und Leben, der Kraft und dem
Reich Jesu Christi, deren Wachstum kein Ende nimmt. Aber
solches ist verborgen den Geistern der Unordnung, die so viel


% \picinclude{./290-299/p_s299.jpg} 
Kampf für die Ordnung im Qnäkertum. Jakob ll. Amnestie. 299
schreiben und drucken gegen die Ordnung, die der Herr durch
seinen Geist und seine Kraft unter seinem Volke ausgerichtet hat.
Ihr, die ihr so viel gegen die Ordnung schreit, ihr seid ja
in ein ,,Land der Finsterniz und dez Dunkelz geraten; ein Land,
da ez stockfinster ist und da keine Ordnung ist, und wo ez ist
wie Finsternis, wenn ez hell wird« (Hiob 10, 21). Jst nicht
diez euer Zustand, wie alle, die in der Wahrheit und nach dem
Evangelium des Lebenz und dez Heilz wandeln, sehen können?« . . .
G. F.
Ich konnte abermalz nicht lange in London bleiben, da ich
die Gingeschlossenheit in der Stadt nicht lange hintereinander
ertragen konnte .... Ehe ich die Stadt verließ, hörte ich von
einem berühmten Gelehrten auz Polen, der kürzlich hergekommen
war; ich lud ihn in meine Wohnung ein und hatte eine lange
Unterredung mit ihm. Nachdem ich mich über allez, was ich zu
wissen wünschte, erkundigt hatte, schrieb ich einen Brief an den
König von Polen, wegen der Freunde in Danzig, die lange
schwer zu leiden gehabt hatten. Ez folgt hier eine Abschrift
davon:
,,An den König von Polen.
An Johann den Dritten, König von Polen, Großherzog von
Lithauen, Rußland und Preußen, Beschützer der Stadt Danzig, ....
wegen der heimgesuchten und unschuldigen Leute, die man im
Groll Quäker nennt, die jetzt bei Wasser und Brot in oben-
genannter Stadt sind, in strenger Gefangenschaft, wo man ihren
Frauen und Kindern kaum erlaubt, sie zu besuchen.
O König!
Die Behörden der Stadt Danzig sagen, ez sei dein Wille,
· daß diesez unschuldige, heimgesuchte Volk solche Unterdrückung zu
erleiden habe. Nun ist die Strafe nur darum über sie verhängt
worden, weil sie zusammenkommen im Namen Jesu Christi ihrez
Erlöserz und Heilandz, der für ihre Sünden starb und zu ihrer
Rechtfertigung von den Toten auferstanden ist, der ihr Prophet
ift, welchen Gott erweckt hat, wie Mosez.
Und nun, in diesen Tagen dez neuen Goangeliumz und dez
neuen Bundez, sollten alle auf ihn hören, die »gewesen wie die
irrenden Schafe, nun aber sich bekehrt haben zum Hirten und
Bischof ihrer Seelen«. ,,Gr hat sein Leben gegeben für seine

