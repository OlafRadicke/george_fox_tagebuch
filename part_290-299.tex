% \picinclude{./290-299/p_s290.jpg} 
einen hervorragenden Lehrer der 
Mennoniten\index{Mennoniten} oder Baptisten\index{Baptisten}.
Ich war bei ihm gewesen, als ich vor sieben Jahren in Holland\index{Holland}
gewesen war, und William Penn\person{Penn, William} 
und George Keith\person{Keith, George} Dispute
mit ihm gehabt hatten. Er war damals sehr hochmütig und
misstrauisch gewesen, so das er nicht von mir angerührt noch
angesehen werden wollte, sondern mich hieß, meine Augen von
ihm abwenden, da sie, wie er behauptete, ihn stechen. Jetzt aber
war er sehr empfänglich und geneigt und bekannte sich bis zu
einem gewissen Grad zur Wahrheit; auch seine Frau und Tochter
waren sehr empfänglich und freundlich, und wir trennten uns
voller Liebe [...].

Wir hatten eine Versammlung in Amsterdam\ort{Amsterdam}, bei welcher
viele außer den Freunden anwesend waren, unter anderm der
große Baptistenlehrer Galenus\person{Galenus}, der dem Zeugnis der Wahrheit
sehr aufmerksam zuhörte und nachher zu mir kam und mir sehr
liebevoll die Hand gab [...].

Wir gingen noch nach Alkmar, Sardam, einer großen Stadt
von Schifsbauern, [...]. dann nach Harlem\ort{Harlem}, wo wir im Haus
eines Freundes eine große Versammlung hatten, [...] darauf nach
Rotterdam\ort{Rotterdam}, wo wir zu zwei Versammlungen blieben, [...] und
am 16. des 5. Monats nach Briel, um uns wieder für England
einzuschiffen [...].

Den folgenden Winter brachte ich in London\ort{London} zu, nur zwei
oder dreimal ging ich mit meiner Frau, die bei mir in der
Stadt war, zu ihrem Sohn nach Kingston\ort{Kingston}. Und obgleich ich
sehr elend war, war ich beständig an der Arbeit, entweder in
öffentlichen Versammlungen\index{Versammlung!Öffentliche}, 
wenn es mir möglich war, sie zu
ertragen, oder in privaten Angelegenheiten unter den Freunden
und mit Besuchen bei solchen, die um der Wahrheit willen litten,
in der Gefangenschaft oder durch Verlust ihrer Habe. Ich schrieb
auch allerlei in dieser Zeit, einiges für den Druck, anderes zu
privatem Gebrauch, so Briefe an den König von 
Dänemark\person{König von Dänemark} und den Herzog von 
Holstein\person{Herzog von Holstein} wegen der Freunde, die in ihren Ländern
zu leiden hatten. Einer dieser Briefe lautet:
\grosszitat{
    An den Herzog von Holstein, 
    den ich in der Liebe Gottes
    bitte, dieses durchzulesen, das ihm in Liebe gesandt wird.
    \bigskip

    Ich hörte, das, als Elisabeth Hendricks\person{Hendricks, Elisabeth} 
    nach Friedrichstadt\ort{Friedrichstadt}
    kam, um die Leute, die ihr Quäker nennt, zu besuchen, dir durch
    einige übelmeinende Personen hinterbracht wurde, es sei ein Ärgernis 
    % \picinclude{./290-299/p_s291.jpg} 
    für die christliche Religion, wenn einer Frau gestattet werde, in
    einer öffentlichen religiösen Versammlung zu reden. Daraufhin hast
    du den Behörden von Friedrichstadt den Befehl gegeben, dafür
    zu sorgen, das besagte Leute die Stadt verlassen, oder sie 
    fortzuschicken. Da aber jene Behörden Arminianer\index{Arminianer} waren, und als ein
    in Holland verfolgtes Volk vor etwa sechzig Jahren hierher 
    gekommen waren, schickten sie die Antwort: \zitat{Wir sind nicht willens,
    andere um der Religion willen zu verfolgen, denn als wir selber
    die Verfolgten waren, so sahen wir die Verfolgung als 
    antichristlich an.} Daraufhin hat das Volk Gottes, zum Spott Quäker
    genannt, an dich, o Herzog, von Friedrichstadt aus geschrieben,
    und seither haben sie ihre Freiheit und friedlichen Versammlungen
    gehabt und konnten frei und ohne Hindernis die letzten beinahe
    zwanzig Jahre in Friedrichstadt Gott dienen und ihn anbeten;
    diese Freiheit betrachten sie als eine große Gunst von dir.
    
    Und in der Tat, o Herzog, der du dich zum Christentum bekennst, 
    das sich nach dem großen, mächtigen Namen Jesu Christi,
    des Könige der Könige, des Herrn aller Herren, nennt und sich
    aus die heiligen Schriften der Wahrheit des alten und neuen
    Testaments gründet, gebrauchst du nicht in deinem Gottesdienst
    oft Worte von Frauen aus dem alten und neuen Testament? Der
    Apostel sagt: ,\zitat{Eure Weiber sollen schweigen in der Gemeinde}
    (1. Cor. 14\bibel{Cor. 1. 14@1. Cor. 14}), und er gestattet 
    den Weibern nicht, das sie lehren,
    sondern sie sollen stille sein, und wenn sie etwas lernen wollen,
    \zitat{so sollen sie daheim ihre Männer fragen, es stehe den Weibern
    übel an, in der Gemeinde zu reden}; und 
    1.~Tim.~2,11+12\bibel{Tim. 1. 02:11+12@1. Tim. 2:11+12}: \zitat{Ein
    Weib lerne in der Stille mit aller Untertänigkeit, und ist ihr nicht
    gestattet, das sie lehre, noch das sie des Mannes Herr sei, sondern
    das sie sei stille}. Der Herzog kann aber sehen, was das für
    Frauen waren, die stille sein sollten und untertänig, denen das
    Geses befiehlt, stille zu sein: es sind die unordentlichen Frauen!
    Denn im gleichen Kapitel gebietet der Apostel den Weibern, \zitat{sich nicht
    mit Zöpfen oder Gold oder Perlen oder köstlichem Gewand zu
    schmücken}. Diese Dinge hat der Apostel verboten, und solche
    Weiber, welche sich also schmücken, die sollen \zitat{in der Stille lernen
    und untertänig sein} und nicht über die Männer herrschen, denn
    solchen stehet es übel, in der Gemeinde zu reden. Sind es aber
    nicht gerade solche Weiber, die Gold, Silber, Perlen und köstliche
    Kleider tragen und sich mit Haarflechten schmücken, die in euern
    % \picinclude{./290-299/p_s292.jpg} 
    Kirchen reden, wenn eure Priester sie die Psalmen singen lassen?
    oder reden sie etwa nicht, wenn sie Psalmen fingen? Solches 
    bedenke, o Herzog! Und dennoch sagst du, eure Weiber sollen schweigen
    in der Kirche und nicht reden, wenn sie aber Psalmen singen in
    der Kirche, sind sie dann still? Und während der Apostel den
    oben genannten Weibern verbietet, in der Kirche zu sprechen, so
    ermutigt er an anderer Stelle die guten und heiligen Frauen, die
    \zitat{Lehrer in guten Dingen zu sein}, wie in 
    Tit.~2,3\bibel{Tit. 02:03@Tit. 2:3}. Der Apostel
    sagt, \zitat{ich bitte dich, mein treuer Geselle, stehe jenen Frauen, die
    mit mir für das Evangelium gekämpft haben, bei und den andern
    meinen Gehilfen, welcher Namen sind im Buche des Lebens}
    (Phil. 4,3\bibel{Phil. 04:03@Phil. 4:3}). Hier anerkennt er 
    diese heiligen Frauen, die mit ihm
    für das Evangelium gekämpft haben, und ermutigt sie und 
    verbietet ihnen nicht, zu reden. Ebenso empfiehlt er Phöbe\person{Phöbe} der
    Kirche von Rom und nennt sie: \zitat{eine Dienerin der Kirche zu
    Kenchreae} (Röm. 16,1\bibel{Röm. 16,01@Röm. 16,1}), sendet 
    durch sie seinen Brief von Corinth
    an die Kirche von Rom und bittet, sie aufzunehmen in dem
    Herrn, wie es den Heiligen geziemt, und sie in allem, was sie
    brauche, zu unterstützen, denn sie sei vielen eine Stütze gewesen,
    so auch ihm selber. Und ferner sagt er: \zitat{Grüset Aquila und
    Priscilla, meine Helfer in Jesus Christus.} 
    (Röm. 16,3\bibel{Röm. 16,03@Röm. 16,3}). Hier
    kann der Herzog sehen, das dieses gute, heilige Frauen waren,
    denen der Apostel nicht verbot zu reden, sondern es gebot.
    Und Aquila und Priscilla legten \zitat{Apollo den Weg Gottes noch
    fleißiger aus} (Act. 18,26\bibel{Act. 18:26}), hier 
    war also Priscilla Lehrer so gut
    wie Aquila, und solchen heiligen Frauen verbietet der Apostel
    das Reden nicht. Ebenso verbot er den vier Töchtern des 
    Philippus, welche Jungfrauen waren, nicht, zu weissagen. Und die
    Weiber dürfen in der Kirche beten und weissagen 
    (1.~Cor.~11,5\bibel{Cor. 1. 11:05@1. Cor. 11:5}).
    Die Apostel zeigten den Juden die Erfüllung der Weissagungen
    des Propheten Joel: \zitat{In den letzten Tagen will ich meinen
    Geist ausgießen über alles Fleisch, und eure Söhne und Töchter,
    Knechte und Mägde sollen weissagen durch den Geist Gottes}.
    Der Apostel ermahnt also die Töchter und Mägde so gut wie die
    Söhne, zu weissagen, und wenn sie weissagen, \zitat{so sollen sie zur
    Gemeinde oder zum Volk reden} 
    (Joel.~2,28\bibel{Joel. 02:28@Joel. 2:28}, 
    Act.~2,18\bibel{Act. 02:18@Act. 2:18}). Sangen
    nicht Mirjam, die Prophetin, und alle Frauen mit ihr, dem Herrn,
    als er die Kinder Israels von Pharao errettet hatte? Pries sie
    nicht den Herrn und prophezeite in der Versammlung der Kinder
    % \picinclude{./290-299/p_s293.jpg} 
    Israel? und geschah das nicht vor der Gemeinde? 
    (2.Mos. 15:21\bibel{.Mos. 2 15:21@2.Mos. 15:21}).
    Moses und Aaron verboten ihr nicht, zu weissagen und zu reden,
    sondern Moses sagte: \zitat{Wollte Gott, das das ganze Volk des Herrn
    weissagte}! (4. Mos. 11:29\bibel{Mos. 4. 11:29@4. Mos. 11:29}). 
    Und \zitat{das Volk des Herrn} sind
    Frauen so gut wie Männer. Deborah war Richterin und Prophetin,
    und gebraucht ihr nicht die Worte Deborahs und Mirjams in euren
    Gottezdiensten? Denket an Deborahs lange Rede oder Gesang!
    Barak verbot es ihr nicht, noch irgend einer der jüdischen Priester, und
    hielt sie nicht diese Rede in der Versammlung oder Kirche 
    Israels? (Rich. 5\bibel{Rich. 05@Rich. 5}). Im Buch Ruth sind 
    gute Reden jener trefflichen Frauen, die man auch nicht verboten hat. Hanna betete
    im Tempel vor Eli, und der Herr erhörte ihr Gebet. Sieh doch,
    was; für eine Rede Hanna hielt, und wie sie Gott pries vor Elis
    Ohren, und er verbot es ihr nicht 
    (1.~Sam.~2:1-10\bibel{Sam. 1. 02:01@1. Sam. 2:1-10}). Josia
    schickte seinen Priester mit mehreren andern zur Prophetin Hulda,
    die in Jerusalem wohnete, um sie um Rat zu fragen 
    (2.~Könige~22:14\bibel{Könige 2. 22:14@2.~Könige~22:14}). 
    Hier haben also der König sowie seine Priester
    den Rat der Prophetin nicht verachtet, und sie weissagte 
    in der Versammlung vor den Israeliten, wie man in diesen
    Kapiteln sehen kann.

    Und sieh bei Lukas den göttlichen Gesang der Elisabeth an
    Maria und den göttlichen langen Gesang Marias, wo Maria sagte:
    \zitat{der Herr hat die Niedrigkeit seiner Magd angesehen.} Und braucht
    ihr nicht in euern Gottesdiensten und Gebete die Worte der
    Maria und Elisabeth, aus Lukas 1:41—55\bibel{Lukas 01:41@Lukas 1:41—55} 
    und verbietet dennoch
    den Frauen in euren Kirchen zu reden? Auch reden alle möglichen
    Frauen in euren Kirchen, wenn sie singen\index{Singen} und Amen sagen. In
    Lukas 2\bibel{Lukas 02@Lukas 2} war die Prophetin Hanna, eine Witwe von 84 Jahren,
    die nicht aus dem Tempel ging, sondern Gott diente mit Fasten
    und Beten Tag und Nacht. Hatte sie nicht von Christus 
    gezeugt im Tempel und dem Herrn Danksagung gebracht, und zu
    allen, die da warteten auf die Erlösung zu Jerusalem, von
    Christum geredet? (Luk. 2:36—38\bibel{Luk. 02:36—38@Luk. 2:36—38}). 
    Solchen heiligen Frauen
    war es nicht verboten, in der Kirche zu reden, weder im Gesetz
    noch im Evangelium. Waren es nicht Maria Magdalena und
    andere Frauen, die zuerst die Auferstehung Christi den Aposteln
    verkündeten? Das Weib war es gewesen (nämlich Eva), die
    zuerst sündigte, und so waren es auch Frauen, die zuerst die  
    erstehung Christi verkündeten; denn Christus sagte zu Maria, \zitat{Gehe
    % \picinclude{./290-299/p_s294.jpg} 
    zu meinen Brüdern und sage ihnen, ich gehe hin zu meinem
    Vater und zu eurem Vater, zu meinem Gott und zu eurem Gott}
    (Joh. 20:17\bibel{Joh. 20:17@Joh. 20:17}). Und 
    Luk. 24:10\bibel{Luk. 24:10} waren es Maria Magdalena und
    Johanna und Maria die Mutter des Jakobus und andere Frauen,
    die mit ihnen waren, die den Aposteln sagten, das Christus von
    den Toten auferstanden sei: \zitat{und es deuchten sie ihre Worte eben,
    als wären es Märlein, und glaubten ihnen nicht.} Und Vers 22:
    \zitat{Auch haben uns erschreckt etliche Weiber der Unsern.} Hier
    kann man sehen, das die Reden der Frauen von der Auferstehung
    Christi die Apostel erstaunten. Christus sandte die Frauen, seine
    Auferstehung zu predigen; es ist also keine Schande für Frauen,
    Christus zu predigen, und wenn Christus sie schickt, so dürfen sie
    nicht schweigen. Der Apostel sagt: \zitat{Jede Zunge soll bekennen}
    (Röm. 14:11\bibel{Röm. 14:11}); und Phil. 2:11\bibel{Phil. 02:11@Phil. 2:11}: 
    \zitat{Jede Zunge soll bekennen, das
    Christus der Herr ist, zur Ehre Gottes des Vaters.} Das zeigt
    klar, das die Frauen so gut wie die Männer Christus bekennen
    sollen, wenn jede Zunge ihn bekennen soll. Und der Apostel sagt,
    \zitat{da ist weder Mann noch Weib, sondern sie sind alle eins in
    Christo Jesu} (Gal. 3:28\bibel{Gal. 03:28@Gal. 3:28}). Und 
    wenn es heißt, die Frauen
    sollen daheim ihre Männer fragen, so weiß ja der Herzog wohl,
    das Jungfrauen keine Männer haben noch die Witwen. Hanna,
    die Prophetin, war eine Witwe. Und da ja Christus der \zitat{Eine
    Mann ist} (2. Cor. 11:2\bibel{Cor. 2. 11:2@2. Cor. 11:2}), 
    so müssen ihn die Männer daheim
    fragen so gut wie die Frauen, ehe sie lehren. Und gesetzt der
    Fall, das Weib eines Türken sei eine Christin, oder das Weib
    eines Papisten eine Lutheranerin oder Calvinistin, müssen sie
    dann auch ihre Männer daheim fragen und von ihnen lernen, ehe
    sie Christus in der Versammlung des Herrn bekennen? Dann
    würden sie den Rat erhalten, Türken oder Papisten zu werden.

    Ich bitte den Herzog, diese Dinge zu bedenken. Ich bitte
    ihn, aus Gottes Gnade und Wahrheit in seinem Herzen, die aus
    Jesus kommen, zu merken, damit er durch diesen Geist der
    Gnade und Wahrheit dazu kommen möge, Gott im Geist und in
    der Wahrheit anzubeten und ihm zu dienen, ihm, dem lebendigen,
    ewigen Gott, der ihn gemacht hat, und Frieden zu haben in
    Christus, den Frieden, den die Welt nicht nehmen kann. Ich
    wünsche ihm Glück, Frieden und Wohlergehen in dieser Welt und
    immerdar Ruhe und Freude im Jenseits. Amen.

\bigskip
\begin{flushright}
London\ort{London}, 26. des 8. Monats 1684\index{Jahr!1684}. G. F.\end{flushright}
}

% \picinclude{./290-299/p_s295.jpg} 
Ich verließ London für einige Zeit, und besuchte die Freunde
in South-Street\ort{South-Street} und an andern Orten, 
und hielt Versammlungen, [...]. Darauf kehrte ich 
im 3. Monat nach London zurück.

Als die Jahresversammlung nahte, war mir angst, es möchte
den Freunden, die vom Lande kommen wollten, etwas zustoßen
unterwegs, da das Land in großer Aufregung war, weil es hieß
der Herzog von Monmouth lande im Westen. Aber es gefiel
dem Herrn nach seiner großen Güte, die Freunde zu bewahren, und
wir durften uns in Ruhe und Frieden versammeln, und er war unter
uns in unsern Versammlungen mit seiner lebendigen, erquickenden
Gegenwart; sein heiliger Name sei gepriesen ewiglich. In 
Anbetracht der Unruhen, die das Land erregten, trieb es mich, am
Schlusse der Versammlungen, einige Zeilen an die Freunde zu
schreiben, um alle zu warnen, sich vor dem Geist dieser Welt, in
welchem Unruhe ist, zu hüten, und in der friedsamen Wahrheit
zu bleiben:

\grosszitat{
    Liebe Freunde und Brüder,
    \bigskip
    die der Herr berufen und erwählt hat in Jesu Christo,
    euerm Leben und Heil, in welchem ihr alle Ruhe und Frieden
    in Gott habt. Gott der Herr hat euch, durch seine mächtige
    Kraft, die über allen ist, bis auf diesen Tag bewahrt, das ihr
    ihm sollt ein auserwähltes, heiliges Volk sein, damit ihr durch
    seinen ewigen Geist und seine Kraft alle vor der Welt bewahrt
    bleiben möget, denn \zitat{in der Welt habt ihr Angst} 
    (Joh. 16:33\bibel{Joh. 16:33}).
    Es ist jetzt der große Tag des allmächtigen Gottes; er erschüttert
    den Himmel und die Erde der irdischen Religionen; \zitat{ihre Elemente
    werden vor Hitze zerschmelze} 
    (2.~Petr.~3:12\bibel{Petr. 2. 03:12@2. Petr. 3:12}) und 
    \zitat{Sonne und
    Mond werden den Schein verlieren, und die Sterne werden vom
    Himmel sfallen} (Matth. 24:29\bibel{Matth. 24:29}), 
    wie bei den Juden vor dem
    Erscheinen Christi. Darum, liebe Brüder, bleibet im Samen,
    in Jesus Christus, im Grund und Felsen, der nicht wanken
    kann; stehet im Licht und dem Geist Jesu Christi, damit ihr wie
    Fixsterne am Firmament Gottes seid. In seiner Kraft und seinem
    Licht werdet ihr über alle jene \zitat{herumirrenden Sterne} sehen,
    über \zitat{die Wolken ohne Wasser, über die Bäume ohne Früchte}
    (Jud.). Was wanken kann, wird jetzt wanken, also alle die 
    abgeirrt sind vom Firmament der Kraft Gottes.

    Liebe Freunde und Brüder, die ihr erlöst seid vom Tod und
    Fall Adams, durch Christus, den zweiten Adam, in ihm habt
    % \picinclude{./290-299/p_s296.jpg} 
    ihr Leben, Ruhe und Frieden! denn Christus sagt: \zitat{in mir habt
    ihr Frieden, aber in der Welt habt ihr Angst}. Und der
    Apostel sagt: \zitat{die glauben, gehen in die Ruhe ein} 
    (Hebr. 4:3\bibel{Hebr. 04:03@Hebr. 4:3})
    nämlich zu Christus, der die Welt überwunden hat und der
    Schlange den Kopf zertritt und den Teufel und seine Werke 
    zerstört und die Zeichen und Weissagungen des alten Testaments
    und der Propheten erfüllt, welcher der Erste und Letzte, Anfang
    und Ende ist, die ewige Ruhe. Darum bleibet in Christus eurer
    Ruhe, ein jeder, der ihn ausgenommen hat.

    Und nun, liebe Freunde und Brüder, was auch für Unruhen,
    Unordnungen, Gewalttaten, Streitigkeiten und Zänkereien in der
    Welt entstehen, haltet euch fern von denselben, und bleibet in
    der Kraft und der Wahrheit, die über allem ist; dann werdet ihr
    den Frieden und das Beste eines jeden suchen. Lebet in der Liebe,
    die Gott in eure Herzen ausgegossen hat durch Jesus Christus.
    In ihr kann nichts euch scheiden von Gott und Christus, weder
    Trübsal noch Verfolgung, weder Hohes noch Tiefes, und nichts
    kann eure himmlische Jüngerschaft im Licht, dem Geist und dem
    Evangelium Christi hindern oder zerstören, noch eure heilige
    Gemeinschaft im heiligen Geist, welcher vom Vater und vom
    Sohn ausgehet, und euch in alle Wahrheit leitet. In diesem
    heiligen Geist habt ihr Gemeinschaft mit dem Vater und dem Sohn
    und untereinander. Er ist es, der die Kirche 
    Christi\index{Kirche Christi}, den Leib,
    zusammenhält und mit Christus, dem himmlischen und geistlichen
    Haupt, verbindet. Er ist das Band des Friedens für alle 
    lebendigen Glieder seiner ganzen Kirche, darin sie Ruhe und ewigen
    Frieden in Christus und Gott haben. Ihm sei Ehre und Preis
    immerdar.

    Liebe Freunde, versäumet nicht, euch untereinander zu 
    versammeln, die ihr verbunden seid im Namen Jesu eures Propheten,
    den Gott im neuen Testament hat erstehen lassen, damit man
    ihn in allen Dingen anhöre. Was er euch eröffnet, kann niemand
    verschließen und was er verschließt, kann niemand öffnen; er ist
    euer Priester durch die Kraft, über alle Himmel erhaben in ein
    ewiges Leben; durch ihn seid ihr zum königlichen Priestertum 
    berufen, Gott geistliche Opfer zu bringen. Er ist der Bischof eurer
    Seelen (1. Petr. 2\bibel{Petr. 1. 02@1. Petr. 2}), das er 
    über euch wache, damit ihr nicht von
    Gott abweichet; er ist der gute Hirte, der sein Leben gelassen hat
    % \picinclude{./290-299/p_s297.jpg} 
    für seine Schafe, und sie hören seine Stimme und folgen ihm und
    er gibt ihnen ewiges Leben (Joh. 11\bibel{Joh. 11}).

    Und nun, liebe Freunde und Brüder, bleibet in Christus,
    dem Weinstock, damit ihr Früchte bringen möget zu Gottes Ehre.
    Und wie ihr nun Christus aufgenommen habt, so wandelt in ihm,
    der nicht von dieser sündigen Welt ist, damit ihr bewahret bleibt
    vor dem eitlen Tun und Treiben dieser Welt, welches des Fleisches
    Lust und der Augen Lust und hofartiges Wesen befriedigten, die
    nicht aus dem Vater sind, sondern von der Welt, die vergehet
    (1.~Joh.~2\bibel{Joh. 1. 02@1.~Joh.~2}). Ein jeder, der 
    sich an das hält, was nicht vom
    Vater ist, oder solches begünstigt, wendet den Sinn ab von Gott
    dem Vater und dem Herrn Jesus Christus. Darum lasset
    Christus in euren Herzen regieren, damit euch Herz, Sinn,
    Seele und Geist bewahrt bleiben mögen vor den Eitelkeiten dieser
    Welt, in Worten und Werken, und ihr ein auserwähltes Volk
    seid, fleißig zu guten Werken, und dem Herrn dienet durch Jesus
    Christns, zu Gottes Ehre und Lob [...].
    \bigskip
    \begin{flushright}
    London\ort{London}, 11. des 4. Monats 1685\index{Jahr!1685}. G. F.\end{flushright}
}

Als die Jahresversammlung vorüber war, ging ich ein wenig
vor die Stadt hinaus, da ich sehr ermüdet war von der Hitze,
dem Gedränge in den Versammlungen und der unausgesetzten
Arbeit. Zuerst ging ich nach South-Street, wo ich einige Tage
blieb. Es kam so recht das Gefühl über mich vom Wachstum
und Zunehmen der Eitelkeit\index{Eitelkeit}, 
Hofsart und Ausschreitung in der
Kleidung, und das nicht nur unter den Weltleuten, sondern
auch viel zu viel unter etlichen, die sich zu uns hielten und schienen,
die Wahrheit zu bekennen. Unter dem Eindruck, wie verderblich
dies sei, schrieb ich eine Schrift, wo es unter andrem heißt:

\grosszitat{
    Ich las von einem weisen Philosophen, der, als er eine Frau
    mit bloßem Hals und Nacken antraf, sie fragte: \zitat{Frau, wollt ihr
    dieses Fleisch verkaufen?} und als sie verneinte, sagte er; \zitat{Dann
    bitte schließt den Laden}, womit er ihre entblößte Brust meinte.
    Also sogar unter den Heiden wurden solche als schlechte Personen
    angesehen, als nicht ehrbare Leute. Darum sollten die, welche
    vorgeben, das wahre Christentum zu kennen, sich solcher Dinge
    schämen. Es gibt sogar von einem Papisten\index{Papisten} 
    ein Buch gegen das
    Entblößen von Brust und Nacken, und ein anderes von Richard
    Baxter\person{Baxter, Richard}, 
    dem Presbyterianer\index{Presbyterianer}. Wenn nun solche, 
    die nur ein äußerliches Bekenntnis haben, so reden, 
    wie viel mehr sollten die, welche
    % \picinclude{./290-299/p_s298.jpg} 
    im Besitz der Wahrheit und des wahren Christentums sind, sich
    solcher Dinge schämen. Bitte, leset das dritte Kapitel des Jesaia,
    wie dieser heilige Prophet betrübt war über die hofärtigen Kleider
    der närrischen Weiber, und wie er vom Herrn gesandt war, sie
    darum zu tadeln. [...].
    \bigskip
    \begin{flushright}
    South-Street, 24. des 4. Monats 1685\index{Jahr!1685}. G. F.    
    \end{flushright}

}
%%%%%%%%%%%%%%%%%%% Kapitel 26. %%%%%%%%%%%%%%%%%%%%%%%%%%%%%%
\chapter[Jakob II. Amnestie]{Jakob II. Amnestie}

\begin{center}
\textbf{Kampf für die Ordnung im Quäkertum. Jakob II. Amnestie.}
\end{center}


Nachdem ich einige Wochen in South-Street\ort{South-Street} gewesen war
und dort manche Versammlungen für die Freunde gehalten hatte,
kehrte ich nach London\ort{London} zurück. Hier half 
ich unter andrem den
Freunden ein Zeugnis\index{Zeugnis} aufsetzen, um sie 
von dem Verdachte zu
reinigen, sie hätten sich am letzten Aufstand im Westen oder an
irgend andern Ausständen oder Verschwörungen gegen die Regierung
beteiligt. Und dieses Zeugnis wurde dann dem obersten Richter
eingereicht, der im Begriff war, nach dem Westen zu gehen, um
die Gefangenen zu verhören.

Ich blieb einige Zeit in London und arbeitete im Dienst der
Wahrheit. Dann ging ich für etwa eine Woche wieder aufs
Land, weil meine Gesundheit unter dem Mangel an frischer Lust
sehr litt [...] und kehrte dann wieder in die Stadt zurück, wo
ich während zwei Monaten die Versammlungen besuchte und mein
Möglichstes tat, um für die Freunde, die in allen Teilen des
Landes viel zu leiden hatten, Erleichterungen zu erwirken. Auch
schrieb ich mehrere Schriften zur Förderung der Wahrheit. Die
eine handelte von der Ordnung in der Kirche Gottes, der sich
etliche unter den Freunden stark 
widersetzt\index{Konflikt!unter Quakern} hatten. Sie lautete:

\grosszitat{
Überall in der Welt besteht für Familie, Gesellschaft oder Stadt
irgend eine Ordnung. Im alten Testament war es die Ordnung\index{Ordnung!der Zusammenlebens}
Arons und Melchisedeks (Hebr. 7:11\bibel{Hebr. 7:11}) und 
danach die Ordnung Jesu
Christi, und er verachtete diese Ordnung nicht. Gott ist ein Gott
der Ordnung in seiner ganzen Schöpfung, so auch in seiner Kirche.
Und alle, die an das Licht glauben, an das Leben in Christus,
durch das man vom Tod ins Leben eingeht, sind in der Ordnung
des heiligen Geistes, und im Licht und Leben, der Kraft und dem
Reich Jesu Christi\index{Reich Gottes}, deren Wachstum 
kein Ende nimmt. Aber
solches ist verborgen den Geistern der Unordnung, die so viel
% \picinclude{./290-299/p_s299.jpg} 
schreiben und drucken gegen die Ordnung, die der Herr durch
seinen Geist und seine Kraft unter seinem Volke aufgerichtet hat.

Ihr, die ihr so viel gegen die Ordnung schreit, ihr seid ja
in ein \zitat{Land der Finsternis\index{Finsternis} und 
des Dunkels geraten; ein Land,
da es stockfinster ist und da keine Ordnung ist, und wo es ist
wie Finsternis, wenn es hell wird} 
(Hiob 10:21\bibel{Hiob 10:21}). Ist nicht
dies euer Zustand, wie alle, die in der Wahrheit und nach dem
Evangelium des Lebens und des Heils wandeln, sehen können? [...]
\bigskip
\begin{flushright}
G. F.\end{flushright}
}


Ich konnte abermals nicht lange in London bleiben, da ich
die Eingeschlossenheit in der Stadt nicht lange hintereinander
ertragen konnte [...]\index{Stadtleben}. Ehe ich die Stadt 
verließ, hörte ich von
einem berühmten Gelehrten aus Polen, der kürzlich hergekommen
war; ich lud ihn in meine Wohnung ein und hatte eine lange
Unterredung mit ihm. Nachdem ich mich über alles, was ich zu
wissen wünschte, erkundigt hatte, schrieb ich einen Brief an den
König von Polen, wegen der Freunde in Danzig, die lange
schwer zu leiden gehabt hatten. Es folgt hier eine Abschrift
davon:

