% \picinclude{./130-139/p_s130.jpg} 
130 Kapitel Il.
in der Zucht der göttlichen Gnade, welche selig macht. Die aber,
die Gottes Gnade in Mutwillen kehren und sein Licht hassen
(Jud.), sind verworfen; darum ermahne ich alle, an das Licht zu
glauben wie Christus gebietet, und die Gnade, die sie umsonst
lehrt, anzunehmen; dann werden sie gewißlich selig, denn sie
genüget. Viele andere Schriftstellen über die Verwersung wurden
auch noch ausgelegt, und die Augen der Leute wurden geöffnet,
so daß eine Quelle des Lebens unter ihnen hervorsprudelte.
Solches kam den Ptiestetct bald zu Ohren; denn den Leuten,
welche durch ihre schrecklichen Lehren irre geführt worden waren,
gingen allmählich die Augen auf, und sie kamen in den- Bund
des Lichts. Die Kunde, daß ich nach Schottland gekommen sei,
verbreitete sich unter den Priestern. Und sie erhoben ein großes
Geschrei, daß jetzt alles aus sei; denn ich hätte schon in England
alle rechten Männer und Frauen abspenstig gemacht, und ihnen
bleibe dann, wie sie selber zugaben, der schlechtere Teil. Sie ver-
anstalteten darum große Zusammenkünfte von Priestern und
stellten eine ganze Reihe von Verdammungen zusammen, welche
in den Turmhäusern verlesen werden sollten, und die Leute sollten
,,Amen« dazu sagen. Ginige davon will ich hier mitteilen. Zuerst
hieß es: ,,Verflucht ist, wer sagt, ein jeder habe ein Licht in sich,
welches genüge, um ihn selig zu machen. Dazu sage ein jeder:
Amen.« .... Nun sagt aber Christus: ,,Glaubet an das Licht,
damit ihr Kinder des Lichtes werdet« (Joh. 12,36) und weiter:
,,wer da glaubt, der soll selig werden« (Mark. 16); und ,,wer
da glaubt kommt vom Tode ins Leben« .... Und der Apostel
sagt: ,,Jhr tut wohl, aus das Licht zu achten, das da scheinet
an einem dunklen Ort, bis der Tag anbreche und der Morgen-
stern ausgehe in euren Herzen« (2. Petr. 1, 19) ..... Was
den 2. Punkt anbelangt, wo es heißt: ,,Verflucht, wer sagt, der
Glaube sei ohne Sünde,« .... so ist er ja eine Gabe Gottes
und jede Gabe Gottes ist rein .... . Der Glaube, dessen
Ursprung Christus ist, ist köstlich, göttlich und ohne Sünde. Dies
ist der Glaube, der die Herrschaft über die Sünde gibt und den
Zugang zu Gott ..... Aber sie sind alle von diesem Glauben
abgefallen .....
Gs waren in Schottland zwei Kirchen der Jndependenten;
in der einen fanden viele Bekehrungen statt; aber der Pre-
diger der andern war sehr erbost über die Wahrheit und die


% \picinclude{./130-139/p_s131.jpg} 
Reise in Schottland. Kampf gegen die Prädestinationslehre usw. 131
Freunde. Sie hatten Alteste die sich oft bestrebten, ihre Gaben
an ihren Gemeindegliedern zu brauchen und sich ost recht empfänglich
zeigten; aber da ihr Prediger so viel gegen uns und gegen das
Licht redete, verdunkelte sich ihr Blick, das; sie ganz blind wurden
und ganz dürr und ihre Empsänglichkeit verloren. Er fuhr sort
gegen die Freunde und gegen das Licht, aus Christus zu
predigen und nannte dasselbe ein natürliches Licht. Eines Tages
beschimpfte er in seiner Predigt das Licht und da fiel er hin wie
tot in seinem Pult. Man trug ihn hinaus und legte ihn auf
einen Grabstein und flößte ihm ein starkes Getränk ein, das ihn u
wieder zum Leben brachte; und sie trugen ihn heim, aber er war
schwachsinnig geworden. Er riß sich die Kleider vom Leib, hüllte
sich in einen schottischen Plaid und ging aufs Land zu den
Milchmädchen. Nachdem er etwa zwei Wochen dort gewesen,
kehrte er zurück und stieg wieder auf die Kanzel. Nun erwarteten
die Leute große Grössnungen von ihm; statt dessen erzählte er,
wie ihm eines der Mädchen abgerahmte Milch, ein anderes
Buttermilch und wieder ein anderes gewöhnliche Milch gegeben
habe; man mußte ihn wie-der von der Kanzel herunter holen und
heim führen. Der, welcher mir dies alles berichtete, ist Andrew
Robinson, einer seiner eisrigsten Zuhörer, der aber später sich
bekehrte und die Wahrheit annahm. Gr sagte mir, daß er nie
etwas davon gehört habe, daß jener Prediger seinen Verstand
wieder bekommen habe. Daran möge ein jeder sehen, wie es
dem geht, der das Licht beschimpft,. das Licht, welches das Leben s
in Christus, dem Wort, ist; und es möge allen zur Warnung s
dienen, welche Übles reden gegen das Licht Christi ..... Z
Viele der schottischen Priester waren sehr in Aufregung über
die Verbreitung der Wahrheit, weil sie dadurch ihre Zuhörer
verloren; und viele von ihnen gingen darum nach Edinburg,
um beim Rate Oliver Cromwells eine Klage gegen mich vorzu-
bringen. Jnsolge dieser eingereichten Klage kam, als ich einmal
aus einer Versammlung zurückkam, ein Beamter und brachte mir
folgenden Befehl:
,,Donnerstag, 8. Oktober 1657, der Rat Seiner Hoheit in
Schottland.
K Es wird befohlen, daß George Fox nächsten Dienstag,
13. Oktober, vormittags, vor dem Rat erscheint-»
G. Downing, Ratsbeamter.«
gt


% \picinclude{./130-139/p_s132.jpg} 
132 Kapitel Il.
Als er mir den Befehl übergab, fragte er mich, ob ich kommen
wolle oder nicht. Jch antwortete ihm nicht darauf, sondern
fragte, ob der Befehl auch nicht gefälscht sei? er erwiderte nein,
es sei ein richtiger Befehl vom Rat, und er sei als Bote damit
gesandt. Jch erschien also zur vorgeschriebenen Zeit und wurde in
einen großen Saal geführt, wo viele angesehene Leute versammelt
waren, die mich alle aufmerksam betrachteten; schließlich wurde
ich ins Ratszimmer geführt und unter der Türe nahm man mir
den Hut ab; ich fragte, warum das geschehe? wer denn drinnen
sei, daß ich den Hut abnehmen müsse? ich habe ihn ja sogar vor
dem Protektor nicht abgenommen. Aber der Hut wurde aufge-
hängt und ich wurde hineingeführt. Als ich schon eine ganze
Weile drinnen war, ohne daß jemand etwas zu mir sagte, trieb
mich der Herr zu sagen; ,,Friede sei mit euch! wartet in der
Furcht Gottes auf den Empfang seiner Weisheit von oben, durch
die alle Dinge geschaffen sind, daß sie euch in allem, was euch
fz zu tun übergeben ist, leite, damit ihr es tut zur Ehre Gottes-«.
t Sie fragten mich, weshalb ich nach Schottland gekommen sei? ich
sagte: um den Samen Gottes aufzusuchen, der solange in den
Banden des Bösen gelegen habe, damit alle, welche sich in diesem
Lande zur Schrift, den Worten Christi, der Apostel und der Pro-
pheten bekennen, zum Licht und Geist und zur Kraft kommen, in
denen jene, die solche Worte geäußert, gewesen sind; und daß sie
in diesem Geist die Schrift verstehen und Christus und Gott er-
kennen und mit ihm und unter einander in der rechten Gemein-
—s— schaft stehen möchten. s Sie fragten mich, ob ich irgend etwas Ge-
schäftliches hier zu besorgen habe? Jch verneinte; darauf fragten
sie weiter: wie lange ich im Lande bleiben wolle? Jch antwortete,
dies könne ich nicht sagen, wahrscheinlich nicht sehr lange; doch
da meine Freiheit dem Herrn gehöre, so müsse ich den Willen
dessen, der mich gesandt habe, tun. Darauf hieß man mich hin-
ausgehen. Bald daraus ließ man mich wieder herein kommen
und erklärte mir, ich müsse Schottland verlassen, von jetzt an in
7 Tagen. Jch fragte: warum? wa-3 ich getan habe? Sie sagten,
sie wollen nicht mit mir verhandeln. Darauf bat ich sie, zu hören,
was ich ihnen zu sagen habe; aber sie wollten nicht. Jch er-
innerte sie daran, daß Pharao, der doch ein Heide gewesen sei,
Moses und Aaron angehört habe, und Herodes hörte Johannes
den Täufer; sie sollten doch nicht schlechter sein als jene! Aber


% \picinclude{./130-139/p_s133.jpg} 
Reise in Schottland. Kampf gegen die Prädeftinationzlehre usw. 133
sie schrien: »hinautz! hinautz!«, woraus ich wieder hinautzgesührt
wurde. Jch kehrte in meine Wohnung zurück und fuhr fort, in
Gdinburg die Freunde zu besuchen und auszurichten im Herrn.
Ich schrieb darauf an den Rat, um ihm sein unchristlichetz Be-
nehmen gegen mich oorzuhalten .....
Nach einiger Zeit ging ich wieder nach Headtz, wo die
Freunde in großer Not gewesen waren; denn die Pretzbyteri-
aner-Priester hatten sie in den Bann getan und befohlen,
etz solle niemand von ihnen kaufen oder ihnen etwatz
verkaufen oder mit ihnen essen und trinken. So konnten sie weder
ihre Ware verkaufen, noch sich datz ihnen Nötige anschaffen, watz
viele in große Bedrängnitz brachte. Denn wenn einer von ihren
Nachbarn ihnen Brot oder andere Lebentzmittel verkauft hätte, so
hätte ihn der Priester derart bedroht, daß er schleunigst gekommen
wäre, die Sachen wieder zu holen. Aber Oberst Ashsield, welcher
der Friedentzrichter jener Gegend war, machte diesem Vorgehen
der Priester ein Ende. Später wurde er selber gewonnen und hielt
Versammlungen in seinem Hause, verkündete selber die Wahrheit
und lebte und starb in derselben .....
Die Wahrheit und die Kraft detz Herm breitete sich autz in
Schottland, und durch die Kraft und den Geist Gottetz wurden
viele zum Herrn Jesutz Ehristutz bekehrt, ihrem Heiland und Lehrer,
der sein Blut für sie vergossen hat; und etz ist seither ein grorßetz
Wachtztum und wird etz immer mehr sein in Schottland. Denn
altz zuerst die Hufe meinetz Pferdetz schottischen Boden berührten,
da fühlte ich, wie überall Funken detz Samentz von Gott um mich
herum aufsprühten, wie unzählige Feuerfunken. Nicht altz ob
nicht noch viel hartetz, schlechtetz Erdreich von Falschheit und
Heuchelei dort gewesen wäre und ein knorriger Boden, der zu-
erst noch durch Gottetz Wort fruchtbar gemacht werden muß
und gepfliigt mit dem Pflug detz Geistetz, ehe der Same Gottetz
geistliche, himmlische Früchte hervorbringen kann zu Gottetz Ehre.
Aber der Landmann muß in Geduld warten (Jar. 5, 7).


% \picinclude{./130-139/p_s134.jpg} 
134 Kapitel 111.
Kapitel Ill.
Erste Jahresoersannnlung. Warnung an Crouiwell vor der Königs-
krone. Trostbries an dessen Tochter. Gesichte vorn Tode Cronuoells
und der kouttnenden Reaktion.
Wir gingen zurück nach England ..... Die Priester
von Newcastel hatten verschiedene Bücher gegen uns geschrieben,
und ein Stadtitltester, Ledger, war uns und der Wahrheit sehr abge-
neigt. Gr sowie die Priester hatten behauptet, die Quäker könnten
nicht in einer Stadt leben, sondern schwirrten wie die Schmetterlinge
in den Hochtälern. Zu diesem Ledger und einigen andern Stadt-
ältesten ging ich, mit Anthony Pearson, mit der Bitte, eine Ver-
sammlung in Neweastel abhalten zu dürfen, nachdem sie so viel
gegen uns geschrieben haben, da wir nun ja in ihre große Stadt
gekommen seien! Aber sie wollten uns keine Versammlung ge-
statten, noch wollten sie mit sich reden lassen. Jch sagte: »Habt
ihr nicht die Freunde Schmetterlinge genannt und gesagt, wir
könnten nicht in Städten leben; nun sind wir in eure Stadt gekommen,
und ihr wollt uns nicht hören. Wer sind nun die Schmetterlinge?«
Ledger sing an, die Sabbathheiligung zu verteidigen. Aber ich
erwiderte ihm, an dem Tage, welcher der Sabbath sei, dem siebenten
Wochentag, hielten sie ja Märkte und Jahrmärkte; während der Tag,
an dem sich die, welche sich jetzt Christen nennen, versammeln, ja
der erste Wochentag sei. Da wir keine öffentliche Versammlung
unter ihnen halten konnten, so veranstaltete ich zu Gateshead
eine kleine im Kreise der Freunde und solcher, die sich zu ihnen
hielten, und dort wird seither eine Versammlung abgehalten im
Namen Jesu. Ms ich über den Marktplatz ging, erfaßte mich
die Kraft des Herrn, und ich ermahnte das Volk, an den Tag
des Herrn zu denken, der über sie kommen werde. Und nicht
lange daraus wurden alle jene Priester von Neweastel rmd ihre
Anhänger Vertrieben, bei der Rückkehr des Königs ..... Von
Heatshead gingen wir nach Durham; es war einer von London
dorthin gekommen, um eine Schule zu errichten, worin »Prediger
Christi'', wie sie sich ausdrückten, ausgebildet werden sollten. Jch
ging zu ihm, um mit ihm zu reden und ihm zu zeigen, daß das
Lehren von Griechisch und Latein und der sieben schönen Künste
nur ein Belehren des natürlichen Menschen sei und nicht das


% \picinclude{./130-139/p_s135.jpg} 
Erste Jahreöversammlung. Warnung an Cromwell usw. 135
Mittel, die Leute zu Predigern Christi zu machen. Die Sprach-
verschiedenheit komme von Babel, und den Griechen, deren Mutter-
sprache griechisch war, war daß Wort vom Kreuz Torheit, und
den Juden, deren Sprache hebräisch war, mar Christuß ein
Stein deß Anstoßeß (1. Cor. 1, 23). Die Römer, die lateinisch
redeten, verfolgten die Christen; und Pilatuß, der römische
Machthaber, schrieb in hebräischer, griechischer und lateinischer
Sprache eine Jnschrift über daß Kreuz Christi; daran, sagte ich,
könne man sehen, daß die Sprachen von Babel kommen, da die
Jnschrist über daß Kreuz in diesen Sprachen geschrieben war.
Johanneß, der daß Wort verkündete, welcheß im Anfang war,
sagt, daß daß Tier und die Hure Macht haben über die
Zungen und Sprachen, welche dem Wasser gleich seien (Offb. 17);
man könne also sehen, daß daß Tier und die Hure diese Macht
haben über die Sprachen, die von der Verwirrung zu Babel her-
rühren. Die Verfolger Christi haben sie dann höher gestellt alß
ihn, alß sie ihn kreuzigten; aber darnach ist er auferstanden, höher
alß alleß andere, er, der vor allen gewesen ist. ,,Gedenkst du
nun,« fragte ich den Mann, ,,Prediger Christi zu bilden, vermittelst
dieser verwirrten äußeren Sprachen, die auß Babel kommen und
dort gut geheißen und von den Verfolgern Christi gebraucht
wurden!« Gr mußte daß zum Teil zugeben; idarauf zeigten
wir ihm weiter, daß Christuß seine Prediger selber lehrte, ihnen
Gaben erteilte und sie hieß, den Herrn der Ernte zu bitten, daß
er Arbeiter sende. Und Petruß und Johanneß, die doch in Sachen
der Schulweißheit unwissend und ungelehrt waren, verkündeten
Christuß, daß Wort, daß am Anfang war, also auch vor Babel. *
Auch Pauluß hat daß Evangelium nicht durch irgend einen
Menschen empfangen, sondern durch Jesuß, welcher auch jetzt
derselbe ist, und so ist auch sein Evangelium unverändert.« —.—’ Der
Priester hörte auf unß und errichtete seine Schule nicht.
Wir zogen über Warwickshire, Northamptonshire und
Leicestershire, wo wir überall viele Freunde aussuchten, nach
Bedforshire, zum Hause John Crookß, wo eine allgemeine Jahreß-
versammlung für daß ganze Land abgehalten wurde; sie dauerte
drei Tage, und die Freunde strömten auß dem ganzen Land herzu,
so daß alle Herbergen und Wohnungen in der Umgegend über-
füllt waren. Und trotz einiger Störungen durch einige böse Leute,
die von ddr Wahrheit abgefallen waren, kam doch die Kraft deß


% \picinclude{./130-139/p_s136.jpg} 
136 Kapitel Ill.
Herrn über alle, so daß wir eine herrliche Versammlung hatten;
das ewige Evangelium wurde gepredigt, und viele nahmen ez
aus, und Leben und unsterblicheß Wesen ging auf in allen und
schien über allen .....
Jch suchte noch da und dort etliche Freunde aus und kam
vom Herrn geleitet nach London, (1658) .....
Jch war noch nicht lange dort, als ich hörte, daß ein Jesuit,
der mit einem Gesandten von Spanien hergekommen war, alle
Quäker aufgefordert hatte, zu einer Dißputation in das-3 Hauö des
Earl von Newport zu kommen. Die Freunde antworteten, daß
etliche kommen werden. Darauf ließ er uns sagen, er wünsche
mit 12 unserer gelehrtesten und weisesten Leuten zu reden; einige
Zeit darauf ließ er sagen, eö möchten nur 6 kommen, und schließlich
nur 3. Wir beeilten uns?-, so viel wir konnten, damit ez nicht am
Ende nach so vieler Prahlerei heiße, eß solle gar niemand kommen.
Alß wir hinkamen, hieß ich Nicolaö Bond und Edward Burrough
hinausgehen, um die Unterredung zu eröffnen; ich wollte eine
Zeitlang mich unten im Hofe aufhalten und dann nachkommen.
Jch riet ihnen, ihm die Frage zu stellen, ob die römische Kirche,
sowie sie jetzt sei, nicht von der wahren Kirche der ersten Zeiten,
von ihrem Leben und ihrer Lehre, ihrer Kraft und ihrem Geist
abgesallen sei. Diese Frage legten sie denn auch dem Jesuiten si,
vor. Er erwiderte, die römische Kirche sei jetzt noch in der Jung-
fräulichkeit und Reinheit der ersten Kirche. Da kam ich dazu.
Wir frugen ihn, ob der heilige Geist über sie wie über die Apostel
auögegossen worden sei? Er antwortete: ,,nein«! ,,Dann,« sagte
ich, ,,wenn nicht derselbe Geist über euch außgegossen worden ist
und dieselbe Kraft wie über die Apostel, so seid ihr vom Geist
und der Kraft der ersten Kirche abgesallen; weiter braucht dann
nicht viel beigefügt zu werden.« Dann fragte ich ihn, auf maß
für Schriftftellen sie sich beriesen bei Errichtung von Nonnen- und
Mönch?-klöstern und Abteien für alle ihre verschiedenen Orden?
und beim Beten mit Rosenkränzen und zu Bildern, und ihrem
Bekreuzen und ihren Verboten wegen allerlei Speisen und beim
Heiraten und bei ihrem Hinrichten um des Glaubens willen?
,,Wenn ihr,« sagte ich, ,,die Gebräuche der ersten Kirche habt, in
ihrer Reinheit und Jungfräulichkeit, so zeiget untz Schristworte,
welche beweisen, daß sie solches getan.« Wir hatten nämlich
vorher gegenseitig abgemacht, daß wir unsere Behauptungen aus-


% \picinclude{./130-139/p_s137.jpg} 
Erste Jahreeversamnsslung. Warnung an Cromwell usw. 137
der Schrift beweisen sollten. Er sprach nun von einem geschrie-
benen und einem ungeschriebenen Wort. Jch fragte ihn, maß er
daß ungeschriebene Wort nenne? Er sagte, daß geschriebene
Wort sei die Schrift, daß ungeschriebene daß mündliche Wort der
Apostel, also alle Überlieferungen, nach denen sie wandeln. Jch
hieß ihn, daß auß der Schrift beweisen. Er kam nun mit der
Stelle, wo der Apostel, 2. Thess. 2, 5, sagt: ,,Gedenket ihr nicht
daran, daß ich euch solcheß sagte, alß ich noch bei euch war?''
»Damit,'' sagte der Jesuit, »meint der Apostel die Klöster und
daß Hinrichten um deß Glaubenß willen und daß Beten mit
Rosenkränzen und zu Bildern und waß noch mehr der Gebräuche
der römischen Kirche sind. Daß ist gemeint mit dem ,,ungeschrie-
benen Wort'', daß der Apostel damalß gesagt und daß sich seither
durch Überlieferung biß auf unsere Zeit erhalten hat.'' Ich hieß
ihn nun, dieseß Wort noch einmal lesen, damit er sehe, wie er eß
verdreht hatte. Denn daß, wovon er dort den Thessalonichern
sagt, ,,daß er ihnen zuvor gesagt'', ist nicht ein »ungeschriebeneß
Wort'', sondern eineß, daß geschrieben ist, nämlich daß »der Mensch
der Sünde, daß Kind deß Verderbenß geoffenbart werde, ehe der Tag
Ehristi kommen werde'' (2. Thess. 2, 3). Er redete also durchauß
hy nicht von den Gebräurhen der römischen Kirche. Ähnlich redete
  der Apostel im dritten Kapitel dieseß Briefeß, von »etlichen argen
Menschen, Vorwitzigen, die nichtß treiben''; um deretwillen hatte
der Apostel mündlich gesagt: »wer nicht arbeiten will, soll auch nicht
essen''; an daß erinnerte er sie nun schriftlich. Diese Schriftstelle
bewieß also nichtß für ihre erfundene Überlieferungen; und eine andere s
Stelle konnte er nicht bringen. Jch sagte ihm darum: ,,dieß ist
eine neue Verirrung eurer Kirche in Überlieferungen und Erfin-
dungen, wie die Apostel und die Heiligen der ersten Kirche sie
niemalß kannten.''
Er ging nun zum Altarsakrament über; er fing beim Pass alamm
und den Schaubroten an und kam zuletzt auf die Worte Jesu:
»Dieß ist mein Leib'', und auf daß, waß der Apostel darüber an
die Corinther schrieb. Er verstand die Stelle so, daß, nachdem der
Priester Brot und Wein geweiht habe, sei daßselbe göttlich und
un-vergänglich, und wer eß genieße, genieße Christuß selber. Jch
folgte ihm bei seiner Aufzählung der Bibelstellen, biß er zu den
Worten Christi und der Apostel kam. Da zeigte ich ihm, wie
derselbe Apostel den Corinthern auch nach ihrem Genuß von Brot und


% \picinclude{./130-139/p_s138.jpg} 
138 Kapitel Ill.
Wem ltsagt habe, sie seien Verworfene, wenn Ehristuö nicht in
ihneniti; wenn aber Brot und Wein, die sie genossen, Ehristuö
ielber Wären, so müßte er ja nun in ihnen sein. übrigen?-, wenn
Bjwt und Wein Christi Leib und Blut wären, wie könnte er dann
semtzts Leib im Himmel haben? Und zu dem haben die Jünger
Christi Leib essen und sein Blut trinken müssen zu seinem Ge-
dlikhkilit sowohl beim Abendmahl als auch nachher, biz daß er
ksms WW ja klar beweise, daß daz Brot und der Wein nicht
iem wirklicher Leib war, denn wenn sie seinen wirklichen Leib ge-
tzfjfsen hätten, so wäre er ja schon gegenwärtig gewesen, und man
hatte es nicht zu seinem Gedächtnis- zu tun brauchen .....
W Vaz die Worte Christi betresse: dietz ist mein Leib, so nenne
lich Christuß auch selbst einen Weinstock (Joh. 1,5) und eine Tür
lJ1gh.1O); und die Schrift nennt ihn einen Felß; ob er darum
i Tuizefiich ein Weinstock, eine Tür, ein Fels sei? »Oh«« sagte der
itesutischdiese Worte muß man auSlegen!« ,,So muß man auch
die Worte: ,,dieS ist mein Leib«, aus-legen«, antwortete ich.
Nachdem ich ihm so den Mund gestopft, machte ich ihm fol-
gmhtsl Vorschlag: ,,Wegen deiner Behauptung, daß Brot und
WM! göttlich und unoergänglich und leibhaftig Christus- seien und
daß leder, der sie genieße, Christus genieße, laßt eine Zusammen-
khsnst veranstalten zwischen einigen von euch, die der Papst und
die Kali-inäle bestimmen sollen, und einigen von uns, und laß eine
Flasche Wein und einen Laib Brot bringen und beide in zwei
Teileteilen und den einen dieser Teile weihen. Dann oerwahret
ionsohi den geweihten alß den ungeweihten Teil an einem sichern
OU, sind laßt sie gut bewachen; und macht den Versuch, ob das
geweshte Brot und der geweihte Wein nicht gerade so schnell
schlecht werden, wie daß; ungeweihte Brot und der ungeweihte
Wem- . . . und so wird die Wahrheit über diese Punkte offen-
bar werden. Wenn daß geweihte Brot und der geweihte Wein
isch nicht verändern, sondern schmackhaft und gut bleiben, so
werden dadurch viele für eure Kirche gewonnen werden; ver-
andem sie sich aber, so müßt ihr nachgeben, euren Jrrtum fahren
Essen Und kein Blut mehr darum vergießen. E6 ist schon viel
Nzuiharum vergossen worden, zum Beispiel zur Zeit der Königin
Stßlsss'' Hierauf erwiderte der Jesuit: ,,Nehmt ein Stück neuen
dan? und schneidet ihn in zwei Hälften, und machet zwei Röcke
Us- und zieht den einen Rock dem König David an und den


% \picinclude{./130-139/p_s139.jpg} 
Erste Jahresversammlung. Warnung an Cromwell usw. 139
andern einem Bettler; beide Röcke werden sich gleichermaßen ab-
tragen«. »Jft das deine Antwort«, stragte ich; ,,ja«, antwortete
er; ,,dann«, entgegnete ich, ,,werden alle Anwesenden überzeugt
sein, das; euer geweihter Wein und euer geweihtes Brot nicht
Christus ist. Habet ihr den Leuten solange oorgeschwatzt, der
geweihte Wein und das geweihte Brot sei Christus, und nun
sagst du, sie verbrauchen sich so gut wie die andern? Jch sage
dir: Christus ist derselbe gestern und heute und verfällt nicht;
sondern er ist der Heiligen hinimliche Speise jetzt und zu allen
Zeiten«. Hieraus erwiderte er nichts mehr, sondern hörte gerne
auf, denn die Anwesenden sahen, daß er im Jrrtum war und sich
nicht verteidigen konnte. Ich fragte ihn weiter, warum seine
Kirche die Leute um des Glaubens willen töte und verfolge?
Gr erwiderte: ,,nicht die Kirche tue das, sondern die Obrigkeit«.
Jch fragte ihn, ob denn diese Obrigkeit sich nicht auch zu den
Gläubigen und Christen zähle? Gr bejahte es. ,,Nun also«,
sagte ich, »stnd sie denn dann nicht Glieder eurer Kirche?« Er
antwortete: »ja«. Daraus überließ ich es den Anwesenden, selber
zu urteilen, ob die römische Kirche nicht die Leute um des Glaubens
willen verfolge und töte. Hiermit trennten wir uns. Seine
Spitzfmdigkeiten erklärten sich durch seine Dummheit.
Gs lag vieles auf mir während der Zeit, da ich in London
war, denn es war eine Zeit großer Not. Es trieb mich, an
Oliver Cromwell zu schreiben, und ihm die Bedrängnis der
Freunde, sowohl in England als auch in Jrland vorzustellen.
Das Gerücht jverbreitete sich damals, man wolle C-romwell zum
König machen. Da trieb es mich, zu ihm zu gehen und ihn da-
vor zu warnen, wie auch vor mancher anderen Gefahr, die seinen
und seiner Nachkommen Untergang herbeiführen würden, wenn
er sie nicht meide. Gr schien alles, was ich ihm sagte, gut aus-
zunehmen, und dankte mir dafür; dennoch trieb es mich, ihm nach-
her noch ausführlicher darüber zu schreiben .....
Um diese Zeit erkrankte Lady Elaypole (die -Lieblings-tochter
Oliver Cromwells), und war sehr niedergeschlagen, und niemand
konnte sie trösten; als ich davon hörte, trieb es mich, ihr folgendes
zu schreiben: —
,,Freundin!
Sei stille und ruhig in deinem Innern, und frei von eigenem
Denken; dann wirst du das Walten Gottes erfahren, wie es

