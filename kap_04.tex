%%%%%%%%%%%%%%%%%%% Kapitel 4. %%%%%%%%%%%%%%%%%%%%%%%%%%%%%%
\chapter[Kampf gegen die Ranter]{Kampf gegen die Ranter}

\begin{center}
\textbf{Erlebnisse im Gefängnis zu Derby. Ein \zitat{Wehe} 
über die Stadt
Lichfield\ort{Lichfield}. Erste Missionsgenossen. Antikirchliche 
Agitation und Kampf gegen die Ranter\index{Ranter}.}
\end{center}

\section{Begegnung mit einem Soldaten}

Während ich noch im Gefängnis war, kam ein Soldat zu
mir und erzählte mir, wie er im Turmhause gewesen sei und dem
Priester zugehört habe, und wie dann auf einmal eine grose Angst
über ihn gekommen sei und die Stimme des Herrn also zu ihm
geschehen sei: \zitat{Weist du nicht, das mein Diener im Gefängnis
ist? Zu ihm gehe und frage ihn um Rat}. Ich redete mit ihm
wie es sein gegenwärtiger Zustand erheischte, und sein Verständnis
% \picinclude{./030-039/p_s036.jpg} 
wurde geöffnet. Ich sagte ihm, das der, welcher ihm seine Sünden
aufdecke und ihn um ihretwillen ängstige, ihm auch die Rettung
zeigen werde; denn der dem Menschen die 
Sünden\index{Sünden} aufdeckt, ist
derselbe, der sie auch hinwegnimmt. Während ich mit ihm redete,
offenbarte sich ihm der Herr, so das er anfing, die Wahrheit des
Herrn und Gottes Gnade zu erkennen; er fing an, unerschrocken
in seinem Regiment unter den Soldaten\index{Soldaten} von 
der Wahrheit zu
reden; denn die Schrift wurde ihm mehr und mehr offenbar, und
er ging soweit zu sagen: sein Oberst sei blind wie Nebukadnezar,
das er den Diener des Herrn ins Gefängnis werfe. Von da an
hegte sein Oberst einen Groll gegen ihn. Als im darauffolgenden
Jahre in der Schlacht von Worcester\index{Schlacht von Worcester} 
die beiden Armeen 
nebeneinander lagen, kamen zwei aus der Armee des Könige und
forderten, das zwei ans der Armee des Parlaments sich mit ihnen
schlagen sollten; da wählte der Oberst ihn und noch einen, um
der Forderung Folge zu leisten. Als sein Kamerad im Kampfe
gefallen war, trieb er seine beiden Gegner zur Stadt hinaus, ohne
einen Schuss auf sie abzufeuern; dies erzählte er mir nach seiner
Rückkehr mit eigenem Munde. Nach Beendigung der Schlacht
sah er die Betrügerei und Heuchelei der Offiziere ein, und im
Gedanken daran, wie wunderbar der Herr ihn bewahrt hatte und
was es eigentlich um den Krieg sei, legte er die Waffen 
nieder.\index{Pazifismus}

\section{Fox lehnt es ab Hauptmann zu werden}

Die Zeit meiner Gefangenschaft war nun fast zu Ende und
da viel neue Soldaten ausgehoben wurden, so wollten mich die
Kommissäre zu ihrem Hauptmann machen, und die Soldaten
erklärten, sie wollten keinen andern als mich haben. Der 
Kerkermeister erhielt den Befehl, mich vor die Soldaten und ihre 
Vorgesetzten auf den Marktplatz zu führen; dort boten sie mir dieses
Ehrenamt, wie sie es nannten, an und fragten mich, ob ich nicht
wolle die Waffen ergreifen für den Commonwealth gegen Karl
Stuart\footnote{1651\jahr{1651} Schlacht von Worcester 
zwischen Cronwell\person{Cronwell} 
(Commonwealth\index{Commonwealth}) und Karl II.\person{Karl II.}}. 
Ich erwiderte ihnen, ich wisse wohl, woher aller
Krieg komme: aus der Begierde, wie schon Jakobus lehre
(Jak. 4\bibel{Jak. 04@Jak. 4}); ich aber stehe in jener 
Kraft und jenem Leben,\index{Pazifismus}
die von vornherein allen Krieg ausschließen. Sie wollten mich
überreden, ihr Anerbieten anzunehmen; sie meinten, ich weigere
mich nur aus Bescheidenheit. Aber ich erklärte ihnen, ich sei in
den Bund des Friedens eingetreten, welcher bestanden, ehe es
% \picinclude{./030-039/p_s037.jpg} 
Krieg und Zank gab. Sie sagten, sie bieten es mir in Liebe und
Zuneigung an wegen meiner Tugend, und ähnliche Schmeicheleien
mehr. Aber ich sagte ihnen, wenn solches ihre Liebe sei, so trete
ich sie mit Füßen. Da wurden sie zornig und sagten: \zitat{Nimm
ihn hinweg, Kerkermeister, und wirf ihn in den untersten Kerker
zu den Schelmen und Verbrechern.} Ich wurde weggeführt und
an einen wüsten, stinkenden Ort\footnote{Die Zustände der 
Gefängnisse und \textit{Korrektionshäuser} im 17. Jahrh.
waren überaus traurig. Überall herrschte große Unreinlichkeit; 
die Verwaltung war der Willkür des Gefängnisvorstehers 
anheim gegeben, der nicht besoldet war, sondern von den 
Gefangenen bezahlt wurde, die die Kosten ihres Aufenthaltes 
selbst tragen mussten. Vgl. Aschrott, Englisches Gefängniswesen.
} gebracht, wo kein Bett war,
mit 30 Verbrechern, wo ich beinahe ein halbes Jahr gefangen
war, außer, wenn sie mich dann und wann ein wenig in den
Garten ließen, weil sie sicher waren, das ich nicht davon laufe.
Es hatte damals, als man mich in diesen Kerker gebracht hatte,
geheißen, ich werde wohl nicht mehr heraus kommen. Aber ich
glaubte an Gott\index{Zuversicht} und das ich zu seiner 
Zeit daraus befreit werde.
Denn der Herr hatte es mir vorausgesagt, das ich nicht bald
von diesem Ort wegkomme, da ich dort eine Aufgabe für ihn zu
erfüllen habe.



Als es bekannt wurde, das ich im Kerker von Derby sei, kamen
meine Angehörigen\index{Angehörige}, um mich wieder zu besuchen; 
denn sie betrachteten
es als eine große Schande für sie, das ich um der Religion
willen gefangen war; und etliche hielten mich für verrückt, weil
ich für die Reinheit, Gerechtigkeit 
und Vollkommenheit\index{Vollkommenheit} eintrat.
Unter denen, die zu mir kamen, war einer aus Nottingham,
ein Soldat, der früher Baptist gewesen war. Im Laufe des 
Gesprächs sagte er zu mir: \zitat{Dein Glaube gründet sich auf einen
Mann, der in Jerusalem gestorben sein soll; solches ist aber nie
geschehen}\index{Atheismus}\index{Sinnbildlichkeit}. 
Es betrübte mich sehr, ihn so reden zu hören, und
ich sagte: \zitat{Wie! hat nicht Christus gelitten vor den Toren 
Jerusalems durch die Juden, die \textit{Frommen}, die Hohenpriester und
durch Pilatus?} Aber er leugnete, das Christus je äußerlich
gelitten habe\index{Leiden!Christi}. Ich fragte ihn, 
ob denn keine Hohenpriester, keine
Juden, kein Pilatus äußerlich dort gewesen sei? und als er das
nicht bestreiten konnte, sagte ich: \zitat{So gewiss ein Hohepriester,
ein Pilatus und Juden äußerlich dort gewesen sind, so gewiss ist
Christus äußerlich verfolgt worden von ihnen und hat durch sie
% \picinclude{./030-039/p_s038.jpg} 
gelitten}. Die Reden dieses Menschen veranlassten eine 
Verleumdung\index{Verleumdung} gegen uns, als ob die Quäker 
bestritten, das Christus
gelitten habe und in Jerusalem gestorben sei. Es war dies ganz
falsch; nie war der leiseste Gedanke davon in unsern Herzen 
gewesen; es war eine bloße Verleumdung, die uns traf, und die
aus dem Gerede dieses Menschen entstanden war. Derselbe
Mensch behauptete auch, niemals habe irgend ein Apostel oder
Prophet, oder Heiliger oder Mann Gottes äußerlich gelitten; alle
ihre Leiden seien innerlich gewesen; aber ich bewies ihm an 
Bespielen, wie viele unter ihnen gelitten und durch wen sie gelitten;
und so widerlegte die Kraft des Herrn seine verkehrten Ansichten.


Eine andere Sorte kam zu mir, die behaupteten, sie könnten
Geister unterscheiden. Ich fragte sie, welches der erste Schritt
zum Frieden sei? und in was der Mensch seine Rettung suchen
müsse? Sie fuhren auf und sagten in ihrem Hochmut, ich sei
verrückt; und solche wollten Geister unterscheiden können und
kannten nicht einmal ihren eigenen Geist!

\section{Kampf gegen Todesstrafe}

Während dieser Zeit meiner Gefangenschaft geriet ich in
große Bekümmernis über das Vorgehen der Richter und Beamten
in ihren Gerichtshöfen. Es trieb mich, an die Richter zu schreiben,
darum das sie das Todesurteil fällten wegen allerlei unwichtiger
Vergehen, in Geldsachen oder das 
Vieh betreffend\index{Kapitalverbrechen}. Ich musste
ihnen zeigen, wie solches von jeher dem Gesetz Gottes zuwider
war; ich war deswegen in meinem Geiste sehr betrübt bis in den
Tod, aber da ich mich unter den Willen Gottes stellte, so 
erwachte ein himmlisches Sehnen nach dem Herrn in meinem Herzen,
ich sah den Himmel offen und freute mich und gab Gott die
Ehre [...]\index{Trost}.


In diesem Zustande trieb es mich, an die Richter zu schreiben,
wie schädlich es für die Gefangenen sei, so lange im Kerker zu
sein, wie sie da schlechtes von einander lernten, wenn sie 
miteinander über ihre bösen Taten reden. Darum sollten die Urteile
rasch gesprochen werden. Denn ich war ein gottseliger Jüngling
und wandelte in der Furcht des Herrn; es betrübte mich, ihre
schlechten Reden zu hören, ich musste ihnen oft Vorstellungen über
ihre bösen Worte machen und über ihr hässliches Betragen 
untereinander. Die Leute wunderten sich, wie ich bewahrt und behütet
blieb; denn nie konnten sie mir ein Wort oder eine Tat 
nachweisen, die sie hätten zu meinen Ungunsten auslegen können
% \picinclude{./030-039/p_s039.jpg} 
während der ganzen Zeit, die ich dort war; denn die unendliche
Kraft des Herrn hielt mich aufrecht und bewahrte mich während
der ganzen Zeit; ihm sei Lob und Ehre immerdar.


Es war eine junge Person mit mir im Gefängnis, die ihrem
Herrn Geld gestohlen hatte. Als sie zum Tode verurteilt werden
sollte, schrieb ich an den Richter und ans Schwurgericht und
stellte ihnen vor, wie es immer gegen das Gesetz Gottes gewesen
sei, die Leute wegen Diebstahls zum Tode zu verurteilen, und
bat um Gnade\index{Gnadenersuch}. Sie wurde aber doch 
verurteilt, und man grub
ihr ein Grab und führte sie zur Hinrichtung. Da schrieb ich noch
einmal ein paar Worte und warnte alle, sich vor Raubgier und
Habsucht zu hüten, da sie von Gott wegführe, und ermahnte alle
den Herrn zu fürchten, allen irdischen Begierden zu entsagen und
die Zeit zu nützen, dieweil sie da ist; solches hieß ich sie unter
dem Galgen vorlesen. Und obgleich sie sie schon auf der Leiter
hatten, bereit gehenkt zu werden, mit einem Tuch über den Augen,
so wurde sie nun nicht hingerichtet, sondern sie führten sie wieder
zurück ins Gefängnis, und im Gefängnis kam sie nachher dazu,
Gottes ewige Wahrheit zu erkennen.



Es war noch ein anderer Gefangener mit mir, ein schlechter,
gottloser Mensch, ein berüchtigter Schwarzkünstler 
und Zauberer\index{Zauberer}.
Er drohte, was er alles zu mir sagen und mir tun wolle, aber
er hatte keine Macht, den Mund gegen mich aufzutun. Einmal
gerieten der Kerkermeister und er aneinander und er drohte, er
wolle den Teufel rufen und das Haus niederreißen, so das der
Kerkermeister Angst bekam. Da trieb mich der Herr hinzugehen
und ihm Einhalt zu gebieten und zu sagen: \zitat{Komm, las sehen
was du kannst, tue dein Äußerstes}. Ich sagte ihm, der Teufel
sei schon in ihm selber bei uns, die Kraft des Herrn binde ihn
aber. Da schlich er sich davon.

\section{Zwangsmusterung von Fox}

Als nun die Zeit der Schlacht von Worcester kam, sandte
der Richter Bennet Konstabler, um mich zu zwingen, Soldat zu
werden, da er gesehen hatte, das ich kein Kommando übernehmen
würde. Ich sagte ihnen, ich sei ganz gegen allen 
äußeren Krieg\index{Krieg}.
Sie kamen wieder, um mir Werbegeld zu geben, aber ich nahm
es nicht. Daraus wurde ich vor den 
Wachtmeister Holes\person{Wachtmeister Holes} gebracht,
der mich eine Weile behielt und dann wieder zurückschickte. Nach
einiger Zeit wurde ich wieder herauf geholt und vor den Kommissar
gebracht, welcher erklärte, ich müsse als Soldat gehen, aber ich
% \picinclude{./040-049/p_s040.jpg} 
sagte ihnen, ich sei hierfür tot. Sie sagten, ich sei ja am Leben.
Ich sagte ihnen, wo Neid und Zank sei, da sei Verderben 
(Jak. 3:16\bibel{Jak. 03:16@Jak. 3:16}).
Sie boten mir zweimal Geld an, aber ich wollte nichts 
annehmen; darauf wurden sie böse und verurteilten mich zum 
Gefängnis [...].

\par

Ich war tief betrübt und bearbeitet in meinem Geist während
meiner Gefangenschaft wegen der Schlechtigkeit, die in der Stadt
herrschte; denn obgleich etliche gewonnen waren, so war doch die
Mehrzahl sehr verhärtet. Ich sah, wie sich das Ausgießen der
Liebe Gottes von ihnen wegwandte. Ich trauerte über sie, und
es kam über mich, folgende Klage über sie zu verbreiten:

\brief{O Derby!}{\index{Kritik!an frömmelei}
    O Derby\ort{Derby}! Wie die Wasser abfließen, wenn die Schleusen
    sich öffnen, also fließet die Liebe Gottes von dir ab, o Derby.
    Darum siehe zu, wo du stehest und auf welchem Grund du bist,
    ehe du gänzlich verlassen wirst. Der Herr hat mich zweimal 
    gerufen, ehe ich zu dir kam, um gegen deine Eitelkeit und 
    Schlechtigkeit aufzutreten und alle zu ermahnen, auf den Herrn und
    nicht auf Menschen zu sehen. \zitat{Wehe der prächtigen Krone der
    Trunkenen! der welken Blume ihrer Herrlichkeit} 
    (Jes. 28:1\bibel{Jes. 28:01@Jes. 28:1}).
    Wehe denen, die mit Worten ihren Glauben zur Schau tragen und
    doch hochmütig und hochfahrend sind und Unterdrückung und Hass
    üben. O Derby! Deine Frömmigkeit und dein Predigen stinken
    gen Himmel! Ihr feiert einen Sabbat in Worten und versammelt
    euch, um euch schön zu kleiden, ihr fröhnet der Eitelkeit. Die
    Weiber gehen mit aufgerichtetem Halse und geschminkten\index{Schminke} 
    Gesichtern, wie es die alten Propheten verurteilt haben 
    (Jes. 03:16\bibel{Jes. 03:16@Jes. 3:16}).
    Eure Versammlungen sind dem Herrn ein Greuel; ihr erhebet
    die Eitelkeit und beuget euch davor; das Laster gedeiht und das
    Böse wird geehrt; das Schlechte wird von den Schlechten 
    geduldet und doch bekennen sie alle Christus mit Worten. O über
    die Schlechtigkeit unter euch! Es bricht mir fast das Herz, zu
    sehen, wie Gott unter euch verachtet ist, o Derby!
}

Als ich gesehen, wie Gottes Liebe sich von diesem Orte 
abwandte, wusste ich, das meine Gefangenschaft hier nun nicht mehr
lange andauern werde, aber ich sah, das, wenn der Herr mich
frei machen werde, so werde es sein, wie wenn man einen
Löwen aus seiner Höhle auf die wilden Tiere des Waldes 
ablässt. Denn alle \textit{Frommen} hatten eine tierische 
Gesinnung, die
der Sünde huldigte, so lange sie lebten. Sie waren alle dem
% \picinclude{./040-049/p_s041.jpg} 
Geist und dem Leben feind, der in der Schrift gegeben ist und
den sie in Worten bekannten. So geschah es, wie man hernach
sehen wird.

\section{Fox wird aus dem Gefängnis entlassen}

Es stand ein Gericht über der Stadt, und den Behörden war
es unbehaglich meinetwegen; aber sie wussten nicht, was sie mit
mir machen sollten. Einmal wollten sie mich vors Parlament
schicken, ein andermal mich nach Irland verbannen. Zuerst
nannten sie mich einen Betrüger und Verführer und 
Gotteslästerer; dann, als Gott seine Strafe über sie 
schickte, sagten sie,
ich sei ein ehrlicher, tugendhafter Mensch. Aber ob sie eine gute
oder schlechte Meinung von mir hatten, war mir gleichgültig;
denn weder richtete mich das eine auf, noch warf mich das andere
nieder, dem Herrn sei Lob. Schließlich mussten sie mich frei
lassen, zu Anfang des Winters 1651\jahr{1651}, nachdem 
ich fast ein Jahr
in Derby gefangen gewesen war, sechs Monate im Zuchthaus
und die übrigen im Kerker.

Als ich nun wieder meine Freiheit hatte, fuhr ich fort wie
zuvor in der Arbeit für den Herrn und zog im Lande umher,
zuerst in der Gegend meiner Heimat, 
Leicestershire\ort{Leicestershire}; ich hielt 
unterwegs Versammlungen, und des Herrn Geist und Kraft war mit
nur [...].

\section{Die Märtyrer von Lichfield}

Einmal als ich mit einigen Freunden unterwegs war und
eine Turmhauzspitze erblickte, ging es mir durch Mark und Bein;
ich fragte, was das für eine Ortschaft sei? es hieß: 
Lichfield\ort{Lichfield}.
Alsobald erging das Wort des Herrn an mich, das ich dorthin
gehen müsse. Als wir bei dem Hause angelangt waren, in das
wir gehen wollten, bat ich die Freunde, die mit mir waren, 
hineinzugehen; ich sagte ihnen aber nicht, wohin ich zu gehen hatte.
Sobald sie im Hause waren, entfernte ich mich und lief über
Hecken und Gräben, bis ich eine Meile weit von Lichsield 
entfernt war; da waren auf einem weiten Felde Schäfer, die ihre
Schafe hüteten. Hier befahl mir der Herr, meine Schuhe 
auszuziehen; ich zögerte, denn es war Winter; doch das Wort des
Herrn war wie Feuer in mir. So zog ich denn meine Schuhe
aus und lies sie bei den Schäfern, und die armen Schäfer zitterten
und waren ganz bestürzt. Darauf lief ich wieder eine Meile,
und sobald ich wieder in der Stadt war, erging das Wort des
Herrn an mich: \zitat{Rufe: wehe der blutigen Stadt Lichfield!} Ich
ging also die Straße auf und ab und rief: \zitat{Wehe der blutigen
% \picinclude{./040-049/p_s042.jpg}
Stadt Lichfield!} Da es Markttag war, ging ich auf den 
Marktplatz, lief auf demselben umher und rief von Zeit 
zu Zeit: \zitat{Wehe der blutigen Stadt Lichfield!} 
Und niemand tat mir etwas.
Während ich rufend durch die Straßen ging, schien es mir, Als
ob ein Bach von Blut durch die Straße fließe, und der 
Marktplatz kam mir vor wie ein Teich von Blut. Als ich mich der
mir aufgetragenen Verkündigung\index{Verkündigung} 
entledigt hatte, verließ ich im
Frieden die Stadt. 

Ich kehrte zu den Hirten zurück, gab ihnen
Geld und erhielt meine Schuhe von ihnen zurück. Aber das
Feuer des Herrn war so in meinen Füßen und in meinem ganzen
Körper, das mir nichts daran lag, meine Schuhe überhaupt wieder
anzuziehen; und ich wusste nicht recht, ob ich es tun sollte oder
nicht, bis ich die Erlaubnis dazu vom Herrn fühlte; nachdem ich
meine Füße gewaschen, zog ich meine Schuhe wieder an. Darauf
verfiel ich in tiefes Nachsinnen, warum und aus welchem Grunde
ich wohl gesandt worden sei, gegen diese Stadt zu reden und sie
die \zitat{blutige Stadt} zu nennen; denn obwohl eine Zeit lang das
Parlament und eine Zeit lang der König die Herrschaft über diesen
Kirchenspengel gehabt hatte und viel Blut in der Stadt vergossen
worden war während des Krieges zwischen beiden, so war es
doch nicht schlimmer gewesen als an vielen anderen Orten auch.
Nach und nach aber fiel es mir ein, wie zur Zeit des Kaisers
Diocletian\person{Kaisers Diocletian} tausend 
Christen in Lichsield gemartert worden waren;
darum hatte ich ohne Schuhe durch den Bach ihres Blutes gehen
müssen, damit die Erinnerung an 
das Blut jener Märtyrer\index{Märtyrer}, das
vor mehr als tausend Jahren vergossen worden und in ihren
Straßen erkaltet war, wach werde. Die Nachwirkung jenes Blutes
war über mich gekommen, so das ich dem Herrn hatte gehorchen
müssen. Man weiß aus alten Überlieferungen, wie viel 
christliche Briten dort gelitten haben. Ich könnte noch viel berichten
über alles, was sich mir offenbarte über das hier während der
zehn Verfolgungen und später vergossene Märtyrerblut, aber ich
überlasse es dem Herrn und seinem Buch, aus welchem alles
gerichtet werden wird; denn sein Buch und fein Geist sind sichere
Überliefere.

Darauf zog ich im Lande umher und hatte vielerorts 
Versammlungen unter den freundlich Gesinnten. Aber meine 
Angehörigen\index{Angehörige} waren böse über mich. 
Nach einiger Zeit kehrte ich nach
Nottinghamshire\ort{Nottinghamshire} zurück und ging 
dann nach Derbshire\ort{Derbshire}, um dort
% \picinclude{./040-049/p_s043.jpg} 
die freundlich Gesinnten aufzusuchen. In 
Yorkshire\ort{Yorkshire} und an einigen
andern Orten predigte ich Buße: darauf kam ich nach Balby,
wo Richard 
Farnsworth\person{Farnsworth, Richard}\footnote{Richard 
Farnsworth, William Dewsbury und James Naylor waren
die ersten bedeutenden Missionsprediger der Quaker. 
(Näheres s. Weingarten, Revolutionskirchen 
Englands. S.~218ff.) James Naylor ist in der Geschichte
berüchtigt geworden durch seinen Messiaseinzug in Bristol, 
dem Höhepunkt der fast zum Wahnsinn gesteigerten 
Schwärmerei des älteren Quakertums.} und einige 
andere gewonnen wurden.
So reiste ich im Lande umher, Buße\index{Buße!predigend} 
predigend und das Wort
des Herrn verkündigend, bis ich in die Gegend von 
Wakefield\ort{Wakefield}
kam, wo James Naylor\person{Naylor, James} lebte; 
er und Thomas Goodyear\person{Goodyear, Thomas}
kamen zu mir; beide wurden gewonnen und nahmen die Wahrheit
auf. Auch William Demsbury\person{Demsbury, William} 
und seine Frau und viele andere
kamen zu mir, wurden gewonnen und nahmen die Wahrheit auf.
Von dort begab ich mich nach Hauptmann 
Pursloes\person{Hauptmann Pursloe} Haus in
die Nähe von Selby\ort{Selby}, und besuchte John 
Leek\person{Leek, John}, der ins Gefängnis
zu mir gekommen war, und er wurde gewonnen. Ich besaß ein
Pferd, musste mich aber leider davon trennen, da ich nicht wusste,
was damit anfangen, weil mich der Herr trieb in manches 
angesehene Haus zu gehen, um die Leute zu ermahnen, sich zum
Herrn zu bekehren. Unter anderm trieb mich der Herr auch ins
Turmhaus von Beverly\ort{Beverly} zu gehen, das damals 
eine Stätte besonderer Frömmigkeit war; da ich vom 
Regen ganz durchnässt war,
ging ich zuerst nach der Herberge. In der Türe kam ein junges
Weib auf mich zu und sagte: \zitat{Wie! seid ihr es? Kommt herein},
wie wenn sie mich schon gekannt hätte; denn die Kraft des Herrn
hatte ihr Herz vorbereitet. Ich nahm etwas zu mir und ging
ins Bett. Am Morgen zog ich meine noch nassen Kleider an
und bezahlte meine Zeche und begab mich ins Turmhaus, wo
einer predigte. Als er geendet, trieb mich die mächtige Kraft Gottes,
zu ihnen zu reden, und ich wies sie auf Christus, ihren Lehrer, hin.
Die Kraft des Herrn war so mächtig, das alle von großer Furcht
ergriffen wurden. Der Bürgermeister kam und sprach ein paar
Worte mit mir, aber niemand hatte Macht, mir etwas zu tun.
Ich verließ die Stadt und ging am Nachmittag in ein anderes
Turmhaus, etwa zwei Meilen weit entfernt. Als der Priester
geendet, trieb es mich, eingehend zu ihm und den Leuten über
den Weg des Lebens und der Wahrheit und den Grund der 
Erwählung und Verdammung zu reden. Der Priester sagte, er sei
% \picinclude{./040-049/p_s044.jpg} 
zu kindlich, um mit mir zu disputieren; ich erklärte ihm, ich sei
nicht gekommen, um zu disputieren, sondern um das Wort des
Lebens und der Wahrheit zu verkünden, und damit sie alle den
Samen kennen lernen möchten, den Gott allen verheißen, den
Männern wie den Frauen. Die Leute waren hier sehr empfänglich
und wünschten, das ich wiederkäme an einem Wochentag, um
ihnen zu predigen, aber ich wies sie an ihren Lehrer Jesus
Christus und verließ sie. Am folgenden Tage ging ich nach
Cranstick zu Hauptmann Pursloe, der mich zu Richter 
Hotham\person{Richter Hotham}
begleitete. 

Dieser war ein gottseliger Mann, der auch Gottes
Wirken schon in seinem Herzen verspürt hatte. Nachdem wir eine
Zeitlang über göttliche Dinge geredet hatten, nahm er mich mit
in sein Zimmer und bekannte mir, das ihm diese Ansichten
schon seit zehn Jahren vertraut seien, und wie er sich freue, das
der Herr sie nun auch verkünden lasse unter den Leuten. 
Nachher kam noch ein Priester zu ihm, mit dem ich auch über die
Wahrheit redete. Aber der war bald zum Schweigen gebracht,
denn er war ein bloser Phantast, der sich das, wovon er redete,
innerlich nicht angeeignet hatte.


Während ich da war, kam eine angesehene Frau aus Beverly,
um Richter Hotham in irgend einer wichtigen Angelegenheit zu
sprechen. In Laufe des Gespräches erzählte sie ihm, das am
vergangenen Sabbat, wie sie diesen Tag nannten, ein Engel oder
ein Geist in die Kirche von Beverly gekommen sei und herrliche
Dinge von Gott geredet habe zur Verwunderung aller Anwesenden,
und als er geendet habe, sei er verschwunden; sie wisse nicht, woher
er gekommen, noch wohin er gegangen sei, alle haben sich 
gewundert, die Priester, die \textit{Frommen} 
und die Behörden der Stadt.
Richter Hotham erzählte mir das nachher wieder, woraus ich ihm
mitteilte, das ich es gewesen, der an jenem Tage im Turmhaus
gewesen und die Wahrheit verkündet hatte [...].

Am Nachmittag ging ich in ein anderes Turmhauö, wo ein
großer, angesehener Priester, ein Doktor, wie sie ihn nannten,
redete, einer von denen, die Richter Hotham wollte kommen lassen.
Ich ging hin und wartete, biS der Priester geendet hatte. Die
Worte, die er alö Text genommen hatte, waren: ,,Wohlan alle,
die ihr dursiig seid, kommet her zum Wasser, und die ihr nicht
Geld habt, kommt her, kauset und esset, kommt her und kauset
ohne Geld, beide:-3 Wein und Milch (Jes. 55, 1).*- Und der Herr


% \picinclude{./040-049/p_s045.jpg} 
Erlebnisse im Gefängnis zu Derby usw. 45
trieb mich zu sagen: ,,Komm herunter, du Verführer; heisest du
die Leute umsonst kommen und umsonst vom Wasser dez Lebenz
nehmen, und nimmst jährlich dreihundert Pfund dafür, das du die
Schrift oerkündest? Errötest du nicht vor Scham? Tat der
Prophet Jesaias und Christus, die diese Worte umsonst geredet
und mitgeteilt hatten, auch also? Sagte nicht Christus- zu seinen
Jüngern, alö er sie auösandte zu predigen: umsonst habt ihr ez
empfangen, umsonst gebet etz auch?« Der Priester machte sich
ganz bestürzt davon; nachdem er seine Herde verlassen hatte,
hatte ich so oiel Zeit, als ich wollte, um zu den Leuten zu sprechen;
ich wieö sie von der Finsternis zum Licht und zur Gnade Gottes,
die sie lehren und ihnen Rettung bringen werde, und zum Geist
Gottes in ihrem Jnnern, der sie umsonst lehre.
Dann kehrte ich zu Richter Hothamö Haus zurück; als ich
eintrat, schlos er mich in seine Arme und sagte, sein Haus sei
mein Hau-3. Denn er freute sich sehr über das Werk dez Herrn
und das seine Kraft kund geworden. Dann erzählte er mir,
warum er am Morgen nicht mit mir zum Turmhauö gegangen
war, und was für Gründe er gehabt hatte; er hatte sich gesagt,
wenn er mit mir in-3 Turmhaus gehe, so würden die Wachen
mich ihm übergeben und da werde er so in die Sache verwickelt;
dann wisse er nicht, mas machen. Darum sei er froh gewesen,
als Hauptmann Pur?-loe gekommen; aber keiner von ihnen war
in Amts-kleidung gewesen oder hatte den Kragen um den Hals I
gehabt. GZ war damalt-3 etwas ganz Ungewöhnliche?-, das einer
ohne Kragen ins Turmhaus kam; aber Hauptmann Pursloe
war ohne einen solchen mit mir ins Turmhauö gekommen, so hatte
die Kraft des Herrn ihn übernommen, das er gar nicht daran
dachte.
Ich zog weiter und kam an einen Abend zu einer Herberge.
Ich bat die Wirtin, mir etwas Fleisch zu bringen, wenn sie solches-
habe; aber weil ich »du« und ,,dich« zu ihr sagte, sah sie mich
besremdet an; ich fragte sie, ob sie Milch habe. Sie sagte: nein.
Ich merkte, das sie nicht die Wahrheit sagte, und um sie noch
weiter zu prüfen, fragte ich sie, ob sie Rahm habe; sie verneinte eö
ebenfalls. Nun stand ein Butterfas im Zimmer und ein kleiner
Knabe, der daneben spielte, steckte seine Hand hinein und sties es
um und oerschiittete allen Rahm vor meinen Augen auf den
Boden; da zeigte es sich, das die Frau eine Lügnerin war. Sie


% \picinclude{./040-049/p_s046.jpg} 
erschrak, sties eine Verwünschung aus, hob das Kind auf und
schlug es tüchtig; aber ich machte ihr Vorwürfe wegen ihrer
Lüge und ihres Betrügens. Nachdem der Herr solcherweise ihre
Betrügerei und Bosheit aufgedeckt hatte, verlies ich das Haus
und ging weiter, bis ich zu einem Heuschober kam und brachte
nun die Nacht darin zu im Regen und Schnee, denn es war
drei Tage vor dem Tag, den sie Ehristfest nennen.
f Am folgenden Tage kam ich nach York, wo etliche sehr gott-
selige Leute waren. Am Ersten Tage der darauffolgenden
Woche hies mich der Herr in das grose Münster gehen und zum
Priester Bowles und seinen Zuhörern reden tn ihrer grosen
Kathedrale. Ich ging hin und als der Ptiestet geendet, sagte ich,
ich habe ihm und der Gemeinde eine Botschaft von Gott dem
Herrn zu bringen. ,,Dann sage sie schnell!« sagte einer der
,,Frotnmen« aus der Versammlung; denn es war gefroren und
schneite und war sehr kaltes Wetter. Ich sagte ihnen, solches
seiüdas Wort des Herrn an sie: ,,Jhr lebet in Worten, aber der
Herr der Allmächtige verlangt Früchte von euch.« Kaum waren
die Worte aus meinem Munde, so stiesen sie mich hinaus und
warfen mich die Stufen hinunter; aber ich stand aus, ohne verletzt
zu sein und ging in meine Wohnung. Etliche wurden überzeugt;
denn schon die Seufzer, die ich aussties unter dem Druck und
dem Zwang des Geistes Gottes in mir, genügten, um vieler
i Herzen zu öffnen und zu ergreifen, sodas sie bekannten, die Seufzer,
die ich ausstose, machen ihnen Eindruck.; mein ganzes Wesen
war bedrückt davon, das sie bekannten und nicht besasen, Worte
machten und keine Früchte brachten.
Nachdem ich für den Augenblick meinen Dienst in York getan
hatte und etliche dort gewonnen worden waren und die Wahrheit
Gottes angenommen und sich zu seiner Lehre bekannt hatten,
verlies ich York und wandte mich nach Cleveland und fand dort
Leute, welche die Kraft Gottes geschmeckt hatten. Ich sah, das
ein Same in jener Gegend war, und das Gott dort ein demiitiges
Volk hatte. Unterwegs holte mich, gegen Abend, ein Päpstlicher
ein und redete mit mir über seine Religion und über ihre Gottes-
dienste, und ich lies ihn alles sagen, was er aus dem Herzen
hatte. Ich brachte die Nacht in einer Schänke zu; am folgenden
Morgen trieb mich der Herr, zu diesem Päpstlichen zu reden. Jch
begab mich in seine Wohnung und zeugte gegen seine Religion


% \picinclude{./040-049/p_s047.jpg} 
Erlebnisse im Gefängnis zu Derbi; usw. 47
und alle ihre abergläubischen Gebräuche und sagte ihm, Gott sei
gekommen, sein Volk selbst zu lehren; das brachte den Papisten
dergestalt auf, das es ihn aus seinem eigenen Hause trieb ....
Obgleich zu der Zeit der Schnee sehr tief war, fuhr ich fort
herutnzureisen und kam zu einem Marktflecken, wo ich viele
,,Fromme« traf, mit denen ich lange Unterredungen hatte. Ich
stellte ihnen viele Fragen, die sie nicht beantworten konnten, weil
sie sagten, man habe sie noch nie in ihrem Leben so schwere
Dinge gefragt. Von da ging ich nach Stath, wo ich ebenfalls
viele ,,Fromme« und einige Ranter traf. Ich hatte grose Ver-
sammlungen unter ihnen undsviele Bekehrungen. Viele nahmen:die
Wahrheit aus, worunter einer, der hundert Jahre alt war; ein
anderer war ein Oberkonstabler und einer war ein Priester, namens
Philipp Scafe. Diesen machte der Herr später durch seinen Geist
zu einem freien Verkündiger seines freien Evangeliums.
Der Priester dieses Ortes war sehr hochfahrend und bedrückte
die Leute sehr mit seinen Abgaben. Wenn sie aus den Fischfang
gingen, so machte er sie Abgaben vom Erlös bezahlen, obgleich
sie dieselben so weit her hatten und sie bis nach Yarmouth zum
verkaufen brachten. Es trieb mich, dort ins Turmhaus zu gehen,
um die Wahrheit zu verkünden und den Priester blos zu stellen.
Als ich mit ihm geredet hatte und ihm die Unterdrückung des
Volkes vorgestellt hatte, lief er davon. Die Ältesten der Gemeinde
waren sehr hochmiitig und leichtfertig; darum verlies ich sie, nach-
dem ich das Wort des Lebens verkündet hatte, weil sie dasselbe
nicht aufnehmen wollten. Aber das Wort des Lebens, das ich
unter ihnen verkündet hatte, blieb bei etlichen von ihnen, so das
etliche der Ersten aus der Gemeinde des Nachts zu mir kamen,
und die meisten wurden gewonnen und bekannten sich zur Wahrheit;
so begann die Wahrheit sich in dieser Gegend auszubreiten, und
wir hatten grose Versammlungen; dadurch wurden die Priester
zornig und die Ranter fingen an, unruhig zu werden und liesen
mir sagen, sie wollten eine Unterredung mit mir haben, die Priester,
welche Unterdrückung übten, und die Runter. Es wurde ein Tag
festgesetzt und die Ranter erichienen; es kam auch noch ein anderer
Priester, ein Schotte, aber der Priester, welcher sich der Unter-
drückung schuldig gemacht hatte, nicht. Philipp Scafe, der be
kehrte Priester, war bei mir und es erschienen viele Leute. Als
wir uns gesetzt hatten, erklärte ein Ranter, namens T. Bushel,


% \picinclude{./040-049/p_s048.jpg}
er habe ein Gesicht von mir gehabt; ich sei an einem grosen Pult
gesessen und er habe kommen müssen und seinen Hut vor mir
abnehmen und sich tief vor mir verbeugen, und er habe es getan;
und noch viele andere Schmeicheleien sagte er mir. Ich sagte
zu ihm, er habe das nur erfunden und er solle zu sich selber
sagen: ,,Schäme dich, du Hund«. Er sagte, es sei nur Neid von
mir, so zu sagen. Darauf fragte ich ihn, was der Neid eigentlich
sei rmd wie er im Menschen entstehe und was das Htindische sei
und wie es im Menschen entstehe. Denn ich sah genau, das er
etwas Hündisrhes hatte, und darum wollte ich von ihm wissen,
wie dieses Hündische in ihm entstanden sei. ,,Denn«, sagte ich
ihm, ,,mir müssen zuerst von dem reden, mas in unserm Leib
geschieht, ehe wir von dem reden können, roas auser dem Leibe
ist.« Damit stopfte ich ihm das Maul und allen seinen Runter-
genossen, denn er war ihr Haupt. Dann ries ich den Priester,
welcher die Leute unterdrückte, aber er kam nicht; nur der schottische
Priester erschien, der mit wenig Worten zum Schweigen gebracht
war; denn es war innerlich kein Leben in ihm von dem, was er-
bekannte. Nun war die Gelegenheit da, mit den Leuten zu reden.
Ich zeigte klar, wie die Ranter waren und verglich sie mit den
Prahlern in Sodom. Ich zeigte, wie ihre Priester die gleiche
Sorte von Mietlingen seien, wie die falschen Propheten früherer
Zeiten, und wie die Priester damals das Volk auch in dieser
Weise regierten, indem sie ihren Gewinn im Auge hatten und
um Geld ihr Amt besorgten und um schnöden Gewinns willen
lehrten. Ich stellte Christus und die wahren Propheten und die
Apostel den Priestern gegenüber und zeigte, wie Ehristus, die
Propheten und die Apostel sie schon lange an ihren Früchten
erkannt hätten. Dann wies ich sie aus den Lehrer in ihrem
Jnnern hin, Jesus Christus, ihren Heiland. Und ich predigte
Christus in den Herzen, nachdem ich alle diese Höhen geebnet
hatte. Die Leute waren alle ruhig und die Widersacher zum
Schweigen gebracht. Denn obgleich es innerlich in ihnen kochte,
so hielt die Kraft sie doch gebunden, so das sie nicht losbrechen
konnten .....
Ein anderer Priester lies mich holen, um mit mir zu reden,
und etliche ,,Freunde« gingen mit mir nach seinem Hauö. Als
er hörte, das wir gekommen seien, entwischte er aus dem Hause
und versteckte sich unter einer Hecke. Die Leute gingen, ihn zu


% \picinclude{./040-049/p_s049.jpg} 
Erlebnisse im Gefängnis zu Derby usw. 49
zu suchen und fanden ihn, aber sie brachten ihn nicht dazu, zu
uns zu kommen. Daraus ging ich in ein nahegelegenes Turm-
haus, wo der Priester und das Volk in groser Erregung waren,
denn eben dieser Priester hatte den Freunden mit allem Ntöglichen,
das er tun werde, gedroht; als ich aber kam, machte er sich davon,
denn die Kraft des Herrn kam über ihn und über die andern.
Ja, des Herrn ewige Kraft kam über die Erde und drang zu
den Herzen der Menschen und machte die Priester und die
,,Frommen« zittern. Sie machte die Geister der Erde und der
Lust erbeben, zu welchen sie Vorgaben zu beten, sodas sie einen
Schreck bekamen, wenn es hies: »Der Mann in den ledernen
Kleidern kommi!«1) An vielen Orten machten sich die Priester,
wenn sie das hörten, davon, so waren sie von Furcht vor der
ewigen Kraft Gottes ergriffen .....
Von hier gingen wir über Scarbvrough .... nach Malton .....
Am Ersten Tag kam eine Frau, eine der angesehensten ,,Frvmmen«
unter den Jndependenten, welche ein solches Vorurteil gegen mich
hatte, das sie sagte, ehe sie kam, sie würde sich freuen, mich er-
hängt zu sehen; aber als sie kam, wurde sie gewonnen und ge-
hört seither zu den »Freunden«.
Daraus hatte ich hier grose Versammlungen; es hätten noch
mehr Leute daran teil genommen, aber sie wagten es nicht, aus
Furcht vor ihren Angehörigen. Es wurde damals als etwas
Unerhörtes angesehen, das man in Häusern predigte statt in der
,,Kirche«, wie sie es nannten; darum wurde sehr gewünscht, das
ich ins Turmhaus gehe und ddrt rede. Einer der Priester schrieb
mir und lud mich ein, im Turmhaus zu predigen, und nannte
mich seinen Bruder. Ein anderer Priester, eine bekannte Persön-
lichkeit, hielt dort eine Stunde. Nun hatte mir der Herr während
meiner Gefangenschaft in Derby kund getan, ich solle in den
Turmhäusern predigen, um die Leute von denselben abzubringen,
und es kamen mir auch zuweilen Bedenken wegen der Kanzeln,
in denen die Priester herumsaulenzten. Die Turmhäuser und
Kanzeln verletzten mein Gefühl, weil sowohl die Priester als auch
das Volk sie Gotteshäuser nannten und im Wahne waren, das
Gott da in äusern sichtbaren Häusern wohne, statt im Gegenteil
1) Fox trug immer Kleider aus Leder, die et wegen ihrer Einfachheit und
Dauerhaftigkeit allen andern Kleidungsstiicken vorzog. (Vgl. Carlyles, Surtor
Resartus: Ein Ereignis in der neuen Geschichte.)
George Fox. 4


% \picinclude{./050-059/p_s050.jpg} 
Verlangen zu tragen, das Gott und Christus in ihren Herzen
und Leibern wohne, aus das sie Tempel Gottes? würden. Denn
der Wostel sagt: ,,Gott wohnet nicht in Tempeln mit Händen
gemacht'' (Act. 7, 48). Weil man aber diese Stätten nun einmal
heilig hielt, so fand man ez- schrecklich, wenn man etwas dagegen
sagte. A18 ich inZ Turmhaus kam, waren nicht mehr altz 11 Zu-
hörer dort, und der Priester hielt ihnen die Predigt. Als nun
in der Stadt bekannt wurde, ich sei im Turmhause, so füllte sich
dasselbe bald mit Menschen. Alk- der Priester, der an dem Tage
zu predigen hatte, geendet hatte, hies er den andern Priester, der
mich aufgefordert hatte zu kommen, mich auf die Kanzel führen,
aber ich lies ihm sagen, ich brauche nicht auf eine Kanzel zu
steigen. Darauf lies, er mir wieder sagen, er wünsche aber, das
ich sie befteige, weil dort ein besserer Platz sei, an dem mich die
Leute sehen könnten. Ich lies ihm darauf sagen, man sehe mich
gut genug, da wo ich sei, ich sei nicht gekommen, solche Stätten
noch aufrecht zu erhalten und ihr Bestehen und den Handel, der
damit getrieben wird. Alk- ich dieö gesagt hatte, fingen sie an,
böse zu werden und sagten: ,,Da8 sind die falschen Propheten
der letzten Zeiten«. Diese Rede Verletzte etliche und sie murrten
darüber; nun stand ich auf und hies alle ruhig sein; ich stieg
auf einen hohen Stuhl und erklärte ihnen, woran man die falschen
Propheten erkenne, und das sie schon gekommen seien; und dann
zeigte ich ihnen im Gegensatz dazu die wahren Propheten, Christus-
und die Apostel. Ich wies sie alle an ihren inneren Lehrer,
Christue, der sie von der Finsterniö zum Lichte führen könne.
Nachdem ich ihnen verschiedene Schriftstellen erklärt hatte, wies
ich sie auf den Geist Gottes in ihren Herzen hin, durch welchen
sie zu ihm kommen könnten und erkennen, wer die falschen
Propheten seien. Nachdem ich so ein reiches Wirken unter ihnen
gehabt hatte, zog ich im Frieden von dannen ....
Hierauf kam ich nach Pickering, wo die Richter im Turm-
hauö ihre Sitzungen hielten; Ftiedenztichtet Robinson war Vor-
sitzender. Ich hatte zur gleichen Zeit eine Versammlung im
Schulhaus und viele »Fromme« und Priester wohnten ihr bei
und stellten allerlei Fragen, die zu ihrer Zufriedenheit beantwortet
wurden. ES war gerade die Zeit der Gerichtösit-zungen, und da
wurden auch vier Oberkonstabler bekehrt. GS kam Richter Robin-
son zu Ohren, das der Priester, den er allen andern Priestern


% \picinclude{./050-059/p_s051.jpg} 
Erlebnisse im Gefängnis zu Terby usw. 51
oorzog, besiegt und überzeugt worden war. Wir gingen nach
der Versammlung in eine Herberge; Richter Robinson’s Priester
war sehr bescheiden und lieb und wollte sogar durchaus mein
Essen bezahlen, was ich aber nicht zulies. Dann bot er mir sein
Turmhaus an, um darin zu predigen, aber ich lehnte es ab und
erklärte ihm und den andern, das ich eben gekommen sei, um die
Leute oon diesen Dingen ab und zu Christus zu bringen.
Am folgenden Morgen ging ich mit den vier Konstablern und
andern, um Richter Robinson zu besuchen, der mir unter der
Türe seines Zimmers entgegenkam. Ich sagte ihm, ich könne ihm
keine menschliche Ehre erweisen; er sagte, er sehe nicht aus das.
Ich ging nun mit ihm ins Zimmer und tat ihm den Unterschied
zwischenswahren und falschen Propheten dar, und wie die wahren
höher stehen als die falschen, und richtete seinen Sinn aus
Christum seinen Lehrer. Ich deutete ihm die Gleichnisse, und
wie es sich mit der Grwählung und Verwersung verhalte, wie
man in der ersten Geburt in der Verwersung sei und in der
zweiten in der Grwählung. Ich zeigte ihm, wer die Verheis-ungen
Gottes habe und wen sein Gericht verdamme. Gr gab alles zu
und war so offen für die Wahrheit, das, wenn ein anderer an-«
wesender Richter eine kleine Ginwendung machen wollte, er ihn
belehrte. Beim Fortgehen sagte er, ich tue sehr gut, diese mir
von Gott verliehene Gabe zu gebrauchen. Gr nahm den obersten
Konstabler beiseite und wollte ihm etwas Geld siir mich geben,
weil er nicht wollte, das ich in ihrer Gegend irgend welche Aus-
gaben habe; aber sie sagten ihm, das ich nicht dazu zu bringen
sei, etwas anzunehmen. Ich schätzte seine Freundlichkeit, das
Geld jedoch lehnte ich ab.
Ich zog im Lande umher und der Priester, der mich Bruder
genannt hatte, zog mit mir. Als wir in eine Stadt kamen, wo
wir im Sinne hatten etwas zu essen, läuteten die Glocken.
Ich fragte, warum sie läuten; man sagte mtr, sie läuten für mich,
damit ich im Turmhaus predige. Bald daraus trieb es mich
dorthin. Als ich kam, sah ich die Leute auf dem Turmhausplatze
versammelt; der alte Priester wollte, das ich ins Turmhaus gehe,
  ich sagte, es sei nicht nötig. Ge besremdete die Leute, das
ich nicht in das gehen wollte, das sie ,,Goiteshaus« nannten. Jch
stellte mich auf den Platz des Turmhauses und erklärte den Leuten,
ich sei nicht. gekommen, ihre göizendienerischen Tempel ausrecht
 


% \picinclude{./050-059/p_s052.jpg}
zu erhalten, noch die Priester mit ihren Zehnten, Zulagen, Ab-
gaben und Pfrtinden, noch ihre jüdischen und heidnischen Zere-
monien und Traditionen; denn die gelten mir alle nicht?-. Ich er-
klärte ihnen, dieses Stück Boden sei nicht heiliger, ale irgend ein
anderes Stück Land. Ich zeigte ihnen, das die Apostel, wenn
sie in die Synagogen und die Tempel der Juden gegangen seien,
die ja Gott selber sogar vorgeschrieben habe, so sei ez nur ge-
schehen, um die Leute davon Iabzubringen und von den Opfern
und Zehnten und den habsüchtigen Ptiestem jener Zeit. Und
die, welche zur Wahrheit belehrt wurden und an den von den
Aposteln gepredigten C-hristuö Hglaubten, hätten sich nachher in
den Wohnhäusern versammelt. Ich sagte ihnen, das alle, welche
Christus, das Wort deö Lebens, predigen, eö umsonst tun sollen
wie die Apostel, und wie Christus es geboten habe. So war ich
gesandt worden von Gott dem Herrn Himmels und der Erden
umsonst zu predigen und die Leute von diesen äuseren Tempeln
mit Händen gemacht, worin Gott nicht wohnt, abzubringen, damit
sie erkennen, das ihre Leiber Tempel Gottes werden sollen. Jch
muste die Leute abbringen von ihren jüdischen Zeremonien, aber-
gläubischen und heidnis chen Gebräuchen, Traditionen und Menschen-
satzruigen, von der Lehre all der Mietlinge, die Zehnten nehmen
und grose Psründen, die um Bestechung predigen und für Geld
weiösagen, die gar nicht von Gott und von Christus gesandt
sind, wie sie ja selber bekennen, wenn sie sagen, sie haben nie
die Stimme Gottes noch Christi vernommen. So ermahnte ich
denn die—Leute, abzulassen von alle dem, und wies sie auf den
Geist und die Gnade Gottetz hin, welche inwendig in ihnen sind,
und auf das Licht Jesu in ihren Herzen, aus das sie dazu kommen
möchten, C-hristum zu kennen, der sie umsonst lehre und ihnen
Rettung bringe und ihnen die Schrift öffne. Alleö war ruhig
und viele wurden gewonnen, der Herr sei gepriesen.
Ich kam daraus in eine andere Stadt, wo wieder eine grose
Versammlung war; der vorhin erwähnte Priester begleitete mich
und allerlei ,,Fromme« kamen dazu herbei. Ich sas mehrere
Stunden auf einem Heuschober und sagte nichte, denn sie sollten
nach Worten hungern. Die ,,Frommen« kamen immer wieder
zu dem alten Priester und fragten ihn, wann ich beginnen werde
zu reden. Er hies sie warten und sagte ihnen, das Volk habe
immer lange gewartet, bis Christus gesprochen habe. Schlieslich


% \picinclude{./050-059/p_s053.jpg} 
Erlebnisse im Gefängnis zu Derby usw. 53
trieb mich der Herr zu reden, und sie wurden von der Kraft deö
Herrn erfast; das Wort detz Lebens erreichte sie und es- geschah
eine allgemeine Bekehrung unter ihnen.
Ich zog weiter; der alte Priester und einige andere waren
mit mir. Unterwegs riesen ihn ein paar Leute an: ,,Mr. Bones,
wir sind euch Geld schuldig für Zehnten; kommt doch und nehmt
eZ!« Aber er wehrte mit der Hand ab und sagte, er habe genug,
er wolle nichts davon, sie sollten ez nur behalten; und er pries
Gott, das er solcheö sagen konnte. Schlieslich kamen wir zu dem
Turmhaus dieses- alten Priesters im Nioor; als wir eingetreten
waren, ging er vorausz und öffnete die Kanzeltür, aber ich sagte ihm,
ich würde nicht hineingehen. Das Turmhaus war stark bemalt;
ich sagte ihm und den Leuten, die dabei waren, das gemalte Tier
(Offb. 17, 3.) habe ein gemaltes Haus. Dann erklärte ich ihnen die
Entstehung aller dieser Häuser und ihre abergläubischen Gebräuche;
ich zeigte ihnen, das die Apostel nicht in die Tempel gegangen
seien, um diese aufrecht zu erhalten, sondern um die Leute zu
Christuö, dem wahren Gut, zu führen; ich zeigte ihnen den wahren
Gotte?-dienst, den Christus gegründet hat; ich zeigte den Unter-
schied zwischen Ehristus dem wahren Weg und allen verkehrten
Wegen, indem ich ihnen die Gleichnisse deutete und sie von der
Finsternitz zum wahren Lichte wieö; damit sie durch dasselbe sich
selbst erkennen möchten und ihre Sünden und ihren Erlöser und
durch den Glauben an ihn erlöst würden von ihren Sünden .....
Nun kam ich nach Eranstick, zu Hauptmann Pursloe und
Friedensrichter Hotham, die mich beide freundlich empsingen, weil
sie sich freuten, das die Kraft des Herrn erschienen war und das
die Wahrheit sich auzbreitete und so viele sie aufnahmen, und das
Frichen?-richter Robinson so freundlich gewesen war. Hotham
sagte, wenn Gott nicht diese Anschauungen von Licht und Leben
hätte kund werden lassen, so wäre das ganze Land von den
Rantern überschwennnt worden und alle Richter des- Landes mit
allen ihren Gesetzen hätten ihnen nicht zu wehren vermocht. ,,Denn«,
sagte er, ,,wenn sie auch gesagt und getan hätten, was- wir ihnen
befehlen, so hätten sie doch nicht von ihren Wisichten gelassen.
Aber eure Grundsätze der Wahrheit werfen alle ihre Grundsätze
und das, worauf sie die ihrigen gründen, über den Hausen«.
Darum war er so froh, das Gott diese Grundsätze des Leben-3
und der Wahrheit hatte durch mich kund werden lassen ....


% \picinclude{./050-059/p_s054.jpg} 
Als am folgenden Tage die Freunde mich verlassen hatten,
reiste ich allein weiter und verkündete den Tag des Herrn überall,
wohin ich kam, und ermahnte zur Buse. Eines Abends kam ich
in die Stadt Patrington, und während ich durch die Stadt ging,
ermahnte ich sowohl die Priester als das Volk Buse zu tun und
sich zum Herrn zu bekehren. Gs wurde finster, ehe ich ans Ende
der Stadt kam, und eine grose Menge hatte sich um mich ver-
sammelt, während ich das Wort des Lebens verkündete. -- Als
ich meine Ausgabe erfüllt hatte, ging ich in eine Herberge und
verlangte Unterkunft für die Nacht, aber sie wurde mir verweigert.
Daraus bat ich um etwas Fleisch und Milch, ich wolle es bezahlen;
aber auch das wollte man mir nicht geben. So verlies ich die
Stadt; einige junge Leute kamen hinter mir drein und fragten
mich, was es neues gebe. Ich hies sie Buse tun und Gott
fürchten. Als ich eine Strecke weiter gegangen war, kam ich wieder
an ein Haus und bat, man solle mir etwas Fleisch und Milch
geben und Nachtherberge, gegen Bezahlung; aber sie schlugen es
mir ab; dann ging ich zu einem andern Haus und verlangte das-
selbe; aber sie wiesen mich ebenfalls ab. Jnzwischen war es so
dunkel geworden, das ich die Landstrase nicht mehr sehen konnte;
ich endeckte einen Wassergraben und schöpfte etwas Wasser um
mich zu erfrischen; dann überschritt ich den Graben und da ich
von der Reise müde war, setzte ich mich unter einen Ginsterstrauch
und wartete bis es Tag war. Mit Tagesanbruch erhob ich mich
und ging weiter. Hinter mir drein kam ein Mann mit einer
Heugabel, der schritt neben mir her bis zu einer Stadt, und
noch ehe die Sonne ausgegangen war, hatte er diese Stadt und
die Polizei gegen mich ausgehetzt; ich verkündete Gottes ewige
Wahrheit unter ihnen und warnte sie vor dem Tag des Herrn,
der kommen würde über alle Sünde und Ungerechtigkeit, und er-
mahnte sie, Buse zu tun. Mer sie.griffen mich und brachten
mich nach Patrington zurück, etwa drei Meilen weit, und be-
wachten mich mit Stöcken, Heugabeln und Hellebarden. Als ich
nach Patrington kam, war die ganze Stadt in Aufruhr. Die
Priester und das Volk berieten sich zusammen; so konnte ich
ihnen abermals das Wort des Lebens verkünden und sie zur Buse
ermahnen. Endlich nahm mich einer der »Frommen«, ein guter
Mann, mit in sein Haus, wo ich mich an etwas Brot und Milch
erlabte, denn ich hatte seit mehreren Tagen nicht-3 gegessen. Dann


% \picinclude{./050-059/p_s055.jpg} 
Erlebnisse ini Gefängnis zu Derby usw. 55
schleppten sie mich etwa neun Meilen weit zu einem Richter.
Als wir nahe bei dessen Haus waren, kam einer hinter uns her
geritten und fragte mich, ob ich der sei, der verhaftet worden
war. Ich fragte, warum er es wissen wolle; er sagte, es ge-
schehe in keiner bösen Absicht; da sagte ich ihm, das ich es
sei; darauf ritt er voraus zum Richter. Meine Begleiter sagten,
hoffentlich sei der Richter nicht betrunken, wenn wir zu ihm
kämen; denn er pflegte schon frtihmorgens betrunken zu sein.
Als ich vor ihn trat und meinen Hut nicht abnahm und ihn mit
Du anredete, fragte er den, welcher uns oorgeritten war, ob ich
verrückt sei, aber er sagte ihm, nein, es sei mein Grundsatz. Jch
ermahnte den Richter, Buse zu tun und sich zum Licht zu be-
kehren, mit dem Christus ihn erleuchtet, damit er durch dasselbe
alle seine bösen Worte und Taten erkennen möge, und zu Christus
zurückzukehren, solange es noch Zeit sei. ,,Ja, ja«, sagte er, ,,das
Licht von dem im dritten Kapitel des Johannes gesprochen wird-«.
Joh bat ihn, er möge doch auf dieses Licht achten und ihm ge-
horchen. Während ich ihn ermahnte, legte ich ihm die Hand auf,
und er ward übernommen von der Kraft des Herm und die
Wächter waren bestürzt. Er führte mich nun in ein kleines Gemach,
um zu untersuchen, was ich von Briefen und Schriften in der
Tasche habe; ich wies ihm meine Kleider und zeigte ihm, das
ich keine Briefe bei mir hatte; er sagte, man sehe an meiner
Wäsche, das ich kein Landstreicher sei, und lies mich frei. Jch
ging mit dem Mann, der vor uns hergeriiten, nach Patrington
zurück, denn er lebte daselbst. Als wir ankamen, wünschte er, ich
solle eine Versammlung auf dem Hauptplatz halten, aber ich sagte
es sei nicht nötig, sein Haus genüge. Gr wollte, das ich zu Bett
gehe oder mich doch aufs Bett lege; dies wünschte er namentlich,
damit er sagen könne, man habe mich in oder doch wenigstens
auf einem Bett gesehen; denn es ging das Gerücht, ich wolle
in keinem Bett schlafen, weil ich damals oft im Freien über-
nachtete. Als der Erste Tag kam, ging ich ins Turmhans
und verkündete dem Priester und dem Volk die Wahrheit; und
die Leute taten mir nichts, denn die Kraft Gottes war über sie
gekommen. Gleich nachher hatte ich eine grose Versammlung in
dem Hause des Mannes, der mich beherbergte, und viele wurden
von Gottes ewiger Wahrheit überzeugt und sind derselben treu
geblieben bis aus den heutigen Tag. Sie bereuten es sehr, das


% \picinclude{./050-059/p_s056.jpg} 
sie mich nicht aufgenommen und beherbergt hatten, als ich zuerst
bei ihnen gewesen war .......

