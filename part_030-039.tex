
% \picinclude{./030-039/p_s030.jpg} 
Grund der Schrift.'' Da eine Bibel zur Hand war, hieß ich sie,
mir die betreffende Stelle zu zeigen, und sie zeigten mir die Stelle,
wo daß Tuch vor Petrus herabgelassen wurde und die Stimme
sagte: ,,WaS Gott gereiniget hat, daö mache du nicht gemein''
(Act. 10, 15). A15 ich ihnen zeigte, daß diese Stelle nichtß für
sie beweise, brachten sie eine.andere vor, die davon handelte, wie
Gott alle mit sich selbst versöhnt im Himmel und auf Erden
(Col. 1, 20). Jch sagte ihnen, daß ich diese Stelle ebenfallö an-
erkenne, daß sie aber ebensowenig sür sie passe. Alß ich nun
vernahm, wie sie sagten, sie seien Gott, fragte ich sie, ob sie
wissen, ob ez morgen regnen werde? Sie antworteten, daß sie
daß nicht sagen könnten. Jch erwiderte ihnen: Gott könne das
sagen. Darauf fragte ich sie, ob sie immer so bleiben würden,
wie sie jetzt seien, oder ob sie sich ändern würden? Sie ant-
worteten: sie wüßten ez nicht.'' Jch erwiderte: ,,Gott kann es-
sagen und Gott verändert sich nicht. Jhr sagt, ihr seid Gott und
wißt nicht, ob ihr euch verändert oder nicht?« Sie wurden ver-
wirrt und für den Augenblick fast überwunden. Nachdem ich sie
wegen ihrer Gotteßlästerungen zurecht gewiesen hatte, ging ich
fort, demr ich merkte, daß sie Ranter 1) waren. Jch war nie
mit solchen zuvor zusammengetrosfen und ich priez die Güte des
Herrn, daß sie mir erschienen war, ehe ich zu ihnen gekommen
war. Nicht lange nachher schrieb einer dieser Ranterö, namenß
Joseph Salmon, ein Buch, in dem er widerrief, worauf sie die
Freiheit erhielten ....
Bei meinem Herumziehen auf den Jahrmiirkten und Märkten
und in den Städten, sah ich Tod und Finsterniö in allen, welche
die Kraft dez Herrn nicht ergriffen hatte. Alß ich durch Leieestershire
zog, kam ich nach Twy-Croß; daselbst waren Steuereinnehmer.
Der Herr trieb mich zu ihnen zu gehen und sie zu ermahnen,
sich vor Unterdrückung der Armen zu hüten. Das machte den
Leuten einen großen Eindruck. GS war in jener Stadt ein ange—
sehener Mann, welcher lange krank gewesen war und von den
Arzten aufgegeben wurde; und etliche Freunde aus- der Stadt
wünschten, daß ich zu ihm gehe. Jch ging zu ihm hinauf in sein
Zimmer und sagte ihm das Wort deß Lebens, und es trieb mich,
1) Runter, eine Sekte von mystischen Schwürmern, die sich rtihmten,
daß Christus in ihnen wohne, aus ihnen rede und sie selbst Christuö seien;
daher der Spottname ,,Ranter«——Prahler.


% \picinclude{./030-039/p_s031.jpg} 
Der Turnnlt in Nottingham. Wachsender Widersnmd usw. 31
mit ihm zu beten. Und der Herr erhörte uns und machte ihn
gesund. Als ich aber darauf in einem untern Raum des Hauses
zu der Dienerschaft und einigen andern Anwesenden redete, stürzte
einer aus einem Nebengemach herein mit dem nackten Degen in
der Hand, gerade auf mich los?-. Jch sah ihn unerschrocken an
und sagte: ,,Wehe Dir, arme Kreatur, was willst Du tun mit
Deiner fleischlichen Waffe? mir ist sie nicht mehr als ein Stroh-
halm.« Die Anwesenden waren sehr bestürzt und er entfernte
sich in Zorn und Wut. Alß sein Herr davon hörte, entließ er
ihn auö feinem Dienst. Also beschützte mich der Herr und half
diesem Schwachen und er wurde später den ,,Freunden« sehr
zugetan; und alö ich wieder in jene Stadt kam, besuchte er mich
mit seinem Weibe .....
Als ich nach Derbi) kam, wohnte ich im Haufe eineß Arztes?-;
eine Frau wurde gewonnen und noch Viele andere. Alß ich in
mein Zimmer ging, läutete die Glocke deß Turmhausej-’; nur schon
sie zu hören, ging mir durch Mark und Bein; ich fragte warum
die Glocke läute? man sagte mir, daß an dem Tage eine große
gotteödienstliche Versammlung stattfinde, dazu viele auß dem
Heer, sowie Priester und Prediger kommen werden. Da trieb
eß mich, auch hin zu gehen; und alö sie fertig waren, redete ich
zu ihnen, wa?. mir der Herr eingab. Sie waren ziemlich ruhig;
aber eine Wache kam, nahm mich bei der Hand und sagte, ich
müsse vor den Rat sowie auch die andern beiden, die mit mir
waren. Um die erste Nachmittags:-’stunde wurde ich-vorgenommen.
Jch wurde gefragt, warum ich hingegangen sei. Jch sagte, Gott
habe mich getrieben, es zu tun, und weiter sagte ich. ,,Gott
wohnet nicht in Tempeln mit Händen gemacht.« Jch sagte ihnen
ferner, all ihr Predigen, ihr Taufen und ihr Opfern werde sie
nie heiligen, und ermahnte sie, auf Ehristum in ihnen zu schauen
und nicht aus Menschen; denn Christuö sei es, welcher sie heilige.
Darauf ergingen sie sich in Vielen Worten, aber ich sagte ihnen,
sie sollten sich nicht über Gott und Christus streiten, sondern
ihm gehorchen. Die Kraft Gotteß donnerte unter ihnen und
sie zerstoben davor wie Spreu. Sie hießen mich mehrmalö
aus- dem Zimmer gehen und dann wieder hereinkommen und
trieben mich hin und her; von ein Uhr an bis abends neun ver-
hörten sie mich. Zuweilen sagten sie mir mit höhnischen Worten,
ich sei nicht bei Sinnen. Zuletzt fragten sie mich, ob ich geheiligt


% \picinclude{./030-039/p_s032.jpg}
sei; ich antwortete: »Ja, denn ich war im Paradies- Gotte?-,«
(2. Cor. 12, 4). Dann fragten sie mich, ob ich keine Sünde habe.
Jch antwortete: ,,ChristuZ, mein Erlöser, hat die Sünde von mir
genommen und in ihm ist keine Sünde.« Sie fragten, wie ich
wüßte, daß Christus in uns wohne? Jch sagte: »Durch seinen
Geist, den er unctz gegeben.« Um mich zu versuchen, fragten sie,
ob einer oon uns Christus sei? Jch antwortete: »Nein, wir
sind nichts, Christuö ist alle?-.« Sie sagten: wenn ein Mann
stehle, ob daß keine Sünde sei? Jch antwortete: »AlleS Unrecht
ist Stinde.« Alß sie ez nun müde geworden, mich zu verhören,
verurteilten sie mich zu sechß Monaten im Korrektionßhauö in
Derby alö Gotteölästerer, wie auS folgendem Verhastbesehl zu
ersehen ist:
An den Oberaufseher des Korrektionöhauseß in Derby.
,,Hiemit senden wir euch die Personen George Fox, oormalß
in Manßfield in der Grafschaft Nottingham, und John Fretwell,
Landwirt, vormalß in Stanießby in der Grafschaft Derby, vor
uns gebracht am heutigen Tag und beschuldigt eingestandener
Äußerungen verschiedener gotteölästerlicher Ansichten, die einem
jüngst verfaßten Parlament?-beschluß1) zuwider sind; sie sollen daher
sogleich nach Einsicht Dieses aufgenommen werden, besagter
George Fox und Johann Fretwell, in euern Gewahrsam und
darin sicher verwahrt werden, für die Dauer oon 6 Monaten,
ohne Möglichkeit einer Bürgschaft oder Abkürzung, es wäre denn,
daß sie sich htulttnglich durch ein gutes Betragen auöweisen, oder
durch unsere eigene Verordnung frei würden. Solcheö zutun
möget ihr nicht versäumen.
Mit unsrer Hand und Siegel gegeben am heutigen Tage
:30. Oktober 1650. Ger. Bennet. «
Nath. Barton.« .....
Während ich im Gefängnis war, kamen oft ,,Fromme«, um eine
Unterredung mit mir zu haben; noch ehe sie etwas sagten, merkte
ich immer, daß sie kamen, um für die bleibende Sündhaftigkeit und
Unoollkommenheit einzutreten. Jch fragte sie, ob sie glänbig seien und
11 Partamentöbeschluß vom 2. Mai 1648 gegen GotteHläster11ng und
Ketzerei. Ein Beschluß, der von der unglaublichen Härte der damals regierenden
Pre-zbyterianer zeugt.


% \picinclude{./030-039/p_s033.jpg} 
Der Tnmult in Nottingham. Wachsender Widerstand usw. Z3
Glauben hätten? Sie sagten: ,,Ja.« Jeh fragte sie: in wen?
Sie sagten: ,,J«n Christu-3.« Jch erwiderte: Wenn ihr wahre
an Christus- Glaubende seid, so seid ihr vom Tode zum Leben
eingegangen, und wenn ihr vom Tode frei seid, dann seid ihr ez
auch von der Sünde, die den Tod bringt. Und wenn euer
Glaube wahr ist, so wird er euch den Sieg geben über Sünde
und—Tc-ufel und eure Herzen und Gewissen reinigen — denn der
wahre Glaube ist in reinen Gewissen (1 Tim. 3) und er wird
machen, daß ihr Gott gefallet und euch wieder Zugang zu ihm
Verschaffen.« Aber sie wollten nicht von Reinheit und von Sieg
über Sünde und Teufel hören; denn sie sagten, sie können nicht
glaubeny daß jemand könne frei von Sünde sein schon dießseitß
des Grabe-3. Jch hieß sie, da-Z Schwatzen über die Schrift, die
das Wort heiliger Männer sei, aufgeben, wenn sie für Unheiligkeit
eintreten wollten. Einmal kam auch eine Anzahl solcher ,,Frommer«
zu mir und fingen an, die Sündhaftigkeit zu befürworten. Jch
fragte sie: ob sie Hoffnung hätten? ,,Ja, ja! daß wäre, wenn
wir keine Hoffnung hötten!« Jch fragte sie: ,,Waö für eine
Hoffnung ist etz, die ihr habt? Jst Christus in euch die Hoffnung
eurer Herrlichkeit? (Col. 1, 27.) Reinigt sie euch, gleich wie er
rein ist?« Aber sie wollten nichtö davon hören, «daß sie selber
hienieden schon rein werden sollten. Darauf gebot ich ihnen,
nicht mehr über die Schrift zu reden, welche daß Wort heiliger
Männer sei. Denn die heiligen Männer, welche die Schrift ge-
schrieben haben, seien für Heiligkeit in Herz, Leben und Wandel
hienieden eingetreten. ,,Jhr aber«, sagte ich, ,,tretet für Unreinheit
und Sünde ein, die vom Teufel sind, wa-3 habt ihr zu schaffen
mit den Worten heiliger Männer?«
Der Kerkermeister, ein großer ,,Frommer«, hatte eine schreck-
liche Wut auf mich und redete sehr schlecht von mir. Aber etz
gefiel dem Herm, ihn eines Tageö so mächtig zu ergreifen, daß
er in großer Angst und innerer Not war. Alö ich in meinem
Zimmer umherlief, hörte ich klägliche Laute und hörte, wie er zu
seiner Frau sagte: ,,Frau, ich habe den Tag des Gerichts gesehen,
und George Fox war da, und ich hatte Angst vor ihm, weil ich
ihm so viel böseß zugefügt hatte und so vieles wider ihn zu den
Vorgesetzten und ,,Frommen« gesagt hatte und zu den Richtern
und in den Wirt?-häusern.« Hierauf kam er gegen Abend zu mir
ins Zimmer und sagte: »Jch bin gegen euch gewesen wie ein
George Fox- 3


% \picinclude{./030-039/p_s034.jpg} 
Löwe; nun aber komme ich wie ein Lamm und wie der Kerker-
meister, der zitternd zu Pauluß und Silaß kam.« Und er bat,
daß er bei mir bleiben dürfe. Jch sagte, ich sei in seiner Macht
und er könne mit mir machen, waß er wolle; aber er sagte:
nein, er wolle meine Grlaubniß haben, und er möchte, daß er
immer mit mir sein könnte, aber nicht mich alß Gefangenen haben;
er und sein Hauß seien meinetwegen geplagt gewesen. Jch erlaubte
ihm denn, bei mir zu sein, und er öfsnete mir sein Herz rmd
sagte, er glaube, daß daß, waß ich vom wahren Glauben und
von der wahren Hoffnung sage, wahr sei, und er wunderte sich,
daß der andere, der mit mir gefangen war, nicht dabei bleibe.
Gr sagte: ,,Jener andere tat unrecht, ihr aber seid ein Gerechter.«
Gr gestand mir auch, daß oft, wenn ich ihn gebeten hatte, mich
unter daß Volk gehen zu lassen, um ihnen daß Wort deß Herrn
zu verkünden, und er eß mir verweigert habe, habe er sich damit
eine große Last auferlegt; denn er sei in große Angst geraten
und einige Zeit ganz verstört und niedergedriickt gewesen, so daß
er gar keine Kraft mehr gehabt habe. Am Morgen ging er fort
und ging zu den Richtern und sagte ihnen, wie er und sein Hauß
meinetwegen geplagt gewesen seien, und einer der Richter erwiderte
ihm, daß auch sie geplagt seien, darum daß sie mich sesthielten.
Eß war Richter Bennet zu Derby, welcher un-3 zuerst Quäkers)
genannt hatte, weil ich ihnen gesagt hatte, gsie müßten erzittem
vor dem Wort Gotteß. Solcheß geschah im Jahre 1650.
Hierauf erlaubten mir die Richter, eine Meile weit zu gehen.
Joh sah, wo sie hinauß wollten und sagte dem Kerkermeister,
wenn sie mir zeigen wollten, wie weit eine Meile sei, so wolle
ich manchmal so weit gehen; denn ich glaube, sie dachten, ich
würde davon laufen. Und der Kerkermeister gestand nachher,
daß sie eß in dieser Absicht gestattet hätten, damit ich entkomme
und sie von ihrer Angst befreit würden; aber ich sagte ihm, daß
ich nicht diesen Geist habe. «
Dieser Kerkermeister hatte eine Schwester, ein kränklicheß
jungeß Weib. Sie kam zu mir, um mich zu besuchen; und nach-
dem sie einige Zeit bei mir gewesen war, und ich Worte der
Wahrheit zu ihr geredet hatte, ging sie hinunter und sagte den
1) Quüker, das heißt ,,Zitterer«, der Spottname, den die Gegner den
Freunden anhängten, wegin der in ihren ersten Versammlungen sich einstellendea
Konvulsionen.


% \picinclude{./030-039/p_s035.jpg} 
Erlebnisse im Gefängnis zu Derbi) usw. Z5
andern, wir seien unschuldige Leute und täten niemand nichts zu
leide, sondern allen nur Gutes, sogar solchen, die uns haßten,
und bat sie, freundlich gegen mich zu sein. ....
Während ich im Korrektionshaus war, besuchten mich meine
Verwandten, und da sie über meine Gefangenschaft bekümmert
waren, gingen sie zu den Richtern und baten sie, daß ich mit
ihnen heim gehen dürfe. Sie erboten sich, sich mit hundert Pfund
zu verbiirgen und einige andere aus Derby, die mit ihnen waren,
je mit siinszig Psund, daß ich nicht mehr dorthin komme, um
gegen die Ptiestet zu reden. So wurde ich vor die Richter
gebracht, und weil ich nicht einwilligen wollte, daß irgendjemand
sich meinetwegen verpftichte, — denn ich war ja keines Vergehens
schuldig und hatte das Wort des Lebens und der Wahrheit ge-
redet, — erhob sich Richter Bennet zornig, und als ich niederkniete, um
Gott zu bitten, ihm zu vergeben, rannte er auf mich los und schlug
mich mit beiden Händen und schrie: »Fort mit ihm! Kerkermeister,
nimm ihn fort!« Hierauf wurde ich wieder in den Kerker gebracht
und mußte dort bleiben, bis meine Zeit von sechs Monaten um
war. Aber ich durfte nun eine Meile weit allein gehen, was ich
tat, als ich fühlte, daß ich es durfte. Ost ging ich auf den Markt
und in die Straßen und ermahnte die Leute, sich von ihrer
Schlechtigkeit zu bekehren, und ging dann wieder ins Gefängnis.
Und da Leute von allerleiislteligionen mit mir im Gefängnis
waren, ging ich hie und da zu ihnen und wohnte ihren Versamm-
lungen an den Ersten Tagen bei ....

%%%%%%%%%%%%%%%%%%% Kapitel 4. %%%%%%%%%%%%%%%%%%%%%%%%%%%%%%
\chapter[Kampf gegen die Ranter]{Kampf gegen die Ranter}

\begin{center}
\textbf{Erlebnisse im Gefängnis zu Derby. Ein \zitat{Wehe} 
über die Stadt
Lichfield\ort{Lichfield}. Erste Missionsgenossen. Antikirchliche 
Agitation und Kampf gegen die Ranter\index{Ranter}.}
\end{center}


Während ich noch im Gefängnis war, kam ein Soldat zu
mir und erzählte hmir, wie er im Turmhause gewesen sei und dem
Priester zugehört habe, und wie dann auf einmal eine grofze Angst
über ihn gekommen sei und die Stimme des Herrn also zu ihm
geschehen sei: ,,Weißt du nicht, daß mein Diener im Gefängnis
ift? zu ihm gehe und frage ihn um Rat«. Jch redete mit ihm
wie es sein gegenwärtiger Zustand erheischte, und sein Verständnis
zbt


% \picinclude{./030-039/p_s036.jpg} 
wurde geöffnet. Jch sagte ihm, daß der, welcher ihm seine Sünden
ausdecke und ihn um ihretwillen ängstige, ihm auch die Rettung
zeigen werde; denn der dem Menschen die Sünden aufdeckt, ist
derselbe, der sie auch hinwegnimmt. Während ich mit ihm redete,
offenbarte sich ihm der Herr, so daß er anfing, die Wahrheit dez
Herrn und Gorteß Gnade zu erkennen; er fing an, unerschrocken
in seinem Regiment unter den Soldaten von der Wahrheit zu
reden; denn die Schrift wurde ihm mehr und mehr offenbar, und
er ging soweit zu sagen: sein Oberst sei blind wie Nebukadnezar,
daß er den Diener deß Herrn inß Gesängnis werfe. Von da an
hegte sein Oberst einen Groll gegen ihn. Alß im darauffolgenden
Jahre in der Schlacht von Worcester die beiden Armeen neben-
einander lagen, kamen zwei auß der Armee des Könige und
forderten, daß zwei anß der Armee des Parlamentß sich mit ihnen
schlagen sollten; da wählte der Oberst ihn und noch einen, um
der Forderung Folge zu leisten. A13 sein Kamerad im Kampfe
gefallen war, trieb er seine beiden Gegner zur Stadt hinauö, ohne
einen Schuß auf sie abzuseuern; dies erzählte er mir nach seiner
Rückkehr mit eigenem Munde. Nach Beendigung der Schlacht
sah er die Betrügerei und Heuchelei der Offiziere ein, und im
Gedanken daran, wie wunderbar der Herr ihn bewahrt hatte und
waß etz eigentlich um den Krieg sei, legte er die Waffen nieder.
Die Zeit meiner Gefangenschaft war nun fast zu Ende und
da viel neue Soldaten aus-gehoben wurden, so wollten mich die
Kommifsäre zu ihrem Hauptmann machen, und die Soldaten
erklärten, sie wollten keinen andern als mich haben. Der Kerken
meister erhielt den Befehl, mich vor die Soldaten und ihre Vor-
gesetzten aus den Marktplatz zu siihren; dort boten sie mir dieseß
Ehrenamt, wie sie eß nannten, an und fragten mich, ob ich nicht
wolle die Waffen ergreifen für den Commonwealth gegen Karl
Stuart.1) Ich erwiderte ihnen, ich wisse wohl, woher aller
Krieg komme: auö der Begierde, wie schon Jakobus- lehre
(Jak. 4); ich aber stehe in jener Kraft und jenem Leben,
die von vornherein allen Krieg ausschließen. Sie wollten mich
überreden, ihr Anerbieten anzunehmen; sie meinten, ich weigere
mich nur aus Bescheidenheit. Aber ich erklärte ihnen, ich sei in
den Bund des- Friedenö eingetreten, welcher bestanden, ehe es
1) 1651 Schlacht von W1-rceöter zwischen Cron1toell(Con1n1onwealth) und
Karl ll.


% \picinclude{./030-039/p_s037.jpg} 
Erlebnisse im Gefängnis zu Derby usw. 37
Krieg und Zank gab. Sie sagten, sie bieten es mir in Liebe und
Zuneigung an wegen meiner Tugend, und ähnliche Schmeicheleien
mehr. Aber ich sagte ihnen, wenn solches ihre Liebe sei, so trete
ich sie mit Füßen. Da wurden sie zornig und sagten: ,,Nimm
ihn hinweg, Kerkermeister, und wirs ihn in den untersten Kerker
zu den Schelmen und Verbrechern.« Jch wurde weggefiihrt und
an einen wiisten, stinkenden Ort 1) gebracht, wo kein Bett war,
mit 30 Verbrechern, wo ich beinahe ein halbes Jahr gefangen
war, außer, wenn sie mich dann und wann ein wenig in den
Garten ließen, weil sie sicher waren, daß ich nicht davon laufe.
Es hatte damals, als man mich in diesen Kerker gebracht hatte,
geheißen, ich werde wohl nicht mehr heraus kommen. Aber ich
glaubte an Gott und daß ich zu seiner Zeit daraus befreit werde.
Denn der Herr hatte es mir vorausgesagt, daß ich nicht bald
von diesem Ort wegkomme, da ich dort eine Ausgabe für ihn zu
erfüllen habe.
Als es bekannt wurde, daß ich im Kerker von Derby sei, kamen
meineAngehörigen, um mich wieder zu besuchen; denn sie betrachteten
es als eine große Schande für sie, daß ich um der Religion
willen gefangen war; und etliche hielten mich für verrückt, weil
ich für die Reinheit, Gerechtigkeit und Vollkommenheit eintrat.
Unter denen, die zu mir kamen, war einer aus Nottingham,
ein Soldat, der früher Baptist gewesen war. Jin Laufe des Ge-
sprächs sagte er zu mir: »Dein Glaube gründet sich auf einen
Mann, der in Jerusalem gestorben sein soll; solches ist aber nie
geschehen«. Gs betrübte mich sehr, ihn so reden zu hören, und
ich sagte: ,,Wie! hat nicht Christus gelitten vor den Toren Jeru-
salems durch die Juden, die ,,Frommen«, die Hohenpriester und
durch Pilatus?« Aber er leugnete, daß Christus je äußerlich
gelitten habe. Jch fragte ihn, ob denn keine Hohenpriester, keine
Juden, kein Pilatus äußerlich dort gewesen sei? und als er das
nicht bestreiten konnte, sagte ich: ,,So gewiß ein Hohepriester,
ein Pilatus und Juden äußerlich dort gewesen sind, so gewiß ist
Christus äußerlich verfolgt worden von ihnen und hat durch sie
1) Die Zustände der Gefängnisse und Korrektionshäuset im 17. Jahrh.
waren überaus traurig. Überall herrschte große Unreinlichkeit; die Verwaltung
war der Willkür des Gefängnisvotstehers anheim gegeben, der nicht besoldet
war, sondern von den Gefangenen bezahlt wurde, die die Kosten ihres Aufent-
haltes selbst tragen mußten. Vgl. Aschrott, Englisches Gesängniswesen.


% \picinclude{./030-039/p_s038.jpg} 
gelitten«.' Die Reden dieses Menschen veranlaßten eine Ver-
leumdung gegen uns, als ob die Quäker bestritten, daß Christus
gelitten habe und in Jerusalem gestorben sei. Gs war dies ganz
falsch; nie war der leiseste Gedanke daoon in unsern Herzen ge
wesen; es war eine bloße Verleumdung, die uns traf, und die
aus dem Gerede dieses Menschen entstanden war. Derselbe
Mässch behauptete auch, niemals habe irgend ein Apostel oder
Prophet, oder Heiliger oder Mann Gottes äußerlich gelitten; alle
ihre Leiden seien innerlich gewesen; aber ich bewies ihm anBei-
spielen, wie viele unter ihnen gelitten und durch wen sie gelitten;
und so widerlegte die Kraft des Herrn seine oerkehrten Ansichten.
Eine andere Sorte kam zu mir, die behaupteten, sie könnten
Geister unterscheiden. Jch fragte sie, welches der erste Schritt
zum Frieden sei? und in was der Mensch seine Rettung suchen
müsse? Sie fuhren auf und sagten in ihrem Hochmut, ich sei
verrückt; und solche wollten Geister unterscheiden können und
kannten nicht einmal ihren eigenen Geist!
Während dieser Zeit meiner Gefangenschaft geriet ich in
große Bekümmernis über das Vorgehen der Richter und Beamten
in ihren Gerichtshösen. Gs trieb mich, an die Richter zu schreiben,
darum daß sie das Todesurteil fällten wegen allerlei unwichtiger
Vergehen, in Geldsachen oder das Vieh betreffend. Jch mußte
ihnen zeigen, wie solches von jeher dem Gesetz Gottes zuwider
war; ich war deswegen in meinem Geiste sehr betrübt bis inden
Tod, aber da ich mich unter den Willen Gottes stellte, so er-
wachte ein himmlisches Sehnen nach dem Herrn in meinem Herzen,
ich sah den Himmel offen und freute mich und gab Gott die
Ehre ....
Jn diesem Zustande trieb es mich, an die Richter zu schreiben,
wie schädlich es für die Gefangenen sei, so lange im Kerker zu
sein, wie sie da schlechtes von einander lernten, wenn sie mit-
einander über ihre bösen Taten reden. Darum sollten die Urteile
rasch gesprochen werden. Denn ich war ein gottseliger Jüngling
rmd wandelte in der Furcht des Herrn; es betrübte mich, ihre
schlechten Reden zu hören, ich mußte ihnen oft Vorstellungen über
ihre bösen Worte machen und über ihr häßliches Betragen unter-
einander. Die Leute wunderten sich, wie ich bewahrt und behiitet
blieb; denn nie konnten sie mir ein Wort oder eine Tat nach-
weisen, die sie hätten zu meinen Ungunsten auslegen können


% \picinclude{./030-039/p_s039.jpg} 
Erlebnisse im Gefängnis zu Derby usw. 39
während der ganzen Zeit, die ich dort war; denn die unendliche
Kraft des Herrn hielt mich aufrecht und bewahrte mich während
der ganzen Zeit; ihm sei Lob und Ehre immerdar.
Es war eine junge Person mit mir im Gefängnis, die ihrem
Herrn Geld gestohlen hatte. Llls sie zum Tode verurteilt werden
sollte, schrieb ich an den Richter und ans Schwurgericht und
stellte ihnen vor, wie es immer gegen das Gesetz Gottes gewesen
sei, die Leute wegen Diebstahls zum Tode zu verurteilen, und
bat um Gnade. Sie wurde aber doch verurteilt, und man grub
ihr ein Grab und führte sie zur Hinrichtung. D,a schrieb ich noch
einmal ein paar Worte und warnte alle, sich vsor Raubgier und
Habsucht zu hüten, da sie von Gott wegführe, und ermahnte alle
den Herrn zu fürchten, allen irdischen Begierden zu entsagen und
die Zeit zu nützen, dieweil sie da ist; solches hieß ich sie unter
dem Galgen vorlesen. Und obgleich sie sie schon auf der Leiter
hatten, bereit gehenkt zu werden, mit einem Tuch über den Augen,
so wurde sie nun nicht hingerichtet, sondern sie führten sie wieder
zurück ins Gefängnis, und im Gefängnis kam sie nachher dazu,
Gottes ewige Wahrheit zu erkennen.
Es war noch ein anderer Gefangener mit mir, ein schlechter,
gottloser Mensch, ein bertichtigter Schwarzkünstler und Zauberer.
Er drohte, was er alles zu mir sagen und mir tun wolle, aber
er hatte keine Macht, den Mund gegen mich aufzutun. Einmal
gerieten der Kerkermeister und er aneinander und er drohte, er
wolle den Teufel rufen und das Haus niederreißen, so daß der
Kerkermeister Angst bekam. Da trieb mich der Herr hinzugehen
und ihm Einhalt zu gebieten und zu sagen: ,,Komm, laß sehen
was du kannst, tue dein Außerstes«. Jch sagte ihm, der Teufel
sei schon in ihm selber bei uns, die Kraft des Herrn binde ihn
aber. Da schlich er sich davon.
Als nun die Zeit der Schlacht von Worcester kam, sandte
der Richter Vennet Konstabler, um mich zu zwingen, Soldat zu
werden, da er gesehen hatte, daß ich kein Kommando übernehmen
würde. Jch sagte ihnen, ich sei ganz gegen allen äußeren Krieg.
Sie kamen wieder, um mir Werbegeld zu geben, aber ich nahm
es nicht. Daraus wurde ich vor den Wachtmeister Holes gebracht,
der mich eine Weile behielt und dann wieder zurückschickte. Nach
einiger Zeit wurde ich wieder heraufgeholt und vor den Kommifsär
gebracht, welcher erklärte, ich müsse als Soldat gehen, aber ich

