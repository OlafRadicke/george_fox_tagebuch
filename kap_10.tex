%%%%%%%%%%%%%%%%%%% Kapitel 10. %%%%%%%%%%%%%%%%%%%%%%%%%%%%%%

\chapter[Warnung an die Kegelspieler. Naylors Fall.]{Warnung an die Kegelspieler. Naylors Fall.}

\begin{center}
\textbf{Warnung an die Kegelspieler. Naylors Fall. Disput mit Paul
Gwin. Besuch bei Cromwell. Herumreisen bei den gefangenen
Freunden. Reise in Wales.}
\end{center}


Alß ich während meiner Gefangenschaft sah, wie sie sich in
Castle-Green mit Kegelschieben oergnügten, hatte ich eine Schrist
geschrieben, worin eß hieß:
,,Gs gehet daß Wort des Herrn an euch, ihr eitlen Müßig-
gänger, die ihr so dem Spiel, dem Vergnügen und solchen ein-
fältigen Übungen zugetan seid, daß ihr bedenken möget, was ihr
tut. Jst das der Zweck eures Daseins? Machte Gott alleß zu
eurem Vergnügen und eurem Gebrauch? Machte nicht Gott alle
Dinge, damit er darin in Furcht und Anbetung, im Geist und in
der Wahrheit, in Gerechtigkeit und Heiligkeit geehrt werde? Wie
könnet ihr Gott dienen, solange ihr euren Vergnügen nachgeht?
Ihr könnet nicht Gott dienen und den weltlichen Vergnügen, dem
Kegeln, Jagen und Trinken und dergleichen; wenn euer Herz bei
derartigem ist, so will Gott eure Lippen nicht, fraget euch, ob daß
nicht wahr ist ..... «


% \picinclude{./120-129/p_s121.jpg} 
Warnung an die Kegclspieler. Naylorß Fall usw. 121
An die Kegler im Castle-Green, geschrieben im Kerker zu Launceston.
Als wir nun frei waren, .... zogen wir . . . . über Laun-
ceston . . . Okington . . . nach Gxeter, wo viele Freunde ge-
fangen waren, unter anderem Jameö Naylor. Kurz ehe wir
srei wurden, hatte James Naylor sich in phantastische Jdeen
verirrt und viele mit ihm, was eine große Verwirrung im Land
anrichtete. Er kam nach Bristol und stiftete dort Unruhe; von
da wollte er nach Launreston gehen, um mich zu besuchen; aber
unterwegö wurde er angehalten und in Exeter gefangen gesetzt,
sowie verschiedene Andere; einer davon, ein ehrlicher, gottseliger
Mensch, starb in der Gefangenschaft; sein Blut kommt aus seine
Verfolger.
Am Abend, alß- wir nach Exeter kamen, redete ich mit Jame-3
Naylor, denn ich sah, daß er ganz in Jrrtum geraten war, sowie
auch seine Genossen. Am folgenden Tag — ez war der Erste
Tag — besuchten wir die Gefangenen und hatten im Gefängniß
eine Versammlung mit ihnen; aber James Naylor und einige
von ihnen konnten es in der Versammlung nicht aushalten. GZ
kam ein Kavallerie-Korporal in die Versammlung; er wurde ge-
wonnen und blieb ein sehr guter Freund.
Am folgenden Tag redete ich wieder mit Jameß Nahlor; er
machte herunter, waß ich ihm sagte, und war verwirrt und ver-
, dreht, dennoch wollte er gerne kommen und mich küssen. Aber
ich sagte, ,,weil er sich der Kraft Gotteß widersetze, so könne ich
seine Freundlichkeitöbezeugungen nicht annehmen«; der Herr trieb
mich, ihn zu verweisen und ihn unter die Kraft dez Herrn zu
stellen. So war nun, nachdem ich gegen die Welt gekämpft, unter
den Freunden ein böser Geist erwacht, gegen den man kämpfen
mußte. Jch ermahnte ihn und seine Genossen. Alö er nach
London kam, wurde ihm sein Widerstand gegen Gottes Kraft in
mir und gegen die ihm durch mich verkündete Wahrheit zur
größten Last. Aber er kam dazu, seine Abirrung einzusehen und
zu verdammen, und nach einiger Zeit kehrte er sich der Wahrheit
wieder zu, wie man in den gedruckten Berichten seiner Buße,
Veroerurteilung und Wiedererhebung ausführlich sieht .....
Von Exeter gingen wir .... zu E. Pyot in Bristol. AM
Morgen deö Ersten Tages ging ich zu der Versammlung in
Broadmead; sie war zahlreich und ruhig. ES wurde eine zVer-
sammlung angezeigt auf den Nachmittag im Garten. EZ war


% \picinclude{./120-129/p_s122.jpg} 
ein ungebildeter, unverschämter Baptist in Bristol, namenß Paul
Gwin, der schon früher in unseren Versammlungen große Störungen
verursacht hatte, ermutigt und angetrieben durch den Bürger-
meister, welcher, wie gesagt wurde, ihm sogar manchmal ein
Mittagessen gab, um ihn zu ermutigen. Er war von einer
solchen Pöbelmenge gefolgt, daß die Zahl derer, die zu unserer
Versammlung im Freien kamen, oft auf 10000 geschätzt wurde.
A18 ich auf dem Wege nach dem Garten war, sagte man mir,
daß Paul Gwin, der zänkische Baptist, zur Versammlung kommen
würde. Jch sagte, man solle sich nicht darum kümmern, ez sei
mir einerlei, wer käme. Jin Garten angekommen, stieg ich auf
einen Stein, auf den die Freunde zu stehen pflegten, wenn sie
sprachen; und der Herr trieb mich, den Hut abzunehmen und so
geraume Zeit zu stehen und mich von den Leuten ansehen zu
lassen; ez waren einige 1000 Leute da. Als ich nun so schweigend
daftand, fing jener Baptist an, mein Haar zu tadeln, aber ich
sagte nichtö zu ihm. Da brach er in einen Wortschwall aus und
rief: ,,Jhr Weisen von Bristol, ich staune über euch, daß ihr hier
steht, um Ginen etwas sagen und behaupten zu hören, der es
nicht beweisen kann«. Da öffnete der Herr meinen Mund, (bis
dahin hatte ich noch nichtö geredet,) und ich fragte die Leute, ob
sie mich je hätten reden hören oder je zuvor gesehen hätten? und
ich hieß sie, nicht zu vergessen, waö für eine Sorte von Mensch
der sei, der so frech sage, daß ich rede und behaupte, wa-3 ich
nicht beweisen könne, da doch weder er noch sie mich je zuvor
gesehen hätten. Darum sei eß ein lügnerischer, bößwilliger und
schlechter Geist, der au-? ihm rede; er sei vom Teufel und nicht
von Gott. Jch gebot ihm, bei der Furcht und Kraft Gotteö zu
schweigen, und die mächtige Kraft Gotteß kam über ihn und alle
seine Anhänger. Darauf hatten wir eine herrliche, friedliche Ver-
sammlung, und datz Wort dez Leben; ward unter ihnen verkündet
und sie kehrten sich von der Finsterniß zum Licht, zu Jesuö
Christus, dem Heiland. IDie Schrift wurde ihnen reichlich ge-
öffnet und die menschlichen Überlieferungen, Zutaten, Mittel und
Lehren darin nachgewiesen; sie wurden auf das Licht Christi hin-
gewiesen, durch daß sie solches- allez erkennen können, sowie auch
Christum selbst, damit er sie erlöse. IJch erklärte ihnen auch die
Zeichen und Sinnbilder von Ehristuö in den Zeiten dez Gesetzes
und zeigte ihnen, daß Ehristuö gekommen war, den Zeichen,


% \picinclude{./120-129/p_s123.jpg} 
Warnung an die Kegelspieler. Naylors Fall usw. 123
Zehnten und Eiden ein Ende zu machen und das Schwören ab-
zuschaffen, und einsetzte, daß man bei ,,ja« und »nein« bleibe,
und daß man umsonst predige, denn er wolle nun sein Volk
selber lehren und sein herrlicher »Ausgang aus der Höhe sei mm
erschienen« (Luk. 1,78). Mehrere Stunden verkündete ich das
Wort des Lebens unter ihnen und die ewige Kraft Gottes, durch
die sie möchten zu dem, der von Anfang war, zurückkehren und
mit ihm oersöhnt werden (2 Cor. 5). Und nachdem ich sie an den
Geist Gottes in ihrem Jnnern gewiesen, der sie in alle Wahrheit
leiten würde (Joh. 16), trieb es mich zu beten in der mächtigen
Kraft Gottes; des Herm Kraft kam über alle. Als ich geendet,
fing jener Kerl aufs Neue an zu schwaizen. John Audland
wurde getrieben, ihn zur Buße und Furcht Gottes zu vermahnen.
Da nun seine eigenen Leute und Anhänger sich seiner schämten,
ging er fort und hat nie mehr eine unserer Versammlungen ge-
stört. Die Versammlung ging ruhig auseinander und des Herm
Kraft und Herrlichkeit leuchtete über allen; es war ein gesegneter
Tag; die Ehre war des Herrn. Einige Zeit darauf ging dieser
Paul Gwin über Meer; Viele Jahre nachher begegnete ich ihm
wieder in Varbadoes .....
Von Kingston ritten wir nach London. Als wir in die Nähe
des HydePark kamen, sahen wir eine große Volks-menge, und bei
näherem Zusehen erblickten wir den Protektor, der in seinem
Wagen daherkam. Jch ritt an die Seite seines Wagens. Einige
seiner Leibgarde suchten mich wegzutreiben, aber er wehrte es
ihnen. So ritt ich neben ihm her und verkündete, was der Herr
mir eingab über seinen Zustand und über die Not der Freunde
im Lande; ich zeigte ihm, wie sehr diese Verfolgungen Christus
und seinen Aposteln und dem Christentum zuwider seien. Als wir
am Tor des James Park ankamen, verließ ich ihn; ehe wir uns
trennten, sagte er noch, ich solle zu ihm nach Hause kommen.
Am folgenden Tag kam eine Magd seiner Frau zu mir in meine
Wohnung und erzählte mir, ihr Herr sei zu ihr gekommen und
habe gesagt, er wolle ihr eine frohe Nachricht mitteilen. Als sie
ihn fragte, was für eine, sagte er ihr: George Fox sei in die
Stadt gekommen. Sie habe geantwortet, das sei in der Tat eine
gute Nachricht (denn sie hatte die Wahrheit angenommen), aber sie
habe es kaum glauben können, bis er ihr gesagt habe, wie ich ihn
getroffen und mit ihm von Hyde Park bis James Park geritten sei.


% \picinclude{./120-129/p_s124.jpg} 
Nach einiger Zeit gingen Edward Pyot und ich nach White-
hall, und als wir vor den Protektor kamen, war Or. Owen, Vize-
kanzler von Oxford, bei ihm. Es trieb uns, Oliver Eromwell die
Not der Freunde vorzustellen.   wiesen ihn auf das Licht
Jesu Christi, das jeden, der in die Welt kommt, erleuchtet.
Er sagte, es sei ein natürliches Licht, aber wir bewiesen ihm das
Gegenteil und legten dar, wie es göttlich und geistig sei, da es
von Christus ausgehe, dem geistigen und himmlischen Menschen
und daß eben das, was das ,,Leben in Ehristus« genannt werde,
das nenne man auch das ,,Licht in uns-.« ,Die Kraft des Herrn
ging aus in mir und trieb mich, ihn zu ermahnen, seine Krone
zu den Füßen Jesu niederzulegenst) Wiederholt redete ich mit
ihm in dieser Absicht. Zuletzt — ich stand neben dem Tisch ——
kam er und setzte sich auf die Tischecke neben mich, sagte, er wolle
so hoch sein wie ich und fuhr fort gegen das Licht Jesu Christi
zu sprechen und ging gleichgültig hinaus. Aber des Herren Macht
kam über ihn, sodaß, als er zu seiner Frau und zu andern
Leuten kam, er sagte: ,,nie habe ich sie in der Weise verlassen«;
denn er war in sich selbst gerichtet .....
Als er fort war, trafen wir beim Hinausgehen mit vielen
angesehenen Leuten zusammen, und einer von ihnen sing an, uns
gegen das Licht und die Wahrheit zu reden, und ich verachtete
ihn deswegen. Da sagte mir ein anderer, er sei der General-
Major von Northamptonshire. ,,Was!«, sagte ich, ,,unser früherer
Verfolger, der so viele unsrer Freunde in die Gefangenschaft ge-
schickt hat und eine Schande ist für die Christenheit und die
Religion? Jch bin froh, sdaß ich dich getroffen habe.« Und ich
redete nun ernstlich mit ihm über sein unchristliches Benehmen;
mid er schlich hinweg, denn er hatte die Verfolgungen in North-
hamptonshire sehr grausam betrieben gehabt .....
Von London ging ich nach Buekinghamshire, . . dann nach
Huntingdonshire, . . . Boston, . . . Edgehill, . . .Warwick . . .
und wieder zurück nach London. Überall hatte ich mich beflissen,
das, was mir der Herr aufgetragen, zu erfüllen. Denn, nachdem
ich aus der Gefangenschaft in Launceston entlassen war, hatte mich
der Herr getrieben im Lande umher zu reisen, wo die Wahrheit
sich verbreitet und recht befestigt hatte, um noch allerlei Ein-
lt Cromwell lehnte 1656 den Königötitel ab.


% \picinclude{./120-129/p_s125.jpg} 
Warnung an die Kegelspieler. Naylorö Fall usw. 125
wände zu beseitigen, welche die bösen Priester und ,,Frommen«
in den Gemütern gegen uns gepflanzt hatten ..... Und in
dieser Absicht trieb es mich nun auch, allerlei Erklärungen ergehen
zu lassen, .... unter anderm folgende: ,,Eß wird den Quäkern
oft vorgeworfen, daß sie das sogenannte Sakrament von Brot und
Wein bestreiten, von dem es heißt, man müsse ez gebrauchen zum
Gedächtnis Christi (Luc. 22,19) biö an der Welt Ende. Wir
hatten dez-wegen und wegen der verschiedenen Arten des- Sakra-
mentö-Gebrauchö im sogenannten Christentum viel Mühe mit den
Priestern und ,,Frommen«, denn manche nehmen es knieend,
manche sitzend; aber keine von allen, die ich je gesehen, nehmen
etz, wie die Jünger etz nahmen, nämlich in einem Zimmer nach
dem Nachtessen, sondern die meisten nehmen es vor dem Mittag-
essen und manche sagen, wemi der Priester Brot und Wein ge-
segnet hat, »eS ist der Leib Christi«. Christuß aber sagte nur, g
»tut etz zu meinem GedächtniZ«. Er sagte ihnen nicht, wie oft
sie es tun müßten oder wie lang; auch gebot er ihnen nicht, eß
ihr Lebenlang zu tun, noch daß alle, die an ihn glauben, es
tun sollten biz an der Welt Ende. Der Apostel Paulus, der erst
nach Christi Tod bekehrt worden, sagt den Corinthern, er habe vom
Herrn empfangen, was er ihnen in dieser Sache mitteile, und er
führt Christi Worte in bezug aus den Kelch also an: ,,DieseS
tut, so ost ihr trinket, zu meinem Gedächtni-8«; und er selbst fügt
bei: ,,denn so ost ihr daß Brot esset und den Kelch trinket, so
verkündet ihr dez Herrn Tod, bis daß er kommt« (1. Cor. 11,26).
Nach dem also, was der Apostel hier mitteilt, gebot weder Christuö
noch er, dietz allezeit zu tun, sondern stellten etz jedem frei. . .
Die Juden pflegten einen Kelch zu gebrauchen und Brot zu
brechen und an ihren Festen unter sich zu verteilen, wie man an
den jüdischen Altertümern sieht; datz Brechen des Broteö und daß
Trinken des Weineö waren also jüdische Gebräuche, die nicht sitr
immer zu bestehen brauchen. Sie tauften auch mit Wasser; da-
rum besremdete ez sie nicht, als- Johannes der Täufer auftrat
mit seiner Wassertause ..... sWaZ aber Brot und Wein an-
belangt, so hatte Christa?. gesagt, daß er das- Brot dez Lebenß
sei (Joh. 6,48), daß vom Himmel kommt, Amd daß er kommen
wolle und in ihnen wohnen. Daß betrachteten die Apostel nun
alß erfüllt und ermahnten die andern, nach dem zu trachten, das
von oben kommt (Col. 3,2). Jhr nun, die ihr diesen äußern Wein


% \picinclude{./120-129/p_s126.jpg} 
trinket und dieseß äußere Brot esset zum Gedächtnis des Todez
Christi, kennet ihr gar nichtö Bessereö, um dem Tode Christi näher
zu kommen? ....
ES muß freilich durch manchen Zustand hindurch gehen, ehe
die Leute dazu gelangen, das, waö von oben kommt, zu sehen
und daran teilzunehmen. Zuerst kommt der Gebrauch dee äußeren
Broteß und Weines zum Gedächtnis Christi; daö war zeitlich und
nicht gezwungen, sondern freiwillig ..... Zweitens kommt da-3
Eingehen in seinen Tod, ein Leiden mit Christuß, und daß ist
notwendig zum Heil und nicht zeitlich, sondern beständig; etz muß
ein täglichez Sterben sein. Drittens ein Begrabensein mit Ehriftuö;
oiertens ein Auferstehen mit Christuß; fünstenß nach dem Aufer-
standensein mit Ehristuß (Röm. 6) ein Trachten nach dem, wa-3
droben ist, ein Suchen nach dem Brote, das vom Himmel her-
unter kommt, ein Essen davon und eine Gemeinschast durch daß-
selbe. .... Die Gemeinschaft, die sich aus den Gebrauch von
Brot, Wein, Wasser, Beschneidung, äußere Tempel und sichtbare
Dinge gründet, wird ein Ende haben; die Gemeinschaft aber, die
sich auf daß Evangelium gründet, auf die Kraft Gotteö, die war,
ehe der Teufel gewesen, und die Leben und unoergänglicheö Wesen
anö Licht bringt, und durch welche die Leute über den Teufel
sehen, der sie versinstert, diese Gemeinschaft wird ewig be-
stehen« .... .
Somit waren die Einwände dieser Pticftet und ,,Frommen«
widerlegt .... und die Wahrheit breitete sich in diesem Jahre
(1656) recht auö, und viele Tausende bekehtten sich zum Herm,
sodaß selten weniger alö tausend im Gefängnis waren um der
Wahrheit willen, etliche wegen des Zehntenwesenß, etliche
weil sie ins Turmhauö gegangen waren, etliche wegen irgend-
welcher sogenannten Mißachtung, etliche wegen deß Schwörenz
oder weil sie ihre Hüte nicht abgenommen .....
Von London zogen wir wieder weiter im Lande umher ....
nach Farnham . . . Bafmgstoke . . . Exeter . . . Bristol, . . . dann
nach Brecknock (Waleö) und dann . . . wieder nach England zu-
rück, nach Shrewöburh .....
E6 war um die Zeit eine große Trockenheit im Lande ....
A13 nun Oliver Csromwell ein Fasten proklamierte um Regen,
nahm man wahr, daß im Norden, soweit sich die Wahrheit auß-
gebreitet hatte, erquickende Niederschläge waren, während sie im


% \picinclude{./120-129/p_s127.jpg} 
Warnung an die Kegclspieler. Noylote Fall usw. 127
Süden vielerortö schier umkanten auö Mangel an Regen. Da
trieb etz mich, eine E-rwidemng auf die Proklamation dez Pro-
tektorö zu schreiben, worin ich ihm sagte, wenn er sich Gottes
Wahrheit zugewendet hätte, so hätte er Regen gehabt, die Trocken-
heit sei ein Zeichen für ihre Dürre und ihren Mangel an Wasser
des Lebenß .....
Wir gingen wieder nach Wales und hatten mehrere Ver-
sammlungen, bie wir nach Tenby kamen. Auf der Straße kam
mir ein Frieden?-richter entgegen und forderte mich auf, zu ihm
in seine Wohnung zu kommen, maß ich denn auch tat. Am
Ersten Tage kam der Stadtmajor und einige der Häupter der
Stadt und blieben während der ganzen Zeit der Versammlung.
John-ap-John aber verließ sie und ging ins Turmhauß, wo ihn
der Gouveneur gefangen nehmen ließ. GS war eine herrliche
Versammlung. Am Morgen dez zweiten Tagetz schickte der Gou-
verneur einen seiner Leute inö Haus dez Friedenßrichterö, um mich
holen zu lassen, wae dem Major und dem Friedenörichter, die
beide bei mir waren, sehr leid tat. Sie gingen darum gleich
vorauö zum Gouverneur, und nach einiger Zeit kam ich ihnen nach
mit dem Beamten. Alö ich eintrat, sagte ich: ,,Friede sei diesem
Hause.« Und ehe noch der Gouverneur mich etwaz fragen konnte,
fragte ich ihn, warum er meinen Freund John-ap-John ine Ge-
fängnis getan habe? »Weil er seinen Hut in der Kirche aufbe-
hielt«, erwiderte der Gouverneur. Ich sagte darauf: ,,Hatte nicht
der Priester zwei Kappen auf dem Kopf. eine weiße und eine
schwarze? Schneide meinem Freund den Rand an seinem Hut
weg, und dann hätte er nur eine solche Kappe, und der Rand
ist nur, um ihn vor dem Regen zu schützen.« ,,DaZ sind dumme
Sachen«, sagte der Gouverneur. ,,Warutn wirfst du dann meinen
Freund ins Gefängniß wegen dummer Sach en ?« sagte ich. E-r fragte
mich nun, ob ich die Grwählimg und Verwerfung annehme?
,,Ja«, sagte ich, ,,und du bist in der Verwerfrmg««. Daß machte
ihn bös und er drohte mir, er wolle mich inö Gefängnis werken,
bis ichö ihm beweise; ich sagte, ich könne ihm daß gleich beweisen,
wenn er der Wahrheit Gehör schenke.-i Nun fragte ich ihn,. ob denn
Wut und Zorn nicht Zeichen der Verwerfung seien? denn der,
welcher auö dem Fleisch geboren sei, verfolge den, der auö dem
Geist geboren sei. Christuö und seine Jünger haben nie jemand
verfolgt oder gefangen genommen. 7,* Daraufhin bekannte er osfen,


% \picinclude{./120-129/p_s128.jpg} 
daß er zu leidenschaftlich und zommütig sei. Ich sagte ihm, er
sei wie Gsau, der Grstgeborene, und nicht wie Jakob .... Die
Kraft dez Herrn kam so mächtig über ihn, daß er sich zur Wahr-
heit bekannte, und der Friedenzrichter kam und schüttelte mir
die Hand ..... Als ich weiter zog, trieb es mich, noch einmal
mit dem Gouverneur zu reden und er lud mich zum Essen ein
und gab meinem Freund die Freiheit ..,.. Wir gingen nach
England zurück . . . nach Liverpool . . . Manchester . . . Lan-
caster . . . nach Swarthmore, wo die Freunde sich sehr freuten,
mich wieder zu sehen. Jch blieb während zwei Ersten Tagen
dort und ging in mehrere Versammlungen. Die Freunde freuten
sich mit mir der Güte Gotteß, die mir durch so viele Gefahren
hindurch geholfen. Jhm sei Preis ewiglich.
