% \picinclude{./180-189/p_s180.jpg} 
das es herein regnete, und wo es schrecklich rauchte, was mir
sehr schadete. Eines Tages besuchte mich der Gouverneur Sir
John Crossland\person{Crossland, Sir John} mit 
Sir Francis Cobb\person{Cobb, Sir Francis}. Ich 
bat den Gouverneur, mich in mein Zimmer zu begleiten, 
um zu sehen, was das für
ein Ort sei. Ich hatte ein kleines Feuer darin angezündet, welches
nun derart rauchte, das man seinen Weg schier nicht fand. Da
der Gouverneur ein Papist war, so sagte ich ihm, es sei sein
Fegefeuer, das sie mir zum Aufenthalt gegeben hätten. Ich musste
etwa 50 Schilling ausgeben\footnote{Die Gefangenen hatten 
zu der Zeit die Kosten ihres Aufenthaltes in
den Gefängnissen selbst zu tragen (s. Aschrott, Engl. 
Gefängniswesen).}, um den Regen abzuhalten und zu
machen, das es nicht so stark rauchte. Und als ich diese 
ausgaben gemacht hatte, und es etwas erträglicher geworden, gaben
sie mir ein noch schlechteres Gelaß, wo ich weder ein Kamin noch
irgend eine andere Vorrichtung, um Feuer zu machen, hatte. Da
es gegen die See gelegen und sehr offen war, so trieb der Wind
den Regen ungehindert herein, so das dass Wasser bis zu meinem
Bett kam und im Zimmer herumlief, und ich es mit einem Gefäß 
ausschöpfen musste. Und wenn meine Kleider nass waren, so
hatte ich kein Feuer, um sie zu trocknen, so das mein Körper ganz
erstarrt war vor Kälte, und meine Finger so geschwollen waren,
das einer so groß war wie sonst zwei. Obgleich ich in diesem
Raum auch zu bezahlen hatte, so gelang es mir doch nicht, Wind
und Regen abzuhalten [...]

Es wurde den Freunden nicht gestattet, mich zu besuchen;
aber sonst führten sie hier und da jemanden zu mir, entweder
um mich anzusehen, oder um sich mit mir zu unterreden. Einmal
kam eine Schar Papisten, um mit mir zu disputieren; sie 
behaupteten, der Papst sei unfehlbar\index{Unfehlbarkeit} 
und sei immer unfehlbar gewesen
seit Petrus Zeit, aber ich bewies ihnen das Gegenteil aus der
Geschichte: ein Bischof von Rom, Marcellinuß mit Namen, habe
den Glauben abgeschworen und den Götzenbildern gehuldigt, dieser
sei also nicht unfehlbar gewesen. Ich sagte ihnen, wenn sie den
unfehlbaren Geist hätten, so bedürften sie keiner Kerker, Schwerter,
Foltern, Scheiterhaufen, Geißeln und Galgen, um ihre Religion
aufrecht zu erhalten, denn wenn sie den unfehlbaren Geist hätten,
so würden sie die Leben der Menschen schützen, statt sie umzubringen, 
und würden in Sachen der Religion nur geistliche Waffen\index{Geistliche Waffen}
% \picinclude{./180-189/p_s181.jpg} 
brauchen. Ich erzählte ihnen auch, was einer der Ihrigen mir
berichtet hatte: eine in Kent lebende Frau war nicht nur selber
Papistin gewesen, sondern hatte auch viele andere für ihren
Glauben gewonnen. Aber als sie zur Wahrheit des Herrn bekehrt 
wurde und durch sie zu Jesus Christus ihrem Heiland kam,
ermahnte sie die Papisten, ein gleiches zu tun, unter anderem
auch einen Schneider, der bei ihr in Arbeit war; sie zeigte ihm
die Verkehrtheit der pästlichen Religion und suchte ihn für die
Wahrheit zu gewinnen, da zog er sein Messer und stellte sich
zwischen sie und die Türe, aber sie trat ihm mutig entgegen und
ermahnte ihn, sein Messer weg zu tun, denn sie kannte seine
Grundsätze; auf die Frage, was er wohl mit dem Messer gemacht 
hätte, antwortete die Frau: "`er hätte mich erstochen"', und
auf die weitere Frage, ob er dies wegen ihrer Religion getan
hätte, erwiderte sie: "`ja, denn es ist der Grundsatz der Papisten,
jeden, der von ihrer Religion abtrünnig wird, womöglich zu töten"'.
Dieses erzählte ich nun den Papisten und fügte bei, ich hätte es
von jemand, der früher zu ihnen gehört, sich jedoch von ihnen
gewandt habe, weil er hinter ihre Handlungsweise gekommen war.
Sie leugneten nicht, das sie solche Grundsätze hätten, fragten
aber, ob ich nun solches weitererzählen werde? Ich erwiderte:
"`ja, denn solche Dinge müssen weitererzählt werden, damit man
erfährt, wie sehr eure Religion vom wahren Christentum 
abweicht"'.\index{Öffentliche Kritik an Anderen}
Darauf gingen sie sehr zornig fort. Ein anderer Papist, welcher
kam, um mit mir zu disputieren, behauptete, alle Patriarchen seien
in der Hölle gewesen, bis Christus zur Hölle\index{Hölle} hinabgefahren sei,
da habe der Teufel\index{Teufel} gesagt: "`was kommst du hierher, unsere sichere
Burg zu sprengen?"' und Christus habe geantwortet, er komme,
um alle diese zu befreien, und sei drei Tage und drei Nächte in
der Hölle gewesen, um sie alle zu befreien. Ich erwiderte ihm,
das sei unrichtig, denn Christus habe ja zum Schächer gesagt:
"`heute noch sollst du mit mir im Paradiese sein."' Und Henoch
und Elias seien in den Himmel gekommen, auch Abraham, denn
es heiße, Lazarus sei in Abrahams Schoß gewesen, und Moses
und Elias seien mit Jesus auf dem Berg gewesen, ehe er leiden
musste. Diese Beispiele stopften dem Papisten den Mund und
brachten ihn in Verlegenheit.

Ein andermal kam Doktor Witty\person{Doktor Witty}, ein berühmter Arzt, mit
Lord Falconbridge; mit ihnen kam auch der Gouverneur der
% \picinclude{./180-189/p_s182.jpg} 
Festung Tynemouth\ort{Festung Tynemouth} und mehrere Adlige. Als ich zu 
ihnen gerufen wurde, sing Witty ein Gespräch mit mir an und fragte mich,
warum ich im Gefängnis. sei. Ich antwortete: "`weil ich den Geboten 
Christi nicht ungehorsam sein will"'. Er sagte, ich hätte
dem König den Treueid leisten sollen. Da er ein eifriger 
Preßbyterianer\index{Preßbyterianer} war, so fragte ich ihn, 
ob er denn nicht zuerst gegen
den König und das Unterhaus geschworen und sich zum
schottischen Covenant bekannt habe und seither wieder zum König
geschworen habe? was denn dann das Schwören nütze? Mein
Huldigungseid, fügte ich bei, bestehe eben nicht im Schwören,
sondern in Wahrheit und Treue. Nach einigem Hin- und Herreden 
wurde ich wieder in meine Zelle zurückgeschickt; nachher
prahlte dieser Arzt bei seinen Patienten in der Stadt herum, er
habe mich besiegt. Als ich von seinem Prahlen hörte, sagte ich
dem Gouverneur, es sei ein geringer Ruhm zu sagen, man habe
einen Gefangenen besiegt. Ich bat, man solle ihm sagen seinen
Besuch zu wiederholen, wenn er wieder ins Schloss komme. Er
kam nach einiger Zeit wieder mit sechzehn oder siebzehn 
angesehenen Leuten und erlitt eine noch größere Niederlage als das
erste mal; er behauptete nämlich, Christus habe nicht alle, die in
die Welt kommen, erleuchtet\index{Erleuchtung}, und die heilsame Gnade Gottes sei
nicht allen Menschen erschienen, und Christus sei nicht für alle
Menschen gestorben\index{Tod Christus}. Ich fragte ihn, was das für Menschen
seien, die Christus nicht erleuchtet habe, denen die heilsame Gnade
nicht erschienen sei, und für die er nicht gestorben sei?\index{Opfertod} Er sagte,
die Ehebrecher, die Götzendiener, die Gottlosen. Ich fragte ihn,
ob die Ehebrecher und Gottlosen keine Sünder seien? Er sagte,
doch. "`Und starb nicht Christus eben für die Sünder?"' fragte ich,
"`kam er nicht, die Sünder zur Buße zu rufen?"'\index{Sünde} Er sagte: "`doch"'.
"`Dann hast du dir selber das Maul gestopft"', sagte ich. Hiermit
hatte ich bewiesen, das die Gnade Gottes allen Menschen erschienen 
ist, obgleich viele sie in Mutwillen kehren und ihr widerstreben, 
und das Christus alle Menschen erleuchtet hat, wenn
schon Viele das Licht hassen. Manche der Anwesenden gaben zu,
das dies wahr sei, der Doktor aber ging fort und kam nie mehr
zu mir.

Ein andermal brachte der Gouverneur einen Priester zu
mir, aber sein Mund war bald gestopft. Bald darauf brachte
er zwei Parlamentsmitglieder, die mich fragten, Ob ich Prediger\index{Prediger}
% \picinclude{./180-189/p_s183.jpg} 
und Bischöfe\index{Bischöfe} gelten lasse. Ich erwiderte: "`ja, solche die Christus
sendet, die umsonst empfangen und umsonst geben, die dazu bestimmt 
sind und den Geist und die Kraft haben, welche auch die
Apostel hatten. Solche Bischöfe und Prediger aber wie eure, die
nichts tun, als was; ihnen ein gutes Einkommen bringt, die
lasse ich nicht gelten, denn sie sind nicht den Aposteln gleich.
Christus sagte zu seinen Jüngern: "`Gehet hin in alle Welt und
predigt das Evangelium umsonst."' Ihr Parlamentsmitglieder,
die ihr euren Bischöfen und Predigern so große Pfründen gebt,
ihr habt sie verdorben. Meinet ihr etwa, diese gehen zu allen
Völkern? oder überhaupt über ihre fetten Pfründen hinaus, um
zu predigen? Urteilt selber, ob sie das tun oder nicht."'
Ein andermal kam die Witwe von Lord Fairfax\person{Lord Fairfax} und viele
mit ihr, unter anderem auch ein Priester. Es trieb mich, ihnen
die Wahrheit zu verkünden; der Priester fragte mich, warum wir
"`du"' und "`ihr"'\footnote{In der ursprünglichen Übersetzung 
wird "`dich"' an dieser Stelle geschrieben.} zu den Leuten sagen? 
denn er hielt uns für
Narren und Dummköpfe deswegen. Ich fragte ihn, ob er finde,
die, welche die Schrift übersetzten und die Grammatik und 
Sprachlehre machten, seien Narren und Dummköpfe gewesen, weil sie
sie so übersetzten und lehrten, das "`du"' für eine Person und
"`ihr"' für mehrere gilt? wenn denn diese Narren und Dummköpfe
gewesen seien, warum denn dann nicht er und die, welche seine
Ansicht teilen und sich für weise halten, die Grammatik und 
Sprachlehre und die Bibel verbessern, und die Mehrzahl statt der 
Einzahl setzen? Wenn es aber weise Männer gewesen seien, die die
Bibel übersetzten und die Sprachlehre und Grammatik machten,
so sollen sie sich fragen, ob nicht etwa sie die Narren und 
Dummköpfe seien, die nicht reden wie die Bibel und die Grammatik
lehre, sondern uns -- die es tun -- darum schelten? So war dem
Priester der Mund gestopft\index{Den Mund stopfen}, und viele wurden 
von der Wahrheit überzeugt und waren recht empfänglich und zugänglich. 
Einige boten mir Geld an, aber ich nahm es nicht.

Hierauf kam Doktor Cradock mit drei weiteren Priestern,
dem Gouverneur und seiner Frau, einer "`Dame"' (lachs) wie
man zu sagen pflegt, und einer andern "`Dame"' und eine ganze
Schar mit ihnen. Doktor Cradock fragte mich, warum ich im
Gefängnis sei; ich antwortete: "`Weil ich den Geboten Christi und
der Apostel, nicht zu schwören, gehorche."' Wenn aber er, ein
Doktor und Friedensrichter, mir beweisen könne, das Christus
% \picinclude{./180-189/p_s184.jpg} 
oder der Apostel den Christen, nachdem er ihnen verboten hatte,
zu schwören, es ihnen nachher wieder zu tun befahl, so wolle
auch ich es tun. Ich Gab ihm die Bibel, damit er mir irgend
ein solches Gebot zeige, wenn er könne. Er sagte: "`Ihr sollt
ohne Heuchelei und heiliglich schwören (Jer. 4, 2)"'. "`Ja, ja,"' sagte
ich, "`so hieß es zu Jeremias Zeiten, aber das war lange bevor
Christus befahl: ihr sollt überhaupt nicht schwören (Matth. 5, 34).
Aus dem alten Testament könnte ich ebenso viele Beispiele oder vielleicht
noch mehr bringen, aber was nützen sie für den Beweis, das dass
Schwören auch im neuen Testament erlaubt war, nachdem Christus
und die Apostel es verboten? Übrigens: zu wem wird dort gesagt, 
sie sollten nicht schwören? zu den Heiden oder zu den Juden?"'
Hierauf gab er keine Antwort. Aber einer der Priester sagte:
"`zu den Juden,"' und Doktor Cradock gab es zu. "`Gut,"' sagte
ich, "`aber wo hat Gott je den Heiden ein Gebot gegeben zu
schwören? und ihr wisset ja, das wir von Natur Heiden sind."'
"`Allerdings,"' sagte Doktor Cradock; "`zwar zur Zeit des 
Evangeliums musste alles aus zweier oder dreier Zeugen Mund bestätigt
werden, aber geschworen wurde nicht"'. "`Warum also,"' fragte ich,
"`zwingst du den Christen Eide ab gegen dein besseres Wissen?
und warum exkommunizierst du die Freunde?"' (er hatte nämlich
viele sowohl in York als auch in Lancashire exkommuniziert). Er
sagte: "`weil sie nicht in die Kirche kamen."' "`So!"' sagte ich,
"`vor mehr als zwanzig Jahren, als wir noch Knaben und Mädchen
waren, da überließet ihr uns den Presbyterianern, den Independenten 
und Baptisten, und viele von diesen nahmen uns Hab und
Gut und verfolgten uns, weil wir uns ihnen nicht anschließen
wollten; damals waren wir noch jung und wussten wenig
von euren Ansichten; hättet ihr nun die alten Leute, denen sie
bekannt waren, bei euch behalten und eure Ansichten in Kraft
erhalten wollen, so hättet ihr sollen entweder euch nicht von uns
wenden, wie ihr getan, oder ihr hättet uns sollen eure Episteln,
Kollekten, Homilien und Abendliturgien senden, wie Paulus ja
auch den Heiligen geschrieben hatte, als er in der Gefangenschaft
von ihnen getrennt gewesen war. Wir hätten allesamt können
Türken oder Juden werden, was das, was wir in dieser Zeit von
euch empfigen, anbelangt; und nun habt ihr uns, alt und jung,
exkommuniziert, also aus eurer Kirche ausgestoßen, ehe ihr uns für
dieselbe gewonnen habt. Ist es nicht ein Unsinn, uns auszuweisen,
% \picinclude{./180-189/p_s185.jpg} 
ehe wir drin waren? Ja, wenn ihr uns für eure Kirche
gewonnen hättet und wir ihr angehört hätten und dann etwa
Unrechtes getan hätten, so wäre es einigermaßen begründet 
gewesen. Was nennst du übrigens "`Kirche?"'\index{Eklesiologi} "`Nun,"' sagte er,
"`das was du "`Turmhaus"' nennst."' Darauf fragte ich ihn, ob
denn Christus sein Blut für das \textit{Turmhaus} vergossen habe. "`Und,"'
sagte ich, "`wenn nun die Kirche die Braut Christi und Christus das 
Haupt der Kirche genannt wird, glaubst du denn, das \textit{Turmhaus}
sei die Braut Christi und er das Haupt dieses alten Gebäudes?
ist er nicht vielmehr das- Haupt der Gemeinde?"' (Eph. 5)\bibel{Eph. 05@Eph. 5}. 
"`Er ist das Haupt der Gemeinde,"' erwiderte er, "`und sie ist die
Kirche."' "`Ihr habt also den Namen Kirche, welcher der Gemeinde
zukommt, einem alten Hause gegeben,"' sagte ich, "`und habt die
Leute gelehrt, solches zu glauben!"' Weiter fragte ich ihn, warum
die Freunde verfolgt werden darum, das sie den Zehnten nicht
geben? Ob Gott je den Heiden geboten habe, den Zehnten zu
bezahlen? Ob Christus nicht die Zehnten aufgehoben habe, als
er das Levitische Priestettum\index{Levitische Priestettum}, 
das Zehnten nahm, aufhob?\index{Zehnten}\index{Kirchensteuer} Und
ob Christus, als er seine Jünger aussandte zu predigen, ihnen
nicht geboten habe, umsonst zu predigen? und ob nicht alle Diener
Christi verpflichtet seien, dieses Gebot zu halten? Er sagte, er
wolle hierüber nicht streiten; er schien überhaupt nicht gern bei
diesem Gegenstand zu verharren, sondern ging bald zu einem
andern über und sagte: "`Ihr verheiratet euch, aber man weiß
nicht, wie ihr dabei verfahrt."' Ich riet ihm, zu kommen und
selbst zu sehen.\index{Öffentlichkeit} Er drohte, uns seine Macht fühlen zu lassen; ich
riet ihm, zu bedenken, das er ein alter Mann sei, und fragte ihn,
wo er von der Genesis bis zur Offenbarung irgendwo lese, das
ein Priester jemand getraut habe; er solle mir ein solches Beispiel
zeigen, wenn er wolle, das wir zu ihnen kommen sollten, um
ums trauen zu lassen. "`Du hast ja,"' sagte ich, "`einen der Freunde
zwei Jahre nach seinem Tode noch exkommuniziert\index{Exkommunizieren} 
wegen seiner Ehe; warum exkommunizierst du nicht auch Jsaak, Jakob, Boaö und
Ruth? Warum machst du deine Macht nicht Lauch an diesen
geltend? Denn es steht nirgends, das sie von einem Priester
getraut worden seien, sondern sie nahmen einander in der 
Versammlung in Gegenwart Gottes und seiner Gemeinde; und so
tun wir. Wir haben also die heiligen Männer und Frauen der
Schrift auf unsrer Seite in dieser Sache."' Wir redeten lange
% \picinclude{./180-189/p_s186.jpg} 
hin und her; als er aber sah, das er nichts über mich vermochte,
ging er fort mit seinen Begleitern [...]

In diesem und dem vorhergehenden Jahre waren viele
Freunde gefangen genommen worden. Viele waren in London, in
Newgate und andern Gefängnissen, wo die Krankheit (Pest)\index{Pest} herrschte,
und starben dort. Viele wurden auch verbannt\index{Scheiterhaufen} und auf des
Königs Befehl auf Schiffe gebracht. Oft wollten die Schiffsherren
sie nicht aufnehmen und setzten sie wieder ans Land; doch gelangten
viele nach Barbados\ort{Barbados}, Jamaika\ort{Jamaika} und Nevis, 
und der Herr segnete sie dort [...]

Nachdem ich mehr als ein Jahr im Schlos zu Scarbro gefangen 
gewesen war, schickte ich einen Brief an den König, in
dem ich ihm von meiner Gefangenschaft berichtete und von der
schlechten Behandlung, die ich während derselben zu erdulden
hatte, und das man mir gesagt habe, niemand als er könne mich
frei machen. Und John Whitehead\person{Whitehead, John} 
begab sich zu Gsquire Marsh,
mit dem er befreundet war, um ihm von mir zu reden, und dieser
versprach, das, wenn John Whitehead einen Bericht über meine
Angelegenheit verfassen wolle, er denselben John Birkenhead, der
über die Begnadigungsgesuche zu entscheiden hatte, einhändigen
und sich um meine Freisprechung bemühen wolle. John
Whitehead und Gllis Hookes\person{Hookes, Gllis} 
verfassten nun einen Bericht über
meine Gefangennahme und meine Leiden während der Gefangenschaft 
und brachten ihn Marsh, der ihn John Birkenhead überbrachte 
und einen Befehl zu meiner Freisprechung erwirkte. In
demselben hieß es, das der König von glaubwürdiger Seite erfahren 
habe, ich sei stets gegen alles Komplottieren und Streiten
gewesen, und habe etwaige Verschwörungen eher entdecken helfen,
als das ich mich selber daran beteiligt hätte, und so sei es Sein
königliches Wohlgefallen, das ich aus meiner Gefangenschaft befreit
werde. Sobald dieser Befehl bekannt war, kam John Whitehead
damit nach Scarbro und übergab ihn dem Gouverneur, der nun
die betreffenden Behörden zusammen berief und ohne weitere
Bürgschaft für mein friedsames Leben, zufrieden mit der Erklärung,
das ich ein stiller Bürger sei, mich frei ließ [...]

Gleich am Tage nach meiner Freilassung brach das Feuer
in London aus, und das Gerücht davon verbreitete sich rasch im
Lande. Da sah ich, das Gott der Herr sein Wort wahr gemacht
hatte, das ins Gefängnis zu Lancaster zu mir geschehen war, als
% \picinclude{./180-189/p_s187.jpg} 
ich den Engel\index{Engel} des Herrn gesehen hatte, wie er mit einem 
leuchtenden Schwerte gen Süden zeigte, wie ich schon berichtet habe.\index{Vorhersehung}
Die Bewohner waren vor diesem Feuer gewarnt worden; aber
wenige hatten es geglaubt oder zu Herzen genommen, vielmehr
wurden sie noch schlechter und hochmütiger. Ein Freund war nämlich
getrieben worden, von Huntingdonshire\ort{Huntingdonshire} herunter zu kommen, kurz
vor der Feuersbrunst, und sein Geld herum zu streuen, sein Pferd
frei in den Straßen herum zuführen, die Kniebänder aufzulösen,
die Strümpfe herunter hängen zu lassen, das Wams aufzuknöpfen
und den Leuten zu sagen: "`so werdet ihr herum laufen und euer
Hab und Gut umherstreuen, halb nackt, wie Wahnsinnige; und so
geschah es, als die Stadt brannte. So machte der Herr seine
Propheten\index{Prophezeiung} und Diener zu Werkzeugen seiner Kraft und gab
ihnen Zeichen seines Gerichts und sandte sie, das Volk zu warnen;
aber statt Buße zu tun, haben sie sie misshandelt und etliche
gefangen genommen, unter der früheren Regierung sowohl als
jetzt; aber der Herr ist gerecht, wohl dem, der seinen Worten
gehorcht! Etliche trieb es, nackt in den Straßen umher zu laufen,\index{Nackt in der Öffentlichkeit}
um zu zeigen, wie Gott ihnen ihre heuchlerische Frömmigkeit
abreißabreißen werde und sie nackt und bloß machen werde. Aber
das Vodas Volk hatte, statt in sich zu gehen, diese oft gegeißelt oder
sonst mißhandelt oder gar gefangen genommen. Andere trieb es,
in Sticken umhetzugehen und die Rache und Strafe Gottes wegen
des großen Hochmutes zu verkünden; aber wenige gaben darauf
acht. In den Tagen der früheren Regierung machten die falschen,
frömmlerischen Priester mehrere Petitionen gegen uns an Oliver
und Richard, die sogenannten Protektoren, und an das Parlament
und die Richter und Räte, voller Lügen, Verleumdungen und
Schmähungen; aber wir verschafften uns Abschriften davon, und
mit Gottes Hilfe antworteten wir auf alle, und wuschen die
Wahrheit und uns rein. Aber o, welche Mächte der Finsternis erhoben
sich in denen, die zum Lügen ihre Zuflucht nahmen! aber der
Herr stürzte sie alle und schützte seine Lämmer durch seine Kraft
und Wahrheit, sein Licht und sein Leben, und deckte sie, wie mit
Adlers Flügeln. Solches gab uns Mut, aus ihn zu vertrauen,
der alle, die sich im Finstern gegen seine Wahrheit und sein Volk
verbünden, stürzt und vernichtet, und der durch diese Wahrheit 
seinem Volk Macht gibt, ihm in der Wahrheit zu dienen. [...]
% \picinclude{./180-189/p_s188.jpg} 

\chapter[Einrichtung der Monatsversammlungen]{Einrichtung der Monatsversammlungen}

\begin{center}
\textbf{Einrichtung der Monatsversammlungen. Regelung der Quäkerehen. 
Gründnng von Knaben- u. Mädchenschulen. Reformation des Quäkertums.}
\end{center}


Nachdem ich nun wieder frei war, zog ich wieder umher,
nach Whitby\ort{Whitby} [...] nach Oran [...] und zuletzt nach 
Marmaduke Storrs\ort{Marmaduke Storrs}. [...] wo ich eine große 
Versammlung hatte [...] Am Tage nach derselben sollten zwei von 
der Freunden sich zur Ehe nehmen,\index{Heirat} und es war 
deshalb eine sehr zahlreiche Versammlung, 
der ich beiwohnte. Es trieb mich, den Leuten unsern Standpunkt 
über die Eheschließung auseinanderzusetzen, indem ich
ihnen zeigte, wie man im Volke Gottes einander zur Ehe genommen 
hatte in der Versammlung der Ältesten und wie es Gott
gewesen, der Mann und Weib zusammengefügte vor dem Fall.
Nachher hätten dann zwar die Menschen sich selber zusammengetan; 
im Stand der Erlösung aber werde das Zusammenfügen
durch Gott als die richtige und ehrenhafte Verbindung angesehen,
und nie lesen wir von irgend einem Priester, von der Genesis
bis zur Offenbarung, das er je zwei zusammengab. Hierauf
redete ich ihnen von den Pflichten der Eheleute, wie sie beide Gott
dienen sollten, als gleichermaßen Erben des Lebens und der Gnade
(1. Petr. 3,7\bibel{Petr. 1. 03:7@1. Petr. 3:7}) [...] Dann besuchte ich die Freunde im Lande
umher, bis ich nach York\ort{York} kam, wo ich eine große Versammlung
hatte. Nach derselben besuchte ich Richter Robinson\person{Richter Robinson}, 
einen früheren Friedensrichter, der von Anfang an mir und den Freunden sehr
wohlgesinnt gewesen war. Es war ein Priester bei ihm, welcher
mir sagte, es heiße von uns, wir liebten niemanden als uns selber.
Ich erwiderte ihm, das wir alle Menschen lieben, als Gottes
Geschöpfe, die ja alle von Adam und Eva abstammen, und das
wir die Brüder lieben, durch den Heiligen Geist. Dies brachte
ihn zum Schweigen, und wir gingen schließlich in Frieden auseinander; 
danach reiste ich weiter.

Ich schrieb um diese Zeit ein Buch, betitelt: \textit{Fürchte Gott
und ehre den König}\index{Fürchte Gott
und ehre den König}. Ich zeigte darin, das niemand wahrhaft
Gott fürchten und den König ehren könne, der nicht mit der Sünde
und dem Bösen breche. Dieses Buch machte großen Eindruck
auf die Soldaten und auf viele andere Leute [...]
Nachdem ich viele Grafschaften durchzogen hatte, wo ich
Freunde besuchte und mit ihnen viele große und gesegnete 
% \picinclude{./180-189/p_s189.jpg} 
Versammlungen hatte, kam ich nach London\ort{London}. Aber ich fühlte mich
sehr schwach nach der beinahe dreijährigen harten und grausamen
Gefangenschaft; alle meine Gelenke und mein ganzer Körper waren
so steif und lahm, das ich fast mein Pferd nicht besteigen noch
mich bewegen konnte; auch konnte ich schier die Nähe eines Feuers
nicht ertragen oder den Genus von warmem Fleisch, nachdem ich
so lange beides entbehrt hatte. In London besuchte ich öfters
die Brandstätten und sah mich aufmerksam darin um. Ich sah,
das die Stadt so aussah, wie mir der Herr einige Jahre zuvor
geoffenbart hatte [...]\index{Vorhersehung}

Um diese Zeit erreichte die Kraft des Herrn etliche, welche
die Wahrheit verlassen und die Freunde angegriffen hatten; sie
strömte so herrlich hernieder, das sie ihre Schmähschriften verdammten 
und zum Teil zerrissen. Wir hatten etliche Versammlungen 
mit ihnen, und des Herrn Kraft war über allen und richtete
die Abtrünnigen. In diesen Versammlungen, welche ganze Tage
dauerten,\index{Versammlungslänge} kamen manche, welche 
mit John Perrot\person{Perrot, John} und andern
abgeirrt waren, zurück und Verurteilten den Geist, der sie verführt
hatte, den Hut aufzubehalten während der Gebete der Freunde
und ihrer eigenen.\index{Hutkontroverse} Etliche von ihnen bekannten, die Freunde
seien besser als sie, und wenn die Freunde nicht gewesen wären,
so wären sie ins Verderben geraten. So ward des Herrn Kraft
herrlich offenbar und goss sich aus über alle.

Darauf trieb mich der Herr, das Einrichten von fünf 
Monntsversammlungen\index{Monntsversammlung} zu beantragen, 
für Männer und Frauen der
Stadt London außer den schon bestehenden Versammlungen für
Frauen und den Vierteljahresversammlungen,\index{Vierteljahresversammlung} 
damit die Herrlichkeit Gottes hoch gehalten werde und die, welche einen 
unordentlichen und leichtsinnigen Wandel führten und nicht nach der 
Wahrheit lebten, ermahnt und zurechtgewiesen würden. Denn weil
die Freunde nur vierteljährliche Versammlungen gehabt hatten, so
trieb es mich, nun, da die Wahrheit sich so ausgebreitet hatte,
und die Freunde zahlreicher geworden waren, das Einrichten von
monatlichen Versammlungen im ganzen Lande zu beantragen.
Und der Herr offenbarte mir, was ich tun müsse und wie die
monatlichen und vierteljährlichen Versammlungen für Männer
und Frauen in diesen und andern Ländern eingerichtet werden
müssen, und das ich denen, zu welchen ich nicht gehen könne,
schreiben solle, das sie es auch so machen. Nachdem die Sache