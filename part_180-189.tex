% \picinclude{./180-189/p_s180.jpg} 
180 Kapitel R7.
daß etz herein regnete, und wo es schrecklich rauchte, waö mir
sehr schadete. Eines Tageß besuchte mich der Gouverneur Sir
John Crossland mit Sir Franeißt Cobb. Jch bat den Gouverneur,
mich in mein Zimmer zu begleiten, um zu sehen, waß daß für
ein Ort sei. Jch hatte ein kleineß Feuer darin angezündet, welches
nun derart rauchte, daß man seinen Weg schier nicht fand. Da
der Gouverneur ein Papist war, so sagte ich ihm, ez sei sein
Fegefeuer, daß sie mir zum Aufenthalt gegeben hätten. Jch mußte
etwa 50 Schilling ausgeben 1), um den Regen abzuhalten und zu
machen, daß eö nicht so stark rauchte. Und alß ich diese AUS-
gaben gemacht hatte, und es- etwaß erträglicher geworden, gaben
sie mir ein noch schlechtereö Gelaß, wo ich weder ein Kamin noch
irgend eine andere Vorrichtung, um Feuer zu machen, hatte. Da
ez gegen die See gelegen und sehr offen war, so trieb der Wind
den Regen ungehindert herein, sodaß dass Wasser biz zu meinem
Bett kam und im Zimmer herumlief, und ich ez mit einem Ge-
fäß ausschöpfen mußte. Und wenn meine Kleider naß waren, so
hatte ich kein Feuer, um sie zu trocknen, sodaß mein Körper ganz
erstarrt war vor Kälte, und meine Finger so geschwollen waren,
daß einer so groß war wie sonst zwei. Obgleich ich in diesem
Raum auch zu bezahlen hatte, so gelang es- mir doch nicht, Wind
und Regen abzuhalten .....
ES wurde den Freunden nicht gestattet, mich zu besuchen;
aber sonst führten sie hier und da jemanden zu mir, entweder
um mich anzusehen, oder um sich mit mir zu unterreden. Einmal
kam eine Schar Papisten, um mit mir zu dißputieren; sie behaup-
teten, der Papst sei unfehlbar und sei immer unsehlbar gewesen
seit Petrus- Zeit, aber ich bewietz ihnen das Gegenteil aus der
Geschichte: ein Bischof von Rom, Marcellinuß mit Namen, habe
den Glauben abgeschworen und den Götzenbildern gehuldigt, dieser
sei also nicht unfehlbar gewesen. IJch sagte ihnen, wenn sie den
tmfehlbaren Geist hätten, so bedürsten sie keiner Kerker, Schwerter,
Foltern, Scheiterhaufen, Geißeln und Galgen, um ihre Religion
aufrecht zu erhalten, denn wenn sie den unfehlbaren Geist hätten,
so würden sie die Leben der Menschen schützen, statt sie umzu-
bringen, und würden in Sachen der Religion nur geistliche Waffen
1) Die Gefangenen hatten zu der Zeit die Kosten ihres- Aufenthaltes in
den Gefängnissen selbst zu tragen (s. Aschrott, Engl. Gesüngniswesen).


% \picinclude{./180-189/p_s181.jpg} 
Ein Gottezgericht. Verhaftung wegen angeblicher Verschwörung usw. 181
brauchen. Jch erzählte ihnen auch, waö einer der Jhrigen mir
berichtet hatte: eine in Kent lebende Frau war nicht nur selber
Papistin gewesen, sondern hatte auch viele andere für ihren
Glauben gewonnen. Aber alß sie zur Wahrheit dez Herm be-
kehrt wurde und durch sie zu Jesuö Christuß ihrem Heiland kam,
ermahnte sie die Papisten, ein gleiches- zu tun, unter anderm
auch einen Schneider, der bei ihr in Arbeit war; sie zeigte ihm
die Verkehrtheit der pästlichen Religion und suchte ihn für die
Wahrheit zu gewinnen, da zog er sein Messer und stellte sich
zwischen sie und die Türe, aber sie trat ihm mutig entgegen und
ermahnte ihn, sein Messer weg zu tun, denn sie kannte setne
Grundsätze; auf die Frage, waß er wohl mit dem Messer ge-
macht hätte, antwortete die Frau: ,,er hätte mich erstochen«, und
aus die weitere Frage, ob er dieö wegen ihrer Religion getan
hätte, erwiderte sie: »ja, denn es- ist der Grundsatz der Papisten,
jeden, der von ihrer Religion abtrünnig wird, womöglich zu töten«.
Dieseö erzählte ich nun den Papisten und fügte bei, ich hätte e-5’
von jemand, der früher zu ihnen gehört, sich jedoch von ihnen
gewandt habe, weil er hinter ihre Handlung?-weise gekommen war.
Sie leugneten nicht, daß sie solche Grundsätze hätten, fragten
aber, ob ich nun solches?. weitererzählen werde? Jch erwiderte:
,,ja, denn solche Dinge müssen weitererzählt werden, damit man
ersährt, wie sehr eure Religion vom wahren Christentum abweicht«.
Darauf gingen sie sehr zornig fort. Gin anderer Papist, welcher
kam, um mit mir zu diöputieren, behauptete, alle Patriarchen seien
in der Hölle gewesen, biö Christus zur Hölle hinabgesahren sei,
da habe der Teufel gesagt: »waS kommst du hierher, unsere sichere
Burg zu sprengen?« und Christuß habe geantwortet, er komme,
um alle diese zu befreien, und sei drei Tage und drei Nächte in
der Hölle gewesen, um sie alle zu befreien. Jch erwiderte ihm,
daß sei unrichtig, denn Christuß habe ja zum Schächer gesagt:
,,heute noch sollst du mit mir im Paradiese sein.« Und Henoch
und Elias seien in den Himmel gekommen, auch Abraham, denn
ez heiße, Lazaruß sei in Abrahamtz Schoß gewesen, und Moseß
und Eliaö seien mit Jesus auf dem Berg gewesen, ehe er leiden
mußte. Diese Beispiele stopften dem Papisten den Mund und
brachten ihn in Verlegenheit.
Ein andermal kam Doktor Witty, ein berühmter Arzt, mit
Lord Falconbridge; mit ihnen kam auch der Gouverneur der


% \picinclude{./180-189/p_s182.jpg} 
182 Kapitel IV.
Festung Tynemouth und mehrere Adlige. Als ich zu ihnen ge-
rufen wurde, sing Witt:) ein Gespräch mit mir an und fragte mich,
warum ich im Gefängnis?. sei. Jch antwortete: »weil ich den Ge-
boten Ehristi nicht ungehorsam sein will«. Gr sagte, ich hätte
dem König den Treueid leisten sollen. Da er ein eifriger Preß-
byterianer war, so fragte ich ihn, ob er denn nicht zuerst gegen
den König und daß Unterhauß geschworen und sich zum
schottischen Covenant bekannt habe und seither wieder zum König
geschworen habe? watz denn dann daß Schwören niitze? Mein
Huldigungseid, fügte ich bei, bestehe eben nicht im Schwören,
sondern in Wahrheit und Treue. Nach einigem Hin- und Her-
reden wurde ich wieder in meine Zelle zurückgeschickt; nachher
prahlte dieser Arzt bei seinen Patienten in der Stadt herum, er
habe mich besiegt. Alö ich von seinem Prahlen hörte, sagte ich
dem Gouverneur, ez sei ein geringer Ruhm zu sagen, man habe
einen Gefangenen besiegt. Jch bat, man solle ihm sagen seinen
Besuch zu wiederholen, wenn er wieder ine; Schloß komme. Er
kam nach einiger Zeit wieder mit sechözehn oder siebzehn ange-
sehenen Leuten und erlitt eine noch größere Niederlage als das
erstemal; er behauptete nämlich, Ehristuß habe nicht alle, die in
die Welt kommen, erleuchtet, und die heilsame Gnade Gotteß sei
nicht allen Menschen erschienen, und Christuß sei nicht für alle
Menschen gestorben. Ich fragte ihn, was daß für Menschen
seien, die Ehristuö nicht erleuchtet habe, denen die heilsame Gnade
nicht erschienen sei, und für die er nicht gestorben sei? Er sagte,
die Ghebrecher, die Götzendiener, die Gottlosen. Jch fragte ihn,
ob die Ghebrecher und Gottlosen keine Sünder seien? Gr sagte,
doch. ,,Und starb nicht Ehristuz eben für die Sünder?« fragte ich,
,,kam er nicht, die Sünder zur Buße zu rufen?« Er sagte: »doch«.
,,Dann hast du dir selber daß Maul gestopft«, sagte ich. Hiermit
hatte ich bewiesen, daß die Gnade Gotteß allen Menschen er-
schienen ist, obgleich viele sie in Mutwillen kehren und ihr wider-
streben, und daß Christuß alle Menschen erleuchtet hat, wenn
schon Viele dasz Licht hassen. Manche der Anwesenden gaben zu,
daß dietz wahr sei, der Doktor aber ging fort und kam nie mehr
zu mir.
Ein andermal brachte der Gouverneur einen Priester zu
mir, aber sein Mund war bald gestopft. z’Bald daraus brachte
er zwei Parlamentsmitglieder, die mich fragten, Ob ich Prediger


% \picinclude{./180-189/p_s183.jpg} 
Ein Gottesgericht. Verhaftung wegen angeblicher Verschwörung usw. 183
und Bischöfe gelten lasse. Jrh erwiderte: ,,ja, solche die Christuß
sendet, die umsonst empfangen und umsonst geben, die dazu be-
stimmt sind und den Geist und die Kraft haben, welche auch die
Apostel hatten. Solche Bischöfe und Prediger aber wie eure, die
nichts tun, alö maß; ihnen ein guteß Einkommen bringt, die
lasse ich nicht gelten, denn sie sind nicht den Aposteln gleich.
Christus sagte zu seinen Jüngern: ,,Gehet hin in alle Welt und
predigt daß Evangelium umsonst.« Jhr Parlamentßmitglieder,
die ihr euren Bischöfen und Predigern so große Pfründen gebt,
ihr habt sie verdorben. Meinet ihr etwa, diese gehen zu allen
Völkern? oder überhaupt über ihre fetten Pfründen hinauß, um
zu predigen? Urteilt selber, ob sie daß tun oder nicht.«.
Ein andermal kam die Witwe von Lord Fairfax und viele
mit ihr, unter anderm auch ein Priester. ES trieb mich, ihnen
die Wahrheit zu verkünden; der Priester fragte mich, warum wir
,,du« und ,,dich« zu den Leuten sagen? denn er hielt uns für
Narren und Dummköpfe deßwegen. Ich fragte ihn, ob er finde,
die, welche die Schrift übersetzten rmd die Grammatik und Sprach-
lehre machten, seien Narren und Dummköpfe gewesen, weil sie
sie so übersetzten und lehrten, daß ,,du« für eine Person und
,,ihr« für mehrere gilt? wenn denn diese Narren und Dummköpfe
gewesen seien, warum denn dann nicht er und die, welche seine
Ansicht teilen und sich für weise halten, die Grammatik und Sprach-
lehre und die Bibel verbessern, und die Mehrzahl statt der Ein-
zahl setzen? Wenn es aber weise Männer gewesen seien, die die
Bibel übersetzten und die Sptachlehre und Grammatik machten,
so sollen sie sich fragen, ob nicht etwa sie die Narren und Dumm-
köpse seien, die nicht reden wie die Bibel und die Grammatik
lehre, sondern uns:-’, die ez tun, darum schelten? So war dem
Priester der Mund gestopft, und viele wurden von der Wahr-
heit überzeugt und waren recht empfänglich und zugänglich. Einige
boten mir Geld an, aber ich nahm es nicht.
Hierauf kam Doktor Cradock mit drei weiteren Priestern,
dem Gouverneur und seiner Frau, einer »Dame« (lachs) wie
man zu sagen pflegt, und einer andern ,,Dame« und eine ganze
Schar mit ihnen. Doktor Eradock fragte mich, warum ich im
Gefängnis sei; ich antwortete: ,,Weil ich den Geboten Christi und
der Apostel, nicht zu schwören, gehorche.« Wenn aber er, ein
Doktor und Friedenörichter, mir beweisen könne, daß Christuz


% \picinclude{./180-189/p_s184.jpg} 
184 Kapitel 17.
oder der Apostel den Christen, nachdem er ihnen verboten hatte,
zu schwören, es ihnen nachher wieder zu tun befahl, so wolle
auch ich es tun. Ich Tgab ihm die Bibel, damit er mir irgend
ein solches Gebot zeige, wenn er könne. Er sagte: »Jhr sollt
ohne Heuchelei und heiliglich schwören (Jer. 4, 2). ,,Ja, ja, sagte
ich, »so hieß es zu Jeremias Zeiten, aber das war lange bevor
Christus befahl: ihr sollt überhaupt nicht schwören (Matth. 5, 34).
Aus dem alten Testament könnte ich ebensoviele Beispiele oder vielleicht
noch mehr bringen, aber was nützen sie für den Beweis, daß das
Schwören auch im neuen Testament erlaubt war, nachdem Christus
und die Apostel es verboten? Übrigens: zu wem wird dort ge-
sagt, sie sollten nicht schwören ? zu den Heiden oder zu den Juden?«
Hierauf gab er keine Antwort. Aber einer der Priester sagte:
,,zu den Juden,« und Doktor Cradock gab es zu. ,,Gut,« sagte
ich, ,,aber wo hat Gott je den Heiden ein Gebot gegeben zu
schwören? und ihr wisset ja, daß wir von Natur Heiden sind.«
,,Allerdings,« sagte Doktor Eradock; ,,zwar zur Zeit des Evan-
geliums mußte alles aus zweier oder dreier Zeugen Mund bestätigt
werden, aber geschworen wurde nicht.« Warum also,« fragte ich,
,,zwingst du den Christen Gide ab gegen dein besseres Wissen?
und warum exkommunizierst du die Freunde?« ser hatte nämlich
viele sowohl in York als auch in Lancashire exkommuniziert). Gr
sagte: ,,weil sie nicht in die Kirche kamen.« »So!« sagte ich,
,,vor mehr als zwanzig Jahren, als wir noch Knaben und Mädchen
waren, da überließet ihr uns den Presbyterianern, den Jndepen-
denten und Baptisten, und viele von diesen nahmen uns Hab und
Gut und verfolgten uns, weil wir uns ihnen nicht anschließen
wollten; damals waren wir noch jung und wußten wenig
von euren Ansichten; hättet ihr nun die alten Leute, denen sie
bekannt waren, bei euch behalten und eure Ansichten in Kraft
erhalten wollen, so hättet ihr sollen entweder euch nicht von uns
wenden, wie ihr getan, oder ihr hättet uns sollen eure E-pisreln,
Kollekten, Homilien und Abendliiurgien senden, wie Paulus ja
auch den Heiligen geschrieben hatte, als er in der Gefangenschaft
von ihnen getrennt gewesen war. Wir hätten allesamt können
Türken oder Juden werden, was das, was wir in dieser Zeit von
euch empfmgen, anbelangt; und nun habt ihr uns, alt und jung,
exkommuniziert, also aus eurer Kirche ausgestoßen, ehe ihr uns für
dieselbe gewonnen habt. Jst es nicht ein Unsinn, uns auszu-


% \picinclude{./180-189/p_s185.jpg} 
Ein Gottetzgeeicht. Verhaftung wegen angeblicher Verschwörung usw. 185
weisen, ehe wir drin waren? Ja, wenn ihr unö für eure Kirche
gewonnen hättet und wir ihr angehört hätten und dann etwa
Unrechteß getan hätten, so wäre etz einigermaßen begründet ge-
wesen. Watz nennst du iibrigenß ,,Kirche?« ,,Nun,« sagte er,
,,daZ was du ,,TurmhauS« nennst.« Darauf fragte ich ihn, ob
denn Ehristuß sein Blut für daß Turmhaue-’ vergossen habe. ,,Und,«
sagte ich, ,,wenn nun die Kirche die Braut Christi und Christus-
datz Haupt der Kirche genannt wird, glaubst du denn, daß- Turm-
haus sei die Braut Christi und er das Haupt dieseö alten Gebäude?-?
ist er nicht vielmehr daß- Haupt der Gemeinde?« (Gph. 5). ,,Er
ist daö Haupt der Gemeinde,« erwiderte er, ,,und sie ist die
Kirche.« ,,Jhr habt also den Namen Kirche, welcher der Gemeinde
zukommt, einem alten Hause gegeben,« sagte ich, ,,und habt die
Leute gelehrt, solcheö zu glauben!« Weiter fragte ich ihn, warum
die Freunde verfolgt werden darum, daß sie den Zehnten nicht
geben? Ob Gott je den Heiden geboten habe, den Zehnten zu
bezahlen? Ob Christuö nicht die Zehnten aufgehoben habe, alß
er daß Levitische Priestettum, daß Zehnten nahm, aufhob? Und
ob Christus, alö er seine Jünger außsandte zu predigen, ihnen
nicht geboten habe, umsonst zu predigen? und ob nicht alle Diener
Christi verpflichtet seien, dieseß Gebot zu halten? Gr sagte, er
wolle hierüber nicht streiten; er schien überhaupt nicht gern bei
diesem Gegenstand zu verharren, sondern ging bald zu einem
andern über und sagte: ,,Jhr verheiratet euch, aber man weiß
nicht, wie ihr dabei verfahrt.« Jch riet ihm, zu kommen und
selbst zu sehen. Er drohte, unö seine Macht fühlen zu lassen; ich
riet ihm, zu bedenken, daß er ein alter Mann sei, und fragte ihn,
wo er oon der Genesiß biz zur Offenbarung irgendwo lese, daß
ein Priester jemand getraut habe; er solle mir ein solcheö Beispiel
zeigen, wenn er wolle, daß wir zu ihnen kommen sollten, um
um? trauen zu lassen. ,,Du hast ja,« sagte ich, ,,einen der Freunde
zwei Jahre nach seinem Tode noch erkommuniziert wegen seiner
Ehe; warum exkommunizierst du nicht auch Jsaak, Jakob, Boaö und
Ruth? Warum machst du deine Macht nicht Lauch an diesen
geltend? Denn ez steht nirgends, daß sie von einem Priester
getraut worden seien, sondern sie nahmen einander in der Ver-
sammlung in Gegenwart Gotteß und seiner Gemeinde; und so
tun wir. Wir haben also die heiligen Männer und Frauen der
Schrift auf unsrer Seite in dieser Sache.« Wir redeten lange


% \picinclude{./180-189/p_s186.jpg} 
186 Kapitel ZW.
hin und her; als er aber sah, daß er nichtß über mich vermochte,
ging er fort mit feinen Begleitern .....
Jn diesem und dem vorhergehenden Jahre waren viele
Freunde gefangen genommen worden. Viele waren in London, in
Newgate und andernGesängnissen, wo die Krankheit (Pest) herrschte,
und starben dort. Viele wurden auch verbannt und auf dez
Königz Befehl auf Schiffe gebracht. Oft wollten die Schiffßherren
sie nicht aufnehmen und setzten sie wieder ank; Land; doch gelangten
viele nach Barbadoeß, Jamaika und Neoiß, und der Herr segnete
sie dort .....
Nachdem ich mehr altz ein Jahr im Schloß zu Searbro ge-
fangen gewesen war, schickte ich einen Brief an den König, in
dem ich ihm von meiner Gefangenschaft berichtete und von der
schlechten Behandlung, die ich während derselben zu erdulden
hatte, und daß man mir gesagt habe, niemand alö er könne mich
frei machen. Und John Whitehead begab sich zu Gßquire Marsh,
mit dem er befreundet war, um ihm von mir zu reden, und dieser
versprach, daß, wenn John Whitehead einen Bericht über meine
Angelegenheit verfassen wolle, er denselben John Birkenhead, der
über die Begnadigung-zgesuche zu entscheiden hatte, einhändigen
und sich um meine Freisprechung bemühen wolle. John
Whitehead und Glliß Hookeß oerfaßten nun einen Bericht über
meine Gefangennahme und meine Leiden während der Gefangen-
schaft und brachten ihn Marsh, der ihn John Birkenhead über-
brachte und einen Befehl zu meiner Freisprechung erwirkte. Jn
demselben hieß e-3, daß der König von glaubwürdiger Seite er-
fahren habe, ich sei stets gegen alles Komplottieren und Streiten
gewesen, und habe etwaige Verschwörungen eher entdecken helfen,
alß daß ich mich selber daran beteiligt hätte, und so sei ez Sein
königlicheö Wohlgefallen, daß ich auö meiner Gefangenschaft befreit
werde. Sobald dieser Befehl bekannt war, kam John Whitehead
damit nach Scarbro und übergab ihn dem Gouverneur, der nun
die betreffenden Behörden zusammen berief und ohne weitere
Bürgschaft für mein friedsameß Leben, zufrieden mit der Erklärung,
daß ich ein stiller Bürger sei, mich frei ließ .....
Gleich am Tage nach meiner Freilassung brach das Feuer
in London auß, und daß Gerücht davon verbreitete sich rasch im
Lande. Da sah ich, daß Gott der Herr sein Wort wahr gemacht
hatte, daß int Gefängniö zu Lancaster zu mir geschehen war, alö


% \picinclude{./180-189/p_s187.jpg} 
Ein Gottezgericht. Verhaftung wegen angeblicher Verschwörung usw. 187
ich den Engel dez Herrn gesehen hatte, wie er mit einem leuch-
tenden Schwerte gen Süden zeigte, wie ich schon berichtet habe.
Die Bewohner waren vor diesem Feuer gewarnt worden; aber
wenige hatten esJ geglaubt oder zu Herzen genommen, vielmehr
wurden sie noch schlechter und hochmütiger. Ein Freund war nämlich
getrieben worden, von Huntingdonshire herunter zu kommen, kurz
vor der Feuerßbrunst, und sein Geld herum zu streuen, sein Pferd
frei in den Straßen herum zuführen, die Kniebänder aufzulösen,
die Strümpfe herunter hängen zu lassen, daß Wamß aufzuknöpfen
und den Leuten zu sagen: »so werdet ihr herum laufen und euer
Hab und Gut utnherstreuen, halb nackt, wie Wahnsinnige; und so
geschah es-, alö die Stadt brannte. So machte der Herr seine
Propheten und Diener zu Werkzeugen seiner Kraft und gab
ihnen Zeichen seineß Gerichtß und sandte sie, daß Volk zu warnen;
aber statt Buße zu tun, haben sie sie mißhandelt und etliche
gefangen genommen, unter der früheren Regierung sowohl als?
jetzt; aber der Herr ist gerecht, wohl dem, der seinen Worten
gehorcht! Etliche trieb e-J, nackt in den Straßen umher zu laufen,
um zu zeigen, wie Gott ihnen ihre heuchlerisehe Frömmigkeit
abreißen werde und sie nackt und bloß machen werde. Aber
das Volk hatte, statt in sich zu gehen, diese ost gegeißelt oder
sonst mißhandelt oder gar gefangen genommen. Andere trieb ez,
in Sticken umhetzugehen und die Rache und Strafe Gotteö wegen
des großen Hochmuteö zu verkünden; aber wenige gaben darauf
acht. In den Tagen der früheren Regierung machten die falschen,
frömmlerischen Priester mehrere Petitionen gegen unß an Oliver
und Richard, die sogenannten Protektoren, und an das Parlament
und die Richter und Räte, voller Lügen, Verleumdungen und
Schmühungen; aber wir oerschafsten unö Abschriften davon, und
mit Gotteß Hilfe antworteten wir auf alle, und wuschen die
Wahrheit und une-’ rein. Aber o, welcheMächte der Finsternis erhoben
sich in denen, die zum Lügen ihre Zuflucht nahmen! aber der
Herr stürzte sie alle und schützte seine Lämmer durch seine Kraft
und Wahrheit, sein Licht und sein Leben, und deckte sie, wie mit
Adlerß Flügeln. Solcheß gab uns Mut, aus ihn zu vertrauen,
der alle, die sich im Finstern gegen seine Wahrheit und sein Volk
verbünden, stürzt und vernichtet, und der durch diese Wahr-
heit seinem Volk Macht gibt, ihm in der Wahrheit zu dienen. ..


% \picinclude{./180-189/p_s188.jpg} 
188 Kapitel IR-’1.
Kapitel Zyl.
Einrichtung der Monatöversommlungm. Regelung
der Qnäletehen. ldriindnng von Knaben- u. Mädchenschulen.
Reformation des Qnäkertuus.
Nachdem ich nun wieder frei war, zog ich wieder umher,
nach Whitbt) . . . nach Oran .... und zuletzt nach Marmaduke
Storrz? ..,. wo ich eine große Versammlung hatte ..... Am
Tage nach derselben sollten zweie von der Freunden sich zur
Ehe nehmen, und es- war deshalb eine sehr zahlreiche Versamm-
lung, der ich beiwohnte. ES trieb mich, den Leuten unsern Stand-
punkt über die Eheschließung auseinanderzusetzen, indem ich
ihnen zeigte, wie man im Volke Gotteß einander zur Ehe ge-
nommen hatte in der Versammlung der Ältesten und wie es Gott
gewesen, der Mann und Weib zusammentigte vor dem Fall.
Nachher hätten dann zwar die Menschen sich selber zusammen-
getan; im Stand der Erlösung aber werde daß Zusammenfügen
durch Gott alö die richtige und ehrenhafte Verbindung angesehen,
und nie lesen wir von irgend einem Priester, von der Genesis
biz zur Offenbarung, daß er je zweie zusammengab. Hierauf
redete ich ihnen von den Pflichten der Eheleute, wie sie beide Gott
dienen sollten, als gleichermaßen Erben deö Lebens und der Gnade
(1. Petr. 3, 7) ..... Dann besuchte ich die Freunde im Lande
umher, biß ich nach York kam, wo ich eine große Versammlung
harte. Nach derselben besuchte ich Richter Robinson, einen früheren
Friedenörichter, der von Anfang an mir und den Freunden sehr
wohlgesinnt gewesen war. EZ war ein Priester bei ihm, welcher
mir sagte, es heiße von un?-, wir liebten niemanden alsz uns selber.
Ich erwiderte ihm, daß wir alle Menscken lieben, als Gotteß
Geschöpfe, die ja alle von Adam und Eve abstammen, und daß
wir die Brüder lieben, durch den Heiligen Geist. Dies brachte
ihn zum Schweigen, und wir gingen schließlich in Frieden aus--
einander; darnach reiste ich weiter.
Jch schrieb um diese Zeit ein Buch, betitelt: Fürchte Gott
und ehre den König. Ich zeigte darin, daß niemand wahrhaft
Gott fürchten und den König ehren könne, der nicht mit der Sünde
und dem Bösen breche. Dieses Buch machte großen Eindruck
aus die Soldaten und auf viele andere Leute .....
Nachdem ich viele Grafschaften durchzogen hatte, wo ich
Freunde besuchte und mit ihnen oiele große und gesegnete Ver-


% \picinclude{./180-189/p_s189.jpg} 
Einrichtung der Monatöversammlungen. Regelung der Quäkerehen usw. 189
sammlungen hatte, kam ich nach London. Aber ich fühlte mich
sehr schwach nach der beinahe dreijährigen harten und grausamen
Gesangenschaft; alle meine Gelenke und mein ganzer Körper waren
so steif und lahm, daß ich fast mein Pferd nicht besteigen noch
mich bewegen konnte; auch konnte ich schier die Nähe eineß Feuerß-
nicht ertragen oder den Genuß von warmem Fleisch, nachdem ich
so lange beides entbehrt hatte. In London besuchte ich öfters?3
die Brandstätten und sah mich aufmerksam darin um. Jch sah,
daß die Stadt so aus-sah, wie mir der Herr einige Jahre zuoor
geoffenbart hatte .....
Um diese Zeit erreichte die Kraft dez Herrn etliche, welche
die Wahrheit verlassen und die Freunde angegriffen hatten; sie
strömte so herrlich hernieder, daß sie ihre Schmähschriften ver-
dammten und zum Teil zerrissen. Wir hatten etliche Versamm-
lungen mit ihnen, und des Herm Krast war über allen und richtete
die Abtriinnigen. Jn diesen Versammlungen, welche ganze Tage
dauerten, kamen manche, welche mit John Perrot und andern
abgeirrt waren, zurück und Verurteilten den Geist, der sie verführt
hatte, den Hut auszubehalten während der Gebete der Freunde
und ihrer eigenen. Etliche von ihnen bekannten, die Freunde
seien besser als sie, und wenn die Freunde nicht gewesen wären,
so wären sie inß Verderben geraten. So ward deö Herm Kraft
herrlich offenbar und goß sich aus über alle.
g,;Daraus trieb mich der Herr, das Einrichten von fünf Monntö-
Versammlungen zu beantragen, für Männer und Frauen der
Stadt London außer den schon bestehenden Versammlungen fiir
Frauen und den Vierteljahre?-Versammlungen, damit die Herrlich-
keit Gotteß hoch gehalten werde und die, welche einen unordent=
lichen und leichtsinnigen Wandel führten und nicht nach der Wahr-
heit lebten, ermahnt und szurechtgewiesen würden. Denn weil
die Freunde nur oierteljährliche Versammlungen gehabt hatten, so
trieb es mich, nun, da die Wahrheit sich so ausgebreitet hatte,
und die Freunde zahlreicher geworden waren, das Eimsichtett von
monatlichen Versammlungen im ganzen Lande zu beantragen.
Und der Herr ofsenbarte mir, maß ich tun müsse und wie die
monatlichen und oierteljährlichen Versammlungen siir Männer
und Frauen in diesen und andern Ländern eingerichtet werden
müssen, und daß ich denen, zu welchen ich nicht gehen könne,
schreiben solle, daß sie etz auch so machen. Nachdem die Sache

