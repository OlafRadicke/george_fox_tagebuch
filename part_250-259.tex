% \picinclude{./250-259/p_s250.jpg} 
250 Kapitel Jllcl
sie in den Tagen der Märtyrer gegen sie gezeugt haben. Denket
auch daran, welches Gericht über die gekommen ist, welche den
Freunden ihre Habe geraubt und sie inz Gefängnis?. getan, um
der Zehnten und Unterstützungen willen. Darum sühret den Krieg
gegen das Tier weiter in der Kraft des Herrn, und süttert ez
nicht, nur damit es euch Friede! zurusen solle; solchen Frieden
sollt ihr nicht annehmen, sondern ihr müsset ihn brechen und ver-
werfen durch den Geist Gotteß. Dann werdet ihr in diesem selben
Geist vom Sohn dez Friedens; jenen Frieden empfangen, den
weder das Tier, noch die Hure, noch die Welt mit allen ihren
irdischen Lehrern empfangen können und ihn euch auch nicht
rauben können. Darum bewahret eure Herrschaft und Macht in
der Kraft, dem Geist und dem Namen Jesu. Jch grüße euch in
seiner Liebe.«
3. Monat dez Jahreß 1677. G. F.
Zur Jahre-?-versammlung kamen zahlreiche Freunde auß-
allen Teilen deß Landes, etliche auch von Schottland, Holland
und andern Ländern, und wir hatten gar herrliche Versammlungen,
in denen die mächtige Gegenwart des Herrn reichlich gespürt
wurde; und die Wahrheit machte gute Fortschritte in der Einig-
keit deö Geisteß zur Freude und Stärkung der Aufrichiigen; ge-
lobt sei der Herr immerdar! ....
Nachdem ich, nach der Jahreßoersammlung, etwa eine Woche
mit Freunden in London zugebracht, ging ich mit William Penn
in sein Hauß nach Sussex .....
Dann blieb ich etwa drei Wochen in Worminghurst, in
welcher Zeit John Burnyeat und ich eine Entgegnung aus ein
sehr böswilligeß Buch schrieben, welches Roger William-Z, ein
Priester in Neu=England, gegen die Wahrheit und die Freunde
geschrieben hatte .....
Dann gingen wir nach Kingston und dann nach London, wo
ich aber nicht lange blieb. Denn es- kam über mich vom Herrn,
nach Holland zu gehen, um die Freunde dort zu besuchen, und
daß Evangelium dort zu predigen, sowie auch in einigen Teilen
in Deutschland. Darum rüstete ich so schnell wie möglich alle?.
zur Abreise und nahm Abschied von den Freunden in London,
und ging mit einigen andern Freunden nach Colchester .... und
von da nach Harwich .....


% \picinclude{./250-259/p_s251.jpg} 
Reise nach Holland. Einrichtung der kirchlichen Ordnung usw. 251
Kapitel Illlll.
Reise nach Holland. Einrichtung der kirchlichen Ordnung sür
Holland und Deutschland. Briefwechsel mit Prinzessin Elisabeth.
Reise nach Deutschland bis Oldenburg. Briefe an verschiedene
Behörden von Holland und Deutschland.
Al-3 daZ Schiff bereit war, nahmen wir, die au?-ersehen waren,
nach Holland zu gehen, von den Freunden Mschied und gingen
an Bord, am Abend deß 25. des 5. Monate- 1677 .....
Am Morgen dez 28. kamen wir in Rotterdam an. Wir
hatten eine gute Überfahrt gehabt, gelobt sei der Name dez Herrn
immerdar .....
Von Rotterdam gingen wir über Ouderkerk und Delft nach
Leyden, welches 18 engl. Meilen von Rotterdam ist, wir hatten
5 Stunden zu fahren, denn unser Boot wurde Von einem Pferd
gezogen, daß am Ufer entlang ging. Tagtz darauf gingen wir
nach Haarlem, wo wir eine zahlreiche Versammlung hatten, und
von da in Begleitung mehrerer Freunde nach Amsterdam.
Am Tag nach unsrer Ankunft war die Vierteljahresversamm-
lung in Amsterdam, zu welcher Freunde von Haarlem und Rotter-
dam kamen und mit ihnen diejenigen unserer Gefährten, die wir
in Rotterdam zurückgelassen hatten, nämlich Robert Barclay,
George Keith und seine Frau und andere. Die Versammlung, die
im Hause von Gertrud Dirick Nieson stattfand, war sehr zahlreich
und segenßreich, denn wir beide, William Penn und ich, wurden
getrieben, une- über die Ordnungen dez Goangeliumß auözusprechen
und den Segen und Nutzen derjährlichen, oierteljährlichen und monat-
lichen Versammlungen für Männer und Frauen zu zeigen. Am
folgenden Tage hatten wir wieder eine Versammlung bei Gertrud
Diriek, mehr öffentlich und sehr zahlreich; ,,Fromme« aller Art
wohnten derselben bei, und der Weg des Lebenß und Heilz wurde
ihnen eingehend und lebendig dargelegt. Sie hörten sehr auf-
merksam zu und niemand erhob irgend einen Widerspruch gegen
daß, waS verkündet wurde. Am Nachmittag hatten wir abermalö
eine Versammlung daselbst, aber im engeren Kreise. Am Tage
darauf hatten wir eine Versammlung nur fitr Freunde, in welcher
nach allgemeinem Übereinkommen Verschiedene Versammlungen ein-
gerichtet wurden, monatliche, Vierteljährliche und jährliche, welche
in Amsterdam gehalten werden sollten für die Freunde aller


% \picinclude{./250-259/p_s252.jpg} 
252 Kapitel ITU.
Provinzen Hollandö, sowie von Emden, der Pfalz, Hamburg-
Friedrichßstadt, Danzig und andern Orten Deutschlandß und der
umliegenden Länder. Darüber waren die Freunde sehr floh, und
die Wahrheit wurde dadurch gefördert .....
An einem Ersten Tage hatten wir eine sehr zahlreiche Ver-
sammlung, zu welcher viele Leute der verschiedensten Richtungen
herbeiströmten: Baptisten, Socinianer, Seeker, Vrawniften und
einige Studierende. George Keith, Robert Barclay, William Penn
und ich verkiindeten die ewige Wahrheit unter ihnen, indem wir
ihnen den Zustand dez gefallenen Menschen dartaten und ihnen
zeigten, wie Männer und Frauen in den Stand der Wiedergeburt
durch Jesum Ehristum gelangen können; daß Geheimniö der Sünd-
haftigkeit und das- Geheimniß der Gottseligkeit wurden deutlich
ausgelegt, und die Versammlung endete ruhig und besiiedigcnd.
Tagß darauf ließen Robert Barclay, George Keith und William
Penn mich und etliche andere Freunde in Amsterdam zurück und
gingen nach Deutschland weiter, wo sie Viele hundert Meilen weit
umherreisten und für den Herrn wirkten, Benjamin Furln ging mit
ihnen alß Dolmetscher .....
Jch schrieb einen Brief an die Prinzessin Elisabeth, welchen
Jsabel Yeomanz ihr iiberbrachte, ale George Keithö Frau und
sie sie besuchten:
»Prinzessin Elisabeth! l)
Ich habe durch Freunde, die dich besucht haben, sowie aus
einigen deiner Briefe, die ich gesehen, erfahren, daß du dem Herm
und seiner heiligen Wahrheit zugetan bist; eö ist ein Großes,
wenn Leute deineß Standeö einen so empsönglichen Sinn haben
  den Herrn und seine köstliche Wahrheit, während so viele in
Uppigkeit und den Vergnügen dieser Welt untergehen; zwar be-
kennen sich alle äußerlich in irgend einer Weise zu Gott und
Christuö, ohne jedoch innerlich etwas Tiesereß siir ihn zu empfinden.
Denn nicht viele Mächtige noch Weise dieser Welt (1. Kor. 1)
sind imstande um Christi willen Narren zu werden, oder sich von
ihrer Höhe herab in die Demut Christi zu erniedrigen, durch
welche sie einen viel miichtigeren Stand und ein mächtigereö König-
1) Prinzessin Elisabeth war die älteste Tochter Friedrichs V., König von
Böhmen; sie hatte ihre Residenz in Herwerden. Sie war eine tnusterhafie
Christin und hielt große Stücke auf die Qnäkcr und ihre Grundsätze, wie ihre
Briefe an verschiedene Personen am englischen Has bezeugen.


% \picinclude{./250-259/p_s253.jpg} 
Reise nach Holland. Einrichtung der kirchlichen Ordnung usw. 253
tum erlangen durch den heiligen Geist, das ewige Licht und die
ewige Kraft Gottes, und eine höhere Weisheit, die von oben ist,
rein und voll Frieden, die Weisheit, die über allem dem Jrdischen,
Sinnlichen und Teuslischen steht, wodurch die Menschen einander
umbringen, sogar um der Religion, der Kirche und der Gottes-
dienste willen. Solches haben sie nicht von Gott noch von Christus.
Die Weisheit von oben, durch welche alle Dinge geschaffen sind,
und die mit der Gottessurcht in den Herzen beginnt, erhält die
Herzen, rein und diese Weisheit soll alle Kinder Gottes regieren,
und durch sie sollen sie alle Dinge zur Ehre Gottes tun. Dieses ist
,,die Weisheit, die sich rechtfertigt durch ihre Kinder« (Math. 11, 19h.
Es ist mein Wunsch, daß du in dieser Weisheit und Gottessurcht
bewahret bleibest zur Ehre Gottes. Denn der Herr ist gekommen,
um sein Volk selbst zu lehren und ein Panter auszurichten unter
den Völkern, aus daß die Völker herzueilen. Viele sind seit den
Tagen der Apostel abgefallen vom göttlichen Licht Christi, das
ihnen hätte die Erleuchtung von der Erkenntnis der Klarheit
Gottes in dem Angesicht Jesu Christi geben sollen und vom heiligen
Geist, der sie in alle Wahrheit geleitet hätte und vom heiligen
Glauben, dessen Anfänger und Vollender Christus ist, welcher
die Herzen reinigt und Sieg gibt über alles, was von Gott trennt,
durch welchen Glauben man Zugang hat zu Gott und in welchem
man Gott angenehm ist, dessen Geheimnis bewahret ist in reinen
Herzen. Sie sind abgesallen vom Evangelium, das in den Tagen
der Apostel gepredigt wurde und welches in Mann und Frau
Leben und unsterbliches Wesen ans Licht bringt, und durch das
die Menschen den Teufel, der sie versinstert hat, überwinden
sollten; dieses Evangelium sollte alle, die es aufnehmen, im Leben
und unsterblichen Wesen bewahren. Sie haben aus die Menschen
gesehen und nicht auf den Herrn, der sein Gesetz allen Kindern
des neuen Bundes in die Herzen schrieb, in welchem Licht, Leben
und Gnade ist und durch welche alle, vom Höchsten bis zum
Niedrigsten, zur Erkenntnis Gottes kommen, sodaß die Gotteser-
kenntnis die Erde bedecket wie das Wasser das Meer. Dieses
Wirken des Herm hebt nun wieder an, wie in den Tagen der
Apostel; die Menschen sollen ,,eine Salbung haben von dem, der
heilig ist und durch den sie alles wissen« (1. Joh. 2,20) und ,,be-
dürfen nicht, daß sie jemand lehre, sondern die Salbung lehret
sie« (1. Joh. 2,27) ..... Jetzt kommen die Menschen wieder


% \picinclude{./250-259/p_s254.jpg} 
254 Kapitel RIU.
zurück vom Abfall in daß Licht und den Geist Christi, und
empfangen den Glauben von ihm und nicht mehr von Menschen,
und empfangen von ihm daß Evangelium und die Salbung und
daß Wort .... Denn Gott ist in seinem Sohne Jesus- Christus
gekommen, sein Volk selbst zu lehren und es- abzubringen von
den Wegen dieser Welt, zu Christus, dem Weg, der Wahrheit
und dem Leben, welcher ist der Weg zum Vater, und von allen
Lehrern und Predigern der Welt zu ihm, dem wahren Lehrer
und von allen weltlichen Gotteödiensten Gott zu dienen
im Geist und in der Wahrheit, welche Art des Gottes-dienstes
Christuß vor 1600 Jahren eingesetzt hat, alZ er den jüdischen
Gotteödienst im Tempel zu Jerusalem zu nichte machte und den
Gotte?-dienst auf jenem Berge beim Jakoböbrunnen, um die
Menschen von allen weltlichen Religionen, die eingeführt wurden
seit den Tagen der Apostel, zu der von Christus und den Aposteln
ausgerichteten Religion zu bringen, die rein und unbefleckt vor
Gott ist und von der Welt unbefleckt erhält (Jar. 1), und von den
weltlichen Kirchen und Gemeinschaften, die seit den Tagen der
Apostel entstanden sind, in die Kirche, die in Gott dem Vater
unsereß .Herrn Jesuß Christuö ist, in die Einigkeit und Gemeinschast
des heiligen Geisteis, welcher tötet, beschneidet und tauft, um die
Sünde und Verderbtheit, welche durch die Ubertretung in den
Männern und Frauen entstanden sind, aus-zutilgen. Jn diesem
heiligen Geist ist heilige Gemeinschaft und Ginigkeit, er ist das
Unterpfand dez Fürsten der Fürsten, des Königß aller Könige,
er ist daß Band des Friedenö dez Herrn aller Herren; diesen
himmlischen Frieden sollen alle wahren Christen sich erhalten, mit
geistlichen Waffen, nicht mit fleischlichen.
ES haben nun, meine Freundin, die heiligen Männer Gotteö
die Schriften geschrieben, wie sie vom heiligen Geist getrieben
wurden, und die ganze Christenheit ist in Verwirrung wegen dieser
Schriften, weil alle nicht von demselben heiligen Geist geleitet sind,
wie die, welche die Schrift geschrieben; diesen heiligen Geist müssen
sie in sich einziehen lassen und sich von ihm leiten lassen, wenn
sie in alle Wahrheit kommen wollen und in den Trost Gottes
und Christi. Denn niemand kann Jesuö Herr nemten, als allein
durch den heiligen Geist und alle, die Christum ohne den heiligen
Geist Herr nennen, mißbrauchen seinen Namen. Alle aber, die
seinen Namen nennen, müssen mit der Sünde brechen, dann nennen


% \picinclude{./250-259/p_s255.jpg} 
Reise nach Holland. Einrichtung der kirchlichen Ordnung usw. 255
sie seinen Namen in Gerechtigkeit und Wahrheit. Achte doch da-
rum auf die Gnade und Wahrheit, die durch Jesum Ehristum
in dein Herz gekommen sind, um dich zu lehren, wie du leben
und was du oerleugnen sollst, sie wollen dein Herz stärken und
deine Rede angenehm machen, dir daß Heil bringen und dir
immerdar ein Lehrer sein. Durch sie wirst du Ehristuz aufnehmen,
und alle, die ihn aufnehmen, denen gibt er Macht, nicht nur dem
Bösen zu widerstehen, sondern Gotteß Kinder zu sein und alß
Kinder dann Erben eineß Lebens, einer Welt und eines Reicheö
ohne Ende und ewiger Reichtümer und Schätze. Diesetz in Eile,
meine Liebe im Herrn Jesus Christuö, welcher den Tod gekostet
für einen jeden und der Schlange den Kopf zerbrach, der zwischen
Gott und dem Menschen stehet, damit durch ihn der Mensch wieder
zu Gott zurückkehre. Er ist der himmlische Fels und geistige
Grund, auf den das Volk Gotteß sich gründen muß. Gelobt sei
der Herr immerdar.«
Amsterdam, 7. dez 6. Monatß 1677. G. F.
PS. Überbringerin dieseö ist eine meiner Schwiegertöchter,
die mit Gertrud Tirick Nieson und George Keith;3 Frau dich be-
suchen will.
Die Antwort der Prinzessin Elisabeth:
»Lieber Freund!
Ich muß alle lieb haben, die den Herrn Jesus lieben und
denen ez gegeben ist, nicht nur an ihn zu glauben, sondern auch
um seinetwillen zu leiden. Darum habe ich mich über euern
Brief wie auch über den Besuch eurer Freunde sehr gefreut, und
ich werde ihren und euern Rat befolgen nach dem Maß der
Salbung und des Lichts, dat; Gott mir schenken wird.
Eure euch stetß liebende Freundin
Hertford, 30. August 1677 ..... Elisabeth«.
Wir gingen über Alkmaar, Horn, Harlingen, Leuwarden, Gro-
ningen nach Delfziel; dieseß ist eine Stadt am Flusse Ems, über
den wir dann nach Emden gelangten, hier waren die Freunde
grausam verfolgt und oft auch verbannt worden. Jch ging nach
einer Herberge und aß zu Mittag mit einigen Leuten, die englisch
konnten; wir unterhielten unö gut, und sie waren sehr empfäng-
lich. . . Von hier gingen wir über Leer, Stickhausen, Detern
nach Pre in Dänemark (?); unterwegö trafen wir mit dem
Grafen von Oldenburg zusammen, der zum Friedenövertrag zu


% \picinclude{./250-259/p_s256.jpg} 
256 Kapitel A11.
Lembach ging. Von da gingen wir nach Oldenburg .... und
dann weiter nach Delmenhorst. Von hier gingß zu Wagen nach
Bremen, einer stattlichen Stadt in Deutschland, hier gingen wir
an ein Wasser, genannt Overdeland, und nahmen ein Boot und
fuhren nach Fischerhude ..... Hier nahmen wir wieder einen
Wagen und fuhren durchs Land dez Bischofö von-Münster nach
Elosterseoen, .... dann am anderen Tage nach Buxtehude.
Die Leute in dieser Gegend deö Bischofs- von Münster waren
sehr im Dunkeln. Während der Reise predigte ich die Wahrheit
unter ihnen und wies sie auf den großen, wichtigen Tag deß
Herrn und ermahnte sie zur Nüchternheit, und daß sie auf den
guten Geist Gotteö in ihrem J-nnem achten sollten.
Von hier gingen wir so schnell wie möglich, teilö zu Wasser,
teil3 zu Wagen, nach Hamburg. Hier kamen wir zeitig genug an,
um noch am gleichen Abend eine Versammlung abzuhalten, sie
war sehr schön und erhebend. GS wohnte ihr unter anderm ein
Baptistenprediger mit seiner Frau bei, und ein angesehener Mann
auß Schweden und seine Frau, und alleß verlief ruhig, gelobt sei
der Herr. Aber eß ist ein arger, versinsterter Ort und die Leute
sind der Wahrheit wenig zugänglich.
ES lebte eine Frau hier, die zur Zeit John Perrotö gegen
mich geredet hatte, obschon sie mich bis jetzt nie gesehen hatte;
daß hatte sie seither immer beunruhigt und sie war nun froh,
Gelegenheit zu haben, ihren Fehler einzugestehen, sie tat etz auch
sehr bereitwillig, und ich vergab ihr ebenso bereitwillig und
völlig.
Wir reisten weiter .... nach Friedrichßstatt; dort gingen
wir zu William Pauls, wo mehrere Freunde zu uns kamen,
denn eö ist eine ziemliche Anzahl von Freunden in dieser Stadt.
Wir hatten am Abend eine schöne erbauliche Versammlung mit-
einander, die uns unsre Ermüdung vergessen ließ, denn wir waren
sehr müde gewesen, nachdem wir zwei Tage ununterbrochen ge-
reist waren, oft ganz durchnäßt vom Regen in den offenen
Wagen. Aber der Herr fügte uns alles zum besten, und wir
freuten uns, die Freunde zu sehen, gelobt sei sein heiliger Name
immerdar.
Die Stadt steht unter der Herrschaft deß Herzogß von Hol-
stein, welcher die Freunde gerne aus Stadt und Land verbannt
hätte und darum den Behörden Befehl gab, solches zu tun; diese


% \picinclude{./250-259/p_s257.jpg} 
Reise nach Holland. Einrichtung der kirchlichen Ordnung usw. 257
aber erklärten, eher wollten sie ihre Ämter niederlegen, denn die
Freunde seien in diese Stadt gekommen, um Gewisseusfreiheit zu
haben. So erfreuen sich denn die Freunde dort ihrer Freiheit,
und sie stehen in gutem Ansehen zu Stadt und Land.
An einem Ersten Tage hatte ich eine Versammlung, zu der
viele Leute kamen, auch etliche Widerspenstige, aber der Herr bannte
sie alle, und der Same des Lebens breitete sich aus über allen.
Während meines dortigen Aufenthaltes hatte ich eine Unterredung
mit einem Juden, einem Leuiten, über das Kommen des Messias,
er wurde sehr zuschanden gemacht in dem, was er sagte, doch be-
nahm er sich anständig und lud mich in sein Haus ein. Ich ging
hin und unterhielt mich dort mit einem andern Juden, der mir
ihren Talmud und andere jüdische Bücher zeigte, aber sie sind
sehr in der Finsternis und Verstehen ihre eigenen Propheten nicht.
Es war ein Baptistenlehrer in dieser Stadt, welcher Schmit-
hungen und Lügen über die Freunde ausgestreut hatte, darum
ging John Claus mit zwei Freunden zu ihm in seine Wohnung
und reinigte die Freunde und die Wahrheit von seinen Ver-
leumdungen und machte ihn zuschanden, indem er die Schwä-
hungen gegen ihn selbst wendete.
Ehe wir weiterreisten, hatte ich noch eine Versammlung, aus-
schließlich sür Freunde, in der ich ihnen den Nutzen der Monats-
oersammlungen vorstellte, um nach den Armen zu sehen und Sorge
zu tragen, daß die Vermählungen und andere kirchliche Ange-
legenheiten sich in guter Ordnung Vollzögen. Da solches mit dem
Zeugnis Gottes in ihren Gewissen übereinstimmte, beschlossen sie
bereitwillig, fürderhin Monatsoersammlungen unter sich abzu-
halten, damit alle, Miinner wie Frauen, sich der äußern kirchlichen
Angelegenheiten annehmen möchten .....
Da ich mich nach dieser Versammlung meiner Pflichten gegen
diese Gegend entledigt fühlte, verabschiedeten wir uns von den
Freunden und gingen wieder nach Hamburg.
Am Tage nach unserer Ankunst hatten wir eine sehr schöne,
friedliche Versammlung. Nach derselben hatte ich eine Unter-
redung mit einem Schweden, der zu den Großen seines Landes
gehörte. Er war aus demselben verbannt worden um seines
Glaubens willen und darum nach Hamburg gekommen und hatte
schon früher einer meiner Versammlungen beigewohnt. Als ich
fertig war mit ihm, hatte ich noch eine Unterredung mit einem
George Ft-;. 17


% \picinclude{./250-259/p_s258.jpg} 
258 Kapitel JTIU.
Baptisten über die sogenannten Sakramente. Beidemale gelang
es mir, ihnen die Wahrheit zu zeigen.
Als ich mich auch in Hamburg meiner Ausgabe entledigt
fühlte, gingen wir in einem Boot in eine Stadt, die zum Lande
des Herzogs von Lüneburg gehörte. Von da gingen wir wieder
nach Bremen zurück, teils zu Wasser, teils zu Land. Unterwegs
hatte ich ost gute Gelegenheit, den Leuten die«Wahrheit zu ver-
künden, besonders in einem Marktflecken, wo wir die Pferde
wechselten; ich verkündete ihnen den Tag des Herrn, der über
alles Fleisch kommen werde, und ermahnte sie zur Rechtschaffenheit.
Ich sagte ihnen, daß Gott gekommen sei, sein Volk selber zu
lehren, und daß sie sich zu ihm kehren und aus das Lehren seines
Geistes in ihren Herzen hören sollten.
Ju Bremen gingen wir in eine Herberge, bis ein anderer
Wagen zur Weiterfahrt bereit war. Obgleich ich fühlte, daß die
Kraft Gottes mit dieser Stadt war und die unsauberen Geister
nieder hielt, so litt doch mein Geist sehr um dieser Leute willen.
Als unser Wagen bereit war, fuhren wir nach Keby, wo wir die
Nacht zubrachten, und srüh am nächsten Morgen weiter nach
Oldenburg. Es war ein trauriger Anblick, wie diese große statt-
liche Stadt niedergebrannt war! Wir begaben uns in eine
Herberge, wo, obgleich es der Erste Tag war, die Soldaten beim
Trinken und Talerschieben waren, und in den wenigen übrig-
gebliebenen Häusern waren die Verkaufsläden offen, und die Leute
handelten miteinander. Es trieb mich, die Wahrheit unter ihnen
zu verkünden und sie vor dem Gericht Gottes zu warnen; sie
hörten mich ruhig an, aber dennoch lastete ihre Schlechtigkeit
schwer auf mir .....
Am nächsten Tage gingen wir nach Leer, durch viel tiefes
Wasser, und am folgenden nach Emden .... tags darauf, zu
Schiff, nach Delfziel. Jn der Herberge kam ein Freund zu uns,
der hier lebte, weil er häufig aus Emden verbannt worden war;
er war Goldschmied und hatte Haus und Geschäft in Emden ge-
habt. Er war, trotz wiederholter Verbannung, immer wieder
dorthin gegangen, bis sie ihn schließlich bei Wasser und Brot ins
Gefängnis taten, ihm Hab und Gut nahmen, ihn mit Weib und
Kind verbannten und ihm nichts mehr zum Unterhalt oder
Aufenthalt ließen. Wir sprachen ihm Mut und Trost im Herrn
zu und ermahnten ihn, dem ihm anoertrauten Zeugnis treu zu


% \picinclude{./250-259/p_s259.jpg} 
Neise nach Holland. Einrichtung der kirchlichen Ordnung usw. 259
bleiben. Nachdem wir uns von ihm verabschiedet hatten, gingen
wir noch am gleichen Tage in einem Boot nach Groningen, wo
wir Cornelius Andries trafen, einen Freund, der auch viel unter
Gefängnis und Verbannung von Emden gelitten hatte.
Von da gingen wir . . . über Amsterdam . . . nach Harlingen.
Am Tage nach unserer Ankunft war hier die Monatsversammlung
für Männer und Frauen. Sie war gut besucht und sehr schön.
Gs wurde beschlossen, daß jeden Monat eine Versammlung statt-
finden solle, sowohl für Männer als fiir Frauen, um für die
Angelegenheiten der Kirche zu sorgen.
Am Nachmittag hatten wir eine öffentliche Versammlung,
zu welcher Leute der verschiedensten Richtungen kamen: Socinianer,
Baptisten, Lutheraner und andere, worunter auch ein Doktor der
Medizin und ein Priester. Nachdem ich ihnen eingehend die
Wahrheit verkündet hatte und den glücklichen Zustand beschrieben,
in welchem die Menschen gewesen, als sie sich noch von Gott
lehren ließen und im Paradies Gottes blieben, und andrerseits
das Elend und den Jammer, die über sie kamen, seit sie Gottes
Lehre verließen und aus die Schlange hörten, Gottes Gebote über-
traten und aus dem Paradies vertrieben wurden, und ihnen dann
den Weg zeigte, aus dem sie wieder in jenen glücklichen Zustand
zurückkehren könnten — da, gerade als ich aufhörte zu reden,
stand ein Priester, ein ernster alter Mann, auf, nahm seinen Hut
ab und sagte: ,,Jch bitte Gott, daß er diese Lehre fördere und
bestätige, denn sie ist Wahrheit und ich habe nichts gegen sie«.
Er wäre gerne bis ans Ende dieser Versammlung geblieben, aber
da er selber an diesem Abend zu predigen hatte und die Zeit
seines Gottesdienstes gekommen war, so konnte er nicht länger
bleiben. Nachdem er dieses Zeugnis für die Wahrheit abgelegt
hatte, eilte er fort, um nachher wieder kommen zu können. Gr
kam auch wieder, aber erst als die Versammlung schon zu Ende
war. Nach der Versammlung hatte ich noch eine andere nur mit
Freunden im Hause Hessel Jakobs, wohin auch der Doktor der
Medizin kam, um mit William Penn zu reden, und dieser ver-
kündete mit Erfolg die Wahrheit. Durch diesen Doktor schickte
mir jener Priester einen Gruß und ließ mir sagen, er habe eine
halbe Stunde früher als gewöhnlich aufgehört zu predigen an
diesem Abend, damit er wieder in unsre Versammlung kommen
könnte, um noch mehr von dieser guten Lehre zu hören.
17-

