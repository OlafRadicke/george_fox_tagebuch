\chapter{Vorwort}
% \section{Vorwort}
\label{sec:vorwort}

Diese Dokument basiert auf ein Scann des Werkes ``George Fox -- 
Aufzeichnungen und Briefe des ersten Quäkers''. Troz intensiver Bemühungen war es mir nicht
möglich zu ermitteln ob und wer das Urheberecht an dem Werk besitzt. Sollte jemand zur
klärung beitragen können, bitte ich um Hinweise an mich!

\begin{center}
Olaf Radicke \\
Ludwig-Richter-Str. 28 \\
80687 München \\
briefkasten@olaf-radicke.de \\
\end{center}

Ich glaube das der Text für das Quakertum in Deutschland so unverzichtbar ist,
das ich mich entschlossenhabe trozdem den Text (wieder) zu veröffentlichen. Meine 
Textbearbeitung stelle ich unter einer \textit{Creative Commons-Lizenz lizenziert}: 
Namensnennung-Weitergabe unter gleichen Bedingungen 3.0 Deutschland (CC BY-SA 3.0).


\section*{Sie dürfen}
%\addcontentsline{toc}{section}{Sie dürfen}

\begin{itemize}
 \item das Werk bzw. den Inhalt vervielfältigen, verbreiten und öffentlich zugänglich machen
 \item Abwandlungen und Bearbeitungen des Werkes bzw. Inhaltes anfertigen
 \item das Werk kommerziell nutzen
\end{itemize}

% \subsection{Zu den folgenden Bedingungen}
\section*{Zu den folgenden Bedingungen}
%\addcontentsline{toc}{section}{Zu den folgenden Bedingungen}

\begin{description}
 \item[Namensnennung] Sie müssen den Namen des Autors/Rechteinhabers in der von ihm 
festgelegten Weise nennen.

 \item[Weitergabe unter gleichen Bedingungen] Wenn Sie das lizenzierte Werk bzw. den 
lizenzierten Inhalt bearbeiten oder in anderer Weise erkennbar als Grundlage für 
eigenes Schaffen verwenden, dürfen Sie die daraufhin neu entstandenen Werke bzw. 
Inhalte nur unter Verwendung von Lizenzbedingungen weitergeben, die mit denen dieses 
Lizenzvertrages identisch oder vergleichbar sind.
 \end{description}

% \subsection{Wobei gilt}
\section*{Wobei gilt}
%\addcontentsline{toc}{section}{Wobei gilt}

\begin{description}
    \item[Verzichtserklärung] Jede der vorgenannten Bedingungen kann aufgehoben werden, 
	  sofern Sie die ausdrückliche Einwilligung des Rechteinhabers dazu erhalten.
    \item[Public Domain (gemeinfreie oder nicht-schützbare Inhalte)] Soweit das Werk, 
	  der Inhalt oder irgendein Teil davon zur Public Domain der jeweiligen Rechtsordnung 
	  gehört, wird dieser Status von der Lizenz in keiner Weise berührt.
    \item[Sonstige Rechte] Die Lizenz hat keinerlei Einfluss auf die folgenden Rechte:
      \begin{itemize}
          \item Die Rechte, die jedermann wegen der Schranken des Urheberrechts oder 
		aufgrund gesetzlicher Erlaubnisse zustehen (in einigen Ländern als 
		grundsätzliche Doktrin des fair use etabliert);
          \item Das Urheberpersönlichkeitsrecht des Rechteinhabers;
          \item Rechte anderer Personen, entweder am Lizenzgegenstand selber oder bezüglich 
		seiner Verwendung, zum Beispiel Persönlichkeitsrechte abgebildeter Personen.
      \end{itemize}
    \item[Hinweis] Im Falle einer Verbreitung müssen Sie anderen alle Lizenzbedingungen 
	  mitteilen, die für dieses Werk gelten. Am einfachsten ist es, an entsprechender 
	  Stelle einen Link auf diese Seite einzubinden.

 \end{description}

Diese "Commons Deed" ist lediglich eine vereinfachte Zusammenfassung des rechtsverbindlichen 
Lizenzvertrages in allgemeinverständlicher Sprache. Deteils, Erleuterungen und vollständigen
Lizenz-Text erhalten Sie unter \url{http://creativecommons.org/licenses/by-sa/3.0/de/}
