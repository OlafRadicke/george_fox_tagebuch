% \picinclude{./180-189/p_s188.jpg} 

%%%%%%%%%%%%%%%%%%% Kapitel 16. %%%%%%%%%%%%%%%%%%%%%%%%%%%%%%

\chapter[Einrichtung der Monatsversammlungen]{Einrichtung der Monatsversammlungen}

\begin{center}
\textbf{Einrichtung der Monatsversammlungen. Regelung der Quäkerehen. 
Gründnng von Knaben- u. Mädchenschulen. Reformation des Quäkertums.}
\end{center}


Nachdem ich nun wieder frei war, zog ich wieder umher,
nach Whitby\ort{Whitby} [...] nach Oran [...] und zuletzt nach 
Marmaduke Storrs\ort{Marmaduke Storrs}. [...] wo ich eine große 
Versammlung hatte [...] Am Tage nach derselben sollten zwei von 
der Freunden sich zur Ehe nehmen,\index{Heirat} und es war 
deshalb eine sehr zahlreiche Versammlung, 
der ich beiwohnte. Es trieb mich, den Leuten unsern Standpunkt 
über die Eheschließung auseinanderzusetzen, indem ich
ihnen zeigte, wie man im Volke Gottes einander zur Ehe genommen 
hatte in der Versammlung der Ältesten und wie es Gott
gewesen, der Mann und Weib zusammengefügte vor dem Fall.
Nachher hätten dann zwar die Menschen sich selber zusammengetan; 
im Stand der Erlösung aber werde das Zusammenfügen
durch Gott als die richtige und ehrenhafte Verbindung angesehen,
und nie lesen wir von irgend einem Priester, von der Genesis
bis zur Offenbarung, das er je zwei zusammengab. Hierauf
redete ich ihnen von den Pflichten der Eheleute, wie sie beide Gott
dienen sollten, als gleichermaßen Erben des Lebens und der Gnade
(1. Petr. 3,7\bibel{Petr. 1. 03:7@1. Petr. 3:7}) [...] Dann besuchte ich die Freunde im Lande
umher, bis ich nach York\ort{York} kam, wo ich eine große Versammlung
hatte. Nach derselben besuchte ich Richter Robinson\person{Richter Robinson}, 
einen früheren Friedensrichter, der von Anfang an mir und den Freunden sehr
wohlgesinnt gewesen war. Es war ein Priester bei ihm, welcher
mir sagte, es heiße von uns, wir liebten niemanden als uns selber.
Ich erwiderte ihm, das wir alle Menschen lieben, als Gottes
Geschöpfe, die ja alle von Adam und Eva abstammen, und das
wir die Brüder lieben, durch den Heiligen Geist. Dies brachte
ihn zum Schweigen, und wir gingen schließlich in Frieden auseinander; 
danach reiste ich weiter.

Ich schrieb um diese Zeit ein Buch, betitelt: \textit{Fürchte Gott
und ehre den König}\index{Fürchte Gott
und ehre den König}. Ich zeigte darin, das niemand wahrhaft
Gott fürchten und den König ehren könne, der nicht mit der Sünde
und dem Bösen breche. Dieses Buch machte großen Eindruck
auf die Soldaten und auf viele andere Leute [...]
Nachdem ich viele Grafschaften durchzogen hatte, wo ich
Freunde besuchte und mit ihnen viele große und gesegnete 
% \picinclude{./180-189/p_s189.jpg} 
Versammlungen hatte, kam ich nach London\ort{London}. Aber ich fühlte mich
sehr schwach nach der beinahe dreijährigen harten und grausamen
Gefangenschaft; alle meine Gelenke und mein ganzer Körper waren
so steif und lahm, das ich fast mein Pferd nicht besteigen noch
mich bewegen konnte; auch konnte ich schier die Nähe eines Feuers
nicht ertragen oder den Genus von warmem Fleisch, nachdem ich
so lange beides entbehrt hatte. In London besuchte ich öfters
die Brandstätten und sah mich aufmerksam darin um. Ich sah,
das die Stadt so aussah, wie mir der Herr einige Jahre zuvor
geoffenbart hatte [...]\index{Vorhersehung}

Um diese Zeit erreichte die Kraft des Herrn etliche, welche
die Wahrheit verlassen und die Freunde angegriffen hatten; sie
strömte so herrlich hernieder, das sie ihre Schmähschriften verdammten 
und zum Teil zerrissen. Wir hatten etliche Versammlungen 
mit ihnen, und des Herrn Kraft war über allen und richtete
die Abtrünnigen. In diesen Versammlungen, welche ganze Tage
dauerten,\index{Versammlung!Länge} kamen manche, welche 
mit John Perrot\person{Perrot, John} und andern
abgeirrt waren, zurück und Verurteilten den Geist, der sie verführt
hatte, den Hut aufzubehalten während der Gebete der Freunde
und ihrer eigenen.\index{Hutkontroverse} Etliche von ihnen bekannten, die Freunde
seien besser als sie, und wenn die Freunde nicht gewesen wären,
so wären sie ins Verderben geraten. So ward des Herrn Kraft
herrlich offenbar und goss sich aus über alle.

Darauf trieb mich der Herr, das Einrichten von fünf 
Monntsversammlungen\index{Monntsversammlung} zu beantragen, 
für Männer und Frauen der
Stadt London außer den schon bestehenden Versammlungen für
Frauen und den Vierteljahresversammlungen,\index{Vierteljahresversammlung} 
damit die Herrlichkeit Gottes hoch gehalten werde und die, welche einen 
unordentlichen und leichtsinnigen Wandel führten und nicht nach der 
Wahrheit lebten, ermahnt und zurechtgewiesen würden. Denn weil
die Freunde nur vierteljährliche Versammlungen gehabt hatten, so
trieb es mich, nun, da die Wahrheit sich so ausgebreitet hatte,
und die Freunde zahlreicher geworden waren, das Einrichten von
monatlichen Versammlungen im ganzen Lande zu beantragen.
Und der Herr offenbarte mir, was ich tun müsse und wie die
monatlichen und vierteljährlichen Versammlungen für Männer
und Frauen in diesen und andern Ländern eingerichtet werden
müssen, und das ich denen, zu welchen ich nicht gehen könne,
schreiben solle, das sie es auch so machen. Nachdem die Sache

% \picinclude{./190-199/p_s190.jpg} 
in London eingerichtet war, [...] ging ich nach Essex\ort{Essex}. Nachdem
die Monatsversammlungen hier eingerichtet waren, ging ich nach
Suffolk\ort{Suffolk} und Norfolk\ort{Norfolk} [...] Als 
auch hier die Monatsversammlungen eingerichtet waren, ging ich 
nach Huntingdonshire\ort{Huntingdonshire}, wo sie
ebenfalls eingerichtet wurden. [...] Ebenso in Bedfordshire,
und Nottinghamshire\ort{Nottinghamshire}, [...] 
Leieestershire\ort{Leieestershire}, [...] Warwickshire\ort{Warwickshire},
[...] In Staffordshire\ort{Staffordshire} hatten wir eine allgemeine 
Männerversammlung und richteten dort ebenfalls eine allgemeine 
Monatsversammlung ein [...] In Chefhire\ort{Chefhire} hatten wir ebenfalls eine
allgemeine Männerversammlung, in der die Monatsversammlung
für diese Grafschaft eingerichtet wurde [...] Auch in Laneashire
wurden die Monateversammlungen für diese Grafschaft eingerichtet, 
nach dem Evangelium [...] Von hier aus sandte ich
Schreiben nach Wesimorland, Durham, Cleveland, Northumberland, 
Eumberland und Schottland, um die Freunde zu ermahnen,
die Monatsversammlungen an diesen Orten einzurichten, maß sie
auch taten. So kam die Kraft des Herrn über alle, und ihre
Erben nahmen von ihr Besitz. Denn unsere Versammlungen
sind von der Kraft Gottes eingesetzt nach dem Evangelium, das
Leben und unvergängliches Wesen ans Licht bringt (2. Tim. 1,10)\bibel{Tim. 2. 01,10@2. Tim. 1,10},
damit alle, die der Teufel\index{Teufel} in Finsternis gebracht hat, wieder
sehend werden, und alle, die Erben des Evangeliums sind, auch in
diesem Evangelium wandeln, und Gott preisen mit Seele, Leib und
Geist, welche sind Gottes. Denn die Ordnungen des herrlichen
Evangeliums sind nicht von Menschen gemacht [...]
Durch Denbigshire und Montgomeryshire kamen wir nach
Merionetshire (Wales). Nachdem wir hier die Monatsversammlungen
eingerichtet, verließen wir Waleß und kehrten nach Shropshire
zurück [...] Dann gingen wir nach Woreestershire, wo wir
eine allgemeine Männerversammlung hatten, in Pafhur, wo ebenfalls
die Monatsversammlungen eingerichtet wurden [...]
In Herefordshire hatten wir mehrere gesegnete Zusammenkünfte. 
Auch hielten wir eine allgemeine Männerversammlung,
in der alle Monatsversammlungen festgesetzt wurden. Es war
gerade eine Verordnung erschienen gegen das Abhalten von Versammlungen. 
A1s wir nun nach Herefordshire kamen, berichtete
man uns von einer großen Versammlung der dortigen Prezbyterianer, 
welche entschlossen waren, alles eher zu ertragen und
auszugeben, als von ihren Versammlungen zu lassen. A1s nun
% \picinclude{./190-199/p_s191.jpg} 
diese Verordnung bekannt geworden sei, so seien die Leute gekommen, 
aber der Priester habe sich davon gemacht und habe sie
im Stich gelassen. Daraufhin kamen sie heimlich in Leominster
zusammen, hielten Brot, Käse und Getränke in Bereitschaft, damit,
wenn die Wachen kommen würden, sie ihre Bibeln bei Seite legen
könnten und sich ans Essen machen. Der Gerichtsdiener kam
ihnen aber auf die Spur, trat unter sie und sagte: "`euer Brot
und Wein hilft euch nicht; gebt eure Redner herans"' Sie 
antworteten: "`was würde dann aus ihren Frauen und Kindern
werden?"' Aber er nahm ihre Redner gefangen und behielt sie
eine Weile. Er erzählte es Peter Young und sagte, dies seien
die ärgsten Heuchler, die je für eine Religion Bekenntnis 
abzulegen suchten.

Ähnliches bewerkstelligten sie an andern Orten. In London\ort{London}
war einer namens Pocock,\person{Pocock} welcher 
Abigail Daray\person{Daray, Abigail} heiratete, eine
sogenannte Dame, und da sie eine Bekännerin der Wahrheit war,
so ging ich in sein Haus, um sie zu besuchen. Dieser Pocock
war ein Erz-Presbyterianer\index{Presbyterianer} und sehr übel gesinnt gegen uns
und pflegte unsre Leute "`Hauskriecher"' (housecreeper\index{housecreeper}) zu nennen.
Als er nun einmal fort war, sagte seine Frau zu mir: "`Ich muss
dir etwas über meinen Mann sagen."' "`Nein,"' sagte ich, "`du
sollst nicht über deinen Mann reden."' "`Doch,"' erwiderte sie,
"`in diesem Falle muss ich es. Am vorigen Ersten Tag hatte
er mit seinen Priestern und Genossen eine Versammlung; sie hatten
Lichter, Tabakspfeifen, Brot und Käse und kaltes Fleisch vor sich
auf dem Tisch, und sie hatten sich verabredet, falls die Beamten
sie überraschen sollten, aufzuhören mit Predigen und Beten und
sich ans- Essen zu machen."' Als ich ihn wieder sah, sagte ich:
"`Ihr, die ihr uns verfolgt und gefangen genommen habt und
unsrer Habe beraubt, weil wir uns eurer Religion nicht anschließen 
wollten, und uns \textit{Kriecher} nanntet, ihr schämt euch nicht,
das ihr nun nicht einmal zu eurer Religion sieht? Habt ihr je
gesehen, das wir uns bei unsern Versammlungen mit Brot und
Käse versahen? oder habt ihr irgendwo in der Schrift gelesen,
das die Heiligen dergleichen taten?"' ,"`Ei,"' sagte der Alte, ,"`wir
sollen ja klug sein wie die Schlangen"' Ich erwiderte: "`Dies
ist allerdings Schlangenklugheit! Wer hätte aber gedacht, das
ihr Presbyterianer und Independenten,\index{Independenten} nachdem ihr solche, die
sich eurem Glauben nicht anschließen wollten, Verfolgtet, gefangennahmt,
% \picinclude{./190-199/p_s192.jpg} 
peitschtet und beraubtet, nun selber zurückweicht und nicht
wagt, zu eurem Glauben zu stehen, sondern denselben mit Hilfe
von Tabakpfeifen, Flaschen, Brot und Käse zu verbergen sucht?"'
Aber ich vernahm später, das solche Heucheleien nur allzuhäufig
betrieben wurden in den Zeiten der Verfolgung.
Als wir in Heresordshire alles die Versammlung Betreffende
geordnet hatten, gingen wir nach Monmouthshire, wo wir mehrere
gesegnete Versammlungen hatten, und bei Walter Jenkins,\person{Jenkins, Walter} einem
früheren Friedensrichter, hatten wir eine große Zusammenkunft
und es wurden mehrere gewonnen. Es war eine ruhige Versammlung; 
in einer früheren hingegen war ein halb betrunkener Gerichtsdiener\index{Andachtsstörung}
erschienen und hatte behauptet, er müsse die Redner abfassen; 
aber die Kraft Gottes war so mächtig gewesen in jener
Versammlung, das sie ihn trotz seines Wütens bannte und er sich
ihr nicht entziehen konnte. Als die Versammlung aus war, war
ich noch ein wenig geblieben und er ebenfalls, ich redete ein
wenig mit ihm und ging dann ruhig weg. In der Nacht kamen
ein paar und schossen mit einer Flinte gegen das Haus, verletzten
aber niemand. So kam die Kraft des Herrn über alle und
band die widerspenstigen Geister, so das wir keinen Schaden
nahmen.

Nun gingen wir nach Gloucestershire\ort{Gloucestershire}, und hatten dort viele
gesegnete Versammlungen in der ganzen Grafschaft herum, und
zuletzt gingen wir weiter nach Bristol,\ort{Bristol} wo nach einer sehr 
ersprießlichen Zeit die Männer und Frauenversammlungen ebenfalls eingerichtet wurden.
Einmal als ich in Bristol in meinem Bett war, geschah das
Wort des Herrn zu mir, ich solle wieder nach London zurückgehen.
Am folgenden Morgen kam Alexander Parker\person{Parker, Alexander} und einige andere
zu mir. Ich fragte sie, was ihnen sei? und ebenso fragten sie
mich, was mir sei? Ich sagte ihnen, ich fühle, das ich nach
London zurückkehren müsse. Sie sagten, gerade so sei es auch
ihnen. So ergaben wir uns drein, nach London zu gehen; denn
welchen Weg auch der Herr uns führte, wir gingen ihn in seiner
Kraft. Wir gingen über Wiltshire\ort{Wiltshire} und ordneten dort die 
Monatsversammlung für Männer, in der Kraft des Herrn, und besuchten
die Freunde, bis wir nach London kamen.

Nachdem wir die Freunde in der Stadt besucht hatten, trieb
es mich, sie zu ermahnen, alle ihre Eheschließungen\index{Heirat} vor die 
% \picinclude{./190-199/p_s193.jpg} 
Versammlungen der Männer und Frauen zu bringen, um sie den
Gläubigen vorzulegen. Diese Vorsorge möge getroffen werden,
um Unordnungen zu verhüten, wie solche von etlichen begangen
worden waren. Denn viele hatten sich gegen den Willen der
Ihrigen Verheiratet, und einige junge Leute, die sich zu uns hielten,
hatten sich mit solchen, die der Welt angehörten,\index{Die der Welt angehörten} verbunden;
Witwen hatten sich wieder verheiratet, ohne Fürsorge zu treffen
für ihre Kinder, trotz meiner Schrift über das Heiraten, die ich
im Jahre 1653\index{Jahr!1653} veröffentlicht hatte, als die Wahrheit noch wenig
verbreitet war. Ich hatte darin die Freunde, für die es in
Betracht kam, ermahnt, die Sache doch ja immer den Gläubigen
vorzulegen, ehe sie etwas abmachten, und erst danach bekannt
zu machen, auf dem Markte oder in der Versammlung, je nachdem 
es sie triebe. Und wenn dann alles ins Reine gebracht
worden sei, wenn sie frei seien von jeder anderweitigen 
Verpflichtung, und ihre Angehörigen einverstanden, so sollten sie eine 
Versammlung bestimmen, in der sie sich dann, in Gegenwart von
mindestens zwölf Zeugen, zur Ehe nehmen. Da nun diese Vorschriften
nicht befolgt wurden, und die Wahrheit sich weiter im Lande
ausgebreitet hatte, so wurde in der Kraft und dem Geist des
Herrn verordnet, das die Eheschließungen den vierteljährlichen
und den monatlichen Versammlungen der Männer vorgelegt werden
sollten. Die Freunde sollten dafür sorgen, das die Angehörigen
beider Teile einverstanden seien, und das die Witwen Bestimmungen
getroffen haben für die Kinder aus erster Ehe, ehe sie wieder
heiraten, und was es sonst noch zu ordnen gibt, damit alles 
geschehe in Reinheit und Gerechtigkeit, zur Ehre Gottes. Später
wurde verordnet durch die Weißheit Gottes, das, wenn der eine
Teil aus einer andern Gegend oder aus einem andern Land komme
oder einer anderen Monatsversammlung zugehörte, so solle er eine
Bescheinigung bringen von der Versammlung, der er zugehörte,
als Gewähr bei der Monatsversammlung, der sie ihre Absicht,
sich zu heiraten, vorlegen.

Nachdem diese Angelegenheit, sowie viele andere Dienste für
Gott, in Ordnung gebracht und geregelt waren in den 
verschiedenen Stadtgemeinden, verließ ich London und ging, wie mich
die Kraft des Herrn führte, nach Hertfordshire\ort{Hertfordshire}. Nachdem ich
viele Freunde dort besucht hatte, und die Monatsversammlungen
für Männer dort geordnet waren, hatte ich eine große 
% \picinclude{./190-199/p_s194.jpg} 
Versammlung in Baldock\ort{Baldock} mit allen möglichen Leuten. Darauf kehrte ich
nach London zurück über Waltham,\ort{Waltham} wo ich ihnen riet, eine Schule\index{Schule}
für Knaben einzurichten, sowie auch in Shacklewell\ort{Shacklewell} eine Schule\index{Quakerschule}
für Mädchen, um sie in allem Guten und Nützlichen zu unterrichten [...]

Wir zogen durch Gloucestershire\ort{Gloucestershire} und besuchten die Freunde,
dann kamen wir nach Monmouthshire,\ort{Monmouthshire} wo wir mit Vertretern
aller Versammlungen des ganzen Bezirks zusammentrafen und in
der Kraft des Herrn auch hier die Monatsversammlungen 
einrichteten, damit alle die Herrlichkeit Gottes feiern möchten und
die, welche nicht nach dem Evangelium wandeln, ermahnt und
zurechtgewiesen würden. Und wirklich bewirkten diese 
Versammlungen eine große Besserung unter den Leuten, so das die 
Obrigkeit ihren Nutzen einsah.\index{Monatsversammlungen, Sinn und Nutzen der }

Wir kamen an einen Ort in der Nähe von Minehead\ort{Minehead}, wo
wir eine große allgemeine Männerversammlung für alle Freunde
von Somersetshire\ort{Somersetshire} hatten. Es war auch einer dabei, ein
Schwindler, von dem einige gutmütige Leute gemeint hatten, ich
solle ihn bleibend zu mir nehmen; aber ich sah, das er ein
Schwindler war, und hieß sie darum, ihn zu mir bringen, damit
ich sehe, ob er mir ins Gesicht sehen könne. Er konnte es
nicht, sondern blickte unruhig hin und her. Er hatte einen
Priester betrogen, indem er ihn glauben machte, er sei ein 
Prediger, hatte sich sein Priesterkleid verschafft und sich in demselben
davon gemacht.

Nach der Versammlung gingen wir weiter nach Minehead,\ort{Minehead}
wo wir rasteten. In der Nacht musste ich ringen mit einem
Geist der Finsternis, der sich gegen die Kirche Christi erheben
wollte, um sie in Verwirrung zu bringen.\index{Anfechtung/Versuchung} Am folgenden Morgen
trieb es mich, einige Zeilen an die Freunde zu schreiben, um sie
zu warnen.

\brief{Quaker-Gemeinde}{
  Liebe Freunde,
  \medskip 
  Lebet in der Kraft Gottes des Herrn, und in seinem Samen,
  der größer ist als alle Versuchungen, die der Geist der Finsternis
  euch anhaben kann, welcher euch ihm Untertan machen und sich
  unter euch erheben möchte; er ist noch nicht gekommen, aber in
  der Kraft Gottes und seines Samens haltet euch über demselben
  und verdammet ihn. Denn ich fühlte einen Geist der Finsternis in
  der vergangenen Nacht, der suchte sich zu erheben und unter euch
  % \picinclude{./190-199/p_s195.jpg} 
  aufzustehen; aber ihr könnet ihn bezwingen mit Gottes Kraft und
  sein Treiben verdammen, ehe er irgendwo Eingang gefunden
  hat. Mehr will ich nicht sagen; meine Liebe im Samen Gottes,
  in welchem kein Wechsel ist."'

  \begin{flushright}
  Minehead in Somersetshire\ort{Somersetshire}, 22. des 4. Monats, 1668.\index{Jahr!1668}

  G. F.
  \end{flushright}
}

Nachdem wir die meisten Versammlungen in Somersetshire
besucht hatten, gingen wir weiter nach Dorsetshire\ort{Dorsetshire} 
zu Georg Harris,\person{Harris, Georg}
in dessen Haus wir eine große Versammlung für Männer hatten.
Hier wurden nun alle Monatsversammlungen für Männer für
den ganzen Bezirk geordnet nach den herrlichen Geboten des
Evangeliums, auf das alle möchten in der Kraft Gottes "`das
Verlorene suchen und das Verirrte wieder holen"' (Hes. 34,4\bibel{Hes. 34:04@Hes. 34:4}),
das Gute ehren und das Böse strafen.

Hierauf kamen wir nach Southampton,\ort{Southampton} wo wir am Ersten
Tage eine große Versammlung hatten. Von da gingen wir zu
Hauptmann Reeves,\person{Hauptmann Reeves} 
wo die allgemeine Versammlung für Männer
für Hampshire stattfand; Es waren viele aus der ganzen Grafschaft 
gekommen, und wir hatten eine gesegnete Zeit. Die Monatsversammlungen 
der Männer für diese Grafschaft wurden geordnet
nach den Vorschriften des Evangeliums, welches Leben und
unsterbliches Wesen in ihnen ans Licht gebracht hatte. Da erschien
eine Bande Ranter,\index{Ranter} die unsre Versammlung recht störten und
sich derselben widersetzten.\index{Versammlung!Störung}

Es war eine Frau dabei, die bei einem Mann gelegen hatte;
dieser erzählte es nun auf dem Marktplatz und rühmte sich
seiner Schlechtigkeit; eine Anzahl dieser liederlichen Leute wohnte
zusammen in einem Haus, ganz nahe bei dem Ort, wo wir unsere
Versammlungen hatten. Ich ging zu ihnen und hielt ihnen ihre
Schlechtigkeit vor. Der Herr des Hauses sagte: nun! warum mich
denn das so sehr erstaune? Ein andrer sagte, es werde mich
wohl auch straucheln machen! Ich erwiderte ihnen, ihre
Schlechtigkeit werde mich nicht zum Straucheln bringen, denn ich stehe
über derselben. Und der Herr trieb mich, ihnen zu sagen, das
die Strafen und das Gericht Gottes über sie kommen werden.
Sie zogen später im Land herum, bis sie schließlich ins Gefängnis
in Winchester\ort{Winchester} geworfen wurden, wo der Mann, der bei der Frau
gelegen hatte, nach dem Kerkermeister stach, ihn jedoch nicht tötete.
Als sie dann aus dem Gefängnis entlassen waren, erhängte sich
% \picinclude{./190-199/p_s196.jpg} 
der Mann, der den Kerkermeister erstechen wollte; die Frau hätte
auch fast einem Kinde den Hals abgeschnitten, wie wir hörten.
Diese Leute hatten früher in der Nähe von London gelebt, und als
die Stadt brannte, prophezeiten sie, das dass ganze übrige London
innerhalb vierzehn Tagen verbrennen würde und flohen aus der
Stadt. Diese Ranter nun, große Gegner der Freunde, und Störer
unsrer Versammlungen, wurden zuweilen in der Gegend, wo die
Leute sie nicht kannten, für Quäker gehalten. Darum trieb mich
der Herr, ein Schreiben zu verfassen, das unter den Behörden
und dem Volk in Hampshire\ort{Hampshire} verbreitet werden sollte, damit man
sehe, das die Wahrheit und die Freunde mit diesen liederlichen
Leuten nichts zu tun haben [...]

So waren nun im ganzen Lande die monatlichen Männerversammlungen 
geordnet. Denn in Berkshire war ich früher
gewesen, damals, als die meisten der ersten Freunde im
Gefängnis waren; ich hatte ihnen den Nutzen dieser 
Monatsversammlungen auseinandergesetzt, und sie hatten sie daraufhin auch
eingerichtet. Auch nach Irland\ort{Irland} und 
Schottland,\ort{Schottland} nach Holland,\ort{Holland}
Barbadoes\ort{Barbadoes} und mehrere Orte in Amerika, sandte ich 
durch zuverlässige Freunde Schreiben, um die Freunde zu ermahnen, ihre
monatlichen Männerversammlungen überall zu ordnen. Vierteljährliche 
Versammlungen hatten sie schon vorher gehabt; aber
jetzt, da die Wahrheit sich unter ihnen verbreitet hatte, sollten sie
auch Monatsversammlungen einrichten in der Kraft Gottes, durch
die sie bekehrt worden waren. Seit diese Versammlungen eingerichtet 
worden sind, und die Getreuen des Herrn, die Erben des
Evangeliums sich versammeln, in der Kraft ihres Meisters, aus
die sich diese Versammlungen gründen, haben viele ihren Mund
aufgetan in Dank und Lobpreisung, und viele haben dem Herrn
mit Tränen gedankt, das er mich in seinem Dienst ausgesandt
hatte. Alle, denen Gottes Ehre und Herrlichkeit am Herzen liegt,
alle, denen es ein Anliegen ist, das sein Name, den sie bekennen,
nicht gelästert werde und das, wer die Wahrheit bekennt, auch
in der Wahrheit, Gerechtigkeit und Heiligkeit wandelt, können nun
das Reich Christi,\index{Reich Christi} dessen Wachstum kein Ende hat, 
kennen und sehen, besitzen und daran teil haben. Der ewige Ruhm und Preis
Gottes ist in jedem Herzen, das treu ist, eingepflanzt; wir
dürfen sagen, das die Ordnung des Evangeliums unter uns
nicht von Menschen, noch durch Menschen, sondern von und
% \picinclude{./190-199/p_s197.jpg} 
durch Jesus Christus; und durch den heiligen Geist ausgerichtet
wurde [...]

Nach London\ort{London} zurückgekehrt, blieb ich einige Zeit dort, um die
Freunde in der Stadt und der Umgegend zu besuchen. Einmal
ging ich zu Esquire Marsh,\person{Marsh, Esquire} der mir und den Freunden viel
Freundlichkeit erwiesen hatte; es traf sich, das er gerade am
Mittagessen war, als ich kam. Kaum hatte er meinen Namen
gehört, so ließ er mich herauf holen und wollte, das ich mich mit
ihm zu Tisch setze; aber ich hatte nicht die Freiheit, es zu tun.
Es waren mehrere hochgestellte Personen mit ihm bei Tisch, und
er sagte zu einem von ihnen, einem angesehenen Papisten:\index{Papisten} "`Hier
ist ein Quäker, den ihr noch nie gesehen habt."' Der Papist fragte
mich, ob ich die Kindertaufe anerkenne?\index{Kindertaufe} Ich erwiderte ihm, es
stehe nichts in der Bibel davon. "`Wie,"' sagte er, "`nichts über
die Kindertaufe?"' Ich sagte: "`nein."' Ich sagte ihm: "`Wir
anerkennen die Eine Taufe durch den Einen Geist in dem Einen
Leib (Kor. 12);\bibel{Kor. 12} jedoch dafür, das man ein wenig Wasser einem
Kinde übers Gesicht schüttet und sagt, das sei nun das Kind taufen
und zu einem Christen machen, gibt es kein Bibelwort."', Ferner
fragte er mich, ob ich den katholischen Glauben anerkenne? Ich
antwortete "`ja,"' fügte aber hinzu, "`weder der Papst noch die
Papisten haben den katholischen Glauben; denn der wahre Glaube
wirket in der Liebe und reinigt das Herz; wenn ihr den Glauben
hättet, welcher den Sieg gibt, und durch den man den Zugang
zu Gott hat, so würdet ihr den Leuten nicht von einem Fegefeuer\index{Fegefeuer}
nach dem Tode reden."'\index{Katholischer Glaube} Ich suchte 
nun zu beweisen, das kein Papst und kein Papist, welcher ein 
Fegefeuer nach dem Tod
annehme, den wahren Glauben habe; denn der wahre, herrliche,
göttliche Glaube, dessen Anfänger Christus ist, gibt den Sieg über
Teufel und Sünde, die den Menschen von Gott getrennt haben.
Wenn sie, die Papisten, den wahren Glauben hätten, so würden
sie nicht solche, die einen andern Glauben haben, verfolgen und\index{Verfolgung}
mit Foltern, Gefängnissen und Geldbußen ihnen ihren Glauben
aufzwingen. Das sei nicht die Art der Apostel\person{Apostel} und ersten Christen
gewesen, die den wahren Glauben Christi besaßen und bezeugten;
sondern die irgläubigen Juden\indexname{Juden} und Heiden machten es so. "`Wenn
Du,"' sagte ich, "`ein Haupt und Führer der Papisten, 
aufgewachsen und erzogen in der Lehre des Papstes, sagst, es gebe kein
Heil außer in eurer Kirche, so möchte ich gerne wissen, was denn
% \picinclude{./190-199/p_s198.jpg} 
in eurer Kirche das Heil bringt?"' Er antwortete: "`Ein gutes
Leben."' "`Sonst nichts?"' sagte ich. "`Doch,"' sagte er, "`gute
Werke."' "`So, daß bringt eurer Kirche Heil,"' sagte ich, "`ein
gutes Leben und gute Werke! das ist also eure Lehre und euer
Grundsatz! Dann wissen weder du, noch der Papst, noch irgend
ein Papist, woher das Heil kommt."' Darauf fragte er mich,
woher denn das Heil in unsrer Kirche komme? Ich sagte ihm:
"`nichts anderes, als was in den Tagen der Apostel das Heil der
Kirche war, ist es auch für unsre Kirche, nämlich, "`die heilsame
Gnade Gottes, die allen Menschen erschienen ist"' (Tit. 2,11).\bibel{Tit. 02:11@Tit. 2:11}
Wie sie einst die Heiligen lehrte, so lehrt sie jetzt uns: "`zu 
verleugnen das ungöttliche Wesen und die weltlichen Lüste, und
gottselig, gerecht und züchtig zu leben in der Welt"' (Tit.2,12).\bibel{Tit. 02:12@Tit.2:12} 
Es sind also weder die guten Werke noch ein gutes Leben, die daß
Heil bringen, sondern die Gnade."'\index{Rechtfertigung} "`Und diese heilsame Gnade
erscheint allen Menschen, sagt ihr?"' rief der Priester. "`Ja,"'
erwiderte ich. "`Das gebe ich euch nicht zu!"' rief er. Ich 
antwortete: "`Alle, die es nicht zugeben, sind Sektierer und haben
nicht den allumfassenden Glauben der Apostel."'\index{Sektierer}

Darauf redete er über die Mutter-Kirche. Ich sagte ihm,
alle die verschiedenen Sekten im Christentum hätten uns 
vorgeworfen, wir verließen die Mutterkirche. Die Papisten warfen
uns den Abfall von der Mutterkirche vor, mit der Behauptung,
Rom sei diese einzige Mutterkirche. Die Bischöflichen beschuldigten
uns des Abfalls vom alten protestantischen Glauben, indem sie
geltend machten, sie hätten die reformierte Mutterkirche. Die
Presbnterianer und Independenten schalten uns, das wir sie 
verlassen, indem beide behaupteten, sie hätten die wahre reformierte
Mutterkirche. "`Allein,"' sagte ich, "`wenn wir irgend einen äußeren
Ort als Mutterkirche anerkennen würden, so wäre es Jerusalem,\ort{Jerusalem}
wo das Evangelium zuerst verkündet wurde durch Christus selbst
und seine Apostel, wo Christus litt, wo die große Vekehrung
zum Christentum durch Petrus stattfand, der Ort der prophetischen
Zeichen und Wunder, die in Christus ihre Erfüllung hatten, und
wo er seinen Jüngern befahl, "`zu warten, bis das sie angetan
würden mit der Kraft aus der Höhe"' (Luk. 24,49)\bibel{Luk. 24:49}. 
Wenn irgend ein äußerer Ort verdiene, eine Mutterkirche genannt zu werden, 
so sei es derjenige, an dem die erste große Bekehrumg zum Christentum 
stattfand. Aber der Apostel sagt, 
Gal. 4,25—27\bibel{Gal. 04:25—27@Gal. 4:25—27}: "`Das
% \picinclude{./190-199/p_s199.jpg} 
Jerusalem, das zu dieser Zeit ist, ist dienstbar mit seinen Kindern.
Aber das Jerusalem, das droben ist, ist die Freie, die ist unser
aller Mutter. Sei fröhlich, du Unfruchtbare, und juble die du
nicht schwanger bist; denn die Einsame hat mehr Kinder als
die den Mann hat."' Der Apostel sagt nicht, das sichtbare
Jerusalem sei die Mutter, obgleich die erste und große Bekehrung
zum Christentum dort stattfand. Und noch weniger berechtigt ist, das
Rom\ort{Rom} oder sonst ein Ort oder eine Stadt so bezeichnet werde von
den Kindern des freien, oberen Jerusalem; auch sind solche nicht
Kinder des freien, oberen Jerusalem, welche das sichtbare Jerusalem 
oder Rom oder irgend einen andern Ort oder eine Sekte
ihre Mutter nennen. Und obgleich von entarteten Christen vielen
Orten und Sekten dieser Titel gegeben wurde, so sagen wir
dennoch wie einst der Apostel: "`Das Jerusalem, das droben ist,
das ist die Freie, die ist unser aller Mutter."' Und wir können
kein anderes Jerusalem, noch ein Rom, noch irgend eine Sekte
als unsre Mutter anerkennen, sondern allein das Jerusalem, das
droben ist, die Freie, die Mutter aller derer, die wiedergeboren\index{wiedergeboren}
sind und wahrhaft an das Licht glauben und eingepflanzt sind
in Christus, den himmlischen Weinstock. Denn alle, welche 
wiedergeboren sind aus dem unvergänglichen Samen durch das Wort
Gottes, welches ewiglich bleibet, nähren sich von der Milch des
Wortes, an den Brüsten des Lebens und wachsen und nehmen
zu durch dieselbe und können keine andere Mutter anerkennen,
als das Jerusalem, welches droben ist. "`O,"' sagte Esquire
Marsh zu dem Papisten, "`Ihr wisset nicht, was für ein Mann
der ist; wenn er nur hier und da in die Kirche kommen wollte,
so wäre er ein ausgezeichneter Mensch."'

Ich nahm Marsh beiseite, um wegen der Freunde mit ihm
zu reden; er war Friedensrichter von Middlesex,\ort{Middlesex} und da er an den
Hof kam, übertragen ihm die andern Richter die Leitung mancher
Geschäftes. Er sagte mir, das er in Verlegenheit sei, wie er zu
unterscheiden habe zwischen uns und einigen Dissentern.\index{Dissentern} "`Denn,"'
sagte er, "`ihr könnt nicht schwören und die Independenten, die
Baptisten und die Fifth-Monarchy-Leute sagen ebenfalls, sie können
nicht schwören; wie soll ich denn nun zwischen euch und ihnen 
unterscheiden, da ihr allesamt sagt, ihr könnt um des Gewissens willen
nicht schwören?"'\index{Schwören}\index{Quaker von Anderen unterscheiden.} 
Ich antwortete ihm: "`Ich will dir zeigen, wie
du uns unterscheiden kannst. Jene, wenigstens die meisten, der

% \picinclude{./200-209/p_s200.jpg} 
von dir Genannten, können in einzelnen Fällen schwören und tun
es auch; wir dagegen schwören in keinem Fall. Wenn man jenen
ihre Kühe oder Pferde stehlen würde und du würdest sie fragen,
ob sie schwören wollen, das es die ihrigen seien, so wären manche
von ihnen gleich bereit, es zu tun. Wenn du es aber bei unsern
Freunden versuchst, so schwören sie auch um ihrer Habe willen
nicht.\index{Schwören} Darum, wenn du von einem von ihnen den Huldigungseid
forderst, so musst du ihn fragen, ob er in andern Fällen schwören
kann, z. B., wenn es seine Kuh oder sein Pferd betreffe, was er,
wenn er wirklich zu uns gehört, nicht kann."' Darauf erzählte
ich ihm folgendes aus einem Verhör in Berkshire:\ort{Berkshire} 
"`Ein Dieb stahl einem unsrer Freunde zwei Stück Vieh; der Dieb wurde erwischt
und ins Gefängnis gebracht, und der Freund trat vor Gericht
als Kläger\index{Quaker als Kläger} gegen ihn aus. Da aber der Richter gehört hatte,
das der Kläger ein Quäker sei und nicht schwören könne, so rief
er, ohne erst anzuhören, was der Freund zu sagen hatte: "`Ist
er Quäker, will er nicht schwören?"' und legte ihm den Huldigungseid 
vor. Und darauf schickte er den Freund ins Gefängnis und unterwarf 
ihn den Strafen gegen Eidverweigerung. Der Dieb aber, der
ihn bestohlen hatte, wurde freigesprochen!"' Richter Marsh sagte:
"`Dieser Richter war ein schlechter Mensch."' "`Wenn wir irgend
schwören könnten,"' sagte ich, "`so wäre es dem König, der zum
Schutz seiner Untertanen die Gesetze aufrecht erhält. Die andern
dagegen, die doch schwören, wenn es gilt, ihr Hab und Gut zu
schützen, weigern sich gerade, dem König den Huldigungseid zu
schwören. Von diesen kannst du uns also leicht kennen und unterscheiden."'

Richter Marsh hat später sehr viel für die Freunde getan,
indem er viele von ihnen vor den Strafen wegen Eidverweigerung
schützte. Wenn während der Verfolgungen Freunde vor ihn gebracht
wurden, so schenkte er ihnen, wenn möglich die Freiheit, und wenn 
er es nicht verhindern konnte, das sie ins Gefängnis kamen, so
machte er, das es nicht für länger als ein paar Stunden oder
für eine Nacht sei. Schließlich ging er zum König und sagte
ihm, er habe einige von uns gegen sein Gewissen ins Gefängnis
geschickt und wolle das fernerhin nicht mehr tun. Er verließ
darum mit seiner Familie Limehouse\ort{Limehouse} und zog in die Nähe von
St. James Park.\ort{St. James Park} Er sagte dem König, wenn er sich dazu 
verstehen könnte, Gewissensfreiheit zu erklären, so würde alles ruhig
% \picinclude{./200-209/p_s201.jpg} 
werden; denn dann könnte man nicht mehr Gewissenskrupeln
geltend machen. Ja, er hat in jenen Tagen viel für die Wahrheit 
und die Freunde getan.

