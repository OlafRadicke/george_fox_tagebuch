
%%%%%%%%%%%%%%%%%%% Kapitel 28. %%%%%%%%%%%%%%%%%%%%%%%%%%%%%%

\chapter[Krankheit und Tod]{Krankheit und Tod}

\begin{center}
\textbf{Ahnung kommender Revolutionen. Christus König. Letzte
Arbeiten. Krankheit und Tod.}
\end{center}

Jm gz. Monat des Jahres 1688 war ich in London .....
Ich war noch nicht lange da, alö es mir sehr schwer nmz Herz
wurde, und ich vom Herrn ein Gesicht hatte von den großen
Unruhen und Trübsalen, Revolutionen und Umwälzungen, die bald
daraus sich vollzogen. Unter diesem Eindruck rmd getrieben vom
Geist dez Herrn, schrieb ich einen Generalbrief an die Freunde,
um sie vor dem nahenden Sturm zu warnen, auf daß sie alle
ihre Zuflucht beim Herrn suchen möchten:
»Lieben Freunde und Brüder allenthalben, die ihr den Herrn
Jesus Christnö angenommen habt, und denen er Macht gegeben
hat, seine Söhne und Töchter zu werden, .... zeiget euch, trotz
der Fluten und Stürme, die die Welt bewegen, als- die unschul-
digen sansten Lämmlein Christi, die in seiner friedsamen Wahrheit
wandeln und in dem Wort seiner Kraft, Weißheit und Geduld
bleiben; dieses Wort wird euch bewahren in den Tagen der


% \picinclude{./300-309/p_s307.jpg} 
Ahnung kommender Revolutionen. Christus3yKönig usw. 307
Heimsuchungen und Versuchungen, welche über die ganze Erde
kommen werden, um alle, die aus der Erde sind, zu prüfen- Denn
das Wort des Herrn war, ehe die Welt gewesen, und alle Dinge
sind durch dasselbe gemacht (Joh. 1); ez ist ein erprobteö Wort,
das zu allen Zeiten dem Volke Gottez Weißheit, Kraft und Ge-
duld verlieh. Darum bleibet und wandelt in Christus Jesuß,
der das Wort Gotteß genannt wird, und in seiner Kraft, die über
allem ist. Suchet, wa-J droben ist, wo Christuö sitzet zur Rechten
Gottes (Kol. 3,1), .... und suchet nicht, maß Oergänglich ist.
Gepriesen sei der Herr, der sich mit seinem ewigen Arm und ,
seiner Kraft ein einiges Volk zubereitet hat, und sich dasselbe treu
erhalten hat durch viel Triibsal, Prüfungen und Versuchungen,
seine Kraft und sein Same, Christus, ist über allem, und in ihm
habet ihr Leben und Frieden in Gott. Darum stehet alle fest in
ihm und sehet in ihm eure Erlösung, den Anfang und das Ende,
das Amen. Der Allmächtige bewahre und erhalte euch alle in
ihm, eurer Arche und eurem Llllerheiligsten; in ihm seid ihr sicher
vor allen Fluten und Stürmen, denn er war, ehe sie waren,
und wird sein, wenn sie nicht mehr sind.«
London, 17. des 8. Monats- 1688. G. F.
EZ kam um diese Zeit eine große Traurigkeit und Schwer-
mut über mich, wie dietz fast jedeömal vor einer großen Um-
wälzung und Änderung der Regierung gewesen war, und meine
Kräfte verließen mich, sodaß ich vor Schwäche fast zusammenbrach,
wenn ich durch die Straßen ging. Zuletzt konnte ich eine Zeit-
lang gar nicht mehr außgehn, so schwachllwar ich, biz ich fühlte,
daß die Kraft Gottes über allem ausging, und er mir die Gewißheit
gab, daß er seine Gläubigen durch alles hindurch bewahren werde.  
Da ich fortwährend leidend war, so ging ich mit meinem
Sohn Mead in sein Hauö nach Essex und blieb einige Wochen
dort. Ich schrieb Verschiedeneß während dieser Zeit, unter anderm
folgende Zeilen:
,,Während die Menschen hienieden sich um Throne streiten,
sitzt Christuö auf seinem Thron und seine heiligen Engel sind um
ihn. Er ist Anfang rmd Ende, der Erste und der Letzte, über
allem. Und der Herr wird Mittel und Wege schaffen für alle,
die aus reinem heiligen Geiste geboren und Kinder des himm-
lischen Jerusalem sind, damit sie heim kommen zu ihrer wahren
Mutter« ....
20*


% \picinclude{./300-309/p_s308.jpg} 
Ferner verfaßte ich eine Schrift, in der ich an Beispielen
aus der Schrift zeigte, daß viele heilige Männer Gottes, Pro-
pheten und Apostel Christi Handwerker und Ackerbauer ge-
wesen waren, damit man den Unterschied zwischen diesen und den
jetzigen Lehrern der Welt sehe:
,,Der gerechte Abel war ein Hirte und hütete die Schafe
(1. Mos. 4) .... Abraham war ein Ackersmann und hatte
große Herden von Schafen (1. Mos. 13) ..... Jakob war ein
Arkersmann (1. Mos. 30) ..... Moses hütete die Schafe
(2. Mos. 3) ..... David hütete die Schafe seines Vaters in
der Wildnis (1. Sam. 16) ..... Elisa war ein Ackersmann;
er wurde vom Pfluge abberufen, um das Volk Gottes zu lehren
(1. Kön. 19,19) ..... Zu Amos geschah das Wort des Herrn,
als er bei den Hirten war (Amos. 1,1) ..... Petrus und
Andreas berief Christus, als sie fischten (Matth. 4,18) . . . und
Matthäus sah er am Zoll sitzen und sagte: »Folge mir nach''
(Matth. 9,9) ..... Paulus war ein Zeltweber, und weil er
das gleiche Handwerk betrieb wie Aquila und Priseilla, so wohnte
er bei ihnen in Corinth und wob« (Act. 18,3).
Gooses, 1. Monat 1689. G. F.
Viel zu viel Zeit wurde damals zugebracht mit Anhören und
Verbreiten von Neuigkeiten und sonstigem Geschwätz ..... Um die
Nichtigkeit solches Treibens zu zeigen und davon abzumahnen,
schrieb ich folgendes:
,,Hienieden in diesem vergänglichen Leben sind alle Neuig-
keiten unsicher, nichts ist gewiß, aber im Reich Christi sind alle
Dinge beständig und sicher, und alle Neuigkeiten stets gut und
sicher. Denn Christus, dem alle Gewalt im Himmel und aus
Erden gegeben ist, regieret der Menschen Reiche, und er, welcher
der Erbherr ist über alle Heiden und über die Enden der Erde,
herrscht in seiner himmlischen Kraft und seinem Licht, er regieret
alle Völker mit seinem eisernen Szepter und schlägt sie in Stücke
wie die Gefäße eines Töpfers (Ps- 2,9, Offb. 2,27), wenn es
Gefäße zur Unehre und löchrichte Gefäße sind, die ein lebendiges
Wasser nicht halten können (Jer. 2,13), seine auserwählten Gefäße
der Ehre und Gnade jedoch behütet er. SeineMacht ist unerschütterlich
und verändert sich nicht, durch sie bewegt er Berge und Hügel
und macht Himmel und Erde erzittern. Löchrichte, schlechte Ge-
fäße, Hügel und Berge und die alten Himmel und Erden, müssen


% \picinclude{./300-309/p_s309.jpg} 
Ahnung kommender Revolutionen. Christus König usw. 309
alle erschüttert und zerschlagen werden, obgleich sie es nicht
merken, noch ihn sehen, der es tut. Seine Auserwiihlten und
Getreuen aber sehen es und kennen ihn und spüren seine Macht
welche nicht erschüttert werden kann und sich nicht verändert«.
5. des 1. Monats 1689. G. F.
Etwa um die Mitte des ersten Monats 1689 ging ich’;nach
London, das Parlament tagte gerade und beschäftigte sich mit
dem Jndulgenzgesetz. Obschon ich mich schwach fühlte und nicht
gut herumlaufen konnte, nahm mein Geist doch so lebhaft Anteil
an der Sache der Wahrheit und der Freunde, daß ich während
vieler Tage mit einigen Freunden den Verhandlungen beiwohnte
und mit den Mitgliedern verhandelte, damit die Sache nachdrück-
lich und wirksam ausfalle .....
So fuhr ich fort in allerlei Arbeit bis gegen das Ende des
zweiten Monats, wo ich dann, ermiidet von der vielen Arbeit,
die Stadt verließ und nach Southgate ging. ,... z Dort schrieb
ich unter anderm einen Brief an die Freunde in Danzig, die da-
mals unter schweren Verfolgungen litten. Joh errnutigte sie darin,
an ihrem Zeugnis festzuhalten und geduldig ihre Leiden zu er-
tragen. Zu gleicher Zeit schrieb ich aber auch an die Behörden
von Danzig und stellte ihnen das Unrecht dieser Verfolgungen vor:
,,Wir haben eure Verordnungen gesehen und euer Wut-
schnauben gegen jenes kleine Trüppchen, die Lämmlein Christi,
die unter eurer Gerichtsbarkeit in der Stadt Danzig leben. Zweie
habt ihr gefangen nehmen und verbannen lassen, und andern
drohtet ihr mit gleicher Strafe für den Fall, daß sie in die Stadt
zurückkommen. Ebenso verhängt ihr Strafen über solche, die
unsern Freunden ihre Häuser anbieten, um darin zu wohnen oder
um Versammlungen oder Gottesdienst darin zu halten. Wahrlich,
ich bin herzlich betrübt, sowohl über eure Behörden als über eure
Priester, die sich Christen nennen lassen und doch solche unmora-
lischen, unmenschlichen und unchristlichen Handlungen begehen,
weit entfernt vom königlichen Gesetz, welches befiehlt: ,,Tut andern
wie ihr wollt, daß man euch tut«. Witrdet ihr es moralisch,
menschlich, christlich und dem königlichen Gesetz entsprechend finden,
wenn der König von Polen, der eine andere Religion als ihr
hat, euch durch den Henker aus der Stadt verbannen ließe und
euch Seelenmörder nennen würde? Würdet ihr solches nach dem
Gesetz«Gottes finden, das befiehlt: ,,andern zu tun, wie man will,

% \picinclude{./310-319/p_s310.jpg} daß sie uns tun''? Und wenn ihr sagt, ihr habet daß Schwert,
die Macht und die Gewalt, so sagen wir: gepriesen sei der Herr,
der euer Schwert, eure Gewalt und eure Macht gekürzt hat, so-
daß es nicht über eure Gericht?-barkeit in Danzig hinaus reicht,
und ihr wisset nicht, wie lange der Herr euch euer Schwert, eure
Gewalt und eure Macht läßt. Wir sind überzeugt, daß ihr nicht
den Geist Christi habt, und der Apostel sagt: ,,Wer Christi Geist
nicht hat, der ist nicht sein« (Röm. 8). Und Chriftuß gebot dem
Petruö, »stecke dein Schwert in die Scheide«. Die, welche um
seinet willen das Schwert zogen, um ihn zu verteidigen, sollten
durchZ Schwert umkommen. Petrusz und die Apostel zogen später
nie mehr das Schwert, sondern sagten, ihre Waffen seien nicht
fleischlich, sondern geistlich und sie kämpfen nicht mit Fleisch und
Blut (Epl). 6). Christus hat den Seinen nie ein Gebot gegeben,
daß sie jemand durch den Henker verbannen sollten, der nicht
ihre Religion habe oder annehmen wolle. Seid ihr nicht ärger
alz die Türken, die dulden, daß allerlei Religionen, sogar Christen,
in ihrem Lande wohnen und sich friedlich versammeln? Ja, die
türkischen Statthalter lassen unsre Freunde, die gefangen waren,
in Algier Versammlungen haben und sagen, sie tuen gut daran.
Jhr seid ärger, als die maurischen Barbaren, die sich gar nicht
zum Christentum bekennen, denn ihr bekennet Christuz in Worten,
aber in euerm Tun verleugnet ihr ihn. Habt ihr je in der Schrift
oder in der Geschichte gesehen, daß E den Verfolgern lange wohl
ging? Jhr seid ärger, alö die im Lande dez Mogulö, welcher,
wie man sagt, sechzig Religionen in seinem Lande zuläßt. Noch
viele andere könnten genannt werden, die ihr alle übertresft mit
eurer Grausamkeit und eurem Verfolgen des Volkeö Gottez, nur
wcil es sich im Namen Jesu versammelt und Gott seinen Schöpfer
anbetet und ihm dient. Sie dürfen nicht einmal frei atmen in
euerm Land, weder leiblich noch geistig. Sagt, wo habt ihr
solches Gebot her? Weder von Christus, noch von seinen Aposteln.
Behauptet ihr nicht, die Schriften des Neuen Testaments seien
euer Gesetz? Aber ich frage euch, was habt ihr für eine Schrift-
stelle für solches Tun? Jhr würdet gut tun, demütig zu sein,
recht zu tun und Barmherzigkeit zu üben; ruft eure Verbannten
zurück und liebt und pflegt sie. Selbst, wenn etz eure Feinde
wären, so solltet ihr sie ja, nach Christi Gebot, lieben. E3 nimmt
mich wunder, wie ihr und die Euren ruhig in euren Betten


% \picinclude{./310-319/p_s311.jpg} 
Ahnung kommender Revolutionen. Christus König usw. 311
schlafen könnt bei diesem grausamen Tun; denkt ihr nicht daran,
daß der Herr euch ein Gleiches tun könnte? Jhr könnt nicht
ganz oerstockt und gefühllos sein, es sei denn, ihr seid der Ver-
wersung anheim gefallen, und habet ein Brandmal in euren
Gewissen (1. Tim. 4, 2). Aber die christliche Liebe hofft, daß ihr
nicht alle in diesem Zustand seid, sondern daß etliche von euch
ihre Handlungen erwägen tmd mildern werden, entweder nach
dein Gesetz Gottes oder dem Evangelium.
Von einem, der euer zeitliches und ewiges Heil und nicht
euer Verderben wünscht.«
Middlesex, 28. des 2. Monats 1689. G. F.
Kurz vor der Jahresversammlung ging ich nach London.
Sie fand in diesem Jahre im dritten Monat statt und war
sehr feierlich und gewichtig; der Herr besuchte, wie er vor Zeiten
getan, sein Volk und würdigte die Versammelten seiner herrlichen
Gegenwart, den Freunden zu Trost und Freude. Nachdem alles
Geschäftliche geordnet war, kam es über mich, dem Brief, der
von der Versammlung an die Freunde gerichtet worden war,
einige Zeilen beizufügen:
,,Liebe Freunde und Brüder!
Die ihr des Herrn ewigen Arm und ewige Kraft erfahren
habt, der euch auf dem ewigen Grund und Fels bewahret und
eure Wohnung darauf erbaut hat; ihr seid in oielen Stürmen
und Wettern gewesen, welche aus den Wassern kommen, daher
das Tier kommt (Off. 13,1) und in vielen Unwettem, die durch
Abtritnnige aller Art entstanden, aber der Same, der der Schlange
den Kopf zertreten hat, und in dem das Volk Gottes gegründet
steht, ftehet fesi.,I Liebe Freunde und Brüder, wenn gleich die Welt
rings um euch erzitteri, so ist die Kraft Gottes über allem und
kann nicht wanken. Darum ihr Kinder Gottes, ihr Kinder des
Lichts und Erben des Reiches, bleibet in den Häusern des
Friedens, und haltet euch fern vom Hader über irdische Dinge;
,,Leget niemand bald die Hände auf« (1. Tim. 5,22), aus daß
ihr nicht aufgeblasen werdet mit dem, was vergänglich ist und so
zu Falle kommt; achtet aber aus die Kraft Gottes, die eure Augen
für das Gegenwärtige und das Zukünftige offen hält. ,,Darinnen
werdet ihr das Wort des Lebens erkennen und betasten« (1. Joh. 1).
G. F .....


% \picinclude{./310-319/p_s312.jpg} 
Bald darauf fand die Jahre?-versammlung in York statt; sie
war während mehrerer Jahre dort abgehalten worden wegen der
Größe der Grafschaft, und weil ez für Viele Freunde bequem war.
Weil durch etliche, die auß der Gemeinschaft der Freunde auß-
getreten waren, Schaden angerichtet worden war, so kam es über
mich, einige Zeilen an die Versammlung zu schreiben, um sie zu
ermahnen, in der reinen, himmlischen Liebe zu bleiben, die zur
wahren Einigkeit sührt und darin bewahrt:
»Liebe Freunde und Brüder in Christa?-,
Welche der Herr durch seinen ewigen Arm und seine Kraft biz
auf den heutigen Tag bewahrt hat, wandelt alle in der Kraft
und dem Geist Gotteö, der über allem ist, in Liebe und Einigkeit;
denn die Liebe einigt und erbauet alle Glieder Christi zu ihm, dem
Haupt. Die Liebe bewahret vor allem Zank und ist auö Gott;
barmherzige Liebe höret nicht auf, sondern erhebt den Sinn über
die äußern Dinge, und über allen Streit über äußere Dinge.
Sie überwindet daß Böse und treibet alle falsche Furcht auß. Sie
ist aus Gott und einigt die Herzen seines Volkes in der himm-
lischen Freude und Einigkeit. Der Gott der Liebe erhalte euch
und gründe euch fest in Christuö, euerm Leben und Heil, in
welchem ihr alle Frieden habt mit Gott. So wandelt nun in
ihm, damit ihr die friedsame Weißheit erlanget, Gott zur Ehre
und euch zum Trost. Amen.«
London, 27. dee 3. Monatö 1689. G. F.
Da ich von den vielen großen Versammlungen sehr ermüdet
und erschöpft war, sowie auch durch die viele Arbeit während der
Jahre-Joersammlung, und meine Gesundheit dadurch sehr ange-
griffen war, verließ ich die Stadt mit meiner Tochter Rouß und
ging nach ihrem Landsitz in die Nähe von Kingston, wo ich den
größten Teil deS folgenden Sommerö zubrachte; ich besuchte zu-
weilen von dort aus- die Freunde in Kingston und schrieb allerlei
zum Nutzen der Wahrheit ..... Viele Freunde besuchten mich,
sowie auch mehrere angesehene Personen, um allerlei Fragen
über Gott mit mir zu verhandeln .... Dann, im 7. Monat, verließ
ich Kingston und ging zu Wasser nach London, unterwegs- besuchte
ich Freunde und ging nach Hammersmith, das mir am Weg lag.
Da ich mich auf dem Lande einigermaßen gekrästigt hatte, ging ich
mm in London von Versammlung zu Versammlung, fleißig im Dienst


% \picinclude{./310-319/p_s313.jpg} 
Ahnung kommender Revolutionen. Christus König usw. 313
dez: Herm und in der Verkündigung göttlicher Geheimnisse, wie
der Geist der- Herrn sie mir ofsenbarte. Aber ich spürte, daß ich
es- nicht lange in der Stadt au?-halten konnte, und ich ging darum,
nachdem ich etwa einen Monat lang die Freunde dort besucht
hatte, nach Tottenham-High-Croß und von da nach Enfteld, wo
ich während etwa 3 Wochen überall die Freunde besuchte und
Versammlungen hielt. Dann, nachdem ich mich wieder ein wenig
erholt hatte, kehrte ich nach London zurück, wo ich etwa biö zum
9. Monat blieb; dann ging ich mit meinem Sohn Mead nach
seiner Wohnung in Essex, wo ich den ganzen Winter blieb ....
Dort schrieb ich Verschiedene?-, unter anderm einen Brief an die
Jahreö- und Vierteljahres-Versammlungen von Pennsylvanien, Neu-
England, Virginia, Maryland, Jersey, Earolina und andern
Niederlassungen in Amerika .... E-3 hieß darin unter anderm:
,,Bleibet in der Liebe Gotteß, die über die Liebe der Welt
erhebt, so daß eure Herzen nicht verderbt und ersüllet werden mit
den äußern Dingen oder mit den Sorgen der Welt, welche ver-
gänglich sind; trachtet nach dem, maß unrergäuglich ist, damit ihr
dessen teilhaftig werdet .,,’ Jhr sollt Alle, die ,,Frommen« wie die Ung-
gläubigen, übertrefsen in Rechtschaffenheit, Menschlichkeit und
Christlichkeit, Bescheidenheit, Mäßigkeit und in gerechtem, göttlichen
Wandel; zeiget ihnen die Früchte deö Geistes- und daß ihr Kinder
seid des lebendigen Gottetz, Kinder dez Lichts und nicht der Finster-
nis. Dienet Gott durch ein neues- Leben, denn ez ist daß Leben und
Wandeln in der Wahrheit, welchez Zeugnitz gibt von dem Gött-
lichen im Menschen, damit ,,sie eure guten Werke sehen und euren
Vater im Himmel preisen« (Matth. 5,16). Darum seid tapfer für
Gottes reine, heilige Wahrheit und verbreitet sie unter den
Gläubigen und den Ungläubigensund unter den Jndianern. Jhr
solltet einmal jährlich von allen euren Jahreöoersammlungen an
die hiesige Jahre?-Versammlung schreiben über euem Eifer sür die
Wahrheit und ihre Verbreitung, und wie die Leute sie aufnehmen,
die Gläubigen und die Ungläubigen und die Jndianer, und über
den Frieden der Kirche Christi unter euch. Denn, gelobt sei der
Herr, die Wahrheit gewinnt dort Boden, und viele werden den
Freunden recht geneigt, und die Kraft und der Same Gotteß sind
über allen; darin erhalte Gott der Allmächtige sein ganzes Volk,
zu seiner Ehre. Amen«.
Gooseß, 28. des 11. Tlltonatz 1689. G. F.


% \picinclude{./310-319/p_s314.jpg} 
Während ich in London gewesen war, hatte ich ein Gesicht
gehabt, von einer doppelten Gefahr die etlichen Bekennern der
Wahrheit drohte. Die eine war, daß viele der jungen Leute dem
Treiben der Welt verfielen, die andere, daß auch Alte sich den
weltlichen Dingen zuwandten. Weil mir daz nun wieder schwer
auf der Seele lag, trieb ez mich, etwasz dagegen zu schreiben .....
»An alle, die sich zu Gotteö Wahrheit bekennen,
E-Z ist mein Wunsch, daß ihr alle demütig in ihr wandelt,
denn alö der Herr mich zuerst berief, da zeigte er mir wie »junge
Leute zusammen in Eitelkeit gerieten und die alten in irdische Ge-
sinn1mg«, und diesen beiden mußte ich ,,ein Fremdling werden«.
Und nun, Freunde, sehe ich gar zu viele junge Leute, die sich
zur Wahrheit bekennen, in Weltlichkeit geraten, und zu viele
Eltern, die ez dulden. Und auch unter den Alten sehe ich viele,
die sich dem Jrdischen zuwenden. Hütet euch, daß ihr euch nicht
euer Grab grabet, dieweil ihr äußerlich noch am Leben seid, und
nicht ,,oiel Schlamm auf euch ladet« (Hab. 2,6). Der Geist der
Welt ist ein Geist der Unruhe; der Geist Christi aber ist Frieden,
darin erhalte Gott alle Gläubigen!« G. F.
Ferner schrieb ich etwaö »über daß Zeichen, von dem Jesaia
sagt, daß ,,Gott ez den Heiden geben werde« (Jes. 11), und zeigte,
daß ez Christuö sei. « .
Ende des- 7. Monate- 1690 ging ich nach London und blieb
dort bis zum Anfang dez 9. Monats. Das Parlament tagte, und
da eö gerade daran war, ein Gesetz zu machen über daö Schwören
und ein andereö über daß Heiraten, so wohnten mehrere Freunde
den Verhandlungen bei, um dahin zu wirken, daß diese Gesetze
so abgefaßt werden, daß sie den Freunden nicht schaden können.
Diese Bestreben suchte auch ich mit zu fördern, indem ich den
Verhandlungen im Parlament beiwohnte und die Sache mit
mehreren Mitgliedern besprach ....
Nachdem ich so mehr alß einen Monat in London gewesen
war, ging ich nach Tottenham und darauf nach Ford Green
und besuchte während mehrerer Wochen die Versammlungen
der Freunde in diesen Gegenden. GS trieb mich auch während
dieser Zeit allerlei zu schreiben, .... so einen Brief an die
Freunde, die alß Prediger nach Amerika gegangen waren.
,,Liebe Freunde und Brüder, Lehrer, Prediger, Ermahner
und Warner, die ihr nach Amerika und den dortigen Inseln ge-


% \picinclude{./310-319/p_s315.jpg} 
Ahnung kommender Revolutionen. Christus König usw. 315
gangen seid. Fachet die euch von Gott verliehenen Gaben in
euch an und die reine Gesinnung, und bildet eure Fähigkeiten
auö, damit ihr daß Licht der Welt werdet, ,,eine Stadt auf dem
Berge, die nicht verborgen sein kann« (Matth. 5). Lasset euer
Licht leuchten vor den Indianern, den Schwarzen und den Weißen,
aus daß ihr der Wahrheit, die in ihnen ist, entgegenkommen
möget und sie unter das Panier bringet, das Gott in Christus
ausgerichtet hat. Denn vom Ausgang der Sonne bis zum Nieder-
gang soll Gotteä Name groß sein unter den Heiden und in
jchem Tempel, d. h. vielmehr jedem geheiligten Herzen ,,soll Gott
ein Opfer dargebracht werden« (Maleachi 1). Habet Salz bei euch,
damit ihr daß Salz der Erde sein könnet, daß sie durch euch ge-
salzen werde und bewahret vor Verderben und Fäulniß, so daß
alle Opfer, die dem Herrn dargebracht werden, gewürzt und dem
Herm angenehm sind. Wachset im Glauben und in der
Gnade Christi, daß ihr nicht wie Zwerge seid, denn ein Zwerg
soll nicht herzu treten, daß er seinem Gott opfere (3. Mos. 21, 20),
wenn gleich er von Gottes Brot essen dars, daß er sich daran
nähre. Meine Freunde, seid nicht lässig, haltet eure Negewer-
sammlungen und eure Familienversammlungen, und haltet Ver-
sammlungen mit den Jndianerkönigen und ihren Räten und
Untertanen und mit andern allenthalben. Bringet sie alle zum
Geist der Taufe und der Beschneidung, durch den sie Gott
erkennen und anbeten können. Vor allem hiitei euch alle, daß
ihr euren Sinn nicht aus irdische Dinge richtet, und nicht darum
zanket und geizet; denn ,,fleischlich gesinnt sein, bringt den Tod«
(Röm. 8,6) und »Geiz ist Abgötterei« (Col. 3, 5). GS ist zuviel deß
Zankenß und Hadernö um dieseß Abgotteö willen, so daß all-
zuviele von der Gotteöfurcht abfallen, und etliche ihre Tugend,
Menschenliebe und wahre christliche Liebe ganz verloren haben .....
Alle Glieder Christi brauchen einander. Der Fuß braucht
die Hand und die Hand den Fuß; das Ohr braucht das Auge
und das Auge daß Ohr. Alle Glieder dienen dem Leibe, davon
Christus- das Haupt ist, und daö Haupt kennt ihre Dienste.
Darum soll keiner auch daß geringste Glied verachten (1. C-or. 12).
Sehet zu, daß ihr die irdische, gewinnsüchtige Gesinnung, die
nach den Gittern und Schätzen dieser Welt trachtet, unterdrürket,
damit ihr nicht herunter sinket zu den Heiden und deß Reicheö
Gotteß verlustig geht, das kein Ende hat. Trachtet am ersten


% \picinclude{./310-319/p_s316.jpg} 
nach diesem Reiche, so wird euch das andre alles zufallen
(Matth. 6, 33). Gott erhält alles im Himmel und auf der Erde.
Jhm sei Lob und Dank für alle seine unaußsprechlichen Gaben
die zeitlichen wie die geistlichen.«
Tottenham, 11. deß 10. Monats: 1690. G. F.
Bald darauf ging ich nach London zurück und wohnte fast
täglich Versammlungen der Freunde bei. Nachdem ich etwa zwei
Wochen dort gewesen war, legten sich die Leiden, die die Freunde
in Jrland zu erdulden hatten, schwer auf meine Seele,1) und es
trieb mich, ihnen einen Trost zu schreiben:
»Liebe Freunde und Brüder in Christo, denen der Herr
durch den starken Arm seiner Kraft durch die vielen Leiden hin-
durch geholfen hat. .... Jch vertraue auf den Herrn, daß er
euch auch ferner hindurch helfen werde, und den, der glaubt, in
seiner Weißheit erhält, damit ihm gerechterweise kein Leid zugefügt
werden kann. Und wenn ihr Unrecht leiden müßt, so möge der
Gott der Gerechtigkeit euch aufrecht erhalten und beistehen und
euch nach eurem Verdienst lohnen ..... Wahrlich, meine Freunde,
wenn ich bedenke, von was für Geistern ihr umgeben seid, so
sehe ich es alß eine große Barmherzigkeit Goiteß an, daß ihr
nicht alle untergegangen seid. Aber der Herr trägt seine—Lämmer
in seinen Armen (Joh. 10) und sie sind ihm teurer alß der—Apsel
de-Z Augeß (Psalm 178) .... . Darum tun alle seine Kinder gut,
sich ihm mit Seele, Herz und Geist ganz hinzugeben, denn er
ist ein treuer Hüter; er schläft noch schlummert nicht, (Psalm 121).
.... Jhm gehört die Macht im Himmel und auf Erden; und
euch, ,,die ihr ihn ausnehmt, gibt er Macht, Gottes Kinder zu
werden« (Joh. 1, 12).
Darum lebet und bleibet in Jesus Christus-, damit nichtß
zwischen euch und Gott sei als- Christuz, in dem ihr Leben, Er-
lösung, Ruhe und Frieden habt mit Gott.
Uber die Sache der Wahrheit, hier und anderwärttz, kann
ich euch mitteilen, daß in Holland und Deutschland überall die
Freunde in Einigkeit und Liebe und im Frieden sind, sowie auch
in Jamaika, Barbadoeö, Neoiö, Antigua, Maryland und Neu-
England. Der Herr bewahre sie alle vor der Welt, in der
Bekümmerniß ist, in Christo, in welchem Frieden und Liebe ist.
1) Die Leiden der Quäker in Jrlaiid in diesen Jahren waren sehr
dtiickend; allein an Habe haben sie in 3 Jahren bei 1110 000 :8 verloren.


% \picinclude{./310-319/p_s317.jpg} 
Ahnung kommender Revolutionen. Christus König usw. 317
Amen. Meine Liebe im Herrn Jesuß Ehriftuz einem jeden Freund
im ganzen Land, wie wenn ich ihn mit Namen genannt hätte.«
London, 10. dez 11. Monats 1690. G. F.
(Nachschrift von William Penn).
Dies, lieber Leser, war nun die Beschreibung dez Lebens-,
der Reisen, der Arbeit und der tnannigfachen Leiden und Prüfungen
dieses heiligen Gotteßmanneö, von seiner Kindheit an biz fast zu
seinem Tode; er hat selber Aufzeichnungen darüber gemacht, auß
denen die vorangehenden Blätter genommen sind. EZ bleibt nun
nur noch übrig, über die Zeit, den Ort und die Art seineß Todeö
und Begräbnissez zu berichten.
Am Tage nachdem er den vorhergehenden Brief nach Jrland
geschrieben hatte, ging er zur Versammlung in Graeechurch=Street,
die sehr zahlreich war, da ez ein Erster Tag war. Der Herr
schenkte ihm die Kraft, die Wahrheit mächtig und eindringlich zu
predigen und viele wichtige und tiefe Dinge mit großer Kraft und
Klarheit darzutun. Nachdem er gebetet hatte, und die Versammlung
zu Gnde war, ging er zu Henry Goldney, in White-Hart-Court,
nahe beim Vetsammlungtzhauß;. Zu einigen Freunden, die mit
ihm gingen, sagte er: ,,Jch fühlte, wie die Kälte mir bi-8 ins
Herz drang, als ich aus der Versammlung kam. Aber ich bin
froh, daß ich gekommen bin, denn nun ist eö getan! völlig getan!«
Sobald die Freunde fort waren, legte er sich auf ein Bett, wie
er oft getan hatte, wenn er nach den Versammlungen müde war.
Bald erhob er sich wieder, um sich aber sogleich wieder nieder
zu legen, immerwährend über Kälte klagend. Seine Kräfte nahmen
zusehendö ab, und er war bald genötigt, ganz zu Bett zu gehen,
wo er ganz ruhig und friedlich und bei vollem Bewußtsein biz
zum Ende lag. Und wie während seines ganzen Lebentz sein
Geist in allumfassender Liebe Gotteö auf die Ausrichtung der
Wahrheit und Gerechtigkeit gerichtet gewesen war, und darauf,
daß der Weg dazu allen Völkern und denen, die noch ferne davon
sind, bekannt werde, so war auch jetzt in seiner Schwachheit fein
Sinn ganz davon in Anspruch genommen, und er ließ einige
nähere Freunde kommen, denen er seinen Wunsch und Willen über
die Verbreitung der Bücher der Freunde und der Wahrheit durch


% \picinclude{./310-319/p_s318.jpg} 
dieselben, aussprach. Mehrere Freunde besuchten ihn in seiner
Krankheit. Zu einigen von ihnen sagte er, ,,Es ist alles gut! Der
Same Gottes herrscht über alles, selbst über den Tod. Und obgleich
mein Körper schwach ist, so ist die Kraft Gottes doch über allem,
und der Same herrscht über alle widerspenstigen Geister.«
So lag er da, in einer himmlischen Gemütsoersassung, den
Geist gänzlich aus den Herrn gerichtet, während seine Kräfte mehr
und mehr abnahmen; und am dritten Wochentage, zwischen neun
und zehn Uhr Abends, schied er friedlich aus diesem Leben und
entschlies sanft im Herrn, dessen herrliche Wahrheit er lebendig
und kräftig noch zwei Tage zuvor gepredigt hatte. So endete
er seine Tage in treuem Zeugnis, in völliger Liebe und Einigkeit
mit seinen Brüdern, und in Frieden und Wohlwollen gegen alle
Menschen, am 13. des 11. Monats 1690, in seinem 67. Lebens-
jahr. ....
Am Tage da George Fox begraben wurde, kam Mittags
eine große Menge von Freunden im Versammlungshause in
White-Hart-Court, nahe bei Graeechurch-Street, zusammen, um
seinen Leib zu Grabe zu geleiten. Die Versammlung dauerte
etwa zwei Stunden. Sie war sehr feierlich und sichtlich gesegnet
mit der Gegenwart des Herrn und seiner herrlichen Kraft; und
mancher legte Zeugnis ab von dem Eindruck und dem Andenken,
welches das Wirken des treuen bewährten Diener Gottes hinter-
ließ; von seinem frühen Eintreten in den Dienst beim Anbruch des
Tages des Evangeliums, seinem reinen Leben, seinen mühsamen Reisen
und s einem unermüdlichen Arbeiten in der Liebe für das unoergäng-
liche Evangelium und für die Bekehrung vieler Tausender von der
Finsternis zum LichteJesu Christi, dem Grund des wahren Glaubens;
von den mancherlei Leiden und Heimsuchungen und allem Widerstand,
den er um seines treuen Bezeugens willen zu erdulden hatte, sowohl
von seiten seiner öffentlichen Gegner als auch von falschen Brüdern;
von allen Befreiungen, aller Hilfe und allen Siegen in diesem
allem durch die Kraft Gottes, welcher immerdar die Ehre und
der Ruhm gebührten, sowie auch fürderhin ihr allein gebühren sollen.
Nachdem die Versammlung zu Ende war, wurde sein Leib
von Freunden nach dem Begräbnisplatz der Freunde, in der Nähe
von Bunhill-Field getragen; dort wurde sein Leib der Erde über-
geben, nach andächtigem Warten aus den Herm, und nachdem
mehrere lebendig Zeugnis abgelegt hatten, die Anwesenden der


% \picinclude{./310-319/p_s319.jpg} 
Ein Brief bon George Fox nach seinem Tode vorgesunden usw. 319
Führung und dem Schutz des heiligen Geistes und der Kraft be-
sehlend, durch welche dieser heilige Mann Gotteß erwählt, auß-
gerüstet, gestärkt und beschützt worden war bis anö Ende seineß
Leben?-; sein Andenken aber soll bleiben und immerdar ein Segen
sein bei den Gerechten.
Anhang.
Ein Brief von George Fox nach seinem Tode votgesunden mit der
Aufschrift: Nicht vor der Zeit zu össnen.
Ein Brief von George Fox,
,,An die Jahreöversammlung und die Versammlung des
Zweiten Tages in London und an alle Kinder Gotteö an allen
Orten der Welt. An alle Kinder Gottes- überall, die von seinem
Geiste geleitet werden und in seinem Lichte wandeln, in welchem
sie daß Leben haben und Einigkeit mit dem Vater und dem Sohn
tmd untereinander. Haltet alle eure Versammlungen im Namen
des Herm Jesu, die ihr in seinem Namen versammelt seid, in
seinem Licht, seiner Gnade, Wahrheit und Kraft, und in seinem
Geist, durch welchen ihr seine gesegnete und erfrischende Gegen-
wart unter euch spüren werdet, unter euch und in euch, zu euerm
Trost und zu Gotteö Ehre.
Liebe Freunde, alle eure Versammlungen, die Männer- wie
die Frauen-Versammlungen, die monatlichen, vierteljährlichesn und
jährlichen, sind durch die Kraft, den Geist und die Weißheit
Gotteö eingesetzt worden. Durch sie habt ihr seine Kraft, seinen
Geist und seine Wet?-heit, seine gesegnete erhebende Gegenwart
gespürt, unter euch und in euch zu seiner Ehre und euerm Trost,
so daß ihr ,,eine Stadt, die aus dem Berge liegt und nicht ver-
borgen sein kann,« (Matth. 5) geworden seid.
Und wenn schon zu Zeiten manche losen und unreinen Geister
sich gegen euch erhoben haben, um sich in Schriften und auf
andere Weise euch zu widersetzen, so habt ihr ja gesehen wie sie
zu nichte geworden sind; der Herr hat sie versengt, hat ihre
Taten antz Licht gebracht, und machte, daß man sah, wie sie
Bäume ohne Früchte, Brunnen ohne Wasser waren, irrige Sterne,
abgesallen am Firmament Gottetz, wütende Wellen dez Meeres,
die ihren Schaum und Schmutz auswerfen (Jud.); und viele von
ihnen sind wie der Hund, der frisset, waö er gespeiet hat, und wie

% \picinclude{./320-329/p_s320.jpg} 
die Sau, die, nachdem sie gewaschen ist, sich wieder im Kot
wälzt (Spr. 26, 11). Dietz ist der Zustand vieler gewesen, wie
Gott und sein Volk weiß. .
Darum bleibet alle fest in Jesus Christus- euerm Haupt, in
welchem ihr alle einß seid, Männer und Weiber, und seine Herr-
schaft kennet. Seine Herrschaft und sein Friede werden nicht
aufhören zu wachsen; aber die Herrschaft deö Teuselö und mit
ihr alle, die nicht in Christo sind, sich ihm und seiner Herrschaft
widersetzten, werden ein Ende haben, ihr Gericht bleibt nicht aus-
und ihre Verdammniß schlummert nicht. Darum lebet und
wandelt alle in Liebe, Unschuld und Reinheit, in Licht, Leben,
Geist und Kraft auß Gott und Ehristue, die über allem sind.
Bleibet in der Rechtschassenheit und Heiligkeit, in der Kraft und
dem heiligen Geist Gotteß, in welchem daß Reich Gottez steht.
Alle, die ihr Kinder des neuen, himmlischen Jerusalem auö der
Höhe seid, richtet eure Blicke dorthin.
Der Geist der Auflehnung und des Widerstandes, der früher
und auch kürzlich wieder sich erhoben hat, stammt nicht auö dem
Reich Gotteß und ist ferne vom Reich Gottes:-’ und vom himmlischen
Jerusalem und fällt dem Gericht und der Verdammniß anheim
mit allen seinen Büchern, Worten und Werken. Darum sollen
die Freunde in der Kraft und dem Geist Gotteß leben und wandeln,
die über jenem Geist sind und im Samen, der ihn vernichten und
in Stücke schlagen wird (1. Mos. 3). Jn diesem Samen habt ihr
Frieden und Freude in Gott, und die Macht, jenen Geist der
Auflehnung zu richten, und eure Einigkeit ist in der Kraft und
dem Geist Gotteß.
Lasset keinen ihm selber leben, sondern alle sollen Gott
leben, wie sie auch ihm sterben sollen (Röm. 14); und suchet den
Frieden der Kirche Christi und den Frieden aller Menschen in
ihm, denn ,,selig sind die Friedfertigen.« Bleibet in der reinen
friedlichen, himmlischen Weiöheit Gotteß, welche friedsam, gelinde
’ und voll Barmherzigkeit ist. Trachtet alle, einerlei Sinnes und
Herzens- zu sein, eine Seele und eine Meinung in Christus,
und habet seinen Sinn und Geist in euch wohnen, ermuntert
euch untereinander in der Liebe Gotteö, welcher den Leib Christi,
seine Kirche, erbauet, deren heiligeß Haupt er ist. Ghre sei Gott
durch Ehristuß, jetzt und immerdar; er ist der Felß und Grund,
der Emmanuel Gott mit und daß Amen in allem, der Anfang


% \picinclude{./320-329/p_s321.jpg} 
rmd daß Ende. Jn ihm lebet und wandelt, in welchem ihr
ewigeß Leben habet; in ihm werdet ihr mich spüren und ich euch.
Alle Kinder dez neuen Jerusalem auß der Höhe, der heiligen
Stadt, deren Licht der Herr und daß Lamm sind, und welche der
Tempel ist (Offb. 2 1), in ihr sind sie wiedergeboren auö dem
Geist; also ist daß; Jerusalem aus- der Höhe die Mutter derer,
die auS dem Geist geboren sind. Die, welche inß himmlische
Jerusalem gekommen sind und noch kommen, nehmen Christuin
auf, und er gibt ihnen Macht, Gottes Kinder zu sein, und sie sind
wiedergeboren auö dem Geist, und so ist daß Jerusalem auß der
Höhe ihre Mutter (Gal. 4). Solche kommen zum himmlischen
Berge Zion, zu der Menge vieler tausend Engel, zu den Geistern
der vollkommenen Gerechten, zu der Stadt deö lebendigen Gottetz,
zu der Gemeinde der Erstgeborenen, die im Himmel angeschrieben
sind und den Namen Gottes tragen (Ebr. 12). Hier ist eine
neue Mutter, von der ein neues himmlische?-, geistigeö Geschlecht
abstammen wird. ES gibt keine Spaltung, keinen Streit, keine
Entzweiung im himmlischen Jerusalem, noch im Leib Christi,
· welcher autz lebendigen Steinen erbaut ist (l.Petr. 2), ein geistiges
Hauß. Bei Christuö ist keine Spaltung; denn in ihm ist Friede.
Christutz sagt: ,,Jn mir habt ihr Friede« (Joh. 16). Und er ist
aus- der Höhe und nicht von dieser Erde. Jn dieser Welt und
in ihrem Geist ist Angst; darum bleibet in Christuö und wandelt
in ihm.
Jerusalem war die Piutter aller wahren Christen, vor dem
Abfall. Seitdem die äußerlichen Christen sich in viele Sekten
gespalten haben, haben sie oiele Mütter; alle aber, die durch
Christi Kraft und Geist vom Abfall zurück gekommen sind, haben
Jerusalem auß der Höhe zur Mutter und keine sonst; sie ernährt
alle ihre geistigen Kinder.« G. F.
(Dieser Brief wurde an der Jahreöoersanunlung, in London,
im Jahre 1691 gelesen.)
George Fox. 21

% \picinclude{./320-329/p_s321z01.jpg} 

% \picinclude{./320-329/p_s321z03.jpg} 

% \picinclude{./320-329/p_s321z04.jpg} 
