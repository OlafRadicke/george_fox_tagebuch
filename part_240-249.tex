% \picinclude{./140-149/p_s140.jpg} 
% Hier ist ein Teil des Textet nach 230-239 verschoben worden
% wegen Formatierung.
% \picinclude{./240-249/p_s241.jpg} 
Um diese Zeit hatte ich einen Krankheitsanfall der mich sehr
angriff und schwächte, und der einige Zeit anhielt, so das einige
Freunde an meiner Wiederherstellung zweifelten. Es kam mir\index{Vision}
vor, ich wandle unter Gräbern und Leichen, aber die unsichtbare
Macht hielt mich innerlich aufrecht und gab mir erquickende Kraft,
selbst wenn ich so schwach war, das ich kaum sprechen konnte.
Einmal, als ich des Nachts auf meinem Bett lag in der Betrachtung 
der Herrlichkeit Gottes, die über alles ist, hörte ich
eine Stimme, das der Herr noch viel Arbeit an seinem Werke für
mich habe, ehe er mich zu sich nehmen könne.

Es wurden Maßregeln getroffen, um meine Freilassung zu
bewirken, wenigstens für so lange, bis ich mich erholt hätte, aber
es zeigte sich, das es schwierig war, sie zu erlangen, denn der
König wolle mich auf keinem andern Weg als dem der Begnadigung\index{Begnadigung}
freilassen, da man ihm gesagt habe, nach dem Gesetz
könne er es nicht tun, und durch Begnadigung wollte ich nicht
frei werden, denn das schien sich mir nicht mit meiner Unschuld
zu vereinbaren [...].

Nun ging meine Frau nach London und erzählte dem König
von meiner langen unverschuldeten Gefangenschaft und der Art
meiner Gefangennahme, wie die Richter mit mir verfahren waren,
indem sie mir den Eid als Falle vorlegten, worauf sie mich den
Strafen des Praemunire unterwarfen, und es stehe nun bei ihm,
ihren Wunsch zu erfüllen und mich frei zu sprechen. Der König
antwortete ihr freundlich und wies sie an den Kanzler; zu diesem
ging sie, konnte jedoch nicht erreichen, was sie wünschte, denn
dieser sagte, der König könne mich nicht anders als durch Begnadigung 
frei lassen, und ich hatte keine Freiheit, mich begnadigen
zu lassen, da ich wusste, das ich nichts Böses getan hatte. Wenn
ich mich hätte wollen begnadigen lassen, so hätte ich nicht solange
zu warten brauchen, denn der König war schon längst willens 
gewesen, mich zu begnadigen, und hatte zu 
Thomas Moore\person{Moore, Thomas} gesagt,
ich brauche mir keine Bedenken zu machen, mich begnadigen zu
lassen, denn es sei schon mancher, der so unschuldig gewesen wie
ein Kind, begnadigt worden, aber ich konnte mich nicht dazu 
verstehen. Lieber wollte ich mein ganzes Leben lang im Gefängnis
bleiben, als aus demselben befreit zu werden auf eine Art, die
irgendwie der Wahrheit zur Unehre gereichen konnte, und ich zog
es darum vor, das die Rechtmäßigkeit meiner Anklage vor Gericht
% \picinclude{./240-249/p_s242.jpg} 
geprüft werde\index{Revision} [...]. So wurde denn ein Befehl
nach Worcester geschickt, mich nochmals in Kings Bench zu
verhören, um meine Anklage zu prüfen. Ich reiste also am 4.
des 12. Monats ab, am 8. kamen wir in London an, und
am 11. wurde ich vor die vier Richter von Kings Bench\ort{Kings Bench} 
gebracht, wo Corbet meine Sache führte. Er brachte einen neuen
Verteidigungsgrund vor, nämlich man dürfe niemand wegen eines
Prämunire\index{Prämunire}\footnote{Prämunire der tarm. techn. einer 
bestimmten Art von Verbrechen, z. B. das Verweigern des Huldigungseides, 
das bestimmte Strafen wie Entziehung des Grundbesitzes, oft auch 
Gefängnis u. a. zur Folge hatte. Vgl. Stephen, Englisches Strafrecht.} 
einsperren. Daraus erwiderte Richter Hale\person{Richter Hale}: \zitat{Mr.
Corbet, diese Verteidigung hätten sie früher bringen sollen}, dieser
erwiderte: \zitat{Wir konnten keine Abschrift der Anklage bekommen.}
Der Richter sagte: \zitat{Das hätten sie uns sagen sollen, dann
hätten wir sie gezwungen, eine solche früher zu schicken.} [...]
Corbet blieb dabei, das man nicht wegen eines Praemunire
jemand gefangen nehmen könne [...] \zitat{Gut}, sagte einer der
Richter, \zitat{wir müssen Zeit haben in unsern Büchern nachzusehen,
und die Statuten nachzuprüfen}. Somit wurde das Verhör auf
den folgenden Tag Verschoben.

Am folgenden Tage ließen sie diesen Verteidigungsgrund
lieber fallen und fingen gleich mit den Irrtümern in der Anklage
an, und als dieselben eröffnet wurden, so waren ihrer so viele
und so große, das die Richter alle die Überzeugung hatten,
die Anklage sei leer und nichtig und man solle mich frei
lassen [...] Es waren an dem Tage mehrere angesehene Leute,
Lords und andere da, die den Huldigungseid ablegen mussten, und
einige meiner Gegner suchten die Richter zu bewegen, mir den
selben auch noch einmal vorzulegen, weil, wie sie sagten, es 
gefährlich sei, mich frei zu lassen. Aber Richter Hale sagte, er habe
allerdings schon dergleichen Gerüchte über mich gehört, aber noch
Viel mehr gute, und so erklärten er und die übrigen Richter mich
öffentlich frei [...].

Während meiner Gefangenschaft in Worcester hatte ich trotz
meiner häufigen Krankheit, und trotzdem ich so oft nach London
und wieder zurückgezerrt wurde, mehrere Bücher für den Druck
geschrieben. Eines derselben war betitelt: \zitat{Buchtitel!Eine Warnung an
England}\index{Eine Warnung an England}. Ein anderes war: \zitat{An die 
Juden, um zu beweisen, das der Messias gekommen ist}\index{Juden}\index{Buchtitel!An die 
Juden, um zu beweisen, das der Messias gekommen ist}.
% \picinclude{./240-249/p_s243.jpg} 
Ein anderes: \zitat{Von der Inspiration, Offenbarung und Weissagung}\index{Buchtitel!Von 
der Inspiration, Offenbarung und Weissagung}. Ein anderes: 
\zitat{Gegen alles unnütze Disputieren.}\index{Buchtitel!Gegen 
alles unnütze Disputieren} Ein anderes: \zitat{An alle Bischöfe und
Prediger, das sie sich nach der Schrift prüfen}\index{Buchtitel!An alle Bischöfe und
Prediger, das sie sich nach der Schrift prüfen}. Ein anderes:
\zitat{An die, welche sagen, wir lieben nur uns selbst}\index{Buchtitel!An 
die, welche sagen, wir lieben nur uns selbst}. Ein anderes
betitelt: \zitat{Unser Zeugnis von Christus}\index{Buchtitel!Unser 
Zeugnis von Christus}. Und ein anderes kleines
Buch: \zitat{Vom Schwören}\index{Buchtitel!Vom Schwören}, welches 
das erste war von den beiden,
die dem Parlament vorgelegt wurden. Außerdem schrieb ich noch
verschiedene Schriften und Briefe an Freunde, um sie zu 
ermutigen und zu stärken im Dienst für den Herrn. Denn etliche,
die sich als Bekenner der Wahrheit ausgegeben hatten, dann
aber einem Geist der Verführung Raum gegeben hatten und von
der Einigkeit und Bruderschaft des Evangeliums, in der die
Freunde stehen, abgewichen waren, hatten versucht, die Freunde
in ihrem Dienst zu entmutigen, besonders in der fleißigen und
wachsamen Fürsorge für die Ordnung der Angelegenheiten der
Kirche Christi [...]

\chapter[Schriften ordnen und für Frauenversammlungen eintreten]{Schriften 
ordnen und für Frauenversammlungen eintreten}

\begin{center}
\textbf{Fox sammelt und ordnet die Bücher und Schriften, die er
geschrieben und tritt für die Frauenversammlungen ein.}
\end{center}


Als ich nun wieder frei war, besuchte ich die Freunde in
London,\ort{London} und da ich mich gar nicht wohl fühlte, ging ich bald
nach Kingston.\ort{Kingston} [...] Ich blieb jedoch nicht lange dort, sondern
kehrte wieder nach London zurück, [...] ging dann nach Coffel,
Preston, [...] Lancaster [...] und am 25. des 4. Monats nach
Swarthmore.\ort{Swarthmore}

Als ich eine Weile dort war, kamen viele Freunde aus verschiedenen 
Gegenden des Landes, um mich zu besuchen. Auch
von Schottland\ort{Schottland} kamen manche; von diesen hörte ich, das vier
junge Studenten in Aberdeen\ort{Aberdeen} in diesem Jahre bekehrt worden
waren, bei einer Disputation\index{Disputation} von Robert Barclay\footnote{Robert 
Barclay, der \zitat{Theo1oge der Quüker}s. Weingarten 
a.a.D.},\person{Barclay, Robert}  und George
Keith\person{Keith, George} mit einigen Schülern der Universität 
[...] Während ich
in dieser Stadt war, lies ich mehrere Bücher drucken. Eines:
\zitat{Über das Schwören}, ein anderes: \zitat{Niemand ist ein Nachfolger
der Apostel und Propheten, als wer ihnen nachfolgt in der gleichen
Kraft und dem gleichen Geist, darin sie waren.} \index{Buchtitel!Niemand 
ist ein Nachfolger der Apostel und Propheten, als wer ihnen nachfolgt in der gleichen
Kraft und dem gleichen Geist, darin sie waren.} Ein anderes:
% \picinclude{./240-249/p_s244.jpg} 
\zitat{Besitzen geht über Bekennen}\index{Buchtitel!Besitzen geht über 
Bekennen}, und das die Bekennenden jetzt
Christus im Geist verfolgen, wie die jüdischen Bekennenden ihn
äußerlich verfolgten in den Tagen, da er im Fleisch wandelte [...] .
Ferner die acht folgenden Bücher: 

\begin{itemize}
 \item \zitat{An die Behörden von Danzig}\index{Buchtitel!An die Behörden von Danzig}
 \item \zitat{Kain gegen Abel}\index{Buchtitel!Kain gegen Abel}, oder eine
 \item \zitat{Antwort auf das Gesetz der Männer von Neu-England}\index{Buchtitel!Antwort 
  auf das Gesetz der Männer von Neu-England}
 \item \zitat{An die Freunde zu Nevis über das Wachsamsein}\index{Buchtitel!An die 
  Freunde zu Nevis über das Wachsamsein}
 \item \zitat{Ein Generalbrief an alle Freunde in Amerika}\index{Buchtitel!Ein 
  Generalbrief an alle Freunde in Amerika}
 \item \zitat{Über das, was des Kaisers und was Gottes ist}\index{Buchtitel!Über 
  das, was des Kaisers und was Gottes ist}
 \item \zitat{Über die Ordnung in den Familien}\index{Buchtitel!Über die Ordnung 
  in den Familien}
 \item \zitat{Der geistliche Mensch richtet alle Dinge}\index{Buchtitel!Der 
  geistliche Mensch richtet alle Dinge}
 \item \zitat{Über die höhere Kraft}\index{ Buchtitel!Über die höhere Kraft}
\end{itemize}

\bigskip 

überdies schrieb ich mehrere Briefe, sowohl an die Freunde
in England, als auch jenseits des Meeres; auch Antworten
aus mehrere Flugblätter über die Abtrünnigkeit mehrerer, die
sich der Ordnung des Evangeliums widersetzt hatten, und viel
Unruhe und Zank in Westmorland\ort{Westmorland} angestiftet hatten. Es trieb
mich darum, an die dortigen Freunde ein paar besondere Zeilen
zu schreiben.

\brief{Quaker-Gemeinde}{
  An die Freunde in Westmorland,

  \bigskip 

  Leber alle in der Kraft Gottes, in seinem Licht und Geist,
  die euch zuerst bekehrten, das ihr durch sie in der ersten Einigkeit
  bleibet, in Demut und in der Furcht Gottes, und seiner friedsamen,
  sanften Wahrheit, welche ihr leicht erbitten könnt, damit ihr in
  dieser Kraft, diesem Licht und Geist alle dienstbereit seid in euren
  Männer- und Frauen-Versammlungen, im Besitz der Ordnung
  des Evangeliums, welches Evangelium Leben und unvergängliches
  Wesen ans Licht gebracht hat [...] Darum, ihr Freunde in
  Westmorland, bleibet in der Kraft Gottes; sie muss euch behüten
  und schützen, wenn ihr wollt geschützt sein. Lasset euren Glauben
  in der Kraft Gottes stehen, nicht in der Weisheit menschlicher
  Worte, auf das ihr nicht fallet. In Gottes Kraft habt ihr
  Friede, Leben und Einigkeit; und weil ihr nicht in der Kraft
  Gottes geblieben seid, in seiner Gerechtigkeit und seinem Heiligen
  Geist, ist dieser Zank unter euch gekommen. 
  \bigskip 
  \begin{flushright}
  G. F.\end{flushright}
}

Ich schrieb auch einen Generalbrief\index{Generalbrief} an die Freunde an der
Jahresversammlung in London, [...] darin hieß es unter
anderem:

\brief{Quaker-Gemeinde}{
  Was die wahren Männer- und Frauen-Versammlungen\index{Autorität der Versammlung}
  anbelangt, die nach Gottes Rat eingesetzt waren, so widersetzt
  sich jeder, der sich ihnen widersetzt, zugleich der Kraft Gottes, auf
  % \picinclude{./240-249/p_s245.jpg} 
  deren Befehl sie eingesetzt\index{Von Gott eingesetzt} sind. Die, welche sich dieser Kraft
  widersetzten, sind nicht Diener des Evangeliums oder Christi [...]
  Unsere Männer- und Frauen-Versammlungen, und alle anderen
  Versammlungen, die im Namen Jesu geschehen, sind nach dem
  Evangelium Christi, nach der Kraft Gottes, eingesetzt, also nicht
  von Menschen oder durch Menschen. Darin sollen alle sich 
  versammeln und Gott anbeten; darin sollen auch alle handeln, und
  darinnen haben auch alle Gemeinschaft untereinander, ein frohes,
  friedsames Beisammensein.

  Alle gläubigen Männer und Frauen in jedem Land und
  jeder Stadt, deren Glauben in der Kraft Gottes steht, im Evangelium 
  Christi, und die dieses Evangelium, diese Kraft Gottes, besitzen, 
  haben alle ein Recht an die Kraft in diesen Versammlungen;
  denn sie sind alle Erben der Kraft, nach der die Männer- und
  Frauen-Versammlungen eingesetzt sind. [...]
}


Um diese Zeit sammelte ich so viel wie möglich von den
Briefen, die ich in früheren Jahren an die Freunde geschrieben
hatte. Ich machte eine Sammlung der Briefe, die ich an 
O. Cromwell\person{Cromwell} und seinen Sohn Richard 
geschrieben hatte zur Zeit ihres
Protektorats, und an die damaligen Parlamente und Behörden.
Ebenso sammelte ich die Briefe, die ich an König Karl II nach
seiner Rückkehr geschrieben hatte und an seine Räte und die
Richter und Beamten unter ihm. Ich machte auch eine Sammlung 
der Zeugnisse, die ich von den verschiedenen Statthaltern,
Richtern und Räten, Parlamentsmitgliedern und andern erhalten
hatte, um mich von allerlei Verleumdungen zu reinigen, welche
die bösen Priester und \zitat{Frommen}, diesseits und jenseits des
Meeres, auf mich geworfen hatten. Solches tat ich um der
Wahrheit willen, weil ich wusste,\index{Archivierung} das der 
Zweck ihrer Verleumdung\index{Verleumdung}
war, die von mir verkündete Wahrheit zu schmähen, und ihre
Verbreitung unter den Leuten zu hindern. Außerdem machte ich
noch zwei Sammlungen, die eine war eine Liste oder ein 
Verzeichnis der Namen derjenigen Freunde, welche im Norden von
England aufgetreten waren, als die Wahrheit dort zuerst 
hervorgebrochen war, um in jenen Gegenden den Tag des Herrn zu
verkünden. Das andere waren die Namen derjenigen Freunde,
welche zuerst das Evangelium in andern Ländern, Gegenden und
Ortschaften predigten, und in welchem Jahr, und wohin sie gegangen. 
Dann machte ich eine andere Sammlung, in zwei Büchern;
% \picinclude{./240-249/p_s246.jpg} 
in dem einen waren die an mich gerichteten Schreiben und Briefe
von Freunden und andern, bei verschiedenen Gelegenheiten; im
andern waren meine Briefe an Freunde und andere. Ich schrieb
auch ein Buch über die Zeichen und Sinnbilder von Christus
und ihre Bedeutung, und manche andere Dinge, die den Freunden
und der Wahrheit künftig von Nutzen sein werden [...].

Weil ich sah, wie die Wahrheit sich immer mehr ausbreitete
im Lande, und die Zahl der Freunde stets zunahm, so trieb mich
die ewige Kraft des Herrn, auch zum Einrichten von Frauen-Versammlungen 
zu raten, damit alle, sowohl Männer wie Frauen,
welche das Evangelium, das Wort vom ewigen Leben, empfangen,
getrieben von der Kraft Gottes, in die Ordnungen des Evangeliums 
kommen möchten, und in der Kraft für den Herrn wirken
und in derselben dienen möchten, zu seiner und der Kirche
Frommen [...].

Etliche von denen, die Vorgaben, zu den Bekennern der
Wahrheit zu gehören und dies zur Schau getragen hatten,
waren, statt bei dem einfachen Evangelium zu bleiben, in
Zänkereien\index{Zänkereien} und Spaltungen\index{Spaltungen} 
geraten und versuchten, den Freunden,
besonders den Frauen, ihre göttliche Wachsamkeit, die sie in der
Kirche gegenseitig aneinander nach der Wahrheit ausübten, zu
verleiden, indem sie sich ihren Versammlungen widersetzten, die
eben zu diesem Zwecke eingerichtet wurden; darum trieb mich der
Herr, einen Brief zu schreiben, und unter den Freunden zu 
Verbreiten, um den Geist, in dem diese Widersacher handelten, 
aufzudecken und die Freunde davor zu warnen; es hieß darin
unter anderem:

\brief{Quaker-Gemeinde}{
  \index{Innere Konflikte}\index{Geschäftsversammlung}
  Etliche, die in diesem Geiste wandeln, haben mir gesagt,
  sie sehen den Nutzen der Frauenversammlungen nicht ein.\index{Frauenversammlungen}
  Meine Antwort an solche war und ist noch, das wenn sie blind
  sind und nichts sehen, sie sich wenigstens nicht den andern
  widersetzen sollen, denn es widersetzt sich ihnen auch niemand;
  Gott hat die Blindheit nie als Verdienst gerechnet, und sein Volk
  soll es auch nicht. Vielmehr hat Christus alle erleuchtet, und so
  Viele ihn aufnahmen, denen gab er Macht, Gottes Kinder zu
  sein (Joh.1,12)\bibel{Joh. 01:12@Joh. 1:12} Die, welche 
  Erben seiner Kraft sind und seines
  Evangeliums, das \zitat{Leben und unsterbliches Wesen ans Licht
  bringt} (2. Tim. 1,10)\bibel{Tim. 2. 01:10@2. Tim. 1,10}, 
  stehen über der Macht der Finsternis, die
  jene verfinstert; sie halten die Gebote des Evangeliums, und sie
  % \picinclude{./240-249/p_s247.jpg} 
  halten ihre Versammlungen in der Kraft Gottes, die sie im Leben
  und dem unsterblichen Wesen bewahret, sie sehen den Nutzen
  der Männer- und Frauen-Versammlungen nach der Ordnung des
  Evangeliums, dieser Gotteskraft, ein, denn sie haben Teil an dieser
  Kraft, auf welche sich die Versammlungen gründen. Darum
  sage ich euch allen, die ihr gegen die Versammlungen der Frauen
  seid, oder auch der Männer, wenn ihr keinen Nutzen in den
  Versammlungen der Frauen seht und euch denselben widersetzt,
  so seid ihr darin nicht in der Kraft Gottes und lebet nicht in
  seinem Geist. Denn Gott sah einen Nutzen in den Zusammenkünften 
  der Frauen in den Tagen des Gesetzes, für allerlei Geschäfte 
  zu seiner Ehre und seinem Dienst, und für die heiligen
  Verrichtungen in seiner Hütte, und so erkennen sie nun die, welche
  seinen Geist haben, in ihrem Dienst am Evangelium. Vieles in
  diesen Versammlungen eignet sich besser für Frauen als für Männer,
  sie sollen darum in der Kraft und Weisheit Gottes die Männer
  über die Dinge, die diese nicht verstehen, belehren, und die
  Männer sollen die Frauen in den Dingen, die die Frauen nicht
  verstehen, belehren\index{Belehrungen}, als gegenseitige Mithelfer. Denn in den Tagen
  des Gesetzes mussten die Frauen so gut wie die Männer opfern,
  wie Vielmehr also sollten sie in den Tagen des Evangeliums ihre
  geistlichen Gaben darbringen, denn sie sind alle, Männer wie
  Frauen, ein königliches Priestertum 
  (1. Petr. 2,9)\bibel{Petr. 1. 02:09@1. Petr. 2:9} genannt, sie
  gehören zu den Genossen des Glaubens, sie sind die lebendigen
  Steine, welche das geistliche Gebäude bilden, dessen Haupt Christus
  ist, und sie sollen im Dienste des Evangeliums ermutigt werden,
  denn alles was sie tun, Männer wie Frauen, sollen sie tun im
  Geiste und in der Kraft Gottes. Alle, die keinen Nutzen in den
  Versammlungen der Frauen sehen oder der Männer, sondern sich
  denselben widersetzen und Streit unter den Freunden anstiften,
  haben den weltlichen Geist, welcher unsern Versammlungen ent-
  gegen ist und sich ihnen widersetzt; sie haben denselben weltlichen
  Geist, welcher gegen das Reden der Frauen in den Versammlungen
  war und noch ist, den Geist derer die sagen, \zitat{die Weiber sollen
  schweigen in der Gemeinde} 
  (1. Cor. 14, 34)\bibel{Cor. 1. 14:34@1. Cor. 14:34}, obgleich der gleiche
  Apostel befiehlt, das die Männer schweigen sollen, so gut wie
  die Frauen, wenn kein Ausleger da sei (14, 28)\bibel{Cor. 1. 14:28@1. Cor. 14:28}. 
  Ihr seht also,
  das der Geist dieser Welt über diese Gegner gekommen ist, wenn
  sie sich schon einen andern Anschein geben; denn sie wollen, das
  % \picinclude{./240-249/p_s248.jpg} 
  wir überhaupt keine Versammlungen haben. Sie sind gegen die
  Versammlungen der Frauen, und etliche auch gegen diejenigen
  der Männer und sagen, sie sehen keinen Nutzen darin. \textbf{Mögen
  sie das Maul halten}\index{Mögen sie das Maul halten} und 
  sich nicht solchen widersetzen, die ihren
  Gottesdienst in diesen Versammlungen sehen [...].
  \bigskip 
  \begin{flushright}Swarthmore, 5. des 8. Monats 
  1676\index{Jahr!1676}. G. F.\end{flushright}
}

Ich verließ Swarthmore am 28. des 1. Monats 1677\index{Jahr!1677} [...]
Später ging ich nach York\ort{York} [...] Von da schrieb ich an meine
Frau folgendes:

\brief{Fell, Margaret}{
Liebes Herz,

\bigskip 

welcher ich liebevollen Gruß sende, sowie deinen Töchtern und
allen Freunden, die nach mir fragen. Es ist mein Wunsch, das
ihr alle bewahret bleibet im ewigen Samen des Herrn, in welchem
ihr Leben und Friede haben werdet, Herrschaft und Wohnort in
der ewigen Heimat, im Hause, das auf Gott gegründet ist. Durch
Gottes Kraft bin ich nach York gelangt, nachdem ich unterwegs
viele Versammlungen gehalten habe. Die Wege waren oft
schlecht und vom Schnee durchweicht gewesen, unsere Pferde
sanken oft ein, so das wir nicht reiten konnten, und oft hatten
wir starken Sturm und Regen, aber durch Gottes Kraft habe ich
alles überwunden. In Scarehouse\ort{Scarehouse} war eine sehr große 
Versammlung, ebenso in Burrowby, in welcher die Freunde von
Durham und Cleveland herbei kamen, und viele andere Versammlungen 
haben wir gehabt. In York hatten wir gestern eine
überaus zahlreiche Versammlung, da die Freunde von weit herum
dazu gekommen waren, alles war ruhig und sehr befriedigt.
O, die Herrlichkeit des Herrn leuchtete über allen! Heute haben
wir eine große Versammlung für Männer und Frauen gehabt,
da viele Freunde, Männer und Frauen, vom Lande gekommen
sind; alles war ruhig, und heute Abend werden wir die Versammlung 
für die Männer und Frauen der Stadt haben. John
Whitehead\person{Whitehead, John} ist hier mit Robert 
Lodge\person{Lodge, Robert} und andern; die Freunde
sind über die Maßen froh. Ich bin also in meinem heiligen
Element, im heiligen Werk für den Herrn, seinem Name sei Ehre
immerdar. Morgen gedenke ich die Stadt zu verlassen und gegen
Tadcaster zu gehen, ich kann nicht reiten wie früher, doch dem
Herrn sei Dank, das ich immerhin so reisen kann. Ich grüß
dich im Brunnen des Lebens, in welchem ihr, so ihr darin bleibet,
Erquickung zum Leben haben werdet, damit ihr wachsen möget
% \picinclude{./240-249/p_s249.jpg} 
und ewige Kräfte sammeln, um dem Herrn zu dienen und Genüge
zu haben. Ich befehle euch alle dem Gott der Kraft, der
allmächtig ist euch zu bewahren.

\bigskip 

\begin{flushright}York, den 16. des 2. Monats 1677. G. F.\end{flushright}
}

Darauf zog ich weiter. [...] Ich bemerkte während meiner
Reise bei manchen, die Vorgaben die Wahrheit zu bekennen, eine
Schlaffheit und Schläfrigkeit im Auftreten gegen das Zehntenwesen;\index{Kirchenseuer} 
denn wo immer der Geist der Spaltung in der Kirche
Eingang fand, schwächte er solche, die ihm Gehör schenkten
im Zeugnis gegen die Zehnten; darum trieb es mich, einen kurzen
Brief ergehen zu lassen, um die reine Gesinnung anzuspornen,
und in allen das christliche Zeugnis gegen des Antichrists\index{Andichrist} 
Bedrückung und Joch zu stärken und zu ermutigen:

\brief{Quaker-Gemeinde}{
  Meine lieben Freunde!

  \bigskip 

  Seid getreu im Herrn in euerm Zeugnis für Christum,
  welcher das levitische zehntennehmende 
  Priestertum\index{Priestertum} Aarons\person{Aaron}
  aufhob und seine Diener aussandte, umsonst zu predigen, wie sie es
  umsonst empfangen hatten, ohne Stab noch Tasche (Math. 10)\bibel{Math. 10}.
  Christi Jünger können nichts mit denen zu tun haben, die ein
  Geschäft aus dem Predigen machen, und gleich wie ein Zeugnis
  abgelegt werden musste gegen jene Zehnten, welche das Gesetz
  für Aaron und Levi\person{Levi} gebot, also muss Zeugnis abgelegt werden
  gegen diese Zehnten, welche von Menschen eingeführt wurden in
  den dunklen Zeiten des Papsttums\index{Papsttum}, und nicht durch Gott oder
  Christus. Nun ist es ein Widerspruch, mit Worten gegen die
  Priester zu schreien und dennoch sie zu unterstützen und zu
  füttern, damit sie nicht Streit anheben sollen. Darum hütet euch,
  denn wenn der Herr euch segnet mit äußeren Gütern, und ihr
  wendet sie den Baalspriestern\index{Baalspriester} zu, so möchte er füglich die
  äußeren Güter, die er euch gab, wieder von euch zurückfordern; sagte
  er doch, das seine Diener umsonst geben sollten, wie sie auch umsonst
  empfangen hätten; darum muss Zeugnis abgelegt werden in der
  Kraft und in dem Geist des Herrn gegen alle, die um Zehnten
  und Geld predigen, und die Zehnten nehmen oder geben, auf
  das alle zusammen stehen mögen zum Zeugnis für Jesus Christus,
  in seiner Kraft und seinem Geist, gegen die Zehnten-Händler.
  Denket daran, wie viele getreue Diener und Kämpfer des Herrn
  ihr Leben ihretwegen gelassen, in den Tagen des Herrn, und wie
  % \picinclude{./250-259/p_s250.jpg} 
  sie in den Tagen der Märtyrer gegen sie gezeugt haben. Denket
  auch daran, welches Gericht über die gekommen ist, welche den
  Freunden ihre Habe geraubt und sie ins Gefängnis getan, um
  der Zehnten und Unterstützungen willen. Darum führet den 
  Krieg\index{führet den Krieg}
  gegen das Tier weiter in der Kraft des Herrn, und füttert es
  nicht, nur damit es euch \zitat{Friede!} zurufen solle; solchen Frieden
  sollt ihr nicht annehmen, sondern ihr müsset ihn brechen und 
  verwerfen durch den Geist Gottes. Dann werdet ihr in diesem selben
  Geist vom Sohn des Friedens; jenen Frieden empfangen, den
  weder das Tier\index{Das Tier}, noch die Hure, noch die Welt mit allen ihren
  irdischen Lehrern empfangen können und ihn euch auch nicht
  rauben können. Darum bewahret eure Herrschaft und Macht in
  der Kraft, dem Geist und dem Namen Jesu. Ich grüße euch in
  seiner Liebe.

\bigskip 
\begin{flushright}3. Monat des Jahres 1677.\index{Jahr!1677} G. F.\end{flushright}

}
