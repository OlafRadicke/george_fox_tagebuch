% \picinclude{./140-149/p_s140.jpg} 
% Hier ist ein Teil des Textet nach 230-239 verschoben worden
% wegen Formatierung.
% \picinclude{./240-249/p_s241.jpg} 
Um diese Zeit hatte ich einen Krankheit?-anfall der mich sehr
angriff und schwächte, und der einige Zeit anhielt, so daß einige
Freunde an meiner Wiederherstellung zweifelten. ES kam mir
vor, ich wandle unter Gräbern und Leichen, aber die unsichtbare
Macht hielt mich innerlich aufrecht und gab mir erquickende Kraft,
selbst wenn ich so schwach war, daß ich kaum sprechen konnte.
Einmal, alö ich des Nachtß auf meinem Bett lag in der Be-
trachtung der Herrlichkeit Gotteß, die über alles ist, hörte ich
eine Stimme, daß der Herr noch viel Arbeit an seinem Werke für
mich habe, ehe er mich zu sich nehmen könne.
ES wurden Maßregeln getroffen, um meine Freilassung zu
bewirken, wenigstens für so lange, biz ich mich erholt hätte, aber
eö zeigte sich, daß ez schwierig war, sie zu erlangen, denn der
König wolle mich auf keinem andern Weg alö dem der Begna-
digung sreilassen, da man ihm gesagt habe, nach dem Gesetz
könne er es nicht tun, und durch Begnadigung wollte ich nicht
frei werden, denn daß schien sich mir nicht mit meiner Unschuld
zu vereinbaren .....
Nun ging meine Frau nach London und erzählte dem König
von meiner langen unverschuldeten Gefangenschaft und der Art
meiner Gefangennahme, wie die Richter mit mir verfahren waren,
indem sie mir den Eid alß Falle oorlegten, worauf sie mich den
Strafen des Praemunire unterwarfen, und es stehe nun bei ihm,
ihren Wunsch zu erfüllen und mich srei zu sprechen. Der König
antwortete ihr freundlich und wieö sie an den Kanzler; zu diesem
ging sie, konnte jedoch nicht erreichen, wa?. sie wünschte, denn
dieser sagte, der König könne mich nicht anders als durch Be-
gnadigung frei lassen, und,-'ich hatte keine Freiheit, mich begnadigen
zu lassen, da ich wußte, daß ich nichtö Böses getan hatte. Wenn
ich mich hätte wollen begnadigen lassen, so hätte ich nicht solange
zu warten brauchen, denn der König war schon längst willenö ge-
wesen, mich zu begnadigen, und hatte zu Thomas Moore gesagt,
ich brauche mir keine Bedenken zu machen, mich begnadigen zu
lassen, denn eß sei schon mancher, der so unschuldig gewesen wie
ein Kind, begnadigt worden, aber ich konnte mich nicht dazu ver-
stehen. Lieber wollte ich mein ganzeö Leben lang im Gefängnis
bleiben, alß aus demselben befreit zu werden auf eine Art, die
irgendwie der Wahrheit zur Unehre gereichen konnte, und ich zog
ez darum vor, daß die Rechtmäßigkeit meiner Anklage vor Ge-
George F-3. 16


% \picinclude{./240-249/p_s242.jpg} 
242 Kapitel IZ.
richt geprüft werde ..... So wurde denn ein Befehl
nach Worcester geschickt, mich nochmals in Kings Bench zu
oerhören, um meine Anklage zu prüfen. Jch reiste also am 4.
des- 12. Monats ab, am 8. kamen wir in London an, und
am 11. wurde ich vor die vier Richter von Kings Bench ge-
bracht, wo Corbet meine Sache führte. Er brachte einen neuen
Verteidigungs-grund nor, nämlich man dürfe niemand wegen eines
Prämunire1) einsperren. Daraus erwiderte Richter Hale: ,,Mr.
Corbet, diese Verteidigung hätten sie früher bringen sollen«, dieser
erwiderte: ,,Wir konnten keine Abschrift der Anklage bekommen.«
Der Richter sagte: ,,Das hätten sie uns sagen sollen, dann
hätten wir sie gezwungen, eine solche früher zu schicken.« ....
Corbet blieb dabei, daß man nicht wegen eines Praemunire
jemand gefangen neh1nen könne ..... »Gut«, sagte einer der
Richter, »wir müssen Zeit haben in unsern Büchern nachzusehext,
und die Statuten nachzuprüfen«. Somit wurde das Verhör auf
den folgenden Tag Verschoben.
Am folgenden Tage ließen sie diesen Verteidigungsgrund
lieber fallen und fingen gleich mit den Jrrttimern in der Anklage
an, und als- dieselben eröffnet wurden, so waren ihrer so oiele
und so große, daß die Richter alle die Überzeugung hatten,
die Anklage sei leer und nichtig und man solle mich frei
lassen ..... Gs waren an dem Tage mehrere angesehene Leute,
Lords und andere da, die den Huldigungs-eid ablegen mußten, und
einige meiner Gegner suchten die Richter zu bewegen, mir den-
selben auch noch einmal norzulegen, weil, wie sie sagten, es- ge-
fährlich sei, mich frei zu lassen. Aber Richter Hale sagte, er habe
allerdings schon dergleichen Gerüchte über mich gehört, aber noch
Viel mehr gute, und so erklärten er und die übrigen Richter mich
öffentlich frei .....
Während meiner Gefangenschaft in Worcester hatte ich trotz
meiner häufigen Krankheit, und trotzdem ich so ost nach London
und wieder zurückgezerrt wurde, mehrere Bücher für den Druck
geschrieben. Eines derselben war betitelt: ,,Eine Warnung an
England«. Ein anderes war: ,,An die Juden, um zu beweisen,
1) Prämunire der darm. techn. einer bestimmten Art von Verbrechen,
z. B. das Berweigern des Huldigungseides, das bestimmte Strafen wie Eni-
ziehung des Grundbcsitzes, oft auch Gefängnis 11. a. zur Folge hatte. Vgl-
Stephen, Englischez Strafrecht.


% \picinclude{./240-249/p_s243.jpg} 
Fox sammelt und ordnet die Bücher und Schristen usw. 243
daß der Messias gekommen ift«. Ein anderes: ,,Von der Jnspi-
ration, Offenbarung und Weissagung«. E-in anderes: »Gegen
alles unniitze Disputieren.« Ein anderes: »An alle Bischöfe und
Prediger, daß sie sich nach der Schrift prüfen«. Ein anderes:
»An die, welche sagen, wir lieben nur uns selbst«. E-in anderes
betitelt: »Unser Zeugnis von Christus«. Und ein anderes kleines
Buch: »Vom Schwören«, welches das erste war von den beiden,
die dem Parlament vorgelegt wurden. Außerdem schrieb ich noch
verschiedene Schriften und Briefe an Freunde, um sie zu er-
mutigen und zu stärken im Dienst für den Herrn. Denn etliche,
die sich als Bekenner der Wahrheit ausgegeben hatten, dann
aber einem Geist der Verführung Raum gegeben hatten und von
der Einigkeit und Bruderschaft des Evangeliums, in der die
Freunde stehen, abgewichen waren, hatten versucht, die Freunde
in ihrem Dienst zu entmutigen, besonders in der fleißigen und
wachsamen Fürsorge für die Ordnung der Angelegenheiten der
Kirche Christi .....
Kapitel II!.
Fox sammelt und ordnet die Bücher und Schriften, die er
geschrieben und tritt für die Frauenversammlungen ein.
Als ich nun wieder frei war, besuchte ich die Freunde in
London, und da ich mich gar nicht wohl fühlte, ging ich bald
nach Kingston. .... Ich blieb jedoch nicht lange dort, sondern
kehrte wieder nach London zurück, .... ging dann nach Eossel,
Preston, .... Lancaster .... und am 25. des 4. Monats nach
Swarthmore.
Als ich eine Weile dort war, kamen viele Freunde aus ver-
schiedenen Gegenden des Landes, um mich zu besuchen. Auch
von Schottland kamen manche; von diesen hörte ich, daß vier
junge Studenten in Aberdeen in diesem Jahre bekehrt worden
waren, bei einer Disputation von Robert Barclay-) und George
Keith mit einigen Schülern der Universität ..... Während ich
in dieser Stadt war, ließ ich mehrere Bücher drucken. Eines:
»Uber das Schwören«, ein anderes: ,,Niemand ist ein Nachfolger
der Apostel und Propheten, als wer ihnen nachfolgt in der gleichen
Kraft und dem gleichen Geist, darin sie waren.« Gin anderes:
11 Robert Barclay, der ,,TheO1oge der Quüker«, s. Weingarten a. a. O.
16-


% \picinclude{./240-249/p_s244.jpg} 
244 Kapitel 1):1.
,,Besitzen geht über Bekennen«, und daß die Bekennenden jetzt
Christus im Geist verfolgen, wie die jüdischen Bekennenden ihn
äußerlich verfolgten in den Tagen, da er im Fleisch wandelte .... .
Ferner die acht folgenden Bücher: ,,An die Behörden von Danzig«,
»Kain gegen Abel«, oder eine »Antwort auf das Gesetz der Männer
von Neu - England«, ,,An die Freunde zu Nevis über das Wach-
sam-sein«, »Ein Generalbrief an alle Freunde in Amerika«,
,,Über das, was des Kaisers und was Gottes ist«, »llber die
Ordnung in den Familien«, ,,Der geistliche Mensch richtet alle
Dinge«, ,,Über die höhere Krast«.
überdies schrieb ich mehrere Briefe, sowohl an die Freunde
in England, als auch jenseits des Meeres; auch Antworten
aus mehrere Flugblätter über die Abtrünnigkeit mehrerer, die
sich der Ordnung des Evangeliums widersetzt hatten, und viel
Unruhe und Zank in Westmvrland angestiftet hatten. Es trieb
mich darum, an die dortigen Freunde ein paar besondere Zeilen
zu schreiben.
,,An die Freunde in Westmorland,
Leber alle in der Kraft Gottes, in seinem Licht und Geist,
die euch zuerst bekehrten, daß ihr durch sie in der ersten Einigkeit
bleibet, in Demut und in der Furcht Gottes, und seiner friedsamen,
sanften Wahrheit, welche ihr leicht erbitten könnt, damit ihr in
dieser Kraft, diesem Licht und Geist alle dienstbereit seid in euern
Männer- und Frauen-Versammlungen, im Besitz der Ordnung
des Evangeliums, welches Evangelium Leben und unvergängliches
Wesen ans Licht gebracht hat ..... Darum, ihr Freunde in
Westmorland, bleibet in der Kraft Gottes; sie muß euch behüten
und schützen, wenn ihr wollt geschützt sein. Lasset euern Glauben
in der Kraft Gottes stehen, nicht in der Weisheit menschlicher
Worte, auf daß ihr nicht sallet. In Gottes Kraft habt ihr
Friede, Leben und Einigkeit; und weil ihr nicht in der Kraft
Gottes geblieben seid, in seiner Gerechtigkeit und seinem Heiligen
Geist, ist dieser Zank unter euch gekommen.« G. F.
Jch schrieb auch einen Generalbrief an die Freimde an der
Jahresversammlung in London, .... darin hieß es unter
anderem:
»Was die wahren Männer- und Frauen-Versammlungen
anbelangt, die nach Gottes Rat eingesetzt waren, so widersetzt
sich jeder, der sich ihnen widersetzt, zugleich der Kraft Gottes, auf


% \picinclude{./240-249/p_s245.jpg} 
Fox sammelt und ordnet die Bücher und Schriften usw. 245
deren Befehl sie eingesetzt sind. Die, welche sich dieser Kraft
widersetzten, sind nicht Diener des Gvangeliumß oder Christi .....
Unsere Männer- und Frauen-Versammlungen, und alle anderen
Versammlungen, die im Namen Jesu geschehen, sind nach dem
Evangelium Christi, nach der Kraft Gottes-, eingesetzt, also nicht
von Menschen oder durch Menschen. Darin sollen alle sich ver-
sammeln und Gott anbeten; darin sollen auch alle handeln, und
darinnen haben auch alle Gemeinschaft untereinander, ein sroheß,
sriedsameö Beisammensein.
Alle gläubigen Männer und Frauen in jedem Land und
jeder Stadt, deren Glauben in der Kraft Gottes- steht, im Evan-
gelium Christi, und die dieses Evangelium, diese Kraft Gotteß, be-
sitzen, haben alle ein Recht an die Kraft in diesen Versammlungen;
denn sie sind alle Erben der Kraft, nach der die Männer- und
Frauen-Versammlungen eingesetzt sind.« ....
Um diese Zeit sammelte ich so viel wie möglich von den
Briefen, die ich in früheren Jahren an die Freunde geschrieben
hatte. Jch machte eine Sammlung der Briefe, die ich an O. Crom-
well und seinen Sohn Richard geschrieben hatte zur Zeit ihres
Protektorat?-, und an die damaligen Parlamente und Behörden.
Ebenso sammelte ich die Briefe, die ich an König Karl ll nach
seiner Rückkehr geschrieben hatte und an seine Räte und die
Richter und Beamten unter ihm. Jch machte auch eine Samm-
lung der Zeugnisse, die ich von den verschiedenen Statthaltern,
Richtern und Räten, Parlamentßmitgliedern und andern erhalten
hatte, um mich von allerlei Verleumdungen zu reinigen, welche
die bösen Priester und ,,Frommen«, dießseitß und jenseitö dee-’
Meere?-, auf mich geworfen hatten. Solcheß tat ich um der
Wahrheit willen, weil ich wußte, daß der Zweck ihrer Verleumdung
war, die von mir verkündete Wahrheit zu schmähen, und ihre
Verbreitung unter den Leuten zu hindern. Außerdem machte ich
noch zwei Sammlungen, die eine war eine Liste oder ein Ver-
zeichnis der Namen derjenigen Freunde, welche im Norden von
England aufgetreten waren, als die Wahrheit dort zuerst hervor-
gebrochen war, um in jenen Gegenden den Tag des Herrn zu
verkünden. Das andere waren die Namen derjenigen Freunde,
welche zuerst das Evangelium in andern Ländern, Gegenden und
Ortschaften predigten, und in welchem Jahr, und wohin sie ge-
gangen. Dann machte ich eine andere Sammlung, in zwei Büchern;


% \picinclude{./240-249/p_s246.jpg} 
246 Kapitel K11.
in dem einen waren die an mich gerichteten Schreiben und Briefe
von Freunden und andern, bei verschiedenen Gelegenheiten; im
andern waren meine Briefe an Freunde und andere. Ich schrieb
auch ein Buch über die Zeichen und Sinnbilder von Christa-Z
und ihre Bedeutung, und manche andere Dinge, die den Freunden
und der Wahrheit künftig von Nutzen sein werden .....
Weil ich sah, wie die Wahrheit sich immer mehr au-Zbreitete
im Lande, und die Zahl der Freunde stetß zunahm, so trieb mich
die ewige Kraft des Herrn, auch zum Einrichten von Frauen-
Versammlungen zu raten, damit alle, sowohl Männer wie Frauen,
welche da-Z Evangelium, daß Wort vom ewigen Leben, empfangen,
getrieben von der Kraft Gotteö, in die Ordnungen deö Evange-
lium?-’ kommen möchten, und in der Kraft für den Herm wirken
und in derselben dienen möchten, zu seiner und der Kirche
Frommen .....
Etliche von denen, die Vorgaben, zu den Bekennern der
Wahrheit zu gehören und dietz zur Schau getragen hatten,
waren, statt bei dem einfachen Evangelium zu bleiben, in
Zänkereien und Spaltungen geraten und versuchten, den Freunden,
besonderß den Frauen, ihre göttliche Wachsamkeit, die sie in der
Kirche gegenseitig aneinander nach der Wahrheit au?-übten, zu
r-erleiden, indem sie sich ihren Versammlungen widersetzten, die
eben zu diesem Zwecke eingerichtet wurden; darum trieb mich der
Herr, einen Vries zu schreiben, und unter den Freunden zu Oer-
breiten, um den Geist, in dem diese *Widersacher handelten, aus-
zudecken und die Freunde davor zu warnen; e3 hieß darin
unter anderem:
,,E-tliche, die in diesem Geiste wandeln, haben mir gesagt,
sie sehen den Nutzen der Frauenoersammlungen nicht ein.
Meine Antwort an solche war und ist noch, daß wenn sie blind
sind und nichtß sehen, sie sich wenigstens- nicht den andern
widersetzen sollen, denn eS widersetzt sich ihnen auch niemand;
Gott hat die Blindheit nie alz Verdienst gerechnet, und sein Volk
soll etz auch nicht. Vielmehr hat Christuß alle erleuchtet, imd so
Viele ihn aufnahmen, denen gab er Macht, Gotteö Kinder zu
sein (Joh.1,12) Die, welche Erben seiner Kraft sind und seineö
Evangelium?-, das ,,Leben und unsterblicheö Wesen anö Licht
bringt« (2. Tim. 1,10), stehen über der Macht der Finsterniß, die
jene verfinstert; sie halten die Gebote dez Goangeliumß, und sie


% \picinclude{./240-249/p_s247.jpg} 
Fox sammelt und ordnet die Bücher und Schriften usw. 247
halten ihre Versammlungen in der Kraft Gottes, die sie im Leben
und dem unsterblichen Wesen bewahret, sie sehen den Nutzen
der Männer- und Frauen-Versammlungen nach der Ordnung des
Eoangeliumß, dieser Gotteßkraft, ein, denn sie haben Teil an dieser
Kraft, auf welche sich die Versammlungen gründen. Darum
sage ich euch allen, die ihr gegen die Versammlungen der Frauen
seid, oder auch der Männer, wenn ihr keinen Nutzen in den
Versammlungen der Frauen seht und euch denselben widersetzt,
so seid ihr darin nicht in der Kraft Gottes und lebet nicht in
seinem Geist. Denn Gott sah einen Nutzen in den Zusammen-
künsten der Frauen in den Tagen dez Gesetze?-, für allerlei Ge-
schäfte zu seiner Ehre und seinem Dienst, und für die heiligen
Verrichtungen in seiner Hütte, und so erkennen sie nun die, welche
seinen Geist haben, in ihrem Dienst am Evangelium. Vieleö in
diesen Versammlungen eignet sich besser für Frauen alß für Männer,
sie sollen darum in der Kraft und Weiöheit Gotteö die Männer
über die Dinge, die diese nicht verstehen, belehren, und die
Männer sollen die Frauen in den Dingen, die die Frauen nicht
verstehen, belehren, als gegenseitige Mithelser. Denn in den Tagen
dez Gesetzeö mußten die Frauen so gut wie die Männer opfern,
wie Vielmehr also sollten sie in den Tagen deö Eoangeliumß ihre
geistlichen Gaben darbringen, denn sie sind alle, Männer wie
Frauen, ein königlicheß Priestertum (1. Petr. 2,9) genannt, sie
gehören zu den Genossen des Glaubens, sie sind die lebendigen
Steine, welche daß geistliche Gebäude bilden, dessen Haupt Christus
ist, und sie sollen im Dienste dez Goangeliumö ermutigt werden,
denn alleß wat-3 sie tun, Männer wie Frauen, sollen sie tun im
Geiste und in der Kraft Gottee-.1 Alle, die keinen Nutzen in den
Versammlungen der Frauen sehen oder der Männer, sondern sich
denselben widersetzen und Streit unter den Freunden anstiften,
haben den weltlichen Geist, welcher unsern Versammlungen ent-
gegen ist und sich ihnen widersetzt; sie haben denselben weltlichen
Geist, welcher gegen da; Reden der Frauen in den Versammlungen
war und noch ist, den Geist derer die sagen, »die Weiber sollen
schweigen in der Gemeinde« (1. Cor. 14, 34), obgleich der gleiche
Apostel befiehlt, daß die Männer schweigen sollen, so gut wie
die Frauen, wenn kein Auöleger da sei (14, 28). Jhr seht also,
daß der Geist dieser Welt über diese Gegner gekommen ist, wenn
sie sich schon einen andern Anschein geben; denn sie wollen, daß


% \picinclude{./240-249/p_s248.jpg} 
248 Kapitel K11.
wir überhaupt keine Versammlungen haben. Sie sind gegen die
Versammlungen der Frauen, und etliche auch gegen diejenigen
der Männer und sagen, sie sehen keinen Nutzen darin. Mögen
sie das Maul halten und sich nicht solchen widersetzen, die ihren
Gottes-dienst in diesen Versammlungen sehen ..... «
Swarthmore, 5. dez 8. Monats 1676. G. F.
Jch verließ Swarihmore am 28. des- 1. Monatiz 1677 ....
Später ging ich nach York ..... Von da schrieb ich an meine
Frau folgendes:
,,LiebeZ Herz,
welcher ich liebevollen Gruß sende, sowie deinen Töchtern und
allen Freunden, die nach mir fragen. GZ ist mein Wunsch, daß
ihr alle bewahret bleibet im ewigen Samen dez Herrn, in welchem
ihr Leben und Friede haben werdet, Herrschaft und Wohnort in
der ewigen Heimat, im Hause, daß auf Gott gegründet ist. Durch
Gottes Kraft bin ich nach York gelangt, nachdem ich unterwegs
viele Versammlungen gehalten habe. Die Wege waren oft
schlecht und vom Schnee durchweicht gewesen, unsere Pferde
sanken oft ein, so daß wir nicht reiten konnten, und oft hatten
wir starken Sturm und Regen, aber durch Gottes Kraft habe ich
alleö überwunden. Ju Searehouse war eine sehr große Ver-
sammlung, ebenso in Burrowby, in welcher die Freunde von
Durham und Cleveland herbei kamen, und viele andere Ver-
sammlungen haben wir gehabt. Ju York hatten wir gestern eine
überaus zahlreiche Versammlung, da die Freunde von weit herum
dazu gekommen waren, alleö war ruhig und sehr befriedigt.
O, die Herrlichkeit des Herrn leuchtete über allen! Heute haben
wir eine große Versammlung für Männer und Frauen gehabt,
da viele Freunde, Männer und Frauen, vom Lande gekommen
sind; alleß war ruhig, und heute abend werden wir die Ver-
sammlung für die Männer und Frauen der Stadt haben. John
Whitehead ist hier mit Robert Lodge und andern; die Freunde
sind über die Maßen froh. Jch bin also in meinem heiligen
Element, im heiligen Werk für den Herrn, seinem Name sei Ehre
immerdar. Morgen gedenke ich die Stadt zu verlassen und gegen
Tadcaster zu gehen, ich kann nicht reiten wie früher, doch dem
Herrn sei Dank, daß ich immerhin so reisen kann. Jch grüße
dich im Brunnen des Lebens, in welchem ihr, so ihr darin bleibet,
Grquickung zum Leben haben werdet, damit ihr wachsen möget


% \picinclude{./240-249/p_s249.jpg} 
Fox sammelt und ordnet die Bücher und Schriften usw. 249
und ewige Kräfte sammeln, um dem Herrn zu dienen und Ge-
nüge zu haben. Jch befehle euch alle dem Gott der Kraft, der
allmächtig ist euch zu bewahren.«
York, den 16. deß 2. Monatß 1677. G. F.
Darauf zog ich weiter. .... Jch bemerkte während meiner
Reise bei manchen, die Vorgaben die Wahrheit zu bekennen, eine
Schlaffheit und Schläfrigkeit im Auftreten gegen daß Zehnten-
wesen; denn wo immer der Geist der Spaltung in der Kirche
Eingang fand, schwächte er solche, die ihm Gehör schenkten
im Zeugniß gegen die Zehnten; darum trieb eß mich, einen kurzen
Brief ergehen zu lassen, um die reine Gesinnung anzuspornen,
und in allen daß christliche Zeugniß gegen des Antichristß Be-
drücknng und Joch zu stärken und zu ermutigen:
,,Meine lieben Freunde!
Seid getreu im Herrn in euerm Zeugniß für Christum,
welcher daß levitische zehnteunehmende Priestertum Aaronß auf-
hob und seine Diener außsandte, umsonst zu predigen, wie sie eß
umsonst empfangen hatten, ohne Stab noch Tasche (Math. 10).
Christi Jünger können nichtß mit denen zu tun haben, die ein
Geschäft aus dem Predigen machen, und gleich wie ein Zeugnis
abgelegt werden mußte gegen jene Zehnten, welche daß Gesetz
für Aaron und Levi gebot, also muß Zeugniß abgelegt werden
gegen diese Zehnten, welche von Menschen eingeführt wurden in
den dunkeln Zeiten deß Papsttumß, und nicht durch Gott oder
Christuß. Nun ist eß ein Widerspruch, mit Worten gegen die
Priester zu schreien und dennoch sie zu unterstützen und zu
füttern, damit sie nicht Streit anheben sollen. Darum hütet euch,
denn wenn der Herr euch segnet mit äußeren Gütern, und ihr
wendet sie den Baalßpriestern zu, so möchte er siiglich die
äußeren Güter, die er euch gab, wieder von euch zurückfordern; sagte
er doch, daß seine Diener umsonst geben sollten, wie sie auch umsonst
empfangen hätten; darum muß Zeugniß abgelegt werden in der
Kraft und in dem Geist des Herrn gegen alle, die um Zehnten
und Geld predigen, und die Zehnten nehmen oder geben, auf
daß alle zusammen stehen mögen zum Zeugniß für Jesuß Ehristuß,
in seiner Kraft und seinem Geist, gegen die Zehnten-Händler.
Denket daran, wie viele getreue Diener und Kämpfer des Herrn
ihr Leben ihretwegen gelassen, in den Tagen deß Herrn, und wie

