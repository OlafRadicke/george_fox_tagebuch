
% \picinclude{./040-049/p_s040.jpg} 
sagte ihnen, ich sei hiestir tot. Sie sagten, ich sei ja am Leben.
Jch sagte ihnen, wo Neid und Zank sei, da sei Verderben (Jak. 3, 16).
Sie boten mir zweimal Geld an, aber ich wollte nichtß an-
nehmen; daraus wurden sie böse und verurteilten mich zum Ge-
fängnis- ....
Jch war tief betrübt und bearbeitet in meinem Geist während
meiner Gefangenschaft wegen der Schlechtigkeit, die in der Stadt
herrschte; denn obgleich etliche gewonnen waren, so war doch die
Mehrzahl sehr oerhärtet. Jch sah, wie sich das- Außgießen der
Liebe Gotteß von ihnen wegwandte. Ich trauerte über sie, und
es kam über mich, folgende Klage über sie zu verbreiten:
,,O Derby! Wie die Wasser abfließen, wenn die Schleusen
sich öffnen, also fließet die Liebe Gotteß von dir ab, o Derby.
Darum siehe zu, wo du stehest und auf welchem Grund du bist,
ehe du gänzlich verlassen wirst. Der Herr hat mich zweimal ge-
rufen, ehe ich zu dir kam, um gegen deine Eitelkeit und Schlech-
tigkeit aufzutreten und alle zu ermahnen, auf den Herm und
nicht auf Menschen zu sehen. ,,Wehe der prächtigen Krone der
Trunkenen! der welken Blume ihrer Herrlichkeit« (Jes. 28, ll.
Wehe denen, die mit Worten ihren Glauben zur Schau tragen und
doch hochmütig und hochfahrend sind und Unterdrückung und Haß
üben. O Derby! Deine Frömmigkeit und dein Predigen stinken
gen Himmel! Jhr feiert einen Sabbat in Worten und versammelt
euch, um euch schön zu kleiden, ihr frönet der Eitelkeit. Die
Weiber gehen mit aufgerichtetem Halse und geschminkten Ge-
sichtern, wie ez die alten Propheten verurteilt haben (Jes. 3, 16).
Eure Versammlungen sind dem Herrn ein Greuel; ihr erhebet
die Eitelkeit und beuget euch davor; das Laster gedeiht und da-Z
Böse wird geehrt; daö Schlechte wird von den Schlechten ge-
duldet und doch bekennen sie alle Christus mit Worten. O über
die Schlechtigkeit unter euch! EH bricht mir fast das Herz, zu
sehen, wie Gott unter euch verachtet ist, o Derby!«
A18 ich gesehen, wie Gottez Liebe sich von diesem Orte ab-
wandte, wußte ich, daß meine Gefangenschaft hier nun nicht mehr
lange andauern werde, aber ich sah, daß, wenn der Herr mich
srei machen werde, so werde eß sein, wie wenn man einen
Löwen auß seiner Höhle auf die wilden Tiere dee Waldes ab-
läßt. Denn alle »Frommen« hatten eine tierische Gesinnung, die
der Sünde huldigte, so lange sie lebten. Sie waren alle dem


% \picinclude{./040-049/p_s041.jpg} 
Erlebnisse im Gefängnis zu Derby usw. 41
Geist und dem Leben seiud, der in der Schrift gegeben ift und
den sie in Worten bekannten. So geschah ez, wie man hernach
sehen wird.
Ez stand ein Gericht über der Stadt, und den Behörden war
ez unbehaglich meinetwegen; aber sie wußten nicht, waz sie mit
mtr machen sollten. Einmal wollten sie mich vorz Parlament
schicken, ein andermal mich nach Jrland oerbannen. Zuerst
nannten sie mich einen Betrüger und Verfiihrer und Gottes-
lästerer; dann, alz Gott seine Strafe über sie schickte, sagten fie,
ich sei ein ehrlicher, tugendhafter Mensch. Aber ob sie eine gute
oder schlechte Meinung von mir hatten, war mir gleichgültig;
denn weder richtete mich daz eine auf, noch warf mich das andere
nieder, dem Herrn sei Lob. Schließlich mußten sie mich frei
lassen, zu Anfang des Winterz 1651, nachdem ich fast etn Jahr
in Derby gefangen gewesen war, sechz Monate im Zuchthauz
und die übrigen im Kerker.
Alz ich nun wieder meine Freiheit hatte, fuhr ich fort wie
zuvor in der Arbeit für den Herrn und zog im Lande umher,
zuerst in der Gegend meiner Heimat, Leicestershire; ich hielt unter-
wegz Versammlungen, und dez Herrn Geist und Kraft war mit
nur ....
Einmal alz ich mit einigen Freunden unterwegz war und
eine Turmhauzspitze erblickte, ging ez mir durch Mark und Bein;
ich fragte, waz daz für eine Ortschaft sei? ez hieß: Lichfield.
Alsobald erging daz Wort dez Herrn an mich, daß ich dorthin
gehen müsse. Alz wir bei dem Hause angelangt waren, in daz
wir gehen wollten, bat ich die Freunde, die mit mir waren, hinein-
zugehen; ich sagte ihnen aber nicht, wohin ich zu gehen hatte.
Sobald sie im Hause waren, entfernte ich mich und lief über
Hecken und Gräben, biz ich eine Meile weit von Lichsield ent-s
fernt war; da waren auf einem weiten Felde Schäfer, die ihre
Schafe hüteten. Hier befahl mir der Herr, meine Schuhe auzzu-
ziehen; ich zögerte, denn ez war Winter; doch daß Wort dez
Herrn war wie Feuer in mir. So zog ich denn meine Schuhe
aus und ließ sie bei den Schäfern, und die armen Schäfer zitterten
und waren ganz bestürzt. Darauf lief ich wieder eine Meile,
und sobald ich wieder in der Stadt war, erging daz Wort dez
Herrn an mich: ,,Rufe: wehe der blutigen Stadt Lichfield!« Ich
ging also die Straße auf und ab und rief: ,,Wehe der blutigen


% \picinclude{./040-049/p_s042.jpg}
Stadt Lichfield!« Da ez Markttag war, ging ich aus den Markt-
platz, lies aus demselben umher und rief von Zeit zu Zeit: ,,W-ehe
der blutigen Stadt Lichfield!« Und niemand tat mir etwaß.
Während ich rufend durch die Straßen ging, schien es mir, 11lS
ob ein Bach von Blut durch die Straße fließe, und der Markt-
platz kam mir vor wie ein Teich von Blut. Alö ich mich der
mir aufgetragenen Verkündigung entledigt hatte, verließ ich im
Frieden die Stadt. Jch kehrte zu den Hirten zurück, gab ihnen
Geld tmd erhielt meine Schuhe von ihnen zurück. Aber das
Feuer dez Herrn war so in meinen Füßen und in meinem ganzen
Körper, daß mir nichts daran lag, meine Schuhe überhaupt wieder
anzuziehen; und ich wußte nicht recht, ob ich ez tun sollte oder
nicht, bis ich die Grlaubniß dazu vom Herrn fühlte; nachdem ich
meine Füße gewaschen, zog ich meine Schuhe wieder an. Darauf
versiel ich in tiefeß Nachstnnen, warum und aus welchem Grunde
ich wohl gesandt worden sei, gegen diese Stadt zu reden und sie
die ,,blutige Stadt« zu nennen; denn obwohl eine Zeitlang daß
Parlament und eine Zeitlang der König die Herrschaft über diesen
Kirchenspengel gehabt hatte und viel Blut in der Stadt vergossen
worden war während des- Krieges zwischen beiden, so war ez
doch nicht schlimmer gewesen alß an vielen anderen Orten auch.
Nach und nach aber fiel es mir ein, wie zur Zeit dez Kaiserß
Diocletian tausend Christen in Lichsield gemartert worden waren;
darum hatte ich ohne Schuhe durch den Bach ihreß Bluteß gehen
müssen, damit die Erinnerung an das Blut jener Märtyrer, daß
vor mehr als tausend Jahren vergossen worden und in ihren
Straßen erkaltet war, wach werde. Die Nachwirkung jenes Bluteß
war über mich gekommen, so daß ich dem Herrn hatte gehorchen
müssen. Man weiß auö alten Uberlieserungen, wie viel christ-
liche Vriten dort gelitten haben. Ich könnte noch viel berichten
über alle:-’, waß sich mir offenbarte über daß hier während der
zehn Verfolgungen und später vergossene Märtyrerblut, aber ich
überlasse es dem Herrn und seinem Buch, au?7 welchem alleß
gerichtet werden wird; denn sein Buch und fein Geist sind sichere
Uberlieserer.
Darauf zog ich im Lande umher und hatte vielerorts Ver-
sammlungen unter den freundlich Gesinnten. Aber meine Ange-
hörigen waren böse über mich. Nach einiger Zeit kehrte ich nach
Nottinghamshire zurück und ging dann nach Derbshire, um dort


% \picinclude{./040-049/p_s043.jpg} 
Erlebnisse im Gefängnis zu Derby usw. 43
die freundlich Gesinnten aufzusuchen. Jn Yorkshire und an einigen
andern Orten predigte ich Buße: darauf kam ich nach Balby,
wo Richard Famöworth 1) und einige andere gewonnen wurden.
So reiste ich im Lande umher, Buße predigend und daö Wort
dez Herm verkündigend, bis-3 ich in die Gegend von Wakefield
kam, wo James Naylor lebte; er und Thomaß Goodyear
kamen zu mir; beide wurden gewonnen und nahmen die Wahrheit
auf. Auch William Dem?-bury und seine Frau und viele andere
kamen zu mir, wurden gewonnen und nahmen die Wahrheit auf.
Von dort begab ich mich nach Hauptmann PurZloe’S Hauß in
die Nähe von Selby, und besuchte John Leek, der inß Gefängniß
zu mir gekommen war, und er wurde gewonnen. Ich besaß ein
Pferd, mußte mich aber leider davon trennen, da ich nicht wußte,
maß damit anfangen, weil mich der Herr trieb in manches- an-
gesehene Hautz zu gehen, um die Leute zu ermahnen, sich zum
Herrn zu bekehren. Unter anderm trieb mich der Herr auch inß
Turmhauß von Beverly zu gehen, daß damalß eine Stätte beson-
derer Frömmigkeit war; da ich vom Regen ganz durchnäßt war,
ging ich zuerst nach der Herberge. Jn der Türe kam ein junge-3
Weib auf mich zu und sagte: ,,Wie! seid ihr ez? Kommt herein«,
wie wenn sie mich schon gekannt hätte; denn die Kraft dez Herrn
hatte ihr Herz vorbereitet. Ich nahm etwaß zu mir und ging
inß Bett. Am Morgen zog ich meine noch nassen Kleider an
und bezahlte meine Zeche und begab mich ins Turmhauö, wo
einer predigte. A15 er geendet, trieb mich die mächtige Kraft Gotteß,
zu ihnen zu reden, und ich wies sie aus Chrisiuz, ihren Lehrer, hin.
Die Kraft dez Herrn war so mächtig, daß alle von großer Furcht
ergriffen wurden. Der Bürgermeister kam und sprach ein paar
Worte mit mir, aber niemand hatte Macht, mir etwaß zu tun.
Jch verließ die Stadt und ging am Nachmittag in ein anderez
Turmhauß, etwa zwei Meilen weit entfernt. A13 der Priester
geendet, trieb ez mich, eingehend zu ihm und den Leuten über
den Weg deö Lebens und der Wahrheit und den Grund der Gr-
wählung und Verdammung zu reden. Der Priester sagte, er sei
1) Richard Farnsworth, William Dewßbury und James Naylor waren
die ersten bedeutenden Missionsprediger der Quätet. (Näheres s. Weingarten,
Revolutionskirchen Englandtz. S. 218ss.) James Naylor ist in der Geschichte
betiichtigt geworden durch seinen Messiatzeinzug in Bristol, dem Höhepunkt der
saft zum Wahnsinn gesteigerten Schwärmerei des älteren Quälertumö.


% \picinclude{./040-049/p_s044.jpg} 
zu kindlich, um mit mir zu dißputieren; ich erklärte ihm, ich sei
nicht gekommen, um zu di?-putieren, sondern um daß Wort dez
Lebenß und der Wahrheit zu verkünden, und damit sie alle den
Samen kennen lernen möchten, den Gott allen verheißeu, den
Männern wie den Frauen. Die Leute waren hier sehr empfänglich
und wünschten, daß ich wiederkäme an einem Wochentag, um
ihnen zu predigen, aber ich wie:3 sie an ihren Lehrer Jesuö
Christus und verließ sie. Am folgenden Tage ging ich nach
Cranstick zu Hauptmann Pnrßloe, der mich zu Richter Hotham
begleitete. Dieser war ein gottseliger Mann, der auch Gottes
Wirken schon in seinem Herzen verspürt hatte. Nachdem wir eine
Zeitlang über göttliche Dinge geredet hatten, nahm er mich mit
in sein Zimmer und bekannte mir, daß ihm diese Ansichten
schon seit zehn Jahren vertraut seien, und wie er sich freue, daß
der Herr sie nun auch verkünden lasse unter den Leuten. Nach-
her kam noch ein Priester zu ihm, mit dem ich auch über die
Wahrheit redete. Aber der war bald zum Schweigen gebracht,
denn er war ein bloßer Phantaft, der sich daß, wovon er redete,
innerlich nicht angeeignet hatte.
Während ich da war, kam eine angesehene Frau auß Beverly,
um Richter Hotham in irgend einer wichtigen Angelegenheit zu
sprechen. Jin Laufe dez Gesprächeß erzählte sie ihm, daß am
vergangenen Sabbat, wie sie diesen Tag nannten, ein Engel oder
ein Geist in die Kirche von Beverly gekommen sei und herrliche
Dinge von Gott geredet habe zur Verwunderung aller Anwesenden,
und alß er geendet habe, sei er verschwunden; sie wisse nicht, woher
er gekommen, noch wohin er gegangen sei, alle haben sich ge-
wundert, die Priester, die »Frommen« und die Behörden der Stadt.
Richter Hotham erzählte mir das- nachher wieder, woraus ich ihm
mitteilte, daß ich e3 gewesen, der an jenem Tage im Turmhauß
gewesen und die Wahrheit verkündet hatte ....
Am Nachmittag ging ich in ein andereß Turmhauö, wo ein
großer, angesehener Priester, ein Doktor, wie sie ihn nannten,
redete, einer von denen, die Richter Hotham wollte kommen lassen.
Jch ging hin und wartete, biS der Priester geendet hatte. Die
Worte, die er alö Text genommen hatte, waren: ,,Wohlan alle,
die ihr dursiig seid, kommet her zum Wasser, und die ihr nicht
Geld habt, kommt her, kauset und esset, kommt her und kauset
ohne Geld, beide:-3 Wein und Milch (Jes. 55, 1).*- Und der Herr


% \picinclude{./040-049/p_s045.jpg} 
Erlebnisse im Gefängnis zu Derby usw. 45
trieb mich zu sagen: ,,Komm herunter, du Verführer; heißest du
die Leute umsonst kommen und umsonst vom Wasser dez Lebenz
nehmen, und nimmst jährlich dreihundert Pfund dafür, daß du die
Schrift oerkündest? Errötest du nicht vor Scham? Tat der
Prophet Jesaiaß und Christuß, die diese Worte umsonst geredet
und mitgeteilt hatten, auch also? Sagte nicht Christus- zu seinen
Jüngern, alö er sie auösandte zu predigen: umsonst habt ihr ez
empfangen, umsonst gebet etz auch?« Der Priester machte sich
ganz bestürzt davon; nachdem er seine Herde verlassen hatte,
hatte ich so oiel Zeit, alß ich wollte, um zu den Leuten zu sprechen;
ich wieö sie von der Finsternis zum Licht und zur Gnade Gotteß,
die sie lehren und ihnen Rettung bringen werde, und zum Geist
Gotteß in ihrem Jnnern, der sie umsonst lehre.
Dann kehrte ich zu Richter Hothamö Hauß zurück; alß ich
eintrat, schloß er mich in seine Arme und sagte, sein Haus sei
mein Hau-3. Denn er freute sich sehr über daß Werk dez Herrn
und daß seine Kraft kund geworden. Dann erzählte er mir,
warum er am Morgen nicht mit mir zum Turmhauö gegangen
war, und was für Gründe er gehabt hatte; er hatte sich gesagt,
wenn er mit mir in-3 Turmhauß gehe, so würden die Wachen
mich ihm übergeben und da werde er so in die Sache verwickelt;
dann wisse er nicht, maß machen. Darum sei er froh gewesen,
alß Hauptmann Pur?-loe gekommen; aber keiner von ihnen war
in Amts-kleidung gewesen oder hatte den Kragen um den Halß I
gehabt. GZ war damalt-3 etwaß ganz Ungewöhnliche?-, daß einer
ohne Kragen inß Turmhauß kam; aber Hauptmann Purßloe
war ohne einen solchen mit mir ins Turmhauö gekommen, so hatte
die Kraft des Herrn ihn übernommen, daß er gar nicht daran
dachte.
Jch zog weiter und kam an einen Abend zu einer Herberge.
Jch bat die Wirtin, mir etwaß Fleisch zu bringen, wenn sie solches-
habe; aber weil ich »du« und ,,dich« zu ihr sagte, sah sie mich
besremdet an; ich fragte sie, ob sie Milch habe. Sie sagte: nein.
Jch merkte, daß sie nicht die Wahrheit sagte, und um sie noch
weiter zu prüfen, fragte ich sie, ob sie Rahm habe; sie verneinte eö
ebenfalls. Nun stand ein Butterfaß im Zimmer und ein kleiner
Knabe, der daneben spielte, steckte seine Hand hinein und stieß eß
um und oerschiittete allen Rahm vor meinen Augen auf den
Boden; da zeigte es sich, daß die Frau eine Lügnerin war. Sie


% \picinclude{./040-049/p_s046.jpg} 
erschrak, stieß eine Verwünschung aus, hob das Kind auf und
schlug es tüchtig; aber ich machte ihr Vorwürfe wegen ihrer
Lüge und ihres Betrügens. Nachdem der Herr solcherweise ihre
Betrügerei und Bosheit aufgedeckt hatte, verließ ich das Haus
und ging weiter, bis ich zu einem Heuschober kam und brachte
nun die Nacht darin zu im Regen und Schnee, denn es war
drei Tage vor dem Tag, den sie Ehristfest nennen.
f Am folgenden Tage kam ich nach York, wo etliche sehr gott-
selige Leute waren. Am Ersten Tage der darauffolgenden
Woche hieß mich der Herr in das große Münster gehen und zum
Priester Bowles und seinen Zuhörern reden tn ihrer großen
Kathedrale. Jch ging hin und als der Ptiestet geendet, sagte ich,
ich habe ihm und der Gemeinde eine Botschaft von Gott dem
Herrn zu bringen. ,,Dann sage sie schnell!« sagte einer der
,,Frotnmen« aus der Versammlung; denn es war gefroren und
schneite und war sehr kaltes Wetter. Jch sagte ihnen, solches
seiüdas Wort des Herrn an sie: ,,Jhr lebet in Worten, aber der
Herr der Allmächtige verlangt Früchte von euch.« Kaum waren
die Worte aus meinem Munde, so stießen sie mich hinaus und
warfen mich die Stufen hinunter; aber ich stand aus, ohne verletzt
zu sein und ging in meine Wohnung. Etliche wurden überzeugt;
denn schon die Seufzer, die ich ausstieß unter dem Druck und
dem Zwang des Geistes Gottes in mir, genügten, um vieler
i Herzen zu öffnen und zu ergreifen, sodaß sie bekannten, die Seufzer,
die ich ausstoße, machen ihnen Eindruck.; mein ganzes Wesen
war bedrückt davon, daß sie bekannten und nicht besaßen, Worte
machten und keine Früchte brachten.
Nachdem ich für den Augenblick meinen Dienst in York getan
hatte und etliche dort gewonnen worden waren und die Wahrheit
Gottes angenommen und sich zu seiner Lehre bekannt hatten,
verließ ich York und wandte mich nach Cleveland und fand dort
Leute, welche die Kraft Gottes geschmeckt hatten. Ich sah, daß
ein Same in jener Gegend war, und daß Gott dort ein demiitiges
Volk hatte. Unterwegs holte mich, gegen Abend, ein Päpstlicher
ein und redete mit mir über seine Religion und über ihre Gottes-
dienste, und ich ließ ihn alles sagen, was er aus dem Herzen
hatte. Ich brachte die Nacht in einer Schänke zu; am folgenden
Morgen trieb mich der Herr, zu diesem Päpstlichen zu reden. Jch
begab mich in seine Wohnung und zeugte gegen seine Religion


% \picinclude{./040-049/p_s047.jpg} 
Erlebnisse im Gefängnis zu Derbi; usw. 47
und alle ihre abergläubischen Gebräuche und sagte ihm, Gott sei
gekommen, sein Volk selbst zu lehren; das brachte den Papisten
dergestalt auf, daß es ihn aus seinem eigenen Hause trieb ....
Obgleich zu der Zeit der Schnee sehr tief war, fuhr ich fort
herutnzureisen und kam zu einem Marktflecken, wo ich viele
,,Fromme« traf, mit denen ich lange Unterredungen hatte. Ich
stellte ihnen viele Fragen, die sie nicht beantworten konnten, weil
sie sagten, man habe sie noch nie in ihrem Leben so schwere
Dinge gefragt. Von da ging ich nach Stath, wo ich ebenfalls
viele ,,Fromme« und einige Ranter traf. Ich hatte große Ver-
sammlungen unter ihnen undsviele Bekehrungen. Viele nahmen:die
Wahrheit aus, worunter einer, der hundert Jahre alt war; ein
anderer war ein Oberkonstabler und einer war ein Priester, namens
Philipp Scafe. Diesen machte der Herr später durch seinen Geist
zu einem freien Verkündiger seines freien Evangeliums.
Der Priester dieses Ortes war sehr hochfahrend und bedrückte
die Leute sehr mit seinen Abgaben. Wenn sie aus den Fischfang
gingen, so machte er sie Abgaben vom Erlös bezahlen, obgleich
sie dieselben so weit her hatten und sie bis nach Yarmouth zum
verkaufen brachten. Es trieb mich, dort ins Turmhaus zu gehen,
um die Wahrheit zu verkünden und den Priester bloß zu stellen.
Als ich mit ihm geredet hatte und ihm die Unterdrückung des
Volkes vorgestellt hatte, lief er davon. Die Ältesten der Gemeinde
waren sehr hochmiitig und leichtfertig; darum verließ ich sie, nach-
dem ich das Wort des Lebens verkündet hatte, weil sie dasselbe
nicht aufnehmen wollten. Aber das Wort des Lebens, das ich
unter ihnen verkündet hatte, blieb bei etlichen von ihnen, so daß
etliche der Ersten aus der Gemeinde des Nachts zu mir kamen,
und die meisten wurden gewonnen und bekannten sich zur Wahrheit;
so begann die Wahrheit sich in dieser Gegend auszubreiten, und
wir hatten große Versammlungen; dadurch wurden die Priester
zornig und die Ranter fingen an, unruhig zu werden und ließen
mir sagen, sie wollten eine Unterredung mit mir haben, die Priester,
welche Unterdrückung übten, und die Runter. Es wurde ein Tag
festgesetzt und die Ranter erichienen; es kam auch noch ein anderer
Priester, ein Schotte, aber der Priester, welcher sich der Unter-
drückung schuldig gemacht hatte, nicht. Philipp Scafe, der be
kehrte Priester, war bei mir und es erschienen viele Leute. Als
wir uns gesetzt hatten, erklärte ein Ranter, namens T. Bushel,


% \picinclude{./040-049/p_s048.jpg}
er habe ein Gesicht von mir gehabt; ich sei an einem großen Pult
gesessen und er habe kommen müssen und seinen Hut vor mir
abnehmen und sich tief vor mir verbeugen, und er habe eß getan;
und noch viele andere Schmeicheleien sagte er mir. Jch sagte
zu ihm, er habe daß nur erfunden und er solle zu sich selber
sagen: ,,Schäme dich, du Hund«. Er sagte, eß sei nur Neid von
mir, so zu sagen. Darauf fragte ich ihn, waß der Neid eigentlich
sei rmd wie er im Menschen entstehe und waß daß Htindische sei
und wie eß im Menschen entstehe. Denn ich sah genau, daß er
etwaß Hündisrheß hatte, und darum wollte ich von ihm wissen,
wie dieseß Hündische in ihm entstanden sei. ,,Denn«, sagte ich
ihm, ,,mir müssen zuerst von dem reden, maß in unserm Leib
geschieht, ehe wir von dem reden können, roaß außer dem Leibe
ist.« Damit stopfte ich ihm daß Maul und allen seinen Runter-
genossen, denn er war ihr Haupt. Dann ries ich den Priester,
welcher die Leute unterdrückte, aber er kam nicht; nur der schottische
Priester erschien, der mit wenig Worten zum Schweigen gebracht
war; denn eß war innerlich kein Leben in ihm von dem, waß er-
bekannte. Nun war die Gelegenheit da, mit den Leuten zu reden.
Jch zeigte klar, wie die Ranter waren und verglich sie mit den
Prahlern in Sodom. Jch zeigte, wie ihre Priester die gleiche
Sorte von Mietlingen seien, wie die falschen Propheten früherer
Zeiten, und wie die Priester damals daß Volk auch in dieser
Weise regierten, indem sie ihren Gewinn im Auge hatten und
um Geld ihr Amt besorgten und um schnöden Gewinnß willen
lehrten. Jch stellte Christuß und die wahren Propheten und die
Apostel den Priestern gegenüber und zeigte, wie Ehristuß, die
Propheten und die Apostel sie schon lange an ihren Früchten
erkannt hätten. Dann wieß ich sie aus den Lehrer in ihrem
Jnnern hin, Jesus Christuß, ihren Heiland. Und ich predigte
Christuß in den Herzen, nachdem ich alle diese Höhen geebnet
hatte. Die Leute waren alle ruhig und die Widersacher zum
Schweigen gebracht. Denn obgleich eß innerlich in ihnen kochte,
so hielt die Kraft sie doch gebunden, so daß sie nicht loßbrechen
konnten .....
Ein anderer Priester ließ mich holen, um mit mir zu reden,
und etliche ,,Freunde« gingen mit mir nach seinem Hauö. Alß
er hörte, daß wir gekommen seien, entwischte er auß dem Hause
und versteckte sich unter einer Hecke. Die Leute gingen, ihn zu


% \picinclude{./040-049/p_s049.jpg} 
Erlebnisse im Gefängnis zu Derby usw. 49
zu suchen und fanden ihn, aber sie brachten ihn nicht dazu, zu
uns zu kommen. Daraus ging ich in ein nahegelegenes Turm-
haus, wo der Priester und das Volk in großer Erregung waren,
denn eben dieser Priester hatte den Freunden mit allem Ntöglichen,
das er tun werde, gedroht; als ich aber kam, machte er sich davon,
denn die Kraft des Herrn kam über ihn und über die andern.
Ja, des Herrn ewige Kraft kam über die Erde und drang zu
den Herzen der Menschen und machte die Priester und die
,,Frommen« zittern. Sie machte die Geister der Erde und der
Lust erbeben, zu welchen sie Vorgaben zu beten, sodaß sie einen
Schreck bekamen, wenn es hieß: »Der Mann in den ledernen
Kleidern kommi!«1) An vielen Orten machten sich die Priester,
wenn sie das hörten, davon, so waren sie von Furcht vor der
ewigen Kraft Gottes ergriffen .....
Von hier gingen wir über Scarbvrough .... nach Malton .....
Am Ersten Tag kam eine Frau, eine der angesehensten ,,Frvmmen«
unter den Jndependenten, welche ein solches Vorurteil gegen mich
hatte, daß sie sagte, ehe sie kam, sie würde sich freuen, mich er-
hängt zu sehen; aber als sie kam, wurde sie gewonnen und ge-
hört seither zu den »Freunden«.
Daraus hatte ich hier große Versammlungen; es hätten noch
mehr Leute daran teil genommen, aber sie wagten es nicht, aus
Furcht vor ihren Angehörigen. Es wurde damals als etwas
Unerhörtes angesehen, daß man in Häusern predigte statt in der
,,Kirche«, wie sie es nannten; darum wurde sehr gewünscht, daß
ich ins Turmhaus gehe und ddrt rede. Einer der Priester schrieb
mir und lud mich ein, im Turmhaus zu predigen, und nannte
mich seinen Bruder. Ein anderer Priester, eine bekannte Persön-
lichkeit, hielt dort eine Stunde. Nun hatte mir der Herr während
meiner Gefangenschaft in Derby kund getan, ich solle in den
Turmhäusern predigen, um die Leute von denselben abzubringen,
und es kamen mir auch zuweilen Bedenken wegen der Kanzeln,
in denen die Priester herumsaulenzten. Die Turmhäuser und
Kanzeln verletzten mein Gefühl, weil sowohl die Priester als auch
das Volk sie Gotteshäuser nannten und im Wahne waren, daß
Gott da in äußern sichtbaren Häusern wohne, statt im Gegenteil
1) Fox trug immer Kleider aus Leder, die et wegen ihrer Einfachheit und
Dauerhaftigkeit allen andern Kleidungsstiicken vorzog. (Vgl. Carlyles, Surtor
Resartus: Ein Ereignis in der neuen Geschichte.)
George Fox. 4

