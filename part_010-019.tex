% \picinclude{./010-019/p_s010.jpg}

nämlich —— damit ihm die Ehre allein gebühre. Alle sind mit
Sünde und Unglauben behaftet, damit Christus, der erleuchtet
und Gnade, Glauben und Kraft gibt, den Vorrang habe [...].
Mein Verlangen nach dem Herrn wurde immer stärker und der
Eifer nach der Erkenntnis Christi und Gottes, ohne jegliche Hilfe
von Menschen oder Büchern. Denn obwohl ich die Schrift las,
die von Gott und Christus sprach, so kannte ich ihn doch nur
durch Offenbarung, als den, der den Schlüssel hat und auftut
(Offb. 3, 7), und als den Vater des Lebens, der mich durch seinen
Geist zum Sohne zog. Dann führte mich der Herr freundlich
weiter und ließ mich seine ewige, unendliche Liebe sehen, die
alles übertrifft, was die Menschen in ihrem natürlichen Zustand,
oder durch Bücher oder die Geschichte erkennen können;
und diese Liebe zeigte mir auch, wie ich selber war ohne ihn.
Ich zog mich zurück von allen anderen, denn durch die Liebe
Gottes sah ich deutlich, wie es um sie stand. Ich hatte keinen
Umgang mit irgendjemand, Priester oder Frommen«, oder
irgendwelchen Separatiften; sondern nur mit Christus, der der
Schlüssel ist, und der mir die Tür zum Licht und zum Leben ge-
öffnet hatte. Jch fürchtete mich vor allem Reden über irdische Dinge;
denn ich sah nur Verderbliches darin, und wie das Leben von Ver-
derben belastet war. Als ich selber in der Tiefe war und unter dem
Druck, da glaubte ich nicht, daß ich je wieder darüber Herr werden
würde; meine Trübsal, Bekümmernis und Versuchung war so
groß, daß ich ost glaubte, verzweifeln zu müssen, so sehr ward
ich versnchet; als aber Christus mir offenbarte, wie er vom gleichen
Satan war versuchet worden, und wie er über ihn Herr geworden
war und ihm den Kopf zertreten hatte (l.Mos. 3, 15), und wie durch
ihn, seine Kraft, sein Licht und seine Gnade und seinen Geist ich
auch siegen werde, da vertraute ich ihm. So war er es, der mir
austat als ich eingeschlossen war und weder Hoffnung noch Glauben
hatte. Christus, der mich erleuchtet hatte, schenkte mir sein Licht,
um daran zu glauben, er schenkte mir Hoffnung, die er selber in
mir ausrichtete, und er gab mir seinen Geist und seine Gnade,
die mir geniigten in meiner Schwachheit. Also erhielt mich der
Herr im tiefsten Elend und Jammer, die oft über mich kamen.
Jch sand in mir zweierlei Durst: nach der Kreatur, um dort
Hilfe und Kraft zu suchen, und nach Gott, dem Schöpfer, und
seinem Sohn Jesus Christus. Jch sah, daß die ganze Welt mir


% \picinclude{./010-019/p_s011.jpg} 

s Erweckung und Krisis bis zum Durchbruch. 11
nicht helfen konnte; wenn ich die Kost, den Palast und die
Dienerschast eines Königs gehabt hätte, so wäre es mir nichts
uütze gewesen; denn nichts konnte mich trösten, als die Kraft
des. Herm. Jch sah, daß die Ptiestet und die ,,Frommen«
und überhaupt die Menschen hohl waren und ganz zufrieden
in dem Zustand, der mich elend machte; und daß sie das
liebten, wovon ich gerne los geworden wäre. Aber der Herr,
von welchem meine Hilfe kam, nahm mein Anliegen auf sich, und
ich wars meine Sorgen aus ihn allein. Darum wartet alle ge-
duldig aus den Herrn, in welchem Zustand ihr auch sein möget;
wartet in der Gnade und Wahrheit, die von Christus kommt;
wenn ihr das tut, so habet ihr eine Verheißung, die der Herr t
an euch erfüllen wird. Wahrlich, selig sind alle, die da hungert
und dürstet nach Gerechtigkeit, denn sie sollen satt werden .....
Wiederum hörte ich eine Stimme, welche sagte: »Du, Schlange,
du suchst das Leben umzubringen, aber kannst es nicht; denn das
Schwert, das den Baum des Lebens (1. Mos. 3.) bewacht, wird
dich umbringen.« Christus, das Wort Gottes, das der Schlange,
dem Mörder, den Kopf zertrat, behütete mich, weil mein Jnneres »
empsänglich war für seinen guten Samen, diesen Samen, der der
Schlange, dem Mörder, den Kopf zertrat. Dieses inwendige
Leben sproßte in mir empor, also daß ich auf alle Einwände der
Priester und der »Frommen« antworten konnte, und brachte mir
Schristworte ins Gedächtnis, um sie zu widerlegen.
Einmal sah ich die große Liebe Gottes und ich wurde mit
Bewunderung über ihre Unendlichkeit erfüllt; ich sah, wer von
Gott ausgestoßen war, und wer ins Reich Gottes einging, und
wie man Einlaß bekommt durch Jesum, der mit seinem himm-
lichen Schlüssel die Tür öffnet; und ich sah den Tod, wie er
über alle Menschen hingegangen war und den Samen Gottes in
den Menschen und auch in mir unterdrückt hatte, wie aber nun
dieser Same in mir ausging und was die Verheißung war. Es
war ein Kampf in meinem Innern: Fragen stiegen in mir
aus über Gaben und Weissagungen; und dann wurde ich
versucht bis zur Verzweiflung, als ob ich gegen den heiligen Geist
gesündigt hätte. Jch war in großer Bangigkeit und Trtibsal
tagelang. Dennoch verließ ich mich ganz aus den Herrn. Ein-
mal als ich von einem einsamen Gang zurückkam, wurde ich so
von der Liebe Gottes eingehüllt, daß ich unaufhörlich die Größe


% \picinclude{./010-019/p_s012.jpg} 
leiner Liebe anftaunen mußte. Während ich in diesem Zustand
war, eröffnete mir die ewige Klarheit und Kraft, daß: ,,alleß ge-
schehen muß in und durch Christum; und daß er jenen Versucher,
den Teufel besiegt und umbringt tmd alle seine Werke und über
ihm steht; und daß alle diese Trübsal gut für mich war, und
die Versuchungen zur Prüfung meineß Glaubenß .dienten, den
C-hristuß mir gegeben. Der Herr schenkte es mir, daß ich durch
alle diese Trübsale und Versuchungen hindurch sehen konnte;
mein lebendiger Glaube wurde erweckt, daß ich sah, wie alleß
durch Christus, daß Leben, geschah, und ich glaubte an ihn. Wenn
irgend einmal meine Stimmung getrübt war, so blieb mein innerer
Glaube fest, und meine tiefgegriindete Hoffnung hielt mich wie
ein Anker im Meere?-grund und ankerte meine unsterbliche Seele
in ihren Bischof (1. Petr. 2,25), indem sie ihr half über den Wassern
der Welt, ihren wilden Wogen, Stürmen und Versuchungen zu
schwimmen. Ach, da wurden mir meine Trübsale, Anfechtungen
und Versuchungen klarer, denn je zuoor. Wenn es Licht ward
in mir, da wurde alleß, waß nicht vom Licht war — Finsterniß
Tod, Versuchung, Unrecht und Gottlostgkeit — offenbar und kam
anß Licht. Darnach entstand ein Feuer in mir, und ich sah »ihn
sitzen wie daß Feuer eineß Goldschmiedß und wie die Seife eineß
Wäscherß«. (Mal. 3, 2). Der Geist der Unterscheidung kam über
mich, durch welchen ich erkannte, maß meine eigenen Gedanken,
mein Seufzen und mein Stöhnen bedeutete, und maß mir die
Erkenntnis trübte, und woher mir die Offenbarungen kamen.
Alleß waß sich nicht in der Geduld bewähren und daß Feuer
nicht erdulden konnte, erkannte ich im Licht alß Seufzer deß
Fleisches, daß sich nicht in Gotteß Willen fügen wollte: dieseß
hatte mich so verdunkelt, daß ich nicht geduldig sein konnte in
Anfechtung, Trtibsal und Verwirrung. Ich konnte mein eigeneß
Jch nicht in den Tod anß Kreuz geben, daß unß die Kraft
verleiht, Gott zu leben; sie bewirkt, daß alleß maß unß
die Gegenwart Christi oerhüllt, maß daß Schwert deß Geisteß
niederschlägt und tötet, nicht weiter leben kann. Jch unter-
schied auch daß Seuszen deß Geisteß, der mir Offenbarungen
eingab und der mich bei Gott vertrat (Röm. 8, 20). Ju diesem
Geiste ist daß wahre Warten im Herrn aus die Erlösung des
Leibeß und der ganzen Kreatur. Durch diesen unsichtbaren Geist,
in dem daß wahre Seufzen geschieht, erkannte ich auch daß ver-



% \picinclude{./010-019/p_s013.jpg} 

Erweckung und Krisi-3 bis zum Durchbruch. 13
kehrte Seufzen und Flehen. Durch diesen unsichtbaren Geist
Unterschied ich in allem, was-«’ ich hörte, sah und schmeckte, das
Falsche, dach sich über den Geist erhebt und ihn dämpst und
betrübt; und ich sah, wie alle die darin waren, im Jrrtum waren
und Schaden nahmete und im falschen Bitten und Flehen und in
jenem Wandeln und Reden, darinnen man Gotteß Namen ver-
geblich anruft; in jenem Geist, der durch das ägyptische Meer
watet und bittet, aber nicht empfängt; denn sie hassen sein
Licht und widerstreisen dem heiligen Geist, sie verwandeln die
Gnade in Wollust und lehnen sich auf wider den heiligen Geist;
und wenden sich ab zoom Glauben, in welchem sie beten sollten,
und vom Geist, in dem sie bitten sollten (Jud.) . . . .
Jch hörte von einer Frau in Laneafhire, die 22 Tage ge-
fastei hatte und ich ging hin, um sie zu sehen; aber alß ich zu
ihr kam, sah ich, daß sie unter großer Versuchung war. Nach-
dem ich zu ihr gereiet von dem, waö ich vom Herrn empfangen
hatte, verließ ich sie denn ihr Vater war ein Großer unter den
,,Frommen«. Von ia ging ich zu den ,,Frommen« in Duckingsield
und Manchester, wo ich einige Zeit blieb und die Wahrheit unter
ihnen verkündete. G3 wurden etliche von ihr überzeugt und nahmen
die Lehre dez Herrn an und wurden durch dieselbe fest gemacht und
blieben in der Wahrheit. Aber die ,,Frommen« waren wütend;
denn sie eiferten alle für die Lehre von der Sündhaftigkeit und
konnten es nicht ertragen, von Vollkommenheit sprechen zu hören
und von einem heilixen, sitndlosen Leben. Aber de; Herrn Macht
war über allen, wenn sie gleich in Finsterniß gebunden waren
und in der Sünde, iiir die sie eiferten und daß- Gottselige in sich
erstickten. GS war zu der Zeit eine große Versammlung der
Baptisten in Brougthon in Leieestershire, mit etlichen, die sich
von ihnen loßgetrenmt hatten; eö gingen auch Leute von anderen
Richtungen hin und ich ging auch; es waren nicht viele Baptisten
aber viele andere dort. Der Herr öfsnete mir den Mund und
die ewige Wahrheit wurde unter ihnen verkündet, und die Macht
dez Herrn war über ihnen allen. Ju diesen Tagen fing die Macht
des Herrn an zu treilsen und ich hatte große Ofsenbarungen über die
Schrift. ES wurdenetliche in dieser Gegend gewonnen und kehrten
sich von der Finstevuiß zum Licht, von der Macht des- Satans zu
Gott, rmd manche werden erweckt zu Gottes Preis-. Ob ich mich an
,,Fromme« oder andere wandte, stets- wurden etliche gewonnen.


% \picinclude{./010-019/p_s014.jpg} 
Ich war damals noch in großen Versuchungen und meine
inneren Leiden waren schwer; aber ich fand keinen, dem ich meinen
inneren Zustand hätte eröffnen können, als allein den Herrn, zu
dem ich Tag und Nacht schrie. Jch ging zurück nach Notting-
hamshire, und dort zeigte mir der Herr, daß das Böse, das sich
in den äußeren Dingen zeigt, inwendig in den Herzen und Ge-
danken unserer bösen Menschennatur ist. Ich sah die Natur der
Hunde, Schweine, Schlangen, die Natur von Sodom und Agypten,
von Pharao, Kain, Jsmael, Esau 2c. inwendig in den Menschen,
während andere sie im Äußern suchten. Jch schrie zum Herrn:
,,Warum muß mir solches geschehen, da ich mich doch nie solchen
Lastern ergeben werde?« Und der Herr antwortete mir, ich
müsse einen Begriff bekommen von diesen Zuständen; wie·sollte
ich sonst zu allen den verschiedenen Zuständen sprechen können?
und ich erkannte die unendliche Liebe Gottes darin. Ich erkannte,
daß es einen Ozean des Todes und der Finsternis gibt, aber
auch einen unendlichen, unerschöpfllichen Ozean des Lichts und
der Liebe, der über den Ozean der Finsternis fließt. Jch sah K
auch darin die unendliche Liebe Gottes, und ich hatte große
Offenbarungen.
Als ich beim Turmhaus (eteeplebouze) 1) von Mansfield vor-
bei kam, sagte der Herr zu mir: ,,Das, was die Leute mit Füßen
treten, muß deine Nahrung sein«. Und während der Herr also
zu mir sprach, offenbarte er mir, daß das Volk und die »Frommen«
das Leben von Christus . . . das Blut des Sohnes Gottes, welches
mein Leben war, mit Füßen treten und von ihren Ginfällen leben,
wenn sie gleich von ihm schwatzen. Gs schien mir zuerst merk-
würdig, daß ich mich nähren sollte mit dem, wa-? die großen
»Frommen« mit Füßen traten; aber der Herr ossenbarte es mir
deutlich durch seinen ewigen Geist und seine Macht.
Die Leute kamen von nah und fern um mich zu sehen;
aber ich vermied, von ihnen ausgesucht zu werden; doch ich mußte
reden und ihnen allerlei eröffnen. Einer, namens Brown, hatte
L große Weissagungen und Gesichte über mich auf dem Totbett.
Er sprach von nichts anderm, als was ich schaffen werde als
1) Fox gebraucht die Bezeichnung ,,Turmhaus« statt Kitche, weil: ,,die
Fechsctzellnnter Kirche nicht ein Gebäude, sondern die Gemeinde der Gläubigen


% \picinclude{./010-019/p_s015.jpg} 

Erweckung und Ktisis bis zum Durchbruch. 15
Werkzeug des Herrn, und von andern sagte er, daß sie in Ver-
derben geraten werden; es erfüllte sich bei einigen, die damals
viel gegolten hatten. Als dieser Mann begraben war, legte sich
die Hand des Herm schwer auf mich, zum Erstaunen vieler, die
glaubten, ich müsse tot gewesen sein; während vierzehn Tagen
kamen viele, nm mich zu sehen. Ich war sehr verändert in Aus-
sehen und Gestalt, als ob mein Körper neu gebildet oder ver-
wandelt worden wäre. Während ich in diesem Zustand war,
schenkte mir der Herr einen Sinn und eine Gabe der Unter-
scheidung, womit ich deutlich erkannte, daß bei vielen, wenn sie
von Gott redeten und von Christus, die Schlange aus ihnen redete;
dies war hatt zu ertragen; doch das Werk des Herrn ging all-
mählich vorwärts, und meine Anfechtungen und Trübsale fingen
an abzunehmen, und Tränen der Freude entrannen mir, so daß
ich Tag und Nacht dem Herrn hätte Freudentränen weinen mögen,
mit demütigem, zerschlagenem Herzen. Jch tat einen Blick in das,
was ohne Ende ist, in Dinge, die nicht ausgesprochen werden
können, und in die Größe und Unendlichkeit der Liebe Gottes,
die sich nicht in Worten ausdrücken läßt; denn ich war durch H
den Ocean der Finsternis und des Todes und durch die Macht
des Satans gebracht worden vermöge der ewigen, herrlichen Kraft
Christi; und selbst durch jene Finsternis wurde ich gebracht, welche
die ganze Welt bedeckt und alles gebunden hält und alle dem Tode
preis gibt. Es war die gleiche Kraft Gottes, die mich durch
solches alles hindurch brachte, welche nachher das ganze Land,
die Priester wie die ,,Frommen« und das Volk ergriff.
Jch konnte von mir sagen, ich sei im geistigen Babylon, Sodom,
Egypten und im Grabe gewesen; aber durch die ewige Kraft Gottes
war ich Herr geworden über jene Mächte und hindurchgedrungen
in die Kraft Christi. Ich sah die Ernte weiß und den Samen
Gottes so dicht im Boden, wie nur je Weizen ausgesäet worden
war, und niemand ihn zu sammeln, darüber trauerte ich mit NT
Tränen.
Es ging das Gerücht über mich, ich sei einer, der den Geist
der Unterscheidung hätte; daraufhin kamen Viele zu mir von nah
und fern, »Fromme«, Priester und Volk. Die Macht des Herm
brach hervor, und ich hatte große Weissagungen; ich redete zu
ihnen von den göttlichen Dingen; sie hörten aufmerksam und an-
dächtig zu, gingen hinweg und machten es ruchbar.


% \picinclude{./010-019/p_s016.jpg} 
Dann kam der Versucher und setzte mir wieder zu und
klagte mich nn, ich hätte wider den heiligen Geist gesündigt; aber
ich mußte nicht, worin. Da kam mir der Zustand Paulus-’ in
den Sinn, wie er in den dritten Himmel verzückt gewesen undf
Dinge gesehen hatte, welche kein Mensch sagen kann, und wie
darauf ein Vote deß Satanß gesandt worden war, ihn mit
Fäusten zu schlagen. So überwand ich durch die Krast Christi
auch diese Versuchung.


%%%%%%%%%%%%%%%%%%% Kapitel 3. %%%%%%%%%%%%%%%%%%%%%%%%%%%%%%
\chapter[Erste Versammlungen]{Erste Versammlungen}

\begin{center}
\textbf{Erste Versammlungen und Proteste.}
\end{center}

Die Macht dez Herrn hatte nun, im Jahre 1648, schon vielen
die Herzen geöffnet, daß ste daß Wort des Lebenß und der Ver:
söhnung aufnahmen. A15 ich nun einmal im Hause eineß Freunde?-,
in Nottinghamshire, saß, erkannte ich, daß ein großeß Krachen
durch die ganze Erde gehen mußte und ein großer Rauch auf-
steigen, überall wo es krachte, und darnach würde ein großes
Beben entstehen: es war die Erde in der Menschen Herzen, die
erbeben mußte, bevor der Same Gotteß au-es der Erde hervor-
gehen konnte. Und so geschah eö: die Macht dez Herrn fing an,
sie erbeben zu machen, und wir fingen an, große Versammlungen
zu haben, und man spürte die mächtige Kraft und daß Wirken
Gottes unter den Leuten, zu ihrer und der Priester Erstaunen ....
Jch ging nach Manöfield, wo eine große Versammlung von
,,Frommen« und andern Leuten stattfand; da trieb etz mich zu
beten, und die Kraft des- Herrn war so mächtig, daß etz schien,
als ob das- ganze Hauö erbebte. Alö ich geendet, sagten etliche
der Frommen, es sei gerade wie in den Tagen der Apostel, da
sich ,,daT- Hauß bewegte, in dem sie versammelt waren« (Act. 2, 2).
Nachdem ich gebetet, wollte einer der ,,Frommen« beten, aber
dadurch kam eine Trübung und etwas toteß über sie und die
andern ,,Frommen« wurden betrübt über ihn und sagten, ez sei
eine Versuchung über ihn gekommen; darauf kam er zu mir und
bat mich, ich solle wieder beten, aber ich konnte nicht auf eineiz
Menschen Geheiß beten.
Bald darauf war abermalß eine Versammlung von ,,Frommen«


% \picinclude{./010-019/p_s017.jpg} 
Erste Versammlungen und Proteste. 17
und ein Hauptmann namenßt Stoddard wohnte ihr bei. Sie
redeten über das Blut Christi, und während sie darüber sprachen,
sah ich durch die unmittelbare Offenbarung des unsichtbaren
Geistes das Blut Christi. Und ich schrie auf und rief: »Seht
ihr nicht das Blut Christi? Seht in eure Herzen, wie ee eure
Herzen und Gewissen besprengt, daß sie, loß von den toten
Werken, dem lebendigen Gott dienen« (Gbr. 9). Denn ich sah
ez, das Blut dez neuen Testamenteß, wie ez ins Herz kam. Daß
erschreckte die »Frommen«; sie wollten daß Blut nur aus?-wendig,
nicht inwendig haben. Aber Hauptmann Stoddard war ergriffen
und sagte: ,,Laßt den Jüngling reden, hört ihn an«, alß er sah,
wie sie mich mit vielen Worten zu besiegen suchten.
ES waren auch eine Anzahl Priester da, die ster gottselig
galten; einer von ihnen hieß Kellett, und etliche, die empfänglichen
Gemüteö waren, gingen hin, um sie zu hören. GS trieb mich,
ihnen nachzugehen, um sie zu ermahnen, auf die Lehre Gotteß in
ihrem Jnnern zu hören. Damals war der Priester Kellett gegen
das Priesteramt; später jedoch nahm er selbst ein solchetz an und
wurde ein Verfolger.
Nachdem ich etliche Arbeit getan hatte in dieser Gegend,
ging ich durch Derbshire in meine Heimat Leicestershire, und
ez wurden mehrere, die empsänglich waren, gewonnen. A15 ich
von dort wegzog, begegnete ich einer großen Zusammenkunft
von »Frommen«, die im Freien beteten und die Schrift auß-
legten. Sie reichten mir die Bibel und ich öffnete sie beim 5. Kap.
des Jtzzatth., wo Ehristuß daß; Gesetz auölegt; und ich erklärte
ihnen en inneren Zustand und den äußeren Zustand worüber sie
in heftigen Streit gerieten und so auszeinandergingen; aber die
Kraft des Herrn nahm überhand.
Darauf hörte ich von einer großen Versammlung, die in
Leicester stattfinden würde; eß sollte eine Die-putation geben, die die
Preßbhterianer, Jndependenten, Baptisten und Common-Payen
Leute gleicherweise angehen sollte. Die Versammlung war in einem
Turmhause, und der Herr trieb mich, dorthin zu gehen und
zugegen zu sein. Jch hörte ihren Verhandlungen und Beweis-
führungen zu. Einige saßen in Kirchenstühlen und der Priester
war aus der Kanzel; es war eine große Menge versammelt.
Zuletzt tat eine Frau eine Frage über die Stelle bei Petrus:
»Wiedergeboren auß ewiglichem Samen, auö dem lebendigen Wort
George Ft--. 2



% \picinclude{./010-019/p_s018.jpg} 
Gottes-, das-8 ewiglich bleibet (1. Petr. 1). Der Priester sagte ihr:
,,Jch erlaube keiner Frau in der Kirche zu reden,« obgleich er
vorher allen die Freiheit erteilt hatte, zu reden. Da wurde ich-
von der Krast dez Herrn übermannt wie in einer Verziickung,
und ich erhob mich und fragte den Priester: »Nennst du dieß
hier, dieseö Turmhauz, eine ,,Kirche«? oder nennst du diese
bunte Menge eine Kirche?« Denn er hätte der Frau auf ihre
Frage antworten sollen, nachdem er vorher allen die Freiheit
erteilt hatte, zu reden. Anstatt mir zu antworten, sragte er mich:
maß eine Kirche sei. Ich sagte: »Die Kirche ist der Pfeiler und
Grund der Wahrheit, auß lebendigen Steinen gemacht, aus
lebendigen Gliedern (1. Petr. 2), eine geistige Haußgemeinde,
deren Haupt Christus- ift; aber er ist nicht daS Haupt einer bunten
Menge oder eines alten Hauseß auß Kalk, Steinen und Holz.«
Diese Worte brachten alleß auß Rand und Band; der Priester
kam auß seiner Kanzel, andere auö ihren Stühlen, und die Ver-
handlungen waren gestört. Ich ging in eine große Herberge und
dißputierte dort mit Priestern und ,,Frommen« aller Richtungen;
und alle waren furchtbar hitzig. Aber ich bestand auf der wahren
Kirche und ihrem wahren Haupt, trotz ihnen allen, bi-3 sie nach-
gaben und auöeinanderstoben. Einer schien sehr geneigt und kam
eine Zeit lang, in der Absicht, sich mir anzuschließen; aber bald
kehrte er sich ganz gegen mich und schloß sich einem Priester an,
trat für die Kindertanse ein, obgleich er vorher selber ein
Baptist gewesen war, und verließ mich. Aber etz wurden an dem
Tage etliche gewonnen; auch die Frau, welche die Frage getan
hatte, wurde gewonnen, samt den Jhrigen; und deö Herrn Kraft
und Herrlichkeit leuchtete über allen.
Hierauf kehrte ich zurück nach Nottinghamshire und ging
ink- Vale of Beavor. Unterwegß predigte ich den Leuten Buße
und ez wurden viele gewonnen, im Vale of Beavor und in den
Städten; denn ich blieb einige Wochen dort. Eineß Morgen?-,
alß ich am Feuer saß, kam eine große Wolke über mich, und eine
große Versuchung überkam mich; aber ich blieb ganz ruhig. Und
ich hörte eine Stimme zu mir sagen: »Alle Dinge gehen auß der
Natur heroor«; und die Elemente und die Sterne kamen über
mich, so daß ich ganz davon eingehiillt war. Aber die andern
im Hause merkten nichtß von all dem, weil ich ganz still und
ruhig war. Und weil ich still und ruhig war und wartete, so


% \picinclude{./010-019/p_s019.jpg} 

Erste Versammlungen und Proteste. 19
stieg eine lebendige Hoffnung in mir auf, und ich Vernahm deutlich
eine Stimme, welche sagte: ,,EZ gibt einen lebendigen Gott, der
alle Dinge geschaffen hat«; und sogleich verschwand die Wolke
und auch die Versuchung, und Leben breitete sich über alles; mein
Herz ward fröhlich und ich prieö den lebendigen Gott. Einige
Zeit darauf traf ich etliche, die behaupteten, ez gebe keinen Gott,
sondern alle Dinge gehen aus-3 der Natur hervor. Ich hatte einen
langen Di?-put mit ihnen und brachte sie herum, so daß mehrere
zugaben, es gebe einen lebendigen Gott. Da sah ich, daß etz gut
gewesen war, daß ich jene Prüfung durchgemacht hatte. Wir
hatten große Versammlungen in jenen Gegenden, denn die Kraft
deß Herm brach hervor in diesem Teil deß Landeö. A13 ich nach
Nottinghamshire zurück kam, traf ich eine Schar von verworrenen
Baptisten und andem; die Kraft des Herrn wirkte mächtig und
gewann viele unter ihnen. Darauf ging ich in die Umgegend von
Manöfield, wo die Kraft deß Herrn herrlich kund ward, in der
Stadt Manßfield und auch in anderen Städten. Jn Derbshire
wirkte sie in herrlicher Weise. Jn Eton in der Nähe von Derby
war eine Versammlung von Freunden; die Kraft dez Herrn tat sich
darin so mächtig kund, daß viele gewaltig erschüttert wurden, und
vieler Mund wurde aufgetan durch die Kraft dez Herrn. Viele wurden
vom Herrn getrieben in die Turmhäuser zu gehen, zu den Priestern
und zum Volk, um ihnen die ewige Wahrheit zu verkünden.
Einmal als- ich in Man?-field war, fand eine Sitzung der
Richter wegen dez Dingenö von Dienstboten statt. ES trieb
mich hinzugehen und den Richtern zu sagen, sie sollten die
Dienstboten richt am Lohn verkürzen. Jch kam in die Nähe
der Herberge, in der die Sitzung abgehalten wurde; aber
alL ich dort eine Musikantenbande traf, ging ich nicht hinein,
sondern gedachte am folgenden Morgen wieder zu kommen, hofsend,
sie dann in ernster Stimmung zu treffen, um mit ihnen zu ver-
handeln; denn ez schien mir jetzt nicht die geeignete Zeit. Aber
alö ich am Morgen kam, war alleö fort; da wurde mir ganz
schwarz vor den Augen, so daß ich fast nichtß mehr sah; ich fragte
den Wirt, wo die Richter an dem Tage Sitzung haben würden;
er sagte mir, in einer etwa acht Meilen entfernten Stadt. Nun
fing ich wieder an zu sehen und lief dorthin, so schnell ich konnte;
altz ich zu dem Hauö kam, in dem sie und ihre zahlreiche Diener-
schaft waren, mahnte ich die Richter, die Dienstboten nicht am


