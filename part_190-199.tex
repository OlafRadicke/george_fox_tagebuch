% \picinclude{./190-199/p_s190.jpg} 
190 Kapitel Zyl.
in London eingerichtet war, .... ging ich nach Essex. Nachdem
die Monatöoersammlungen hier eingerichtet waren, ging ich nach
Suffolk und Norfolk ..... Alö auch hier die Monatßoersamm-
lungen eingerichtet waren, ging ich nach Huntingdonshire, wo sie
ebenfallß eingerichtet wurden. .... Gbenso in Bedfordshire,
und Nottinghamshire, .... Leieestershire, .... Warwickshire,
. . . . Jn Staffordshire hatten wir eine allgemeine Männeroer-
sammlung und richteten dort ebenfalls eine allgemeine Monatö-
oersammlung ein ..... In Eheshire hatten wir ebenfallö eine
allgemeine Männeroersammlung, in der die Monat;-versammlung
für diese Grafschaft eingerichtet wurde ..... Auch in Laneashire
wurden die Monate-versammlungen für diese Grafschaft einge-
richtet, nach dem Evangelium ..... Von hier aus sandte ich
Schreiben nach Wesimorland, Durham, Cleveland, Northumber-
land, Eumberland und Schottland, um die Freunde zu ermahnen,
die Monat?-Versammlungen an diesen Orten einzurichten, maß sie
auch taten. So kam die Kraft des Herrn über alle, und ihre
Erben nahmen von ihr Besitz. Denn unsere Versammlungen
sind von der Kraft Gotteß eingesetzt nach dem Evangelium, das
Leben und unvergänglicheß Wesen anö Licht bringt (2. Tim. 1,10),
damit alle, die der Teufel in Finsternitz gebracht hat, wieder
sehend werden, und alle, die Erben des Evangeliumö sind, auch in
diesem Evangelium wandeln, und Gott preisen mit Seele, Leib und
Geist, welche sind Gotteß. Denn die Ordnungen deö herrlichen
Evangeliums sind nicht von Menschen gemacht .....
Durch Denbigshire und Montgomeryshire kamen wir nach
Merionetshire (Wales). Nachdem wir hier die Monatßversammlungen
eingerichtet, verließen wir Waleß und kehrten nach Shropshire
zurück ..... Dann gingen wir nach Woreestershire, wo wir
eine allgemeine Männewersammlung hatten, in Pafhur, wo eben-
fallö die Monatßversammlungen eingerichtet wurden .....
Ju Herefordshire hatten wir mehrere gesegnete Zusammen-
künfte. Auch hielten wir eine allgemeine Männewersammlung,
in der alle Monat?-Versammlungen festgesetzt wurden. ES war
gerade eine Verordnung erschienen gegen daß Abhalten von Ver-
sammlungen. A13 wir nun nach Herefordshire kamen, berichtete
man unö von einer großen Versammlung der dortigen Prezby-
terianer, welche entschlossen waren, alles- eher zu ertragen und
auszugeben, als- von ihren Versammlungen zu lassen. A13 nun


% \picinclude{./190-199/p_s191.jpg} 
Einrichtung der Monatsversammlungen. Regelung der Quäkerehen usw. 191
diese Verordnung bekannt geworden sei, so seien die Leute gitz
kommen, aber der Priester habe sich davon gemacht und habe sie
im Stich gelassen. Daraufhin kamen sie heimlich in Leominster
zusammen, hielten Brot, Käse und Getränke in Bereitschaft, damit,
wenn die Wachen kommen würden, sie ihre Bibeln bei Seite legen
könnten und sich anö Essen machen. Der Gerichts-diener kam
ihnen aber auf die Spur, trat unter sie und sagte: »euer Brot
und Wein hilft euch nicht; gebt eure Redner heran?-.« Sie ant-
worteten: »waZ würde dann autz ihren Frauen und Kindern
werden?« Aber er nahm ihre Redner gefangen und behielt sie
eine Weile. Gr erzählte eö Peter Young und sagte, dieö seien
die ärgsten Heuchler, die je für eine Religion Bekenniniö abzu-
legen suchten.
Ähnlicheß bewerkstelligten sie an andern Orten. Ju London
war einer namentz Pocock, welcher Abigail Darcy heiratete, eine
sogenannte Dame, und da sie eine Bekennerin der Wahrheit war,
so ging ich in sein Haus, um sie zu besuchen. Dieser Pocock
war ein Erz-Preßbyterianer und sehr übel gesinnt gegen unß
und pflegte unsre Leute »HauSkriecher« (boueecreepcr) zu nennen.
A13 er nun einmal fort war, sagte seine Frau zu mir: »Jch muß
dir etwas- über meinen Mann sagen.« »Nein,« sagte ich, »du
sollst nicht über deinen Mann reden.« »Doch,« erwiderte sie,
»in diesem Falle muß ich eß. Am vorigen Ersten Tag hatte
er mit seinen Priestern und Genossen eine Versammlung; sie hatten
Lichter, Tabak?-pfeifen, Brot und Käse und kaltes Fleisch vor sich
aus dem Tisch, und sie hatten sich verabredet, falls die Beamten
sie überraschen sollten, aufzuhören mit Predigen und Beten und
sich uns- Gssen zu machen.« Alß ich ihn wieder sah, sagte ich:
»Jhr, die ihr unß verfolgt und gefangen genommen habt und
unsrer Habe beraubt, weil wir unö eurer Religion nicht an-
schließen wollten, und unß Kriecher nanntet, ihr schämt euch nicht,
daß ihr nun nicht einmal zu eurer Religion sieht? Habt ihr je
gesehen, daß wir uns- bei unsern Versammlungen mit Brot und
Käse versahen? oder habt ihr irgendwo in der Schrift gelesen,
daß die Heiligen dergleichen taten?« ,,Ei,« sagte der Alte, ,,wir
sollen ja klug sein wie die Schlangentts Jch erwiderte: ,,DieZ
ist allerdingß Schlangenklugheit! Wer hätte aber gedacht, daß
ihr Preöbyterianer und Jndependenten, nachdem ihr solche, die
sich eurem Glauben nicht anschließen wollten, Versolgtet, gefangen-


% \picinclude{./190-199/p_s192.jpg} 
192 Kapitel JT71.
nahmt, peitschtet und beraubtet, nun selber zurückweichi und nicht
wagt, zu eurem Glauben zu stehen, sondern denselben mit Hilfe
von Tabak?-pfeifen, Flaschen, Brot und Käse zu verbergen sucht?«
Aber ich vernahm später, daß solche Heucheleien nur allzuhäufig
betrieben wurden in den Zeiten der Verfolgung.
A16 wir in Heresordshire alleß die Versammlung Betreffende
geordnet hatten, gingen wir nach Monmouthshire, wo wir mehrere
gesegnete Versammlungen hatten, und bei Walter Jenkinö, einem
früheren Friedenßrichter, hatten wir eine große Zusammenkunft
und etz wurden mehrere gewonnen. GZ war eine rtchige Versamm-
lung; in einer früheren hingegen war ein halb betrunkener Gerichtß-
diener erschienen und hatte behauptet, er müsse die Redner ab-
fassen; aber die Kraft Gotteß war so mächtig gewesen in jener
Versammlung, daß sie ihn trotz setneß Wütenö bannte und er sich
ihr nicht entziehen konnte. A16 die Versammlung aus war, war
ich noch ein wenig geblieben und er ebenfallö, ich redete ein
wenig mit ihm und ging dann ruhig weg. In der Nacht kamen
ein paar und schossen mit einer Flinte gegen daß Hauö, verletzten
aber niemand. So kam die Kraft dez Herrn über alle und
band die widerspenstigen Geister, so daß wir »keinen Schaden
nahmen.
Nun gingen wir nach Gloeestershire, und hatten dort viele
gesegnete Versammlungen in der ganzen Grafschaft herum, und
zuletzt gingen wir weiter nach Bristol, wo nach einer sehr erspriesz-
lichen Zeit die Männer und Frauenoersammlungen ebenfalltz ein-
gerichtet wurden.
Einmal alz ich in Bristol in meinem Bett war, geschah das
Wort dez Herrn zu mtr, ich solle wieder nach London zurückgehen.
Am folgenden Morgen kam Alexander Parker und einige andere
zu mir. Jch fragte sie, waß ihnen sei? und ebenso fragten sie
mich, waö mir sei? Jch sagte ihnen, ich fühle, daß ich nach
London zurückkehren müsse. Sie sagten, gerade so sei ez auch
ihnen. So ergaben wir unß drein, nach London zu gehen; denn
welchen Weg auch der Herr unß führte, wir gingen ihn in seiner
Kraft. Wir gingen über Wiltshire und ordneten dort die Monatß-
oersammlung für Männer, in der Kraft deß Herrn, und besuchten
die Freunde, bis wir nach London kamen.
Nachdem wir die Freunde in der Stadt besucht hatten, trieb
es mich, sie zu ermahnen, alle ihre Gheschließungen vor die Ver-


% \picinclude{./190-199/p_s193.jpg} 
Einrichtung der Monatöversammlungen. Regelung der Quäierehen usw. 193
sammlungen der Männer und Frauen zu bringen, um sie den
Gläubigen vorzulegen. Diese Vorsorge möge getroffen werden,
um Unordnungen zu verhüten, wie solche von etlichen begangen
worden waren. Denn viele hatten sich gegen den Willen der
Jhrigen Verheiratet, und einige junge Leute, die sich zu unß hielten,
hatten sich mit solchen, die der Welt angehörten, verbunden;
Witwen hatten sich wieder verheiratet, ohne Fürsorge zu treffen
für ihre Kinder, trotz meiner Schrift über daß Heiraten, die ich
im Jahre 1653 veröffentlicht hatte, alß die Wahrheit noch wenig
verbreitet war. Jch hatte darin die Freunde, sür die etz in
Betracht kam, ermahnt, die Sache doch ja immer den Gläubigen
vorzulegen, ehe sie etwaö abmachten, und erst darnach bekannt
zu machen, auf dem Markte oder in der Versammlung, je nach-
dem e3 sie triebe. Und wenn dann alleß ins'- Reine gebracht
worden sei, wenn sie frei seien von jeder anderweitigen Verpflich-
tung, und ihre Angehörigen einverstanden, so sollten sie eine Ver-
sammlung bestimmen, in der sie sich dann, in Gegenwart von
mindeftenß zwölfZeugen, zur Ehe nehmen. Da nun diese Vorschriften
nicht befolgt wurden, und die Wahrheit sich weiter im Lande
auögebreitet hatte, so wurde in der Kraft und dem Geist dez
Herrn verordnet, daß die Gheschließungen den vierteljährlichen
und den monatlichen Versammlungen der Männer vorgelegt werden
sollten. Die Freunde sollten dafür sorgen, daß die Angehörigen
beider Teile einverstanden seien, und daß die Witwen Bestimmungen
getroffen haben siir die Kinder aus erster Ehe, ehe sie wieder
heiraten, und wa-Z e3 sonst noch zu ordnen gibt, damit alleß ge-
schehe in Reinheit und Gerechtigkeit, zur Ehre Gotteö. Später
wurde verordnet durch die Weißheit Gottes, daß, wenn der eine
Teil auö einer andern Gegend oder auö einem andern Land komme
oder einer anderen Monatßversammlung zugehörte, so solle er eine
Bescheinigung bringen von der Versammlung, der er zugehörte,
alß Gewähr bei der Monatßoersammlung, der sie ihre Absicht,
sich zu heiraten, vorlegen.
Nachdem diese Angelegenheit, sowie viele andere Dienste für
Gott, in Ordnung gebracht und geregelt waren in den verschie-
denen Stadtgemeinden, verließ ich London und ging, wie mich
die Kraft dez Herrn führte, nach Hertsordshire. Nachdem ich
viele Freunde dort besucht hatte, und die Monatöversammlungen
für Männer dort geordnet waren, hatte ich eine große Versamm-
George Fox. 13


% \picinclude{./190-199/p_s194.jpg} 
194 Kapitel IN.
lung in Baldock mit allen möglichen Leuten. Darauf kehrte ich
nach London zurück über Waltham, wo ich ihnen riet, eine Schule
für Knaben einzurichten, sowie auch in Shacklewell eine Schule
für Mädchen, um sie in allem Guten imd Nützlichen zu unter-
richten .....
Wir zogen durch Gloueestershire und besuchten die Freunde,
dann kamen wir nach Monmouthshire, wo wir mit Vertretern
aller Versammlungen des ganzen Bezirks zusammentrafen und in
der Kraft des Herrn auch hier die Monatsoersammlungen ein-
richteten, damit alle die Herrlichkeit Gottes feiern möchten und
die, welche nicht nach dem Evangelium wandeln, ermahnt und
zurechtgewiesen würden. Und wirklich bewirkten diese Versamm-
lungen eine große Besserung unter den Leuten, so daß die Obrig-
keit ihren Nutzen einsah.
Wir kamen an einen Ort in der Nähe von Minehead, wo
wir eine große allgemeine Männerversammlung sür alle Freunde
von Somersetshire hatten. Gs war auch einer dabei, ein
Schwindler, von dem einige gutmütige Leute gemeint hatten, ich
solle ihn bleibend zu mir nehmen; aber ich sah, daß er ein
Schwindler war, und hieß sie darum, ihn zu mir bringen, damit
ich sehe, ob er mir ins Gesicht sehen könne. Gr konnte es
nicht, sondern blickte unruhig hin und her. Er hatte einen
Priester betrogen, indem er ihn glauben machte, er sei ein Pre-
diger, hatte sich sein Priesterkleid verschafft und sich in demselben
davon gemacht.
Nach der Versammlung gingen wir weiter nach Minehead,
wo wir rasteten. In der Nacht mußte ich ringen mit einem
Geist der Finsternis, der sich gegen die Kirche Christi erheben
wollte, um sie in Verwirrung zu bringen. Am folgenden Morgen
trieb es mich, einige Zeilen an die Freunde zu schreiben, um sie
zu warnen.
,,Liebe Freunde,
Lebet in der Kraft Gottes des Herrn, und in seinem Samen,
der größer ist als alle Versuchungen, die der Geist der Finsternis
euch anhaben kann, welcher euch ihm untertan machen und sich
unter euch erheben möchte; er ist noch nicht gekommen, aber in
der Kraft Gottes und seines Samens haltet euch über demselben
und verdammet ihn. Denn ich fühlte einen Geist der Finsternis in
der vergangenen Nacht, der suchte sich zu erheben und unter euch


% \picinclude{./190-199/p_s195.jpg} 
Einrichtung der Monatsversammlungen. Regelung der O-uittereben usw, 195
aufzustehen; aber ihr könnet ihn bezwingen mit Gotteö Kraft und
sein Treiben oerdammen, ehe er irgendwo Eingang gefunden
hat. Mehr will ich nicht sagen; meine Liebe im Samen Gottes,
in welchem kein Wechsel ist.«
Minehead in Somersetshire, 22. des 4. Monatß, 1668.
G. F. . .
Nachdem wir die meisten Versammlungen in Somersetshire
besucht hatten, gingen wir weiter nach Dorsetshire zu Georg Harriz,
in dessen Hauß wir eine große Versammlung für Männer hatten.
Hier wurden nun alle Monats-Versammlungen für Männer für
den ganzen Bezirk geordnet nach den herrlichen Geboten dez
Eoangeliumß, auf daß alle möchten in der Kraft Gottez ,,daZ
Verlorene suchen und daß Verirrte wieder holen« (Hes. 34, 4),
daz Gute ehren und daß Böse strasen.
Hierauf kamen wir nach Southampton, wo wir am Ersten
Tage eine große Versammlung hatten. Von da gingen wir zu
Hauptmann Reeoe?-, wo die allgemeine Versammlung für Männer
für Hampshire stattfand; eS waren viele au-3 der ganzen Graf-
schaft gekommen, und wir hatten eine gesegnete Zeit. Die Monatö-
oersammlungen der Männer für diese Grafschaft wurden geordnet
nach den Vorschriften des Evangeliumß, welches Leben und
unsterblicheö Wesen in ihnen anß Licht gebracht hatte. Da erschien
eine Bande Ranter, die unsre Versammlung recht störten und
sich derselben widersetzten.
ES war eine Frau dabei, die bei einem Mann gelegen hatte;
dieser erzählte es nun auf dem Marktplatz und rühmte sich
seiner Schlechtigkeit; eine Anzahl dieser liederlichen Leute wohnte
zusammen in einem Haus, ganz nahe bei dem Ort, wo wir unsere
Versammlungen hatten. Jch ging zu ihnen und hielt ihnen ihre
Schlechtigkeit vor. Der Herr deö Hauseö sagte: nun! warum mich
denn das- so sehr erstaune? Ein andrer sagte, eö werde mich
wohl auch straucheln machen! Jch erwiderte ihnen, ihre Schlech-
tigkeit werde mich nicht zum Straucheln bringen, denn ich stehe
über derselben. Und der Herr trieb mich, ihnen zu sagen, daß
die Strafen und daß Gericht Gotteß über sie kommen werden.
Sie zogen später im Land herum, bis sie schließlich inö Gefängniß
in Winchester geworfen wurden, wo der Mann, der bei der Frau
gelegen hatte, nach dem Kerkermeister stach, ihn jedoch nicht tötete.
Als sie dann auö dem Gefängnis entlassen waren, erhängte sich
13*


% \picinclude{./190-199/p_s196.jpg} 
196 Kapitel 171.
der Mann, der den Kerkermeister erstechen wollte; die Frau hätte
auch fast einem Kinde den Hals abgeschnitten, wie wir hörten.
Diese Leute hatten früher in der Nähe von London gelebt, und ala
die Stadt brannte, prophezeiten sie, daß daß ganze übrige London
innerhalb vierzehn Tagen verbrennen würde und flohen aus der
Stadt. Diese Ranter nun, große Gegner der Freunde, und Störer
unsrer Versammlungen, wurden zuweilen in der Gegend, wo die
Leute sie nicht kannten, für Quäker gehalten. Darum trieb mich
der Herr, ein Schreiben zu verfassen, das unter den Behörden
und dem Volk in Hampshire verbreitet werden sollte, damit man
sehe, daß die Wahrheit und die Freunde mit diesen liederlichen
Leuten nichtö zu tun haben .....
So waren nun im ganzen Lande die monatlichen Männer-
versammlungen geordnet. Denn in Berkshire war ich früher
gewesen, damalß-, als- die meisten der ersten Freunde im
Gefängnis waren; ich hatte ihnen den Nutzen dieser Monatsver-
sammlungen außeinandergesetzt, und sie hatten sie daraufhin auch
eingerichtet. Auch nach Jrland und Schottland, nach Holland,
Barbadoeß und mehrere Orte in Amerika, sandte ich durch zuver-
lässige Freunde Schreiben, um die Freunde zu ermahnen, ihre
monatlichen Männeroersammlungen überall zu ordnen. Viertel-
jährliche Versammlungen hatten sie schon vorher gehabt; aber
jetzt, da die Wahrheit sich unter ihnen verbreitet hatte, sollten sie
auch Monatßversammlungen einrichten in der Krafi Gotteß, durch
die sie bekehrt worden waren. Seit diese Versammlungen einge-
richtet worden sind, und die Getreuen dez Herrn, die Erben deö
Gvangeliumö sich versammeln, in der Kraft ihreß Meister?-, aus
die sich diese Versammlungen gründen, haben viele ihren Mund
aufgetan in Dank und Lobpreisung, und viele haben dem Herm
mit Tränen gedankt, daß er mich in seinem Dienst außgesandt
hatte. Alle, denen Gotteß Ehre und Herrlichkeit am Herzen liegt,
alle, denen eK-’ ein Anliegen ist, daß sein Name, den sie bekennen,
nicht gelästert werde und daß, wer die Wahrheit bekennt, auch
in der Wahrheit, Gerechtigkeit und Heiligkeit wandelt, können nun
das Reich Christi, dessen Wach-k-tum kein Ende hat, kennen und
sehen, besitzen und daran teil haben. Der ewige Ruhm und Preis
Gottes ist in jedem Herzen, daß treu ist, eingepflanzt; wir
dürfen sagen, daß die Ordnung dee Eoangeliumß unter unö
nicht von Menschen, noch durch Menschen, sondern von und


% \picinclude{./190-199/p_s197.jpg} 
Einrichtung der Monatöversamnüungen. Regelung der Quükerehen usw. 197
durch Jesuß Christus; und durch den heiligen Geist ausgerichtet
wurde .....
Nach London zurückgekehrt, blieb ich einige Zeit dort, um die
Freunde in der Stadt und der Umgegend zu besuchen. Einmal
ging ich zu Cßquire Marsh, der mir und den Freunden viel
Freundlichkeit erwiesen hatte; es- traf sich, daß er gerade am
Mittagessen war, alß ich kam. Kaum hatte er meinen Namen
gehört, so ließ er mich herauf holen und wollte, daß ich mich mit
ihm zu Tisch setze; aber ich hatte nicht die Freiheit, es- zu tun.
GS waren mehrere hochgestellte Personen mit ihm bei Tisch, und
er sagte zu einem von ihnen, einem angesehenen Papisten: ,,Hier
ist ein Quäker, den ihr noch nie gesehen habt.« TDer Papist fragte
mich, ob ich die Kindertause anerkenne? Jch erwiderte ihm, es
stehe nichtß in der Bibel davon. z,,Wie,« sagte er, ,,nichtß über
die Kindertaufe?« Jch sagte: »nein.« Jch sagte ihm: ,,Wir
anerkennen die Eine Taufe durch den Einen Geist in dem Einen
Leib (Cor. 12); jedoch dafür, daß man ein wenig Wasser einem
Kinde überßz Gesicht schiittet und sagt, daß sei nun daß Kind taufen
und zu einem Christen machen, gibt eß kein Bibelwort.«, Ferner
fragte er mich, ob ich den katholischen Glauben anerkenne? Jch
antwortete ,,ja,« fügte aber hinzu, ,,weder der Papst noch die
Papisten haben den katholischen Glauben; denn der wahre Glaube
wirket in der Liebe und reinigt daß Herz; wenn ihr den Glauben
hättet, welcher den Sieg gibt, und durch den man den Zugang
zu Gott hat, so würdet ihr den Leuten nicht von einem Fegefeuer
nach dem Tode reden.« Jch suchte nun zu beweisen, daß kein
Papst und kein Papist, welcher ein Fegefeuer nach dem Tod
annehme, den wahren Glauben habe; denn der wahre, herrliche,
göttliche Glaube, dessen Anfänger Christus ist, gibt den Sieg über
Teufel und Sünde, die den Menschen von Gott getrennt haben.
-iWenn sie, die Papisten, den wahren Glauben hätten, so würden
sie nicht solche, die einen andern Glauben haben, verfolgen und
mit Foltern, Gefängnissen und Geldbußen ihnen ihren Glauben
aufzwingen. Das- sei nicht die Art der Apostel und ersten Christen
gewesen, die den wahren Glauben Christi besaßen und bezeugten;
sondern die rmgläubigen Juden und Heiden machten ez so.-s »Wenn
Du,« sagte ich, ,,ein Haupt und Führer der Papisten, aufge-
wachsen und erzogen in der Lehre des Papstes, sagst, ee gebe kein
Heil außer in eurer Kirche, so möchte ich gerne wissen, was denn


% \picinclude{./190-199/p_s198.jpg} 
198 Kapitel 171.
in eurer Kirche das Heil bringt?« Er antwortete: ,,Ein gutes
Leben.« ,,Sonst nichts?'' sagte ich. ,,Doch,« sagte er, ,,gute
Werke.'' ,,So, daß bringt eurer Kirche Heil,« sagte ich, ,,ein
gutes Leben und gute Werke! das ist also eure Lehre nnd euer
Grundsatz! Dann wissen weder du, noch der Papst, noch irgend
ein Papist, woher daß Heil kommt.'' Darauf fragte er mich,
woher denn das Heil in unsrer Kirche komme? Jch sagte ihm:
,,nichts anderes, als was in den Tagen der Apostel das Heil der
Kirche war, ist es auch sür unsre Kirche, nämlich, ,,die heilsame
Gnade Gottes, die allen Menschen erschienen ist'' (Tit. 2, 11).
Wie sie einst die Heiligen lehrte, so lehrt sie jetzt uns: ,,zu ver-
leugnen daß ungöttliche Wesen und die weltlichen Lüfte, und gott-
selig, gerecht und züchtig zu leben in der Welt'' (Tit.2, 12). Gs
sind also weder die guten Werke noch ein gutes Leben, die daß
Heil bringen, sondern die Gnade.« ,,Und diese heilsame Gnade
erscheint allen Menschen, sagt ihr?'' rief der Priester. »J—a,«
erwiderte ich. »Das gebe ich euch nicht zul'' rtes er. Ich ant-
wortete: ,,Alle, die es nicht zugeben, sind Sektierer und haben
nicht den allumfassenden Glauben der Apostel.''
Darauf redete er über die Slltutter-Kirche. Ich sagte ihm,
alle die verschiedenen Sekten im Christentum hätten uns vorge-
worfen, wir verließen die Ntutterkirche. Die Papisten warfen
uns den Abfall von der Mutterkirche vor, mit der Behauptung,
Rom sei diese einzige Mutterkirche. Die Bischöflichen besrhuldigten
uns des Absalls vom alten protestantischen Glauben, indem sie
geltend machten, sie hätten die reformierte Mutterkirche. Die
Presbnterianer und Jndependenten schalten uns, daß wir sie ver-
lassen, indem beide behaupteten, sie hätten die wahre reformierte
Mutterkirche. ,,Allein,« sagte ich, »wenn wir irgend einen äußeren
Ort als Mutterkirche anerkennen würden, so wäre es Jerusalem,
wo das Evangelium zuerst verkündet wurde durch Christum selbst
und seine Apostel, wo Christus litt, wo die große Vekehrung
zum Christentum durch Petrus stattfand, der Ort der prophetischen
Zeichen und Wunder, die in Christus ihre Erfüllung hatten, und
wo er seinen Jüngern befahl, ,,zu warten, bis daß sie angetan
würden mit der Kraft aus der Höhe'' (Luc. 24, 49). Wenn irgend ein
äußerer Ort verdiene, eine Mutterkirche genannt zu werden, so
sei es derjenige, an dem die erste große Bekehrumg zum Christen-
tum stattfand. Aber der Apostel sagt, Gal. 4,, 25—27: ,,Das


% \picinclude{./190-199/p_s199.jpg} 
Einrichtung der Monatßversammlungen. Regelung der Quäkerehen nsw. 199
Jerusalem, daß zu dieser Zeit ist, ist dienstbar mit seinen Kindern.
Aber daß Jerusalem, daß droben ist, ist die Freie, die ist unser
aller Mutter. Sei fröhlich, du Unsruchtbare, und juble die du
nicht schwanger bist; denn die Einsame hat mehr Kinder alß
die den Mann hat.« Der Apostel sagt nicht, daß sichtbare
Jerusalem sei die Mutter, obgleich die erste und große Bekehrung
zum Christentum dort stattfand. Und noch weniger berechtigt ist, daß
Rom oder sonst ein Ort oder eine Stadt so bezeichnet werde von
den Kindern deß freien, oberen Jerusalem; auch sind solche nicht
Kinder deß freien, oberen Jerusalem, welche daß sichtbare Jeru-
salem oder Rom oder irgend einen andern Ort oder eine Sekte
ihre Mutter nennen. Und obgleich von entarteten Christen vielen
Orten und Sekten dieser Titel gegeben wurde, so sagen wir
dennoch wie einst der Apostel: ,,Daß Jerusalem, daß droben ist,
daß ist die Freie, die ist unser aller Mutter.« zi Und wir können
kein andereß Jerusalem, noch ein Rom, noch irgend eine Sekte
alß unsre Mutter anerkennen, sondern allein daß Jerusalem, daß
droben ist, die Freie, die Mutter aller derer, die wiedergeboren
sind und wahrhaft an daß Licht glauben und eingepflanzt sind
in Ehristum, den himmlischen Weinstock. Denn alle, welche wieder-
geboren sind auß dem unvergänglichen Samen durch daß Wort
Gotteß, welcheß ewiglich bleibet, nähren sich von der Milch deß
Worteß, an den Brüsten deß Lebenß und wachsen und nehmen
zu durch dieselbe und können keine andere Mutter anerkennen,
alß daß Jerusalem, welcheß droben ist. ,-i»O,« sagte Gßquire
Marsh zu dem Papisten, ,,il)r wisset nicht, waß für ein Mann
der ist; wenn er nur hier und da in die Kirche kommen wollte,
so wäre er ein außgezeichneter Mensch.«
Jch nahm Marsh beiseite, um wegen der Freunde mit ihm
zu reden; er war Friedenßrichter von Middlesex, und da er an den
Hof kam, übertragen ihm die andern Richter die Leitung mancher
Geschästezt Gr sagte mir, daß er in Verlegenheit sei, wie er zu
unterscheiden habe zwischen unß und einigen Dissentern. ,,Denn,«
sagte er, ,,ihr könnt nicht schwören und die Jndependenten, die
Baptisten und die Fifth-Monarchy-Leute sagen ebensallß, sie können
nicht schwören; wie soll ich denn nun zwischen euch und ihnen unter-
scheiden, da ihr allesamt sagt, ihr könnt um deß Gewissenß willen
nicht schwören?« Jch antwortete ihm: ,,Jch will dir zeigen, wie
du unß unterscheiden kannst. Jene, wenigstenß die meisten, der

