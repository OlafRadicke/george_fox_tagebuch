
%%%%%%%%%%%%%%%%%%% Kapitel 15. %%%%%%%%%%%%%%%%%%%%%%%%%%%%%%

\chapter[Ein Gottesgericht]{Ein Gottesgericht}

\begin{center}
\textbf{Ein Gottesgericht. Verhaftung wegen angeblicher Verschwörung
und schreckliche Gefangenschaft in Lancaster und Cearbo. Disput
im Gefängnis mit Baptisten und andern. Fox steht den Brand
von London voraus.}
\end{center}

Wir gingen nach Tenterden und hatten dort eine Versammlung, 
zu der viele Freunde aus der Umgegend kamen. Nach der
Versammlung ging ich ein wenig mit Thomas Briggs\person{Briggs, Thomas} ins Freie,
während man unsre Pferde bereit machte. Als wir uns umwandten, 
sahen wir einen Hauptmann und einen Haufen Soldaten
mit geladenen Gewehren auf uns zukommen; einige von ihnen
hießen uns zu ihrem Hauptmann kommen. Als wir vor ihn
traten, fragte er: "`Welcher ist George Fox?"' "'`Jch"' erwiderte
"`Ich bin es"'. Da trat er auf mich zu und sagte: "`Ich werde
dafür sorgen, das dir nichts geschieht bei diesen Soldaten."' 
Darauf tief er sie und hieß sie, mich festnehmen; auch Thomas
Briggs und unsern Hauswirt nahmen sie fest, aber die Kraft des
Herrn war mächtig über ihnen. Nun kam der Hauptmann
wieder zu mir und sagte, ich müsse mit ihm in die Stadt;
er war ganz höflich mit mir und hieß die Soldaten mit den
andern nachkommen. Ich fragte ihn unterwegs, warum er
eigentlich solches tue, denn es war mir schon lange nichts 
derartiges mehr vorgekommen, und ich ermahnte ihn, seine Mitmenschen,
wenn sie ruhig leben, doch auch in Ruhe zu lassen. Als wir in
die Stadt kamen, brachten sie uns in eine Herberge, die zugleich
das Haus des Kerkermeisters war. Und bald darauf kam der
Bürgermeister und jener Hauptmann und einer seiner Leute, die
auch Friedensrichter waren, und fragten mich, warum ich 
hergekommen sei um Unruhe zu stiften? Ich erwiderte, ich sei nicht
gekommen, um Unruhe zu stiften und habe das auch nicht getan.
Sie sagten, es gebe aber ein Gesetz speziell gegen 
Quäkerversammlungen. Ich antwortete, ich wisse von keinem derartigen

% \picinclude{./170-179/p_s170.jpg} 
Gesetz. Darauf brachten sie die Verordnungen gegen Quäker und
andere. Ich sagte, das gehe ja gegen solche, welche die Untertanen 
des Königs gefährden und Grundsätze haben, welche der
Obrigkeit gefährlich seien; also gehe es nicht gegen uns, denn wir
hätten keine der Obrigkeit gefährlichen Grundsätze und unsere
Versammlungen seien friedliche. Sie behaupteten, ich sei ein
Feind des Königs. Ich antwortete: "`Wir lieben jedermann und
sind niemands Feind; was mich betrifft, so bin ich ins Gefängnis 
zu Derby gebracht worden, weil ich nicht wollte die
Waffen gegen den König nehmen, und nachher bin ich von Oberst
Hacker nach London gebracht worden als ein Mitverschworener
für die Rückkehr König Karls und dort gefangen gewesen, bis
Oliver mir die Freiheit schenkte"'. Sie fragten mich, ob ich während
des Aufstandes gefangen gewesen sei? Ich sagte: "`Ja, ich war
damals gefangen und seither wieder und erhielt die Freiheit auf
des Königs Befehl"'. Ich erklärte ihnen die Verordnung und
machte sie auf die letzte Kundmachung des Königs aufmerksam und
brachte ihnen Beispiele von andern Friedensrichtern und was
das Oberhaus darüber gesagt hatte. Ich redete auch mit
ihnen über ihren Seelenzustand und ermahnte sie, in der Furcht
Gottes zu wandeln und gegen ihre gottesfürchtigen Mitmenschen
mild zu sein und auf Gottes Weisheit zu achten, durch welche alle
Dinge geschaffen seien, damit diese Weisheit ihnen zu teil werde
und sie leite, so das sie in derselben alles zu Gottes Ehre
regieren möchten. Sie verlangten, das wir uns verpflichten
sollten, bei der nächsten Gerichtssitzung zu erscheinen, aber wir 
verweigerten jegliche Verpflichtung auf Grund unserer Unschuld. 
Daraus wollten sie uns versprechen machen, nie mehr hierher zu
kommen, aber wir ließen uns auch darauf nicht ein. Als sie sahen,
das sie nichts erreichten, sagten sie, sie wollten uns zeigen, das
sie gewillt seien, uns höflich zu behandeln; der Bürgermeister habe
nämlich die Güte, uns die Freiheit zu schenken. Ich erwiderte,
ihr höfliches Benehmen bekunde eine anständige Gesinnung, und
so gingen wir von dannen [...]

Joseph Hellen\person{Hellen, Joseph} und G. 
Bewley\person{Bewley, G.} waren im Loo gewesen, um
Blanch Pope\person{Pope, Blanch}, eine Ranterfrau\index{Ranter}, 
zu besuchen, angeblich um sie zu
bekehren; aber ehe sie sie wieder verließen, waren sie so verstrickt
in ihre Ansichten, das sie fast im Begriffe schienen, eher ihre Anhänger 
zu werden, besonders Joseph Hellen. Sie hatte sie unter
% \picinclude{./170-179/p_s171.jpg} 
andrem gefragt: "`Wer machte den Teufel? war es nicht Gott?"'\index{Teufel}
Diese einfältige Frage verblüffte die Beiden so, das sie nicht 
antworten konnten. Sie legten mir nachher die Frage vor, ich 
verneinte sie, "`denn"` sagte ich, "`alles was Gott machte, war gut,
und der Teufel ist nicht gut; er hieß Schlange, ehe er Teufel und
Feind hieß, und danach wurde er Teufel genannt. Später
wurde er Drache genannt, weil er ein Zerstörer war. Der Teufel
blieb nicht in der Wahrheit (Joh. 8,44) und als er die Wahrheit
verließ, wurde er der Teufel. Von den Juden hieß es, als sie
die Wahrheit verließen, sie seien vom Teufel, und man nannte
sie Schlangen (Matth. 23)\bibel{Matth. 23}. Für den Teufel 
gibt es keine Verheißung, 
das er je wieder zur Wahrheit zurückkehren werde, aber
für die Menschen, die von ihm verführt werden, steht die Verheißung, 
das der Samen des Weibes der Schlange den Kopf zertreten 
und ihre Macht zertrümmern werde (1. Mos. 3)\bibel{Mos. 1. 03@1. Mos. 3}. Nachdem
diese Fragen ausführlich zur Beruhigung der Freunde erörtert
worden waren, sahen sich die Beiden, die den Geist der Runtersfrau 
hatten aufkommen lassen, von der Wahrheit gerichtet; der
eine, Joseph Heilen, 
wandte sich ganz von uns ab und die Freunde
erkannten ihn nicht mehr als zu ihnen gehörend; der andre dagegen, 
George Bewley, wurde wieder zurückgewonnen und wurde
später recht brauchbar [...]

Ich hörte von einem Oberst Robinson in Cornwall, einem
bösen Menschen, der bei der Rückkehr des Königs zum 
Friedensrichter gemacht worden war, das er die Freunde grausam 
verfolge und viele von ihnen ins Gefängnis getan habe; als er hörte,
das ihnen durch die Gunst des Kerkermeisters einige kleine 
Freiheiten zugestanden wurden und sie ausgehen durften, um Weib
und Kinder zu sehen, erhob er deswegen beim Gericht eine Anklage 
gegen den Kerkermeister, und dieser musste eine Buße von
20 Pfund bezahlen, und die Freunde wurden einige Zeit sehr knapp
gehalten. Nach der Gerichtssitzung schickte dann Robinson zu
einem benachbarten Friedensrichter und ließ ihm sagen, er solle
ihm helfen, auf diese Fanatiker Jagd zu machen. An dem Tage,
als sie nun ihr Vorhaben ausführen wollten, schickte, er feinen
Knecht mit den Pferden Voraus und ging zu Fuß von seiner
Wohnung nach einer Farm, auf der er seine Kühe und seine
Milchwirtschaft hatte und wo seine Knechte und Mägde gerade
am Melken waren. Als er kam, fragte er nach dem Stier; die
% \picinclude{./170-179/p_s172.jpg} 
Mägde sagten, sie hätten ihn auf dem Felde eingesperrt, weil er
störrig sei bei den Kühen und sie am Melken hindere. Da ging
er ins Feld und begann nach seiner Gewohnheit seinen Stock gegen
den Stier zu schwingen, der Stier schnaubte nach ihm und holte
nach rückwärts aus, dann kehrte er sich und rannte wütend auf
ihn los und bohrte ihm die Hörner in die Seite, nahm ihn auf
die Hörner, schleuderte ihn über sich hinweg und riss ihm die Seite
auf bis zum Bauch, dann wühlte er mit den Hörnern im Boden
und brüllte und leckte seines Herrn Blut aus. Als eine der
Mägde den Herrn schreien hörte, rannte sie ins Feld, packte den
Stier bei den Hörnern und riss ihn von ihrem Meister weg.
Der Stier stieß sie ganz sanft mit seinen Hörnern zur Seite, ohne
ihr weh zu tun, und ließ nicht ab, sein Opfer zu durchstechen und
sein Blut auszulecken. Nun rannte sie davon und holte ein paar
Männer, die in einiger Entfernung arbeiteten, um ihrem Meister
zu helfen. Aber es gelang ihnen erst den Stier weg zubringen, als
sie die Kettenhunde auf ihn hetzten, da rannte er wutschnaubend
davon. Als die Schwester Robinsons hörte, was geschehen, kam
sie heraus und sagte: "`Ach, Bruder, welch schweres Gericht hat
dich betroffen!"' Er antwortete: "`Ja wahrlich ein schweres 
Gericht! lass den Stier töten und sein Fleisch den Armen geben."'
Sie brachten ihn nach Hause, aber er starb bald darauf. Der
Stier war so wild geworden, das sie ihn erschießen mussten, denn
niemand konnte sich ihm nähern, um ihn zu töten. So gibt der
Herr oft Beweise seines gerechten Gerichts über die Verfolger
seines Volkes, auf das man sich fürchte und sich in acht nehme. . .

Ich kam nach Swarthmore\ort{Swarthmore}, wo man mir sagte, Oberst Kirby
habe seine Leute geschickt, um mich festzunehmen. Während der
Nacht, als ich in meinem Bett lag, trieb mich der Herr, am
nächsten Tage nach Kirbyhall zu Oberst 
Kirby\person{Oberst Kirby} zu gehen, fast zwei
Stunden weit, um mit ihm zu reden; ich ging denn auch [...]
und sagte ihm, ich hätte gehört, er wolle etwas von mir, ob er
irgend etwas gegen mich habe? Er sagte vor allen Anwesenden,
das er ein Gentleman sei und darum nichts gegen mich habe,
hingegen solle Mistres Fell keine Versammlungen in ihrem Hause
haben, das sei gegen die Verordnungen. Ich erklärte ihm, diese
Verordnungen treffen nicht uns, sondern die, welche sich 
versammeln, um Komplotte und Verschwörungen zu machen; [...]
die, welche sich bei Margaret Fell versammelten, seien friedliche
% \picinclude{./170-179/p_s173.jpg} 
Leute. Nachdem wir längere Zeit miteinander geredet, gab er
mir die Hand und wiederholte, das er nichts gegen mich habe.
So kehrte ich nach Swarthmore zurück [...] Bald darauf
ging Oberst Kirby nach London in eine Privatsitzung der Richter
in Holkerhall\indexname{Holkerhall}, und dort wurde ein 
Verhaftbefehl gegen mich aufgesetzt [...] Ich hörte davon 
und hätte gut entwischen können, [...]
aber da das Gerücht ging von einer Verschwörung, so fürchtete
ich, sie würden, wenn ich mich davon machte, über die Freunde
herfallen, wenn ich aber bleibe, so würden sie mich nehmen, und
die Freunde könnten sich eher davon machen, und ich blieb also [...]
Am folgenden Tage kam ein Beamter mit Pistole und Schwert.
Ich sagte ihm, ich wisse, warum er komme, und sei dageblieben,
um mich festnehmen zu lassen; [...] ich verlangte, das er mir den
Befehl zeige, aber er weigerte sich. So ging ich mit ihm, und
Margaret Fell\person{Fell, Margaret} begleitete uns nach 
Holkerhall [...] Dort wurde
mir unter anderem der Suprematseid vorgelegt; als ich ihn nicht
schwören wollte, verlangten einige, das ich ins Gefängnis von
Lancaster geschickt werde, andere wollten nur, das ich verspreche
an der Gerichtssitzung zu erscheinen, worauf ich entlassen wurde,
und ich kehrte also wieder mit Margaret Fell nach Swarthmore
zurück.

Am Gerichtstage ging ich wie verabredet war, nach Lancaster [...]
Der alte Richter Rawlinfon, der Vorsitzende, fragte mich, ob ich
um die Verschwörung wisse? Ich sagte, ich habe in Yorkshire
davon gehört. Er fragte mich, ob ich es den Behörden angezeigt? 
Ich erwiderte, ich hätte ja Schriften gegen Verschwörungen
geschrieben [...] Sie legten mir den Supremats- und Huldigungseid 
vor; ich sagte ihnen, das ich nicht schwören könne, weil
Christus und seine Apostel es verboten hätten, und sie hätten ja
schon genugsam erfahren, wie es bei solchen gehe, welche schwören,
ich aber habe noch nie in meinem Leben einen Eid geleistet. Hierauf
fragte mich Rawlinson, ob ich es für gesetzwidrig halte, zu
schwören? Diese Frage stellte er absichtlich, um mich zu fangen;
denn es war eine Verordnung gemacht worden, das alle, die
sagen, es sei gesetzwidrig zu schwören, verbannt oder hart bestraft
würden. Aber weil ich die Falle merkte, vermied ich sie und erklärte 
ihm, das in den Tagen des Gesetzes, bevor Christus gekommen sei, 
das Gesetz den Juden geboten habe, zu schwören
(3. Mos. 19)\bibel{Mos. 3. 19@3. Mos. 19}; Christus aber, der in den 
Tagen des Evangeliums
% \picinclude{./170-179/p_s174.jpg} 
das Gesetz erfüllte, befehle, überhaupt nicht zu schwören 
(Matth. 5)\bibel{Matth. 5},
und der Apostel Jakobus verbiete das Schwören selbst denen, die
Juden waren und das Gesetz Gottes hatten. Nach vielem Hin und 
Herreden riefen sie den Gefangenwärter und verurteilten mich
zum Gefängnis. Ich trug die Schrift bei mir, die ich gegen
Verschwörungen geschrieben hatte, und bat, das man sie vor dem
ganzen Gerichtshofe vorlese oder lesen lasse, aber sie wollten nicht.
Als ich nun solchermaßen eingesperrt war, dafür, das ich mich
geweigert hatte zu schwören, war mir daran gelegen, das sie und
alle Leute wissen möchten, das ich um der Lehre Christi willen
leide und darum, das ich seine Gebote gehalten. Ich hörte später,
das die Richter sagten, sie hätten besondere Befehle vom Oberst
Kirby gehabt, mich zu verfolgen, trotz seinem schönen Benehmen
und seiner anscheinenden Freundlichkeit damals, als er vor allen
Anwesenden erklärt hatte, er habe nichts gegen mich [...]

Ich wurde bis zur Gerichtsverhandlung gefangen gehalten, und
da Richter Turner und Richter Twistden gerade an der Reihe waren,
wurde ich vor Richter Twistden gebracht, am 14. Tage desk Monats, 
den man März nennt, im Jahre 1663\index{Jahr!1663}. A1s ich 
vorgeführt wurde, sagte ich: "`Friede sei mit euch allen"'. Der
Richter sah mich an und fragte: "`Warum kommst du hier vor
Gericht mit dem Hut aus dem Kopf?"' A1s der Kerkermeister mir
ihn hierauf weg nahm, sagte ich: "`Das Hutabnehmen ist doch nicht
eine Ehre, die vor Gott gilt!"' Daraus fragte mich der Richter:
"`Wollet ihr den Huldigungseid leisten, George Fox?"' Ich erwiderte: 
"`Ich habe nie in meinem Leben einen Eid geleistet, noch
mich zu irgend einem Vertrag verpfiichtet"'; darauf fragte er:
"`Wollt ihr schwören oder nicht?"' Ich erwiderte: "`Ich bin ein
Christ, und Christus befiehlt, nicht zu schwören, ebenso der Apostel
Jakobuß, und ob ich Gott oder Menschen gehorchen soll, darüber
urteile du selbst"'. Er sagte: "`Ich frage euch nochmals, ob ihr
schwören wollt oder nicht?"' Ich antwortete abermals: "`Ich bin
weder Türke, noch Jude, noch Heide, sondern ein Christ und
werde mich zum Christentum bekennen"'. Und darauf fragte ich
ihn, ob er nicht wisse, das die Christen der ersten Zeiten unter
den 10 Verfolgungen, sowie auch einige Märtyrer in den Tagen
der Königin Maria sich weigerten zu schwören, weil Christus
und die Apostel es verboten hätten; ferner sagte ich ihm, sie
hätten ja genügsam die Erfahrung gemacht, wie viele zuerst dem
% \picinclude{./170-179/p_s175.jpg} 
König geschworen hatten und nachher gegen ihn; was mich betreffe, 
so habe ich nie in meinem Leben einen Eid geleistet, und
meine Huldigung bestehe nicht im Leisten eines Eides, sondern
darin, das ich Wahrheit und Treue halte,. denn, sagte ich, ich
ehre jedermann, wie vielmehr denn den König. Christus aber,
der große Prophet und der König aller Könige und Heiland der
Welt, der große Richter der ganzen Erde, hat gesagt, das man
nicht schwören soll, soll ich nun Christus oder dir gehorchen?
Denn es geschiehet aus Gewissenszartheit und aus Gehorsam gegen
Christi Gebote, das ich nicht schwöre, und wir haben ja ein
Königswort für zarte Gewissen. Daraus fragte ich den
Richter, ob er den König anerkenne: "`Ja"', sagt er, "`ich 
anerkenne den König"'. "`Warum"', fragte ich, "`befolgst du denn dann
nicht seinen Erlass von Breda\index{Erlass von Breda} und 
seine Versprechen, die er bei
seiner Rückkehr machte, das niemand um der Religion willen 
vefolgt werde, solange er ruhig lebe? Wenn du den König anerkennst, 
warum verfolgst du mich, verhöhnst mich und treibst mich
dazu, einen Eid zu leisten, maß doch Sache des Glaubens ist, und
siehst doch, das weder du noch sonst jemand mich eines unfriedlichen 
Lebens zeihen kann"'. Hierauf wurde er sehr gereizt und
sagte: "`Kerl, wollt ihr schwören!"' Ich sagte darauf, ich sei keiner
seiner "`Kerls"', sondern ein Christ und es stehe einem alten Richter nicht
an, hier zu sitzen und den Gefangenen Spottnamen zu geben,
weder seinen grauen Haaren noch seinem Amt. Darauf sagte er:
"`Ich bin auch ein Christ"'. "`So handle auch christlich"', sagte ich;
"`Kerl"', sagte er, "`willst du mir mit deinen Reden Angst machen?
Aber"', fügte er Verlegen hinzu, "`jetzt brauche ich ja dieses Wort
wieder!"' und er bezwang sich. Ich sagte: "`Ich rede in Liebe so
mit dir, weil eine solche Sprache dir als; Richter nicht ansteht.
Du solltest deinem Gefangenen das Gesetz erklären, wenn er unwissend 
ist und einen verkehrten Weg geht"'. "`Ich rede ebenfalls in Liebe 
mit dir"', sagte er: "`aber"', erwiderte ich, "`die Liebe gebraucht 
keine Spottnamen"'. Darauf erhob er sich und sagte: "`Ich
lasse mich nicht von dir einschüchten, du sprichst so laut, deine
Stimme übertäubt die meinige und alle andern, ich müsste drei
oder vier Ausrufer kommen lassen, um dich zu übertönen, du hast
gute Lungen"'. Ich erwiderte: "`Ich bin hier gefangen um Jesu
Willen, um seinetwillen leide ich und stehe ich heute hier, und
wenn meine Stimme fünfmal so laut wäre, so würde ich sie erheben 
% \picinclude{./170-179/p_s176.jpg} 
und erschallen lassen für Christus, für dessen Sache ich
heute vor dem Richtstuhl stehe im Gehorsam gegen Christus,
welcher gebietet, nicht zu schwören, vor dessen Richtstuhl ihr alle
stehen und Rechenschaft ablegen müsst"'. "`So antworte mir nun
George For"', sagte er, "`ob du den Eid leisten willst oder nicht"'.
Ich erwiderte: "`Ich frage dich nochmals, ob ich Gott oder den
Menschen gehorchen soll? beurteile du das selber. Wenn ich
überhaupt einen Eid leisten wollte, so wäre es dieser; aber ich
leugne überhaupt alle Eide, nicht nur den oder jenen, nach der
Lehre Christi, der seinen Nachfolgern gebot, überhaupt nicht zu
schwören. Wenn nun du oder sonst jemand von euch, oder eure
Prediger oder Priester mir beweisen wollen, das Christus oder
seine Apostel irgend einmal, nachdem sie alles Schwören verboten
hatten, es den Christen wieder geboten, so will ich schwören"'.
Ich sah, das verschiedene Priester zugegen waren, aber nicht ein
einziger wollte reden. "`Nun denn"', sagte der Richter, "`ich bin
ein Diener des Königs und der König hat mich nicht geschickt,
um mit dir zu disputieren, sondern das Gesetz an dir auszuüben;
legt ihm also den Huldigungseid vor"'. "`Wenn du den König
lieb hast"', sagte ich, "`warum hälst du dich nicht an das, was er
sagt? und an seine Erklärung, in der er uns Gewissensfreiheit 
zugesagt hatte? Ich bin ein Mann mit einem zarten Gewissen und
kann aus Gehorsam gegen Christi Gebot nicht schwören"'. "`Wenn
er also nicht schwören will"', sagte der Richter, "`so führet ihn
in den Kerker"'. Ich sagte, es sei um Christi willen, das ich
nicht schwören könne, ihm müsse ich gehorchen; aber der Herr
möge ihnen allen vergeben. So führte mich der Kerkermeister
hinweg, aber ich fühlte, das der Herrn mächtige Kraft über ihnen
allen war [...]

Während ich nun hier im Kerker war, trieb es mich, an
Richter Flemming, einen der heftigsten Verfolger der Freunde,
folgendermaßen zu schreiben: 


\brief{Richter Flamming}{
  O, Richter Flemming! 
  \medskip 

  Barmherzigkeit, 
  Milde und Güte zieret die Menschen und auch die Behörden.
  O, hörest du nicht das Schreien derer, die durch die Verfolgungen
  Witwen und Waisen geworden sind? Sind sie nicht wie
  Schafe von Konstabler zu Konstabler getrieben worden, wie wenn
  sie die größten Übeltäter und Bösewichter im Lande wären? Es
  betrübt die Herzen vieler einsichtiger Leute, zu sehen, wie man
  ihre ehrlichen Mitmenschen, die ein friedsames, stilles Leben geführt,
  % \picinclude{./170-179/p_s177.jpg} 
  behandelt hat. Wieder ist einer gestorben, den ihr ins Gefängnis 
  geworfen; er hat fünf Kinder hinterlassen, die nun verwaist 
  sind. Solltest du nun nicht für diese vaterlosen Kinder
  sorgen, sowie auch für die Weiber und Hinterlassenen der andern?
  Ist es nicht deine Pflicht? Denke an Hiob, Kap. 29\bibel{Hiob. 29}:
  "`Er war ein Vater der Armen; er errettete den Armen, der da schrie und
  die Waisen, die keinen Helfer hatten, er brach die Kinnbacken des
  Ungerechten und riss den Raub aus seinen Zähnen."' Und nun
  vergleiche dein Leben mit dem seinen und hüte dich vor dem Tage
  des Gerichts, welcher kommen wird, und vor dem Urteil Christi,
  wenn ein jeder muss Rechenschaft ablegen und den Lohn empfangen
  für seine Taten. Als dann wird es heißen; O, wo sind die verlorenen 
  Tage! — Als John Stubbs\person{Stubbs, John} vor dich gebracht wurde, der
  ein Weib und vier kleine Kinder hatte, und mit seiner Hände
  Arbeit nur den dürftigsten Lebensunterhalt verdiente, da riesest
  du: fordert diesem Menschen den Eid ab! und als er dir vorstellte, 
  dass er ein armer Mann sei, ließest du kein Mitleid aufkommen 
  und wolltest ihn nicht hören, und nun ist er im Gefängnis, 
  weil er nicht schwören konnte, also nicht das Gebot
  Christi und der Apostel übertreten konnte. Hoffentlich wirst du für
  seine Familie sorgen, damit seine Kinder nicht Hungers sterben.
  Ist denn das dem König gehuldigt, wenn man tut, wovon Christus
  und die Apostel sagen, es sei Unrecht und führe in die Verdammnis? 
  Ihr würdet wohl auch Christus und die Apostel, welche
  das Schwören verboten, ins Gefängnis geworfen haben, wenn sie
  zu eurer Zeit gelebt hätten.

  Denke auch an deinen armen Mitmenschen William Wilson\person{Wilson, William},
  der allgemein als ein fleißiger Mann bekannt war, und der sein
  Weib und seine Kinder ehrlich durchbrachte, obgleich er nichts besaß, 
  als was er durch seiner Hände Arbeit erwarb. Sogar auf
  den Märkten wird über den Tod dieser Beiden geredet; man hört
  das Schreien derer, die um der Gerechtigkeit willen Witwen und
  Waisen geworden sind. Wenn John Stubbs und William Wilson 
  geschworen hätten, so hätten sie damit ihre Freiheit wieder
  erlangt, wenn sie auch daneben es mit den Marktschreiern und
  Schnurranten gehalten hätten. O gehet in euch! es ist solches
  nicht nach des Herrn Sinn. Und auch der König hat erklärt, es
  solle gegen keinen seiner Untertanen, der friedlich lebe, eine 
  Grausamkeit ausgeübt werden. Sodann sind einigen sehr rechtschaffenen
  % \picinclude{./170-179/p_s178.jpg} 
  Leuten Bußen auferlegt worden, obgleich sie selber nichts hatten,
  und es eher am Platze gewesen wäre, ihnen etwas zu geben, als
  ihnen noch etwas zu nehmen. Weil du weißt, das sie um ihrer zarten
  Gewissen willen keinen Eid schwören können, so stellst du ihnen damit 
  eine Falle. Wie denkst du, das das Volk über ein derartiges
  Tun redet? Sie sagen: Wir wissen, das die Quäker sich an ihr
  \textit{ja} und \textit{nein} halten, andere dagegen sehen 
  wir schwören und wieder abschwören! 
  Ich weise dich an den Geist Gottes in deinem Gewissen, 
  Richter Fleming, der du so eifrig die Gefangennahme
  des George Fox betriebest und so böse warst über die, die ihn
  nicht gefangen nahmen. Wo ist dein Erbarmen mit den armen,
  verwaisten Kindern? Hüte dich vor der Grausamkeit des Herodes\person{Herodes},
  der kein Mitleid kannte; Esau hat es also gemacht und nicht
  Jakob! Thomas Walters\person{Walters, Thomas} 
  von Bolton ist auch hier im Gefängnis 
  und wird darin festgehalten, weil er sich nach Christi Gebot
  weigert zu schwören, und dabei hat er fünf kleine Kinder und seine
  Frau ist ihrer Niederkunft nahe; du solltest dich doch seiner 
  annehmen und dafür sorgen, das seine Frau und seine Kinder nicht
  Mangel leiden, da sie durch deine Schuld verwaist dastehen.
  Klingt dir das Schreien der Verwaisten nicht in den Ohren, und
  siehest du das Blut derer, die durch dich umgekommen sind, nicht
  vor dir? Es wird dich am Tage des Gerichts ein schweres Urteil 
  treffen, wie willst du dich verantworten, wenn du nach deinen
  Werken gerichtet werden wirst und vor den Richterstuhl des
  Allmächtigen treten musst? [...] Aber trotz alledem sagen wir
  Quäker: der Herr vergebe dir und rechne dir diese Dinge nicht
  an, wenn es sein heiliger Wille ist.
  \medskip 
  \begin{flushright}G. F.\end{flushright}
}

Bald danach starb Richter Flemings Weib, und hinterließ
ihm dreizehn oder vierzehn mutterlose Kinder [...]
Einige Zeit vorher war Margaret Fell\person{Fell, Margaret} auch von Richter
Fleming als Gefangene nach Lancaster geschickt worden, und als
sie, an der Gerichtssitzung, den Eid nicht schwören wollte, wurde
sie weiter zum Gefängnis verurteilt. [...]
Während ich im Gefängnis zu Lancaster war, hieß es, der
Türke werde über die Christenheit herfallen, und viele kamen in
große Angst. Eines Tages, als ich in meiner Zelle auf und
nieder ging, kam es über mich vom Herrn, das ich sah, wie die
Kraft des Herrn sich gegen den Türken kehrte, so das er wieder
umkehren musste, und ich teilte einigen mit, was der Herr mich
% \picinclude{./170-179/p_s179.jpg} 
hatte sehen lassen, und binnen eines Monats kam die Nachricht,
das er geschlagen worden war!\footnote{1664 Sieg der 
Abendländer (Deutschland und Frankreich) über die
Türken bei der Abtei St. Gotthardt.}\index{Jahr!1664}\ort{Abtei St. Gotthardt}

Ein andermal als ich in meiner Zelle auf und niederging
und zum Herrn aufschaute, sah ich den Engel des Herrn, wie er
mit einem leuchtenden Schwert gen Süden wies, und das ganze
Schloss schien in Feuer zu stehen. Nicht lange darauf brach der
Krieg in Holland aus,\footnote{1665 Krieg zwischen England 
und den Niederlanden.}\index{Jahr!1665} und dann eine große Seuche und dann das
Feuer in London;\footnote{Oktober 1665 die große Pest in London, 
September 1666 der große Brand in London.} da war wahrlich 
das Schwert des Herrn gezogen.

Durch die lange Gefangenschaft an diesem ungesunden Orte
war ich sehr angegriffen in meiner Gesundheit, aber die Kraft
des Herrn war stärker als alles, sie half mir hindurch und hielt
mich aufrecht und half mir für den Herrn wirken, so viel der
Ort es erlaubte. Ich antwortete denn auch während dieser Zeit
auf mehrere Bücher, wie: "`die Messe"', "`das Common Prayer
Buch"', "`daß Directory"', "`das Kirchenbekenntnis"', welches die
vier mächtigsten Religionen\footnote{Römische, Bischöfliche, 
Preßbyterianer und Independenten.} sind, die sich seit den Tagen der
Apostel erhoben.

Nach der Gerichtsverhandlung war es einigen der Richter
etwas ungemütlich, das ich in Lancaster war, denn ich hatte sie
bei den Verhandlungen tüchtig geärgert, und sie bemühten sich
sehr darum, das man mich anderswohin bringe [...] Etwa sechs
Wochen nach der Gerichtsverhandlung erhielten sie denn auch den
Befehl vom König und dem Rat, mich von Lancaster fortzubringen,
und zugleich kam ein Brief vom Earl von 
Anglesea\person{Earl von Anglesea}, worin es
hieß, das, wenn alles dessen man mich beschuldigt hatte, wahr
sei, so verdiene ich keinerlei Nachsicht noch Milde. Und doch war
das Ärgste, was sie gegen mich vorgebracht hatten das, das ich
einem Gebot Christi nicht ungehorsam sein konnte [...]

Sie brachten mich nun nach Schloss Scarbro, wo sie mich in
ein Gemach führten und mir einen zur Wache setzten. Da ich seht
schwach war und öfters ohnmächtig wurde, so ließen sie mich
manchmal mit der Wache an die frische Luft gehen; nach einiger
Zeit brachten sie mich in ein anderes Eelaß\footnote{Das Wort 
ist i Original nicht zu entziffern}, das offen war, so

% \picinclude{./180-189/p_s180.jpg} 
das es herein regnete, und wo es schrecklich rauchte, was mir
sehr schadete. Eines Tages besuchte mich der Gouverneur Sir
John Crossland\person{Crossland, Sir John} mit 
Sir Francis Cobb\person{Cobb, Sir Francis}. Ich 
bat den Gouverneur, mich in mein Zimmer zu begleiten, 
um zu sehen, was das für
ein Ort sei. Ich hatte ein kleines Feuer darin angezündet, welches
nun derart rauchte, das man seinen Weg schier nicht fand. Da
der Gouverneur ein Papist war, so sagte ich ihm, es sei sein
Fegefeuer, das sie mir zum Aufenthalt gegeben hätten. Ich musste
etwa 50 Schilling ausgeben\footnote{Die Gefangenen hatten 
zu der Zeit die Kosten ihres Aufenthaltes in
den Gefängnissen selbst zu tragen (s. Aschrott, Engl. 
Gefängniswesen).}, um den Regen abzuhalten und zu
machen, das es nicht so stark rauchte. Und als ich diese 
ausgaben gemacht hatte, und es etwas erträglicher geworden, gaben
sie mir ein noch schlechteres Gelaß, wo ich weder ein Kamin noch
irgend eine andere Vorrichtung, um Feuer zu machen, hatte. Da
es gegen die See gelegen und sehr offen war, so trieb der Wind
den Regen ungehindert herein, so das dass Wasser bis zu meinem
Bett kam und im Zimmer herumlief, und ich es mit einem Gefäß 
ausschöpfen musste. Und wenn meine Kleider nass waren, so
hatte ich kein Feuer, um sie zu trocknen, so das mein Körper ganz
erstarrt war vor Kälte, und meine Finger so geschwollen waren,
das einer so groß war wie sonst zwei. Obgleich ich in diesem
Raum auch zu bezahlen hatte, so gelang es mir doch nicht, Wind
und Regen abzuhalten [...]

Es wurde den Freunden nicht gestattet, mich zu besuchen;
aber sonst führten sie hier und da jemanden zu mir, entweder
um mich anzusehen, oder um sich mit mir zu unterreden. Einmal
kam eine Schar Papisten, um mit mir zu disputieren; sie 
behaupteten, der Papst sei unfehlbar\index{Unfehlbarkeit} 
und sei immer unfehlbar gewesen
seit Petrus Zeit, aber ich bewies ihnen das Gegenteil aus der
Geschichte: ein Bischof von Rom, Marcellinuß mit Namen, habe
den Glauben abgeschworen und den Götzenbildern gehuldigt, dieser
sei also nicht unfehlbar gewesen. Ich sagte ihnen, wenn sie den
unfehlbaren Geist hätten, so bedürften sie keiner Kerker, Schwerter,
Foltern, Scheiterhaufen, Geißeln und Galgen, um ihre Religion
aufrecht zu erhalten, denn wenn sie den unfehlbaren Geist hätten,
so würden sie die Leben der Menschen schützen, statt sie umzubringen, 
und würden in Sachen der Religion nur geistliche Waffen\index{Geistliche Waffen}
% \picinclude{./180-189/p_s181.jpg} 
brauchen. Ich erzählte ihnen auch, was einer der Ihrigen mir
berichtet hatte: eine in Kent lebende Frau war nicht nur selber
Papistin gewesen, sondern hatte auch viele andere für ihren
Glauben gewonnen. Aber als sie zur Wahrheit des Herrn bekehrt 
wurde und durch sie zu Jesus Christus ihrem Heiland kam,
ermahnte sie die Papisten, ein gleiches zu tun, unter anderem
auch einen Schneider, der bei ihr in Arbeit war; sie zeigte ihm
die Verkehrtheit der pästlichen Religion und suchte ihn für die
Wahrheit zu gewinnen, da zog er sein Messer und stellte sich
zwischen sie und die Türe, aber sie trat ihm mutig entgegen und
ermahnte ihn, sein Messer weg zu tun, denn sie kannte seine
Grundsätze; auf die Frage, was er wohl mit dem Messer gemacht 
hätte, antwortete die Frau: "`er hätte mich erstochen"', und
auf die weitere Frage, ob er dies wegen ihrer Religion getan
hätte, erwiderte sie: "`ja, denn es ist der Grundsatz der Papisten,
jeden, der von ihrer Religion abtrünnig wird, womöglich zu töten"'.
Dieses erzählte ich nun den Papisten und fügte bei, ich hätte es
von jemand, der früher zu ihnen gehört, sich jedoch von ihnen
gewandt habe, weil er hinter ihre Handlungsweise gekommen war.
Sie leugneten nicht, das sie solche Grundsätze hätten, fragten
aber, ob ich nun solches weitererzählen werde? Ich erwiderte:
"`ja, denn solche Dinge müssen weitererzählt werden, damit man
erfährt, wie sehr eure Religion vom wahren Christentum 
abweicht"'.\index{Öffentliche Kritik an Anderen}
Darauf gingen sie sehr zornig fort. Ein anderer Papist, welcher
kam, um mit mir zu disputieren, behauptete, alle Patriarchen seien
in der Hölle gewesen, bis Christus zur Hölle\index{Hölle} hinabgefahren sei,
da habe der Teufel\index{Teufel} gesagt: "`was kommst du hierher, unsere sichere
Burg zu sprengen?"' und Christus habe geantwortet, er komme,
um alle diese zu befreien, und sei drei Tage und drei Nächte in
der Hölle gewesen, um sie alle zu befreien. Ich erwiderte ihm,
das sei unrichtig, denn Christus habe ja zum Schächer gesagt:
"`heute noch sollst du mit mir im Paradiese sein."' Und Henoch
und Elias seien in den Himmel gekommen, auch Abraham, denn
es heiße, Lazarus sei in Abrahams Schoß gewesen, und Moses
und Elias seien mit Jesus auf dem Berg gewesen, ehe er leiden
musste. Diese Beispiele stopften dem Papisten den Mund und
brachten ihn in Verlegenheit.

Ein andermal kam Doktor Witty\person{Doktor Witty}, ein berühmter Arzt, mit
Lord Falconbridge; mit ihnen kam auch der Gouverneur der
% \picinclude{./180-189/p_s182.jpg} 
Festung Tynemouth\ort{Festung Tynemouth} und mehrere Adlige. Als ich zu 
ihnen gerufen wurde, sing Witty ein Gespräch mit mir an und fragte mich,
warum ich im Gefängnis. sei. Ich antwortete: "`weil ich den Geboten 
Christi nicht ungehorsam sein will"'. Er sagte, ich hätte
dem König den Treueid leisten sollen. Da er ein eifriger 
Preßbyterianer\index{Preßbyterianer} war, so fragte ich ihn, 
ob er denn nicht zuerst gegen
den König und das Unterhaus geschworen und sich zum
schottischen Covenant bekannt habe und seither wieder zum König
geschworen habe? was denn dann das Schwören nütze? Mein
Huldigungseid, fügte ich bei, bestehe eben nicht im Schwören,
sondern in Wahrheit und Treue. Nach einigem Hin- und Herreden 
wurde ich wieder in meine Zelle zurückgeschickt; nachher
prahlte dieser Arzt bei seinen Patienten in der Stadt herum, er
habe mich besiegt. Als ich von seinem Prahlen hörte, sagte ich
dem Gouverneur, es sei ein geringer Ruhm zu sagen, man habe
einen Gefangenen besiegt. Ich bat, man solle ihm sagen seinen
Besuch zu wiederholen, wenn er wieder ins Schloss komme. Er
kam nach einiger Zeit wieder mit sechzehn oder siebzehn 
angesehenen Leuten und erlitt eine noch größere Niederlage als das
erste mal; er behauptete nämlich, Christus habe nicht alle, die in
die Welt kommen, erleuchtet\index{Erleuchtung}, und die heilsame Gnade Gottes sei
nicht allen Menschen erschienen, und Christus sei nicht für alle
Menschen gestorben\index{Tod Christus}. Ich fragte ihn, was das für Menschen
seien, die Christus nicht erleuchtet habe, denen die heilsame Gnade
nicht erschienen sei, und für die er nicht gestorben sei?\index{Opfertod} Er sagte,
die Ehebrecher, die Götzendiener, die Gottlosen. Ich fragte ihn,
ob die Ehebrecher und Gottlosen keine Sünder seien? Er sagte,
doch. "`Und starb nicht Christus eben für die Sünder?"' fragte ich,
"`kam er nicht, die Sünder zur Buße zu rufen?"'\index{Sünde} Er sagte: "`doch"'.
"`Dann hast du dir selber das Maul gestopft"', sagte ich. Hiermit
hatte ich bewiesen, das die Gnade Gottes allen Menschen erschienen 
ist, obgleich viele sie in Mutwillen kehren und ihr widerstreben, 
und das Christus alle Menschen erleuchtet hat, wenn
schon Viele das Licht hassen. Manche der Anwesenden gaben zu,
das dies wahr sei, der Doktor aber ging fort und kam nie mehr
zu mir.

Ein andermal brachte der Gouverneur einen Priester zu
mir, aber sein Mund war bald gestopft. Bald darauf brachte
er zwei Parlamentsmitglieder, die mich fragten, Ob ich Prediger\index{Prediger}
% \picinclude{./180-189/p_s183.jpg} 
und Bischöfe\index{Bischöfe} gelten lasse. Ich erwiderte: "`ja, solche die Christus
sendet, die umsonst empfangen und umsonst geben, die dazu bestimmt 
sind und den Geist und die Kraft haben, welche auch die
Apostel hatten. Solche Bischöfe und Prediger aber wie eure, die
nichts tun, als was; ihnen ein gutes Einkommen bringt, die
lasse ich nicht gelten, denn sie sind nicht den Aposteln gleich.
Christus sagte zu seinen Jüngern: "`Gehet hin in alle Welt und
predigt das Evangelium umsonst."' Ihr Parlamentsmitglieder,
die ihr euren Bischöfen und Predigern so große Pfründen gebt,
ihr habt sie verdorben. Meinet ihr etwa, diese gehen zu allen
Völkern? oder überhaupt über ihre fetten Pfründen hinaus, um
zu predigen? Urteilt selber, ob sie das tun oder nicht."'
Ein andermal kam die Witwe von Lord Fairfax\person{Lord Fairfax} und viele
mit ihr, unter anderem auch ein Priester. Es trieb mich, ihnen
die Wahrheit zu verkünden; der Priester fragte mich, warum wir
"`du"' und "`ihr"'\footnote{In der ursprünglichen Übersetzung 
wird "`dich"' an dieser Stelle geschrieben.} zu den Leuten sagen? 
denn er hielt uns für
Narren und Dummköpfe deswegen. Ich fragte ihn, ob er finde,
die, welche die Schrift übersetzten und die Grammatik und 
Sprachlehre machten, seien Narren und Dummköpfe gewesen, weil sie
sie so übersetzten und lehrten, das "`du"' für eine Person und
"`ihr"' für mehrere gilt? wenn denn diese Narren und Dummköpfe
gewesen seien, warum denn dann nicht er und die, welche seine
Ansicht teilen und sich für weise halten, die Grammatik und 
Sprachlehre und die Bibel verbessern, und die Mehrzahl statt der 
Einzahl setzen? Wenn es aber weise Männer gewesen seien, die die
Bibel übersetzten und die Sprachlehre und Grammatik machten,
so sollen sie sich fragen, ob nicht etwa sie die Narren und 
Dummköpfe seien, die nicht reden wie die Bibel und die Grammatik
lehre, sondern uns -- die es tun -- darum schelten? So war dem
Priester der Mund gestopft\index{Den Mund stopfen}, und viele wurden 
von der Wahrheit überzeugt und waren recht empfänglich und zugänglich. 
Einige boten mir Geld an, aber ich nahm es nicht.

Hierauf kam Doktor Cradock mit drei weiteren Priestern,
dem Gouverneur und seiner Frau, einer "`Dame"' (lachs) wie
man zu sagen pflegt, und einer andern "`Dame"' und eine ganze
Schar mit ihnen. Doktor Cradock fragte mich, warum ich im
Gefängnis sei; ich antwortete: "`Weil ich den Geboten Christi und
der Apostel, nicht zu schwören, gehorche."' Wenn aber er, ein
Doktor und Friedensrichter, mir beweisen könne, das Christus
% \picinclude{./180-189/p_s184.jpg} 
oder der Apostel den Christen, nachdem er ihnen verboten hatte,
zu schwören, es ihnen nachher wieder zu tun befahl, so wolle
auch ich es tun. Ich Gab ihm die Bibel, damit er mir irgend
ein solches Gebot zeige, wenn er könne. Er sagte: "`Ihr sollt
ohne Heuchelei und heiliglich schwören (Jer. 4, 2)"'. "`Ja, ja,"' sagte
ich, "`so hieß es zu Jeremias Zeiten, aber das war lange bevor
Christus befahl: ihr sollt überhaupt nicht schwören (Matth. 5, 34).
Aus dem alten Testament könnte ich ebenso viele Beispiele oder vielleicht
noch mehr bringen, aber was nützen sie für den Beweis, das dass
Schwören auch im neuen Testament erlaubt war, nachdem Christus
und die Apostel es verboten? Übrigens: zu wem wird dort gesagt, 
sie sollten nicht schwören? zu den Heiden oder zu den Juden?"'
Hierauf gab er keine Antwort. Aber einer der Priester sagte:
"`zu den Juden,"' und Doktor Cradock gab es zu. "`Gut,"' sagte
ich, "`aber wo hat Gott je den Heiden ein Gebot gegeben zu
schwören? und ihr wisset ja, das wir von Natur Heiden sind."'
"`Allerdings,"' sagte Doktor Cradock; "`zwar zur Zeit des 
Evangeliums musste alles aus zweier oder dreier Zeugen Mund bestätigt
werden, aber geschworen wurde nicht"'. "`Warum also,"' fragte ich,
"`zwingst du den Christen Eide ab gegen dein besseres Wissen?
und warum exkommunizierst du die Freunde?"' (er hatte nämlich
viele sowohl in York als auch in Lancashire exkommuniziert). Er
sagte: "`weil sie nicht in die Kirche kamen."' "`So!"' sagte ich,
"`vor mehr als zwanzig Jahren, als wir noch Knaben und Mädchen
waren, da überließet ihr uns den Presbyterianern, den Independenten 
und Baptisten, und viele von diesen nahmen uns Hab und
Gut und verfolgten uns, weil wir uns ihnen nicht anschließen
wollten; damals waren wir noch jung und wussten wenig
von euren Ansichten; hättet ihr nun die alten Leute, denen sie
bekannt waren, bei euch behalten und eure Ansichten in Kraft
erhalten wollen, so hättet ihr sollen entweder euch nicht von uns
wenden, wie ihr getan, oder ihr hättet uns sollen eure Episteln,
Kollekten, Homilien und Abendliturgien senden, wie Paulus ja
auch den Heiligen geschrieben hatte, als er in der Gefangenschaft
von ihnen getrennt gewesen war. Wir hätten allesamt können
Türken oder Juden werden, was das, was wir in dieser Zeit von
euch empfigen, anbelangt; und nun habt ihr uns, alt und jung,
exkommuniziert, also aus eurer Kirche ausgestoßen, ehe ihr uns für
dieselbe gewonnen habt. Ist es nicht ein Unsinn, uns auszuweisen,
% \picinclude{./180-189/p_s185.jpg} 
ehe wir drin waren? Ja, wenn ihr uns für eure Kirche
gewonnen hättet und wir ihr angehört hätten und dann etwa
Unrechtes getan hätten, so wäre es einigermaßen begründet 
gewesen. Was nennst du übrigens "`Kirche?"'\index{Eklesiologi} "`Nun,"' sagte er,
"`das was du "`Turmhaus"' nennst."' Darauf fragte ich ihn, ob
denn Christus sein Blut für das \textit{Turmhaus} vergossen habe. "`Und,"'
sagte ich, "`wenn nun die Kirche die Braut Christi und Christus das 
Haupt der Kirche genannt wird, glaubst du denn, das \textit{Turmhaus}
sei die Braut Christi und er das Haupt dieses alten Gebäudes?
ist er nicht vielmehr das- Haupt der Gemeinde?"' (Eph. 5)\bibel{Eph. 05@Eph. 5}. 
"`Er ist das Haupt der Gemeinde,"' erwiderte er, "`und sie ist die
Kirche."' "`Ihr habt also den Namen Kirche, welcher der Gemeinde
zukommt, einem alten Hause gegeben,"' sagte ich, "`und habt die
Leute gelehrt, solches zu glauben!"' Weiter fragte ich ihn, warum
die Freunde verfolgt werden darum, das sie den Zehnten nicht
geben? Ob Gott je den Heiden geboten habe, den Zehnten zu
bezahlen? Ob Christus nicht die Zehnten aufgehoben habe, als
er das Levitische Priestettum\index{Levitische Priestettum}, 
das Zehnten nahm, aufhob?\index{Zehnten}\index{Kirchensteuer} Und
ob Christus, als er seine Jünger aussandte zu predigen, ihnen
nicht geboten habe, umsonst zu predigen? und ob nicht alle Diener
Christi verpflichtet seien, dieses Gebot zu halten? Er sagte, er
wolle hierüber nicht streiten; er schien überhaupt nicht gern bei
diesem Gegenstand zu verharren, sondern ging bald zu einem
andern über und sagte: "`Ihr verheiratet euch, aber man weiß
nicht, wie ihr dabei verfahrt."' Ich riet ihm, zu kommen und
selbst zu sehen.\index{Öffentlichkeit} Er drohte, uns seine Macht fühlen zu lassen; ich
riet ihm, zu bedenken, das er ein alter Mann sei, und fragte ihn,
wo er von der Genesis bis zur Offenbarung irgendwo lese, das
ein Priester jemand getraut habe; er solle mir ein solches Beispiel
zeigen, wenn er wolle, das wir zu ihnen kommen sollten, um
ums trauen zu lassen. "`Du hast ja,"' sagte ich, "`einen der Freunde
zwei Jahre nach seinem Tode noch exkommuniziert\index{Exkommunizieren} 
wegen seiner Ehe; warum exkommunizierst du nicht auch Jsaak, Jakob, Boaö und
Ruth? Warum machst du deine Macht nicht Lauch an diesen
geltend? Denn es steht nirgends, das sie von einem Priester
getraut worden seien, sondern sie nahmen einander in der 
Versammlung in Gegenwart Gottes und seiner Gemeinde; und so
tun wir. Wir haben also die heiligen Männer und Frauen der
Schrift auf unsrer Seite in dieser Sache."' Wir redeten lange
% \picinclude{./180-189/p_s186.jpg} 
hin und her; als er aber sah, das er nichts über mich vermochte,
ging er fort mit seinen Begleitern [...]

In diesem und dem vorhergehenden Jahre waren viele
Freunde gefangen genommen worden. Viele waren in London, in
Newgate und andern Gefängnissen, wo die Krankheit (Pest)\index{Pest} herrschte,
und starben dort. Viele wurden auch verbannt\index{Scheiterhaufen} und auf des
Königs Befehl auf Schiffe gebracht. Oft wollten die Schiffsherren
sie nicht aufnehmen und setzten sie wieder ans Land; doch gelangten
viele nach Barbados\ort{Barbados}, Jamaika\ort{Jamaika} und Nevis, 
und der Herr segnete sie dort [...]

Nachdem ich mehr als ein Jahr im Schlos zu Scarbro gefangen 
gewesen war, schickte ich einen Brief an den König, in
dem ich ihm von meiner Gefangenschaft berichtete und von der
schlechten Behandlung, die ich während derselben zu erdulden
hatte, und das man mir gesagt habe, niemand als er könne mich
frei machen. Und John Whitehead\person{Whitehead, John} 
begab sich zu Gsquire Marsh,
mit dem er befreundet war, um ihm von mir zu reden, und dieser
versprach, das, wenn John Whitehead einen Bericht über meine
Angelegenheit verfassen wolle, er denselben John Birkenhead, der
über die Begnadigungsgesuche zu entscheiden hatte, einhändigen
und sich um meine Freisprechung bemühen wolle. John
Whitehead und Gllis Hookes\person{Hookes, Gllis} 
verfassten nun einen Bericht über
meine Gefangennahme und meine Leiden während der Gefangenschaft 
und brachten ihn Marsh, der ihn John Birkenhead überbrachte 
und einen Befehl zu meiner Freisprechung erwirkte. In
demselben hieß es, das der König von glaubwürdiger Seite erfahren 
habe, ich sei stets gegen alles Komplottieren und Streiten
gewesen, und habe etwaige Verschwörungen eher entdecken helfen,
als das ich mich selber daran beteiligt hätte, und so sei es Sein
königliches Wohlgefallen, das ich aus meiner Gefangenschaft befreit
werde. Sobald dieser Befehl bekannt war, kam John Whitehead
damit nach Scarbro und übergab ihn dem Gouverneur, der nun
die betreffenden Behörden zusammen berief und ohne weitere
Bürgschaft für mein friedsames Leben, zufrieden mit der Erklärung,
das ich ein stiller Bürger sei, mich frei ließ [...]

Gleich am Tage nach meiner Freilassung brach das Feuer
in London aus, und das Gerücht davon verbreitete sich rasch im
Lande. Da sah ich, das Gott der Herr sein Wort wahr gemacht
hatte, das ins Gefängnis zu Lancaster zu mir geschehen war, als
% \picinclude{./180-189/p_s187.jpg} 
ich den Engel\index{Engel} des Herrn gesehen hatte, wie er mit einem 
leuchtenden Schwerte gen Süden zeigte, wie ich schon berichtet habe.\index{Vorhersehung}
Die Bewohner waren vor diesem Feuer gewarnt worden; aber
wenige hatten es geglaubt oder zu Herzen genommen, vielmehr
wurden sie noch schlechter und hochmütiger. Ein Freund war nämlich
getrieben worden, von Huntingdonshire\ort{Huntingdonshire} herunter zu kommen, kurz
vor der Feuersbrunst, und sein Geld herum zu streuen, sein Pferd
frei in den Straßen herum zuführen, die Kniebänder aufzulösen,
die Strümpfe herunter hängen zu lassen, das Wams aufzuknöpfen
und den Leuten zu sagen: "`so werdet ihr herum laufen und euer
Hab und Gut umherstreuen, halb nackt, wie Wahnsinnige; und so
geschah es, als die Stadt brannte. So machte der Herr seine
Propheten\index{Prophezeiung} und Diener zu Werkzeugen seiner Kraft und gab
ihnen Zeichen seines Gerichts und sandte sie, das Volk zu warnen;
aber statt Buße zu tun, haben sie sie misshandelt und etliche
gefangen genommen, unter der früheren Regierung sowohl als
jetzt; aber der Herr ist gerecht, wohl dem, der seinen Worten
gehorcht! Etliche trieb es, nackt in den Straßen umher zu laufen,\index{Nackt in der Öffentlichkeit}
um zu zeigen, wie Gott ihnen ihre heuchlerische Frömmigkeit
abreißabreißen werde und sie nackt und bloß machen werde. Aber
das Vodas Volk hatte, statt in sich zu gehen, diese oft gegeißelt oder
sonst mißhandelt oder gar gefangen genommen. Andere trieb es,
in Sticken umhetzugehen und die Rache und Strafe Gottes wegen
des großen Hochmutes zu verkünden; aber wenige gaben darauf
acht. In den Tagen der früheren Regierung machten die falschen,
frömmlerischen Priester mehrere Petitionen gegen uns an Oliver
und Richard, die sogenannten Protektoren, und an das Parlament
und die Richter und Räte, voller Lügen, Verleumdungen und
Schmähungen; aber wir verschafften uns Abschriften davon, und
mit Gottes Hilfe antworteten wir auf alle, und wuschen die
Wahrheit und uns rein. Aber o, welche Mächte der Finsternis erhoben
sich in denen, die zum Lügen ihre Zuflucht nahmen! aber der
Herr stürzte sie alle und schützte seine Lämmer durch seine Kraft
und Wahrheit, sein Licht und sein Leben, und deckte sie, wie mit
Adlers Flügeln. Solches gab uns Mut, aus ihn zu vertrauen,
der alle, die sich im Finstern gegen seine Wahrheit und sein Volk
verbünden, stürzt und vernichtet, und der durch diese Wahrheit 
seinem Volk Macht gibt, ihm in der Wahrheit zu dienen. [...]
% \picinclude{./180-189/p_s188.jpg} 

