% \picinclude{./200-209/p_s200.jpg} 
von dir Genannten, können in einzelnen Fällen schwören und tun
es auch; wir dagegen schwören in keinem Fall. Wenn man jenen
ihre Kühe oder Pferde stehlen würde und du würdest sie fragen,
ob sie schwören wollen, das es die ihrigen seien, so wären manche
von ihnen gleich bereit, es zu tun. Wenn du es aber bei unsern
Freunden versuchst, so schwören sie auch um ihrer Habe willen
nicht.\index{Schwören} Darum, wenn du von einem von ihnen den Huldigungseid
forderst, so musst du ihn fragen, ob er in andern Fällen schwören
kann, z. B., wenn es seine Kuh oder sein Pferd betreffe, was er,
wenn er wirklich zu uns gehört, nicht kann."' Darauf erzählte
ich ihm folgendes aus einem Verhör in Berkshire:\ort{Berkshire} 
"`Ein Dieb stahl einem unsrer Freunde zwei Stück Vieh; der Dieb wurde erwischt
und ins Gefängnis gebracht, und der Freund trat vor Gericht
als Kläger\index{Quaker als Kläger} gegen ihn aus. Da aber der Richter gehört hatte,
das der Kläger ein Quäker sei und nicht schwören könne, so rief
er, ohne erst anzuhören, was der Freund zu sagen hatte: "`Ist
er Quäker, will er nicht schwören?"' und legte ihm den Huldigungseid 
vor. Und darauf schickte er den Freund ins Gefängnis und unterwarf 
ihn den Strafen gegen Eidverweigerung. Der Dieb aber, der
ihn bestohlen hatte, wurde freigesprochen!"' Richter Marsh sagte:
"`Dieser Richter war ein schlechter Mensch."' "`Wenn wir irgend
schwören könnten,"' sagte ich, "`so wäre es dem König, der zum
Schutz seiner Untertanen die Gesetze aufrecht erhält. Die andern
dagegen, die doch schwören, wenn es gilt, ihr Hab und Gut zu
schützen, weigern sich gerade, dem König den Huldigungseid zu
schwören. Von diesen kannst du uns also leicht kennen und unterscheiden."'

Richter Marsh hat später sehr viel für die Freunde getan,
indem er viele von ihnen vor den Strafen wegen Eidverweigerung
schützte. Wenn während der Verfolgungen Freunde vor ihn gebracht
wurden, so schenkte er ihnen, wenn möglich die Freiheit, und wenn 
er es nicht verhindern konnte, das sie ins Gefängnis kamen, so
machte er, das es nicht für länger als ein paar Stunden oder
für eine Nacht sei. Schließlich ging er zum König und sagte
ihm, er habe einige von uns gegen sein Gewissen ins Gefängnis
geschickt und wolle das fernerhin nicht mehr tun. Er verließ
darum mit seiner Familie Limehouse\ort{Limehouse} und zog in die Nähe von
St. James Park.\ort{St. James Park} Er sagte dem König, wenn er sich dazu 
verstehen könnte, Gewissensfreiheit zu erklären, so würde alles ruhig
% \picinclude{./200-209/p_s201.jpg} 
werden; denn dann könnte man nicht mehr Gewissenskrupeln
geltend machen. Ja, er hat in jenen Tagen viel für die Wahrheit 
und die Freunde getan.

%%%%%%%%%%%%%%%%%%% Kapitel 17. %%%%%%%%%%%%%%%%%%%%%%%%%%%%%%

\chapter[Reise nach Irland und Heirat mit Margaret Fell]{Reise 
nach Irland und Heirat mit Margaret Fell}

\begin{center}
\textbf{Reise nach Irland. Rückkehr und Heirat mit Margaret Fell.
Ihre abermalige Gefangennahme. Schwere innere Anfechtungen.}
\end{center}

Der Herr trieb mich, nach Irland\ort{Irland} zu gehen um dort den
Samen Gottes zu besuchen [...] Als wir auf dem Schiff
waren, rief ich meinen Gefährten zu: "`Lasset uns fröhlich sein
im Herrn, denn wir werden gute Winde haben!"' Viele waren
krank während der Überfahrt, aber niemand von den Unsrigen.
Der Kapitän und auch viele der Mitreisenden waren uns sehr
zugetan, und da wir an einem Ersten Tage auf dem Wasser
waren, trieb es mich, die Wahrheit unter ihnen zu verkünden,
worauf der Kapitän zu den andern sagte: "`Das sind Dinge, von
denen ihr noch nie in eurem Leben gehört habt"'. Vor Dublin\ort{Dublin}
nahmen wir ein Boot und fuhren ans Land, und mir schien, das
der Boden und die Luft von der Verdorbenheit dieser Nation
übel rieche; wenigstens empfand ich einen ganz andern Geruch als
in England, was ich den päpstlichen Gräueltaten, die hier 
begangen wurden, zuschrieb, und dem Blute, das hier vergossen
worden war und nun Fäulnis ausströmte [...] Wir fanden
nicht gleich Freunde und begaben uns darum nach einer Herberge
und ließen welche aufsuchen und zu uns kommen. Sie waren
alle sehr froh über unser Kommen, und empfingen uns mit großer
Freude. Wir blieben zur Wochenversammlung,\index{Wochenversammlung} 
die sehr zahlreich war und gesegnet mit der Kraft und dem Leben aus 
Gott [...] Danach gingen wir zu einer Versammlung in der Provinz [...]
und zogen dann nach einem andern Orte wo wir eine sehr schöne,
erbauliche Versammlung hatten, aber einige Papisten\index{Papisten} die ihr
beigewohnt hatten, waren nachher sehr zornig und wütend. Als
ich dies vernahm, lies ich einen von ihnen zu mir kommen, einen
Schulmeister, aber er wollte nicht kommen. Darauf schickte ich
an ihn, sowie an alle Mönche, Klosterbrüder, Priester und Jesuiten
eine Aufforderung, ihren Gott und ihren Christus, die sie aus
Brot und Wein gemacht,\index{Abendmahl} zu erproben; aber ich konnte keine
Antwort von ihnen erlangen. Dann erklärte ich ihnen, sie seien
% \picinclude{./200-209/p_s202.jpg} 
ärger als die Baalzpriester:\index{Baalzpriester} denn die 
Baalzpriester hätten ihren hölzernen Gott erprobt, sie aber 
dürften nicht wagen, ihren Gott aus Brot und Wein\index{Gott 
aus Brot und Wein}\index{Eucharist} zu erproben, und die Baalzpriester und ihre
Anhänger hätten ihren Gott nicht gegessen wie sie und nachher
einen andern gemacht [...]

Der damalige Bürgermeister von Cork\ort{Cork} war den Freunden
und der Wahrheit übel gesinnt und hielt viele Freunde gefangen,
und weil er wusste, das ich im Lande war, hatte er Befehle
erlassen, mich zu verhaften, darum wollten die Freunde nicht,
das ich durch Cork reiste. Aber als ich in Bandon\ort{Bandon} war, erschien
mir in einem Gesicht ein sehr hässlicher Mensch mit finsterem,\index{Vision}
bösem Blick; mein Geist schlug nach ihm in der Kraft Gottes,
und es war mir, als ob ich über ihn weg ritte mit dem Pferde,
und das Pferd den Fuß auf sein Gesicht setze. Als ich am Morgen
hinunter kam, berichtete ich einem der Freunde, was mir 
widerfahren, und das des Herrn Befehl an mich ergangen, durch Cork
zu reiten, aber ich bat ihn, es niemand zu sagen. So ritt ich
mit vielen Freunden von dannen; als wir uns der Stadt näherten,
hätten sie mir gerne einen Weg hinten um die Stadt herum 
gezeigt, aber ich sagte ihnen, mein Weg gehe durch die Straßen.
Ich nahm also einen von ihnen, er hieß Paul 
Morrice,\person{Morrice, Paul} mit mir,
um mir den Weg durch die Stadt zu zeigen und ritt hinein.
Als wir über den Marktplatz ritten und am Hause des 
Bürgermeisters vorbei, sagte dieser, als er mich vorüberreiten sah: "`da
geht George Fox vorbei"', aber er hatte nicht Macht, mich 
anzuhalten. Als wir die Wachen und die Brücke passiert hatten,
gingen wir zum Hause eines Freundes und stiegen ab. Hier
berichteten mir die Freunde, was für eine Erbitterung in der
Stadt herrsche, und wie viele Befehle erlassen werden, um mich
gefangen zu nehmen. Während ich hier mit den Freunden zusammen 
saß, spürte ich, wie der böse Geist am Werk war in der
Stadt, um Unheil gegen mich anzustiften, und ich spürte auch,
wie des Herrn Kraft diesen bösen Geist schlug. Andere Freunde,
die nach und nach herein kamen, berichteten mir, das in der
Stadt und unter den Behörden meine Anwesenheit bekannt geworden 
sei. Ich sagte: "`Last den Teufel sein Äußerstes tun"'.
Nachdem die Freunde sich gegenseitig gestärkt hatten und wir
Reisende uns auch gestärkt hatten, lies ich mein Pferd holen und
ging mit einem Freunde, der Mich führte, meiner Wege. Aber
% \picinclude{./200-209/p_s203.jpg} 
der Zorn war groß bei den Behörden und den Leuten von Cork,
das sie mich verfehlt hatten, und sie gaben sich in der Folge
große Mühe, mich zu erwischen, indem sie, wie ich hörte, überall
ihre Späher hatten, um zu forschen, welchen Weg ich gehe, und
in fast jeder öffentlichen Versammlung, der ich beiwohnte, kamen
Späher, um zu sehen, ob ich da sei. Die Behörden sandten 
einander Berichte über mich, in denen sie mich nach Haaren, Hut,
Kleidern und Pferd beschrieben, so das man, als ich schon 
hundert Meilen von Cork weg war, Bericht und Beschreibung über
mich hatte, ehe ich ankam. Einer, der zu den Schlimmsten in
der Behörde gehörte, und zugleich Priester und Friedensrichter
war, erhielt eine Vollmacht vom Richter, mich zu verhaften; dieser
Befehl erstreckte sich über seinen ganzen Bezirk, der mehr als
hundert Meilen umfasste. Aber der Herr machte alle ihre
Anschläge zu nichte und vereitelte alle ihre Vorhaben. Die treue
Hand seiner Vorsehung behütete mich vor allen ihren Fallstricken
und gab uns manche gesegnete Gelegenheit, Freunde zu besuchen
und die Wahrheit im Lande zu verbreiten. Die Versammlungen
waren sehr zahlreich, da die Freunde von nah und fern sie
besuchten, und auch andere Leute herzu strömten. Die mächtige
Kraft des Herrn machte sich herrlich fühlbar mit und unter uns;
dadurch wurden viele Weltlichgesinnte ergriffen und überzeugt
und für die Wahrheit gewonnen; die Herde des Herrn wuchs,
und die Freunde wurden erquickt und gestärkt durch das Gefühl
der Liebe Gottes. O, wie wurden sie ergriffen von den Strömen
des Lebens, so das viele in der Kraft und dem Geist Gottes
miteinander in Singen\index{Singen} ausbrachen, und dem Herrn spielten in
ihren Herzen!

Viele angesehene Personen kamen ins Haus von James 
Hutchinson\person{Hutchinson, James} in Jrland, um 
mit mir über Erwählung und Verwerfung zu\index{Prädistinationslehre}
reden. Ich sagte ihnen: "`Wenn ihr schon unsre Ansichten als
verrückt verwerft: sie sind eben zu hoch für euch, ihr könnt sie
mit eurer Weisheit nicht verstehen, darum will ich mich in dem,
was ich sage, nach eurem Verständnis richten. Ihr sagt, Gott
habe die meisten Menschen zur Hölle verdammt, und sie seien
dazu verordnet von Anbeginn der Welt, und bringt als Beweis
dafür den Judasbrief.\bibel{Judasbrief} Ihr sagt, 
Esau\person{Esau} sei verworfen gewesen
und die Ägypter und die Nachkommen des Ham. Aber Christus
sagt seinen Jüngern: "`gehet hin und prediget allen Völkern"' und
% \picinclude{./200-209/p_s204.jpg} 
"`gehet hin in alle Lande."' Wenn sie nun zu allen Völkern gehen
mussten, mussten sie denn dann nicht auch zu den Nachkommen
Esaus und Hams gehen? Ist nicht Christus für alle gestorben,
also auch für die Nachkommen Esaus, Hams und der Ägypter?\index{Opfertod}
Sagt nicht die Schrift, Gott will, das allen Menschen geholfen werde?
Merket wohl: allen Menschen, also auch den Nachkommen Esaus
und Hamtz. Sagt nicht Gott: "`Agypten mein Volk ?"' 
(Jes. 19,25)\bibel{Jes. 19:25}
und, das er einen Altar in Agypten haben wolle? 
(Jes. 19,19)\bibel{Jes. 19:19}.
Waren nicht viele Christen früher in Agypten?\ort{Agypten} Und berichtet
nicht die Geschichte, das der Bischof von Alexandrien\ort{Alexandrien} Papst 
gewesen ist? Und hat nicht Gott eine Kirche in Babylonien?\ort{Babylonien} Ich
gebe zu: "`das Wort geschah zu Jakob und das Recht an Israel"'
(Ps. 147,19)\bibel{Ps. 147:19}; solches kam den andern Nationen nicht zu
denn das Gesetz Gottes war nur Israel gegeben, das Evangelium
jedoch sollte allen Völkern gepredigt werden und soll es noch.
Für alle Menschen gilt die gute Botschaft des Friedens: "`wer
da glaubet, der wird selig, wer aber nicht glaubt, ist schon
gerichtet"'\index{gerichtet} (Mark. 16,16)\bibel{Mark. 16:16}. 
Die Verdammung\index{Verdammung} kommt also durch
den Unglauben. Und wenn Judas von etlichen sagt, sie seien
vor Zeiten zur Verdammung bestimmt, so sagt er nicht vor 
Anbeginn der Welt, sondern "`geschrieben vor Zeiten"', was sich auf
die Schrift Moses beziehen kann, in welcher von denen geschrieben
steht, die Juda erwähnt, nämlich Kain, Korah, Bileam und die
Engel, die ihr Fürstentum nicht behielten. Die Christen nun,
welche solchem Wandel nachfolgen und abgefallen sind vom Stand
der ersten Christenheit, waren und sind zur Verdammung bestimmt
durch das Licht und die Wahrheit, davon sie abgefallen sind.
Und obgleich der Apostel sagt: "`Gott liebte Jakob und haste
Esau"'\person{Esau} (Röm.9,13),\bibel{Röm. 09:13@Röm. 9:13} so erinnert er 
doch die Gläubigen, "`wir waren alle Kinder des Zorns von 
Natur, gleich wie die andern"' (Eph. 2,3).\bibel{Eph. 02:3@Eph. 2:3}
Dies schließt auch den Stamm Jakobs ein, welchem der Apostel
selber und alle gläubigen Juden angehörten. Die Juden wie die
Heiden standen also unter der Sünde\index{Sünde} und somit 
unter der Verdammung, damit Gott sich aller erbarme in Jesus Christus.
Die Erwählung stehet bei Christus, und "`wer da glaubt, wird
selig"' und "`wer nicht glaubt, ist schon verdammt"'. Jakob 
repräsentiert die zweite Geburt, die Gott liebt; und sowohl Juden
wie Heiden müssen wiedergeboren\index{wiedergeboren} werden, 
ehe sie ins Reich Gottes\index{Reich Gottes} eingehen können. Wenn ihr wiedergeboren 
sein werdet, so werdet
% \picinclude{./200-209/p_s205.jpg} 
ihr verstehen, was Erwählung und Verwerfung bedeutet, denn
die Grwählung stehet in Christus, dem Samen, der "`gewesen,
ehe der Welt Grund gelegt war"'; die Verwerfung dagegen liegt
im schlechten Samen, welcher erst nach der Erschaffung der Welt
entstand. In dieser Weise, nur etwas ausführlicher, redete ich
mit diesen Leuten, und sie gestanden, das sie dergleichen noch
nie gehört hätten.

Nachdem ich in Irland umher gereist war und die Freunde
in ihren Versammlungen besucht hatte, sowohl in geschäftlichen
Angelegenheiten als um mit ihnen Gottesdienst zu halten, und
verschiedene Schreiben von Mönchen, Klosterbrüdern und 
protestantischen Priestern beantwortet hatte,\index{Verteidigungsschriften} 
(denn sie waren alle wütend über uns und suchten das Werk des Herrn zu hindern,
und wir hörten, wie einige Jesuiten schworen, wenn wir auch
kämen um unsere Ideen im Lande zu verbreiten, so solle es uns
nicht gelingen), kehrte ich nach Dublin zurück, um mich nach England
einzuschiffen [...]

Es sind in Irland gute, tüchtige und aufrichtige Menschen,
die die Kraft Gottes spüren und empfänglich sind für die 
Wahrheit, und sie halten gute Ordnung in ihren Versammlungen,
denn sie stehen für Heiligkeit und Gerechtigkeit ein, welche den
Weg der Schlechtigkeit versperren. Ich könnte noch viel über
die Leute dieses Landes schreiben und über meine Reisen unter
ihnen, was zu weit führen würde; dieses aber erwähnte ich gern,
damit der Gerechte sich an dem Gedeihen der Wahrheit freuen
moge.

James Lancaster, Robert Briggs und Robert Lodge kehrten
mit mir zurück. John Stubbö, der noch zu tun hatte, blieb zurück.
Wir waren zwei Nächte auf dem Wasser; in der einen Nacht
brach ein starker Sturm aus, der das Schiff in große Gefahr
brachte. Aber ich sah, das Gottes Macht größer war als Wind
und Sturm. Er hielt sie in seiner Hand, und seine Macht bändigte
sie. Die gleiche Macht des Herrn, welche uns hinüber gebracht
hatte, brachte uns wieder zurück, und sein Leben hatte uns Herrschaft
gegeben über alle bösen Geister, welche uns dort entgegen gewesen
waren. Wir landeten in Liverpool. Danach gingen wir nach
Gloucerstershire, wo wir ein Gerücht\index{Gerücht} vernahmen, das sich in der
Gegend verbreitet hatte: George Fox sei Presbyterianer geworden,
man habe eine Kanzel für ihn errichtet im Freien, und es würden
% \picinclude{./200-209/p_s206.jpg} 
am folgenden Tage Tausende von Menschen kommen, um ihn zu
hören. Ich wunderte mich, wie ein solches Gerücht sich über
mich verbreiten konnte. Wir hörten jedoch beim nächsten
Freund, zu dem wir kamen, ebenfalls davon. Wir sahen im
Vorbeigehen die Kanzel auf dem Felde stehen; dann zogen wir
weiter an den Ort, wo am folgenden Tage die Versammlung der
Freunde stattfinden sollte, und blieben dort über Nacht. Am 
folgenden Tage, es war der Erste Tag, hatten wir eine sehr große
Versammlung, und des Herrn Kraft und Gegenwart war unter
uns. Die Veranlassung zu jenem Gerücht war, wie ich hörte,
folgende gewesen. Einer namens John Fox,\indexname{Personen!Fox, John} 
ein Presbyterianer-Priester, reiste herum und predigte und es hieß, einige hätten
statt John George gesagt und ausgestreut, George Fox habe
seinen Glauben geändert und sei aus einem Quäker ein 
Presbyterianer geworden und werde an dem und dem Tage an dem
und dem Orte predigen. Daraufhin entstand eine solche Neugierde 
unter den Leuten, das viele, welche niemals gegangen
wären, John Fox zu hören, liefen, um diesen presbyterianisch
gewordenen Quäker zu hören. Auf diese Weise, hieß es, hätten
sie Tausende von Menschen zusammengebracht. Als sie aber kamen
und sahen, das man ihnen einen Streich gespielt hatte, das es
nur ein falscher George Fox\index{falscher George Fox} sei, 
und erfuhren, der rechte George
Fox sei ganz in der Nähe, kamen Hunderte von ihnen in unsere
Versammlung und waren sehr aufmerksam und ruhig. Ich wies
sie auf die Gnade Gottes, die in ihnen sei, welche sie lehren und
ihnen das Heil bringen könne. Nach der Versammlung sagten
einige, sie hörten den Quäker George Fox lieber predigen als den
Presbyterianer George Fox. So war durch mein providentielles
Erscheinen in dieser Gegend, gerade zu dieser Zeit, dieses falsche
Gerücht entdeckt worden und seine Urheber wurden zu Schanden.
Bald darauf wurde John Fox im Unterhause angeklagt, er
halte stürmische Versammlungen, in welchen verräterische Worte
geredet würden. Damit verhielt es sich, nach dem, was ich darüber
erfahren konnte, also: Er war früher Priester in Mansfield\ort{Mansfield} in
Wiltshire gewesen, und als er von dort fortgeschickt worden war,
hatte ihm ein Common-Prayer Priester erlaubt, manchmal in
seinem Turmhaus zu predigen. Schließlich wurde aber dieser
Presbyterianer-Priester kecker als erwünscht war und machte
allzu sehr von der ihm gegebenen Erlaubnis Gebrauch und 
% \picinclude{./200-209/p_s207.jpg} 
versuchte dort zu predigen, ob der Ortspfarrer es gern hatte oder
nicht. Dies führte zu viel Zwist und Reibereien zwischen den
beiden Priestern und ihren Hörern, wobei sogar das Common-
Prayer Buch zerschnitten wurde und es fielen einige verräterische
Worte bei einigen der Anhänger des John Fox. Dies wurde
sogleich bekannt, und einige böswillige Presbyterianer verbreiteten
die Sache so, als ob sie von George Fox, dem Quäker, herrührten,
obgleich ich etwa zweihundert Meilen weit weg war. Als ich
davon hörte, verschaffte ich mir sofort eine Bescheinigung von
einem Mitglied des Unterhauses, das diesen John Fox kannte,
und lies verbreiten, das es sich um John Fox, den früheren
Priester von Mansfield handle [...] Dieser John Fox erwies
sich auch später als ein schlechter Kerl. [...]

Wir zogen weiter bis wir nach Bristol\ort{Bristol} kamen, wo ich Margaret
Fell\person{Fell, Margaret} traf, die gekommen war, um ihre Tochter Yeomans 
zu besuchen. Der Herr hatte mir schon vor längerer Zeit gezeigt, das
ich Margaret Fell zur Frau nehmen solle. Und als ich das erste
mal mit ihr redete, kam ihr die Antwort darauf von oben. Aber
obgleich mir der Herr solches eröffnet hatte, so hatte ich doch
damals noch keinen Befehl von ihm erhalten, es auszuführen.
Darum lies ich die Sache ruhen und fuhr wie bisher fort in der
Arbeit und dem Dienst des Herrn, wie er mich führte, hier im
Lande wie in Irland. Als ich nun aber in Bristol war und
Margaret Fell da traf, offenbarte mir der Herr, das die Sache
nun ausgeführt werden müsse. Nachdem wir miteinander darüber
geredet, sagte ich ihr, wenn es ihr auch recht sei, es jetzt zu tun,
so solle sie zuerst ihre Kinder kommen lassen, was sie auch tat.
Als alle ihre Töchter beisammen waren, so fragte ich dieselben
sowie auch die Schwiegersöhne, ob sie irgend etwas dafür oder
dawider hätten? Sie sprachen alle einmütig ihre Zufriedenheit
darüber aus. Darauf fragt ich Margaret, ob sie den Willen ihres
Gatten den Kindern gegenüber erfüllt und ausgeführt habe?
Sie erwiderte, "`die Kinder wissen, das ich es tat"'. Darauf fragte
ich die Kinder, ob es nicht ein Verlust für sie bedeute, wenn
ihre Mutter wieder heirate? Und Margaret fragte ich, ob sie
ihren Kindern irgend welche Bürgschaft dafür geleistet habe?
Die Kinder sagten, sie habe es getan, und ich möchte nicht mehr
davon sprechen. Ich sagte ihnen, ich sei ein Mann der Geradheit\index{Geradheit}
und möchte, das alles offen zugehe, denn ich suche keinerlei
% \picinclude{./200-209/p_s208.jpg} 
äuseren Vorteil für mich bei dieser Sache. Nachdem ich nun
den Kindern die Sache vorgelegt hatte, wurde unsere Absicht,
uns zu ehelichen, vor die Freunde gebracht, sowohl einzeln als
öffentlich; alle waren sehr einverstanden, und viele von ihnen
bezeugten, das komme vom Herrn. Darauf wurde eine Versammlung
veranstaltet im Broad-Mead Versammlungshaus in Bristol,\ort{Bristol} damit
es vollzogen werde, und wir nahmen einander, indem der Herr
uns verband in der ehrbaren Ehe, in dem ewigen Bund und
dem unsterblichen Samen.\index{Heirat} Unter dem Eindruck davon legten
manche der Freunde lebendiges und ergreifendes Zeugnis ab,
getrieben von der himmlischen Kraft, die uns miteinander verband.
Darauf wurde ein Ausweis über die Verhandlungen und über
die Eheschließung öffentlich verlesen und unterzeichnet von den
Verwandten, den meisten der ältesten Freunde der Stadt sowie
von vielen andern aus verschiedenen andern Teilen des Landes [...]

Als ich wieder in London war, trieb es mich, den Freunden
im ganzen Lande darüber zu schreiben, das man arme Kinder
bei Handwerkern gegen Arbeit unterbringen solle. Ich sandte
darum an die Vierteljahrsversammlung\index{Vierteljahrsversammlung} 
jeder Grafschaft folgenden Brief:

\brief{Quaker-Gemeinde}{
  Meine lieben Freunde,
  \medskip 
  Jede Vierteljahrsversammlung soll sich bei allen monatlichen
  und andern Versammlungen erkundigen nach allen Witwen und
  andern unter den Freunden, die Kinder haben, welche fähig
  wären, ein Handwerk zu erlernen, so das man jedes Vierteljahr
  von der Vierteljahrsversammlung aus einen in die Lehre schicken
  kann; so können jährlich vier in jeder Grafschaft ausgeschickt
  werden oder auch mehr, wenn sich die Gelegenheit bietet. Diese
  Lehrlinge können dann, wenn sie ausgelernt haben, den Eltern
  helfen und der Familie, wenn sie heruntergekommen ist, wieder
  aufhelfen, so das alle nach und nach behaglich leben können.
  Wenn ihr dies in euren vierteljährlichen Versammlungen tut, so
  werdet ihr in den monatlichen Versammlungen von den geeigneten
  Meistern in der Grafschaft hören und von Gewerben, wie die
  Eltern und ihr sie wünscht, oder für die die Kinder besondere
  Neigung haben. Wenn sie in der Weise bei Freunden untergebracht 
  sind, so werden sie in der Wahrheit unterwiesen; ihr
  könnet dadurch die Kinder der Freunde in der Wahrheit erhalten
  durch die Weisheit, die von Gott kommt, und sie instand setzen
  % \picinclude{./200-209/p_s209.jpg} 
  den Ihrigen Stütze und Hilfe zu sein und sie in ihren alten Tagen
  zu erhalten. Auch werdet ihr, wenn diese Dinge solcherweise nach
  der Weisheit aus Gott geordnet sind, die beständigen Unterstützungen 
  aufheben und euch viel Verdrus ersparen. Ihr könnt
  ja bekanntlich auf dem Lande einen die verschiedensten Gewerbe
  erlernen lassen, wie Maurer, Schreiner, Wagner, Pflugschmied,
  Schneider, Gerber, Schmied, Schuhmacher, Nagelschmied, Metzger,
  und Weber in Leinen, Wolle, Stoff und Tuch.\index{Ausbildung}\index{Beruffe} 
  Ihr tut auch wohl daran, wenn ihr zu diesem Zweck eine Kasse in euren
  Versammlungen einrichtet. Alles, was Freunde bei ihrem Tode
  hinterlassen, wenn es nicht ausdrücklich einer andern Person
  oder Sache bestimmt ist, möge zu diesem Zweck in die allgemeine
  Kasse gehen.\index{Armenversorgung} Dadurch wird es möglich sein, viele Arme unter
  euch zu unterstützen und vielen armen Familien wieder auszuhelfen.
  In mehreren Grafschaften wird es schon so gemacht; einige 
  Vierteljahrsversammlungen schicken jeweilen zwei in eine Lehre, zuweilen
  auch Kinder, die der Gemeinde zur Last gefallen waren. Ihr
  könnt sie für eine beliebige Anzahl von Jahren binden, je nach
  ihren Fähigkeiten. In allem dem wird euch die Weisheit von
  Gott lehren, durch welche es uns gelingen möge, den Kindern
  armer Freunde zu ermöglichen, die Ihrigen zu unterstützen und in
  der Furcht Gottes zu bleiben. Ich schließe; meine Liebe im
  ewigen Samen, durch den ihr Weisheit empfangen werdet, um
  alles zur Ehre Gottes einzurichten."'

  \begin{flushright}
  London, 1. des 11. Monats 1669. 

  G. F 
  \end{flushright}

}

Wir zogen weiter und kamen nach Leicestershire;\ort{Leicestershire} anstatt
hier meine Frau zu treffen, hörte ich, das sie aus ihrem Hause
wieder ins Gefängnis von Lancaster geschleppt worden war,
auf einen Befehl des Königs und des Rats, sie wegen des
früheren Vergehens ins Gefängnis zurück zu bringen, obgleich
sie auf Befehl des Königs und Rats das Jahr vorher aus dieser
Gefangenschaft freigesprochen worden war. Darum kehrte ich nach
London zurück.

Sobald ich in London ankam, schickte ich Mary Lower\person{Lower, Mary} und
Sarah Fell,\person{Fell, Sarah} zwei Töchter meiner Frau, zum König, um ihm 
mitzuteilen, wie man ihre Mutter behandle, und zu sehen, ob sie
nicht eine völlige Lossprechung für sie erwirken könnten, damit sie
ihre Freiheit und ihren Besitz unbelästigt geniesen könne. Es
hatte einige Schwierigkeit; doch erreichten sie es schließlich durch