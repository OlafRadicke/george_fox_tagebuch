% \picinclude{./200-209/p_s200.jpg} 
200 Kapitel 171.
von dir Genannten, können in einzelnen Fällen schwören und tun
es- auch; wir dagegen schwören in keinem Fall. Wenn man jenen
ihre Kühe oder Pferde stehlen würde und du würdest sie fragen,
ob sie schwören wollen, daß ez die ihrigen seien, so wären manche
von ihnen gleich bereit, etz zu tun. Wenn du etz aber bei unsern
Freunden versuchst, so schwören sie auch um ihrer Habe willen
nicht. E Darum, wenn du von einem oon ihnen den Huldigungßeid
sorderst, so mußt du ihn fragen, ob er in andern Fällen schwören
kann, z. B., wenn ez seine Kuh oder sein Pferd betreffe, waö- er,
wenn er wirklich zu unß gehört, nicht kann.« Darauf erzählte
ich ihm solgendeß auö einem Verhör in Berkshire: ,,Gin Dieb stahl
einem unsrer Freunde zwei Stück Vieh; der Dieb wurde erwischt
und inß Gefängniß gebracht, und der Freund trat vor Gericht
alö Kläger gegen ihn aus. Da aber der Richter gehört hatte,
daß der Kläger ein Quäker sei und nicht schwören könne, so rief
er, ohne erst anzuhören, waß der Freund zu sagen hatte: »Jst
er Quäker, will er nicht schwören?« und legte ihm den Huldigungß-
eid oor. Und darauf schickte er den Freund ins Gefängniß und unter- .
warf ihn den Strafen gegen Gidoerweigerung. Der Dieb aber, der
ihn bestohlen hatte, wurde freigesprochen!« Richter Marsh sagte:
,,Dieser Richter war ein schlechter Mensch.« ,,Wenn wir irgend
schwören könnten,« sagte ich, ,,so wäre eß dem König, der zum
Schutz seiner Untertanen die Gesetze aufrecht erhält. Die andern
dagegen, die doch schwören, wenn etz gilt, ihr Hab und Gut zu
schützen, weigern sich gerade, dem König den Huldigungs?-eid zu
schwören. Von diesen kannst du unß also leicht kennen und unter-
scheiden.«
Richter Marsh hat später sehr viel für die Freunde getan,
indem er viele von ihnen vor den Strafen wegen Eidverweigerung
schätzte. Wenn während der Versolgungen Freunde vor ihn gebracht
wurden, so schenkte er ihnen, wenn möglich die Freiheit, und wenn —
er es nicht verhindern konnte, daß sie ins Gefängniß kamen, so
machte er, daß eß nicht für länger als ein paar Stunden oder
für eine Nacht sei. Schließlich ging er zum König und sagte
ihm, er habe einige von unß gegen sein Gewissen inß Gefängnis
geschickt und wolle daß sernerhin nicht mehr tun. Er verließ
darum mit seiner Familie Limehouse und zog in die Nähe oon
St. Jameß Park. Er sagte dem König, wenn er sich dazu ver-
stehen könnte, Gewissenöfreiheit zu erklären, so würde alles- ruhig


% \picinclude{./200-209/p_s201.jpg} 
Reise nach Jtland. Rückkehr und Heirat mit Margaret Fell usw. 201
werden; denn dann könnte man nicht mehr Gewissen?-skrupeln
geltend machen. Ja, er hat in jenen Tagen viel für die Wahr-
heit und die Freunde getan.
Kapitel Tcyll.
Reise nach Irland. Rückkehr nnd Heirat mit Margaret Fell.
Jhre abermalige Gesangennahme. Schwere innere Ansechtnngen.
Der Herr trieb mich, nach Jrland zu gehen um dort den
Samen Gotteö zu besuchen ..... Alß wir aus dem Schiff
waren, rief ich meinen Gefährten zu: ,,Lasset uns- fröhlich sein
im Herrn, denn wir werden gute Winde haben!« Viele waren
krank während der Überfahrt, aber niemand von den Unsrigen.
Der Kapitän und auch viele der Mitreisenden waren uns sehr
zugetan, und da wir an einem Ersten Tage auf dem Wasser
waren, trieb etz mich, die Wahrheit unter ihnen zu verkünden,
worauf der Kapitän zu den andern sagte: ,,DaS sind Dinge, von
denen ihr noch nie in eurem Leben gehört habt«. Vor Dublin
nahmen wir ein Boot und fuhren anß Land, und mir schien, daß
der Boden und die Luft von der Verdorbenheit dieser Nation
iibel rieche; wenigstenß empfand ich einen ganz andern Geruch als
in England, was ich den päpstlichen Greueltaten, die hier be-
gangen wurden, zusehrieb, und dem Blute, das hier vergossen
worden war und nun Fänlniö auzströmte ..... Wir fanden
nicht gleich Freunde und begaben unß darum nach einer Herberge
und ließen welche aufsuchen und zu unz kommen. Sie waren
alle sehr froh über unser Kommen, und empfingen unö mit großer
Freude. Wir blieben zur Wochenversammlung, die sehr zahlreich
war und gesegnet mit der Kraft und dem Leben aus- Gott .....
Darnach gingen wir zu einer Versammlung in der Provinz ....
und zogen dann nach einem andern Orte wo wir eine sehr schöne,
erbauliche Versammlung hatten, aber einige Papisten die ihr
beigewohnt hatten, waren nachher sehr zornig und wütend. Alß
ich dieö vernahm, ließ ich einen von ihnen zu mir kommen, einen
Schulmeister, aber er wollte nicht kommen. Darauf schickte ich
an ihn, sowie an alle Mönche, Klosterbrüder, Priester und Jesuiten
eine Aufforderung, ihren Gott und ihren Christuß, die sie aus
Brot und Wein gemacht, zu erproben; aber ich konnte keine
Antwort von ihnen erlangen. Tamm erklärte ich ihnen, sie seien


% \picinclude{./200-209/p_s202.jpg} 
202 Kapitel Iksll.
ärger alz die Baalzpriester: denn die Baalzpriester hätten ihren
hölzernen Gott erprobt, sie aber dürften nicht wagen, ihren Gott
auz Brot und Wein zu erproben, und die Baalzpriester und ihre
Anhänger hätten ihren Gott nicht gegessen wie sie und nachher
einen andern gemacht .....
Der damalige Bürgermeister oon Cork war den Freunden
und der Wahrheit übel gesiimt und hielt viele Freunde gefangen,
und weil er wußte, daß ich im Lande war, hatte er Befehle
erlassen, mich zu oerhasten, darum wollten die Freunde nicht,
daß ich durch Cork reiste. Aber alz ich in Bandon war, erschien
mir in einem Gesicht ein sehr häßlicher Mensch mit finsterem,
bösem Blick; mein Geist schlug nach ihm in der Kraft Gottez,
und ez war mir, alz ob ich über ihn weg ritte mit dem Pferde,
und daz Pferd den Fuß auf sein Gesicht setze. Alz ich am Morgen
hinunter kam, berichtete ich einem der Freunde, waz mir wider-
fahren, und daß dez Herrn Befehl an mich ergangen, durch Cork
zu reiten, aber ich bat ihn, ez niemand zu sagen. So ritt ich
mit vielen Freunden von dannen; alz wir nnz der Stadt näherten,
hätten sie mir gerne einen Weg hinten um die Stadt herum ge-
zeigt, aber ich sagte ihnen, mein Weg gehe durch die Straßen.
Ich nahm also einen von ihnen, er hieß Paul Morrice, mit mir,
um mir den Weg durch die Stadt zu zeigen und ritt hinein.
Alz wir über den Marktplatz ritten und am Hause dez Bürger-
meisterz Vorbei, sagte dieser, alz er mich ooriiberreiten sah: ,,da
geht George Fox vorbei«, aber er hatte nicht Macht, mich anzu-
halten. Alz wir die Wachen und die Brücke passiert hatten,
gingen wir zum Hause einez Freundez und stiegen ab. Hier
berichteten mir die Freunde, waz für eine Erbitterung in der
Stadt herrsche, und wie viele Befehle erlassen werden, um mich
gefangen zu nehmen. Während ich hier mit den Freunden zu-
sammensaß, spürte ich, wie der böse Geist am Werk war in der
Stadt, um Unheil gegen mich anzustisten, und ich spürte auch,
wie dez Herrn Kraft diesen bösen Geist schlug. Andere Freunde,
die nach und nach herein kamen, berichteten mir, daß in der
Stadt und unter den Behörden meine Anwesenheit bekannt ge-
worden sei. Ich sagte: ,,-Laßt den Teufel sein Äußerftez tun«.
Nachdem die Freunde sich gegenseitig gestärkt hatten und wir
Reisende unz auch gestärkt hatten, ließ ich mein Pferd holen und
ging mit einem Freunde, der Mich führte, meiner Wege. Aber


% \picinclude{./200-209/p_s203.jpg} 
Reise nach Jrland. Rückkehr und Heirat mit Margaret Fell usw. 203
der Zorn war groß bei den Behörden und den Leuten von Cork,
daß sie mich verfehlt hatten, und sie gaben sich in der Folge
große Mühe, mich zu erwischen, indem sie, wie ich hörte, überall
ihre Späher hatten, um zu forschen, welchen Weg ich gehe, und
in fast jeder öffentlichen Versammlung, der ich beiwohnte, kamen
Späher, um zu sehen, ob ich da sei. Die Behörden sandten ein-
ander Berichte über mich, in denen sie mich nach Haaren, Hut,
Kleidern und Pferd beschrieben, so daß man, alß ich schon hun-
dert Meilen von Cork weg war, Bericht und Beschreibung über
mich hatte, ehe ich ankam. Giner, der zu den Schlimmsten in
der Behörde gehörte, und zugleich Priester und Friedenßrichter
war, erhielt eine Vollmacht vom Richter, mich zu verhaften; dieser
Befehl erstreckte sich über seinen ganzen Bezirk, der mehr alß
hundert Meilen umfaßte. Aber der Herr machte alle ihre
Anschläge zu nichte und vereitelte alle ihre Vorhaben. Die treue
Hand seiner Vorsehung behütete mich vor allen ihren Fallstricken
und gab un?7 manche gesegnete Gelegenheit, Freunde zu besuchen
und die Wahrheit im Lande zu verbreiten. Die Versammlungen
waren sehr zahlreich, da die Freunde von nah und sem sie be-
suchten, und auch andere Leute herzu strömten. Die mächtige
Krast dez Herrn machte sich herrlich fühlbar mit und unter uns-;
dadurch wurden viele Weltlichgesinnte ergriffen und überzeugt
und für die Wahrheit gewonnen; die Herde des Herrn wuchs,
und die Freunde wurden erquickt und gestärkt durch daß Gefühl
der Liebe Gottetz. O, wie wurden sie ergriffen von den Strömen
dez Leben?-, so daß viele in der Kraft und dem Geist Gottez
miteinander in Singen auöbrachen, und dem Herrn spielten in
ihren Herzen!
Viele angesehene Personen kamen inö Haus von Jametz Hutchin-
son in Jrland, um mit mir über Erwählung und Verwersung zu
reden. Ich sagte ihnen: »Wenn ihr schon unsre Ansichten alö
verrückt verwerst: sie sind eben zu hoch für euch, ihr könnt sie
mit eurer Wei?-heit nicht verstehen, darum will ich mich in dem,
waö ich sage, nach eurem Verständnis richten. Jhr sagt, Gott
habe die meisten Menschen zur Hölle verdammt, und sie seien
dazu verordnet von Anbeginn der Welt, und bringt als Beweis
dafür den Judas?-brief. Jhr sagt, Esau sei verworfen gewesen
und die Ägypter und die Nachkommen des Ham. Aber Christus
sagt seinen Jüngern: »gehet hin und prediget allen Völkern« und


% \picinclude{./200-209/p_s204.jpg} 
204 Kapitel 1711.
,,gehet hin in alle Lande.« Wenn sie nun zu allen Völkern gehen
mußten, mußten sie denn dann nicht auch zu den Nachkommen
Esauß und Hamö gehen? Jst nicht Christus für alle gestorben,
also auch für die Nachkommen Esauß, Hamß und der Ägypter?
Sagt nicht die Schrift, Gott will, daß allen Menschen geholfen werde?
Merket wohl: allen Menschen, also auch den Nachkommen Esauß
und Hamtz. Sagt nicht Gott: ,,Agypten mein Volk ?« (Jes. 19, 25)
und, daß er einen Altar in Agypten haben wolle? (Jes. 19,19).
Waren nicht viele Christen früher in Agypten? Und berichtet
nicht die Geschichte, daß der Bischof von Lllexandrien Papst ge-
wesen ist? Und hat nicht Gott eine Kirche in Babylonien? Jch
gebe zu: ,,daS Wort geschah zu Jakob und daß Recht an JZrael«
(Ps. 147, 19); solches:3 kam den andem Nationen nicht zu
denn daß Gesetz Gotteß war nur J-?-rael gegeben, das- Evangelium
jedoch sollte allen Völkern gepredigt werden und soll etz noch.
Für alle Menschen gilt die gute Botschaft deß Friedens: ,,wer
da glaubet, der wird selig, wer aber nicht glaubt, ist schon
gerichtet« (Mark. 16, 16). Die Verdammung kommt also durch
den Unglauben. Und wenn Judaß von etlichen sagt, sie seien
vor Zeiten zur Verdammung bestimmt, so sagt er nicht vor Anbe-
ginn der Welt, sondern ,,geschrieben vor Zeiten«, waß sich auf
die Schrift Moseß beziehen kann, in welcher von denen geschrieben
steht, die Juda erwähnt, nämlich Kain, Korah, Bileam und die
Engel, die ihr Fürstentum nicht behielten. Die Christen nun,
welche solchem Wandel nachsolgen und abgefallen sind vom Stand
der ersten Christenheit, waren und sind zur Verdammung bestimmt
durch das Licht und die Wahrheit, davon sie abgefallen sind.
Und obgleich der Apostel sagt: ,,Gott liebte Jakob und haßte
Esau« (Röm.9, 13), so erinnert er doch die Gläubigen, ,,wir waren
alle Kinder des Zornß von Natur, gleich wie die andern« (Eph.2,3).
Diez schließt auch den Stamm Jakobs ein, welchem der Apostel
selber und alle gläubigen Juden angehörten. Die Juden wie die
Heiden standen also unter der Sünde und somit unter der Ver-
dammung, damit Gott sich oller erbamie in Jesuß Christuß
Die Erwählung stehet bei Christuß, und ,,wer da glaubt, wird
selig« und ,,wer nicht glaubt, ist schon verdammt«. Jakob reprä-
sentiert die zweite Geburt, die Gott liebt; und sowohl Juden
wie Heiden müssen wiedergeboren werden, ehe sie inö Reich Gottes-
eingehen können. Wenn ihr wiedergeboren sein werdet, so werdet


% \picinclude{./200-209/p_s205.jpg} 
Reise nach Jrland. Rückkehr und Heirat mit Margaret Fell usw. 205
ihr verstehen, waö Erwählung und Verwerfung bedeutet, denn
die Grwählung stehet in C-hristuö, dem Samen, der ,,gewesen,
ehe der Welt Grund gelegt war«; die Verwerfung dagegen liegt
im schlechten Samen, welcher erst nach der Erschaffung der Welt
enistand.« In dieser Weise, nur etwaö außsührlicher, redete ich
mit diesen Leuten, und sie gestanden, daß sie dergleichen noch
nie gehört hätten.
Nachdem ich in Jrland umher gereist war und die Freunde
in ihren Versammlungen besucht hatte, sowohl in geschäftlichen
Angelegenheiten alß um mit ihnen Gotteödienst zu halten, und
verschiedene Schreiben von Mönchen, Klosterbrüdern und prote-
stantischen Priestern beantwortet hatte, (denn sie waren alle
wütend über una und suchten daß Werk dez Herrn zu hindern,
und wir hörten, wie einige Jesuiten schworen, wemi wir auch
kämen um unsere Jdeen im Lande zu verbreiten, so solle eß unz
nicht gelingen), kehrte ich nach Dublin zurück, um mich nach England
einzuschisfen .....
E-Z sind in Jrland gute, tüchtige und aufrichtige Menschen,
die die Krast Gottes spüren und empfänglich sind für die Wahr-
heit, und sie halten gute Ordnung in ihren Versammlungen,
denn sie stehen für Heiligkeit und Gerechtigkeit ein, welche den
Weg der Schlechtigkeit oersperren. Jch könnte noch viel über
die Leute dieseß Landeß schreiben und über meine Reisen unter
ihnen, maß zu weit führen würde; dieses aber erwähnte ich gern,
damit der Gerechte sich an dem Gedeihen der Wahrheit freuen
moge.
Jameß Lancaster, Robert Briggß und Robert Lodge kehrten
mit mir zurück. John Stubbö, der noch zu tun hatte, blieb zurück.
Wir waren zwei Nächte aus dem Wasser; in der einen Nacht
brach ein starker Sturm auö, der das Schiff in große Gefahr
brachte. Aber ich sah, daß Gottez Macht größer war alß Wind
und Sturm. Gr hielt sie in seiner Hand, und seine Macht bändigte
sie. Die gleiche Macht des Herrn, welche uns hinüber gebracht
hatte, brachte unß wieder zurück, und sein Leben hatte uns Herrschaft
gegeben über alle bösen Geister, welche uns dort entgegen gewesen
waren. Wir landeten in Liverpool. Danach gingen wir nach
Gloucerstershire, wo wir ein Gerücht vernahmen, das sich in der
Gegend verbreitet hatte: George Fox sei Preöbyterianer geworden,
man habe eine Kanzel für ihn errichtet im Freien, und es würden


% \picinclude{./200-209/p_s206.jpg} 
206 Kapitel Icllll.
am folgenden Tage Tausende von Menschen kommen, um ihn zu
hören. Jch wunderte mich, wie ein solches Gerücht sich über
mich verbreiten konnte. Wir hörten jedoch beim nächsten
Freund, zu dem wir kamen, ebenfalls daoon. Wir sahen im
Vorbeigehen die Kanzel auf dem Felde stehen; dann zogen wir
weiter an den Ort, wo am folgenden Tage die Versammlung der
Freunde stattfinden sollte, und blieben dort über Nacht. Am sol-
genden Tage, es war der Erste Tag, hatten wir eine sehr große
Versammlung, und des Herm Kraft und Gegenwart war unter
uns. Die Veranlassung zu jenem Gerücht war, wie ich hörte,
folgende gewesen. Einer namens John Fox, ein Presbyteriancr-
Priester, reiste herum und predigte und es hieß, einige hätten
statt John George gesagt und ausgestreut, George Fox habe
seinen Glauben geändert und sei aus einem Quäker ein Presby-
terianer geworden und werde an dem und dem Tage an dem
und dem Orte predigen. Daraufhin entstand eine solche Neu-
gierde unter den Leuten, daß viele, welche niemals gegangen
wären, John Fox zu hören, liefen, um diesen presbyterianisch
gewordenen Quäker zu hören. Aus diese Weise, hieß es, hätten
sie Tausende von Menschen zusammengebracht. Als sie aber kamen
und sahen, daß man ihnen einen Streich gespielt hatte, daß es
nur ein falscher George Fox sei, und erfuhren, der rechte George
Fox sei ganz in der Nähe, kamen Hunderte von ihnen in unsere
Versammlung und waren sehr aufmerksam und ruhig. Jch wies
sie aus die Gnade Gottes, die in ihnen sei, welche sie lehren und
ihnen das Heil bringen könne. Nach der Versammlung sagten
einige, sie hörten den Quäker George Fox lieber predigen als den
Presbyterianer George Fox. So war durch mein prooidentielles
Erscheinen in dieser Gegend, gerade zu dieser Zeit, dieses falsche
Gerücht entdeckt worden und seine Urheber wurden zu Schanden.
Bald darauf wurde John Fox im Unterhause angeklagt, er
halte stiirmische Versammlungen, in welchen oerräterische Worte
geredet würden. Damit verhielt es sich, nach dem, was ich darüber
erfahren konnte, also: Er war früher Priester in Mansfield in
Wiltshire gewesen, und als er von dort fortgeschickt worden war,
hatte ihm ein Common-Prayer Priester erlaubt, manchmal in
seinem Turmhaus zu predigen. Schließlich wurde aber dieser
Presbyterianer-Priester kecker als erwünscht war und machte
allzu sehr von der ihm gegebenen Erlaubnis Gebrauch und oer-


% \picinclude{./200-209/p_s207.jpg} 
Reise nach Jrland. Rückkehr und Heirat mit Margaret Fell usw. 207
suchte dort zu predigen, ob der Orttzpfarrer etz gern hatte oder
nicht. Dietz führte zu viel Zwist und Reibereien zwischen den
beiden Priestern und ihren Hörern, wobei sogar daß Common-
Prayer Buch zerschnitten wurde und ez fielen einige verräterische
Worte bei einigen der Anhänger des John Fox. Dietz wurde
sogleich bekannt, und einige bößwillige Preßbyterianer verbreiteten
die Sache so, als- ob sie von George Fox, dem Quäker, herrührten,
obgleich ich etwa zweihundert Meilen weit weg war. Alß ich
davon hörte, verschaffte ich mir sofort eine Bescheinigung von
einem Mitglied deS Unterhauseö, daß diesen John Fox kannte,
und ließ verbreiten, daß eß sich um John Fox, den früheren
Priester von Mans?-field handle ..... Dieser John Fox erwieß
sich auch später alß ein schlechter Kerl. ....
Wir zogen weiter bis wir nach Bristol kamen, wo ich Margaret
Fell traf, die gekommen war, um ihre Tochter Yeomanß zu be-
suchen. Der Herr hatte mir schon vor längerer Zeit gezeigt, daß
ich Margaret Fell zur Frau nehmen solle. Und als ich daß erste
mal mit ihr redete, kam ihr die Antwort darauf von oben. Aber
obgleich mir der Herr solches eröffnet hatte, so hatte ich doch
damalß noch keinen Befehl von ihm erhalten, ez auzzuführen.
Darum ließ ich die Sache ruhen und suhr wie biöher fort in der
Arbeit und dem Dienst des Herrn, wie er mich führte, hier im
Lande wie in Jrland. A16 ich nun aber in Bristol war und
Margaret Fell da traf, ossenbarte mir der Herr, daß die Sache
nun außgesührt werden müsse. Nachdem wir miteinander darüber
geredet, sagte ich ihr, wenn es ihr auch recht sei, ez jetzt zu tun,
so solle sie zuerst ihre Kinder kommen lassen, maß sie auch tat.
Alß alle ihre Töchter beisammen waren, so fragte ich dieselben
sowie auch die Schwiegersöhne, ob sie irgend etwas dafür oder
dawider hätten? Sie sprachen alle einmütig ihre Zufriedenheit
darüber auö. Darauf fragt ich Margaret, ob sie den Willen ihreß
Gatten den Kindern gegenüber erfüllt und außgeführt habe?
Sie erwiderte, .,die Kinder wissen, daß ich’Z tat«. Darauf fragte
ich die Kinder, ob ez nicht ein Verlust für sie bedeute, wenn
ihre Mutter wieder heirate? Und Margaret fragte ich, ob sie
ihren Kindern irgend welche Bürgschaft dafür geleistet habe?
Die Kinder sagten, sie habe es getan, und ich möchte nicht mehr
davon sprechen. Jch sagte ihnen, ich sei ein Mann der Geradheit
und möchte, daß alles offen zugehe, denn ich suche keinerlei


% \picinclude{./200-209/p_s208.jpg} 
208 Kapitel 1711.
äußeren Vorteil für mich bei dieser Sache. Nachdem ich nun
den Kindern die Sache vorgelegt hatte, wurde unsere Absicht,
unß zu ehelichesn, vor die Freunde gebracht, sowohl einzeln als-
öffentlich; alle waren sehr einverstanden, und viele von ihnen
bezeugten, da?-kommevomHerrn. Daraus wurde eine Versammlung
veranstaltet im Broad-Mead=VersammlungShauZ in Bristol, damit
es vollzogen werde, und wir nahmen einander, indem der Herr
unß verband in der ehrbaren Ehe, in dem ewigen Bund und
dem unsterblichen Samen. Unter dem Eindruck davon legten
manche der Freunde lebendigeß und ergreifendeß Zeugniö ab,
getrieben von der himmlischen Kraft, die uns miteinander verband.
Darauf wurde ein Au?-weitz über die Verhandlungen und über
die Eheschließung öffentlich verlesen und unterzeichnet von den
Verwandten, den meisten der ältesten Freunde der Stadt sowie
von vielen andern aus verschiedenen andern Teilen dez? Landes .....
A16 ich wieder in London war, trieb etz mich, den Freunden
im ganzen Lande darüber zu schreiben, daß man arme Kinder
bei Handwerkern gegen Arbeit unterbringen solle. Jch sandte
darum an die Vierteljahrßversammlung jeder Grafschaft folgenden
Brief:
,,Meine lieben Freunde,
Jede Vierteljahr?-versammlung soll sich bei allen monatlichen
und andern Versammlungen erkundigen nach allen Witwen und
andern unter den Freunden, die Kinder haben, welche fähig
wären, ein Handwerk zu erlernen, so daß man jedes Vierteljahr
von der Vierteljahrßversammlung auö einen in die Lehre schicken
kann; so können jährlich vier in jeder Grafschaft außgeschickt
werden oder auch mehr, wenn sich die Gelegenheit bietet. Diese
Lehrlinge können dann, wenn sie au?-gelernt haben, den Eltern
helsen und der Familie, wenn sie heruntergekommen ist, wieder
aufhelfen, so daß alle nach und nach behaglich leben können.
Wenn ihr dietz in euren vierteljährlichen Versammlungen tut, so
werdet ihr in den monatlichen Versammlungen von den geeigneten
Meistern in der Grafschaft hören und von Gewerben, wie die
Eltern und ihr sie wünscht, oder für die die Kinder besondere
Neigung haben. Wenn sie in der Weise bei Freunden unter-
gebracht sind, so werden sie in der Wahrheit unterwiesen; ihr
könnet dadurch die Kinder der Freunde in der Wahrheit erhalten
durch die Weiß-heit, die von Gott kommt, und sie instand setzen


% \picinclude{./200-209/p_s209.jpg} 
Reife nach Ji-land. Rückkehr und Heirat mit Margaret Fell usw. 209 1
den Jhrigen Stütze und Hilfe zu fein und sie in ihren alten Tagen
zu erhalten. Auch werdet ihr, wenn diese Dinge solcherweise nach
der Weisheit aus Gott geordnet sind, die beständigen Unter-
stützungen aufheben und euch viel Verdrnß ersparen. Jhr könnt
ja bekanntlich auf dem Lande einen die verschiedensten Gewerbe
erlernen lassen, wie Maurer, Schreiner, Wagner, Pflugfchmied,
Schneider, Gerber, Schmied, Schuhmacher, Nagelschmied, Metzger,
und Weber in Leinen, Wolle, Stoff und Tuch. Jhr tut auch
wohl daran, wenn ihr zu diesem Zweck eine Kasse in euren
Versammlungen einrichtet. Alles, was Freunde bei ihrem Tode
hinterlassen, wenn es nicht ausdrücklich einer andern Person
oder Sache bestimmt ist, möge zu diesem Zweck in die allgemeine
Kasse gehen. Dadurch wird es möglich sein, viele Arme unter
euch zu unterstützen und vielen armen Familien wieder auszuhelfen.
Jn mehreren Grasschasten wird es schon so gemacht; einige Viertel-
jahrsversammlungen schicken jeweilen zwei in eine Lehre, zuweilen
auch Kinder, die der Gemeinde zur Last gefallen waren. Jhr
könnt sie für eine beliebige Anzahl von Jahren binden, je nach
ihren Fähigkeiten. Jn allem dem wird euch die Weisheit von
Gott lehren, durch welche es uns gelingen möge, den Kindern
armer Freunde zu ermöglichen, die Jhrigen zu unterstützen und in
der Furcht Gottes zu bleiben. Jch schließe; meine Liebe im
ewigen Samen, durch den ihr Weisheit empfangen werdet, um
alles zur Ehre Gottes einzurichten.«
London, 1. des 11. Monats 1669. G. F .....
Wir zogen weiter und kamen nach Leieestershire; anstatt
hier meine Frau zu treffen, hörte ich, daß sie aus ihrem Hause
wieder ins Gefängnis von Lancaster geschleppt worden war,
aus einen Befehl des Königs und des Rats, sie wegen des
früheren Vergehens ins Gefängnis zurück zu bringen, obgleich
sie auf Befehl des Königs und Rats das Jahr vorher aus dieser
Gefangenschaft freigesprochen worden war. Darum kehrte ich nach
London zurück.
Sobald ich in London ankam, schickte ich Mary Lower und
Sarah Fell, zwei Töchter meiner Frau, zum König, um ihm mit-
zuteilen, wie man ihre Mutter behandle, und zu sehen, ob sie
nicht eine völlige Lossprechung für sie erwirken könnten, damit sie
ihre Freiheit und ihren Besitz unbeläftigt sgenießen könne. Gs
hatte einige Schwierigkeit; doch erreichten sie es schließlich durch
George For. 14

