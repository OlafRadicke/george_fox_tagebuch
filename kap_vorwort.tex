% \picinclude{./vorwort/p_v01.jpg}
George Fox.
Aufzeichnungen und Briefe dez ersten Quäkerz.
« Jn Außwahlüberfetztvon
Marg. stähelin.
Mit einer Einführung von
V Professor 1). Paul Wernle.
M
P.-Iizgiöze czezzelizcbczsss (ier l-reuncso
(HDMI-LCS-:)
S S s Z i si N W 7
I’rin:—i.Oui5—’s:c-iscjZcjoncsssir. Z
Tübing en.
Verlag von J. E. B. Mohr (Paul Siebeck).
1908.


% \picinclude{./vorwort/p_v02.jpg}

Alle Rechte vorbehalten.


% \picinclude{./vorwort/p_v03.jpg}

Jnhaltzverzetchntz.
I: Seite
Zur Einführung. Von Professor 1). Paul Wernle ....... 7
Kap. 1. Erweckung und Krisis bis zum Durchbruch ..... 1
Kap. ll. Erste Versammlungen und Proteste ........ 16
Kap. lll. Der. Tumult in Nottingham. Wachsender Widerstand,
bis szum Gefängnis in Derby .......... 26
Kap. 17. Erlebnisse im Gefängnis zu Derby. Ein ,,Wehe« über
die Stadt Lichfield. Erste Missionsgenossen. Antikirch-
liche Agitation und Kampf gegen die Ranter .... 35
Kap. 7. Christus in uns. Erkenntnis der Quäkerischen Welt-
mission. Das Haus Richter Fells in Swarthmore. Der
Pöbel von Ulverstone. Rechtfertigung vor dem Gericht
in Lancaster ................ 56
Kap. 71. Fox der Hexerei verdächtigt. Falsche Osfenbarungen bei
Freunden. Gefangenschaft in Earlisle ....... 71
Kap. 711. Kämpfe mit schwärmerischen Rantern und zehntengierigen
Priestern. Fox in Wetstone verhaftet und vor Cromwell
geschickt .........   ......... 83
Kap. 7111. Brief an den Papst. Die Studenten von Cambrigde.
Die Quäker in der Bibel. Wachsende Entfremdung von
Cromwell ................. 96
Kap. llc. Angriffe der Jndependenten und Presbyterianer. Ahnungen,
Heilungen, Bekehrungen. Dispute über Taufe und Er-
wählung. Gefangennahme auf Grund angeblicher Ver-
schwörung. Wirken während der Gefangenschaft . . . 104
Kap. 1. Warnung an die Kegelspieler. Naylors Fall. Dis-put
mit Paul Gwin. Besuch bei Eromwell. Herumreisen bei
den gefangenen Freunden. Reise in Wales ..... 120
Kap. I1. Reise nach Schottland. Kampf gegen die Prädestinations-
lehre und Widerstand der schottischen Geistlichkeit . . . 128
Kap. Ill. Erste Jahresversammlung. Warnung an Cromwell vor
der Königskrone. Trostbrief an dessen Tochter. Gesichte
vom Tode Cromwells und der kommenden Reaktion . . 134
Kap. 1111. Ein Gottesgericht. Ermahnung zur Barmherzigkeit bei
Schissbrüchen. Quäkersreundlicher Erlaß des General
Monk. Fox als Königsfeind gefangen und schließlich .
auf Befehl Karls ll. befreit .......... 145


% \picinclude{./vorwort/p_v04.jpg}

17 Jnhaltsverzeichnis.
Kap. 117. Beginn neuer Quäkerverfolgungen bei Anlaß der Ver-
schwörung der Fifthmonarchy -Leute. Des Q-uäkers
John Perots Verirrungen. O-uäker mißhandelt in Neu-
England und Malta .... . ........ 154
Kap. 17. Ein Gottesgericht. Verhaftung wegen .angeblicher Ver-
schwörung und schreckliche Gefangenschaft in Lancaster und .
Scarbro. Dispute im Gefängnis mit Baptisten und andern.
Fox sieht den Brand von London voraus ...... 169
Kap. 171. Einrichtung der Monatsversammlungen. Regelung der
Osuäkerehen. Gründung von Knaben- und Mädchenschulen.
Reformation des Quäkertums .......... 188
Kap. 1711. Reise nach Jrland. Rückkehr, und Heirat mit Margaret
Fell. Jhre abermalige Gefangennahme. Schwere innere
Anfechtungen ................ 201
Kap. )c71l1. Reise nach Amerika. Barbadoes. Jamaika ..... 215
Kap. X11. Arbeit in Nordamerika unter Engländern und Jndianern 224
Kap. IX. Ankunft in Bristol. Zusammentreffen mit William Penn
und andern. Verteidigung der Frauenversammlungen.
Vorgeahnte Gefangenschaft in Worcefter. Brief an den
König über die Grundsätze der Quäker. Krankheit. Be-
freiung. Während der Gefangenschaft verfaßte Schriften 235
Kap. 111. Fox sammelt und ordnet die Bücher und Schriften, die er
geschrieben, und tritt für die Frauenversammlungen ein . 243
Kap. JTI11. Reife nach Holland. Einrichtung der kirchlichen Ordnung
für Holland und Deutschland. Briefwechsel mit Prinzessin
Elisabeth. Reise nach Deutschland bis Oldenburg. Briefe
an verschiedene Behörden von Holland und Deutschland 251
Kap. Jllllll. Rückkehr nach England. Kampf der Ordnungspartei gegen
die unbotmiißigen Quäker. Briefe über Toleranz an den
König von Polen, den Großmogul und andere .... 266
Kap. ZR17. Allerlei Mahn- und Troftschreiben ........ 281
Kap. D17. Zweite Reise nach Holland. Brief an den Herzog von *
Holstein zur Verteidigung des öffentlichen Redens der
Frauen ...... . ........... 285
Kap. II71. Kampf für die Ordnung im Quäkertum. Jakobsll.Amneftie 298
Kap. )T?(71l. Wirken in London unter dem Zeichen der Toleranz . . 303
Kap. II7111. Ahnung kommender Revolutionen. Christus König. Letzte
—— Arbeiten. Krankheit und Tod ......... 306
Anhang. Ein Brief von George Fox nach seinem Tode vorgefunden mit
der Aufschrift: Nicht vor der Zeit zu öffnen ..... 319
Zeittafel ......... . ............. 322
Berichtigungen .................... 324



% \picinclude{./vorwort/p_v05.jpg} 

zur Einführung.
Der Mann, der aus den folgenden Auszeichnungen zu uns
redet, ist schon von seinen Zeitgenossen als ein Rätsel und Ge-
heimnis angestaunt worden. Den einen erschien er als ein Mann,
der gewaltig predigte und nicht wie die Schriftgelehrten, den andern
als ein Verrückter, der selber andere verhexen könne. Man er-
zählte von ihm, er schlafe in keinem Bett, er könne fliegen, er
könne nicht ertrinken und man könne ihn nicht bluten machen,
weil er ein Zauberer sei. Seine magische Wirkung auf die
Zuhörer erklärten sich die einen daraus, daß er Flaschen bei sich
trage und den Leuten daraus zu trinken gebe, damit sie ihm
nachfolgten, andere meinten, er lege den Leuten Bänder um den Arm.
Selbst auf die Tiere gehe eine Kraft von ihm aus: die Hunde
mucksen nicht gegen die Quäker. Den Menschen sehe er den
Teufel im Gesicht geschrieben; ,,durchbohre mich nicht so mit
deinen Augen, wende deine Augen ab von mir«, rief ihm ein
streitsüchtiger Täufer zu. Anders wirkte er auf eine Frau in
Beverly durch eine kurze Ansprache in der Kirche daselbst; sie
erzählte nachher, ein Engel oder ein Geist sei in die Kirche ge-
kommen und habe herrliche Dinge von Gott geredet zur Ver-
wunderung aller Anwesenden, und als er geendet habe, sei er
verschwunden, sie wisse nicht, woher er gekommen noch, wohin er
gegangen sei.
Gin Jahrhundert später hat ihn Voltaire mit Jesus ver-
glichen. Der Vergleich war im Sinne einer Herabsetzung Jesu
gemeint. Voltaire glaubte, daß der englische Kanzelredner Tillotson
unendlich geschmackvoller als Jesus gepredigt habe; um nun die
richtige Analogie für das Bildungs-niveau Jesu zu finden, verglich
er ihn mit einem ungebildeten Schwärmer und Narren aus der
neuern Zeit, mit George Fox.



% \picinclude{./vorwort/p_v06.jpg} 


71 Zur Einführung.
Unter allem, was Voltaire von Jesnö zu sagen weiß, ist doch
dieser Vergleich mit Fox fast daö Beste. Allerdingz wird sich bei
jedem genaueren Zusammenschauen beider die handgreisliche
Uberlegenheit Jesu aufdrängen müssen, aber die Analogien sind
zahlreich und überraschend genug. E8 sind beideö Laien, die auf
Grund einer unmittelbaren inneren Berufung und Erleuchtung
sich getrieben fühlen, eine neue Weise, wie man Gott dienen soll,
zu verkünden, im Gegensatz zu allem, maß gerade von den Frommen
ihrer Zeit als göttlich autzgegeben wurde, heißen sie nun Pharisäer
oder Puritaner. Auch die wunderbaren Begleiterscheinungen haben
sie gemein; Heilkräfte gehen von ihnen aus, selbst auf Sterbende,
die von den Arzten ausgegeben sind, Vorahnungen und Gesichte
scheinen sie über den Zeitverlauf zu erheben, manche Antworten
und Weisungen gibt ihnen direkt der Geist, während sie bei an-
deren Gelegenheiten durch die Selbsteoidenz ihreß gesunden
Menschenoerstandetz überraschen. Nächst den Evangelien ist es
besonderß die Erzählung der Apostelgeschichte, an die man durch
Fox erinnert wird. Das beim Gebet erbebende Versammlungs-
hauö in Jerusalem, die Steinigung und Wiederbelebung des
Paulus in Lystra. die Gesängniöszene in Philippi mit dem
Kerkermeister, die Seefahrt nach Rom mit der Angst der Schiffs-
leute und der göttlichen Zuoersicht deö Apostel?-, alleö daß
wiederholt sich im Leben zdes Fox mit wenig veränderten Um-
ständen. Man glaubt, in die Tage des- UrchristentumZ zurück-
versetzt zu sein, nur mit dem Unterschied, daß, maß dort in der
Regel erst nach Jahrzehnten durch sekundäre Berichterstatter
schriftlich aufgezeichnet wurde, hier in einer eigenhändigen
Niederschrift des Manne?-, der all daß erlebt hat, unö entgegen-
tritt. Man darf hoffen, daß die neuteftamentlichen Gxegeten sich
künftig diesen Laienkommentar zu den Erlebnissen Jesu und der
Apostel nicht entgehen lassen, nicht im Jnteresse einer klein-
gläubigen Apologetik, sondern um ihre Einsicht in das, was in
einer enthusiastischen Zeit bei einem »Mann Gottes'' möglich ist,
zu erweitern und mehr Leben und Farbe der Wirklichkeit in ihre
oft so erstaunlich dürftige Auzlegerphantasie zu bekommen.
Aber nicht nur sür den biblischen Auöleger, für jeden Re-
ligion-Jforscher muß diese Quäkerselbstbiographie von höchster
Anziehungßkrast sein. Die noch junge Wissenschaft der Religions-
psychologie findet hier eineö ihrer allerinstruktiosten Dokumente.


% \picinclude{./vorwort/p_v07.jpg} 
Zur Einsührung. 711
Was unsere heutige Religionsforschung vor den früheren Zeiten
voraus hat, das ist ja eben die Wendung zu den ursprünglichen
religiösen Erlebnissen, während die frühere Forschung allzulange
sich bei der nachträglichen Verarbeitung dieser Erlebnisse in Dog-
men und Systemen aufgehalten hatte. Wir Theologen erkennen
heute, daß es für uns nichts Wichtigeres gibt, als aus die Personen
in der Geschichte zu lauschen, die Gott gehört und gesehen haben,
in denen also, wie der technische Ausdruck heißt, Religion aus
erster Hand uns vorliegt. Die schönste, sruchtbarste Religions-
psychologie der Gegenwart, William James ,,Religiöse Gr-
fahrung in ihrer Mannigsaltigkeitch hat ihren Wert darin, daß
sie den Zeugnissen aus erster Hand möglichst unvoreingenommen
nachgegangen ist. Zu ihnen gehört als eines der merkwürdigsten
eben das ,,Journal« des George Fox.
Man kann hier studieren, «wie die Bekehrung bei einem
solchen Mann Gottes vorgegangen ist. Alle ihre Vorbedingungen
läßt er uns erkennen, die Abstammung, das Milieu, den moralischen
Habitus vor der religiösen Krisis; nur eins, nicht das Umvichtigste,
sehlt in seinen Grinnerungen: die enthusiastische Zeit mit ihren
unerhörten weltgeschichtlichen und kirchlichen Umwälzungen, in die
Weingartens ,,Revolutionskirchen Englands-« immer noch die
klassische Einführung sind. Dann verfolge man den ,,Durchbruch«
selbst mit seiner ganze Jahre ausfüllenden Langsamkeit, dem
Wechsel der Seligkeitsgefühle mit den surchtbarsten anhaltendsten
Depressionen, den vielen pathologischen Begleiterscheinungen bis
zu dem Höhepunkt der Krisis, da Fox 14 Tage lang wie tot
daliegt, so verändert in Aussehen und Gestalt, als ob sein Körper
neu gebildet oder verwandelt wäre. Und dann als Folge das
souveräne Bewußtsein göttlicher Grwählung und Sendung, das
ihn keinen Augenblick in der Seligkeit der Gottesliebe ausruhen
läßt, sondern sofort ihn zu den Brüdern treibt, nicht um sie zu
bekehren, sondern um das schlummernde Bewußtsein des Gottes-
aeistes und seiner Kraft auch in ihnen zu wecken. Sein ganzes
Leben lang geht ihm dies Pneumatische nach, das- die Psychis
aker so gern in ihre Domäne ziehen möchten, während es für
ihn selber der Geist Gottes gewesen ist: plötzliche Stimmen, Ge-
stchte und Gefühle, Heilungen und Bewahrungen der mannig-
sachsten Art. Die Berichte darüber sind erstklassig wegen der er-
staunlichen Schlichtheit und Ausrichtigkeit dieses Berichterstatters,


% \picinclude{./vorwort/p_v08.jpg} 
7lll Zur Einführung.
der seine Wunder so natürlich erzählt, daß wir sie zu verstehen
glauben, und seine Gefichte und ihre Deutung, resp. Erfüllung
so auseinanderhält, daß er uns oft die Mittel der Kritik selber
in die Hand gibt. Nur davor darf vielleicht gewarnt werden,
sich zu einseitig auf die Sammlung dieser außerordentlichen
pneumatischen Erlebnisse zu beschränken. Das königliche Gott-
oertrauen in all den rasenden E-xzessen des englischen Pöbels,
in den schauerlichen Kerkern des damaligen Englands, in See-
sturm und Seeräubergefahr, in den Wäldern und Sümpsen
Nordamerikas breitet das Wunder über sein alltägliches Leben
aus. Dieser Mann scheint aus anderem Stoss zu sein und andere
Kräfte in sich zu tragen, als wir andere Menschen, wir verstehen,
daß man ihn für einen Zauberer hielt, wenn nicht so manche
Krankheiten, Hemmungen und Versuchungen rms wieder daran
erinnern würden, daß auch er ein Mensch gewesen ist.
Aber mir ist, als sehe ich ihn schon lange mit merkbarem Grimm
seine Erregung darüber bemeistern, daß er für uns eine historische
Merkwürdigkeit, ein religionspsychologisches Objekt geworden sei.
Soll das der ganze Wert meines göttlichen Auftrags gewesen
sein, euch interessanten Stoss für eure sogenannte Wissenschaft zu
geben? dazu mein Wahrheitszeugnis, meine Kämpfe und namen-
losen Leiden, meine Sammlung der Kinder Gottes in aller
Welt, damit ihr subtile psychologische und psychiatrische Unter-
suchungen an mir anstellen könnt? An das Licht und an den
Samen Gottes in euch appelliere ich: behandelt den lebendigen
Geist Gottes nicht wie einen Toten!
Welches ist der Platz des George Fox und seiner Quäker
in der Geschichte gewesen? Jndem wir das in Kürze feststellen,
wird deutlich, ob der Mami uns noch heute etwas zu sagen hat.
Weingarten hat einmal treffend das Quäkertum die geist-
liche Nachhut des Enthusiasmus der englischen Revolutionszeit
genannt. Ungeheure Bewegungen sind ihm vorangegangen, auf
denen es fußt, deren Gewinn es voraussetzt. Fox hat gut die
unpolitische neutestamentliche Ethik der Wehrlosigkeit und absoluten
Friedlichkeit predigen, nachdem zuvor der alttestamentliche Pari-
tanismus in einer gewaltigen kriegerischen Erhebung England
zur Vormacht des Protestantismus erhob; ohne das Heldentum
des Schwertes kein Raum für sein stilles, friedliches Heldentum.
Und ebenso hat Fox gut am Beispiel der puritanischen Revo-


% \picinclude{./vorwort/p_v09.jpg} 
Zur Einführung. 11
lutionßkirchen seine Kirchenkritik zu Ende denken, nachdem zuvor
der puritanische Kirchensturm daö prunkvolle Gebäude der anglis
kanischen Staatökirche mit ihrem ss- katholischen Apparat hinweg-
gefegt hatte; ohne die gewaltigen kirchlichen Reformationen und
Reduktionen der Puritaner keine Möglichkeit seine-3 antikirchlichen
Radikalißmuö. Auch der Gnthustaömuö, das Lauschen aus die
Stimmen des gegenwärtigen Gottes-geisteß, ist vor ihm in England
aufgetreten und hat seinen eigenen Enthusiaömuß angesteckt. Man
lernt auö seinen Auszeichnungen die Puritaner fast nur nach ihren
schlechten Seiten kennen; und doch ist der ganze Fox und sein
Quäkertum nur denkbar auf der Grundlage des puritanischen
Befreiung?-kampfß.
ES bleibt darum doch denkwürdig, daß es zu einem so
scharfen Gegensatz zwischen Fox und den Puritanerkirchen der
Preöbyterianer, Jndependenten und Täuser gekommen ist. Was
ist der Grund dieseß Kampfeß?
Die Puritanerkirchen erhohen sich alle auf objektiver, historischer
Grundlage. Daß historische Erlösung?-werk Christi war für sie
alle der Grund der Seligkeit und darum stand der Glaube, das
Bekenntniö, an der Spitze ihreß Christentum?-. Jin Ernstmachen
mit der absoluten Autorität der Bibel suchte jede Gemeinschaft
die andere zu überbieten, jede Kirche wollte reiner nach Gotteß
Wort geordnet sein. So wichtig ihnen auch die Reformation dez
Lebens war, der Nachdruck beim Einzelnen wie in der Offent-
lichkeit ruhte auf einem prononcierten Zur-Schau-stellen des Ve-
kenntnisseß, der Bibel, der kirchlichen Ordnung. EZ ist vielleicht
nie in der Geschichte so viel in der Bibel gelesen, so eifrig gebetet,
so lang und viel gepredigt worden, wie unter der Herrschaft des
Puritanertumß. Und da von der Reformation her die Lehre von
der auch im Ehristenstande bleibenden Sündhaftigkeit sich diesen
Frommen eingeprägt hatte, so lag ez allerdingtz nahe, zu meinen,
daß elementare wie feinere sittliche Gebrechen durch den geistlichen
Habituö, das Bekenntniö, genugsam aufgehoben würden; darin
liegt die Verwandtschaft des Puritanißmuö und jedes Pietißmuß
mit dem Pharisäertum.
An diesem Punkt setzt die Kritik, der Protest, der Gotteßzom
unseres- Quäkerß ein. Ich sehe seine Eigenttimlichkeit gar nicht
in seinem Gnthusiaömuß, sondern in seiner moralischen Gesundheit
und gründlichen Ehrlichkeit. Er scheint mir der ausrichtigste,


% \picinclude{./vorwort/p_v10.jpg} 
K Zur Einsühtung.
lauterste Mann seines Zeitalters zu sein und darin allerdings im
Sinn Thomas Carlyles ein ganzer Held. Er hatte einen
einzigen Sinn, den Sinn für Recht und Unrecht, vorausgesetzt,
daß man das Wort ,,Recht« in seinem weiten Sinn nimmt,
da es alle Liebe und Menschlichkeit in sich schließt. Wenn man
alles übersieht was er in seinem ganzen Leben angreift, wofür
er kämpft, es ist — ein paar Außerlichkeiten, die bei ihm sehr
innerlich gemeint waren, abgerechnet — immer die schlichte
natürliche Moral, für die er eintritt, Recht und Liebe und Treue
(Mt. 23): nicht lügen, nicht Unrecht tun, nicht schwören, sluchen,
stehlen, nicht Gottes Namen mißbrauchen, für alle Menschen den
Frieden und das Gute suchen und friedlich leben mit ihnen. Gr
ist jedesmal empört, wenn er sieht, daß Dienstboten am Lohn
verkürzt werden, daß Wirte ihre Gäste betrunken machen, daß
Steuereinnehmer die Armen bedrücken, daß arme Reisende lieblos
behandelt werden, daß bei einem Schifsbruch die benachbarte Ve-
völkerung sich auf den Raubstürzt, statt sich der Schissbrüchigen
anzunehmen. Aus eigener Anschauung lernte er die Sünden des
englischen Rechts und Strafwesens kennen: die lange Ver-
schleppung der Prozesse, die vorschnelle Füllung von Todes-
urteilen wegen unwichtiger Vergehen in Geldsachen oder das
Vieh betreffend, den Mißbrauch des Eides, die schauerlichen
Gefängnisse Tmit den barbarischen Kerkermeistern, die Ansteckung
der Gefangenen durch die schlechte Gesellschaft, die Unterschlagung
der für die Gefangenen bestimmten Speisen, die gänzliche Ver-
wahrlosung der Familien der Gefangenen während ihrer Ge-
fangenschaft. Wahr ist, daß sich ihm dabei zuweilen Unwichtiges
als wichtig saufdrängte und er aus dem Duzen aller Menschen
und dem Aufbehalten des Hutes selbst vor den Richtern mit
einem Gigensinn bestand, den wir bei Jesus und selbst seinen
Jüngern nicht finden. Aber diese Außerlichkeiten der Konvention
hat er eben anders betrachtet; er verstand nicht und konnte nicht
verstehen, wie Menschen, die doch alle Brüder sind, künstlicher
Formen unter einander bedürfen. Daß er dann in seinem Recht-
sinn schlechterdings keinen Unterschied zwischen Frauen und Mün-
nern, zwischen Engländern und Negern oder Jndianern machen
kann, daß er sie alle schlechtweg als Menschen nimmt mit dem
vollen Anspruch aus menschliche Behandlung, braucht kaum hin-
zugefügt zu werden. Gr nimmt das alle-3 auch gar nicht als


% \picinclude{./vorwort/p_v11.jpg} 
Zur Einführung. Il
christlich in Anspruch; das Licht, das einen jeden Menschen er
leuchtet, das Gewissen, wird in allen Menschen dasselbe Recht
und Unrecht erkennen müssen.
Aber da drängte sich ihm nun die entsetzliche Frage auf;
wo ist bei den Christen, bei diesen Puritanern und Frommen
(Bekennern), diese Wahrhaftigkeit, Menschlichkeit und Liebe? Als
er selbst noch Trost bei einzelnen ihrer Führer suchte, erlebte er
nichts als Enttäuschungen; der eine schwatzte seine Leiden und
Bekümmernisse den Dienstboten aus, ein anderer geriet mitten
im Gespräch, als Fox aus Versehen auf den Rand eines Garten-
beetes trat, in solche Wut, als ob sein Haus in Flammen stünde.
Sie besaßen das gar nicht, was sie bekannten. Und später machte
er die Beobachtung, daß ihm und seinen ,,Freunden« aus diesen
Puritanerkirchen die roheste, gemeinste Verfolgung erwuchs, die
sich in Amerika, in den puritanischen Musterländern, bis zur
Hinrichtrmg einzelner Quäker steigerte. Diese Kirchen, die eben
aus jahrzehntelanger Verfolgungszeit zur Freiheit gelangt waren,
zeigten sich genau so tmduldsam, iso heerschsüchtig- so pfäfsisch
wie ihre früheren Verfolger, Katholiken und Anglikaner. Man
kann dies Urteil als einseitig beanstanden, als einen Ausf-luß des
sektenhaften Richtgeistes, der sich der Quäker bemächtigte; so wie
es hier aussieht, haben sich Licht und Finsternis nicht verteilt,
man denke nur an Richard Baxter und an Oliver Cromwell.
Aber daß Fox auch guten Grund zu dieser Beurteilung hatte, wer
wird das leugnen? Eine so ausgesprochen sromme Bewegung
wie der Puritanistnus fordert den strengsten Maßstab heraus.
Von da aus stellte sich ihm das scharfe Entweder — Oder
einer doppelten Frömmigkeit aus: die der frommen Formen und
Worte, und die der Kraft. Auf der einen Seite standen ihm alle
vorhandenen Religionen, Puritaner und Anglikaner und Katholiken
allesamt, deren Unterschiede doch nur in den Formen bestehen,
aus der andern Seite das, was Gott will, wozu Jesus in die
Welt gekommen ist. Fromm sein, das heißt die Kraft Gottes
besitzen, von ihr allein beherrscht werden, dies, und dies allein.
Gr nannte diese Kraft den Geist oder mit Vorliebe den Samen
Gottes, glaubte kühn, daß in jedem Menschen dieser Same oet-
borgen sei, eben als das Zeugnis seines Gewissens, und daß der
ein Christ sei, in dem der Same durch Gottes Wunder lebendig
und mächtig über alles geworden sei. Nen waren diese Gedanken


% \picinclude{./vorwort/p_v12.jpg} 
Xll Zur Einführung.
nicht, wir finden sie in der Resormationszeit besonders bei
Sebastian Franck und später bei Jakob Böhme, der in der
Zeit des Fox auch in England englisch gelesen wurde. E3
kommt aber nicht aus die Priorität an, sondern darauf, daß sie
hier bei Fox mehr als Gedanken waren, daß sie die Lebenskraft
einer ganzen Gemeinschaft wurden. Und sie traten hier nicht wie
in Deutschland zu einem toten Kirchentum und einer starren
Orthvdoxie in Gegensatz, sondern zku den lebendigsten, jugend-
srischesten Puritanergemeinschaften. tzhnen galt der für fromme
Ohren wahrhaft entsetzliche Kriegsrus: nicht Bibel, nicht Ve—
kenntnis, nicht Kirchen, sondern allein der Geist, der Gott, der in
uns selber als Lehrer tmd als Kraft lebendig ist!
Es war eine höchst gefährliche Losung, die Fox damit auf-
nahm, die Losung aller Schwarmgeister und Fanatiker, aus der
von Jahrhundert zu Jahrhundert die unheinilichsten und grau-
sigsten Exzesse der Religionsgeschichte geboren worden sind. Wie
leicht verbergen sich dunkles Triebleben und oerworrene mensch-
liche Einbildung unter dem hohen Titel des Geistes Gottes! Als
Fox auftrat, wimmelte es in England von Enthusiasten aller
Art, entfesselt durch die allgemeine Emanzipation des Revolutions-
zeitalters. Sie traten mit Träumen, Gesichten und Stimmen aus,
gaben sich selbst für Christus aus und erklärten, sündlos zu sein.
Alle Geschichte war ihnen bloßes Symbol ihrer eigenen Erlebnisse,
man hörte geradezu die Leugnung, daß Jesu Tod eine geschichtliche
Tatsache sei; sein Leiden sei ja in uns. Diese Gott- und Christus-
trunkenen Schwärmer wurden von den Kirchlichen Runter,
d. h. Prahler, genannt. Und bevor der Quäkername sich allgemein
verbreitete, sind auch Fox und seine ersten »Freunde« als Ranter
angeschrien und verfolgt worden. Obschon Fox von Anfang an
dagegen protestierte, es ist Tatsache, daß die ersten Quäker und
die Ranter sich wie ein Haar vom andern unterschieden, daß
einige der hervorragendsten Quäker den Rantergeist nicht los
geworden sind. Der messianische Einzug des James Naylor
in Bristol ist ein echtes Ranterstück. Und äußerlich betrachtet, i
wer will die Ossenbarungen des Fox von den Eingebungen der
Runter unterscheiden?
Aber während diese Ranter spurlos und namenlos in der
Geschichte der religiösen Schwtirmerei wieder untergegangen sind,
bilden die »Freunde« bis heut eine blühende religiöse Gemein-


% \picinclude{./vorwort/p_v13.jpg} 
Zur Einsührung. Xlll
schaft von charakteristischer Eigenart. Daß kommt daher, daß sie
die Niichternen, die moralisch Gesunden in dem enthusiastischen
Wirbelsturm waren. Der Gnthusiaßmus ist bei ihnen nur die
Form, die Grstlingöform, in der ihre neue moralische Kraft sich
manifestiert, nicht anders, als eß beim Urchristentum der Fall war.
Man erkennt an diesem Enthusiasmus die absolute Energie, mit
der sie von ihrer Wahrheit erfaßt waren, mochte die ganze Welt
widersprechen. Sobald man aber aus den Jnhalt achtet, stößt man
aus jene schlichte Menschlichkeit, das- Einfachste und Nüchternste, was
jemalß Grweckungßprediger gefordert haben. Fox ist kein Schwärmer
gewesen, obschon ihn die von ihm erkannte Wahrheit beherrschte
wie eine tiefe Schwärmerei. In keinem Augenblick seineß Lebenß
hat er seinen klaren Sinn für Recht und Unrecht, gut und böse
verloren. Daß rettete ihn an schwindelnden Abgrtinden vorbei
und durch alle Verlorkungen des religiösen Wahnsinnß, dem manche
seiner Genossen erlagen. Er war vielleicht nicht immer Herr über
die von ihm entsachte Bewegung, wie ihm überhaupt das Herrscher-
talent abging, aber er war immer Herr über sich selbst. Daß
ist der eine Grund, daß daß Geistprinzip ihm nichts geschadet
hat: seine gründliche moralische Festigkeit und Gesundheit. Dazu
kommt aber, daß der Gegensatz zum historischen Christentum nicht
von ferne so tief war, wie ihn die Kampslosung deß Fox: ,,nicht
die Bibel, sondern der Geist« könnte erscheinen lassen. Kein
Mensch seiner Zeit hat mehr in seiner Bibel und auö seiner
Bibel gelebt alß eben Fox; seine individuelle, kernige Sprache ist
ihm sogar durch den biblischen Dialekt ganz abhanden gekommen.
Selbst in seinen eigenen Liebling;-’gedanken steht er aus dem festen
Grund dez Reformationöeoangeliuniß von der den Menschen
durch Gottes Gnade geschenkten und durch gar kein Eigenwerk
von ferne zu oerdienenden Erlösung. Jn allem, wat; er redet und
tut, beruft er sich aus das- Wort Jesu und der Apostel, und daß
cthische Jdeal, daß er darauß ableitet, berührt sich ausß engste
mit dem täuferischen Ideal der Friedsamkeit und Wehrlosigkeit,
von dem die Täufer selbst sich in den Reoolutionökriegen hatten
abdräugen lassen. So oerleugnet er nirgends die Kontinuität
mit dem Christentum der von ihm so radikal verworsenen Kirchen;
er ist auch schon viel zu bescheiden und ehrlich, um den Anspruch
zu erheben, autz dem Geist Gotteö heraus ein neuer Religionß-
stifter zu sein. Was er eigentlich will, ist nur die radikale Re-


% \picinclude{./vorwort/p_v14.jpg} 
W1 Zur Einführung.
formation in Tat und Leben, das Emftmachen mit Jesu Wort
und Geist in rücksichtslosem Kampf mit allem, was Welt und
Tradition dariiberlegten. Stellt er also Geist und Bibel in so
scharfen Gegensatz, so meint er letztlich zwei Dinge: das Recht
des Laienverständnisses der Bibel im Gegensatz zum theologischen
Privileg — nicht Gelehrsamkeit, sondern allein Frömmigkeit kann
Gott recht verstehen — und die Notwendigkeit, die Kraft der
biblischen Religion im Leben zu beweisen, statt im Besitz und
Lesen des Bibelbuchs. So sbetrachtet, ist er keine Jnstanz, die
sich gegen den Wert der Bibel und des historischen Christen-
tums anführen läßt, sondern ein Zeugnis der Lebenskraft,
welche aus der Berührung der Geschichte — Gottes in der Ge-
schichte — mit einem aufrichtigen, tapferen Menschenherzen quillt.
So ist auch, was wir heute brauchen, keine neue Offenbarung
verborgener Seiten Gottes, sondern ein ganz anderes Grnstmachen
mit der uns in der Geschichte geschenkten Erkenntnis Gottes und
unserer Pflicht, als das Namenchristentum es kennt.
Damit scheint mir das Wesentliche zum Verständnis des Fox
angedeutet zu sein. Höchstens fehlt noch ein ganz auffallender
Punkt: sein massiver Vergeltungsglaube. Jn seiner sittlichen
Forderung iiberschritt er bewußt die alttestamentliche Stufe und
lebte die Ethik der Bergpredigt wie wenig Christen vor ihm.
Aber in seinem Vergeltungsglauben, der ihn mit größtem Eifer
und innerer Zufriedenheit die jeweiligen Strafen der Quäker-
Verfolger, besonders die auffallenden, plötzlichen Gottesgerichte
notieren ließ, kommt er uns seltsam alttestamentlich zurückgeblieben
oor. Für ihn war das die notwendige Ergänzung seines extremen
Spiritualismus. Der Gott, der ihn innerlich bewegte, war der-
selbe, der die Vorfälle des äußeren Lebens in seiner Hand hielt
und durch sichtbare Strafen oder Segnungen kundgab, auf welcher
Seite das Recht war. Genau so haben es Cromwell und die
Jndependenten geglaubt. Deshalb ist es Fox doch keinen Augen-
blick eingefallen, etwa nun auch seine und der ,,Freunde« Leiden
als Strafen Gottes aufzunehmen und nach der sie verdienenden
menschlichen Verschuldung zu fragen. Die entsetzlichsten Miß-
handlungen und die empörendsten Ungerechtigkeiten der Justiz
nahm er als Kind Gottes gelassen und sogar fröhlich auf, ohne
an eine Strafe Gottes dabei zudenken. Das sind Jnkonsequenzen.
Man kann es doch auch wohl verstehen, daß ein Mann mit


% \picinclude{./vorwort/p_v15.jpg} 
Zur Einführung. 17
solchem Rechtsfmn so fest sich an den Glauben an eine moralische
Weltordnung auch im Außern geklammert hat.
Die Geschichte des Fox und der Freunde, die diese Auf-
zeichnungen uns vorführen, zerfällt in zwei deutlich unterschiedene
Perioden. Zuerst die Sturm und Drangzeit, die Periode des
extremen Enthusiasmus und der extremen antikirchlichen Agitation.
Es ist die Zeit der ersten Liebe, des größten Heroismus, aber
auch einer rohen Unreife, die sich später in den langen Leidens-
jahren korrigiert. Fox ist damals ein Kirchenstürmer der wildesten
Art gewesen, er ging darauf aus, die Leute aus den Puritaner-
kirchen und von den Puritanerpfarrern weg zu reißen mit allen
Mitteln der Agitation. Wenn sich dann der ganze Haß der
Pfarrer und ihrer Anhänger in der brutalsten Weise über ihm
entlud, so ist ihm das sicher nicht unnerschuldet begegnet, wenn
er auch mit Recht darin eine wunderliche Manifestation des puri-
tanischen Christentums, dieser Gxtrafrömmigkeit, sah. Es bestand
für ihn selbst vorübergehend die Gefahr, daß das Nein in der
Agitation das Ja, das Recht und die Liebe, übertöne. Es bildete
sich damals ein Richtgeist unter den Freunden aus, der sie in
allen andern Gemeinschaften Babylon und den Antichrist erblicken
ließ, während sie allein Jerusalem vertraten. Mit Cromwells
Protektorat standen sie auf gespanntestem Fuß; es ist nur mensch-
lich, daß dem toleranten Mann ihnen gegenüber wiederholt die
Geduld ausging. Dazu die gräßliche Schwärmerei des James
Naylor, eines der Führer der Bewegung, und auf der andern
Seite die Phantastik der Weltmission, die gleich den Papst in
Rom und den Sultan in Konstantinopel zu bekehren strebte und
Schreiben an alle Potentaten der Welt ergehen ließ, damit sie
dem ,,Kommen des Herrn« nichts in den Weg stellten. Man
kann diese Zeit in ihrer Weise mit der Sichtungszeit der
Brüdergemeinde vergleichen, es ist fast ein Wunder, daß die
Gemeinschaft diese Schwärmerei überstand. Nun, sie haben auch
dafür gelitten, und unter den Blättern aus der Geschichte religiösen
Heldentums ragen zweifellos die Erzählungen des Fox aus dieser
ersten Zeit immer hervor.
Die zweite Periode, welche ungefähr mit der Restauration
des Königtums einsetzt, ist dann die Periode der Ernüchterung
und der Organisation. Erst jetzt beginnt im vollen Sinn ihr
unschuldiges Leiden, nachdem das frühere so vielfach durch eigene


% \picinclude{./vorwort/p_v16.jpg} 
171 Zur Einführung.
Schuld provoziert worden war. Nicht nur hielten sie sich inder
politischen Neuordnung durchaus neutral, ja grundsätzlich un-
politisch, sodaß auch nicht der Schein einer Gefahr für den Staat
bestand, ihre Kirchenstürmerei hatten sie längst in dem Maß auf-
gegeben, als sie sich zu selbständigen Gemeinschaften mit eigenem
,,Gottesdienste« zusammenschloss en; sie taten schlechterdings niemand
etwas zu leide. Dennoch hat die Verfolgung seitens des neu-
befestigten anglikanischen Staatskirchentums gerade sie mit beson-
derer Härte getroffen, z. T. wegen ihrer absoluten Eidverweigerung,
die ihnen schlimm aus-gedeutet werden konnte, und hat dadurch
dazu beigetragen, daß auch der letzte revolutionäre Gedanke sich
verlor, und gar nichts anderes übrig blieb als die gänzliche
Wehrlosigkeit und Leidsamkeit, durch die einst das alte Täufertum
sich ein bleibendes Andenken in der Geschichte erwarb. Der
stürmische Enthusiasmus war verflogen, aber der stille Enthusias-
mus, mit dem das Gotteskind alles Leid, das Menschen ihm
antun können, friedlich, unverbittert, ja selig im Grunde, hinnimmt,
blieb als die Frucht der großen Zeit. Zugleich aber ist diese
Leidenszeit die Zeit des Bauens, der Organisation. Aus der
kirchenstürmerischen Bewegung geht selbst eine neue Kirche — Fox
selbst braucht den Ausdruck Kirche dafür — hervor, und das
Erstaunlichste ist, daß nicht etwa Epigonen im Gegensatz zur
ursprünglichen Tendenz des Stifters diese Verkirchlichung durch-
setzen, sondern daß der Begründer des quäkerischen Enthusiasmus
auch der kirchliche Organisator ist. Zuerst hatten sich die Jahres-
versammlung und die Vierteljahrsversammlungen eingebürgert zu
wichtigeren Beratungen, während man am ,,Ersten Tag'' (dem
Sonntag) zu zwanglosen Aussprachen aus der Eingebung des
Geistes zusammenkam. Das war ein Anfang von Ordnung, aber
dem einzelnen blieb eine ungebundene Freiheit während des
ganzen Vierteljahrs. Da tat Fox im Jahr 1666 den entschei-
denden Schritt zur geschlossenen kirchlichen Organisation mit der
Einrichtung von Monatsversammlungen sowohl fiir die Männer
als für die Frauen, vornehmlich zur Durchführung der Kirchen-
zucht gegen unordentliche Mitglieder, auch zur Regelung der
Quäkerehen 2c. Er hat damals ganz England im Jnteresse
dieser Organisation bereist und alle Quäker im Ausland, auf
dem Kontinent, in Jrland, Schottland und Amerika, zur Nach-
ahmung dieser Organisation aufgefordert. Eine große Reformation


% \picinclude{./vorwort/p_v17.jpg} 
Zur Einführung. 1711
des Quäkertumß leitet er selbst von dieser neuen Verfassung her.
Allein es fehlte nicht an ganz energischem Widerstand aus den
Kreisen der »Freunde«, welche die christliche Freiheit durch eine
neue Menschensatzung bedroht glaubten, in der Kirchenzucht ein
uneoangelischeß Richten aufkommen sahen und speziell von der
den Frauen in dieser Organisation gewährten Stellung nichtö
wissen wollten. EZ ist kein Zufall, daß in diesem Zusammenhang
wieder von den Rantern die Rede ist. Der alte Rantergeist, der
extreme Subjektiviömuß und Jndividualiömuß, sah sich durch diese
kirchliche Organisation in-8 Herz getroffen. GS berührt in der
. Tat seltsam, wenn man den alten Fox jetzt den göttlichen Ursprung
dieser »eoangelischen Ordnungen« verfechten sieht; war das noch
der Prophet und Enthusiast von ehemal-Z? Und doch ist er nicht
von sich abgefallen, als er für seine Gemeinschaft die für ihren
Bestand notwendigen Formen schuf. Jndioidualist im extremen
Sinn war er nie gewesen, sondern von Anfang an Gemeinschaftß-
mann, und darum Mann der Liebe und Ordnung. Er hat einfach
gelernt, was- jeder gereifte Mensch einmal lernen muß, daß der
Geist zerfließt und zerflattert, wenn er nicht durch Organisationen
sich einen dauerhaften Körper geben kann. Waß müßten wir
heute vom Quäkertum ohne diese kirchliche »Reformation«! Zudem
ist die Quäkerorganisation unter allen mir bekannten kirchlichen
Ordnungen die freieste, formloseste geblieben. Gar kein Glaubens-
bekenntnis, und daß Grftaunlichste — keine Sakramente! Der
Gottesdienst so, daß man zusammenkommt, auf den Geist wartet,
und wenn einmal der Geist niemand zum Reden treibt, sich nach
stiller Versammlung die Hand gibt und friedlich autzeinandergeht.
Bekenntnis und Verfassung und Kultuö sind hier nichtß, das
Leben ist allez, und will und soll sein, wa-3 es- von Anfang war:
Recht und Liebe und Treue, schlichte Menschlichkeit.
Man soll eß nie vergessen, daß —— nächst dem einzelnen
Roger Williams- —— die Quäker die ersten waren, die mit einer
unterschiedölosen religiösen Toleranz praktisch ernst machten, nicht
aus Jndifferenz sondern aus Glauben. Andere, wie die Puritaner,
hatten nur, solang sie selbst verfolgt waren, Toleranz begehrt,
und, zur Herrschaft gelangt, sie schmählich verleugnet. In Penn-
stlvanien hat tatsächlich jeder seineö Glaubenö frei gelebt. Hier
bei den Quäkern zuerst ist den Frauen die volle kirchliche Gleich-
stellung mit dem Mann gegeben worden; ez gibt kein doppeltes


% \picinclude{./vorwort/p_v18.jpg} 
Illlll Zur Einführung.
Recht vor Gott. Von Fox erging während seiner amerikanischen
Reise die Mahnung, die Negersklaven mild und freundlich zu
behandeln und sie frei zu lassen, nachdem sie einige Jahre gedient.
Die Quäker sind damit im Kampf gegen die Sklaverei Voran-
gegangen. Und wenn mehr alz ein Jahrhundert später Elisabeth
Fry alz Resomiatorin deö Gesängnißwesenß England und den
Kontinent durchzogen hat, so sehen wir auß den Aufzeichnungen
des Fox, daß sie nur seine Aufgabe zu Ende führte. EZ gibt kein
großes Werk der Menschlichkeit und Barmherzigkeit, an dem nicht
die Quäker beteiligt sind, und daß nicht letztlich in dem wurzelt,
waz Fox ale die Kraft des Samenö Gotteö erkannte.
Daß ist doch mehr als eine historische oder religionöpsycho-
logische Merkwiirdigkeit, ee ist die beste Kraft dez Evangelium-3,
es- ist Jesuß selbst, der hier wieder einmal die Umhiillungeu, in
denen ihn menschliche Schwachheit und Kleinglaube konferoieren,
konseroieren mtiss en, frei herau?-tritt, um die Menschen einen ganzen
großen Schritt auswärtz zu führen in der Richtung auf sein
Gotteßreich. Wir in der Schweiz und in Deutschland sind nicht
Quäker und werden auch nach unserer Eigenart keine Quäker-
gemeinschaften bilden, aber wir sind Jünger dez Evangeliumß
nur dann, wenn wir ganz allein daß wollen, wa-3* die Quäker
wollten, ein Leben in der Kraft Gotteß statt in den Formen
und Worten, und die Richtung der Kraft: Recht und Liebe
und Treue, Menschlichkeit. Viele haben gemeint, daß die Quäker
doch keine rechten Christen seien, weil sie gar keine Sakra-
mente haben. Aber dem steht daß Wort Jesu entgegen: an
den Früchten sollt ihr sie erkennen! Hätte eine unserer Kirchen
solche Früchte wie die Quäker!
Daß nach dem Tode dee George Fox von William Penn
heraußgegebene Journal, auß dem im folgenden eine Auswahl
gegeben wird, ist nicht, wie der Titel vermuten ließe, ein wirk-
liches Tagebuch, sondern eine zusammenhängend geschriebene
Selbstbiographie, die allerdingß tagebuchartige Notizen voraußsetzt.
GZ ist ein Werk deß Alter;-’, geschrieben in der Restaurationözeit,
wie ich vermute, etwa im Jahre 1677, als Fox sich überhaupt
an die Sammlung und Ordnung seiner älteren Dokumente machte,
und dann in den folgenden Jahren noch ergänzt. Für seine
relativ einheitliche Abfassung spricht einmal der durchweg einheit-
liche Stil, der gar keine Wandlungen aufweist, sodann die Hinzu-


% \picinclude{./vorwort/p_v19.jpg} 
Zur Einführung. 111
fügung einer ganzen Reihe späterer Notizen bei viel früheren
Jahrgängen, vor allem der religiöse und kirchliche Geist des
Ganzen. Die Anerkennung Karl Stuarts als des von Gott be-
stimmten rechtmäßigen Königs beherrscht das ganze Buch, von
dem ursprünglichen Enthusiasmus ist wohl die Erinnerung be-
halten, aber aus ihm heraus geschrieben ist keine Seite, ja es
läßt sich eine gewisse apologetische Absicht nicht verkenneu, auch
das Quäkertum der Vergangenheit als politisch harmlos und in
jeder Weise ungefährlich hinzustellen. Die Verirrungen einzelner
Quäker, z. B. des James Naylor sind behutsam angedeutet, aber
man gewinnt keinen Eindruck, wie kritisch sie damals siir das
Quäkertum gewesen sind. Die Vermengung der Quäker mit den
Rantern wird von Anfang an säuberlich abgewehrt, die wirkliche
Entstehung des Quäkernamens aus den krankhasten Konvulsionen
der ,,Freunde« (Zitterer) wird verdeckt. Etwas Unwahres möchte ich
in dieser Darstellung nicht sehen, wohl aber da und dort eine
unwillkiirliche Verschiebung, veranlaßt durch die eigene Ernüchterung
und den Zwang, sich der Anklagen und Verleumdungen zu er-
wehren. Jch glaube, daß das, was erzählt ist, immer historische
Wahrheit ist; aber ob immer alles, zumal aus der Sturm- und
Drangzeit, erzählt ist, was man später noch wußte, muß für uns
dahin gestellt bleiben; über James Naylor wußte Fox jedenfalls
noch mehr. Aus dem Gedächtnis aber kann er diesen unermeßlich
reichen, im einzelnen so detaillierten Stoss nicht niedergeschrieben
haben. Eine eigene Aufzeichnung der Stationen muß ihm vor-
gelegen haben und zugleich wohl kurze Notizen über die merk-
wiirdigsten Erlebnisse an den einzelnen Orten. Leider hatte er
gar kein Jnteresse an der Chronologie; Jahreszahlen finden sich
im ganzen Buch nur in den mitgeteilten Dokumenten, eigenen
Briefen, Hastbefehlen 2c. Dazwischen erwähnt er jedoch eine
Reihe weltgeschichtlicher Begebenheiten, die der Quäkergeschichte
doch ein gewisses chronologisches Gerippe geben und deshalb in
der Übersetzung mit Fleiß gesammelt sind.
Die Beschränkung aus eine Auswahl ergab sich uns statt
einer ganzen Ubersetzung mit Notwendigkeit, weil das Ganze so
gut wie keine Leser gesunden hätte. Nicht nur des Umfangs
wegen. Die Erzählung wiederholt sich unendlich, und die ein-
gelegten Briefe sind von einer ermiidenden Breite und Monotonie.
Fox ist doch sicher ursprünglich ein origineller Laie gewesen mit



% \picinclude{./vorwort/p_v20.jpg}
II Zur Einführung.
realistischem Ausdruck und oft ungewöhnlichen: Mutterwitz.
Allein die Notwendigkeit, 40 Jahre lang unaufhörlich reden zu
müssen, und eigentlich doch immer dasselbe, hat seine Originalität
stark vermindert und ihm in Sprache und Schrift die Monotonie
verliehen, die im allgemeinen den Gemeinschaft-'spredigern nach-
zugehen pflegt. Einen wesentlichen Vorzug wird man ihm gerade
deöhalb doch zugestehen müssen: er erzählt durchaus schlicht, un-
gesucht, und daz gibt der Sache, die er erzählt, eine um so ge-
waltigere Wirkung, eZ steckt ’auch nicht ein Schimmer Eitelkeit
darin. Die Ubersetzerin — es ist die Tochter des Basler Kirchen-
historikertz und Zwinglibiographen Rud. Stähelin — hat sich
bemüht, so schmuckloö schlicht zu erzählen wie er selbst und die
Sache durch sich selbst reden zu lassen. Aufgenommen haben wir
alles, waß unß sür die Charakteristik deö Fox und die Geschichte
des Quäkertumö wesentlich schien, speziell auch möglichst alle
religiösen Merkwürdigkeiten und die zerstreuten welt- und kirchen-
geschichtlichen Notizen, welche die Verbindung mit der allgemeinen
Geschichte ermöglichen. In dieser beschränkten Außwahl, die im
Grunde doch alletz Wesentliche wiedergibt, wird, wie wir nicht —
zweifeln, die Lektüre unsereö Bucheß Vielen Genuß bringen, und,
hoffen wir, etwaö Besseres alß Genuß. Carlyle hat einmal im
Sartor Resartuz das merkwürdigste Greigniö der neuem Geschichte
den George Fox genannt, der sich einen Anzug von Leder machte.
Wer diese Aufzeichnungen lesen wird, der wird seine Paradoxsie
verstehen. P. Wernle.

