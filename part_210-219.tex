% \picinclude{./210-219/p_s210.jpg} 
unermüdliches Anhalten; der König gab Sir John Otway den
Befehl, dem Sheriff in einem Brief seinen diesbezüglichen Willen
kund zu tun, sowie in betrefs anderer aus der Gegend. Diesen
Brief nahm Sarah Fell mit, als sie mit ihrem Bruder und ihrer
Schwester Rouß nach Lancaster ging; und durch sie schrieb ich
folgendes an meine Frau:

\grosszitat{
  Mein liebes Herz in der Wahrheit und dem Leben, welches
  sich nimmermehr verändert.

  Es kam über mich, Mary Lower und Sarah sollten zum
  König gehen und zu Kirby, daß die Kraft des Herrn sich an
  ihnen allen kund tun möge zu deiner Befreiung. Sie gingen,
  wollten aber dann wieder zurückkommen; aber es kam über mich,
  sie noch ein wenig länger zu halten, damit sie die Lossprechung
  zu Ende bringen; dies ist nun geschehen, und wie du siehst, sende
  ich sie dir hier. Meine letzte Erklärung ist sehr förderlich gewesen,
  man war im ganzen damit zufrieden. Soviel für heute, meine
  Liebe im heiligen Samen.

  \begin{flushright}
  G. F.\end{flushright}
}

Die erwähnte Erklärung war ein gedrucktes Blatt, das ich
bei Anlass einer neuen Verfolgung geschrieben. Zu der Zeit
nämlich, da ich von Leicester nach London zurückkehrte, hatte
sich ein neuer Sturm erhoben infolge einer sehr stürmischen
Versammlung im Turmhauß in Gloucestershire;\index{Gloucestershire} es hieß, einige
Parlamentsmitglieder hätten sie dazu benutzt, um ein Gesetz gegen
verführerische Konventikel\index{Konventikel} durchzusetzen. Das selbe wurde bald
darauf Veröffentlicht und wurde auf uns angewandt, die wir doch
Vor allen andern frei waren von Verführung und Tumult. Darauf
schrieb ich eine Erklärung und zeigte an Hand der Ausdrücke
dieses Gesetzes das wir nicht derartige Leute seien noch
unsere Versammlungen derart, wie sie in dem Gesetz beschrieben
seien. [...]

Wir machten uns auf den Weg nach Rochester. Unterwegs,
als ich einen Hügel hinunterstieg, wurde meine Seele von einer
schweren Last bedrückt; ich bestieg mein Pferd, aber der Druck
blieb dermaßen, das ich kaum fähig war, weiter zu reiten. Endlich
kamen wir nach Rochester; aber ich war sehr erschöpft, weil die
Geister der Welt mich so schwer bedrückten und so schwer auf
mir lasteten. Mit Mühe erreichte ich Gravesend und lag dort
in einer Herberge, aber ich konnte kaum essen noch schlafen. Am
% \picinclude{./210-219/p_s211.jpg} 
folgenden Tage machten sich John Routz und Alexander Parker,
auf nach London; ich ging mit John Stubbs, welcher zu mir
gekommen war, mit der Fähre nach Essex. Wir kamen nach
Hornchurch, wo am Ersten Tage eine Versammlung war. Darauf
ritt ich unter großen Beschwerden nach Stratford zu einem
Freunde namens; Williamß, der früher Hauptmann gewesen war.
Hier lag ich in großer Schwachheit und verlor schließlich Gehör
und Gesicht.\index{Erkrankung} Mehrere Freunde kamen von London, um mich zu
besuchen, und ich sagte ihnen, ich müsse ein Zeichen sein für die,
welche die Wahrheit nicht sehen und hören wollten. Ich blieb
einige Zeit in diesem Zustand. Es kamen etliche zu mir, und
obgleich ich sie nicht sehen konnte, so durchschaute ich doch ihr
Inneres, welche aufrichtig waren und welche nicht. Verschiedene
Freunde, die Arzneikunde trieben, kamen zu mir und wollten mir
Medizin\index{Medizin} geben, aber ich durfte mich mit keinem einlassen, denn
ich spürte, das ich durch eine Heimsuchung hindurch müsse, und
darum wollte ich nur zuverlässige ernste Freunde um mich haben.
Unter großen Leiden und Beschwerden, in großer Gedrücktheit und
Niedergeschlagenheit, lag ich mehrere Wochen krank, und ich kam
so herunter und wurde so schwach, das die wenigsten glaubten,
ich würde am Leben bleiben. Einige, die bei mir waren, gingen
weg und sagten, sie wollten mich nicht sterben sehen; es hieß in
London und in der Umgegend, ich sei gestorben, aber ich fühlte,
das mich innerlich die Kraft des Herrn aufrecht erhielt. Als die,
welche um mich waren, mich aufgegeben hatten, hieß ich sie, mir
einen Wagen holen, um mich zu Gerrard Robert\index{Personen!Robert, Gerrard} zu bringen,
etwa zwölf Meilen weit weg, denn ich erkannte, das es das
richtige sei, dorthin zu gehen. Ich hatte wieder einen Schimmer,
so das ich die Leute und die Gegend erkennen konnte, aber das
war alles [...]

Ich litt zu dieser Zeit mehr, als sich mit Worten sagen lässt,\index{Vision}
denn ich musste in die Tiefe, und ich sah alle Religionen der Welt
und die Menschen, die darin leben, und die Priester, die sie vertraten, 
und die wie eine Bande von Menschenfressern waren; sie
fraßen die Menschen auf wie Brot und nagten das Fleisch von
ihren Knochen. Wahrer Glaube aber und Anbetung, wahre
Diener Gottes, ach! da waren keiner unter denen, die sich dafür
ausgaben! Denn die, welche behaupteten, eine Kirche zu sein,
waren nur eine Gesellschaft von Menschenfressern, Menschen mit
% \picinclude{./210-219/p_s212.jpg} 
harten Gesichtern und langen Zähnen; und wenn sie gleich über
die Menschenfresser in Amerika geschrieen, ich sah, das sie ganz
gleich waren. Den großen Frommen unter den Juden\index{Juden} sind sie
gleich, die \zitat{Gottes-Volk fressen wie Brot} (Micha 3,3),\index{Bibel!Micha 03:03@Micha 3:3} den
falschen Propheten und Priestern, die dem Volk Frieden predigten,
so lange als es ihnen zu fressen gab; wo man ihnen aber
nichts in das Maul gibt, da predigen sie, es müsse ein Krieg
kommen, \zitat{sie fressen das Fleisch meines Volkes und zerlegen es
wie Fleisch in einem Kessel} (Micha 3,5).\index{Bibel!Micha 03:05@Micha 3:5}

So sind sie, die sich jetzt als Christen ausgeben, sowohl
Priester als Fromme; sie sind nicht in der Kraft und dem Geist,
in welchem Christus, die heiligen Propheten\index{Prophet} und die Apostel\index{Apostel}
waren; sie sind von der gleichen Art wie die alten jüdischen
Frommen und sind Menschenfresser so gut wie jene. Sie haben
die Verfolgungen angezettelt und haben die bösen Angeber 
aufgestiftet, so das ein Freund kaum ruhig im engsten Familienkreise
sich aussprechen kann, wenn er sich zum Essen setzt, ohne das
nicht ein paar andere schon bereit wären, ihn zu verklagen [...]
Obgleich es eine Zeit grausamer Verfolgungen war, so war
doch Gottes Kraft über allen, und sein ewiger Same trug den Sieg
davon; und es wurde den Freunden gegeben, festzustehen und treu
zu bleiben in der Kraft des Herrn. Einige einsichtsvolle Leute
von anderen Glaubensrichtungen bekannten, wenn die Freunde
nicht ausharrten, so würde das Land dem Laster verfallen.
Obgleich ich durch meine Schwachheit verhindert war, wie
gewohnt bei den Freunden herumzureisen, so sandte ich doch nach
einem inneren Antrieb folgende Zeilen als Aufmunterung:

\grosszitat{
  Liebe Freunde,
  \medskip 
  Der Same über allen. Wandelt darin, in ihm habt ihr\index{Leiden der Gerechten}
  alle das Leben, lasset euch nicht irre machen durch die böse Zeit,
  denn der Gerechte hatte je und je vom Ungerechten zu leiden,
  aber der Gerechte trug den Sieg davon zu jeder Zeit. Es ist
  allezeit so gewesen; durch den Glauben wurden Berge bezwungen,
  und wurden der Zorn der Gottlosen und die feurigen Pfeile des
  Bösewichts ausgelöscht. Wenn gleich die Wellen und Stürme
  hoch gehen, so wird euer Glaube euch doch darüber halten, denn
  jene sind zeitlich, die Wahrheit aber ist ewig. Darum bleibet auf
  dem heiligen Berge,\index{Heiligen Berge} wo kein Unheil euch treffen wird. Denket
  nicht, das irgend etwas die Wahrheit überdauern werde; sie
  % \picinclude{./210-219/p_s213.jpg} 
  stehet fest und ist über allem, das nicht aus der Wahrheit ist;
  das Gute wird das Böse überwinden, das Licht die Finsternis,
  die Tugend das Laster, die Gerechtigkeit das Unrecht. Der falsche
  Prophet kann den wahren nicht überwinden, aber der wahre
  Prophet,\index{Wahrer Prophet} Christus, wird alle falschen überwinden. Darum bleibet
  treu und harret aus in Geduld.

  \begin{flushright}G. F.\end{flushright}

} 

Einige Zeit darauf gefiel es dem Herrn, die Hitze dieser
grausamen Verfolgung zu dämpfen, und ich fühlte in meiner
Seele trotz meiner äußern Schwachheit den Sieg über die Geister
jener Menschenfresser,\index{Menschenfresser} welche sie angestiftet und bis zu solcher
Grausamkeit weiter geführt. Und ich fühlte deutlich, und die
Freunde, die bei mir waren und zu mir kamen, sahen dies, das,
mit dem Aufhören der Verfolgung, ich frei wurde von dem Druck
und den Leiden, die so schwer auf mir gelegen hatten, so das ich
gegen den Frühling anfing, mich zu erholen und umherzugehen
über alle Erwartung vieler, die nie gedacht hätten, das ich je
wieder herumreisen würde.

Während ich unter dieser seelischen Anfechtung war, ward
mir der Zustand des Neuen Jerusalem,\index{Neuen Jerusalem} das vom Himmel herunter
kommt, geoffenbart, das einige fleischlich Gesinnte sich als eine
sichtbare, aus greifbaren Stoffen gemachte Stadt vorgestellt hatten.
Ich sah seine Schönheit und Herrlichkeit, seine Länge, Breite und
Höhe, alles in schönem Verhältnis. Ich sah, das alle, die im
Lichte Christi und im Glauben an ihn sind und im heiligen
Geiste, in welchem Christus und seine Apostel und Propheten
waren, und in der Gnade, der Wahrheit und der Kraft Gottes,
in dieser Stadt sind, Glieder derselben sind und das Recht haben,
vom Baum des Lebens\index{Baum des Lebens} zu essen, welcher jeden Monat seine Frucht
gibt, und dessen Blätter den Völkern Heilung bringen (Offb. 22)\index{Bibel!Offb. 22}.
Die aber nicht in der Gnade Gottes, der Wahrheit, dem Licht,
dem Geist und der Kraft Gottes sind, und die \zitat{dem heiligen Geist
widerstehen und die Gnade Gottes aus Mutwillen ziehen} (Jud. 4)\index{Bibel!Jud. 4},
die vom Glauben abgeirrt sind und die \zitat{Verheißungen, 
Offenbarungen und Eingebungen verachten}, dieses sind die Hunde
und die Ungläubigen\index{Ungläubigen}, die draußen sind (Offb. 22). Diese bilden
die große Stadt Babylon,\index{Babylon} die Verwirrung, und ihr Behältnis ist
die Macht der Finsternis, und der böse Geist des Irrtums umgibt
und bedeckt sie. In dieser großen Stadt Babylon sind die
% \picinclude{./210-219/p_s214.jpg} 
falschen Propheten, die in einem verkehrten Geist und einer
falschen Kraft stehen, das Tier, das in der Gewalt des Drachen
ist, die Hure, die den heiligen Geist und Christum ihren Gemahl
verlassen hat (Hos.1)\index{Bibel!Hos. 01@Hos. 1}. Aber des Herrn Macht ist größer als
alle Macht der Finsternis, alle falschen Propheten und ihre Anbeter: 
diese \zitat{gehören in den feurigen Pfuhl} (Off. 19,20)\index{Bibel!Off. 19:20}. [...]
Ich sah noch viele Dinge über das himmlische Jerusalem, welche
aber schwer zu beschreiben und noch schwerer zu Verstehen wären [...]

Während ich in Enfield\index{Enfield} wirkte, spürte ich einen Schaden,
der öfters unter den Bekennern der Wahrheit vor kam; nämlich,
wenn sie in ein anderes Land zogen, heirateten\index{Heiraten} sie unter Freunden,
bei denen sie fremd waren, und von denen man nicht wusste, ob sie
makellos\index{makellos}: und ordentlich seien oder nicht. Und eine innere Stimme
hieß mich ihnen folgende Verfahren zur Verhütung derartiger
Missstände zu empfehlen:

\grosszitat{
  Alle Freunde, die sich verheiraten, Männer wie Frauen,
  sollen, wenn sie aus einem anderen Land, einer andern Gegend
  oder Insel kommen, der Männerversammlung, der sie ihre Absicht
  zu heiraten, vorlegen, eine Bescheinigung von der 
  Männeversammlung aus dem Ort ihrer Herkunft bringen. Denn da die
  Männeversammlung aus Gläubigen besteht, so werden die
  herumschwärmenden bösen Geister gebannt. Kommt nun einer
  mit einer Bescheinigung oder einem Empfehlungschreiben\index{Empfehlungschreiben} einer
  Männeroersammlung zu einer andern, so wird diese durch jene
  erquickt, und sie kann die Sache getrost unternehmen. Dies wird
  viel Verdruss ersparen. Und was ihr ihnen dann in der Kraft
  Gottes zu sagen habt in Ermahnung und Lehre, das tut in der
  Kraft und dem Geist Gottes; lasset sie die Pflichten und die 
  Bedeutung der Ehe wissen. Die Einigkeit im Geist und Kraft,
  Licht und Weisheit von Gott möge unter allen Männerversammkungen 
  in der ganzen Welt herrschen in dem Einen, dem Leben.

  Lasst hiervon Abschriften in jedes Land, jede Gegend und
  Insel, wo Freunde sind, senden, damit alle Dinge heilig, rein
  und gerecht bewahret bleiben in Einigkeit und Frieden, und das
  Gott über alles gepriesen werde unter euch, seinen Auserwählten,
  seinem Volk und Erbe, die ihr seine erwählten Söhne und Töchter
  und Erben seines Lebens seid. Soviel davon; meine Liebe in
  dem, das nicht ändert.

  \medskip 

  \begin{flushright}
  Den 14.des 1. Monats 1671\index{Jahr!1671}. G. F.
  \end{flushright}
}


% \picinclude{./210-219/p_s215.jpg} 
Zu dieser Zeit trieb es mich, den Herrn also anzurufen:

\grosszitat{
  Herr, Gott, Allmächtiger!\index{Gebet}

  \medskip 

  Fördere die Arbeit und schütze die Gerechtigkeit und Billigkeit 
  im Land. Steure der Bösheit und Ungerechtigkeit, Bedrückung
  und Falschheit, Grausamkeit und Unbarmherzigkeit, auf das
  Barmherzigkeit und Gerechtigkeit möge überhand nehmen.
  \medskip 
  O, Herr, Gott! Richte die Wahrheit im Lande aus und
  schütze sie. Tilge aus alles Laster, Hurerei, Abgötterei, den Geist
  der Unzucht, welcher macht, das das Volk dich nicht ehrt, noch
  ihre Seelen, noch ihren Leib, noch das Christentum, noch Zucht,
  noch Menschenwürde.

  \medskip

  O, Herr, gib der Obrigkeit ins Herz, all diesem ungöttlichen
  Wesen, dieser Gewalttätigkeit und Grausamkeit, der Gottlosigkeit und
  dem Fluchen zu wehren, und alle schlechten Häuser und Spielhäuser 
  auszurotten, welche die Jugend und das Volk verderben,
  und sie deinem Reich entführen, in welches nichts Unreines je
  eingehen kann. Solches Treiben führet die Leute in die Hölle.
  Herr, reinige das Land von allen diesen Dingen nach Deiner
  Barmherzigkeit, das Dein Zorn gestillet werde, o Gott, und nicht
  über das Land hereinbreche.
  \medskip
  \begin{flushright}17. des 2. Monats 1671.\index{Jahr!1671} G. F.\end{flushright}
}



\chapter[Reise nach Amerika, Barbadoes und Jamaika.]{Reise nach Amerika, Barbadoes und Jamaika.}

\begin{center}
\textbf{Reise nach Amerika, Barbadoes und Jamaika.}
\end{center}


Wie schon erwähnt, hatte ich zwei Töchter meiner Frau
zum König geschickt, um ihre Freisprechung zu erwirken, und sie
hatten auch seinen diesbezüglichen Befehl dem Befehlshaber in
Laneashire gebracht; [...] aber der Sturm der Verfolgung war
gerade so mächtig geworden, das man Mittel fand, sie weiter
gefangen zu halten. Als nun aber die Verfolgungen etwas
nachließen, trieb es mich, Martha 
Fischer\index{Personen"!Fischer, Martha} und eine andere Frau
aus dem Kreise der Freunde zu veranlassen, abermals zum König
zu gehen, um ihre Freilassung zu erbitten. Sie gingen im Glauben
an die Kraft des Herrn, welcher sie Gnade finden ließ vor dem
König, so das er einen besiegelten Freilassungsbefehl bewilligte,
nachdem sie fast zehn Jahre gefangen gewesen war, und ihre Gitter
mit Beschlag belegt, dergleichen kaum je in England war erhört
worden. Ich schickte die Freisprechang sofort zu ihr durch einen
% \picinclude{./210-219/p_s216.jpg} 
Freund, und zugleich schrieb ich ihr, wie sie den Frestassungsbefehl
müsse der Richter zukommen lassen und teilte ihr auch
mit, daß es über gekommen sei vom Herrn, übers Meer zu
gehen nach Amerika; sie möge darum, sobald es ihr möglich sei,
nach London eilen, da das Schiff sich schon zur Abreise rüste.
In der Zwischenzeit ging ich nach Kigston zu John Rous, bis
meine Frau kam, und dann rüstete ich mich zur Reise. Doch
weil die Jahresversammlung\index{Jahresversammlung} bald stattfand, so blieb ich noch bis
zu derselben [...] Dann, als unsera Schiff und die Freunde, die,
mich zu begleiten beabsichtigten bereit waren, ging ich, am 12. des
6. Monates 1671\index{Jahr!1671} nach Gravesend, und meine Frau und mehrere der
Freunde begleiteten mich über die Downs. Die Freunde,
die die Reise mit mir machten waren: 
\begin{itemize}
 \item Thomas Briggs,\index{Personen!Briggs, Thomas}
 \item William Edmumdson,\index{Personen!Edmumdson, William} 
 \item John Stubbs,\index{Personen!Stubbs, John} 
 \item Salomon Eccles,\index{Personen!Eccles, Salomon} 
 \item James Lancaster,\index{Personen!Lancaster, James} 
 \item John Cartwright,\index{Personen!Cartwright, John} 
 \item Robert Widders,\index{Personen!Widders, Robert} 
 \item George Pattison,\index{Personen!Pattison, George}
 \item John Hull,\index{Personen!Hull, John} 
 \item Elisabeth Hooton und\index{Personen!Hooton, Elisabeth} 
 \item Elisabeth Miers.\index{Personen!Miers, Elisabeth}
\end{itemize}

Unser Schiff eine Jacht und hieß "`Industrie"', der Kapitän hieß Thomas
Forster, und wir waren etwa 50 Passa [...]

Als wir etwa drei Wochen auf dem Wasser waren, bemerkten
wir etwa vier Seemeilen hinter uns ein Schiff. Unser Kapitän
sagte, es sein maurisches Piratenschiff, das uns verfolgen
scheine. "`komt"' sagte er "`wir wollen zum Abendessen gehen,
wenn es dunkel geworden ist, sie unsere Spur verlieren."'
Dies sagte er, um die Reisenden zu beruhigen, denn es fingen
schon einige an sich zu ängstigen. Die Freund jedoch waren
guten Mutes, weil sie Gott vertrauten, und keinerlei Furcht ihr
Seele bedrücke. Als die Sonne untergegangen war, sah ich von
meiner Kajüte aus, wie das Schiff auf uns zukam. Als es
dunkel wurde, änderten wir die Richtung um ihm auszuweichen;
aber es änderte die seine auch, In der Nacht
kamen der Kapitän und andere zu mir in Kajüte und fragten
mich, was sie tun sollten. Ich antwortete,  kein Schiffsmann
und fragte sie, Was sie für das Beste hielten? Sie sagten,
Es gäbe nur zwei Wege: entweder wir müssten das Schiff überholen
oder hin und herkreuzen und die gleiche Richtung einhalten
wie vorher. Ich fragte, wenn es Räuber seien, so werden sie
sicherlich auch hin- und herkrezen und was das Überholen anbelangt,
so sei daran gar nicht zu denken, da man ja sehe wie
viel schneller fahren als wir. Sie fragten mich wieder, was
% \picinclude{./210-219/p_s217.jpg} 
sie denn tun sollten: \zitat{denn,} sagten sie, \zitat{wenn die Schiffsleute 
damals den Rat des Paulus befolgt hätten, so wäre es ihnen
nicht so schlimm ergangen.} Ich erwiderte: \zitat{Ee ist eine 
Glaubensprüfung, und darum muss man auf den Herrn warten und auf
seinen Rat.} Während ich mich nun innerlich sammelte, zeigte
mir der Herr, das er mit seinem Leben und mit seiner Kraft
zwischen uns und dem Schiff, das uns verfolgte, stehe. Ich
teilte dies dem Kapitän und den anderen mit, und das es nun
das Beste sei, zu kreuzen und den rechten Kurz einzuschlagen.
Ich hieß sie auch alle Lichter auslöschen außer dem einen, das
sie beim Steuer brauchten, und den Reisenden sagen, sie sollten
sich still und ruhig verhalten. In der Nacht etwa um 11 Uhr,
kam die Wache und sagte, sie seien ganz nahe hinter uns. Das
beunruhigte einige der Reisenden. Ich richtete mich in meiner
Kajüte auf, und da der Mond noch nicht untergegangen war. sah
ich durch die Luke, das sie ganz nahe waren. Ich wollte aufstehen 
und hinausgehen; aber ich erinnerte mich der Worte des
Herrn, \zitat{das er mit seinem Leben und seiner Kraft zwischen uns
und ihnen stehe,} und legte mich wieder nieder. Der Kapitän
und einige der Schiffsleute kamen abermals und fragten, ob sie
nicht nach dieser oder jener Richtung steuern sollten? Ich sagte
ihnen, sie sollten machen, wie sie wollten. Da ging der Mond
vollends unter, ein neuer Wind erhob sich, und der Herr verbarg
uns vor ihnen; wir segelten rasch und sahen sie nicht mehr. Am
folgenden Tag, einem Ersten Tag, hatten wir eine öffentliche
Versammlung auf dem Schiffe, wie wir sie gewöhnlich während
der ganzen Reise an diesem Tage zu halten pflegten; und des
Herrn Gegenwart war mächtig unter uns. Und ich ermahnte die
Leute, an Gottes Barmherzigkeit zu denken, die sie errettet; denn
sie wären jetzt vielleicht alle in den Händen der Türken,\index{Türken} wenn
des Herrn Hand sie nicht errettet hätte. Etwa eine Woche
darauf suchten der Kapitän und einige der Schiffsleute den
Reisenden einzureden, es seien nicht türkische Seeräuber\index{Seeräuber} gewesen,
die uns verfolgten, sondern ein Kaufmannssschiff, das nach den
Kanarischen Inseln ging. Als ich das hörte, fragte ich sie,
warum sie denn dann solches zu mir gesagt hätten? warum sie
die Reisenden beunruhigt hätten? und warum sie, um ihnen
davon zu fahren, den Kurs geändert hätten? Sie sollten sich
hüten, Gottes Barmherzigkeit zu verachten. Später, als wir in
% \picinclude{./210-219/p_s218.jpg} 
Barbados waren, kam ein maurischer Kaufmann und erzählte den
Leuten, die Mannschaft eines maurischen Piratenschiffs habe auf
dem Meer ein ungeheures Jachtschiff gesehen, das größte, das
sie je gesehen hätten, sie hätten es verfolgt, und seien schon ganz
nahe gewesen, aber es sei ein Geist darin gewesen, so das sie es
nicht erobern konnten. Dies bestätigte uns in unserer Überzeugung,
das es ein maurisches Piratenschiff war, das uns verfolgte, und
das es der Herr gewesen, der uns befreit hatte.

Ich war nicht seekrank gewesen auf der Reise, wie so viele
der Freunde und andere Reisende; aber alle die Wunden und
Schläge, die ich früher erlitten, die Krankheiten, die ich mir durch
die Kälte und die Entbehrungen während meiner Gefangenschaften
zugezogen hatte, machten sich nun während der Reise wieder
geltend, so das mein Magen sehr angegriffen war, und ich heftige
Schmerzen in allen Gliedern hatte. Es fing an, nachdem ich
etwa einen Monat auf der See war; zuerst schwitzte ich stark,\index{Erkrankung}
und an Kopf und Leib zeigten sich überall Pusteln,\index{Pusteln} und meine
Hände und Füße wurden so geschwollen, das ich nur mit Mühe
und unter großen Schmerzen meine Strümpfe und Pantoffeln
anziehen konnte; auf einmal hörte das Schwitzen auf, und
als ich in das heiße Klima kam, wo die anderen tüchtig schwitzten,
konnte ich gar nicht schwitzen, sondern mein Körper war heiß
und trocken und brennend, und was vorher in Pusteln 
ausgebrochen war, schlug jetzt nach innen auf Herz und Magen, so
das ich sehr krank war und über alle Maßen schwach; dies
dauerte während der ganzen übrigen Zeit der Reise, während
der etwa vier Wochen, die wir noch auf dem Wasser waren.
Am frühen Morgen des 3. Tages des 8. Monats erblickten wir
die Insel Barbadoes,\index{Barbadoes} aber es dauerte noch bis zwischen neun und
zehn des Abends, ehe wir in den Hafen der Carlisle-Bay 
einfuhren. Wir gingen sobald wie möglich ans Land, und ich 
begab mich mit einigen Freunden in das Haus eines Freundes,
eines Kaufmanns namens Richard Forstall,\index{Personen!orstall, Richard} der etwa zehn
Minuten von der Landungsbrücke wohnte. Aber ich war so
krank und schwach, das ich sehr müde wurde von diesem kurzen
Gang, und Vollständig erschöpft ankam. Ich lag dort mehrere
Tage krank, und obgleich man mir mehrmals Mittel gab, um
mich schwitzen zu machen, so kam es doch nie zu einem rechten
Schweiß. Was sie mir gaben, vertrocknete eher meinen Körper
% \picinclude{./210-219/p_s219.jpg} 
noch mehr, und machte mich noch kränker, als ich sonst gewesen
wäre. Diese Schmerzen in allen Gliedern dauerten etwa drei
Wochen, und ich litt sehr, so das ich kaum je Ruhe finden
konnte, aber ich war ziemlich getrost und der Geist ward Herr
über alles. Auch hinderte mich meine Krankheit nicht am Dienst
für die Wahrheit, sondern sowohl auf der See als in Barbadoeß,
ehe ich herum reisen konnte, gab ich verschiedene Schriften
heraus, die ein Freund für mich schrieb, und von denen ich einige
mit der ersten Gelegenheit nach England schickte, um gedruckt
zu werden [...]

Weil ich so schwach war, das ich nicht an die verschiedenen
Versammlungen reisen konnte, nahmen sich die andern Freunde
des Werkes des Herrn an; schon am Tage nach unserer Ankunft
hatten sie eine große Versammlung an der Landungsbrücke, und
nach derselben noch mehrere in verschiedenen Teilen der Insel,
maß die Bevölkerung sehr in Aufregung brachte, so das viele zu
den Versammlungen kamen, worunter mehrere von hohem Rang;
denn sie hatten gehört, das ich auf der Insel angekommen sei
und erwarteten, mich bei den Versammlungen zu sehen, da sie
nicht wussten, das ich zu schwach war, um zu kommen. Meine
Schwachheit wich darum so lange nicht von mir, weil mein 
Gemüt zuerst sehr niedergedrückt\index{Nidergeschlagenheit} 
war von der Schmutzigkeit und
Ungerechtigkeit und Gemeinheit der Leute, maß wie eine schwere
Last auf mir lag. Aber nachdem ich etwa einen Monat auf der
Insel gewesen war, wurde es mir etwas leichter zu Mut, und
ich fühlte mich wieder etwas kräftiger, so das ich wieder umher
gehen konnte zu den Freunden [...]

Weil ich aber doch nicht gut viel umher reisen konnte, so
kamen die Freunde auf der Insel überein, die Männer- und
Frauen-Versammlungen zur Ordnung der kirchlichen Angelegenheiten 
im Hause Thomas Rous, bei dem ich wohnte, abzuhalten,
so das ich bei allen Versammlungen dabei war und recht für den
Herrn wirken konnte. Denn sie hatten in manchen Dingen Belehrung\index{Belehrung}
nötig, weil sich aus Mangel an Vorsicht und Wachsamkeit 
allerlei Unordnungen eingeschlichen hatten. Ich ermahnte sie,
besonders in der Männerversammlung, recht vorsichtig und 
wachsam in Bezug auf das Heiraten\index{Heirat} zu sein und die Freunde zu
verhindern, in die Verwandtschaft zu heiraten, sowie auch zu
hastig vorzugehen bei Wiederverheiratung nach dem Tode des