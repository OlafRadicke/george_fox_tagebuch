% \picinclude{./210-219/p_s210.jpg} 
210 Kapitel IN1!.
unermüdlicheö Anhalten; der König gab Sir John Otway den
Befehl, dem Sheriff in einem Brief seinen dieöbeziiglichen Willen
kund zu ttm, sowie in betresf anderer autz der Gegend. Diesen
Brief nahm Sarah Fell mit, als sie mit ihrem Bruder und ihrer
Schwester Rouß nach Lancaster ging; und durch sie schrieb ich
folgendeö an meine Frau:
,,Mein liebeö Herz in der Wahrheit und dem Leben, welcheß
sich nimmermehr verändert.
ES kam über mich, Mary Lower und Sarah sollten zum
König gehen und zu Kirby, daß die Kraft des Herrn sich an
ihnen allen kund tun möge zu deiner Befreiung. Sie gingen,
wollten aber dann wieder zurückkommen; aber ez kam über mich,
sie noch ein wenig länger zu halten, damit sie die Loösprechung
zu Ende bringen; dietz ist nun geschehen, und wie du siehst, sende
ich sie dir hier. Meine letzte Erklärung ist sehr förderlich gewesen,
man war im ganzen damit zufrieden. Soviel für heute, meine
Liebe im heiligen Samen.«
G. F.
Die erwähnte Erklärung war ein gedruckteß Blatt, daß ich
bei Anlaß einer neuen Verfolgung geschrieben. Zu der Zeit
nämlich, da ich von Leicester nach London zurückkehrte, hatte
sich ein neuer Sturm erhoben infolge einer sehr stürmischen
Versammlung im Turmhauß in Gloucestershire; etz hieß, einige
Parlamentömitglieder hätten sie dazu benutzt, um ein Gesetz gegen
verführerische Konventikel durchzusetzen. Das-selbe wurde bald
darauf Veröffentlicht und wurde auf unß angewandt, die wir doch
Vor allen andern frei waren von Verführung und Tumult. Darauf
schrieb ich eine Erklärung und zeigte an Hand der Ausdrücke
dieseß Gesetzes daß wir nicht derartige Leute seien noch
unsere Versammlungen derart, wie sie in dem Gesetz beschrieben
seien. ....
Wir machten unß auf den Weg nach Rochester. Unterwegß,
altz ich einen Hügel hinunterstieg, wurde meine Seele von einer
schweren Last bedrückt; ich bestieg mein Pferd, aber der Truck
blieb dermaßen, daß ich kaum fähig war, weiter zu reiten. Endlich
kamen wir nach Rochester; aber ich war sehr erschöpft, weil die
Geister der Welt mich so schwer bedrückten und so schwer auf
mir lasteten. Mit Mühe erreichte ich Gravesend und lag dort
in einer Herberge, aber ich konnte kaum essen noch schlafen. Am


% \picinclude{./210-219/p_s211.jpg} 
Reise nach Irland. Rückkehr und Heirat mit Margaret Fell usw. 211
folgenden Tage machten sich John Routz und Alexander Parker,
auf nach London; ich ging mit John Stubbß, welcher zu mir
gekommen war, mit der Fähre nach Essex. Wir kamen nach
Hornchurch, wo am Ersten Tage eine Versammlung war. Daraus
ritt ich unter großen Beschwerden nach Stratford zu einem
Freunde namens; Williamß, der früher Hauptmann gewesen war.
Hier lag ich in großer Schwachheit und verlor schließlich Gehör
und Gesicht. Mehrere Freunde kamen von London, um mich zu
besuchen, und ich sagte ihnen, ich müsse ein Zeichen sein für die,
welche die Wahrheit nicht sehen und hören wollten. Ich blieb
einige Zeit in diesem Zustand. GS kamen etliche zu mir, und
obgleich ich sie nicht sehen konnte, so durchschante ich doch ihr
Inneres, welche aufrichtig waren und welche nicht. Verschiedene
Frennde, die Arzneikunde trieben, kamen zu mir und wollten mir
Medizinen geben, aber ich durfte mich mit keinem einlassen, denn
ich spürte, daß ich durch eine Heimsuchung hindurch müsse, und
darum wollte ich nur zuverlässige ernste Freunde um mich haben.
Unter großen Leiden und Beschwerden, in großer Gedrticktheit und
Riedergeschlagenheit, lag ich mehrere Wochen krank, und ich kam
so herunter und wurde so schwach, daß die wenigsten glaubten,
ich würde am Leben bleiben. Einige, die bei mir waren, gingen
weg und sagten, sie wollten mich nicht sterben sehen; es hieß in
London und in der Umgegend, ich sei gestorben, aber ich fühlte,
daß mich innerlich die Kraft des Herm aufrecht erhielt. Alß die,
welche um mich waren, mich aufgegeben hatten, hieß ich sie, mir
einen Wagen holen, um mich zu Gerrard Robert zu bringen,
etwa zwölf Meilen weit weg, denn ich erkannte, daß ez daß
richtige sei, dorthin zu gehen. Jch hatte wieder einen Schimmer,
so daß ich die Leute und die Gegend erkennen konnte, aber daß
war alleß .....
Ich litt zu dieser Zeit mehr, alö sich mit Worten sagen läßt,
denn ich mußte in die Tiefe, und ich sah alle Religionen der Welt
und die Menschen, die darin leben, und die Priester, die sie ver-
traten, und die wie eine Bande von Menschenfressern waren; sie
fraßen die Menschen auf wie Brot und nagten idaß Fleisch von
ihren Knochen. Wahrer Glaube aber und Anbetung, wahre
Diener Gotteß, ach! da waren keiner unter denen, die sich dasür
ausgaben! Denn die, welche behaupteten, eine Kirche zu sein,
waren nur eine Gesellschaft von Menschenfressern, Menschen mit
14*


% \picinclude{./210-219/p_s212.jpg} 
212 Kapitel 1711.
harten Gesichtern und langen Zähnen; und wenn sie gleich über
die Menschenfresser in Amerika geschrieen, ich sah, daß sie ganz
gleich waren. Den großen Frommen unter den Juden sind sie
gleich, die ,,Gotte;«- Volk fressen wie Brot« (Micha 3,3), den
falschen Propheten und Priestern, die dem Volk Frieden predigten,
so lange alö eS ihnen zu fressen gab; wo man ihnen aber
nichtö in daß Maul gibt, da predigen sie, etz müsse ein Krieg
kommen, »sie fressen das Fleisch meineö Volkeß und zerlegen etz
wie Fleisch in einem Kessel« (Micha 3,5).
So sind sie, die sich jetzt alß Christen aus-geben, sowohl
Priester als Fromme; sie sind nicht in der Kraft und dem Geist,
in welchem Ehristuö, die heiligen Propheten und die Apostel
waren; sie sind von der gleichen Art wie die alten jüdischen
Frommen und sind Menschenfresser so gut wie jene. Sie haben
die Verfolgungen angezettelt und haben die bösen Angeber aus-
gestiftet, so daß ein Freund kaum ruhig im engsten Familienkreise
sich aussprechen kann, wenn er sich zum Essen setzt, ohne daß
nicht ein paar andere schon bereit wären, ihn zu verklagen .....
Obgleich es eine Zeit grausamer Verfolgungen war, so war
doch Gottes Kraft über allen, und sein ewiger Same trug den Sieg
davon; und es wurde den Freunden gegeben, sestzustehen und treu
zu bleiben in der Kraft dez Herrn. Einige einsichtßvolle Leute
von anderen Glaubenßrichtungen bekannten, wenn die Freunde
nicht außharrten, so würde dat? Land dem Laster verfallen.
Obgleich ich durch meine Schwachheit verhindert war, wie
gewohnt bei den Freunden herumzureisen, so sandte ich doch nach
einem inneren Antrieb folgende Zeilen alö Aufmunterung:
,,Liebe Freunde,
Der Same über allen. Wandelt darin, in ihm habt ihr
alle das Leben, lasset euch nicht irre machen durch die böse Zeit,
denn der Gerechte hatte je und je vom Ungerechten zu leiden,
aber der Gerechte trug den Sieg davon zu jeder Zeit. GZ ist
allezeit so gewesen; durch den Glauben wurden Berge bezwungen,
und wurden der Zorn der Gottlosen und die feurigen Pfeile des
Bösewichtß ausgelöscht. Wenn gleich die Wellen und Stürme
hoch gehen, so wird euer Glaube euch doch darüber halten, denn
jene sind zeitlich, die Wahrheit aber ist ewig. Darum bleibet auf
dem heiligen Berge, wo kein Unheil euch tressen wird. Denket
nicht, daß irgend etwas die Wahrheit überdauern werde; sie


% \picinclude{./210-219/p_s213.jpg} 
Reise nach Jrland. Rückkehr und Heirat mit Margaret Fell usw. 213
stehet fest und ist über allem, das nicht auß der Wahrheit ist;
das Gute wird daß Böse überwinden, das Licht die Finsterniz,
die Tugend das Laster, die Gerechtigkeit daß Unrecht. Der falsche
Prophet kann den wahren nicht überwinden, aber der wahre
Prophet, Christus, wird alle falschen überwinden. Darum bleibet
treu und harret auß in Geduld.«
G. F.
Einige Zeit darauf gesiel eß dem Herrn, die Hitze dieser
grausamen Verfolgung zu dämpfen, und ich fühlte in meiner
Seele trotz meiner äußern Schwachheit den Sieg über die Geister
jener Menschensresser, welche sie angestiftet und bis zu solcher
Grausamkeit weiter geführt. Und ich fühlte deutlich, und die
Freunde, die bei mir waren und zu mir kamen, sahen dies, daß,
mit dem Aufhören der Verfolgung, ich frei wurde von dem Druck
und den Leiden, die so schwer auf mir gelegen hatten, so daß ich
gegen den Frühling anfing, mich zu erholen und umherzugehen
über alle Erwartung vieler, die nie gedacht hätten, daß ich je
wieder herumreisen würde.
Während ich unter dieser seelischen Anfechtung war, ward
mir der Zustand des Neuen Jerusalem, das vom Himmel herunter
kommt, geoffenbart, daß einige fleischlich Gesinnte sich als eine
sichtbare, auß greifbaren Stoffen gemachte Stadt vorgestellt hatten.
Jch sah seine Schönheit und Herrlichkeit, seine Länge, Breite und
Höhe, alleß in schönem Verhältniß. Ich sah, daß alle, die im
Lichte Christi und im Glauben an ihn sind und im heiligen
Geiste, in welchem Christuö und seine Apostel und Propheten
waren, und in der Gnade, der Wahrheit und der Kraft Gotteß,
in dieser Stadt sind, Glieder derselben sind und daß Recht haben,
vom Baume de-8 Lebens- zu essen, welcher jeden Monat seine Frucht
gibt, und dessen Blätter den Völkern Heilung bringen (Ofstz. 22).
Die aber nicht in der Gnade Gottes, der Wahrheit, dem Licht,
dem Geist und der Kraft Gottes- sind, und die ,,dem heiligen Geist
widerstehen und die Gnade Gotteß aus Mutwillen ziehen« (Jud. 4),
die vom Glauben abgeirrt sind und die ,,Verheißungen, Offen-
barungen und Gingebrmgen verachten«, dieseö fmd die Hunde
und die Ungläubigen, die draußen sind (Offb. 22). Diese bilden
die große Stadt Babylon, die Verwirrung, und ihr Behältniö ist
die Macht der Finsterniö, und der böse Geist dez Jrrtumß umgibt
und bedeckt sie. Jn dieser großen Stadt Babylon sind die


% \picinclude{./210-219/p_s214.jpg} 
214 Kapitel 3711.
falschen Propheten, die in einem oerkehrten Geist und einer
falschen Krast stehen, daß Tier, das in der Gewalt des Drachen
ist, die Hure, die den heiligen Geist und Chriftuin ihren Gemahl
verlassen hat (Hos.-1). Aber dez Herrn Macht ist größer ale
alle Macht der Finsternis, alle falschen Propheten und ihre An-
beter: diese ,,gehören in den feurigen Pfuhl« (Oss. 19, 20). ....
Jch sah noch viele Dinge über daß himmlische Jerusalem, welche
aber schwer zu beschreiben und noch schwerer zu Verstehen wären ....
Während ich in Enfield wirkte, spürte ich einen Schaden,
der öfters- unter den Bekennern der Wahrheit vorkam; nämlich,
wenn sie in ein anderes Land zogen, heirateten sie unter Freunden,
bei denen sie stemd waren, und von denen man nicht wußte, ob sie
makellos: und ordentlich seien oder nicht. Und eine innere Stimme
hieß mich ihnen folgende-H Verfahren zur Verhütung derartiger
Mißstände zu empfehlen:
,,Alle Freunde, die sich verheiraten, Männer wie Frauen,
sollen, wenn sie aus einem anderen Land, einer andern Gegend
oder Jnsel kommen, der Männeroersammlung, der sie ihre Absicht
zu heiraten, vorlegen, eine Bescheinigung von der Männewer-
sammlung aus dem Ort ihrer Herkunft bringen. Denn da die
Männewersammlung aus Gläubigen besteht, so werden die
herumschwärmenden bösen Geister gebannt. Kommt nun einer
mit einer Bescheinigung oder einem Empfehlung?-schreiben einer
Männeroersammlung zu einer andern, so wird diese durch jene
erquickt, und sie kann die Sache getrost unternehmen. Dies wird
viel Verdruß ersparen. Und waö ihr ihnen dann in der Kraft
Gottes- zu sagen habt in Ermahnung und Lehre, daß tut in der
Kraft und dem Geist Gotteß; lasset sie die Pflichten und die Be-
deutung der Ehe wissen. Die Einigkeit im Geist und Kraft,
Licht und Weißheit von Gott möge unter allen Männeroersamm-
kungen in der ganzen Welt herrschen in dem Einen, dem Leben.
Laßt hiervon Abschriften in jedeö Land, jede Gegend und
Jnsel, wo Freunde sind, senden, damit alle Dinge heilig, rein
und gerecht bewahret bleiben in Einigkeit und Füedctt, und daß
Gott über alleß gepriesen werde unter euch, seinen Auöerwählten,
seinem Volk und Erbe, die ihr seine et-wählten Söhne und Töchter
und Erben seines Lebens seid. Soviel davon; meine Liebe in
dem, daß nicht ändert.
Den 1-1.deS 1. Monats 1671. G. F.


% \picinclude{./210-219/p_s215.jpg} 
Reise nach Amerika. Varbadoez. Jamaika. 215
Zu dieser Zeit trieb eö mich, den Herm also anzurufen:
,,Herr, Gott, Allmächtiger!
Fördere die Arbeit und schütze die Gerechtigkeit und Billig-
keit im Land. Steure der Boßheit und Ungerechtigkeit, Bedrückung
und Falschheit, Grausamkeit und Unbarmherzigkeit, auf daß
Barmherzigkeit und Gerechtigkeit möge überhand nehmen.
O, Herr, Gott! Nichte die Wahrheit im Lande aus und
schütze sie. Tilge auß alleß Laster, Hurerei, Abgötterei, den Geist
der Unzucht, welcher macht, daß das Volk dich nicht ehrt, noch
ihre Seelen, noch ihren Leib, noch das Christentum, noch Zucht,
noch Menschenwürde.
O, Herr, gib der Obrigkeit ins Herz, all diesem ungöttlichen
Wesen, dieser Gewalttätigkeit und Grausamkeit, der Gottlosigkeit und
dem Fluchen zu wehren, und alle schlechten Häuser und Spiel-
häuser au-Jzurotten, welche die Jugend und daß Volk verderben,
und sie deinem Reich entführen, in welcheß nichtß Unreineö je
eingehen kann. Solcheö Treiben fiihret die Leute in die Hölle.
Herr, reinige daß Land von allen diesen Dingen nach Deiner
Barmherzigkeit, daß Dein Zorn gestillet werde, o Gott, und nicht
über daß Land hereinbreche.
17. des 2. Monatß 1671. G. F.
Kapitel Islll.
Reise nach Amerika. Barbadoes. Jamaika.
Wie schon erwähnt, hatte ich zwei Töchter meiner Frau
zum König geschickt, um ihre Freisprechung zu erwirken, und sie
hatten auch seinen dießbezüglichen Befehl dem Besehlßhaber in
Laneashire gebracht; . . . aber der Sturm der Verfolgung war
gerade so mächtig geworden, daß man Mittel fand, sie weiter
gefangen zu halten. Als nun aber die Verfolgungen etwas
nachließen, trieb eS mich, Martha Fischer und eine andere Frau
aus- dem Kreise der Freunde zu veranlassen, abermalß zum König
zu gehen, um ihre Freilassung zu erbitten. Sie gingen im Glauben
an die Kraft dez Herrn, welcher sie Gnade finden ließ vor dem
König, so daß er einen befiegelten Freilassungöbefehl bewilligte,
nachdem sie fast zehn Jahre gefangen gewesen war, und ihre Gitter
mit Beschlag belegt, dergleichen kaum je in England war erhört
worden. Ich schickte die Freisprechang sofort zu ihr durch einen


% \picinclude{./210-219/p_s216.jpg} 
216
F Kapitel :07111.
reund, und zu ls . . »
befehl müsse deis essiceschsecse-Kleb ich she- wie sie den Frestassungs-
mit, daß eg über m. zu emmen lassen und teilte ihr auch
gehen nach Amerikaschsieesessssässsxsszn sel v1smHerM« überß Meer zu
Ich London eilen: da dag Sghrtzxm,. sobald es:-’ ihr möglich sei,
em. dee Zwischenzeit Sing ich nach fdisschstschon Re Abreise süsse
meme Frau kam, und dann rütt «g on-zu sohn Nous, biz
weil die Jahreßversammlung IMO etetgch mich zur Reise. Doch
zu derselben .... Dann, als unseraSsand, so blieb ich noch bis
mlch zu begleiten beabsichtigten bereit wchsss uns dee Freunde, die
6. Monate- 1671 nach Gravesend und elren, gmg sch, am 12« dez
 Z2 hegteszeten mich ck-xr ds2«’WtH?’e UI-meh reee de
. esse mit mir machten waren. T ' sse Freunde,
Idmundson, John RMS Joh;1StubbH Hofnaß Briggö, William
aueuskek, John Eartwr« t R s s o omon Eeeleee- Jameß
John Hull, Elisabeth HUIZVL u::exl·WHdderS,.George Pattison,
war eine Jacht und hieß ,,IndufMe,,ssT eeh ME ' Unser Schiss
Forster, und wir waren etwa 50 Passa er Kapttan hieß Thomaz
1 A15 wir etwa drei Wochen auf demgßxzzv. . . .
Wlr etwa yjer Seemeilm hinter UW ein S ter waren, bemerkten
sßzgse, es E ein maursscheß Piratensehifs djsdnsunxnser Kapitän
eme. ,, ommt,« sagte er - « — zu Verfolgen
Mgmt ez dunkel geworden ist sßkjxrxollen zum Abendessen Sehen,
Dietz sa te e · Z en sse unsere Spur Verlie «
ich .9. r, um die Reisenden zu beruhi en d ssrens
UI Mge an sich zu äugstigen. Die Fseutsdcesnis eg singen
Heek: b dtezt weil sie Gott vertrauten und keinerleseßzsechchsssaren
e ru e. Als d' , uk ihr
meiner Kajüte aus ie Sonne untergegangen war, sah ich om:
dunkel wurde, ände1:tersVs:1irds;IseKske1ss aus uns zukam. Alß ez
aber es änderte die seine auch, Untzchtßlxßtz   ihm auzzuweichen;
kamen der Kapitän und andere zu mir in o. e Unßj » Km der Nacht
mich, was sie mn sollten. Ich Mw t:neme.Kaiute und fragten
mann, und fragte sie, Wag sie für das HejI:,hFes?tsetz kein Schiffß-
ZK-lgabe nur zwei Wege: entweder wir müßtee dns; Fw fagtem
O en, oder hin d tkt . . u a Schiff über-
wie vorher. Gun he euzen und die gleiche Richtun ein al
uch sagte wenn etz Rss - g h ten
sicherlich auch hin- und derkreuzen U gubergseien, so werden sie
tzelange, fo sei daran gar nicht zu dxsnkxxlasddaz Uberholen an-
te viel schneller fahren sz - . . s a man ja sehe wie
a wir. Sie fragten mich wieder, maß


% \picinclude{./210-219/p_s217.jpg} 
Reise nach Amerika. Bardadoczß. Jamaika. 217
sie denn tun sollten: ,,denn,« sagten sie, ,,wenn die Schifföleute da-
malß den Rat deß Paulutz befolgt hätten, so wäre es ihnen
nicht so schlimm ergangen.« Jch erwiderte: ,,EZ ist eine Glaubens-
prüfung, und darum muß man auf den Herrn warten und auf
seinen Rat.« Während ich mich nun innerlich sammelte, zeigte
mir der Herr, daß er mit seinem Leben und mit seiner Kraft
zwischen uns und dem Schiff, das unö verfolgte, stehe. Ich
teilte diez dem Kapitän und den anderen mit, und daß ez nun
daß Beste sei, zu kreuzen und den rechten Kurz einzuschlagen.
Jch hieß sie auch alle Lichter au?-löschen außer dem einen, das
sie beim Steuer brauchten, und den Reisenden sagen, sie sollten
sich still und ruhig verhalten. Jn der Nacht etwa um 11 Uhr,
kam die Wache und sagte, sie seien ganz nahe hinter unö. Daß
beunruhigte einige der Reisenden. Jch richtete mich in meiner
Kajüte auf, und da der Mond noch nicht untergegangen war. sah
ich durch die Luke, daß sie ganz nahe waren. Jch wollte auf-
stehen und hinaußgehen; aber ich erinnerte mich der Worte des
Herrn, ,,daß er mit seinem Leben und seiner Kraft zwischen uns-
und ihnen stehe,« und legte mich wieder nieder. Der Kapitän
und einige der Schiff?-leute kamen abermals- und fragten, ob sie
nicht nach dieser oder jener Richtung steuern sollten? Jch sagte
ihnen, sie sollten machen, wie sie wollten. Da ging der Mond
vollends unter, ein neuer Wind erhob sich, und der Herr oerbarg
nnß vor ihnen; wir segelten rasch und sahen sie nicht mehr. Am
folgenden Tag, einem Ersten Tag, hatten wir eine öffentliche
Versammlung auf dem Schiffe, wie wir sie gewöhnlich während
der ganzen Reise an diesem Tage zu halten pflegten; und deö
Herrn Gegenwart war mächtig unter uns. Und ich ermahnte die
Leute, an Gottes Barmherzigkeit zu denken, die sie errettet; denn
sie wären jetzt vielleicht alle in den Händen der Türken, wenn
des Herrn Hand sie nicht errettet hätte. Etwa eine Woche
darauf suchten der Kapitän und einige der Schiffs-leute den
Reisenden einzureden, etz seien nicht türkische Seeräuber gewesen,
die uns verfolgten, sondern ein Kaufmamisßschiff, das nach den
Kanarischen Jnseln ging. A16 ich das hörte, fragte ich sie,
warum sie denn dann solcheß zu mir gesagt hätten? warum sie
die Reisenden beunruhigt hätten? und warum sie, um ihnen
davon zu fahren, den Kurz geändert hätten? Sie sollten sich
hüten, Gotteß Barmherzigkeit zu verachten. Später, als wir in


% \picinclude{./210-219/p_s218.jpg} 
218 Kapitel Kslll.
Varbadoes waren, kam ein maurischer Kaufmann und erzählte den
Leuten, die Mannschaft eines maurischen Piratenschiffs habe auf
dem Meer ein ungeheures Jachtschiss gesehen, das größte, das
sie je gesehen hätten, sie hätten es verfolgt, und seien schon ganz
nahe gewesen, aber es sei ein Geist darin gewesen, so daß sie es
nicht erobern konnten. Dies bestätigte uns in unserer Uberzeugung,
daß es ein maurisches Piratenschiff war, das uns verfolgte, und
daß es der Herr gewesen, der uns befreit hatte.
Ich war nicht seekrank gewesen auf der Reise, wie so viele
der Freunde und andere Reisende; aber alle die Wunden und
Schläge, die ich früher erlitten, die Krankheiten, die ich mir durch
die Kälte und die Gntbehrungen während meiner Gesangenschasten
zugezogen hatte, machten sich nun während der Reise wieder
geltend, so daß mein Magen sehr angegriffen war, und ich heftige
Schmerzen in allen Gliedern hatte. Es fing an, nachdem ich
etwa einen Monat auf der See war; zuerst schwitzte ich stark,
und an Kopf und Leib zeigten sich überall Pusteln, nnd meine
Hände und Füße wurden so geschwollen, daß ich nur mit Mühe
und unter großen Schmerzen meine Strümpfe und Pantoffeln
anziehen konnte; auf einmal hörte das Schwitzen auf, und
als ich in das heiße Klima kam, wo die anderen tüchtig schwitzten,
konnte ich gar nicht schwitzen, sondern mein Körper war heiß
und trocken und brennend, und was vorher in Pusteln aus-
gebrochen war, schlug jetzt nach innen auf Herz und Magen, so
daß ich sehr krank war und über alle Maßen schwach; dies
dauerte während der ganzen übrigen Zeit der Reise, während
der etwa vier Wochen, die wir noch auf dem Wasser waren.
Am frühen Morgen des 3. Tages des 8. Monats erblickten wir
die Jnsel Barbadoes, aber es dauerte noch bis zwischen neun und
zehn des Abends, ehe wir in den Hafen der Carlisle=Bay ein-
fuhren. Wir gingen sobald wie möglich ans Land, und ich be-
gab mich mit einigen Freunden in das Haus eines Freundes,
eines Kaufmanns namens Richard Forstall, der etwa zehn
Minuten von der Landungsbrücke wohnte. Aber ich war so
krank und schwach, daß ich sehr müde wurde von diesem kurzen
Gang, und Vollständig erschöpft ankam. Jch lag dort mehrere
Tage krank, und obgleich man mir mehrmals Mittel gab, um
mich schwitzen zu machen, so kam es doch nie zu einem rechten
Schweiß. Was sie mir gaben, oertrocknete eher meinen Körper


% \picinclude{./210-219/p_s219.jpg} 
Reise nach Amerika. Barbadoes. Jamaika. 219
noch mehr, und machte mich noch kränker, als ich sonst gewesen
wäre. Diese Schmerzen in allen Gliedern dauerten etwa drei
Wochen, und ich litt sehr, so daß ich kaum je Ruhe finden
konnte, aber ich war ziemlich getrost und der Geist ward Herr
über alle?-. Auch hinderte mich meine Krankheit nicht am Dienst
für die Wahrheit, sondern sowohl auf der See als in Barbadoeß,
ehe ich herum reisen konnte, gab ich verschiedene Schriften
heraus, die ein Freund für mich schrieb, und von denen ich einige
mit der ersten Gelegenheit nach England schickte, um gedruckt
zu werden .....
Weil ich so schwach war, daß ich nicht an die verschiedenen
Versammlungen reisen konnte, nahmen sich die andern Freunde
dez Werkes dez Herrn an; schon am Tage nach unserer Ankunft
hatten sie eine große Versammlung an der Landungzbrücke, und
nach derselben noch mehrere in verschiedenen Teilen der Jnsel,
maß die Bevölkerung sehr in Aufregung brachte, so daß viele zu
den Versammlungen kamen, worunter mehrere von hohem Rang;
denn sie hatten gehört, daß ich aus der Jnsel angekommen sei
und erwarteten, mich bei den Versammlungen zu sehen, da sie
nicht wußten, daß ich zu schwach war, um zu kommen. Meine
Schwachheit wich darum so lange nicht von mir, weil mein Ge-
müt zuerst sehr niedergedriickt war von der Schmutzigkeit und
Ungerechtigkeit und Gemeinheit der Leute, maß wie eine schwere
Last auf mir lag. Aber nachdem ich etwa einen Monat auf der
Jnsel gewesen war, wurde es mir etwaß leichter zu Mut, und
ich fühlte mich wieder etwaß kräftiger, so daß ich wieder umher
gehen konnte zu den Freunden .....
Weil ich aber doch nicht gut viel umher reisen konnte, so
kamen die Freunde auf der Jnsel überein, die Männer- und
Frauen-Versammlungen zur Ordnung der kirchlichen Angelegen-
heiten im Hause Thomaß Rouz, bei dem ich wohnte, abzuhalten,
so daß ich bei allen Versammlungen dabei war und recht für den
Herrn wirken konnte. Denn sie hatten in manchen Dingen Be-
lehrung nötig, weil sich auö Mangel an Vorsicht und Wachsam-
keit allerlei Unordnungen eingeschlichen hatten. Jch ermahnte sie,
besonderß in der Männeroersammlung, recht vorsichtig und wach-
sam in bezug auf das Heiraten zu sein und die Freunde zu
verhindern, in die Verwandtschaft zu heiraten, sowie auch zu
hastig vorzugehen bei Wiederoerheiratung nach dem Tode des

