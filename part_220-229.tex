% \picinclude{./220-229/p_s220.jpg} 
Mannes oder der Frau. Ich ermahnte sie, das in solchen Fällen
dem verstorbenen Teile die geziemende Ehrerbietung sollte bezeugt
werden. Ich wies sie auch daraus hin, wie unziemlich es sei,
ihre Kinder so früh einander zu verheiraten, mit dreizehn und
vierzehn Jahren, und was für Schäden und Nachteile aus solchen
frühen Heiraten entstehen. Ich ermahnte sie ferner, ihre 
Fußböden gründlich zu reinigen,\index{Reinlichkeit} ihre Häuser rein zu halten und auch
außerhalb der Versammlungen einander nicht mit verleumderischen\index{Verleumdung}
Reden zu schaden. Ich ermahnte sie, genaue Verzeichnisse 
zu führen über Geburten, Heiraten und 
Beerdigungen,\index{Gemeindebuch}\index{Buch über Geburten, Heiraten und Beerdigungen}
in eigens dazu bestimmten Büchern; auch sollten sie ein 
besonderes Buch führen über die Bestrafungen\index{Buch über Bestrafungen}
solcher, die von der Wahrheit abweichen und einen unordentlichen Wandel führen,
und über Buße und Wiederaufnahme solcher, die wieder zurück
kommen. Ich empfahl ihnen an, für geeignete Begräbnisplätze\index{Begräbnis}
zu sorgen, die an etlichen Orten noch fehlten. Ich gab ihnen
auch einige Räte inbetreff der Vermächtnisse,\index{Erbschaft} welche Freunde zu
beliebigem Gebrauch hinterlassen hatten, und wie sie darüber
verfügen sollten, und über allerlei andere kirchliche Angelegenheiten.
Inbetreff der Schwarzen oder Neger\footnote{Das Wort \zitat{Neger} wurde
im Originaltext verwendet. Würde die Übersetzerin und der Autor vermutlich 
Heute nicht mehr benutzen},\index{Sklaverei} hieß ich versuchen, 
dieselben in der Furcht Gottes zu unterweisen, sowohl die gekauften
als die, welche in der Familie geboren wurden, damit alle dazu
kommen möchten, den Herrn zu kennen, so das jeder Hausvater
mit Josua sagen könne: \zitat{ich aber und mein Haus wollen dem
Herrn dienen}. Ich ermahnte sie auch ihre Aufseher dazu zu
bringen, mild und freundlich gegen die Neger zu sein, und sie
nicht grausam zu behandeln, wie viele es taten und noch tun,
und sie, wenn sie einige Jahre als Sklaven gedient, frei zulassen
Viele köstliche, herrliche Dinge wurden in diesen Versammlungen
offenbar, durch den Geist und die Kraft Gottes, zur Erbauung,
Ausrichtung und Stärkung der Freunde im Glauben und der
Heiligen Ordnung des Evangeliums [...]

Als es mir wieder besser ging, machten wir dem 
Gouverneur\index{Personen!Gouverneur von Barbados}
einen Besuch. Er empfing uns sehr höflich und behandelte 
uns sehr freundlich und hieß uns mit ihm zu Mittag essen. In der
gleichen Woche ging ich nach Bridgetown\index{Bridgetown}; und da die Behörden,
die militärischen wie die andern, von meinem Besuch beim Gouverneur 
und seiner freundlichen Aufnahme gehört hatten, so kamen
aus allen Teilen des Landes viele Leute von hohem Rang,
% \picinclude{./220-229/p_s221.jpg} 
Richter, Friedensrichter, Oberste, Hauptleute zu dieser 
Versammlung [...]\index{Prominente Versammlungsbesucher} 

Von den Freunden, die mit mir gekommen waren,
gingen viele nach Jamaika und andere Orte, so das wenige mit
mir in Barbados blieben. Wir hatten viele große und schöne
Versammlungen [...] sie waren friedlich und nicht gestört von
Seiten der Regierung; jedoch gehässige Priester\index{Andachsstörung} 
[...] und Baptisten\index{Baptisten} [...] und Fromme brachten 
Schmähschriften\index{Schmähschriften} gegen uns [...]
diesen traten wir mit einer Schrift entgegen,\index{Verteidigungsschriften} 
die im Namen der sogenannten Quäker sollte verbreitet werden, um die Wahrheit
und die Freunde von solchen falschen Anschuldigungen zu reinigen.
[...] Es hieß darin unter anderem: [...] 

\grosszitat{ 
  eine Verleumdung,
  die sie gegen uns ausstreuten ist, das wir die Neger zu Aufständen\index{Sklavenbefreihung}
  anstiften, und gerade das Verabscheuen wir im Innersten; der
  Herr, der die Herzen prüft, weiß es, und kann uns das Zeugnis
  geben, das dies eine ganz abscheuliche Unwahrheit ist.  Wir haben
  in Bezug auf sie gesagt, man solle sie lehren, nüchtern und 
  rechtschaffen zu sein, Gott zu fürchten und ihre Herren und Herrinnen
  zu lieben und treu und fleißig ihren Herren zu dienen; dann
  würden ihre Herren und ihre Aufseher sie lieben und gütig und
  freundlich behandeln; auch sollten sie ihre Weiber nicht schlagen
  noch die Weiber ihre Männer, und die Männer sollten nicht
  mehrere Weiber haben; sie sollten nicht stehlen noch sich betrinken,
  nicht Ehebruch noch Unzucht treiben, nicht fluchen, nicht schwören,
  nicht lügen oder sich unter einander beschimpfen, denn es sei
  etwas in ihnen, das; ihnen sage, sie sollten diese und andere
  schlechte Dinge nicht tun. Wenn sie sie aber dennoch tun, so
  lehrten wir sie, das es nur zwei Wege\index{Dualismus} gibt, der eine, der zum
  Himmel führt, den die Gerechten gehen, und der andere, der
  zur Hölle führt, den die Gottlosen gehen und die Ehebrecher,
  Hurer, Mörder und Lügner. Zu den Einen wird der Herr sagen:
  Kommet, ihr Gerechten meines Vaters, erbet das Reich!\index{Reich Gottes} Zu
  den Andern aber wird er sagen: Gehet hin, ihr Verfluchten in
  das ewige Feuer! und so werden die Ungerechten in die ewige
  Pein gehen, die Gerechten aber in das ewige Leben (Matth. 25).\index{Bibel!Matth. 25}

  Wisset, Freunde es ist keine Schande für einen Hausvater,
  die Seinen selber zu unterweisen oder jemand anders es für ihn
  tun heißen, vielmehr ist es eine wichtige Pflicht, die ihm zu
  tun auferlegt ist. Abraham und Josua haben es also gemacht.
  Vom ersteren heißt es Genesis 18,19,5\index{Bibel!Genesis 18,19:5}: \zitat{er wird befehlen seinen
  % \picinclude{./220-229/p_s222.jpg} 
  Kindern und seinem Haus, das sie des Herrn Wege halten [...] },
  und der zweite sagt, Josua 24,15\index{Bibel!Josua 24:15}: 
  \zitat{erwählet euch heute welchem
  ihr dienen wollt; ich aber und mein Haus wollen dem Herrn
  dienen}. Wir erklären, das wir es für unsere Pflicht halten,
  mit denen und für die zu beten, die unsrem Hause angehören,
  und sie zu lehren und zu ermahnen; denn es ist dies ein Befehl
  vom Herrn und der Ungehorsam hieregen wird sein Missfallen
  erregen, wie wir Jeremias 10, 25\index{Bibel!Jeremias 10:25} 
  sehen können: \zitat{Schütte deinen
  Zorn über die Heiden, die dich nicht kennen und über die 
  Familien, die deinen Namen nicht anrufen}. Nun bilden die
  Reger, die Rothäuter\index{Rothäuter}, die Twanies\index{Twanies}, 
  die Indianer\index{Indianer} überall einen
  großen Teil der Familien hier auf dieser Insel, und es wird 
  Rechenschaft über sie gefordert werden von dem, der kommen wird zu
  richten die Lebendigen und die Toten am großen Tage des
  Gerichts: \zitat{da ein jeder empfangen wird seinen Lohn, nach dem
  er gehandelt hat, es sei gut oder böse}, wenn er \zitat{wird 
  geoffenbaret werden mit Feuerflammen, Rache zu geben über die so
  Gott nicht erkennen} [...] und \zitat{es werden in den letzten Tagen
  Spötter kommen, die nach den eigenen Lüsten wandeln} [....] \zitat{es
  wird aber des Herrn Tag kommen wie ein Dieb in der Nacht} [...]
  wie 2. Thess. 1,8\index{Bibel!Thess. 2. 01:8@2. Thess. 1,8} und 
  2. Pet. 3\index{Bibel!2. Pet. 3@2. Pet. 3} zu sehen ist.
}


Die Veranlassung zu diesem Gerücht, das wir versuchten die
Neger aufzuhetzen, hatten unsre Gegner daraus geschöpft, das
wir Versammlungen mit und unter den Negern gehabt hatten,
denn sowohl ich als andere der Freunde hatten mehrere 
Versammlungen mit ihnen in verschiedenen Plantagen, in denen wir
sie zu Rechtschaffenheit, zur Keuschheit, zur Nüchternheit und zur
Frömmigkeit ermahnten, und zum Gehorsam gegen ihre Herrn
und Meister, also gerade das Gegenteil von dem, was unsre
übelwollenden Gegner böswillig gegen uns ausstreuten. [...]
Ehe ich die Insel verließ, schrieb ich folgenden Brief an
meine Frau:

\grosszitat{
Mein liebes Herz,

\bigskip 

Welcher meine Liebe gehört, sowie allen Kindern im Samen
des Lebens, der sich nicht verändert, sondern größer ist als alles,
gelobt sei der Herr ewiglich. Ich habe unaussprechlich an Seele
und Leib zu erdulden gehabt; aber der Gott des Himmels sei
gelobt, seine Wahrheit geht über alles. Ich bin jetzt gesund, und
so der Herr will, gehe ich in einigen Tagen von Barbados nach
% \picinclude{./220-229/p_s223.jpg} 
Jamaika\index{Jamaika} und gedenke nur kurze Zeit dort zu bleiben. Ich hoffe,
das ihr alle im Samen des Lebens\index{Samen des Lebens}, frei von aller Kümmernis,
bewahret bleibet. Die Freunde sind im allgemeinen wohl. Grüße
mir die Freunde, die nach mir fragen. Soviel diesmal. Meine
Liebe im Samen und Leben, die nicht wechseln.

\bigskip 

Barbados, 6. des 11. Monats 1671\index{Jahr!1671}. G. F.
}

Ich schiffte mich am 8. des 11. Monats 1671 in Barbados
für Jamaika ein [...] Wir hatten eine gute, rasche Überfahrt, [...]
und trafen in Jamaika James Lancaster\index{Personen!Lancaster, James}, 
John Eartwright\index{Personen!Eartwright, John} und
George Pattison\index{Personen!Pattison, George} wieder, die 
eifrig im Dienste der Wahrheit gearbeitet hatten, dem wir 
uns nun auch widmeten, wir reisten
aus der Insel hin und her; es ist ein recht schönes Land, doch
sind die Leute zum Teil recht verdorben und ausschweifend. Wir
wirkten viel. Es war eine große Belehrung und viele nahmen
die Wahrheit auf, worunter manche angesehene Leute. Wir
hatten viele Versammlungen hier, die zahlreich und ganz ruhig
waren. Die Leute begegneten uns sehr anständig und niemand
tat den Mund gegen uns auf. Ich war zweimal beim Gouverneur 
und den Behörden, die sehr freundlich gegen mich waren.

Etwa eine Woche nach meiner Ankunft in Jamaika schied
Elisabeth Hooton\index{Personen!Hooton, Elisabeth}, eine sehr 
alte Frau, die viel im Dienst der
Wahrheit umhergereist war und viel dafür gelitten, aus diesem
Leben. Sie war noch am Tage vor ihrem Tode gesund und
schied in Frieden, und gab noch im Sterben der Wahrheit die
Ehre. Nachdem wir etwa sieben Wochen in Jamaika gewesen,
und unter den dortigen Freunden etwas Ordnung geschaffen und
mehrere Versammlungen unter ihnen eingerichtet hatten, ließen
wir Solomon Eccles\index{Personen!Eccles, Solomon} dort, und 
schifften uns für Maryland\index{Maryland} ein [...]

Ehe ich Jamaika verließ schrieb ich noch einmal einen Brief
an meine Frau:

\grosszitat{
Mein liebes Herz,

\bigskip 

Dir und den Kindern meine Liebe in dem, das über allem
ist und sich nicht verändert, und allen Freunden die bei euch sind,
Ich bin nun etwa fünf Wochen in Jamaika gewesen. Den
Freunden geht es im ganzen gut und wir haben große Bekehrungen 
gehabt; aber es würde zu weit führen, über alles zu
schreiben; überall warten Leiden meiner, aber der gesegnete Same
ist über allem. Der Herr sei gelobt, welcher Herr ist über Land
% \picinclude{./220-229/p_s224.jpg} 
und Meer und alles was darinnen ist. Wir haben im Sinn,
etwa anfangs des nächsten Monats von hier abzureisen nach
Maryland zu, so der Herr will. Bleibet alle miteinander im
Samen des Herrn; in seiner Wahrheit bleibe ich in der Liebe zu
euch allen

\begin{flushright}Jamaika, 23. des 12. Monats 1671. G. F.\end{flushright}
}

\chapter[Nordamerika unter Engländern und Indianern.]{Nordamerika unter Engländern und Indianern.}

\begin{center}
\textbf{Arbeit in Nordamerika unter Engländern und Indianern.}
\end{center}


Wir schifften uns am 8. des 1. Monats 1671 ein; und da
wir schlechten Wind hatten, segelten wir eine ganze Woche hin
und her, ehe wir von Jamaika fort kamen. Es war eine 
schwierige und gefahrvolle Reise, besonders als wir den Golf von
Florida passierten, wo wir manche Schwierigkeiten durch Wind
und Sturm zu bestehen hatten. Aber der große Gott, welcher
Herr ist über Meer und Land, welcher aus den Flügeln des
Windes dahin fährt, bewahrte uns durch seine Kraft vor vielen
großen Gefahren, wenn bei dem Ungestüm des Wetters unser
Schiff oft nahe daran war umzuschlagen, und das Tauwerk 
größtenteils zerbrochen wurde. Wahrlich, wir merkten, das der Herr ein
Gott der Nähe ist, und hört auf das Flehen seines Volkes. Denn
als die Winde so stark und heftig tobten, und es so mächtig
stürmte, das die Schiffsleute sich nicht zu helfen wussten, und
das Schiff sich selbst überließen, da beteten wir zum Herrn, welcher
uns gnädig erhörte, Wind und Wellen stillte und uns günstiges
Wetter gab, so das wir uns unsrer Errettung freuen durften.
Gelobt und gepriesen sei der herrliche Name des Herrn, der
Macht hat über alles, dem Wind und Wellen gehorchen. [...]

Wir waren etwa sechs bis sieben Wochen unterwegs von
Jamaika nach Maryland [...] Dort trafen wir John Burnyeat,\index{Personen!Burnyeat, John}
der die Absicht hatte, sich bald nach England einzuschiffen, aber
als wir kamen, änderte er seinen Vorsatz und schloss sich uns an
zum Dienst für den Herrn [...] Er hatte eine Versammlung für
alle Freunde von Maryland veranstaltet, damit er sie alle 
miteinander sehe, um Abschied von ihnen zu nehmen; und nun
fügte es die Vorsehung Gottes\index{Vorsehung Gottes} so, 
das wir gerade zur rechten
Zeit landeten, um dieser Versammlung beizuwohnen; [...] Es
war eine sehr große Versammlung, die vier Tage dauerte [...]
% \picinclude{./220-229/p_s225.jpg} 
Nach der allgemeinen Versammlung fingen die Männer- und
Frauen-Versammlungen an [...] Hernach gingen wir nach einem
andern Orte, die Klippen genannt, wo eine andere große 
Versammlung stattfinden sollte. Wir gingen einen Teil des Weges
zu Land, den Rest zu Wasser; und da sich ein Sturm erhob, stieß
unser Boot auf und wäre fast zertrümmert worden, und das
Wasser drang herein. Ich schwitzte stark, da ich sehr warm aus
der Versammlung gekommen war, und nun wurde ich vom Wasser
ganz durchnässt; aber weil ich Glauben\index{Glaube} hatte in die göttliche Kraft,
wurde ich vor Schaden bewahrt, der Herr sei gepriesen [...]
Wir hatten hier auch eine Männer- und Frauen-Versammlung,
und in vielen dieser Versammlungen wurde die Angelegenheit der
Kirche geordnet.

Nach diesen Versammlungen trennten wir uns und verteilten
uns auf die verschiedenen Küsten, zum Dienst der Wahrheit.
James Lancaster\index{Personen!Lancaster, James} und 
John Eartwright\index{Personen!Eartwright, John} gingen zu Wasser nach
Neu-England\index{Neuengland}; William Edmundson\index{Personen!Edmundson, William}
und drei andere Freunde schifften sich für Virginia\index{Virginia} ein, 
wo die Dinge sehr in Unordnung geraten waren; John Burnyeat\index{Personen!Burnyeat, John}, 
Robert Widders\index{Personen!Widders, Robert}, George 
Pattison\index{Personen!Pattison, George}
und ich mit einigen andern Freunden gingen mit einem Boot
zur Ostküste\index{Ostküste (Nordamerika)} und hatten dort am Ersten Tag eine 
Versammlung, wo viele die Wahrheit mit Freude aufnahmen und die
Freunde reichlich erquickt wurden. Es war eine große, selige
Versammlung, und es waren mehrere Personen von Rang aus der 
Gegend dabei, darunter zwei Friedensrichter. Es kam über
mich vom Herrn, dem \zitat{Kaiser} der Indianer\index{Indianer in 
der Andacht} und seinen \zitat{Königen}
sagen zu lassen, sie sollten zu dieser Versammlung kommen. Der
\zitat{Kaiser} kam und wohnte ihr bei, aber seine \zitat{Könige}, welche weiter
weg wohnten, konnten nicht zur rechten Zeit kommen; aber sie
kamen später nach, mit ihren Leuten. Ich hatte am Abend zwei
mal eine gute Zeit mit ihnen, und sie hörten das Wort des Herrn
gerne und bekannten sich dazu. Ich bat sie, das, was ich ihnen
sagte, dann auch ihrem Volke zu sagen und ihm zu verkünden,
das Gott jetzt die Hütte des Zeugnisses in der Wüste aufrichte
und das Panier und segensreiche Zeichen seiner Gerechtigkeit. Sie
benahmen sich sehr anständig und fragten, wann die nächste 
Versammlung sein werde, sie wollten dazu herkommen; aber sie
erzählten uns, sie hätten eine heftige Auseinandersetzung gehabt
mit ihren Räten, wegen ihres Kommens. Am folgenden Tage
% \picinclude{./220-229/p_s226.jpg} 
traten wir unsre Reise nach Neu-England an, eine schwierige
Reise, durch Wälder und Sümpfe und große Flüsse; dann mussten
wir die Wildnis passieren, die jetzt West-Jersy\index{West-Jersy} genannt wird,
die aber damals nicht von Engländern bewohnt war; einen ganzen
Tag reisten wir, ohne Mann oder Frau, Haus oder Wohnort
zu treffen. Zuweilen nächtigten wir im Wald bei einem Feuer,
zuweilen in den Hütten und Häusern der Imdianer. Eines Abends
kamen wir in eine indianische Ortschaft und übernachteten beim
\zitat{König}, der ein sehr achtenswerter Mann war; er sowohl als sein
Weib nahmen uns sehr liebevoll auf, und seine Dienerschaft 
behandelte uns sehr ehrerbietig. Sie gaben uns Matten, um darauf
zu schlafen, aber zu essen hatten sie wenig, da sie an dem Tage
wenig gefangen hatten. In einer andern indianischen Ortschaft, in
der wir uns aufhielten, kam der \zitat{König} zu uns; er sprach ein
wenig englisch. Ich redete viel mit ihm und auch mit seinen
Leuten, und sie waren sehr lieb mit uns. Schließlich kamen
wir nach Middletown,\index{Middletown} einer englischen Pflanzung 
in Ost-Jersy\index{Ost-Jersy};
doch konnten wir nicht zu einer Versammlung bleiben, da es uns
trieb, rechtzeitig zur Halbjahresversammlung in der Oysterbay,\index{Oysterbay}
auf Long-Island,\index{Long-Island} zu sein [...] Ein Freund, 
Richard Hartshorn,\index{Personen!Hartshorn, Richard}
setzte uns in seinem eigenen Boote über nach Long-Island, und
am zweitfolgenden Morgen erreichten wir die Oysterbay, wo wir der
Halbjahresversammlung beiwohnten [...] Nachdem dann die
Freunde wieder nach Hause gegangen waren, blieben wir noch
einige Tage auf der Insel und warteten dann in der Oysterbay
auf günstigen Wind, um nach Rhode-Island\index{Rhode-Island} zu gehen [...]
Dort kamen wir am 13. des 3. Monats an und wurden mit
Freuden von den Freunden aufgenommen. Am nächsten
Ersten Tage hatten wir eine große Versammlung, welcher der
Unterstatthalter und mehrere von der Behörde beiwohnten. Sie
wurden mächtig von der Wahrheit ergriffen. In der daraus
folgenden Woche sand hier die Jahresveriammlung für alle Freunde
von Neu-England und den übrigen angrenzenden Kolonien statt;
außer sehr vielen Freunden, die in diesen Gegenden lebten, kamen
noch John Stubbs\index{Personen!Stubbs, John} aus Barbados 
und James Lancaster\index{Personen!Lancaster, James} und
John Cartwrigth,\index{Personen!Cartwrigth, John} von 
verschiedenen Seiten, um ihr beizuwohnen.
Diese Versammlung dauerte sechs Tage; an den vier ersten Tagen
waren allgemeine, gottesdienstliche Versammlungen, zu welchen
eine große Menge Leute kamen; denn sie hatten keinen Priester
% \picinclude{./220-229/p_s227.jpg} 
in Rhode-Island und darum keinerlei bestimmte Form irgend
einer Art von Gottesdienst; und weil der Unterstatthalter mit
mehreren von der Regierung täglich\index{Tägliche Andacht} 
zu den Versammlungen kamen, wurden die Leute ermutigt, so 
das sie von allen Seiten herbei strömten. Wir hatten ein 
gesegnetes Wirken unter ihnen, und
die Wahrheit fand gute Aufnahme. Ich habe selten Leute von
dieser Art mit mehr Aufmerksamkeit, Fleiß und Liebe zuhören
sehen, als diese es im ganzen während der Vier Tage taten, was
auch andere Freunde beobachteten. Nachdem die öffentlichen
Versammlungen zu Ende waren, begann die Männerversammlung,
welche sehr zahlreich, köstlich und feierlich war; und tags daraus
war die Frauenversamnrlung, die ebenfalls sehr zahlreich und
feierlich war. Da diese beiden Versammlungen zur Ordnung
kirchlicher Angelegenheiten veranstaltet waren, so wurden viele
wichtige Dinge eröffnet und mitgeteilt, durch Rat\index{Rat}, 
Belehrung\index{Belehrung} und Unterweisung,\index{Unterweisung} 
für das Verhalten in den verschiedenen in Frage
kommenden Verrichtungen, damit alles rein, lieblich und kräftig
unter ihnen erhalten werde. Verschiedene Männer- und Frauen-
Versammlungen für andere Gegenden wurden in diesen beiden
Versammlungen beschlossen und eingerichtet, zur Fürsorge für die
Armen und andere kirchliche Angelegenheiten, und damit dafür
gesorgt werde, das alle, die die Wahrheit bekennen, auch nach
dem Evangelium wandeln. Als diese große Versammlung auf
Rhode-Island zu Ende war, wurde den Freunden der Abschied
etwas schwer; denn die herrliche Macht des Herrn, die über allen
war, und seine gesegnete Wahrheit und sein Leben, die sich über sie
ausgossen, hatten sie so untereinander verbunden und vereinigt,
das sie zwei Tage damit zubrachten, um sich unter einander und
von den Freunden auf der Insel zu verabschieden. Darauf gingen
sie fort, mächtig erfüllt von der Gegenwart und der Kraft Gottes,
freudigen Herzens, jeder in seine Heimat, nach den verschiedenen
Kolonien, in denen sie lebten. [...]

Während dieser Zeit fand eine Vermählung von zweien von
den Freunden statt, der wir beiwohnten. Es war im Hause
eines Freundes, der früher Gouverneur\index{Gouverneur} dieser Gegend gewesen war;
drei Friedensrichter und viele, die nicht zu uns 
gehörten\index{Fremde in der Andacht}, waren
zugegen; aber alle, diese sowohl als die Freunde, sagten, sie hätten
noch nie eine so andächtige Zusammenkunft gesehen bei einem
derartigen Anlass, noch eine so feierliche Vermählung und eine
% \picinclude{./220-229/p_s228.jpg} 
so vorzügliche Ordnung. Dermaßen durchdrang die Wahrheit
alle. Es ist hoffentlich für viele ein gutes Beispiel gewesen, denn
sie sind aus allen Teilen des Landes dazu gekommen. Ich hatte
innerlich viel durchzumachen wegen der Ranter\index{Renter}\index{Ranter} in dieser Gegend,
die eine Versammlung,\index{Versammlungsstörung} der ich nicht beigewohnt hatte, gestört
hatten. Ich zeigte darum eine Versammlung unter ihnen an, im
Glauben, das der Herr mir Macht über sie geben werde, was
er auch tat, zu seiner Ehre und Verherrlichung. Sein Name
sei gelobt ewiglich [...]

Darauf hatten wir eine Versammlung in Providence\index{Providence} [...]
ebenso in Narraganset\index{Narraganset} [...] Von dort ging ich zu der Insel
Shelter.\index{Shelter} [...] Dort hatten wir am Tage nach unsrer Ankunft,
einem Ersten Tage, eine Versammlung. In der gleichen Woche hatte
ich eine unter den Indianern,\index{Versammlung mit Indianern} 
ihr \zitat{König} war zugegen und sein
Rat und einige hundert Indianer; sie setzten sich unter uns, ganz
wie die Freunde taten, und hörten aufmerksam zu, während ich
durch einen Dolmetscher, einen Indianer, der gut englisch sprach,
zu ihnen redete. Sie waren nach der Versammlung sehr lieb
und bekannten, das, was man ihnen gesagt habe, sei Wahrheit
gewesen [...].

Wir reisten nun umher und kamen schließlich nach Shrewsbury,\index{Shrewsbury}
wo sich etwas ereignete, das damals eine wichtige Probe für uns
war. John Jay,\index{Personen!Jay, John} ein Freund aus Barbadoes, der mit uns von
Rhode-Island gekommen war und uns durch die Wälder von
Maryland begleiten wollte, bestieg ein Pferd, um es zu versuchen;
das Pferd sing an zu galoppieren und warf ihn ab, so das er
auf den Kopf fiel und den Hals brach, wie die Leute sagten.
Einige hoben ihn auf, in der Meinung, er sei tot, und trugen
ihn ein Stück weit und legten ihn unter einen Baum. Ich ging
sogleich zu ihm, und als ich ihn anrührte, hielt ich ihn auch für
tot. Während ich neben ihm stand, und ihn und die Seinen
beklagte, fuhr ich ihm durch die Haare, wobei sein Kopf sich hin
und her drehte, so schlaff war sein Hals. Nun nahm ich seinen
Kopf zwischen meine beiden Hände, und, indem ich meine Knie
gegen den Baum stemmte, hob ich seinen Kopf in die Höhe und
sah, das da nichts zerrissen oder gebrochen war; ich faste ihn
nun mit einer Hand unter dem Kinn und mit der andern hinten
am Kopf und bewegte seinen Kopf zwei oder dreimal mit aller
Kraft hin und her und renkte ihn ein. Ich bemerkte bald, wie
% \picinclude{./220-229/p_s229.jpg} 
sein Hals wieder anfing, Halt zu bekommen; darauf fing er an
zu röcheln und gleich darauf zu atmen. Die Leute waren entsetzt, 
aber ich hieß sie, guten Muts zu sein, Glauben zu haben
und ihn ins Haus zu tragen. Sie taten es und legten ihn neben
das Feuer. Ich hieß sie, ihm etwas Warmes zu trinken zu geben
und ihn zu Bett zu bringen. Nach einer Weile fing er an zu sprechen,
aber er wusste nicht, was mit ihm geschehen war. Am folgenden
Tage zogen wir weiter, und er mit uns, ziemlich wohl, etwa
16 Meilen zu einer Versammlung nach Middletown\index{Middletown} durch Wälder
und Sümpfe und über einen Fluss, wo wir unsere Pferde hinüber
schwimmen ließen und selber aus einem hohlen Baumstamm hinüber
setzten. Er reiste noch viele hundert Meilen mit uns [...]

Wir hatten hier eine herrliche Versammlung [...] Nach der
selben gingen wir nach Middletown-Harbour, etwa fünf Meilen weit,
um am nächsten Tage unsre große Reise anzutreten, durch die
Wälder nach Maryland. Unsre Führer waren Indianer. Ich
beschloss, den Weg durch die Wälder auf der andern Seite der
Delawara-Bay\index{Delawara-Bay} zu nehmen, um die Flüsse und Buchten so viel wie
möglich zu vermeiden. Wir machten uns am 9. des 7. Monats.
auf den Weg und kamen durch viele indianische Ortschaften und über
mehrere Flüsse und Sümpfe; als wir etwa 40 Meilen weit
geritten waren, machten wir ein Feuer für die Nacht und legten
uns daneben. Wenn wir zu Indianern kamen, verkündeten
wir ihnen den Tag des Herrn. Tags darauf reisten wir 50 Meilen.
Nachts fanden wir ein altes Haus, aus dem die Indianer die
Leute vertrieben hatten; wir machten ein Feuer und blieben dort
am Eingang der Delawara-Bay. Am folgenden Tage ließen wir
unsre Pferde etwa eine Meile weit über den Fluss schwimmen,
nach der Insel Upper-Dinidock, und darauf aufs Festland. Wir
selber hatten von den Indianern ein Kanu gemietet und uns
darin von ihnen übersetzen lassen [...]

Dann gingen wir nach Newcastle,\index{Newcastle} jetzt Neu-Amsterdam\index{Neu-Amsterdam}
genannt; [...] am 16. des 7. Monats zogen wir weiter [....]
Nach einer beschwerlichen Reise erreichten wir das Haus Robert
Harwoods\index{Personen!Harwoods, Robert} in Miles River\index{Miles River} 
in Maryland\index{Maryland} und wohnten am
folgenden Tage einer Versammlung bei. [...] Von da gings
nach dem Kentischen Ufer, [...] und dann zu Wasser, etwa zwanzig
Meilen weit, zu einer sehr großen Versammlung [...] Es war
eine gesegnete Versammlung und von großem Nutzen sowohl zur