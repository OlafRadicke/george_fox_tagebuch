% \picinclude{./220-229/p_s220.jpg} 
220 Kapitel Illlll.
Manneö oder der Frau. Jch ermahnte sie, daß in solchen Fällen
dem verstorbenen Teile die geziemende Ehrerbietung sollte bezeugt
werden. Jch wieß sie auch daraus hin, wie unziemlich eö sei,
ihre Kinder so früh einander zu verheiraten, mit dreizehn und
vierzehn Jahren, und waß für Schäden und Nachteile auö solchen
frühen Heiraten entstehen. Jch ermahnte sie ferner, ihre Fuß-
böden gründlich zu reinigen, ihre Häuser rein zu halten und auch
außerhalb der Versammlungen einander nicht mit verleum-
derischen Reden zu schaden. Jch ermahnte sie, genaue Ver-
zeichnisse zu führen über Geburten, Heiraten und Beerdigungen,
in eigenö dazu bestimmten Büchern; auch sollten sie ein be-
sonderez Buch führen über die Bestrafungen solcher, die von der
Wahrheit abweichen und einen unordentlichen Wandel führen,
und über Buße und Wiederaufnahme solcher, die wieder zurück
kommen. Jch empfahl ihnen an, sür geeignete Begräbniöplätze
zu sorgen, die an etlichen Orten noch fehlten. Jch gab ihnen
auch einige Räte inbetreff der Vermäehtnisse, welche Freunde zu
beliebigem Gebrauch hinterlassen hatten, und wie sie darüber
verfügen sollten, und über allerlei andere kirchliche Angelegenheiten.
 Jnbetreff der Schwarzen oder Neger, hieß ich versuchen, die-
selben in der Furcht Gotteö zu unterweisen, sowohl die gekauften
alö die, welche in der Familie geboren wurden, damit alle dazu
kommen möchten, den Herm zu kennen, so daß jeder Hau?-vater
mit Josua sagen könne: ,,ich aber und mein Haus wollen dem
Herrn dienen«. Jch ermahnte sie auch ihre Aufseher dazu zu
bringen, mild und freundlich gegen die Neger zu sein, und sie
nicht grausam zu behandeln, wie viele es taten und noch run,
und sie, wenn sie einige Jahre alß Sklaven gedient, freizulasseng
Viele köstliche, herrliche Dinge wurden in diesen Versammlungen
offenbar, durch den Geist und die Kraft Gotteß, zur Erbauung,
Ausrichtung und Stärkung der Freunde im Glauben und der
Heiligen Ordnung dez Evangeliums .....
A18 e3 mir wieder besser ging, machten wir dem Gouverneur
einen Besuch. Er empfing unö sehr höflich und behandelte uns-
sehr freundlich und hieß unß mit ihm zu Mittag essen. Jn der
gleichen Woche ging ich nach Bridge-Town; und da die Behörden,
die militärischen wie die andern, von meinem Besuch beim Gou-
verneur und seiner freundlichen Aufnahme gehört hatten, so kamen
aus allen Teilen det? Lnndetz viele Leute von hohem Rang,


% \picinclude{./220-229/p_s221.jpg} 
Reise nach Amerika. Barbadoeö. Jamaika. 221
Richter, Friedenßrichter, Oberste, Hauptleute zu dieser Versamm-
lung ..... Von den Freunden, die mit mir gekommen waren,
gingen viele nach Jamaika und andere Orte, so daß wenige mit
mir in Barbadoeö blieben. Wir hatten viele große und schöne
Versammlungen .... sie waren friedlich und nicht gestört von
Seiten der Regierung; jedoch gehässige Priester .... und Bap-
tisten .... und Fromme brachten Schmähschriften gegen unß ....
diesen traten wir mit einer Schrift entgegen, die im Namen der
sogenannten Quäker sollte verbreitet werden, um die Wahrheit
und die Freunde von solchen falschen Anschuldigungen zu reinigen.
.... E3 hieß darin unter anderem: .... ,,eine Verleumdung,
die sie gegen unß au?-streuten ist, daß wir die Neger zu Aufftänden
anstiften, und gerade das Verabscheuen wir im Jnnersten; der
Herr, der die Herzen prüft, weiß eß, und kann uns daß Zeugniz
geben, daß dietz eine ganz abscheuliche Unwahrheit ist.  Wir haben
in bezug aus sie gesagt, man solle sie lehren, nüchtern und recht-
schaffen zu sein, Gott zu fürchten und ihre Herren und Herrinnen
zu lieben und treu und fleißig ihren Herren zu dienen; dann
würden ihre Herren und ihre Aufseher sie lieben und gütig und
freundlich behandeln; auch sollten sie ihre Weiber nicht schlagen
noch die Weiber ihre Männer, und die Männer sollten nicht
mehrere Weiber haben; sie sollten nicht stehlen noch sich betrinken,
nicht Ehebruch noch Unzucht treiben, nicht fluchen, nicht schwören,
nicht lügen oder sich unter einander beschimpfen, denn es sei
etwas in ihnen, daß; ihnen sage, sie sollten diese und andere
schlechte Dinge nicht tun. .s’’ Wenn sie sie aber dennoch tun, so
kehrten wir sie, daß es- nur zwei Wege gibt, der eine, der zum
Himmel sührt, den die Gerechten gehen, und der andere, der
zur Hölle führt, den die Gottlosen gehen und die Ghebrecher,
Hurer, Mörder und Lügner. Zu den Einen wird der Herr sagen:
Kommet, ihr Gerechten meinetz Vaterö, erbet daß Reich! Zu
den Andern aber wird er sagen: Gehet hin, ihr Verfluchten in
das ewige Feuer! und so werden die Ungerechten in die ewige
Pein gehen, die Gerechten aber in das ewige Leben (Matth. 25).
Wisset, Freunde eS ist keine Schande für einen Haußvater,
die Seinen selber zu unterweisen oder jemand andertz etz für ihn
tun heißen, vielmehr ist es eine wichtige Pflicht, die ihm zu
tun auferlegt ift. Abraham und Josua haben etz also gemacht.
Vom ersteren heißt ez Genesiö 18, 19.5: ,,er wird befehlen seinen


% \picinclude{./220-229/p_s222.jpg} 
222 Kapitel Il-’1ll.
Kindern und seinem Hauß, daß sie dez Herrn Wege halten .... «,
und der zweite sagt, Josua 24,15: ,,erwählet euch heute welchem
ihr dienen wollt; ich aber und mein Hauß wollen dem Herrn
dienen«. Wir erklären, daß wir eö für unsere Pflicht halten,
mit denen und für die zu beten, die unsrem Hause angehören,
und sie zu lehren und zu ermahnen; denn eß ist dietz ein Befehl
vom Herrn und der Ungehorsam hiegegen wird sein Mißfallen
erregen, wie wir Jeremiaß I, 25 sehen können: ,,Schütte deinen
Zorn über die Heiden, die dich nicht kennen und über die Fa-
milien, die deinen Namen nicht anrufen«. Nun bilden die
Reger, die Rothäuter, die Twanieß, die Jndianer überall einen
großen Teil der Familien hier auf dieser Jnsel, und etz wird Rechen-
schaft über sie gefordert werden von dem, der kommen wird zu
richten die Lebendigen und die Toten am großen Tage deß
Gerichtß: ,,da ein jeder empfangen wird seinen Lohn, nach dem
er gehandelt hat, eö sei gut oder böse«, wenn er ,,wird geofsen-
baret werden mit Feuerflammen, Rache zu geben über die so
Gott nicht erkennen;« .... und ,,eß werden in den letzten Tagen
Spötter kommen, die nach den eigenen Lüften wandeln« .... ,,eZ
wird aber dez Herrn Tag kommen wie ein Dieb in der Nacht«. . . .
wie 2. Thess. 1,8 und 2. Pet. Z zu sehen ist.«
Die Veranlassung zu diesem Gerücht, daß wir versuchten die
Neger aufzuhetzen, hatten unsre Gegner darauö geschöpft, daß
wir Versammlungen mit und unter den Negern gehabt hatten,
denn sowohl ich ale; andere der Freunde hatten mehrere Ver-
sammlungen mit ihnen in verschiedenen Plantagen, in denen wir
sie zu Rechtschafsenheit, zur Keuschheit, zur Nüchternheit und zur
Frömmigkeit ermahnten, und zum Gehorsam gegen ihre Herm
und Meister, also gerade das Gegenteil von dem, wasz unsre
übelwollenden Gegner bößwillig gegen unß au?-streuten. ....
Ehe ich die Jnsel verließ, schrieb ich folgenden Brief an
meine Frau:
,,Mein liebetz Herz,
Welcher meine Liebe gehört, sowie allen Kindern im Samen
dez Lebenß, der sich nicht verändert, sondem größer ist alß alleß,
gelobt sei der Herr ewiglich. Jch habe unauösprechlich an Seele
und Leib zu erdulden gehabt; aber der Gott dez Himmels sei
gelobt, seine Wahrheit geht über alleß. Ich bin jetzt gesund, und
so der Herr will, gehe ich in einigen Tagen von Barbadoeö nach


% \picinclude{./220-229/p_s223.jpg} 
Reise nach Amerika. Varbadoes. Jamaika. 223
Jamaika und gedenke nur kurze Zeit dort zu bleiben. Jrh hoffe,
daß ihr alle im Samen des Lebens, frei von aller Kümmernis,
bewahret bleibet. Die Freunde sind im allgemeinen wohl. Grüße
mir die Freunde, die nach mir fragen. Soviel diesmal. Meine
Liebe im Samen und Leben, die nicht wechseln.«
Varbadoes, 6. des 11. Monats 1671. G. F.
Jch schisste mich am 8. des 11. Monats 1671 in Barbadoes
fiir Jamaika ein ..... Wir hatten eine gute, rasche Überfahrt, ....
und trafen in Jamaika James Lancaster, John Eartwright und
George Pattison wieder, die eifrig im Dienste der Wahrheit ge-
arbeitet hatten, dem wir uns nun auch widmeten, wir reisten
aus der Jnsel hin und her; es ist ein recht schönes Land, doch
sind die Leute zum Teil recht verdorben und ausschweifend. Wir
wirkten viel. Gs war eine große Belehrung und viele nahmen
die Wahrheit aus, worunter manche angesehene Leute. Wir
hatten viele Versammlungen hier, die zahlreich und ganz ruhig
waren. Die Leute begegneten uns sehr anständig und niemand
tat den Mund gegen uns auf. Jch war zweimal beim Gouver-
neur und den Behörden, die sehr freundlich gegen mich waren.
Etwa eine Woche nach meiner Ankunft in Jamaika schied
Elisabeth Hooton, eine sehr alte Frau, die viel im Dienst der
Wahrheit umhergereist war und viel dafür gelitten, aus diesem
Leben. Sie war noch am Tage vor ihrem Tode gesund und
schied in Frieden, und gab noch im Sterben der Wahrheit die
Ehre. Nachdem wir etwa sieben Wochen in Jamaika gewesen,
und unter den dortigen Freunden etwas Ordnung geschaffen und
mehrere Versammlungen unter ihnen eingerichtet hatten, ließen
wir Solomon Gccles dort, und schisften uns für Maryland
ein .....
Ehe ich Jamaika verließ schrieb ich noch einmal einen Brief
an meine Frau:
,,Mein liebes Herz,
Dir und den Kindern meine Liebe in dem, das über allem
ist und sich nicht verändert, und allen Freunden die bei euch sind,
Jch bin nun etwa fünf Wochen in Jamaika gewesen. Den
Freunden geht es im ganzen gut und wir haben große Be-
kehrungen gehabt; aber es würde zu weit führen, über alles zu
schreiben; überall warten Leiden meiner, aber der gesegnete Same
ist über allem. Der Herr sei .-gelobt, welcher Herr ist über Land


% \picinclude{./220-229/p_s224.jpg} 
224 Kapitel X11.
und Meer und alleß waz darinnen ist. Wir haben im Sinn,
etwa anfangß deß nächsten Monatö von hier abzureisen nach
Maryland zu, so der Herr will. Bleibet alle miteinander im
Samen dez Herrn; in seiner Wahrheit bleibe ich in der Liebe zu
euch allen.«
Jamaika, 23. deß 12. Monatß 1671. Gs Fs
Kapitel III.
Arbeit in N-tdcmetita unter Engländern nnd Indianern.
Wir schissten unß am 8. des 1. Monats 1671 ein; und da
wir schlechten Wind hatten, segelten wir eine ganze Woche hin
und her, ehe wir von Jamaika fort kamen. EZ war eine schwie-
rige und gefahroolle Reise, besonderö alö wir den Golf von
Florida passierten, wo wir manche Schwierigkeiten durch Wind
und Sturm zu bestehen hatten. Aber der große Gott, welcher
Herr ist über Meer und Land, welcher aus den Flügeln des
Windeö dahin fährt, bewahrte unß durch seine Kraft vor vielen
großen Gefahren, wenn bei dem Ungestüm deß Wetterö unser
Schiff oft nahe daran war umzuschlagen, und daß Tauwerk großen-
teiltz zerbrochen wurde. Wahrlich, wir merlten, daß der Herr ein
Gott der Nähe ift, und hört auf daß Flehen seines Volkeß. Denn
alö die Winde so stark und heftig tobten, und ez so mächtig
stürmte, daß die Schifföleute sich nicht zu helfen wußten, und
daß Schiff sich selbst überließen, da beteten wir zum Herrn, welcher
unß gnädig erhörte, Wind und Wellen siillte und unö günstiges
Wetter gab, so daß wir unö unsrer Errettung freuen durften.
Gelobt und gepriesen sei der herrliche Name dez Herrn, der
Macht hat über alleß, dem Wind und Wellen gehorchen. ....
Wir waren etwa sechß biß sieben Wochen untemzegß von
Jamaika nach Maryland .... Dort trafen wir John Burnyeat,
der die Absicht hatte, sich bald nach England einzuschifsen, aber
als wir kamen, änderte er seinen Vorsatz und schloß sich uns an
zum Dienst für den Herm .... Er hatte eine Versammlung für
alle Freunde von Maryland veranstaltet, damit er sie alle mit-
einander sehe, um Abschied von ihnen zu nehmen; und nun
fügte etz die Votsehung Gotteö so, daß wir gerade zur rechten
Zeit landeten, um dieser Versammlung beizuwohnen; .... ES
war eine sehr große Versammlung, die vier Tage dauerte ....


% \picinclude{./220-229/p_s225.jpg} 
Arbeit in Nordamerika unter Engländern und Indianern. 225
Nach der allgemeinen Versammlung fingen die Männer- und
Frauen-Versammlungen an ..... Hernach gingen wir nach einem
andern Orte, die Klippen genannt, wo eine andere große Ver-
sammlung stattfinden sollte. Wir gingen einen Teil dez Wegeö
zu Land, den Rest zu Wasser; und da sich ein Sturm erhob, stieß
unser Boot aus und wäre fast zertrümmert worden, und daß
Wasser drang herein. Ich schwitzte stark, da ich sehr warm au?-
der Versammlung gekommen war, und nun wurde ich vom Wasser
ganz durchnäßt; aber weil ich Glauben hatte in die göttliche Kraft,
wurde ich vor Schaden bewahrt, der Herr sei gepriesen .....
Wir hatten hier auch eine Männer- und Frauen-Versammlung,
und in vielen dieser Versammlungen wurde die Angelegenheit der
Kirche geordnet.
Nach diesen Versammlungen trennten wir uns und verteilten
unö auf die verschiedenen Küsten, zum Dienst der Wahrheit.
Jameö Lancaster und John Eartwright gingen zu Wasser nach
Neu-England; William Edmundson und drei andere Freunde
schifften sich für Virginia ein, wo die Dinge sehr in Unordnung
geraten waren; John Burnyeat, Robert Widderö, George Pattison
und ich mit einigen andern Freunden gingen mit einem Boot
nach der Ostküste und hatten dort am Ersten Tag eine Versamm-
lung, wo viele die Wahrheit mit Freude aufnahmen und die —
Freunde reichlich erquickt wurden. ES war eine große, selige
Versammlung, und es waren mehrere Personen von Nang auß-
der Gegend dabei, darunter zwei Zriedensrichter. ES kam über
mich vom Herrn, dem »Kaiser« der Indianer und seinen ,,Königen«
sagen zu lassen, sie sollten zu dieser Versammlung kommen. Der
,,Kaiser« kam und wohnte ihr bei, aber seine »Könige«, welche weiter
weg wohnten, konnten nicht zur rechten Zeit kommen; aber sie
kamen später nach, mit ihren Leuten. Ich hatte am Abend zwei-
mal eine gute Zeit mit ihnen, und sie hörten daß- Wort des Herrn
gerne und bekannten sich dazu. Ich bat sie, daß, wa-3 ich ihnen
sagte, dann auch ihrem Volke zu sagen und ihm zu verkünden,
daß Gott jetzt die Hütte des Zeugnisses in der Wüste aufrichte
und das Panier und segenöreiche Zeichen seiner Gerechtigkeit. Sie
benahmen sich sehr anständig und fragten, wann die nächste Ver-
sammlung sein werde, sie wollten dazu herkommen; aber sie
erzählten uns, sie hätten eine heftige Auszeinandersetzung gehabt
mit ihren Räten, wegen ihreö Kommenö. Am folgenden Tage
Stotz- F.-3. 15


% \picinclude{./220-229/p_s226.jpg} 
226 Kapitel R11.
traten wir unsre Reise nach Neu-England an, eine schwierige
Reise, durch Wälder und Sümpfe und große Flüsse; dann mußten
wir die Wildnis passieren, die jetzt West-Jets:) genannt wird,
die aber damalß nicht von Engländern bewohnt war; einen ganzen
Tag reisten wir, ohne Mann oder Frau, Hauß oder Wohnort
zu tressen. Zuweilen nächtigten wir im Wald bei einem Feuer,
zuweilen in den Hütten und Häusern der Jmdianer. Eines Abends
kamen wir in eine indianische Ortschaft und übernachteten beim
,,König«, der ein sehr achten?-werter Mann war; er sowohl alö sein
Weib nahmen uns sehr liebevoll auf, und seine Dienerschaft be-
handelte unö sehr ehrerbietig. Sie gaben uns Matten, um darauf
zu schlafen, aber zu essen hatten sie wenig, da sie an dem Tage
wenig gefangen hatten. In einer andern indianischen Ortschaft, in
der wir unß aufhielten, kam der »König« zu uns; er sprach ein
wenig englisch. Ich redete viel mit ihm und auch mit seinen
Leuten, und sie waren sehr lieb mit uns. Schließlich kamen
wir nach Middletown, einer englischen Pslanzung in Ost=Jersy;
doch konnten wir nicht zu einer Versammlung bleiben, da e-3 unß
trieb, rechtzeitig zur Halbjahreßversammlung in der Oysterbay,
auf Long-JS-land, zu sein ..... Ein Freund, Richard Hartshorn,
setzte unö in seinem eigenen Boote über nach Long-Jtzland, und
am zweitfolgenden Morgen erreichten wir die Oysterbay, wo wir der
Halbjahreßversammlung beiwohnten ..... Nachdem dann die
Freunde wieder nach Hause gegangen waren, blieben wir noch
einige Tage auf der Jnsel und warteten dann in der Oysterbat)
auf günstigen Wind, um nach Rhode- Jöland zu gehen .....
Dort kamen wir am 13. dez 3. Monatß an und wurden mit
Freuden von den Freunden ausgenommen. Am nächsten
Ersten Tage hatten wir eine große Versammlung, welcher der
Unterstatthalter und mehrere von der Behörde beiwohnten. Sie
wurden mächtig von der Wahrheit ergriffen. Jn der daraus
folgenden Woche sand hier die Jahreöveriammlung für alle Freunde
von Neu-England und den übrigen angrenzenden Kolonien statt;
außer sehr vielen Freunden, die in diesen Gegenden lebten, kamen
noch John Stubbß auß Barbadoet? und Jameö Lancaster und
John Cartwrigth, von verschiedenen Seiten, um ihr beizuwohnen.
Diese Versammlung dauerte sechß Tage; an den vier ersten Tagen
waren allgemeine, gotteödienstliche Versammlungen, zu welchen
eine große Menge Leute kamen; denn sie hatten keinen Priester


% \picinclude{./220-229/p_s227.jpg} 
Arbeit in Nordamerika unter Engländern und Jndianern. 227
in Rhode=J-sland und darum keinerlei bestimmte Form irgend
einer Art von Gottesdienst; und weil der Unterstatthalter mit
mehreren von der Regierung täglich zu den Versammlungen kamen,
wurden die Leute ermutigt, so daß sie von allen Seiten herbei-
strömten. Wir hatten ein gesegnetes Wirken unter ihnen, und
die Wahrheit fand gute Ausnahme. Jch habe selten Leute von
dieser Art mit mehr Aufmerksamkeit, Fleiß und Liebe zuhören
sehen, als diese es im ganzen während der Vier Tage taten, was
auch andere Freunde beobachteten. Nachdem die öffentlichen
Versammlungen zu Ende waren, begann die Männerversammlung,
welche sehr zahlreich, köstlich und feierlich war; und tags daraus
war die Frauenversamnrlung, die ebenfalls sehr zahlreich und
feierlich war. Da diese beiden Versammlungen zur Ordnung
kirchlicher Angelegenheiten veranstaltet waren, so wurden viele
wichtige Dinge eröffnet und mitgeteilt, durch Rat, Belehrung und
Unterweisung, für das Verhalten in den verschiedenen in Frage
kommenden Verrichtungen, damit alles rein, lieblich und kräftig
unter ihnen erhalten werde. Verschiedene Männer- und Frauen-
Versammlungen für andere Gegenden wurden in diesen beiden
Versammlungen beschlossen und eingerichtet, zur Fürsorge sitr die
Armen und andere kirchliche Angelegenheiten, und damit dafür
gesorgt werde, daß alle, die die Wahrheit bekennen, auch nach
dem Evangelium wandeln. Als diese große Versammlung auf
Rhode-Jsland zu Ende war, wurde den Freunden der Abschied
etwas schwer; denn die herrliche Macht des Herrn, die über allen
war, und seine gesegnete Wahrheit und sein Leben, die sich über sie
ausgossen, hatten sie so untereinander verbunden und vereinigt,
daß sie zwei Tage damit zubrachten, um sich unter einander und
von den Freunden auf der Jnsel zu verabschieden. Darauf gingen
sie fort, mächtig erfüllt von der Gegenwart und der Krast Gottes,
sreudigen Herzens, jeder in seine Heimat, nach den verschiedenen
Kolonien, in denen sie lebten. ....
Während dieser Zeit fand eine Vermählung von zweien von
den Freunden statt, der wir beiwohnten. Gs war im Hause
eines Freundes, der früher Gouverneur dieser Gegend gewesen war;
drei Friedens-richter und viele, die nicht zu uns gehörten, waren
zugegen; aber alle, diese sowohl als die Freunde, sagten, sie hätten
noch nie eine so andächtige Zusammenkunft gesehen bei einem
derartigen Anlaß, noch eine so feierliche Vermählung rmd eine
15’


% \picinclude{./220-229/p_s228.jpg} 
228 Kapitel X11.
so Oorzügliche Ordnung. Dermaßen durchdrang die Wahrheit
alle. Es ist hoffentlich sür viele ein gutes Beispiel gewesen, denn
sie sind aus allen Teilen dez Landes dazu gekommen. Jch hatte
innnerlich viel durchzumachen wegen der Ranter in dieser Gegend,
die eine Versammlung, der ich nicht beigewohnt hatte, gestört
hatten. Jch zeigte darum eine Versammlung unter ihnen an, im
Glauben, daß der Herr mir Macht über sie geben werde, was
er auch tat, zu seiner Ehre und Verherrlichung. Sein Name
sei gelobt ewiglich .....
Darauf hatten wir eine Versammlung in Prooidence ....
ebenso in Narraganset ..... Von dort ging ich nach der Jnsel
Shelter. .... Dort hatten wir am Tage nach unsrer Ankunft,
einem Ersten Tage, eine Versammlung. Jn der gleichen Woche hatte
ich eine unter den Indianern, ihr ,,König« war zugegen und sein
Rat und einige hundert Indianer; sie setzten sich unter uns, ganz
wie die Freunde taten, und hörten aufmerksam zu, während ich
durch einen Dolmetscher, einen Jndianer, der gut englisch sprach,
zu ihnen redete. Sie waren nach der Versammlung sehr lieb
und bekannten, das, was man ihnen gesagt habe, sei Wahrheit
gewesen .....
Wir reisten nun umher und kamen schließlich nach Shrewsbury,
wo sich etwas ereignete, das damals eine wichtige Probe für uns
war. John Jay, ein Freund aus Barbadoes, der mit uns von
Rhode-Jsland gekommen war und uns durch die Wälder von
Maryland begleiten wollte, bestieg ein Pferd, um es zu versuchen;
das Pferd sing an zu gallopieren und warf ihn ab, so daß er
auf den Kopf fiel und den Hals brach, wie die Leute sagten.
Einige hoben ihn auf, in der Meinung, er sei tot, und trugen
ihn ein Stück weit und legten ihn unter einen Baum. Jch ging
sogleich zu ihm, und als ich ihn anriihrte, hielt ich ihn auch für
tot. Während ich neben ihm stand, und ihn und die Seinen
beklagte, fuhr ich ihm durch die Haare, wobei sein Kopf sich hin
und her drehte, so schlaff war sein Hals. Nun nahm ich seinen
Kopf zwischen meine beiden Hände, und, indem ich meine Knie
gegen den Baum stemmte, hob ich seinen Kopf in die Höhe und
sah, daß da nichts zerrissen oder gebrochen war; ich faßte ihn
nun mit einer Hand unter dem Kinn und mit der andern hinten
am Kopf und bewegte seinen Kopf zwei- oder dreimal mit aller
Kraft hin und her und renkte ihn ein. Jch bemerkte bald, wie


% \picinclude{./220-229/p_s229.jpg} 
Arbeit in Nordamerika unter Engländern und Jndianern. 229
sein Hale wieder anfing, Halt zu bekommen; darauf sing er an
zu röcheln und gleich darauf zu atmen. Die Leute waren ent-
setzt, aber ich hieß sie, guten Mutß zu sein, Glauben zu haben
und ihn inß Haus zu tragen. Sie taten eö und legten ihn neben
daß Feuer. Jch hieß sie, ihm etwas Warmeö zu trinken zu geben
und ihn zu Bett zu bringen. Nach einer Weile fing er an zu sprechen,
aber er wußte nicht, was mit ihm geschehen war. Am folgenden
Tage zogen wir weiter, und er mit uns-, ziemlich wohl, etwa
16 Meilen zu einer Versammlung nach Middletown durch Wälder
und Sümpse und über einen Fluß, wo wir unsere Pferde hinüber
schwimmen ließen und selber aus einem hohlen Vaumstanun hinüber
setzten. Er reiste noch viele hundert Meilen mit unß .....
Wir hatten hier eine herrliche Versammlung ..... Nach der-
selben gingen wir nach Middletown-Harbour, etwa siins Meilen weit,
um am nächsten Tage unsre große Reise anzutreten, durch die
Wälder nach Maryland. Unsre Führer waren Jndianer. Jch
beschloß, den Weg durch die Wälder auf der andern Seite der
Delawara-Bay zu nehmen, um die Flüsse und Buchten so viel wie
möglich zu vermeiden. Wir machten unö am 9. des 7. Monat?.
auf den Weg und kamen durch viele indianische Ortschaften und über
mehrere Flüsse und Sümpse; alö wir etwa 40 Meilen weit
geritten waren, machten wir ein Feuer für die Nacht und legten
unß daneben. Wenn wir zu Jndianern kamen, verkündeten
wir ihnen den Tag deß Herrn. Tagß daraus reisten wir 50 Meilen.
Nachtß fanden wir ein alteö Hauß, aus dem die Indianer die
Leute vertrieben hatten; wir machten ein Feuer und blieben dort
am Eingang der Delawara-Bay. Am folgenden Tage ließen wir
unsre Pferde etwa eine Meile weit über den Fluß schwimmen,
nach der Jnsel Upper-Dinidock, und darauf ausß Festland. Wir
selber hatten von den Jndianern ein Canoe gemietet und unö
darin von ihnen übersetzen lassen .....
Dann gingen wir nach Newcastle, jetzt Neu-Amsterdam
genannt; .... am 16. dez 7. Monats zogen wir weiter .... .
Nach einer beschwerlichen Reise erreichten wir daß; Haus Robert
Harwoodz in Mileö River in Maryland und wohnten am
folgenden Tage einer Versammlung bei. .... Von da gingö
nach dem Kentischen User, .... und dann zu Wasser, etwa zwanzig
Meilen weit, zu einer sehr großen Versammlung ..... E3 war
eine gesegnete Versammlung und von großem Nutzen sowohl zur

