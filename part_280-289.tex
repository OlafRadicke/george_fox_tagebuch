% \picinclude{./280-289/p_s280.jpg} 
Von Kingston ging ich nach London\ort{London} [...] und dann nach
Hertford\ort{Hertford} [...]. Dort traf ich John 
Story\person{Story, John}\footnote{John Story, der Führer einer Partei, 
die gegen die Frauenversammlungen und andere Einrichtungen 
auftrat.}, und etliche seiner
Richtung; aber das Zeugnis der Wahrheit hielt sie nieder, so
das wir eine ruhige Versammlung hatten. Dies war an einem Ersten
Tage, und da am folgenden Tage die geschäftliche Männer- und
Frauen-Versammlung war, so blieb ich auch noch zu dieser, um
so mehr, als etliche eine Geringschätzung derselben hatten aufkommen
lassen. Darum trieb es mich, über den Nutzen dieser Versammlungen
zu reden und über ihren Segen für die Kirche Christi, wie der
Herr es mir eingab, und dies tat den Freunden einen guten
Dienst [...].

Ich blieb den größten Teil dieses Winters 1680—81\index{Jahr!1680-81} in London,
eifrig im Dienst des Herrn, sowohl in Versammlungen als auch
sonst; denn da es eine Zeit schwerer Prüfungen für die Freunde
war, so zog es mich, ihre Versammlungen noch mehr als sonst
zu besuchen, um sie zu ermutigen und zu ermahnen durch Wort
und Beispiel. Das Parlament\index{Parlament} tagte, und die Freunde warteten
gespannt, bis sie ihre Anliegen Vorbringen konnten; denn wir
hörten fast jeden Tag von neuen Leiden, die die Freunde in 
verschiedenen Teilen des Landes zu erdulden hatten. Ich brachte
viel Zeit damit zu, meinen leidenden Brüdern Linderung zu 
verschaffen; mit einigen andern Freunden, die von sich aus sich auch
dieser Sache annahmen, wartete ich manchen Tag im 
Parlamentshaus und benutzte jede Gelegenheit, Parlanientsmitglieder, die
unsre berechtigten Klagen anhören wollten,\index{Politik}\index{Lobbyismus} 
zu sprechen. Einige
waren auch sehr entgegenkommend und Versprachen, für uns zu
tun, was sie könnten. Aber weil das Parlament gerade mit
aller Macht daran war, das päpstliche Komplott zu entdecken,
und die dabei Beteiligten herauszufinden, so benützten unsre Gegner,
die wussten, das wir nicht schwören noch die Waffen gebrauchen
dürfen, die Gelegenheit, die Strafen die über die Päpstlichen 
verhängt wurden, auch uns zuzuwenden, obgleich ihnen ihr Gewissen
sagte, das wir nicht päpstlich waren, und sie aus Erfahrung
wussten, das wir uns nicht an Verschwörungen beteiligten. Um
nun unsre Unschuld darzutun und unsern Gegnern das Maul zu
stopfen, verfasste ich ein kurzes Schreiben und lies es im 
Parlament vorweisen:
% \picinclude{./280-289/p_s281.jpg} 

\brief{Parlament}{

Wir haben den Grundsatz, uns von allen Verschwörungen\index{Verschwörung}
gegen den König oder seine Untertanen fernzuhalten; denn wir
haben den Geist Christi und durch ihn die Gesinnung Christi,
und er \zitat{ist gekommen, die Menschen zu erretten und nicht sie zu
verderben} (Luk. 9, 56\bibel{Luk. 09:56@Luk. 9:56}). Wir wollen die Sicherheit des Königs
und aller seiner Untertanen, darum erklären wir hiermit, das
wir trachten werden, alle Verschwörungen gegen ihn, von denen
wir hören, entdecken zu helfen. Solches versprechen wir euch
aufrichtig; was aber das Schwören\index{Schwören} und Kriegen\index{Krieg} anbelangt, so
können wir das ums unserer Gewissen willen nicht tun, wie ihr
ja wisset, denn wir haben alle diese Jahre hindurch um dieser
Weigerung willen viel gelitten. Wir hoffen nun, da euch der
Herr hier zusammengeführt hat, ihr werdet uns von diesen Leiden
befreien und nicht Dinge von uns verlangen, um deretwillen wir
schon so viel und so lange gelitten haben; dadurch würdet ihr
unsre Bande noch härter machen, als sie schon sind, anstatt sie
uns zu erleichtern.
\bigskip 

\begin{flushright}
G. F.\end{flushright}
}

\chapter[Mahn- und Trostschreiben]{Mahn- und Trostschreiben}

\begin{center}
\textbf{Allerlei Mahn- und Trostschreiben.}
\end{center}


Ungefähr zur gleichen Zeit erhielt ich zwei sehr gehässige
Bücher, die gegen die Wahrheit und die Freunde gerichtet waren;
das eine war von einem sogenannten Doktor aus Bremen\ort{Bremen} in
Deutschland, das andere von einem Priester aus Danzig\ort{Danzig}. Beide
waren Voll arger Falschheit und verleumderischer Vorwürfe\index{Verleumdung}. Es
kam über mich, auf beide zu antworten, und um nicht durch
andere Geschäfte und Besuche gestört zu werden, ging ich nach
Kingston\ort{Kingston} an der Themse\ort{Themse}, wo ich eine Antwort auf jedes der
Bücher schrieb, sowie aus einige andere gehässige Schriften, die
geschrieben und verbreitet worden waren, um die Freunde falsch
darzustellen [...].

Die Sheriff für die Stadt sollten neu gewählt werden und
die, welche für die Wahl vorgeschlagen waren, wünschten die
Stimmen der Freunde zu erhalten; da schrieb ich einige Zeilen,
um zu erfahren, wessen Geistes sie wären, und wie sie sich zu der
wahren Freiheit stellten. Ich tat es in Form einer Frage 
folgendermaßen:


% \picinclude{./280-289/p_s282.jpg} 
\brief{Sheriff}{
    Gibt irgend einer, der hier in London möchte zum Sheriff
    gewählt werden, zu, das Christus, der vor den Toren Jerusalems
    gekreuzigt wurde, \zitat{das Licht der Welt ist, das jeden, der in die
    Welt kommt, erleuchtet} und sagt, \zitat{glaubet an das Licht, aus das
    ihr Kinder des Lichts seid} (Joh. 12,36\bibel{Joh. 12:36})? Widersetzt sich einer,
    das man die Leute verfolge um der Religion willen, und darum, das
    sie Gottes Gebot halten und ihn im Geist und in der Wahrheit
    anbeten? Denn Christus sagt: \zitat{Ich bin nicht von dieser Welt}
    (Joh.17\bibel{Joh.17}), noch ist \zitat{mein Reich von 
    dieser Welt} (Joh. 18\bibel{Joh.18}), darum
    hält er seine Religion nicht mit weltlichen Waffen aufrecht. Christus
    sagte: \zitat{Ihr sollt überhaupt nicht schwören}, und sein Apostel
    Jakobus sagt dasselbe, und nun wollet ihr uns zwingen, zu
    schwören und somit die Gebote Christi und seiner Apostel zu
    brechen, indem ihr uns Eide vorlegt? Christus sagt zu seinen
    Aposteln: \zitat{Umsonst habt ihr es empfangen, umsonst gebt es
    auch} (Matth. 10,8\bibel{Matth. 10:08@Matth. 10:8}). Werdet 
    ihr uns nicht zwingen, Zehnten
    und Abgaben zu zahlen an solche Lehrer, von denen wir wissen,
    das Gott sie nicht gesandt hat? Werden wir frei sein, Gott
    zu dienen und ihn anzubeten und seine und seines Sohnes Gebote
    zu halten, wenn wir euch freiwillig unsre Stimmen geben? Denn
    wir sind nicht willens, unsre Stimmen solchen zu geben, die uns
    gefangen nehmen und uns unsre Habe nehmen [...].

}

Ich schrieb auch, während ich in London war, so oft ich
zwischen den Versammlungen und andern öffentlichen 
Gottesdiensten\index{Versammlung!öffentliche} Zeit hatte, 
verschiedene Bücher und Schriften, von denen
einige gedruckt und andere im Manuskript verbreitet wurden.
Eines davon richtete sich an die Bischöfe und andere, welche 
Verfolgungen anzetteln, und bewies ihnen aus der heiligen Schrift,
das sie nicht nach derselben wandelten und nach dem \zitat{königlichen
Gesetz, das gebietet, seinen Nächsten zu lieben wie sich selbst}
(Jak. 2,8\bibel{Jak. 02:08@Jak. 2:8}), und andern zu tun, wie man möchte, das andere uns
tun. Eine andere Schrift war: \buchtitel{An die Menge derer, Protestanten
wie Papisten\index{Papisten}, die sich für Christen ausgeben, deren Gottesdienst
und Religion aber in äußern Formeln und Zeremonien besteht}
Ich wies sie mit Nachdruck auf die Worte des Apostels Paulus
hin, Galater 5,2-4\bibel{Galater 05:02@Galater 5:2-4}: \zitat{Ich, 
Paulus, sage euch, wo ihr euch beschneiden lässt, so ist euch 
Christus kein nütze. Ich sage aber
jedem, der sich beschneiden lässt, das er noch das ganze Gesetz
schuldig ist. Ihr habt Christus verloren, die ihr noch durch das
% \picinclude{./280-289/p_s283.jpg} 
Gesetz gerecht werden wollt und seid aus der Gnade gefallen}.
Eine andere Schrift war: \zitat{Jeder wende sich dem Geiste Gottes
zu, um durch denselben ein rechtes Verständnis zu bekommen und
fähig zu werden, zwischen Recht und Unrecht zu unterscheiden,
zwischen Wahrheit und Irrtum und nicht unter dem Vorwand,
Übeltäter zu bestrafen, sich selber Übles zu tun, indem man den
Gerechten verfolgt}\index{Glaubenskrieg}\index{Verfolgung} \buchtitel{Jeder 
wende sich dem Geiste Gottes
zu, um durch denselben ein rechtes Verständnis zu bekommen und
fähig zu werden, zwischen Recht und Unrecht zu unterscheiden,
zwischen Wahrheit und Irrtum und nicht unter dem Vorwand,
Übeltäter zu bestrafen, sich selber Übles zu tun, indem man den
Gerechten verfolgt}[...].

Eine andere Schrift schrieb ich über \buchtitel{Betrachtung, Ergötzen,
Übung, Streben und Forschen}; ich zeigte aus der Schrift der
Wahrheit, worüber die wahren Christen nachdenken sollten und
worin ihren Sinn üben, was ihr Ergötzen sein sollte, und was sie sich
zu tun bestreben sollten. Denn in diesen Dingen sind nicht nur
die Weltlichen und die leichtfertigen Leute in großem Irrtum,
sondern auch die großen \zitat{Frommen}, sie freuen sich über die 
irdischen, vergnüglichen Dinge, während sie über himmlische Dinge
nachdenken und sich am Gesetz Gottes ergötzen und sich bestreben
sollten, immer ein reines Gewissen gegen Gott und Menschen zu
haben, gleich dem Apostel« [...].

Da die Leiden immer noch schwer und drückend auf den
Freunden lasteten, nicht nur in der Stadt, sondern auch in fast
allen Gegenden des Landes, setzte ich ein Schreiben auf, das dem
König eingereicht werden sollte. Ich brachte darin unsre 
Bekümmernisse vor und bat für die Fälle, die mir in seiner Macht
zu stehen schienen, um Abhilfe. Da ich aber keine Hilfe von
ihm erhielt, kam es über mich, einen Brief an die Freunde zu
schreiben, um sie in ihren Leiden zu ermutigen\index{Ermunterung}, 
damit sie in Geduld die vielen Prüfungen ertragen möchten, die über sie 
gebracht wurden, sowohl von Seiten der Behörden, als auch von
falschen Brüdern und Abtrünnigen\index{Abtrünnige}, deren böse 
Bücher und gemeine Verleumdungen die Rechtschaffenen betrübten [...].

Ich blieb den größten Teil des Winters in London im Dienst
der Wahrheit unter den Freunden, ausgenommen eine kurze Zeit,
die ich in Kingston zubrachte, im 10. Monat des Jahres, wo ich
ein Buch schrieb über: 
\buchtitel{Das Wesen der zeitlichen Geburt und
das der geistigen}, worin ich die Pflicht und Stellung eines Kindes,
Jünglings, Erwachsenen und Greises der Wahrheit gegenüber darlete [...].

An einem Ersten Tage kam es über mich, am Nachmittage
in die Versammlung in Devonshire House\ort{Devonshire House} zu gehen, 
und da ich
% \picinclude{./280-289/p_s284.jpg} 
erfuhr, das die Freunde dort am Morgen nicht eingelassen worden
waren, wie es an dem Tage bei den meisten Versammlungen in
der Stadt geschehen war, ging ich früher hin und ging in den
Hof, ehe die Soldaten kamen, um die Eingänge zu bewachen,
aber die Konstabler waren vor mir dort und standen im Torweg
mit ihren Stäben\index{Versammlung!trotz Verbot}. Ich bat 
sie, mich hinein zu lassen; sie sagten,
sie können und dürfen es nicht, man habe ihnen das Gegenteil
geboten, es tue ihnen Leid. Ich sagte, ich wolle nicht in sie
dringen; so stand ich neben ihnen und sie waren sehr höflich.
Ich stand, bis ich müde war, und dann gab mir einer einen
Stuhl, um mich zu setzen, und nach einer Weile fing die Kraft
des Herrn an, unter den Freunden lebendig zu werden, und
einer fing an zu reden. Die Konstabler verboten es gleich und
sagten, er solle nicht sprechen; und da er nicht aufhörte, wurden
sie zornig. Da legte ich sanft meine Hand. auf die des einen 
Konstablers und bat ihn, den Mann ruhig zu lassen, der Konstabler tat
es und war still, und der Mann redete nicht lang. Nachdem er
geendet, trieb es mich, aufzustehen und zu reden, und in meiner
Verkündigung sagte ich, sie brauchen nicht gegen uns vorzugehen
mit Stäben und Schwertern, denn wir seien friedliche Leute und
es sei nichts als Wohlwollen in unsern Herzen gegen den König
und die Behörden und gegen alle Menschen auf der ganzen Welt.
Wir gebrauchen die Religion nicht als Vorwand, um uns zu
Verschwörungen und Bündnissen oder Aufständen 
zu versammeln,\index{Religion und Gewalt}
sondern wir versammeln uns, um Gott im Geist und in der
Wahrheit anzubeten. Wir haben Christus zum Bischof, Priester
und Hirten (1. Petr. 2\bibel{Petr. 1. 02@1. Petr. 2}), 
er erlabt und leitet unsre Seelen, darum
können wir alle hier stille sitzen und unsres Lehrers- genießen und
uns seiner Lehre freuen, und ich befahl sie alle Christus, ihrem
Bischof und Hirten. Darauf setzte ich mich nieder, und nach einer
Weile trieb es mich, zu beten\index{Gebet}, und die Kraft des Herrn war über
allen, und die Versammelten, die Soldaten und die Konstabler,
nahmen ihre Hüte ab. Als die Versammlung zu Ende war, und
die Freunde anfingen hinaus zu gehen, nahm der Konstabler,
seinen Hut ab und bat den Herrn, das er uns segne, denn die
Kraft des Herrn war über ihm und allen andern und überwältigte sie [...].

Im ersten Monat des Jahres 1683 ging ich nach Kingston\ort{Kingston}
an der Themse [...]. Daraus nach Guildsord in Snrrey, und
% \picinclude{./280-289/p_s285.jpg} 
nachdem ich die Freunde dort besucht hatte, weiter nach 
Worminghurst\ort{Worminghurst} in Sussex\ort{Sussex}, wo ich 
eine sehr gesegnete Versammlung
mit den Freunden hatte, ohne jegliche Störung. Während ich
dort war, wurde James Claypole\person{Claypole, James} aus London, der mit seiner
Frau auch dort war, plötzlich krank\index{Krankheit}; es war ein so heftiger Anfall,
das er weder stehen noch liegen konnte und vor heftigen Schmerzen
schrie. Als ich es hörte, wurde ich sehr betrübt im Geist um ihn
und ging zu ihm. Nachdem ich einige Worte mit ihm gesprochen
hatte, um seinen Sinn nach innen zu richten, trieb es mich, ihm
meine Hand aufzulegen\index{Handauflegen}, und ich bat den Herrn, seine Krankheit
von ihm zu nehmen; während ich ihm meine Hand auslegte, kam
die Kraft des Herrn über ihn, und durch den Glauben an diese
Kraft wurde es ihm gleich leichter, und er fiel in einen Schlaf.
Als er erwachte, war er so wohl, das er am nächsten Tage
25 Meilen mit mir in einem Wagen fuhr, während er früher,
wie er mir sagte, gewöhnlich zwei Wochen, manchmal einen
Monat an einem solchen Anfall darniederlag. Aber der Herr
war für ihn angerufen worden und schenkte ihm durch seine Kraft
diesmal schnelle Besserung; sein heiliger Name sei dafür gelobt
und gepriesen [...]\index{Heilung}.

\chapter[Zweite Reise nach Holland.]{Zweite Reise nach Holland.}

\begin{center}
\textbf{Zweite Reise nach Holland. Brief an den Herzog von Holstein
zur Verteidigung des öffentlichen Redens der Frauen.}
\end{center}


Ich reiste umher, besuchte Freunde und wohnte den Versammlungen 
bei [...]. und im 6. Monat 1683\index{Jahr!1683} war ich wieder
in London. Hier besuchte ich namentlich die Freunde, die im
Gefängnis\index{Gefängnisbesuch} waren, weil sie für Jesus 
Zeugnis abgelegt hatten;
ich ermutigte sie in ihren Leiden und ermahnte sie treu zu bleiben
in ihrem Zeugnis, welches der Herr ihnen auferlegt hatte. Auch
zu solchen ging ich, die krank, schwach und im Gemüt angefochten
waren und half ihnen, das ihr Geist nicht in ihrer Trübsal 
versank. Unsre Versammlungen waren oft ruhig und friedlich, oft
wurden sie aber auch von den Beamten gestört und aufgelöst.
An einem Ersten Tag kam es über mich, zu einer sehr großen
Versammlung in Savoy\ort{Savoy} zu gehen, es waren viele \zitat{Fromme} und
Ernstgesinnte anwesend. Der Herr offenbarte\index{Offenbarung} mir viele wichtige
Dinge, die ich den Leuten verkündete und sie aus den Geist
% \picinclude{./280-289/p_s286.jpg} 
Gottes in ihrem Innern hinwies, durch den alle möchten die
Schrift\index{Bibel} verstehen\index{Exegese}, die er 
eingegeben hat, und Christus erkennen,
welchen Gott gesandt hat [...] Während ich noch redete in der
Kraft des Herrn, und die Leute sehr ergriffen waren, brachen
plötzlich der Pöbel und die Konstabler wie eine Welle herein.\index{Versammlung!Störung}
Einer der Konstabler rief mir zu: \zitat{Komm herunter!} und legte
Hand an mich. Ich fragte ihn: \zitat{Bist du ein Christ? Wir sind
Christen}. Er hatte mich bei der Hand gepackt und wollte mich
herunterreißen, aber ich blieb stehen und redete ein paar Worte
zu den Leuten, indem ich den Herrn bat, das der Segen Gottes
auf ihnen sein möge. Der Konstabler rief mir immer noch zu,
herunter zu kommen und zerrte mich schließlich herunter und hieß
einen andern mit einem Stab mich ergreifen und gefangen nehmen.
Ich wurde ins Haus eines andern Beamten geführt, der anständiger 
war; nach einer Weile wurden vier weitere Freunde,
die sie gefangen genommen hatten, hereingebracht. Ich war sehr
ermüdet und erhitzt und viele Freunde kamen zu mir, als sie
hörten, wo ich sei; aber ich bat sie alle, ihrer Wege zu gehen,
damit die Konstabler und Aufseher sie nicht ergreifen. Nach einer
Weile führten uns die Konstabler eine Meile weit zu einem Richter,
einem sehr zornigen, heftigen Mann; der Konstabler hatte ihm mitgeteilt, 
das ich in Versammlungen predige; nachdem er mich nach dem
Namen gefragt, und der Schreiber ihn aufgeschrieben hatte, sagte er
mit ärgerlicher Stimme: \zitat{Wisst ihr nicht, das es gegen des Königs
Gesetz ist, in solchen Konventikeln etwas zu predigen, das im
Widerspruch mit der Liturgie der Kirche von England steht?}
Es war einer da, Shad\person{Shad}, ein böser Aufseher, von dem es hieß,
er sei aus dem Gefängnis zu Coventry\ort{Gefängnis zu Coventry} 
ausgebrochen und in
London in die Hand gebrandmarkt worden; als der den Richter
so zu mir sprechen hörte, ging er auf ihn zu und sagte ihm: er
habe uns schon schuldig erklärt, nach Paragraph 22 des Gesetzes
König Karls ll. \zitat{Wie! ihr sie schuldig erklären!} rief der
Richter. \zitat{Ja}, antwortete Shad, \zitat{ich erklärte sie schuldig
und ihr müsst es auch tun nach dem Gesetz}. Hierauf wurde
der Richter ärgerlich über ihn und sagte: \zitat{Ihr wollt mich lehren?
Wer seid ihr? Ich werde sie wegen Aufruhrs schuldig erklären}.
Als der Ankläger das hörte und den Zorn des Richters sah,
ging er verdrießlich hinweg, denn er war in seinem Vorhaben
getäuscht. Ich vermutete, das er jemand wollte gegen mich
% \picinclude{./280-289/p_s287.jpg} 
schwören lassen, darum sagte ich: \zitat{Last niemand gegen mich
schwören, denn es ist mein Grundsatz, nicht zu schwören; darum
möchte ich nicht, das jemand gegen mich schwört}. Der Richter
fragte mich darauf, ob ich nicht in den Versammlungen predige?
Ich antwortete: \zitat{Ich bekenne, was Gott und Christus für meine
Seele getan haben, und preise Gott; ich kann solches auch in den
Straßen und auf allen Plätzen tun, und schäme mich nicht, dies zu
bekennen, auch widerspricht es der Liturgie der Kirche von England
nicht}. Der Richter sagte, das Gesetz sei gegen die Versammlungen, 
welche der Liturgie der Kirche von England widersprechen.
Ich sagte: \zitat{Ich kenne kein Gesetz gegen unsre Versammlungen,
wenn er aber jene Verordnung meine, die sich gegen die richte,
welche zusammen kommen, um Komplotte, Verschwörungen und
Aufstände gegen den König zu machen, so gehörten wir nicht 
dazu, sondern wir verabscheuten alle solche Dinge. Wir hegten auf-
richtige Liebe und Wohlwollen gegen den König und alle Menschen
auf Erden.} Der Richter fragte mich darauf, ob ich dem geistlichen 
Stand angehört habe, ich antwortete: \zitat{nein}. Darauf
nahm er sein Buch und suchte nach Gesetzen gegen uns und hieß
den Schreiber unterdessen, die Namen der übrigen notieren; als
er aber kein anderes Gesetz gegen uns finden konnte, lies er den
Konstabler gegen uns schwören. Einige der Freunde hießen den
Konstabler, sich in acht nehmen, das er nicht einen Meineid tue,
da er sie am Eingang und nicht in der Versammlung festgenommen 
habe. Aber der Konstabler war ein schlechter Kerl und
schwor, sie seien in der Versammlung gewesen.
Dennoch sagte der Richter, weil er nur einen Zeugen habe,
so müsse er die andern freisprechen; aber mich wolle er nach
Newgate\ort{Newgate} schicken; ich könne dann dort predigen. Ich fragte ihn,
ob er es; mit seinem Gewissen vereinigen könne, mich nach 
Newgate zu schicken dafür, das ich Gott preise und Christus bekenne?
Er rief: \zitat{Gewissen, Gewissen!}, aber ich sah, das meine Worte
sein Gewissen getroffen hatten. Er hieß den Konstabler, mich
wegführen, er werde einen Verhaftbefehl machen, um mich ins
Gefängnis zu schicken, wenn er gegessen habe. Ich sagte ihm,
ich wünsche ihm Frieden und den Seinen alles Gute, und das sie
in der Furcht Gottes bewahrt bleiben möchten; damit ging ich
fort. Der Konstabler nahm einigen Freunden das Versprechen
ab, das ich am folgenden Morgen um acht zu ihm kommen werde.
% \picinclude{./280-289/p_s288.jpg} 
Dies tat ich denn auch. Der Konstabler teilte mir mit, als er
zum Richter gekommen sei nach dem Mittagessen, um den Verhaftbefehl 
zu holen, habe dieser ihn geheißen, nach dem Abendgottesdienst 
noch einmal zu kommen. Dies habe er getan, und da
habe ihm der Richter gesagt, er solle mich laufen lassen. \zitat{Somit},
sagte der Konstabler, \zitat{seid ihr freigesprochen}. Ich machte dem
Konstabler Vorwürfe, das er als Ankläger aufgetreten sei, und
er sagte, er wolle es nicht mehr tun [...].

Da es über mich kam, Verschiedenes zu schreiben, ging ich
nach Kingston\ort{Kingston}, damit ich vor Unterbrechung\index{Rückzug} 
sicher sei [...]. Hier
schrieb ich ein kleines Buch, das bald darauf gedruckt wurde, betitelt: 
\buchtitel{Der Heiligen himmlische und geistige Anbetung, Einigkeit,
Gemeinschaft usw.}, in welchem ich dartat, was die wahre 
Anbetung nach dem Evangelium sei, und worin die wahre Einigkeit
und Gemeinschaft der Heiligen\index{Gemeinschaft der Heilige} 
bestehe, nebst einer Bloßstellung
derer, die von dieser heiligen Einigkeit und Gemeinschaft 
abgewichen waren und sich gegen die Heiligen, die darin blieben, 
gewandt hatten [...].

Die Jahresversammlung war im 3. Monat 1684\index{Jahr!1684}. Sie war 
gesegnet und herrlich, und die Freunde wurden erquickt und gehoben,
denn der Herr war unter uns und teilte uns seine himmlischen
Schätze mit. Und trotzdem es eine Zeit großer Not und Gefahr
war, wegen der vielen Verfolgungen durch die Obrigkeiten, so
schützte doch der Herr die Seinen.
Ich fühlte einen Zug in meinem Geist, nach Holland zu gehen,
um dort den Samen Gottes zu besuchen, und sobald die 
Jahresversammlung zu Ende war, rüstete ich mich für die Reise.
Alexander Parker\person{Parker, Alexander}, George
Watts\person{Watts, George} und Nathaniel 
Brassey\person{Brassey, Nathaniel}, die
auch einen Zug nach diesem Land verspürten, gingen mit mir [...].

In Harwich schifften wir uns ein [...]. Wir hatten eine gute
Überfahrt und landeten am Morgen des darauf folgenden Tages
in Briel in Holland [...]. Am Morgen nach unsrer Ankunft
gingen wir nach Rotterdam\ort{Rotterdam}, wo wir einige Tage blieben. Am
Tag nach unsrer Ankunft in Rotterdam lud mich einer namens
Wilbert Frouzen\person{Frouzen, Wilbert}, ein Bürgermeister, auf sein Landgut ein. Er
war ein Verwandter von Aarent Sunneman\person{Sunneman, Aarent}, und als er von
meinem Kommen gehört, wünschte er, mit mir eine Angelegenheit,
Sunnemans Töchter betreffend, zu besprechen. Ich nahm George
Watts mit, und ein Bruder von Aarent Sunnemann brachte uns
% \picinclude{./280-289/p_s289.jpg} 
hin. Der Bürgermeister empfing uns sehr freundlich und freute
sich sehr, mich zu sehen; im Lauf des Gesprächs über die Töchter
seines Verwandten merkte ich, das er fürchtete, man werde sie,
da ihr Vater gestorben war und ihnen ein beträchtliches Vermögen
hinterlassen hatte, übervorteilen und ungünstig verheiraten. Ich 
erklärte ihm, es sei unser Prinzip, das niemand bei uns sich 
verheirate\index{Heirat} ohne eine Einwilligung oder ein Gutachten der Verwandten
oder des Vormunds; denn es sei unsre Christenpflicht, alle jungen
Leute, die zu uns kommen, zu überwachen und ihnen nachzugehen,\index{Gemeindezucht}
besonders solchen, deren rechtmäßige Angehörige gestorben seien.
Und was die Töchter seines Verwandten anbelange, so würden
wir Sorge tragen, das ihnen nichts vorgeschlagen werde, als was
mit der Wahrheit und Gerechtigkeit übereinstimme, und das sie
in der Furcht Gottes bewahrt bleiben, nach dem Sinn ihres
Vaters. Dies schien ihn sehr zu beruhigen. Während ich dort
war, kamen viele angesehene Leute zu mir, und ich ermahnte sie
alle, in der Furcht Gottes zu bleiben und aus seinen guten Geist
in ihrem Innern zu achten, und ihren Sinn aus Gott zu richten.
Nachdem ich zwei oder drei Stunden dort gewesen war und mit
ihm über verschiedene Dinge geredet hatte, nahm ich Abschied von
ihm, und er lies mich sehr freundlich in seinem Wagen nach
Rotterdam zurück führen [...].

Von da gingen wir nach Amsterdam\ort{Amsterdam} [...]. In Amsterdam
findet die Jahresversammlung für die Freunde von Holland,
Deutschland und andere Länder statt, und sie begann nun gerade,
am 8. des 4. Monats und endete am 12. Da hatten wir nun
eine schöne Gelegenheit, Freunde von überall her zu sehen und
gemeinsam in der Liebe Gottes erquickt zu werden. Nach den
Versammlungen, bevor die, welche aus den verschiedenen Gegenden
hergekommen waren, wieder fort gingen, hatten wir eine 
Versammlung mit einigen Freunden besonders, um zu beraten über
die Gegenden und Orte, in die wir im Dienst des Herrn gehen
sollten, und um zu erfahren, wer unter ihnen sich eigne, als 
Dolmetscher\index{Dolmetscher} mit uns zu gehen. Nachdem man übereingekommen war,
schissten sich William Bingley\person{Bingley, William} und Samuel 
Waldensield\person{Waldensield, Samuel} mit Jakob
Claus\person{Claus, Jakob}, ihrem Dolmetscher, für Friesland\ort{Friesland} ein.
Alexander Parker\person{Parker, Alexander} und George Watts blieben mit mir
noch einige Tage länger in Amsterdam, wo ich noch zu tun
hatte [...]. Ehe ich fort ging, besuchte ich Galenus 
Abrahams\person{Abrahams, Galenus},