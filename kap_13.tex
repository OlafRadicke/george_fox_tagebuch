%%%%%%%%%%%%%%%%%%% Kapitel 13. %%%%%%%%%%%%%%%%%%%%%%%%%%%%%%

\chapter[Ein Gottesgericht.]{Ein Gottesgericht.}

\begin{center}
\textbf{Ein Gottesgericht. Ermahnung zur Barmherzigkeit bei 
Schiffbrüchen. Qnakerfreundlicher Erlass des General Monk. 
Fox als Königsfeind
gefangen und schließlich auf Befehl Karls II. befreit.}
\end{center}

\section{Provokationen und Störungen in den Versammlungen}

Als ich nun meine Arbeit in London getan, ging ich nach
Surrey und Sussex,\ort{Sussex} [...] dann nach Hampshire,
\ort{Hampshire} Dorsetshire,\ort{Dorsetshire}
Ringwood\ort{Ringwood} und Poole,\ort{Poole} wo ich überall 
Freunde besuchte und große
Versammlungen unter ihnen hatte.
In Dorchester hatten wir eine große Abendversammlung in
unserer Herberge,\index{Versammlung!in Herberge} bei der 
viele Soldaten\index{Soldaten} zugegen waren, die alle
ziemlich anständig waren. Aber da erschienen die Wachen und
Schutzleute der Stadt unter dem Vorwand, sie müssten einen 
geschorenen Jesuiten suchen und verlangten, das alle ihre Hüte 
abnähmen,\index{Hut!abnehmen} oder sie würden sie abnehmen, 
um die Tonsur des Jesuiten\index{Schikanen}
zu finden. So nahmen sie mir den Hut ab und untersuchten
mich genau, denn mich hatten sie im Verdacht; aber als sie keine
kahle oder geschorene Stelle fanden, gingen sie beschämt fort. Und
% \picinclude{./140-149/p_s146.jpg} 
die Soldaten und andere Leute ärgerten sich sehr über sie. Aber
es förderte die Sache Gottes und alles diente zum Guten, denn
es machte den Leuten Eindruck, und nachdem die Beamten fort
waren, hatten wir eine schöne Versammlung, und viele wurden
zum Herrn Jesus bekehrt, ihrem Lehrer, der sie erkauft hat und
sie versöhnen will mit Gott. 


Von da gingen wir nach Somersetshire, wo die Presbyterianer 
\index{Presbyterianer} und andere \textit{Fromme} sehr böse
waren und oft die Versammlungen der Freunde störten. Einmal
\index{Versammlung!Störung} \index{Versammlung!Provokation}
hatten sie einen sehr schlechten Menschen dazu veranlasst, 
in eine Versammlung der Quäker zu gehen und eine Bärenhaut 
anzuziehen und Unsinn zu treiben. Er setzte sich gerade dem Freund,
der redete, gegenüber mit seiner Bärenhaut über dem Rücken
und streckte die Zunge heraus und machte eine große Unruhe.
Aber ein schweres Gericht kam über ihn, und seine Strafe schlummerte
nicht; als er aus der Versammlung heim ging, kam er an einer
Stierhetze vorüber und blieb stehen, um zuzusehen; als er aber
nahe bei dem Stier war, stieß dieser dem Mann das Horn in
den Hals, so das seine Zunge heraus hing, gerade wie er es 
vorher in der Versammlung gemacht hatte. Und der Stier stieß
sein Horn durch den Kopf des Mannes hindurch, und schwang
ihn schrecklich in der Luft herum. So kam er, der dem Volke 
Gottes hatte Schaden zufügen wollen, selber zu Schaden, und
es wäre gut, wenn solche Beispiele der Rache Gottes andere lehren
würden, sich zu hüten [...]. \index{Gott!Rache} \index{Strafe Gottes}

\section{Appell zur Rettung Schiffbrüchiger}

Wir gingen durch Somersetshire,\ort{Somersetshire} 
Plymouth\ort{Plymouth} und Devonshire\ort{Devonshire}
nach Cornwall\ort{Cornwall} [...]. Während ich hier war, 
geschahen große Schiffbrüche in der Nähe von Landsend. 
Nun war es Brauch \index{Schiffbruch}
in jener Gegend, das bei einem solchen Anlass reich und arm
hinaus ging, um so viel wie möglich von den Überresten an
sich zu bringen, unbekümmert um die Rettung der Menschen. An
einigen Orten nannten sie sogar einen Schiffbruch eine 
Gottesgnade.\index{Gottesgnade} Es betrübte mich, von 
solchem unchristlichen Treiben zu
hören und zu sehen, wie tief diese Leute unter den Heiden von
Melite stehen, die Paulus aufnahmen, ihm ein Feuer machten
und freundlich waren gegen ihn und die anderen Schiffbrüchigen
(Act. 28\bibel{Act. 28}). Darum trieb es mich, ein Schreiben 
an alle Gemeinden,
Priester und Behörden zu senden, um sie wegen ihres habsüchtigen
Treibens zu tadeln und sie zu ermahnen, so oft sie könnten, mit
Eifer behilflich zu sein, wenn es gelte, Menschenleben zu retten
% \picinclude{./140-149/p_s147.jpg} 
und Schiffe und Waren zu schützen; auch sollten sie auch bedenken, 
wie grausam es ihnen vorkommen würde, wenn sie selber
Schiffbruch litten und die Leute suchen würden, ihnen soviel wie
möglich zu rauben, ohne sich um ihre Rettung zu kümmern [...].
Diese Schrift hatte viel Erfolg beim Volk; die Freunde bemühten
sich um die Rettung der Schiffbrüchigen und den Schutz der
Schiffe und der Habe, ja, Freunde haben Schiffbrüchige, die halb
tot und Verhungert waren, bei sich aufgenommen und sie gepflegt
und unterstützt; alle wahren Christen sollten so handeln [...].

\section{Fox wird erneut verhaftet}

Die Soldaten, die unter dem Befehl des General 
Monk\person{General Monk} standen, waren damals oft sehr grob 
und störten die Versammlungen\index{Versammlung!Störung} 
der Freunde an manchen Orten. Als man sich darüber
beim General Monk beklagte, erließ er folgenden Befehl, worauf
es etwas besser wurde:

\grosszitat{
\begin{flushright}St. James, 9. März 1659.\end{flushright}
Ich will, das alle Offiziere und Soldaten sich hüten, die
friedlichen Versammlungen der Quäker zu stören, da sie nichts
tun, das dem Parlament oder dem Commonwealth von England 
zuwider ist. 
\bigskip
George Monk
}


Wir gingen [...] über Oldeston\ort{Oldeston} [...]
Nailsworth\ort{Nailsworth} [...] Drayton\ort{Drayton} [...]
Lancaster\ort{Lancaster} nach Swarthmore.\ort{Swarthmore} 
Ich war noch nicht lange dort, als Henry 
Porter,\person{Porter, Henry} ein Friedensrichter, einen Verhaftbefehl
sandte, um mich zu greifen. Ich hatte dies vorausgefühlt, und
so kam denn auch, während ich mit Richard 
Richardson\person{Richardson, Richard} und
Margaret Fell\person{Fell, Margaret} zusammen im Zimmer saß, 
ihre Dienerschaft und
meldete, es seien einige da, die durchsuchten das Haus, 
angeblich um zu sehen, ob Waffen darin seien. Es kam über mich, zu
ihnen hinaus zu gehen, und als ich an einem von ihnen vorüber
ging, redete ich ihn an, woraus sie mich nach meinem Namen
fragten; ich sagte ihn ohne weiteres, worauf sie mich ergriffen
und sagten, ich sei gerade der Mann, den sie suchten. Und sie
führten mich fort nach Ulverstone\ort{Ulverstone} [...]. 
Von da brachten sie mich nach Lancaster\ort{Lancaster} [...].


Als ich dorthin kam, war das Volk
sehr aufgeregt; ich blieb stehen und sah sie fest an, und sie
schrien: \zitat{seht diese Augen!} Nach einer Weile redete ich mit
ihnen, und da waren sie ziemlich ruhig. Ein junger Mann nahm
mich mit in seine Wohnung, und nach einiger Zeit kam ein 
Beamter und brachte mich zu Major Porter,\person{Major Porter} 
der den Befehl gegen mich erlassen hatte; es waren noch ein 
paar andere bei ihm. Als
% \picinclude{./140-149/p_s148.jpg} 
ich hereinkam, sagte ich: \zitat{Friede sei mit euch}. Porter fragte
mich, warum ich in dieser unruhigen Zeit hierher komme? Ich
erwiderte: \zitat{Um meine Mitmenschen zu besuchen}. \zitat{Ihr habt
überall herum große Versammlungen}, sagte er; ich erwiderte
ihm, diese Versammlungen seien aber im ganzen Lande als 
friedliche und ruhige bekannt, und wir seien ein friedliches Volk. Er
sagte: \zitat{Ihr seht aber den Teufel den Leuten im 
Gesicht geschrieben}.\index{Teufel}\index{Fox!Sieht den 
Teufel in Anderen} Ich erwiderte: 
\zitat{Wenn ich einen Trunkenbold oder
einen Schwörer oder einen Grobian sehe, so kann ich doch nicht
sagen, ich sehe den Geist Gottes in ihm.}\index{Geist Gottes} 
Und ich fragte ihn, ob. \index{Alkohol}
er den Geist Gottes sehen könne? Er sagte, wir treten gegen
ihre Prediger auf. Ich antwortete: \zitat{Als wir noch wie Saulus waren
und unter den Priestern saßen, da hat man uns nicht schädliche
Männer genannt} (Act. 24:5\bibel{Act. 24:05@Act. 24:5}) oder Sektenmacher; aber als wir
anfingen, Gott zu leben, so wurden wir schädliche Leute genannt
wie Paulus. 


Er sagte, wir könnten recht gut reden, er wolle
lieber nicht mit uns disputieren, aber greifen wolle er uns lassen. Ich
fragte ihn, warum und auf wessen Befehl er einen Verhaftbefehl
gegen mich ergehen lasse, und beklagte mich über die Behandlung
der Beamten bei meiner Gefangenschaft und auf dem Wege
hierher. Er hörte nicht auf mich, sondern sagte, er habe einen
Befehl, aber er wolle mich ihn nicht sehen lassen, denn er
wolle die Geheimnisse des Königs nicht preisgeben; und überdies 
brauche ein Gefangener nicht zu wissen, warum er verhaftet 
sei. Ich sagte ihm, das sei unvernünftig, wie der Gefangene 
sich denn dann verteidigen solle? er solle mir eine Abschrift
geben. Er sagte, es sei einmal ein Richter bestraft worden, weil
er einem Gefangenen den Verhaftbefehl gezeigt habe [...] und
er sagte mir, ich sei ein Friedenstörer im Land. Ich sagte, ich
sei im Gegenteil ein Segen für das Land durch die Kraft und
die Wahrheit des Herrn, und der Geist Gottes in den Gewissen
gebe Zeugnis hiervon. Dann beschuldigte er mich, ich sei ein
Feind des Königs und beabsichtige einen neuen Krieg anzustiften
und neues Blutvergießen über das Land zu bringen. Ich erklärte
ihm, ich habe nie die Gebräuche des Krieges gelernt und sei in
diesen Dingen so unwissend wie ein Kind. Da kam der Schreiber
mit dem ausgefertigten Verhaftbefehl und der Kerkermeister wurde
gerufen, und er musste mich ins Loch tun und niemand durfte
mich besuchen. Dort sollte ich nun gefangen bleiben, bis mich der
% \picinclude{./140-149/p_s149.jpg} 
König oder daß Parlament frei sprechen würde [...]. Ich ließ
nun Thomas Eummins\person{Eummins, Thomas} und Thomas 
Green\person{Green, Thomas} bitten, zum Gefangenwärter 
zu gehen und ihn um eine Abschrift des Verhaftbefehls zu
bitten, damit ich wisse, warum ich verurteilt sei. Sie gingen hin;
der Wärter sagte, er könne ihnen die Abschrift nicht geben, weil
einmal einer bestraft worden, der dies getan, aber sie könnten sie
durchlesen. Soviel sie sich nachher erinnerten, lautete die 
Anklage also: das ich im Verdacht stehe, ein Störer des 
Landfriedens zu sein, ein Feind des Königs und eine Hauptstütze der
Quäker-Sekte, und das ich, zusammen mit andern dieser 
Fanatiker, kürzlich versucht habe, Aufstände in dieser Gegend 
anzustiften und das Land in Blut zu tauchen. Darum müsse der
Kerkermeister mich in sicherem Gewahrsam behalten, bis ich auf
Befehl des Königs; oder Parlamentes befreit würde.


Als mir nun meine Anklage in der Hauptsache bekannt war,
schrieb ich eine kurze Erwiderung, um meine Unschuld zu zeigen;
sie lautete:

\grosszitat{

  Ich bin Gefangener in Lancaster, durch Friedensrichter
  Porter\person{Friedensrichter Porter} verhaftet. 
  Ich kann keine Abschrift der Anklage erhalten;
  doch erfahre ich, das sie Behauptungen enthält, die durchaus 
  unrichtig sind, z. B. das ich im Verdacht stehe, ein Friedenstörer
  zu sein und ein Feind des Königs, und das ich versuche, mit
  andern zusammen Aufstände anzustiften und Blutvergießen übers
  Land zu bringen; das ist gänzlich falsch und ich bestreite etc. Es
  trifft mich keinerlei Verdacht, ein Friedenstörer zu sein; denn
  ich bin über jeden dieser Punkte schon früher verhört worden, im
  ganzen Land herum. In den Tagen Cromwells bin ich gefangen
  genommen worden, weil es hieß, ich habe die Waffen\index{Waffen} 
  gegen ihn ergriffen, was falsch war, denn ich habe überhaupt nie
  Waffen getragen; dennoch wurde ich als Gefangener nach London 
  gebracht und vor ihn geführt; dort bewies ich ihm meine Unschuld
  und das ich ja überhaupt gegen das Gebrauchen irgend einer
  fleischlichen Waffe sei, da meine Waffen geistliche seien, solche,
  die die Ursachen des Krieges hinwegnehmen und zum Frieden
  führen. Daraufhin sprach mich Oliver frei. Darnach wurde ich
  gefangen durch Major Ceely\person{Major Ceely} in Cornwall 
  \ort{Cornwal} ins Gefängniß
  gebracht; er behauptete vor Gericht, ich hätte ihn beiseite 
  genommen und ihm gesagt, ich könne in Zeit einer Stunde 
  vierzigtausend Mann stellen, um das Land in Blutvergießen zu stürzen
  % Erster Durchlauf
  % \picinclude{./150-159/p_s150.jpg} 
  und den König Karl\person{König Karl} zurück zu bringen. 
  Das war alles ganz
  falsch und eine Lüge, die er selber erfunden, wie ihm auch bewiesen 
  wurde. Ich harte nie so etwas zu ihm gesagt; ich hatte
  mich nie an einer Verschwörung beteiligt; ich hatte nie einen Eid
  geschworen, nie Kriegsübungen\indexname{Kriegsübungen} 
  gemacht. Wie jenes falsche 
  Anschuldigungen gewesen, so sind es jetzt die, die Major Porter 
  vorgebracht [...] Ich bin kein Störer des Landfriedens, sondern
  ich suche den Frieden aller Menschen [...] Und ebenso ist es
  falsch, wenn Major Porter\person{Major Porter} sagt, ich sei ein Feind des Königs,
  denn ich liebe ihn und alle Menschen, wenn sie schon Feinde
  Gottes sind und ihre eigenen Feinde und meine.
  \index{Feinde Gottes} Ich weiß, das
  seine Rückkehr vom Herrn kommt, damit er viel begangenes Unrecht 
  wieder gut mache. \person{Fox!Vorhersehung} Ich hatte 
  ein Gesicht davon, drei Jahre
  ehe er zurück kam. Es ist eigentümlich, zu sagen, ich sei ein
  Feind des Königs; ich habe keinerlei Grund es zu sein, trotzdem
  ich allerdings viel verfolgt und eingesperrt gewesen bin während
  der letzten 11 oder 12 Jahre, von den Gegnern sowohl des
  jetzigen Königs als seines Vaters, also eben von der Partei, die
  Porter zum Major gemacht und für die er die Waffen führte,
  aber nicht durch die, die für den König war. Ich war nie ein
  Feind des Königs, noch irgend eines andern auf Erden. Ich
  habe die Liebe, die des Gesetzes Erfüllung ist, die nichts Böses
  denkt, sondern sogar die Feinde liebt, und möchte, das der König
  errettet würde und die Wahrheit erkennete und dazu käme, Gott
  zu fürchten und die Weisheit von oben zu erlangen, durch die
  alle Dinge gemacht sind, damit er in dieser Weisheit regierete zur
  Ehre Gottes [...].

  \medskip 

  Weil ich nun hier gefangen bin, bis ein Befehl vom König
  oder dem Parlament mich frei macht, so habe ich solches geschrieben, 
  damit ihr und der König und das Parlament es
  leset, und alles bedenket, ehe ihr etwas in der Sache tut; und
  in der Weisheit Gottes untersucht, was für Absichten zugrunde 
  liegen, damit ihr nicht etwas tut, womit 
  ihr die Hand \index{Gottesstrafe}
  des Herrn gegen euch wendet, wie viele Machthaber zuvor getan,
  die dann gestürzt wurden von dem Gotts, den wir fürchten und
  dem wir trauen und zu dem wir Tag und Nacht schreien, und
  der uns gehöret hat und noch erhört und uns rächen wird. Viel
  unschuldig Blut ist schon vergossen worden, und viele sind bis in
  den Tod verfolgt worden durch die, die vor euch die Herrschaft
  % \picinclude{./150-159/p_s151.jpg} 
  hatten; und Gott hat sie ausgespien, weil sie sich gegen das
  Recht kehrten. Darum prüfet, wie es um euch steht, solange es
  Tag ist, und nehmet dieses auf als eine Warnung in Liebe
  an euch.

  \medskip 

  Von einem der unschuldig in Lancaster gefangen liegt, genannt
  George Fox [...]

}

Bald darauf gab ich eine Schrift gegen das Verfolgen heraus:

\brief{Verfolger}{
  Die Papisten\index{Papisten}, die 
  Common-Prayerleute\index{Common-Prayerleute}, die 
  Presbyterianer\index{Presbyterianer}, Independenten\index{Independenten} 
  und Baptisten\index{Baptisten} verfolgen einander um ihrer eigenen
  Erfindungen willen, ihren Messen, ihren Common-Prayer Bücherm,
  ihrem \zitat{Directory} und Bekenntnis, dies sie aufgesetzt haben,
  aber nicht zum Nutzen der Wahrheit; denn sie wissen nicht, wes
  Geistes Kind sie sind, wenn sie verfolgen und die Leben der Menschen
  zu zerstören suchen um des Kirchendienstes und der Religion
  willen, während Christus sagte, er sei nicht gekommen, das
  Leben der Menschen zu zerstören, sondern es zu retten (Luc. 9\bibel{Luc. 09@Luc. 9}).
  Wir können uns doch nicht solchen anvertrauen, die nicht wissen,
  wes Geistes Kind sie sind [...] Ihr möchtet gerne ein Gebot
  haben, um zu zerstören, wie einst die Jünger wollten Feuer vom
  Himmel regnen lassen, um die, welche Christus nicht aufnehmen
  wollten, zu zerstören [...] Die, welche das Leben der Menschen
  zerstören, sind nicht Jünger Christi, des Heilands, [...] wenn ihr
  die Leben anderer zerstört und verfolget und nicht Buße tut, werdet
  ihr nicht auferstehen zum Leben mit Gott. Die aber, die wissen
  wes Geistes Kinder sie sind, die haben den untadeligen Eifer und
  geben durch den Geist Gottes dem Herrn Leib, Seele und Geist,
  die sein sind, das er sie bewahre. [...]. 

  \bigskip 

  \begin{flushright}G. F.\end{flushright}

}

Es trieb mich auch, an den König zu schreiben, um ihn zu
ermahnen, Barmherzigkeit zu üben gegen seine Feinde und der
Zügellosigkeit und Gottlosigkeit, die bei seiner Rückkehr im Lande
aufgekommen war, zu steuern.

\brief{König Karl}{
  \begin{center}An den König:\end{center}
  \medskip 

  O König Karl,
  \medskip 

  Du kamst nicht ins Land durch Schwert noch durch Sieg im
  Kriege, sondern durch die Kraft des Herrn; wenn du nun nicht
  in derselben lebest, so wirst du nicht gedeihen. Wenn der Herr
  dir Barmherzigkeit erzeigt hat und dir vergeben hat, und du
  übest nun nicht auch Barmherzigkeit und Vergebung, so wird der
  % \picinclude{./150-159/p_s152.jpg} 
  Herr deine Gebete nicht erhören, noch die Gebete derer, die für
  dich beten. Wenn du nicht den Verfolgungen Einhalt gebietest
  und nicht die Gesetze, welche das Verfolgen um des Glaubens
  willen gestatten, abschaffst [...] so wirst du so blind werden
  wie deine Vorgänger; denn das Verfolgen hat immer die Verfolger 
  blind gemacht. Solche aber stürzet Gott durch seine Macht
  und verfährt streng mit ihnen; aber den Bedrückten schickt er
  Rettung. Wenn du das Schwert umsonst trägst und lässest
  Trunkenheit, Schwören, Spielen und dergleichen eitles Treiben
  ungestraft, wie z.B. das Aufstellen der Maibäume mit dem Bild
  der Krone oben drauf und dergleichen, so wird das Land bald
  sein wie Sodom und Gomorra und so schlecht, wie die alte Welt,
  die den Herrn so betrübte, das er sie untergehen ließ. So wird
  er auch euch tun, wenn ihr solche Dinge nicht abschafft. Es hat
  kaum je solche Freiheit, Unrecht zu tun, geherrscht wie jetzt, als
  ob es nicht Gewalt noch Schwert der Obrigkeit mehr gäbe; und
  solches ist weder der Regierung noch denen, die recht tun, zum
  Nutzen. Wir beten für die, welche die Herrschaft haben, das
  wir ein ruhiges Leben unter ihnen führen können, in Frieden
  und Gottseligkeit, und das wir nicht durch sie in Gottlosigkeit
  fallen. Höre und denke darüber nach und tue Gutes, so lange
  du kannst und Macht hast. Sei barmherzig und vergib; dies
  ist der Weg, aus dem du überwindest und das Reich Christi
  erlangst.

  \medskip 

  \begin{flushright}G. F.\end{flushright}
}

Es ging lange, ehe der Sheriff einwilligte, mich nach London\ort{London}
überzuführen, es sei denn, das ich die Kosten trage, was ich 
verweigerte. Schließlich, als sie sahen, das es nicht anders ging,
gab der Scheriff zu, das ich mit einigen Freunden nach London
gehe, ohne andere Verpflichtungen als mein Versprechen, an
dem und dem Tage vor den Richtern in London zu erscheinen,
so der Herr es zulasse, woraus ich entlassen wurde [...]
Etwa drei Wochen nach meiner Freilassung gelangte ich nach
London. Als wir nach Charing Cros\ort{Charing Cros} kamen, war dort eine
ungeheure Menschenmenge versammelt, um zu sehen, wie die
Überreste einiger der früheren Richter des Königs verbrannt
wurden, die erhängt, ertränkt und gevierteilt worden waren [...].

Als wir den Richtern die gegen uns gerichtete Anklage eingereicht 
hatten, und sie die Worte lasen: meine Freunde und ich
trachteten Blutvergießen im Lande anzurichten, schlugen sie mit
% \picinclude{./150-159/p_s153.jpg} 
der Hand auf den Tisch; ich erklärte, das ich der Mann sei,
gegen den diese Anklage gehe, aber ich sei an allem derartigen
so unschuldig, wie ein neugeborenes Kind, und habe die Anklage
selber hierher gebracht, und meine Freunde seien ohne Wache mit
mir gekommen. Sie hatten bis jetzt meinen Hut noch nicht
beachtet, aber jetzt fiel er ihnen auf und sie fragten: \zitat{Was! ihr
steht hier im Hut?} Ich sagte ihnen, es geschehe keineswegs
aus Mangel an Achtung vor ihnen. Darauf befahlen sie, das
man mir ihn abnehme, und dann riefen sie den Marschall von
Kings-Bench und sagten zu ihm: \zitat{Ihr müsst diesen Mann in
Gewahrsam bringen, gebt ihm aber ein Zimmer und legt ihn nicht
unter die anderen Gefangenen.} Als der Marschall erklärte, er
habe kein Zimmer frei, das er mir geben könnte, so fragten sie
mich: \zitat{Wollt ihr morgen um zehn in Westminsterhall vor 
Kings-Bench erscheinen?} Ich antwortete: \zitat{Ja, wenn der Herr mir die
Kraft gibt.} Hierauf sagte Richter Foster\person{Richter Foster} zum anderen Richter:
\zitat{Wenn er sagt ja und es verspricht, so könnet ihr auf sein Wort
gehen.} Somit war ich entlassen. Am nächsten Tage erschien
ich zur bestimmten Stunde vor Kings-Bench\ort{Kings-Bench} [...] und als ich
eintrat, trieb es mich zu sagen: \zitat{Friede sei mit euch,} und die
Kraft des Herrn kam über alle. Meine Anklage wurde öffentlich
verlesen. Die Leute waren ziemlich still und die Richter ruhig
und freundlich, die Gnade des Herrn war mit ihnen. Aber als
sie an die Stelle kamen, wo es hieß, ich und meine Freunde
wollten Blutvergießen über das Land bringen und einen neuen
Krieg anstiften, und das ich ein Feind der Königs sei, da hoben
sie ihre Hände auf. Da erhob ich meinen Arm und sagte: \zitat{Ich
bin der Mann, gegen den sich diese Anklage richtet, aber ich bin
so unschuldig wie ein neugeborenes Kind in dieser Sache, und
habe nie den Gebrauch der Waffen gelernt. Und meint ihr, wenn
ich oder meine Freunde solche wären, wie es in der Anklage
heißt, so hätten wir unsere Anklage selber hierher gebracht?} [...]
Sie fragten mich, was man mit der Anklage tun solle? ich sagte:
\zitat{Ihr seid die Richter und könnt hoffentlich in dieser Sache
richten; tut also was ihr wollt. [...]}

Sie sagten, sie wollten mich nicht verurteilen, denn sie hätten
nichts gegen mich. Esquire Marsh\footnote{Esquire Marsh, eine 
angesehenen Persönlichkeit am Hofe Karl II, war den
Ouäkern geneigt und bemühte sich oft für sie und schützte sie vor 
Verfolgungen.}\person{Marsh, Esquire} erhob sich und sagte, es sei
% \picinclude{./150-159/p_s154.jpg} 
des Königs Wohlgefallen, das ich freigesprochen werde, wenn sich
kein Kläger gegen mich erhebe. Sie fragten mich, ob ich es dem
König und dem Rat überlassen wolle? Ich sagte: \zitat{Ja, gerne.}
Hierauf schickten sie des Scheriffs Bericht, der die Anklage enthielt,
dem König, damit er sehe, wessen man mich beschuldigte [...]

Der König, nachdem er dies gelesen und von der ganzen
Angelegenheit unterrichtet war, war von meiner Unschuld überzeugt 
und sandte einen Befehl, mich frei zu lassen, welcher lautete:

\brief{König}{
  Es beliebt Seiner Majestät, zu befehlen, das man dem Manne
  George Fox, bislang Gefangener im Kerker von Lancaster, die
  volle Freiheit schenke [...] Und diese Kundgebung von Seiner
  Majestät Belieben soll euch als Befehl genügen.

  \begin{flushright}Whitehall, 24. Oktober,1660\index{Jahr!1660}. 
Edward Nicholat\person{Nicholat, Edward}\end{flushright}
}

Als ich nun mehr als zwanzig Wochen gefangen gewesen
war, war ich auf Befehl des Königs rechtmäßig frei geworden;
die Macht des Herrn hatte meine Unschuld herrlich kund getan.
Porter wagte nicht, die Anklage, die er fälschlich gegen mich
erhoben hatte, öffentlich zu berichtigen [...]


