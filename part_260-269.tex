% \picinclude{./260-269/p_s260.jpg} 
260 Kapitel 1311.
Jch hörte später, daß seine Hörer ihn wegen dessen, was er
in unsrer Versammlung gesagt, zu Rede stellten, und als er dazu
stand, ihn bei den andern Priestern der Stadt oerklagten, die ihn
darüber zur Rechenschaft zogen, aber vom Ausgang der Sache
konnte ich nichts erfahren .....
Am folgenden Tage gingen wir nach Amsterdam, wo wir
etwas nach Mitternacht ankamen, und da die Tore geschlossen
waren, so blieben wir bis zum folgenden Tage auf dem Schiff
und gingen dann ins Haus oon Gertrud Ditick; hier besuchten
uns viele Freunde, froh, daß wir wohlbehalten wieder zurück
waren. Am folgenden Tage fühlte ich mich im Geist beunruhigt
wegen etlicher verführerischer Geister, die Uneinigkeit tmter die
Freunde brachten, und weil ich merkte, daß sie sich suchten in Grmst
zu bringen, so trieb es mich, einige Zeilen deswegen an die
Freunde zu schreiben: ,,Alle, die sich in die Gunst der Leute ein-
schmeicheln wollen, trachten, sich in Gunst zu bringen statt Christus.
Aber Freunde, euer friedsames Bleiben in der Wahrheit, die ewig
ist und sich nicht verändert, wird alles, was nicht aus der Wahr-
heit ist, überdauern, auch wenn es mit noch so viel Worten aus-
tritt. Lasset denn die, welche so für J. S. und J. W. auftreten
zu ihnen halten und sich von euch trennen, und ihr, die ihr Zeug-
nis abgelegt gegen diesen Geist, beharret bei diesem Zeugnis, bis
sie euch mit Anschuldigungen angreifen. Zanket nicht, lasset euch
nicht ein mit etwas, das nicht in der Wahrheit steht, noch suchet
lebendig zu erhalten, was sollte Gott zum Opfer gebracht werden,
damit ihr nicht des Reiches oerlustig geht«.
Amsterdam, 14. des 7. Monats 1677. G. F.
An einem großen Fasttage wohnte ich einer der Versamm—
lungen der Freunde bei. Jch hatte eigentlich vorgehabt, nach
Haarlem zu gehen, aber ich wurde in meinem Geist gehalten, zu
bleiben. Wir hatten eine sehr große Versammlung, eine große
Menge Leute strömte herbei, worunter viele angesehene Personen.
Die Kraft des Herrn war über der Versammlung, und in den
Offenbarungen, die ich während derselben hatte, trieb es mich,
darzutun, daß niemand, mit allem Studieren und allem Verstand
oder mit dem Lesen der Geschichte, wenn er sie nach seinem eigenen
Willen lese, die Abstammung von Christus wisse, der nicht nach
dem Willen eines Menschen, sondern nach dem Willen Gottes
gezeugetssei. . . . Nachdem ich ihnen das ausführlich erklärt hatte,


% \picinclude{./260-269/p_s261.jpg} 
Reise nach Holland. Einrichtung der kirchlichen Ordnung usnz. 261
erklärte ich ihnen den Unterschied zwischen wahrem und falschem
Fasten. Jch zeigte ihnen, wie alle, ob sie sich nun Christen, Juden
oder Türken nennen, nicht in der rechten Weise fasten, sondern sie
fasten, »daß sie hadern und zanken und mit gottloser Faust schlagen«
(Jes. 58,4), sie erheben nicht reine Hände zu Gott. Und wenn sie
schon vor den Leuten tun, als ob sie fasteten, und ,,deS Tagß den
Kopf hängen wie ein Schilf, so ist es doch nicht das Fasten, das
Gott erwählt« (Jes. 58,5). Darum sind ihre Gebeine oertrocknet,
und wenn sie den Herrn anrufen, so hört er sie nicht und »ihre
Besserung wächst nicht« (Jes. 58,8), weil sie ihr eigeneö Fasten
halten und nicht daß des Herrn. Jch ermahnte sie, daß Fasten
deß Herrn zu halten, welchetz ein Fasten von der Ungerechtigkeit
und der Sünde sei, vom Streiten, Hadern und Unterdrücken, und
auch allen bösen Schein zu meiden. Die Leute, die Fasttage
hielten, wunderten sich sehr über diese Eröffnungen, und die Ver-
sammlung nahm ein schöne-Z friedlicheß Ende.
Am folgenden Tage ging ich nach Haarlem, wo ich zuvor eine
Versammlung angesagt hatte. Peter Hendricke und Gertrud
Dirick Nieson gingen mit mir und wir hatten eine gesegnete Ver-
sammlung. ES waren verschiedene ,,Fromme« dabei, auch ein
Priester der Lutheraner, der mehrere Stunden andächtig zuhörte,
während ich ihnen die Wahrheit verkündete, Gertrud verdol-
metschte. Als die Versammlung zu Ende war, sagte der Priester,
er habe nichts darin gehört, daß nicht nach dem Worte Gottes
gewesen wäre, und er wünschte unß, daß der Segen Gotteß mit
unß und unsren Versammlungen sein möge. Auch andere erklärten,
man habe ihnen noch nie zuoor die Dinge so verständlich auß-
einandergesetzt.
Wir brachten die Nacht im Hause eineö Freundes, Dirirk
Klassen, zu, am folgenden Tage kehrten wir nach Amsterdam zu
Gertrud Dirick zurück; wir waren noch nicht lange da, alß ein
berühmter Priester kam, der früher unter dem deutschen Kaiser
gestanden, mits einem anderen deutschen Priester, um mit mir zu
reden..««Jch ergriff die Gelegenheit, um ihnen den Weg der
Wahrheit zu erklären, indem ich ihnen zeigte, wie sie dazu kommen
können, Gott und Christuß und sein Goangelium und Gesetz zu
kennen; ich zeigte ihnen, daß sie niemals durch Studieren und
durch Philosophie dazu kommen skönnen, sondern durch göttliche
Offenbarung, durch den Geist Gotteö, der ihnen in der Stille sdeß-


% \picinclude{./260-269/p_s262.jpg} 
262 Kapitel 25111.
Herzenß kund werde. Die Beiden waren empfänglich und gingen
befriedigt fort.
Am folgenden Ersten Tage war ich in einer Versammlung
der Freunde in Amsterdam; außer vielen Verschiedenen ,,Frommen«
war auch ein Doktor auß Polen anwesend, der um seiner Religion
willen auS seiner Heimat verbannt war; während der Versamm-
lung wurde er ergriffen vom Zeugniß der Wahrheit und kam
nachher zu mir und wünschte eine Unterredung mit mir, und
nachdem wir eine Zeitlang miteinander geredet, und ich ihm die
Dinge noch mehr erklärt hatte, ging er sehr empfünglich und in
Liebe zur Wahrheit fort.
Während ich in Amsterdam mar, brachte ich viel Zeit mit
Schreiben für die Wahrheit zu. Jch schrieb von hier mehrere Briefe
an die Freunde in England, ebenso: »Eine Warnung an die
Bewohner der Stadt Oldenburg«, die kürzlich abgebrannt war,
ferner: ,,Eine Warnung an die Bewohner der Stadt Hamburg« . . .
Ferner schrieb ich einen Brief an die Gesandten, die zu Nym-
wegen über den Frieden oerhandelten .....
Jch schrieb auch an die Behörden und Priester von Emden,
um ihnen zu zeigen, wie unchristlich etz sei, die Freunde zu ver-
folgen. Mehrere andere Bücher schrieb ich, Antworten an Priester
und andere, in Hamburg, Danzig und anderwärtß, um die Freunde
und die Wahrheit von allen Beschuldigungen und Verleumdungen
zu reinigen ..... Ferner ,,Ein Brief über daß wahre Fasten,
daß wahre Beten, und die wahre Ehre, gegen die Verfolgungen
und für die wahre Freiheit in Christuß Jesu?-, damit ihr in eurem
Halten von Tagen, Monaten, Zeiten und Festen Sorge tragen
möget, daß der Apostel nicht umsonst an euch gearbeitet habe,
und ihr nicht von neuem »den dürftigen Satzungen dienet«
(Gal. 4, 9) und sie anderen auferlegt.«
,,Wo haben je Christus oder seine Apostel den Gläubigen oder
den Christen befohlen, Feste oder Tage zu halten? Zeiget unö,
wo in den Schriften dez Neuen Testamentß, in den vier Evan-
gelien, in den Briefen oder in der Offenbarung geschrieben steht,
daß Christuß oder die Apostel je befahlen, die Zeit, die man
Ch:-istfest nennt, zu feiern, oder den Tag von der Geburt Christi,
oder die Zeit, die man Ostern nennt, oder den weißen Sonntag,
oder Petruö, Pauluß, Lukatz oder Markus, oder irgend eines-
andern Heiligen Tag? . . .


% \picinclude{./260-269/p_s263.jpg} 
Reise nach Holland. Einrichtung der kirchlichen Ordnung usw. 263
GS war des Apostelö Arbeit, sie auö den Banden dieser
Satzungen zu befreien. Und alß sie sich dann dem Halten der
Tage wieder zuwandten, fürchtete er, er habe umsonst an ihnen
gearbeitet; und er ermahnte sie, Gal. 5, 1: ,,So bestehet nun
in der Freiheit, damit euch Christus befreit hat, und fallet
nicht wieder in daß knechtische Joch der Siinde.« Hiemit sagt
er, daß sie einst in dem knechtischen Joch gefangen waren. Aber
ach, wie sehr die sogenannten Christen seit den Tagen der Apostel
wiederum in dieseö Joch gekommen sind, indem sie wieder Fasten
und Tage hielten, daß sieht man an ihrem Tun. Ja, zwingen
nicht sogar sowohl die Papisten wie die Protestanten die Leute,
Tage, Monden und Jahreszeiten zu halten? . . . ET- war und
ist Christus, der die Menschen von diesen dürftigen Satzungen frei
macht, darum sollen die Erlöften fest stehen in der Freiheit, womit
Christus sie befreit hat .... Die so in diesen Satzungen stehen, und
andere dazu zwingen wollen, sind abgewichen von der Grkenntniß
Gottetz und stehen nicht sest in der Freiheit, mit der C-hristuS befreiet.
Was- daß Beten anbelangt, so sehen wir nirgends, daß Christuö
oder die Apostel je jemanden zwangen, mit ihnen zu beten oder
zu fasten. Sondern Christus zeigte, wie man beten solle und sich
von den Heuchlern unterscheiden. Seine Worte sind: ,,Wenn du
beteft, so sollst du nicht sein wie die Heuchler, die da gerne stehen
in den Schulen und an den Ecken aus den Gassen, auf daß sie i
von den Leuten gesehen werden ..... Wenn du betest, so gehe
in dein Kämmerlein und schließe die Türe zu und bete zu deinem
Vater im Verborgenen, und dein Vater, der ins Verborgene sieht,
wird dirö oergelien öfsentlichss (Matth. 6, 5) ..... Und wir
tun nun, wie ez die Apostel und Heiligen getan. Wir beten im
Verborgenen und öffentlich, je nachdem der Geist es uns ein-
gibt, welcher unsrer Schwachheit hilft, wie er den Aposteln und
allen wahren Christen half; so beten wir für unö und für alle
Menschen, hoch und niedrig. .... über daö Fasten sagt
Christus: »Wenn ihr fastet, so sollt ihr nicht sauer sehen wie die
Heuchler, sie verstellen ihr Angesicht, auf daß sie vor den Leuten
scheinen mit ihrem Fasten. Wenn du saftest, so salbe dein Haupt
und wasche dein Angesicht, und dein Vater, der ins Verborgene
sieht, wird dirs vergelten öffentlich-- (Matth. 6). Jn Jesaia 58
heißt es: ,,Rufe laut und schone nicht, erhebe deine Stimme wie
eine Posaune und oerkiindige meinem Volk ihr lildertreten und


% \picinclude{./260-269/p_s264.jpg} 
264 Kapitel 12111.
dem Hause Jakobs ihre Sünde; sie suchen mich täglich und wollen
meine Wege wissen, als- ein Volk, das Gerechtigkeit schon getan,
und das Recht ihres Gottes- nicht verlassen hätte; sie fordern
mich zum Recht und wollen mit ihrem Gott rechten. Warum
fasten wir und du siehest es nicht an? warum tun wir unserm
Leib wehe und du willst es nicht wissen? .... Siehe, ihr fastet,
daß ihr hadert und zanket und schlaget mit der—Faust ungöttlich.
Fastet nicht also, wie ihr jetzt tut, daß ein Geschrei von euch
in der Höhe gehört wird. Sollte das ein Fasten sein, das- ich
erwählen soll, .... wollt ihr das ein Fasten nennen und einen
Tag dem Herrn angenehm? Das ist ein Fasten, das ich erwähle:
»Laß los-, welche du mit Unrecht gebunden hast; laß ledig, welche
du beschwerest; gib frei, welchH du drängest; reiß weg allerlei
Last.« .... Das- Fasten also, das der Herr verlangt, ist nicht,
daß man Lasten auserlege und die Banden der Sünde noch be-
sestige, sondern solche Bande zu lösen und zu sprengen.
Und nun darüber, daß wir den Hut nicht abnehmen vor den
Leuten. Viele, die sich Christen nennen, haben Anstoß an uns-
genommen, weil wir den Hut nicht abnahmen und uns- nicht vor
ihnen Verneigten. Wir finden nirgends, daß Christus das- geboten
hat, sondern eher das Gegenteil. Christus sagt: »nehmet nicht
Ehre von den Menschen;« und ferner sagt Christus: ,,wie könnet
ihr glauben, gdie ihr Ehre von einander nehtnet, und die Ehre
die von Gott kommt, sucht ihr nicht« (Joh. 5). Christus nennt
es ein Kennzeichen der Ungläubigen, Ehre voneinander zu nehmen
und die Ehre, die von Gott kommt, nicht zu suchen, und ist denn
nicht das Mnehmen des- Hutes- und das Vet-neigen eine Ehre, die
sich die Menschen untereinander erzeigen, nach welcher sie trachten
und beleidigt sind, wenn sie ihnen nicht erzeigt wird? Haben sie .
nicht sogar etliche gebüßt, versolgt und gefangen genommen, weil
sie den Hut nicht abnahmen? Ja, verhöhnen nicht die Türken
die Christen in ihrem Sprichwort, welches sagt, die Christen
bringen einen großen Teil ihrer Zeit damit zu, ihre Hüte abzu-
nehmen und einander ihre kahlen Köpfe zu zeigen? Sollten nun
die, welche den edlen Namen Christen tragen diirsen, nicht über
den Türken stehen und über dem Trachten nach Menschenehre
und dem Verfolgen solcher, die ihnen diese Ehre nicht erweisen
wollen, wie überhaupt alle wahren, gläubigen Christen allein die
Ehre suchen sollten, die von Gott kommt? ES heißt: ,,Wer an


% \picinclude{./260-269/p_s265.jpg} 
Reise nach Holland. Einrichtung der kirchlichen Ordnung usw. 265
den Sohn Gottes glaubt, der hat daß ewige Leben; wer aber
nicht glaubt, der wird verdammt werden'' (Joh. 3, 36). Jst
nicht die Redentzart der Türken, daß die Christen so viel Zeit
darauf verwenden, ihre Hüte abzuziehen und einander ihre kahlen
Köpfe zu zeigen, ein Vorwurf für die Christen? Habt ihr nicht
viele gefangen genommen und bestraft, weil sie den Hut nicht vor
euch abnehmen wollten und euch ihre kahlen Köpfe zeigen? Ja,
in vielen eurer Städte und Staaten haben solche, die ihre Hüte
nicht abnehmen und ihre kahlen Köpfe nicht zeigen, weder Freiheit
noch Recht, obgleich sie treue Untertanen sind. Habt ihr nicht
ein Gesetz gegen sie erlassen, daß sie zwei Gulden bezahlen müssen,
wenn sie es nicht tun? Und trachtet ihr nicht, sie dazu zu zwingen
und bestraft sie, wenn sie es- nicht tun, wie in Lan?-meer in
Waterland? Jst denn das nicht trachten nach Menschenehre?
Taten nicht die Pharisäer und Juden also? ....
Jhr habet keinerlei Befehl von Ehristuö oder einem seiner
Apostel, irgend jemanden zu verfolgen, zu bestrafen oder gefangen
zu nehmen um seiner Religion willen.«
Harlingen in Frießland, 11.deS 6.MonatZ 1677. G. F.
Bald daraus kamen William Penn und George Keith von
Deutschland nach Amsterdam zurück und hatten einen Diöput mit
Galenus Abrahamö, einem der bekanntesten Baptisten in Holland.
Viele »Fromme« waren zugegen; da sie nicht Zeit hatten, den
Diöput zu beendigen, kamen sie am folgenden Tage noch einmal
zusammen, und da wurde der Baptist gänzlich geschlagen, und die
Wahrheit gewann Boden ..... Als wir nun unsern Dienst in
Amsterdam getan, gingen wir, zu Wagen, nach Leyden. Wir
kamen dort mit einem Deutschen zusammen, der teilweise bekehrt
wurde. Er sagte uns von einem hervorragenden Mann, der die
Wahrheit suche. Etliche fanden ihn auf und besuchten ihn und
sanden einen ernst gesinnten Mann in ihm. Jch redete auch mit
ihm, und er bekannte sich zur Wahrheit. William Penn und Benjamin
Furly besuchten noch einen andern angesehenen Mann, der ein wenig
außerhalb von Leyden wohnte, von dem es hieß, er sei General
beim König von Dänemark gewesen. Gr und seine Frau waren
sehr liebevoll mit ihnen und nahmen die Wahrheit mit Freuden auf.
Von Leyden gingen wir nach dem Haag, wo der Prinz von
Oranien seinen Hof hielt, und wir besuchten einen von der Regierung
von Holland, mit dem wir eine ziemlich lange Unterredung hatten.


% \picinclude{./260-269/p_s266.jpg} 
266 Kapitel IIUI.
. . .. Von da gingen wir über Delft nach Rotterdam, wo wir
einige Tage blieben und mehrere Versammlungen hatten. Hier
verfaßte ich auch ein Buch an die Juden, mit denen ich gerne,
alö ich in Amsterdam war, mich unterredet hätte, aber sie wollten
nicht. Jch erhielt hier auch einige andere Bücher und Schristen,
die ich früher herauzigegeben und die nun übersetzt waren.
Kapitel Ickclll.
Rückkehr nach England. Kampf der Orduungöpartei gegen die
nnbotmsißigen Quiiker. Briefe über Toleranz an den König von
Polen, den Großmogul und andere.
Da wir spürten, daß wiryunser Werk in Holland getan
hatten, nahmen wir Abschied von den Freunden in Rotterdam ....
Am 21. des 8. Monate reisten wir nach England ab, William
Penn, George Keith und ich und Gertrud Dirick Nieson mit ihren
Kindern. Wir hatten eine lange und gefahroolle Überfahrt ....
aber der Herr, der den Winden gebieten kann und die stürmischen
Wellen des Meeres- stille macht, daß sie auf und nieder gehen,
wie es ihm gefällt, er behütete unß ..... Am Abend des 23.
kamen wir in Harwich an. Am nächsten Morgen gingen William
Penn und George Keith mit mir nach Eolchester .... Dort blieben
wir biz zum Ersten Tag, da es mich oerlangte, der Versammlung
der Freunde beizuwohnen. EZ war eine riberauß zahlreiche und
wirksame, denn ale- die Freunde von meiner Rückkehr hörten,
strömten sie von allen Seiten herbei vom Lande und auch autz
der Stadt, so daß etwa tausend Menschen anwesend waren ....
Am 9. dez 9. Monats kam ich nach London, wo ich mit großer
Freude empfangen wurde.
Als ich einige Zeit in London war, schrieb ich folgenden Brief
an meine Fran: ,
,,LiebeS Herz,
Dir und den Kindern meine Liebe und allen andern Freunden
in der Wahrheit, der Kraft und dem Samen deß Herrn, der über
allem ist. Dem Herrn sei Ehre und sein Name sei immerdar
hochgelobt! Er hat mich durch allerlei Trübsal und Gefahr hin-
durch geführt, in seiner ewigen Kraft; ich bin zweimal in der Ver-
sammlung in Gracechurchstreet gewesen, und obgleich auch feindliche
Geister zugegen waren, mar doch alleö ruhig; der Tau des Himmelß
fiel aus die Anwesenden, und die Herrlichkeit dez Herrn schien


% \picinclude{./260-269/p_s267.jpg} 
Rückkehr nach England. Kampf der Ordiunigöpartei usw. 267
über allen. Jch muß wohl oder übel täglich zu Versammlungen
gehen, in geschäftlichen Angelegenheiten und wegen allerlei Drang-
sal, deren ez viele gibt rings umher, und oiele Freunde haben
gegenwärtig darunter zu leiden, darum in Eile euch alle gr«üßend.«
London, 24. des 9. Monats 1677. G. F.
Um diese Zeit erhielt ich Briefe aus Neu-England, welche
berichteten, wie die Behörden grausam und unchristlich gegen die
dortigen Freunde oerfuhren, indem sie sie abscheulich mifzhandelten
und peitschten; sie peitschten viele Frauen unter den Freunden.
Eine Frau banden sie an einen Karren und fchleppten sie halb-
entblößt durch die Straßen. Sie peitschten einige Schisfökapitäne,
die selber keine Freunde waren, nur weil sie Freunde hergebracht
hatten. Währenddem sie aber in dieser barbarischen Weise die
Freunde verfolgten, schlugen die Jndianer sechzig ihrer Leute,
nahmen einen der Führer gefangen und zogen ihm bei lebendigein
Leib die Haut vom Kopf und trugen sie im Triumph davon.
Manche einsichtige Leute sagten: »GotteS Gericht ist über sie ge-
kommen, weil sie die Quiiker oerfolgten.« Aber die oerblendeteu,
verfinsterten Priester sagten, ez sei, weil sie sie nicht genug ver-
folgt hätten. Jch hatte große Mühe, für die fernen leidenden
Freunde Erleichterung zu schaffen, damit sie nicht unter die Rute
der Bösewichter kämen .....
Ich blieb etwa einen Monat in London; darauf ging ich
nach Buckinghamshire und besuchte die dortigen Freunde und
hatte mehrere Versammlungen. Qfterß machten während derselben ,
solche, die von der wahren Einigkeit der Freunde in der Wahrheit
abgewichen und in Zank, Zwiespalt und Auflehnung geraten
waren, große Störungen, besonder-3 während der Männeroer=
sammlungen bei Thomaß Gllwoodß in Hunger Hill; ihr Anführer
kam von Wickham und versuchte die Freunde zu stören und
an der weiteren Abhaltung der Versammlung zu hindern. A18
ich ihr Vorhaben merkte, ermahnte ich sie, ruhig und vernünftig
zu sein und die Versammlung nicht durch Unterbrechungen zu
stören; sondern, wenn sie mit dem Vorgehen der Freunde nicht
einverstanden seien und etwas dagegen einzuwenden hätten, dafür
eine Versammlung auf einen andern Tag zu veranstalten. Die
Freunde boten ihnen an, an einem folgenden Tag eine Versamm-
lung für sie abzuhalten, und schließlich wurde eine solche für die
darauffolgende Woche bei Thoma?. Ellwood festgesetzt. Die


% \picinclude{./260-269/p_s268.jpg} 
268 Kapitel 12c11l.
Freunde trafen sie dort, und die Versammlung sand in der
Scheune statt, weil so viele gekommen waren, daß das Haus sie
nicht fassen konnte. Nachdem wir eine Zeitlang dagesessen hatten,
fingen sie an mit ihren Zänkereien. Die meisten ihrer Pfeile
waren gegen mich gerichtet; aber der Herr war mit mir und
stärkte mich, daß ich in seiner Kraft die Pfeile der Bosheit und
Falschheit gegen sie selber zurück schleudern konnte. Jhre Ent-
gegnungen wurden widerlegt, und manches wurde den Leuten
geoffenbart, und die Wahrheit wurde gefördert; viele, die zuvor
schwach gewesen, wurden gestärkt und gefestigt; etliche, die ge-
schwankt und gezweifelt, wurden überzeugt und befestigt, und die
gläubigen Freunde wurden erquickt und ermuntert im Wachstum
des Lebens. Denn die Kraft wuchs unter uns, und das Leben
gedieh, und manch lebendiges Zeugnis wurde abgelegt gegen die
bösen, trennenden und spaltenden Geister, von denen jene Gegner
getrieben wurden, und die Versammlung endete zur Zufriedenheit
der Freunde. Jch iibernachtete mit anderen Freunden bei
Thomas Ellwood; in der gleichen Woche hatte ich noch eine Ver-
sammlung mit den Gegnern in Wickhatn, wo sie abermals ihre
Bosheit zeigten und vor den Rechtgefinnten blvßgestellt wurden ....
Hierauf besuchte ich die Freunde in Henley in Oxfordshire,
und dann gings durch Cosham nach Reading, wo ich eine große
Versammlung mit Freunden hatte. Am folgenden Tage in einer
Versammlung zur Besprechung über die Einrichtung einer Frauen-
versammlung gerieten etliche, die dem Geist der Uneinigkeit Raum
gegeben hatten, in Streit und waren eine Zeitlang widerspenstig,
bis die Wucht der Wahrheit sie bezwang. Daraus hatte ich
Versammlungen an verschiedenen Orten, und am 24. des 11.Mo-
nats, gerade zum Jahrmarkt, kam ich nach Bristol.
Jch blieb während der ganzen Zeit des Jahrmarkts da und
noch einige Zeit nachher. Wir hatten viele schöne Versammlungen.
Aus allen Gegenden des Landes waren viele Freunde da, teils
in Geschäften, teils um Sachen der Wahrheit willen. Groß war
die Liebe und Einigkeit unter denjenigen Freunden, die der
Wahrheit treu blieben. Jedoch etliche, die von der heiligen
Einigkeit abgewichen waren und in Streit, Uneinigkeit und Feind-
seligkeit geraten, waren grob und beleidigend und benahmen sich
unchristlich gegen mich. Aber die Krast des Herm war über
allen; weil sie mich in der himmlischen Geduld erhielt, welche


% \picinclude{./260-269/p_s269.jpg} 
Rückkehr nach England. Kampf der Ordnungspartei usw. 269
kann Schmähungen um seines Namens willen ertragen, so fühlte
ich mich Herr über die groben und ungeregelten Widerspenftigen
und überließ sie dem Herrn, der meine Unschuld kannte und sich
meiner Sache annehmen würde. Je eisriger diese waren, um
mich zu schmähen und zu erniedrigen, desto mehr Liebe strömte
mir von den ausrichtigen, wahren, ehrlichen Freunden entgegen,
und etliche, die von den Gegnern verführt worden waren, trennten
sich von ihnen, als sie ihre Schlechtigkeit und Bosheit und ihr
grobes Benehmen sahen; sie haben alle Ursache, Gott für ihre
Errettung zu preisen .....
Am 8. des 3. Monats 1678 kam ich nach London; das Parla-
ment tagte gerade, und Freunde, die eine Klage über ihre Leiden ein-
gereicht hatten, warteten nun auf die Erklärung, daß das Gesetz
gegen päpstliche Rekusanten uns nicht treffe. Man wußte zwar
wohl, daß wir nichts mit diesen zutun hatten; aber dennoch
hatten einige böswillige Behörden davon gegen uns Gebrauch
gemacht, um uns in verschiedenen Gegenden zu verfolgen. Ich
schloß mich nun den Freunden, die sich in dieser Sache bemühten,
an, und es war Aussicht vorhanden, etwas zur Erleichterung der
Freunde aus diesem Wege zu erreichen, weil viele der Parlaments-
mitglieder den Freunden geneigt und wohlgesinnt waren und
einsahen, daß uns unsere Gegner oft falsch darstellten. Als ich
aber eines Morgens mit George Whitehead zum Parlamentsgebäude
kam, war das Parlament vertagt .....
Etwa zwei Wochen nach meiner Ankunft in London fand
die Jahres?-versammlung statt .... worüber ich meiner Frau
bald darauf in einem Brief berichtete: L
»Liebes Herz,
Dir meine Liebe in dem ewigen Samen des Lebens, welcher
alles regieret. Große Versammlungen sind hier gewesen, und die
Kraft des Herrn hat alle gepackt wie noch nie. Der Herr hat
durch seine Kraft die Freunde herrlich untereinander verbunden,
und seine glorreiche Gegenwart erschien unter ihnen. Und jetzt,
da die Versammlungen vorüber sind, lobe man den Herm in
Ruhe und Frieden. Aus Holland oernehme ich, daß dort alles
gut geht. Es sind einige Freunde hingegangen, um der Jahres-
versammlung in Amsterdam beizuwohnen. In Emden sind
Freunde, die verbannt gewesen waren, wieder in die Stadt
zurückgekehrt. Jn Danzig waren Freunde im Gefängnis und die

