% \picinclude{./260-269/p_s260.jpg} 
Ich hörte später, das seine Hörer ihn wegen dessen, was er
in unsrer Versammlung gesagt, zu Rede stellten, und als er dazu
stand, ihn bei den andern Priestern der Stadt verklagten, die ihn
darüber zur Rechenschaft zogen, aber vom Ausgang der Sache
konnte ich nichts erfahren [...].

Am folgenden Tage gingen wir nach Amsterdam\ort{Amsterdam}, wo wir
etwas nach Mitternacht ankamen, und da die Tore geschlossen
waren, so blieben wir bis zum folgenden Tage auf dem Schiff
und gingen dann ins Haus von Gertrud Ditick\person{Ditick, Gertrud}; hier besuchten
uns viele Freunde, froh, das wir wohlbehalten wieder zurück
waren. Am folgenden Tage fühlte ich mich im Geist beunruhigt
wegen etlicher verführerischer Geister, die Uneinigkeit\index{Uneinigkeit} unter die
Freunde brachten, und weil ich merkte, das sie sich suchten in Gunst
zu bringen, so trieb es mich, einige Zeilen deswegen an die
Freunde zu schreiben: 

\grosszitat{
  Alle, die sich in die Gunst der 
  Leute einschmeicheln wollen, trachten, sich in Gunst zu 
  bringen statt Christus.
  Aber Freunde, euer friedsames Bleiben in der Wahrheit, die ewig
  ist und sich nicht verändert, wird alles, was nicht aus der 
  Wahrheit ist, überdauern, auch wenn es mit noch so viel Worten 
  austritt. Lasset denn die, welche so für J. S. und J. W. auftreten
  zu ihnen halten und sich von euch trennen, und ihr, die ihr 
  Zeugnis abgelegt gegen diesen Geist, beharret bei diesem Zeugnis, bis
  sie euch mit Anschuldigungen angreifen. Zanket nicht, lasset euch
  nicht ein mit etwas, das nicht in der Wahrheit steht, noch suchet
  lebendig zu erhalten, was sollte Gott zum Opfer gebracht werden,
  damit ihr nicht des Reiches verlustig geht.

  \begin{flushright}Amsterdam\ort{Amsterdam}, 14. des 7. Monats 
  1677\index{Jahr!1677}. G. F.\end{flushright}

}

An einem grosen Fasttage wohnte ich einer der Versammlungen 
der Freunde bei. Ich hatte eigentlich vorgehabt, nach
Haarlem\ort{Haarlem} zu gehen, aber ich wurde in meinem Geist gehalten, zu
bleiben. Wir hatten eine sehr große Versammlung, eine große
Menge Leute strömte herbei, worunter viele angesehene Personen.
Die Kraft des Herrn war über der Versammlung, und in den
Offenbarungen\index{Offenbarungen}, die ich während derselben 
hatte, trieb es mich,
darzutun, das niemand, mit allem Studieren und allem Verstand
oder mit dem Lesen der Geschichte, wenn er sie nach seinem eigenen
Willen lese, die Abstammung von Christus wisse, der nicht nach
dem Willen eines Menschen, sondern nach dem Willen Gottes
gezeuget sei. [...] Nachdem ich ihnen das ausführlich erklärt hatte,
% \picinclude{./260-269/p_s261.jpg} 
erklärte ich ihnen den Unterschied zwischen wahrem und falschem
Fasten\index{Fasten}. Ich zeigte ihnen, wie alle, ob sie sich 
nun Christen, Juden\index{Juden} oder Türken\index{Türken} nennen, 
nicht in der rechten Weise fasten, sondern sie
fasten, \zitat{das sie hadern und zanken und mit gottloser Faust schlagen}
(Jes. 58,4)\bibel{Jes. 58:04@Jes. 58:4}, sie erheben nicht 
reine Hände zu Gott. Und wenn sie
schon vor den Leuten tun, als ob sie fasteten, und \zitat{des Tags den
Kopf hängen wie ein Schilf, so ist es doch nicht das Fasten, das
Gott erwählt}(Jes. 58,5\bibel{Jes. 58:05@Jes. 58:5}). Darum sind 
ihre Gebeine vertrocknet,
und wenn sie den Herrn anrufen, so hört er sie nicht und \zitat{ihre
Besserung wächst nicht}(Jes. 58,8\bibel{Jes. 58:08@Jes. 58:8}), 
weil sie ihr eigenes Fasten
halten und nicht das des Herrn. Ich ermahnte sie, das Fasten
des Herrn zu halten, welches ein Fasten von der Ungerechtigkeit
und der Sünde sei, vom Streiten, Hadern und Unterdrücken, und
auch allen bösen Schein zu meiden. Die Leute, die Fasttage
hielten, wunderten sich sehr über diese Eröffnungen, und die 
Versammlung nahm ein schönes friedliches Ende.

Am folgenden Tage ging ich nach Haarlem\ort{Haarlem}, wo ich zuvor eine
Versammlung angesagt hatte. Peter Hendricke\person{Hendricke, Peter} 
und Gertrud Dirick Nieson\person{Nieson, Gertrud Dirick} 
gingen mit mir und wir hatten eine gesegnete Versammlung. 
Es waren verschiedene \zitat{Fromme} dabei, auch ein
Priester der Lutheraner\index{Lutheraner}, der mehrere 
Stunden andächtig zuhörte, \index{Andachtsdauer}
während ich ihnen die Wahrheit verkündete, Gertrud verdolmetschte. 
Als die Versammlung zu Ende war, sagte der Priester,
er habe nichts darin gehört, das nicht nach dem Worte Gottes
gewesen wäre, und er wünschte uns, das der Segen Gottes mit
uns und unsren Versammlungen sein möge. Auch andere erklärten,
man habe ihnen noch nie zuvor die Dinge so verständlich 
auseinandergesetzt.

Wir brachten die Nacht im Hause eines Freundes, Dirirk
Klassen\person{Klassen, Dirirk}, zu, am folgenden Tage 
kehrten wir nach Amsterdam zu
Gertrud Dirick zurück; wir waren noch nicht lange da, als ein
berühmter Priester kam, der früher unter dem deutschen Kaiser
gestanden, mit einem anderen deutschen Priester, um mit mir zu
reden. Ich ergriff die Gelegenheit, um ihnen den Weg der
Wahrheit zu erklären, indem ich ihnen zeigte, wie sie dazu kommen
können, Gott und Christus und sein Evangelium und Gesetz zu
kennen; ich zeigte ihnen, das sie niemals durch Studieren\index{Studieren} und
durch Philosophie\index{Philosophie} dazu kommen können, sondern durch göttliche
Offenbarung,\index{Offenbarung} durch den Geist Gottes, der ihnen in der Stille des
% \picinclude{./260-269/p_s262.jpg} 
Herzens kund werde. Die Beiden waren empfänglich und gingen
befriedigt fort.

Am folgenden Ersten Tage war ich in einer Versammlung
der Freunde in Amsterdam; außer vielen Verschiedenen \zitat{Frommen}
war auch ein Doktor aus Polen anwesend, der um seiner Religion
willen aus seiner Heimat verbannt war; während der 
Versammlung wurde er ergriffen vom Zeugnis der Wahrheit und kam
nachher zu mir und wünschte eine Unterredung mit mir, und
nachdem wir eine Zeitlang miteinander geredet, und ich ihm die
Dinge noch mehr erklärt hatte, ging er sehr empfänglich und in
Liebe zur Wahrheit fort.

Während ich in Amsterdam war, brachte ich viel Zeit mit
Schreiben für die Wahrheit zu. Ich schrieb von hier mehrere Briefe
an die Freunde in England, ebenso: \zitat{Eine Warnung an die
Bewohner der Stadt Oldenburg}\index{Buchtitel!Eine Warnung an die
Bewohner der Stadt Oldenburg}\ort{Oldenburg}, die kürzlich abgebrannt war,
ferner: \zitat{Eine Warnung an die Bewohner der Stadt Hamburg}
\index{Buchtitel!Eine Warnung an die Bewohner der Stadt Hamburg}\ort{Hamburg} [...]
Ferner schrieb ich einen Brief an die Gesandten, die zu 
Nymwegen über den Frieden verhandelten [...].

Ich schrieb auch an die Behörden und Priester von Emden\ort{Emden},
um ihnen zu zeigen, wie unchristlich es sei, die Freunde zu 
verfolgen. Mehrere andere Bücher schrieb ich, Antworten an Priester
und andere, in Hamburg\ort{Hamburg}, Danzig\ort{Danzig} und 
anderwärts, um die Freunde
und die Wahrheit von allen Beschuldigungen und Verleumdungen
zu reinigen [...]. Ferner 

\grosszitat{Ein Brief über das wahre Fasten,
das wahre Beten, und die wahre Ehre, gegen die Verfolgungen
und für die wahre Freiheit in Christus Jesus, damit ihr in eurem
Halten von Tagen, Monaten, Zeiten und Festen Sorge tragen
möget, das der Apostel nicht umsonst an euch gearbeitet habe,
und ihr nicht von neuem \zitat{den dürftigen Satzungen dienet}
(Gal. 4,9\bibel{Gal. 04:09@Gal. 4:9}) und sie anderen auferlegt.
Wo haben je Christus oder seine Apostel den Gläubigen oder
den Christen befohlen, Feste oder Tage zu halten? Zeiget uns,
wo in den Schriften des Neuen Testaments, in den vier 
Evangelien, in den Briefen oder in der Offenbarung geschrieben steht,
das Christus oder die Apostel je befahlen, die Zeit, die man
Christfest(Weihnachten)\index{Weihnachten} nennt, zu feiern, oder den Tag von der 
Geburt Christi, oder die Zeit, die man Ostern\index{Ostern} 
nennt, oder den weisen Sonntag,
oder Petrus, Paulus, Lukas oder Markus, oder irgend eines
andern Heiligen Tag? [...]
}
% \picinclude{./260-269/p_s263.jpg} 

Es war des Apostels Arbeit, sie aus den Banden dieser
Satzungen zu befreien. Und als sie sich dann dem Halten der
Tage wieder zuwandten, fürchtete er, er habe umsonst an ihnen
gearbeitet; und er ermahnte sie, Gal. 5, 1\bibel{Gal. 05:01@Gal. 5:1}: 
\zitat{ So bestehet nun
in der Freiheit, damit euch Christus befreit hat, und fallet
nicht wieder in das knechtische Joch der Sünde.} Hiermit sagt
er, das sie einst in dem knechtischen Joch gefangen waren. Aber
ach, wie sehr die sogenannten Christen seit den Tagen der Apostel
wiederum in dieses Joch gekommen sind, indem sie wieder Fasten
und Tage hielten, das sieht man an ihrem Tun. Ja, zwingen
nicht sogar sowohl die Papisten\index{Papisten} wie die 
Protestanten\index{Protestanten} die Leute,
Tage, Monden und Jahreszeiten zu halten? [...]. Es war und
ist Christus, der die Menschen von diesen dürftigen Satzungen frei
macht, darum sollen die Erlösten fest stehen in der Freiheit, womit
Christus sie befreit hat [...]. Die so in diesen Satzungen stehen, und
andere dazu zwingen wollen, sind abgewichen von der Erkenntnis
Gottes und stehen nicht fest in der Freiheit\index{Freiheit}, mit der Christus befreiet.

Was das Beten anbelangt, so sehen wir nirgends, das Christus
oder die Apostel je jemanden zwangen\index{Zwang im Glauben}, mit ihnen zu beten oder
zu fasten. Sondern Christus zeigte, wie man beten solle und sich
von den Heuchlern unterscheiden. Seine Worte sind: \zitat{Wenn du
betest, so sollst du nicht sein wie die Heuchler, die da gerne stehen
in den Schulen und an den Ecken auf den Gassen, auf das sie 
von den Leuten gesehen werden [...]. Wenn du betest, so gehe
in dein Kämmerlein und schließe die Türe zu und bete zu deinem
Vater im Verborgenen, und dein Vater, der ins Verborgene sieht,
wird dirs vergelten öffentlich} 
(Matth. 6, 5\bibel{Matth. 06:05@Matth. 6:5}) .[...]. Und wir
tun nun, wie es die Apostel und Heiligen getan. Wir beten im
Verborgenen und öffentlich, je nachdem der Geist es uns eingibt, 
welcher unsrer Schwachheit hilft, wie er den Aposteln und
allen wahren Christen half; so beten wir für uns und für alle
Menschen, hoch und niedrig. [...] über das Fasten sagt
Christus: \zitat{Wenn ihr fastet, so sollt ihr nicht sauer sehen wie die
Heuchler, sie verstellen ihr Angesicht, auf das sie vor den Leuten
scheinen mit ihrem Fasten. Wenn du fastest, so salbe dein Haupt
und wasche dein Angesicht, und dein Vater, der ins Verborgene
sieht, wird dirs vergelten öffentlich.} (Matth. 6\bibel{Matth. 06@Matth. 6}). 
In Jesaia 58\bibel{Jesaia 58@Jesaia 58} heißt es: 

  \grosszitat{Rufe laut 
  und schone nicht, erhebe deine Stimme wie
  eine Posaune und verkündige meinem Volk ihr Übertreten und
  % \picinclude{./260-269/p_s264.jpg}
  dem Hause Jakobs ihre Sünde; sie suchen mich täglich und wollen
  meine Wege wissen, als ein Volk, das Gerechtigkeit schon getan,
  und das Recht ihres Gottes nicht verlassen hätte; sie fordern
  mich zum Recht und wollen mit ihrem Gott rechten. Warum
  fasten wir und du siehest es nicht an? warum tun wir unserm
  Leib wehe und du willst es nicht wissen? [...] Siehe, ihr fastet,
  das ihr hadert und zanket und schlaget mit der Faust ungöttlich.
  Fastet nicht also, wie ihr jetzt tut, das ein Geschrei von euch
  in der Höhe gehört wird. Sollte das ein Fasten sein, das ich
  erwählen soll, [...] wollt ihr das ein Fasten nennen und einen
  Tag dem Herrn angenehm? Das ist ein Fasten, das ich erwähle:
  \zitat{Las los, welche du mit Unrecht gebunden hast; las ledig, welche
  du beschwerest; gib frei, welchs du drängest; reis weg allerlei
  Last.} [...] 

  Das Fasten also, das der Herr verlangt, ist nicht,
  das man Lasten auferlege und die Banden der Sünde noch 
  befestige, sondern solche Bande zu lösen und zu sprengen.

  Und nun darüber, das wir den Hut nicht abnehmen\index{Hutabnehmen} vor den
  Leuten. Viele, die sich Christen nennen, haben Anstoß an uns 
  genommen, weil wir den Hut nicht abnahmen und uns nicht vor
  ihnen Verneigten. Wir finden nirgends, das Christus das geboten
  hat, sondern eher das Gegenteil. Christus sagt: \zitat{nehmet nicht
  Ehre von den Menschen;} und ferner sagt Christus: \zitat{wie könnet
  ihr glauben, die ihr Ehre von einander nehmet, und die Ehre
  die von Gott kommt, sucht ihr nicht} (Joh. 5)\bibel{Joh. 5}. Christus nennt
  es ein Kennzeichen der Ungläubigen, Ehre voneinander zu nehmen
  und die Ehre, die von Gott kommt, nicht zu suchen, und ist denn
  nicht das Abnehmen des Hutes und das Verneigen eine Ehre, die
  sich die Menschen untereinander erzeigen, nach welcher sie trachten
  und beleidigt sind, wenn sie ihnen nicht erzeigt wird? Haben sie
  nicht sogar etliche gebüßt, verfolgt und gefangen genommen, weil
  sie den Hut nicht abnahmen? Ja, verhöhnen nicht die Türken
  die Christen in ihrem Sprichwort, welches sagt, die Christen
  bringen einen großen Teil ihrer Zeit damit zu, ihre Hüte 
  abzunehmen und einander ihre kahlen Köpfe zu zeigen? Sollten nun
  die, welche den edlen Namen Christen tragen dürfen, nicht über
  den Türken stehen und über dem Trachten nach Menschenehre
  und dem Verfolgen solcher, die ihnen diese Ehre nicht erweisen
  wollen, wie überhaupt alle wahren, gläubigen Christen allein die
  Ehre suchen sollten, die von Gott kommt? Es heißt: \zitat{Wer an
  % \picinclude{./260-269/p_s265.jpg} 
  den Sohn Gottes glaubt, der hat das ewige Leben; wer aber
  nicht glaubt, der wird verdammt werden} (Joh. 3, 36\bibel{Joh. 3:36@Joh. 03:36}). 
  Ist nicht die Redensart der Türken\index{Türken}, das die Christen so viel Zeit
  darauf verwenden, ihre Hüte abzuziehen und einander ihre kahlen
  Köpfe zu zeigen, ein Vorwurf für die Christen? Habt ihr nicht
  viele gefangen genommen und bestraft, weil sie den Hut nicht vor
  euch abnehmen wollten und euch ihre kahlen Köpfe zeigen? Ja,
  in vielen eurer Städte und Staaten haben solche, die ihre Hüte
  nicht abnehmen und ihre kahlen Köpfe nicht zeigen, weder Freiheit
  noch Recht, obgleich sie treue Untertanen sind. Habt ihr nicht
  ein Gesetz gegen sie erlassen, das sie zwei Gulden bezahlen müssen,
  wenn sie es nicht tun? Und trachtet ihr nicht, sie dazu zu zwingen
  und bestraft sie, wenn sie es nicht tun, wie in Landsmeer in
  Vaterland? Ist denn das nicht trachten nach Menschenlehre?
  Taten nicht die Pharisäer und Juden also? [...]

  Ihr habet keinerlei Befehl von Christus oder einem seiner
  Apostel, irgend jemanden zu verfolgen\index{Verfolgung}, 
  zu bestrafen oder gefangen
  zu nehmen um seiner Religion willen.

  \begin{flushright} Harlingen\ort{Harlingen} in 
  Friesland\ort{Friesland}, 11.des 6.Monats 
  1677\index{Jahr!1677}. G. F.\end{flushright}
}

Bald darauf kamen William Penn\person{Penn, William} 
und George Keith\person{Keith, George} von
Deutschland nach Amsterdam\ort{Amsterdam} zurück 
und hatten einen Disput\index{Disput} mit
Galenus Abrahams\person{Abrahams, Galenus}, einem 
der bekanntesten Baptisten\index{Baptisten} in Holland.
Viele \zitat{Fromme} waren zugegen; da sie nicht Zeit hatten, den
Disput zu beendigen, kamen sie am folgenden Tage noch einmal
zusammen, und da wurde der Baptist gänzlich geschlagen, und die
Wahrheit gewann Boden [...]. Als wir nun unsern Dienst in
Amsterdam getan, gingen wir, zu Wagen, nach Leyden\ort{Leyden}. Wir
kamen dort mit einem Deutschen zusammen, der teilweise bekehrt
wurde. Er sagte uns von einem hervorragenden Mann, der die
Wahrheit suche. Etliche fanden ihn auf und besuchten ihn und
sahen einen ernst gesinnten Mann in ihm. Ich redete auch mit
ihm, und er bekannte sich zur Wahrheit. William Penn und Benjamin
Furly\person{Furly, Benjamin} besuchten noch einen andern 
angesehenen Mann, der ein wenig
auserhalb von Leyden wohnte, von dem es hieß, er sei General
beim König von Dänemark gewesen. Er und seine Frau waren
sehr liebevoll mit ihnen und nahmen die Wahrheit mit Freuden auf.

Von Leyden gingen wir nach dem Haag\ort{Haag}, wo der Prinz von
Oranien\person{Prinz von Oranien} seinen Hof hielt, und 
wir besuchten einen von der Regierung
von Holland, mit dem wir eine ziemlich lange Unterredung hatten.
% \picinclude{./260-269/p_s266.jpg} 

[...] Von da gingen wir über Delft nach Rotterdam\ort{Rotterdam}, wo wir
einige Tage blieben und mehrere Versammlungen hatten. Hier
verfasste ich auch ein Buch an die Juden\index{Juden}, mit denen ich gerne,
als ich in Amsterdam war, mich unterredet hätte, aber sie wollten
nicht. Ich erhielt hier auch einige andere Bücher und Schriften,
die ich früher herausgegeben und die nun übersetzt waren.

\chapter[Rückkehr nach England]{Rückkehr nach England}

\begin{center}
\textbf{Rückkehr nach England. Kampf der Ordnungspartei gegen die
unbotmäßigen Quäker. Briefe über Toleranz an den König von
Polen, den Großmogul und andere.}
\end{center}


Da wir spürten, das wir unser Werk in Holland getan
hatten, nahmen wir Abschied von den Freunden in Rotterdam [...].
Am 21. des 8. Monate reisten wir nach England ab, William
Penn\person{Penn, William}, George 
Keith\person{Keith, George} und ich und Gertrud Dirick 
Nieson\person{Nieson, Gertrud Dirick} mit ihren
Kindern. Wir hatten eine lange und gefahrvolle Überfahrt [...].
aber der Herr, der den Winden gebieten kann und die stürmischen
Wellen des Meeres stille macht, das sie auf und nieder gehen,
wie es ihm gefällt, er behütete uns [...]. Am Abend des 23.
kamen wir in Harwich an. Am nächsten Morgen gingen William
Penn und George Keith mit mir nach Colchester\ort{Colchester} [...]. Dort blieben
wir bis zum Ersten Tag, da es mich verlangte, der Versammlung
der Freunde beizuwohnen. Es war eine überaus zahlreiche und
wirksame, denn als die Freunde von meiner Rückkehr hörten,
strömten sie von allen Seiten herbei vom Lande und auch aus
der Stadt, so das etwa tausend Menschen anwesend waren [...]\index{Versammlung!große}.
Am 9. des 9. Monats kam ich nach London\ort{London}, wo ich mit großer
Freude empfangen wurde.

Als ich einige Zeit in London war, schrieb ich folgenden Brief
an meine Frau\person{Fell, Margaret}:

\grosszitat{
  Liebes Herz,

  \bigskip 

  Dir und den Kindern meine Liebe und allen andern Freunden
  in der Wahrheit, der Kraft und dem Samen des Herrn, der über
  allem ist. Dem Herrn sei Ehre und sein Name sei immerdar
  hochgelobt! Er hat mich durch allerlei Trübsal und Gefahr 
  hindurch geführt, in seiner ewigen Kraft; ich bin zweimal in der 
  Versammlung in Gracechurch Street\ort{Gracechurch Street}\footnote{Eine 
  Straße in London, in der die Quaker bis 1821 ein Meeting-Haus, 
  bis es abbrannte. Siehe her zu 
  \url{http://en.wikipedia.org/wiki/Gracechurch_Street}} 
  gewesen, und obgleich auch feindliche
  Geister zugegen waren, war doch alles ruhig; der Tau des Himmels
  fiel auf die Anwesenden, und die Herrlichkeit des Herrn schien
  % \picinclude{./260-269/p_s267.jpg} 
  über allen. Ich muss wohl oder übel täglich zu
  Versammlungen\index{Versammlung!unfreiwillige}
  gehen, in geschäftlichen Angelegenheiten und wegen allerlei 
  Drangsal, deren es viele gibt rings umher, und viele Freunde haben
  gegenwärtig darunter zu leiden, darum in Eile euch alle grüsend.

  \bigskip 

  London, 24. des 9. Monats 1677\index{Jahr!1677}. G. F.
}

Um diese Zeit erhielt ich Briefe aus Neu-England\ort{Neu-England}, welche
berichteten, wie die Behörden grausam und unchristlich gegen die
dortigen Freunde verfuhren, indem sie sie abscheulich misshandelten
und peitschten; sie peitschten viele Frauen unter den Freunden.
Eine Frau banden sie an einen Karren und schleppten sie 
halb entblößt durch die Straßen. Sie peitschten einige Schiffskapitäne,
die selber keine Freunde waren, nur weil sie Freunde hergebracht
hatten. Währenddem sie aber in dieser barbarischen Weise die
Freunde verfolgten, schlugen die Indianer\index{Indianer} sechzig ihrer Leute,
nahmen einen der Führer gefangen und zogen ihm bei lebendigen
Leib die Haut vom Kopf und trugen sie im Triumph davon.
Manche einsichtige Leute sagten: \zitat{Gottes 
Gericht\index{Gottes Gericht} ist über sie 
gekommen, weil sie die Quäker verfolgten.} Aber die verblendeten,
verfinsterten Priester sagten, es sei, weil sie sie nicht genug 
verfolgt hätten. Ich hatte große Mühe, für die fernen leidenden
Freunde Erleichterung zu schaffen, damit sie nicht unter die Rute
der Bösewichter kämen [...].

Ich blieb etwa einen Monat in London; darauf ging ich
nach Buckinghamshire\ort{Buckinghamshire} und besuchte die dortigen Freunde und
hatte mehrere Versammlungen. Ofters machten während derselben,
solche, die von der wahren Einigkeit der Freunde in der Wahrheit
abgewichen und in Zank, Zwiespalt und Auflehnung geraten\index{Streit unter Quäkern}
waren, große Störungen, besonders während der Männerversammlungen 
bei Thomas Ellwoods\person{Ellwood, Thomas} in Hunger Hill; ihr Anführer
kam von Wickham\ort{Wickham} und versuchte die Freunde zu stören und
an der weiteren Abhaltung der Versammlung zu hindern. Als
ich ihr Vorhaben merkte, ermahnte ich sie, ruhig und vernünftig
zu sein und die Versammlung nicht durch Unterbrechungen zu
stören; sondern, wenn sie mit dem Vorgehen der Freunde nicht
einverstanden seien und etwas dagegen einzuwenden hätten, dafür
eine Versammlung auf einen andern Tag zu veranstalten. Die\index{Konfliktbeseitibung}
Freunde boten ihnen an, an einem folgenden Tag eine Versammlung 
für sie abzuhalten, und schließlich wurde eine solche für die
darauf folgende Woche bei Thomas Ellwood festgesetzt. Die
% \picinclude{./260-269/p_s268.jpg} 
Freunde trafen sie dort, und die Versammlung fand in der
Scheune statt, weil so viele gekommen waren, das das Haus sie
nicht fassen konnte. Nachdem wir eine Zeitlang dagesessen hatten,
fingen sie an mit ihren Zänkereien. Die meisten ihrer Pfeile
waren gegen mich gerichtet;\index{Persönliche Angriffe} aber der Herr war mit mir und
stärkte mich, das ich in seiner Kraft die Pfeile der Bosheit und
Falschheit gegen sie selber zurück schleudern konnte. Ihre 
Entgegnungen wurden widerlegt, und manches wurde den Leuten
geoffenbart, und die Wahrheit wurde gefördert; viele, die zuvor
schwach gewesen, wurden gestärkt und gefestigt; etliche, die 
geschwankt und gezweifelt, wurden überzeugt und befestigt, und die
gläubigen Freunde wurden erquickt und ermuntert im Wachstum
des Lebens. Denn die Kraft wuchs unter uns, und das Leben
gedieh, und manch lebendiges Zeugnis wurde abgelegt gegen die
bösen, trennenden und spaltenden Geister, von denen jene Gegner
getrieben wurden, und die Versammlung endete zur Zufriedenheit
der Freunde. Ich übernachtete mit anderen Freunden bei
Thomas Ellwood; in der gleichen Woche hatte ich noch eine 
Versammlung mit den Gegnern in Wickham, wo sie abermals ihre
Bosheit zeigten und vor den Rechtgesinnten bloßgestellt\index{Bloßstellung} wurden [...].

Hierauf besuchte ich die Freunde in Henley in Oxfordshire\ort{Oxfordshire},
und dann gings durch Cosham nach Reading, wo ich eine große
Versammlung mit Freunden hatte. Am folgenden Tage in einer
Versammlung zur Besprechung über die Einrichtung einer 
Frauenversammlung\index{Frauenversammlung} gerieten etliche, 
die dem Geist der Uneinigkeit Raum
gegeben hatten, in Streit und waren eine Zeitlang widerspenstig,
bis die Wucht der Wahrheit sie bezwang. Darauf hatte ich
Versammlungen an verschiedenen Orten, und am 24. des 
11.~Monats, gerade zum Jahrmarkt, kam ich nach Bristol\ort{Bristol}.

Ich blieb während der ganzen Zeit des Jahrmarkts da und
noch einige Zeit nachher. Wir hatten viele schöne Versammlungen.
Aus allen Gegenden des Landes waren viele Freunde da, teils
in Geschäften, teils um Sachen der Wahrheit willen. Groß war
die Liebe und Einigkeit unter denjenigen Freunden, die der
Wahrheit treu blieben. Jedoch etliche, die von der heiligen
Einigkeit abgewichen waren und in Streit, Uneinigkeit und 
Feindseligkeit geraten, waren grob und beleidigend und benahmen sich
unchristlich gegen mich.\index{Streit unter Quakern}
\index{Quaker greifen Fox an} Aber die Kraft des Herrn war über
allen; weil sie mich in der himmlischen Geduld erhielt, welche
% \picinclude{./260-269/p_s269.jpg} 
kann Schmähungen um seines Namens willen ertragen, so fühlte
ich mich Herr über die groben und ungeregelten Widerspenstigen
und überließ sie dem Herrn, der meine Unschuld kannte und sich
meiner Sache annehmen würde. Je eisiger diese waren, um
mich zu schmähen und zu erniedrigen, desto mehr Liebe strömte
mir von den aufrichtigen, wahren, ehrlichen Freunden entgegen,
und etliche, die von den Gegnern verführt worden waren, trennten
sich von ihnen, als sie ihre Schlechtigkeit und Bosheit und ihr
grobes Benehmen sahen; sie haben alle Ursache, Gott für ihre
Errettung zu preisen [...].

Am 8. des 3. Monats 1678\index{Jahr!1678} kam ich nach London\ort{London}; 
das Parlament tagte gerade, und Freunde, die eine Klage über ihre Leiden 
eingereicht hatten, warteten nun auf die Erklärung, das das Gesetz
gegen päpstliche Rekusanten uns nicht treffe. Man wusste zwar
wohl, das wir nichts mit diesen zu tun hatten; aber dennoch
hatten einige böswillige Behörden davon gegen uns Gebrauch
gemacht, um uns in verschiedenen Gegenden zu verfolgen. Ich
schloss mich nun den Freunden, die sich in dieser Sache bemühten,
an, und es war Aussicht vorhanden, etwas zur Erleichterung der
Freunde aus diesem Wege zu erreichen, weil viele der 
Parlamentsmitglieder den Freunden geneigt und wohlgesinnt waren und
einsahen, das uns unsere Gegner oft falsch darstellten. Als ich
aber eines Morgens mit George Whitehead\person{Whitehead, George} 
zum Parlamentsgebäude kam, war das Parlament vertagt [...].

Etwa zwei Wochen nach meiner Ankunft in London fand
die Jahresversammlung\index{Jahresversammlung} statt [...] 
worüber ich meiner Frau
bald darauf in einem Brief berichtete: 

\grosszitat{

  Liebes Herz,

  \bigskip 

  Dir meine Liebe in dem ewigen Samen des Lebens, welcher
  alles regieret. Große Versammlungen sind hier gewesen, und die
  Kraft des Herrn hat alle gepackt wie noch nie. Der Herr hat
  durch seine Kraft die Freunde herrlich untereinander verbunden,
  und seine glorreiche Gegenwart erschien unter ihnen. Und jetzt,
  da die Versammlungen vorüber sind, lobe man den Herrn in
  Ruhe und Frieden. Aus Holland\ort{Holland} vernehme ich, das dort alles
  gut geht. Es sind einige Freunde hingegangen, um der 
  Jahresversammlung in Amsterdam\ort{Amsterdam} beizuwohnen. 
  In Emden\ort{Emden} sind
  Freunde, die verbannt gewesen waren, wieder in die Stadt
  zurückgekehrt. In Danzig\ort{Danzig} waren Freunde im Gefängnis und die
  % \picinclude{./270-279/p_s270.jpg} 
  Behörden drohten ihnen mit noch härterer Gefangenschaft. Aber
  am darauf folgenden Tage machten die Lutheraner\index{Lutheraner} einen Aufstand
  und zerstörten das papistische Kloster, und so haben sie nun
  genug mit sich selber zu tun. Der König von Polen\person{König von Polen} hat meinen
  Brief erhalten und selbst gelesen, und Freunde haben ihn seither
  Hochdeutsch\index{Hochdeutsch} gedruckt. Durch Briefe von der halbjährlichen 
  Versammlung in Irland\ort{Irland} höre ich, das sie dort alle in der Liebe
  bleiben. In Barbados\ort{Barbados} haben die Freunde Ruhe, und ihre 
  Versammlungen verlaufen friedlich. Auch in Antigua\ort{Antigua} und Nevis\ort{Nevis}
  gedeiht die Wahrheit, und die Freunde haben geordnete und
  schöne Versammlungen. In Neu-England\ort{Neu-England} und an anderen Orten
  geht ebenfalls alles in Betreff der Freunde und der Wahrheit
  seinen guten Gang; an diesen Orten sind die Männer- und
  Frauen-Versammlungen geordnet, gelobt sei der Herr. So bleibet
  denn im Samen und in der Kraft Gottes, die über allem ist,
  durch welche wir Leben und Heil haben, denn der Herr regieret
  alles in seinem Reich und seiner Herrlichkeit, ihm sei Ehre 
  ewiglich, Amen. In Eile grüße ich euch alle und alle Freunde.

  \bigskip 
  \begin{flushright}
  London, 26. des 3. Monats 1678\index{Jahr!1678}. G. F.\end{flushright}

}
