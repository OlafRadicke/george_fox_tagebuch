% \picinclude{./160-169/p_s160.jpg} 
Edward Burrough sagte: \glqq So tue es eilends, denn wir können nicht
wissen, wie viele in Bälde noch hingerichtet werden.\grqq Der König
sagte: \glqq So bald ihr wollt,\glqq und befahl einem der Anwesenden:
\grqq holt den Sekretär, so will ich es sogleich tun.\glqq Als der Sekretär
kam, wurde sofort ein Erlass zugesagt. Ein paar Tage darauf
ging Edward Burrough wieder zum König, um ihn zu bitten,
den Erlass abzuschicken; der König antwortete, er habe jetzt keine
Gelegenheit ein Schiff dorthin zu schicken; wenn wir es aber tun
wollten, so stehe uns das frei, so bald wir wollten. Darauf fragte
Edward den König, ob er allenfalls auch einen sogenannter:
\grqq Quäker\grqq mit seiner Sendung betrauen würde? Der König 
antwortete: \glqq Ja, es kann gehen, wer will.\grqq Hierauf nannte 
Edward ihm Samuel Chattok, der aus Neu-England, seiner Heimat
verbannt worden war und nicht zurückkehren durfte, es sei denn
mit dieser Sendung. Dann ließ er Ralph Goldsmith kommen,
den Besitzer eines guten Schiffes, und einigte sich mit ihm auf
300 Pfund, in Waren oder bar, und Abfahrt in zehn Tagen. Er
rüstete sich alsbald, unter Segel zu gehen, und, vom Winde 
begünstigt, kam er nach etwa sechs Wochen, am Morgen eines
Ersten Tages, in Boston in Neu-England an. Es reisten viele
mit ihm, aus Alt- und Neu-England, Freunde, die der Herr
trieb, mitzugehen und aufzutreten gegen die blutigen Verfolger,
welche alle Übrigen an Grausamkeit übertrafen.

Als die Bewohner von Boston ein Schiff mit englischen
Farben in den Hafen von Boston fahren sahen, kamen sie
gleich aufs Schiff und fragten nach dem Kapitän, und Ralph
Goldsmith sagte ihnen, das er es sei. Sie fragten ihn, ob er
Briefe habe? Er sagte: \glqq ja\grqq. Sie fragten, ob er sie 
ausliefern wolle? er antwortete: \glqq nein, heute nicht.\grqq 
Darauf begaben sie sich ans Ufer und berichteten, es sei 
ein ganzes Schiff voll Quäker angekommen, und Samuel Shattock 
sei darunter, der nach den Gesetzen hingerichtet werden müsse, 
wenn er aus der Verbannung zurückkomme! denn sie wussten nichts 
von seiner Sendung. Den ganzen Tag wurden alle streng abgesperrt, 
und keiner von der Schiffsmannschaft durfte landen. Am 
folgenden Morgen begaben sich die Gesandten des Königs, 
Samuel Shattock und der Befehlshaber des Schiffs, Ralph 
Goldsmith ans Ufer, und nachdem sie die Männer, die sie ans 
Land geführt hatten, zurückgeschickt hatten,
gingen sie durch die Stadt zum Haus des Gouverneurs, John
% \picinclude{./160-169/p_s161.jpg} 
Endicott, und klopften. Der Gouverneur schickte jemand heraus,
um sie nach ihrem Begehren zu fragen. Sie ließen ihm sagen,
sie kämen vom König von England und werden ihre Botschaft
niemand übergeben, als dem Gouverneur selbst. Darauf wurden
sie vorgelassen. Der Gouverneur erschien und nachdem er ihre
Botschaft vernommen und ihren Auftrag, nahm er seinen Hut ab
und betrachtete sie. Dann verließ er sie und begab sich zum
Untergouverneur, und nach einer kurzen Unterredung mit diesem
kam er zu den Freunden zurück und sagte ihnen: \glqq Wir
werden Seiner Majestät Befehl gehorchen.\grqq Hierauf erhielten
die Reisenden die Erlaubnis zu landen, und rasch verbreitete sich
die Kunde von dem Vorgefallenen in der Stadt, und die Freunde
aus der Stadt vereinigten sich mit den Reisenden des Schiffes,
um Gott zu loben und zu danken, das er sie so wunderbar aus
den Zähnen derer, die sie umbringen wollten, befreite. Während
sie beisammen waren, kam ein Freund herein, der von ihrem
blutigen Gesetz zum Tode verurteilt worden war und lange Zeit
in Fesseln gelegen und auf seine Hinrichtung gewartet hatte. Da
wurde die Freude noch größer, und alle erhoben ihre Herzen in
inbrünstigem Loben Gottes, welcher würdig ist zu nehmem Preis,
Ruhm und Ehre; denn er allein kann frei machen und erretten
und helfen allen denen, die ihr Vertrauen auf ihn setzen [...]

Vorher, als ich noch im Gefängnis zu Lancaster war, war
ein Buch von mir (\textit{The Battledore}) veröffentlicht 
worden, das zeigen sollte, wie in allen Sprachen \glqq du\grqq 
und \glqq dich\grqq die eigentliche Anrede an eine einzelne 
Person sei und \glqq ihr\grqq nur an mehrere.
Ich hatte es an Beispielen aus der Schrift und aus Lehrbüchern
in etwa dreißig Sprachen nachgewiesen. J. Stubbs und Benjamin
Furly hatten sich auf meine Veranlassung sehr Mühe gegeben,
das Material zu sammeln, und ich fügte dann noch einiges bei.
Als es fertig war, erhielten der König und die Räte, die Bischöfe
von Canterbury und London und die beiden Universitäten eine
Abschrift und es wurde viel gekauft. Der König sagte, es sei
richtig, das diese Völker so sprechen; und als man den Bischof
von Canterbury fragte, was er davon halte, so wusste er nicht,
was er sagen sollte; denn es wirkte so überzeugend auf die Leute,
das viele daraufhin sich kaum mehr ärgerten, wenn wir \glqq du\grqq
und \glqq dir\grqq zu ihnen sagten, während man uns das vorher sehr
übel genommen hatte [...]
% \picinclude{./160-169/p_s162.jpg} 

Da die Priester und Bischöfe gerade eifrig am Werk waren,
ihre Gottesdienste einzurichten und alle zu zwingen, daran teil
zu nehmen, trieb es mich, folgendes zu schreiben, um die Art der
wahren Gottesdienste, die Christus eingesetzt hat und die Gott
annimmt, zu zeigen:

\glqq Der wahre Gottesdienst Christ geschieht im Geist und steht
allen Menschen offen. Die im Geist und in der Wahrheit anbeten, 
die sind Gott angenehm (Join). Es gibt dem Volk Odem
und den Geist denen, die auf der Erde sind (Jes. 42, 5), und er
gibt ihnen eine unsterbliche Seele; sie sind sie Tempel, in denen
er wohnen will (1. Cor. 3, 16). Die, welche äußerlich
Juden waren, mussten nach Jerusalem gehen, um anzubeten, so
lange sie dort ihren äußeren Tempel hatten; [...] nun aber
sollen alle \glqq Gott im Geist und in der Wahrheit anbeten\grqq. Dies
ist ein Gottesdienst der Freiheit, denn \glqq wo der Geist ist, da ist
Freiheit\grqq (2. Cor. 3, 17). Die Früchte des Geistes werden
offenbar werden; und man soll der Geist keine Schranken setzen,
sondern in ihm wandeln und lebens, damit man seine Früchte
hervorbringen kann. [...] Denn ,an ihren Früchten sollt ihr
sie erkennen\grqq (Matth. 7, 16) [...] 
 \begin{flushright}G. F. \end{flushright}

Viele Papisten und Jesuiten fingen damals an, den Freunden
zu schmeicheln und zu sagen, so oft sie einen von ihnen sahen,
von allen Sekten haben die Quäkeren meisten Selbstverleugnung,
und es sei schade, das sie nicht in die heilige Mutterkirche 
zurückkehrten. In dieser Weise schwatzten sie den Leuten vor und 
behaupteten, sie würden gern mit den Freunden unterhandeln; aber
die Freunde verabscheuten es, sich mit ihnen einzulassen und
hielten es für gefährlich und sogar anstößig, weil es Jesuiten
waren. Als ich aber davon hörte, sagte ich: \glqq lasst uns mit ihnen
unterhandeln, seien sie, wer sie wollen.\grqq Somit wurde die Zeit
festgesetzt, zu der zwei, die wie Höflinge aussahen, kamen; sie
fragten nach unsern Namen, die wir ihnen nannten; wir aber
fragten nicht nach ihren Namen, denn wir wussten ja, das sie
Papisten waren und wir Quäker. Ich fragte sie das selbe, was
ich schon früher einen Jesuiten gefragt hatte, nämlich, ob die
Römische Kirche nicht abgefallen sei von der Kraft, dem Geist
und den Grundsätzen der apostolischen Zeiten? [...] Als sie
sahen, das wir es genau nahmen, wichen sie aus, indem sie sagten,
es sei eine Anmaßung zu behauptet, irgend jemand habe den
% \picinclude{./160-169/p_s163.jpg} 
Geist und die Kraft, den die Apostel hatten. Aber ich sagte, es
sei eine Anmaßung von ihnen, die Worte Christi, der Apostel und
Propheten zu benützen und die Leute glauben zu machen, sie seien
Nachfolger der Apostel und Propheten, da sie doch zugeben müssen,
sie haben nicht den Geist und die Kraft der Apostel. Ich zeigte
ihnen, wie verschieden ihr Tun und ihre Früchte von denen der
Apostel seien. Darauf erwiderte mir einer von ihnen: \glqq Ihr seid
eine Gesellschaft von Träumern.\grqq \glqq Nein,\grqq erwiderte ich, 
\glqq sondern ihr seid widerwärtige Träumer, die ihr euch als die 
Nachfolger der Apostel träumt, während ihr doch zugebt, das ihr nicht ihren
Geist und ihre Kraft habt. Und ist es nicht Befleckung des Fleischer?
zu sagen, es sei Anmaßung zu behaupten, man habe den Geist
und die Kraft der Apostel? Und wenn ihr nun zugebt, das ihr
nicht den Geist und die Kraft der Apostel habt,\grqq sagte ich, \glqq 
so ist es klar, das ihr von einem anderen Geist und einer anderen
Kraft geleitet werdet als die erste Kirche und die Apostel.\grqq Ich
erklärte ihnen, das es ein böser Geist sei, der sie leite und sie zu
dem Beten mit Rosenkränzen und zu Bildern geführt habe und
zum errichten von Klöstern \index{Kloster} und zum Töten um des Glaubens
willen. Ich wies sie darauf hin, wie solches Tun gesetzlich und
nicht nach dem Evangelium der Freiheit sei. Sie waren dieser
Reden bald überdrüssig und gingen fort, und wir Vernahmen,
das sie den Papisten rieten, nicht mit uns zu disputieren, noch
von unsern Büchern zu lesen; somit waren wir sie los. Aber
wir setzten uns mit allen andern Sekten auseinander, mit den
Presbyterianern\index{Presbyterianern}, 
den Independenten\index{Independenten}, 
den Seekers\index{Independenten}, 
den Baptisten\index{Baptisten},
den Episkopalen\index{Episkopalen}, 
den Socinianern\index{Socinianern}, 
den Brownisten\index{Brownisten}, 
den Lutheranern\index{Lutheranern},
den Calvinisten\index{Calvinisten}, 
den Arminianern\index{Arminianern}, 
den Fifthmonarchyleuten\index{Fifthmonarchyleuten}, 
den Feministen\index{Feministen}, 
den Rantern\index{Rantern}. 
Von diesen allen behauptete niemand,
den gleichen Geist und die gleiche Kraft wie die Apostel zu haben.
In diesem Geist und dieser Kraft verlieh uns also der Herr den
Sieg über sie alle. Was die Fisthmonarchyleute betrifft, so trieb
es mich, eine Schrift zu schreiben, um ihren Irrtum aufzudecken.
Sie erwarteten Christi persönliche Wiederkunft\index{Christi 
(persönliche) Wiederkunft} in äußerer Form
und Weise und setzten dazu das Jahr 1666 fest, und viele, wenn
es um diese Zeit donnerte und regnete, machten sich bereit, weil
sie meinten, nun komme Christus, um sein Reich aufzurichten,
und bildeten sich ein, sie müssten nun die Hure draußen in der
Welt töten (Offb. 17\index{Bibel!Offb. 17}). Aber ich sagte ihnen, 
die Hure\index{Hure} sei lebendig
% \picinclude{./160-169/p_s164.jpg} 
in ihnen und noch nicht verzehrt vom Feuer Gottes und von
ihnen im Geist und der Kraft des Herrn vernichtet. Und ihre
Erwartungen, daß Christus äußerlich wiederkomme, um sein Reich
aufzurichten, sei wie das \glqq siehe hier, siehe da\grqq (Luc. 
17,28\index{Bibel!Luk. 17:28)} der
Pharisäer. Aber Christus sei vor mehr als 1600 Jahren gekommen, 
um sein Reich auszurichten, wie Nebukadnezar\index{Personen!Nebukadnezar} 
geträumt und Daniel\index{Personen!Daniel} prophezeit, und 
habe die vier Reiche zertrümmert und
das große Volk mit dem goldenen Kopf und den Armen und
Beinen aus Silber, alles habe der Wind Gottes weggeblasen
wie im Sommer die Spreu beim Dreschen (Dan. 2, 32\index{Bibel!Dan. 2:32)}).
Christus habe gesagt, als er auf Erden war: \glqq Mein Reich ist
nicht von dieser Welt.\grqq Wenn es von dieser Welt gewesen wäre,
so hätten seine Diener gekämpft (Joh. 18); aber es war nicht
von dieser Welt, darum kämpften sie nicht. Alle diese Fifth-
monarchyleute, die mit fleischlichen Waffen kämpfen, sind keine
Diener Christi, sondern Diener deß Tierez und der Hure; Christus
sagt: ,,Mir ist gegeben alle Gewalt im Himmel und auf E-rden«
(Matth. 28,18), und sein Reich, daß vor 1600 Jahren aufgerichtet
wurde, herrschet noch. Und der Apostel sagt: ,,Wir sehen Ehristuö
regieren, und er wird fort regieren, biz daß alle Dinge ihm unter-
tan find« (1. Cor. 15).
In diesem Jahre, 1661, trieb es viele Freunde überß Meer.,
zu gehen, um die Wahrheit in fremden Ländern zu verkünden.
John Stubbe-, Henry Fell und Richard Costrob trieb ez nach
China und Priester Johannes Gegend zu gehen; aber kein Schiff
wollte sie nehmen. Mit vieler Mühe erhielten sie eine Vollmacht
vom König; aber die Ostindische Gesellschaft fand Mittel und
Wege, sie zu umgehen, und die Schiff?-herren wollten sie nicht
nehmen. Sie begaben sich nun nach Holland, in der Hoffnung,
dort überfahren zu können; aber auch dort wollte sie niemand
nehmen. Nun nahm Henry Fell und John Stubbß ein Schiff,
daß nach Alexandrien in Ägypten ging, in der Absicht, von dort
aus sich einer Karawane anzuschließen. Doch da kam Daniel
Baker und veranlaßte Richard Costrop gegen seine innere Frei-
heit, mit ihm nach Smyrna zu gehen. Auf der Überfahrt wurde
Richard krank, da kümmerte sich Daniel Baker gar nicht um ihn
und er starb. Aber der hartherzige Mann verlor später seine Stelle.
John Stubbß und Henry Fell erreichten Alexandrien, aber
sie waren kaum dort, alö der engliche Konsul sie schon verbannte;


% \picinclude{./160-169/p_s165.jpg} 

Beginn neuer Quäkerversolgnugen bei Anlaß der Verschwötungen usw. 165
doch verbreiteten sie, ehe sie sort gingen, viele Bücher und Schriften,
um den Türken und Griechen den Weg der Wahrheit zu zeigen;
daß Buch betitelt: »Die Gewalt des Papsteß gebrochen«, gaben
sie einem alten Mönch, damit er eß dem Papst bringe oder
schicke; alß der Mönch ez durchgelesen, legte er die Hand aufs
Herz und sagte: ,,Waß hier geschrieben steht, ist Wahrheit; wenn
ich eß aber öffentlich bekennen würde, so würden sie mich oer-
brennen.« John Stubbß und Henry Fell kehrten nach Eng-
land zurück, weil eß ihnen nicht erlaubt wurde, weiter zu gehen,
und kamen wieder nach London. Stubbß hatte eine Vision, daß
die Engländer und Holländer, die sich verbündet hatten, sie nicht
überzuschisfen, sich untereinander entzweien werden, und so kam
eß auch .....
Wir hatten aber nicht nur Schwereß von außen zu erdulden,
sondern auch unter unß durch John Perrot und seine Anhänger.
Einem trügerischen Geiste nachgebend, suchte er unter den Freunden
den schlechten, unziemlichen Brauch einzuführen, daß man während
deß allgemeinen Gebetß den Hut aufbehalten solle. Viele Freunde
hatten mit ihm und seinen Anhängern darüber gesprochen, und
ich hatte einigen deßwegen geschrieben, aber er und andere taten
sich nur noch mehr gegen unß zusammen .....
Eine der Sorgen, die die Freunde von außen trafen, war,
daß man die Art, wie sie sich verheirateten, beanstandete. So
kam zum Beispiel folgender Fall vor daß Gericht von Notting-
ham: etwa zwei Jahre vorher hatten sich zwei auß der Gemein-
schaft der Freunde geehelicht; da starb der Mann und hinterließ
der Frau, die guter Hoffnung war, einen Besitz an Land und Zinß-
lehen. Alß daß Kind geboren war, erklärte eß daß Gericht alß
Erbe seineß Vater-Z, und eß wurde alß solcher anerkannt. Später
heiratete ein anderer Freund die Witwe. Daraufhin kam ein
naher Verwandter deß ersten Manneß und verklagte den Freund,
der die Witwe geheiratet, und suchte ihm seinen Besitz zu entreißen
und daß Kind seineß Grbeß zu berauben und alleß an sich zu
bringen, alß nächster Erbe deß ersten Manneß. Um dieß zu
begründen, suchte er die illegitime Geburt deß Kindeß zu beweisen
mit der Behauptung, die Ghe sei nicht nach dem Gesetz gewesen.
Bei den Verhandlungen gebrauchte der Kläger ungebührliche Auß-
drücke gegen die Freunde und sagte, sie täten sich zusammen wie
daß Vieh; und andere scheußliche Dinge. Nachdem die Anwälte


% \picinclude{./160-169/p_s166.jpg} 

166 Kapitel R17.
beider Parteien gesprochen hatten, nahm der Richter die Sache
in die Hand und sagte, eö sei eine Ehe im Paradies geschlossen
worden, ale- der Adam die E-oa und die Eva den Adam genommen
hatte; es sei eben die Zustimmung der beiden Teile, maß eine
Ehe au?-mache. Was die Quäker anbelange, so kenne er ihre
Ansichten nicht, aber er glaube nicht, daß sie sich zusammentun
wie das unvernünftige Vieh, wie man von ihnen behaupte, sondern
wie Christen, und darum glaube er, die Ehe sei gesetzlich gewesen-
und daß Kind legitimer Erbe. Um daß E-nicht zu überzeugen,
brachte er einen anderen Fall: Gin Mann, der schwach und bett-
lägerig war, hatte in diesem Zustande den Wunsch, sich zu ver-
ehelichen und erklärte vor Zeugen, daß er diese Frau zum
Weibe nehme und die Frau erklärte, daß sie diesen zum Mann
nehme; diese Ehe wurde später angefochten, aber alle Bischöfe
erklärten damalö die Ehe für gültig. Daraufhin entschied daß
Gericht auch zu Gunsten des Quäkerkindeß, gegen den Mann,
der etz um sein Erbe bringen wollte.
Um diese Zeit wurde der Suprematß- und Huldigungßeid
von den Freunden gefordert alö eine Falle, denn man wußte,
daß wir nicht schwören konnten, und ez wurden in der Folge
viele gefangen gesetzt. Bei dieser Gelegenheit veröffentlichten
die Freunde die Schrift: »Die Gründe und Ursachen, warum wir
nicht schwören« und es trieb mich, derselben einige Linien bei-
zufügen, damit man sie dem Magistrate gebe:
»Die Welt sagt: ,,Küsse das Vuch« 1); daß Buch aber sagt:
,,Küsse den Sohn, daß er nicht zürne« (Ps. 2, 12). Der Sohn sagt:
»Bleibet bei Ja und Nein in euren Reden, denn maß darüber ist,
daß ist vom Ubel« (Matth. 5, 37). Wiederum sagt die Welt: »Leget
die Hand auf daß Buch«; aber daß Buch sagt: ,,WaS unsre Hände
betastet haben vom Worte dez Leben-i-« (1. Joh. 1, 1) .... Und
Gott sagt: »Dieö ist mein lieber Sohn, den sollt ihr hören«;
er ist daß- Leben, die Wahrheit, daß Licht und der Weg zu Gott.«
G. F.
Weil so viele Freunde gefangen waren, verfaßten Richard
Hubberthorn und ich eine Schrift und ließen sie dem König überreichen,
damit er erfahre, wie wir von seinen Beamten behandelt wurden
sie lautete:
1) Aus der Formel beim Schwören des Eides.


% \picinclude{./160-169/p_s167.jpg} 
Beginn neuer Quäketversolgungen bei Anlaß der Verschwörungen usw. 167
An den König:
»Fre1md,
Der du der Herrscher dieses Reiches bist! Hier ist eine Auf-
zählung eines Teiles der Leiden, die das Volk Gottes, das man
im Ärger Quäker nennt, zu erdulden hat. Unter dem Wechsel
der Mächte, die deiner Regierung vorangingen, haben sie viel
gelitten; 3170 wurden gefangen genommen um des Gewissens
millen, und weil sie Zeugnis ablegten für die Wahrheit, die in
Christus ist; und noch jetzt sind 73 Personen im Namen des Common-
wealth gefangen; 32 Personen starben im Gefängnis während
der Zeit des Commonwealth und unter Oliver und Richard, in
harter, grausamer Gefangenschaft, aus schmutzigem Stroh und in
gräulichen Löchern. Und 3068 Personen sind seit deiner Rückkehr
gefangen genommen worden durch solche, die sich damit bei dir
einzuschmeicheln suchten. Zudem werden unsre Versammlungen
täglich gestört durch Männer mit Waffen und Kntitteln, obwohl
mir friedlich zusammenkommen, nach der Art des Volkes Gottes
der ersten Zeiten; unsre Freunde werden ins Wasser geworfen
und werden blutig geschlagen; ja, es können gar nicht alle die
Gräueltaten aufgezählt werden. Nun möchten wir gerne von dir
erbitten, daß du alle, die im Namen des Commonwealth und im
Namen der beiden Protektoren und in deinem eigenen Namen
um des Gewissens und der Wahrheit willen gefangen sind, frei
gebest; haben sie doch nie die Hand erhoben gegen dich oder
irgend sonst jemand, und daß, wenn sich die Freunde friedlich
versammeln, um Gott anzubeten, sie nicht mehr durch rohe Be-
wassnete gestört werden. Gin Hauptgrund dieser frühern Gefangen-
nahme war der, daß wir den Protektoren und den verschiedenen
Regierungen keine Eide leisten konnten; und nun tut man uns
ins,Gefängnis, weil wir den Huldigrmgseid nicht leisten können.
Z Wenn nun dir oder irgend einem Menschen gegenüber unser
ja nicht ja und unser nein nicht nein sein sollte, dann laß uns
dafür das leiden, was andere leiden müssen, wenn sie einen Eid
brechen.! Wir haben alle diese Jahre viel gelitten an unserm
eigenen Leib und an unserer Habe, unter mancherlei Regierungen,
weil wir nicht schwören, sondern Christi Gebot folgen, das sagt,
,,ihr sollt überhaupt nicht schwören«; dieses besiegeln wir mit Leib
und Gut, mit unserm ja und nein, wie Christus es befiehlt.
Bedenke das in der Weisheit, die aus Gott ist, damit du in



% \picinclude{./160-169/p_s168.jpg} 

168 Kapitel R17.
derselben solchem Tun Einhalt gebietest, du, der du die Herrschaft
hast und solches oermagst. Wir möchten, daß alle, die jetzt im
Gefängni-8 sind, frei werden und nicht wieder um der Wahrheit
und des Gewissenß willen gefangen genommen werden. Und wenn
du untersuchst, ob sie unschuldig leiden, so laß ihre Ankläger vor
dich kommen, und wir wollen, wenn nötig, au?-führlich Bericht
über ihre Leiden erstatten.« G. F. und R. H .....
Zwei Freunde, beides Frauen, waren auf Malta bei der
Jnquisition gefangen, Katharine Gvanß und Sarah Chevertz; da
es hieß, ein Lord D'Aubenh, ein römisch-katholischer Priester,
könne ihnen die Freiheit verschaffen, so ging ich zu ihm. Nach-
dem ich ihn über alletz, was ihre Gefangennahme betraf, unter-
richtet hatte, bat ich ihn, an die dortigen Behörden um ihre
Freilassung zu schreiben. Gr versprach bereitwilligst, ez zu tun
und daß, wenn ich in einem Monat wieder komme, man mir
ihre Freisprechung mitteilen wolle. Als ich zur bestimmten Zeit
wieder hinkam, sagte er, sein Brief sei scheintz nicht angekommen,
denn er habe keine Antwort erhalten, aber er versprach, nochmalß
zu schreiben, und tat es auch, und sie wurden beide frei.
Mit diesem hohen Herm redete ich viel über Religion, und
er gab zu, daß Christus jeden, der in die Welt kommt, erleuchtet
mit seinem geistigen Licht, und daß er den Tod für einen jeden
gekostet hat, und daß die heilfame Gnade Gottes allen Menschen
erschienen ist und sie lehrt und ihnen daß Heil bringt, wenn sie
ihr gehorchen. Jch fragte ihn darauf, wozu denn die Papisten
alle ihre Bilder und Reliquien brauchen, wenn sie an dieseö Licht
glauben und die Gnade, die sie lehrt und ihnen daß Heil bringt,
annehmen? Er antwortete, das seien nur Mittel, um daß
Volk in Unterwürsigkeit zu erhalten. Gr zeigte sich in dieser
Unterredung sehr weitherzig; ich hörte nie einen Papisten soviel
zugeben wie diesen .....
Jm gleichen Jahre, alö ich in Eambridgeshire war, hörte ich,
daß Edward Burrough gestorben war; und da ich wußte, wie
schwer und traurig dieser Verlust für die Freunde war, schrieb
ich folgende Zeilen zur Aufrichtung und Beruhigung ihrer Ge-
müter:
,,Freunde
Seid stille und ergeben und gefaßt im Samen Gottes, der
sich nicht ändert, damit ihr den lieben Edward Burrough unter


% \picinclude{./160-169/p_s169.jpg} 

Ein Gottesgerichi. Verhaftung wegen angeblicher Verschwörung usw. 169
euch spüren möget in diesem Samen, durch den er euch bei Gott,
bei dem er jetzt ist, vertreten wird; durch diesen Samen könnet
ihr ihn alle sehen und fühlen, denn in diesem ist Einigkeit und
Leben; freuet euch seiner im unoergänglichen Leben, das unsicht-
bar ist.« .... G. F.
Kapitel Ill
Ein Gottesgericht. Verhaftung wegen angeblicher Verschwörung
und schreckliche Gefangenschaft in Lancaster und Searlvo. Disput
im Gefängnis mit Baptisten und andern. Fox steht den Brand
von London voraus.
Wir gingen nach Tenterden und hatten dort eine Versamm-
lung, zu der viele Freunde aus der Umgegend kamen. Nach der
Versammlung ging ich ein wenig mit Thomas Briggs ins Freie,
während man unsre Pferde bereit machte. Als wir uns um-
wandten, sahen wir einen Hauptmann und einen Hausen Soldaten
mit geladenen Gewehren aus uns zukommen; einige von ihnen
hießen uns zu ihrem Hauptmann kommen. Als wir vor ihn
traten, fragte er: ,,Welcher ist George Fr-x?« Jch erwiderte:
»Jch bin es«. Da trat er auf mich zu und sagte: ,,Jch werde
dafür sorgen, daß dir nichts geschieht bei diesen Soldaten.« Dar-
auf tief er sie und hieß sie, mich festnehmen; auch Thomas
Briggs und unsern Hauswirt nahmen sie fest, aber die Kraft des
Herm war mächtig über ihnen. Nun kam der Hauptmann
wieder zu mir und sagte, ich müsse mit ihm in die Stadt;
er war ganz höflich mit mir und hieß die Soldaten mit den
andern nachkommen. Jch fragte ihn unterwegs, warum er
eigentlich solches tue, denn es war mir schon lange nichts der-
artiges mehr vorgekommen, und ich ermahnteihn, s eine Mitmenschen,
wenn sie ruhig leben, doch auch in Ruhe zu lassen. Als wir in
die Stadt kamen, brachten sie uns in eine Herberge, die zugleich
das Haus des Kerkermeisters war. Und bald darauf kam der
Bürgermeister und jener Hauptmann und einer seiner Leute, die
auch Friedensrichter waren, und fragten mich, warum ich herge-
kommen sei um Unruhe zu stiften? Jch erwiderte, ich sei nicht
gekommen, um Unruhe zu stiften und habe das auch nicht getan.
Sie sagten, es gebe aber ein Gesetz speziell gegen Quäkerver-
sammlungen. Jch antwortete, ich wisse von keinem derartigen

