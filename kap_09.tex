
% \picinclude{./100-109/p_s104.jpg} 

%%%%%%%%%%%%%%%%%%% Kapitel 9. %%%%%%%%%%%%%%%%%%%%%%%%%%%%%%

\chapter[Angriffe der Independenten und Presbyterianer.]{Angriffe der Independenten und Presbyterianer.}

\begin{center}
\textbf{Angriffe der Independenten und Presbyterianer. Ahnungen,
Heilungen, Bekehrungen. Dispute über Taufe und Erwählung.
Gefangennahme auf Grund angeblicher Verschwörungen. Wirken
während der Gefangenschaft.}
\end{center}

\section{Zaubereivorwürfe und Besuch in der Heimat}

Nachdem ich meine Arbeit in London\ort{London} getan, ging ich nach
Bedfordshire\ort{Bedfordshire} und
Northamptonshire\ort{Northamptonshire}. In 
Wellingborough\ort{Wellingborough} hatte
ich eine grose Versammlung [...]. Die \textit{Frommen} waren in
groser Aufregung hier, den die bösen Priester der 
Presbyterianer\index{Presbyterianer} und 
Independentens\index{Independenten} hatten fälschlich ausgestreut, 
wir trügen\index{Gerüchte}\index{Zauberei}
Flaschen mit uns herum, aus denen wir den Leuten zu trinken
geben, damit sie uns nachfolgen; aber die Kraft und der Geist
Gottes liesen die Freunde über diesen falschen Gerüchten stehen [...].


Von Wellingborough ging ich nach Leicestershire\ort{Leicestershire}, 
wo Oberst Hacker\person{Hacker, Oberst} drohte, wenn ich hierher 
käme, würde er mich wieder gefangen nehmen lassen; aber als 
ich nach Whetstone\ort{Whetstone} kam, wo er
mich das letzte Mal in der Versammlung hatte festnehmen lassen,
war alles ruhig. Oberst Hackers Frau kam in die Versammlung 
und wurde bekehrt, [...] es waren auch zwei Friedensrichter 
in dieser Versammlung, aus Wales, namens Walter
Jenkin\person{Jenkin, Walter} und Peter Price\person{Price, Peter}, 
die beide später treue Diener des Herrn wurden.

Von da gingen wir nach Sileby\ort{Sileby} [...] und dann nach 
Drayton,\ort{Drayton} meiner Heimat, wo früher so viele 
Priester und \textit{Fromme} gegen
mich aufgetreten waren, jetzt aber rührte sich keiner. Ich fragte
einen meiner Verwandten,\index{Fox!Verwandte} wo alle Priester 
und \textit{Frommen} seien? Man sagte mir, der Priester von 
Nun~Caton\ort{Nun Caton} sei gestorben, und nun bewerben 
sich acht oder neun um seine Stelle.
\zitat{Sie werden dich diesmal in Ruhe lassen}, sagten sie zu mir,
\zitat{denn wie die Krähen sich um ein totes Schaf schaaren, so tun die
Priester, wenn eine Pfründe frei ist}. Das waren von ihren
eigenen Zuhörern, die so redeten! [...].\index{Gier}

\section{Begegnung mit Naylor}

Als ich nach Derbshire\ort{Derbshire} kam, kam James 
Naylor\person{Naylor, James} zu mir
und sagte mir, sieben oder acht Priester hätten ihn zu einer
Unterredung aufgefordert. Ich war nun seinetwegen sehr 
bekümmert in meinem Geist, und der Herr gebot mir ihm zu sagen,
er solle der Aufforderung folgen; denn der Herr der Allmächtige
wolle bei ihm sein und ihm durch feine Kraft den Sieg geben.
% \picinclude{./100-109/p_s105.jpg} 
Und der Herr tat es, so das die Leute merkten, das die Priester
geschlagen waren; und sie riefen: \zitat{ein Nagler 
(Naylor engl.: Nagler)
hat sie alle zu Grunde gerichtet!} Er kam nach dem Disput zu
mir voll Dank gegen Gott [...].

Nun zogen wir durch Worrestershire\ort{Worrestershire}; ich 
hatte in Birmingham\ort{Birmingham} eine Versammlung [...]. 
Dann kamen wir nach Worcester\ort{Worcester} [...]. Von da 
nach Tewkesbury\ort{Tewkesbury} [...]. Dann nach Warwirk\ort{Warwirk}, 
wo ich im Hause einer Witwe eine Versammlung hatte [...].
Nach derselben, als ich gerade fort gehen wollte, [...] kam ein
Gerichtsdiener herein und fragte: \zitat{Wen hören die Leute zu so
später Stunde?} Er verhaftete John Crook\person{Crook, John}, 
Amor Stoddart\person{Stoddart, Amor},
Gerrard Roberts\person{Roberts, Gerrard} und mich, erlaubte 
uns jedoch in unsere Herberge zu gehen; nur sollten wir am 
Morgen wiederkommen [...].


Aber am nächsten Morgen hieß es, wir können unsrer Wege
gehen [...]. Nun gingen wir weiter nach Coventry\ort{Coventry}, [...]
dann durch Leicestershire\ort{Leicestershire} nach 
Swannington\ort{Swannington} und Baldock\ort{Baldock}. Hier
fragte ich, ob keinerlei Art von besonderem Bekenntnis vertreten
sei? Es hieß, es gebe einige Baptisten\index{Baptisten} und 
eine kranke Baptistenfrau. John 
Rush\person{Rush, John} ging mit mir zu ihr [...]. Als wir zu
ihr kamen, waren viele fromme Leute bei ihr. Man sagte mir,
diese Frau gehöre nicht mehr diesem Leben an; wenn ich ihr aber
etwas über das zukünftige sagen könne, so solle ich es tun. Der
Herr trieb mich, zu ihr zu reden, und sie erholte sich wieder,
\person{Fox!Heilung}\index{Heilung} zum Erstaunen der Stadt 
und des ganzen Landes. Diese 
Baptistenfrau und ihr Mann wurden gewonnen, und viele Hunderte
von Leuten haben sich seither in ihrem Hause versammelt [...].


Wir gingen nun über Market Street\ort{Market Street} [...] 
und St.~Albans\ort{St.~Albans}
nach London\ort{London} [...]. Nachdem ich mich einige Zeit in London
aufgehalten hatte und die dortigen Freunde in ihren 
Versammlungen besucht hatte, verließ ich die Stadt, wo 
ich James Naylor\person{Naylor, James}
zurückließ. Als ich mich von ihm trennte, fiel mein Blick auf
ihn und eine Angst befiel mich seinetwegen, aber ich ging doch
weg und ritt nach Newgate in Surrey\ort{Newgate in Surrey} [...].

\section{Baptisten überlassen den Quäkern nicht ihr Versammlungshaus}

Von da gingen wir nach Dorchester\ort{Dorchester} und stiegen in einer
Herberge, die einem Baptisten gehörte, ab; wir baten die in der
Stadt wohnenden Baptisten\index{Baptisten}, uns ihr Versammlungshaus zu
überlassen, damit wir in demselben Versammlungen halten könnten,
aber sie verweigerten es; wir ließen sie fragen, warum sie es
verweigerten; dadurch ward die Sache in der Stadt ruchbar.
% \picinclude{./100-109/p_s106.jpg} 
Wir ließen ihnen nun sagen, das sie und alle, die Gott fürchteten,
in unsere Herberge kommen könnten, wenn sie wollten. Sie
waren in großer Aufregung, und viele ihrer Lehrer und andere
von ihren Leuten kamen in unsere Herberge und schlugen mit den
Bibeln\index{Bibeln!schlagen mit} auf die Tische. 
Ich fragte sie, worüber sie denn so aufgebracht seien, ob 
sie gegen die Bibel so aufgebracht seien? Da
fingen sie an mit Auseinandersetzungen über ihre 
Wassertaufe.\index{Taufe!mit Wasser}
Ich fragte sie, ob sie behaupten könnten, von Gott gesandt zu
sein, die Leute zu taufen, wie Johannes 
(Joh. 1:6\bibel{Joh. 01:06@Joh. 1:6}) und ob sie
den gleichen Geist haben wie die Apostel? Sie sagten: nein.
Darauf fragte ich sie, wie vielerlei Kräfte es denn gebe! ob es
noch andere gebe als die Kraft Gottes und die des Teufels?\index{Teufel}
Sie sagten, es gebe keine andere außer diesen beiden; darauf
sagte ich: \zitat{Wenn ihr nicht die Kraft Gottes habt, welche auch
die Apostel hatten, dann handelt ihr in der Macht des Teufels.}
Viele der Anwesenden, die nüchterne verständige Leute waren,
sagten: \zitat{die Baptisten treten den Rückzug an!} Viele angesehene
Leute wurden an dem Abend gewonnen, und wir hatten einen 
köstlichen Gottesdienst und des Herrn Kraft war über allen. Am
folgenden Morgen, als wir fortgingen, schüttelten die Baptisten in
ihrer Wut hinter uns her den Staub von ihren Füßen. \zitat{So}
sagte ich, \zitat{ihr tut solches in der Macht der Finsternis? 
dann tun wir es auch gegen euch, aber in der Kraft Gottes.}
Wir verließen Dorchester und gingen nach Weymouth\ort{Weymouth} [...].


Es war ein Kaoalleriehauptmann in der Stadt, der mich zu sich
kommen lies und mich gerne länger gehalten hätte; aber ich
durfte nicht länger bleiben. Er und ein Diener ritten etwa
sieben Meilen mit mir; Edward Pyot\person{Pyot, Edward} war 
auch dabei. Dieser
Hauptmann war der behäbigste, fröhlichste, leutseligste und 
lachlustigste Mensch,\index{Ausgelassenheit} der mir je 
begegnete, so das es mich einige Male
trieb, ihm in der gewaltigen Kraft des Herrn zuzusprechen, aber
es war ihm so zur Gewohnheit geworden, das er immer wieder
über alles, was er sah, lachte. Aber ich ermahnte ihn immer
wieder, ernsthaft zu werden und gottesfürchtig. Wir brachten die
Nacht in einem Wirtshaus zu; am Morgen trieb es mich, noch
einmal mit ihm zu reden, ehe wir uns trennten. Als ich ihn
das nächste Mal sah, teilte er mir mit, das die Kraft des Herrn
ihn so übernommen habe, während ich damals mit ihm redete
beim Abschied, das er ganz ernsthaft geworden sei, ehe er heim
% \picinclude{./100-109/p_s107.jpg} 
kam und sein Lachen gelassen habe; er bekehrte sich später und
wurde ernsthaft und gut und starb in der Wahrheit [...].


Wir kamen nach Kings Bridge,\ort{Kings Bridge} wo wir in unsrer Herberge
nach den Ernstgesinnten in der Stadt fragten. Sie schickten uns
zu Nieclas Tripe\person{Tripe, Nieclas} und seiner Frau und wir 
gingen dorthin. Sie
ließen den Priester holen, mit dem wir uns längere Zeit unterredeten,
aber da er unterlag, verließ er uns bald. Nicolas 
Tripe\person{Tripe, Nicolas} und
seine Frau wurden gewonnen; und seitdem kommen in jener Gegend
häufig Freunde zusammen. 


Als wir am Abend in unsere
Herberge kamen und viele dort antrafen, welche 
tranken,\index{Alkohol} trieb \person{Predigen!in Kneipe}
mich der Herr zu ihnen zu gehen und sie auf das Licht 
hinzuweisen, welches Christus ihnen allen angezündet habe, durch das 
sie ihr böses Tun erkennen könnten, ihre bösen Reden und auch
Jesus Christus ihren Heiland. Dem Wirt wurde es unbehaglich,
weil er sah, das ich seine Leute vom Trinken abhielt, und sowie
ich die letzten Worte geredet, nahm er ein Licht und 
sagte: \zitat{Kommt,
hier ist ein Licht, mit dem ihr in euer Zimmer gehen könnt}. Am
nächsten Morgen, als er abgekühlt war, stellte ich ihm vor, wie
unziemlich er sich benommen hatte und ermahnte ihn beim Abschied,
an den Tag des Herrn zu denken [...]. 

\section{Erneute Verhaftung}

Wir zogen durch
Penryn nach Helston [...] und von da nach 
Market-Jew,\ort{Market-Jew} wo
wir in eine Herberge gingen [...]. Am nächsten Morgen 
versammelten sich die Behörden und schickten ihre Konstabler, um
uns vor sie zu holen. Wir fragten sie nach dem Verhaftbesehl;
sie sagten, sie hätten keinen; [...] es kamen auch mehrere
andere höhere Beamte, und wir stellten ihnen Vor, was das für
ein schmähliches Betragen sei, Reisende in ihrer Herberge zu 
behelligen [...]. Ehe wir die Stadt verließen, verfasste ich noch
ein Schreiben an die sieben Gemeinden in Lands-End. Es hieß
darin zum Schluss: 


\brief{Schreiben an sieben Gemeinden in Lands-End}
{

Nützet eure Zeit, dieweil sie euch gegeben
ist; denn jetzt ist \zitat{eure angenehme Zeit, jetzt ist euer Tag des
Heils} (2. Cor. 6:2\bibel{Cor. 2. 06:02@2. Cor. 6:2}). 
In einem jeden von Euch ist ein Licht
von Christus, das euch zeigt, das ihr nicht lügen, nicht unrecht
tun, nicht schwören, nicht fluchen, nicht stehlen, noch Gottes Namen
missbrauchen sollt. Wenn ihr dieses Licht lieb habt und ihm
folgt, so wird es euch zu Christus führen, welcher der Weg zum
Vater ist, dem Vater des Lichts, bei welchem nichts Ungöttliches
ist. Wenn ihr dieses Licht hasset, so wird es euch zum Verderben 
werden; wenn ihr es aber liebt, so bringt es euch ab von
% \picinclude{./100-109/p_s108.jpg} 
den Lehrern der Welt, damit ihr von Christus lernt, und 
bewahrt euch vor dem Unrecht der Welt und allen ihren 
Verführern. 

\bigskip

\begin{flushright}
G. F.\end{flushright}

}

Dieses Schreiben trug ein Freund, der mich begleitete, bei
sich; als wir nun etwa drei Meilen von Market-Jew gegen
Westen weiter gegangen waren, begegnete er einen Mann, dem
er eine Abschrift davon gab. Es stellte sich heraus, das dieser
Mann ein Diener vom Gefolge des Peter 
Ceely\person{Ceely, Peter} war, des
Obersten der Armee und Friedensrichters jener Gegend. Der Mann
ritt nun voraus und zeigte das Schreiben dem Major Ceely. Als
wir nach St. Ives\ort{St. Ives} kamen, verlor Edward 
Pyots\person{Pyots, Edward} Pferd ein Hufeisen
und wir hielten an, um es wieder beschlagen zu lassen. 
Währenddessen ging ich zum Meeresstrand hinunter. Als ich zurück kam,
fand ich die Stadt in Aufruhr, und sie schleppten eben Edward
Pyot und einen andern vor Major Ceely. Ich folgte ihnen ins
Richthaus, obgleich mich niemand dazu zwang. [...] Man
legte uns den Abschwörungseid vor, worauf ich meine Hand in
die Tasche steckte [...]. Major Ceely hatte einen albernen
Priester bei sich, der uns viele nichtssagende Fragen stellte; unter
andrem verlangte er, ich solle mein Haar, das damals ziemlich
lang war,\person{Fox!lange Haare}\index{Haare!lange} 
schneiden lassen, aber ich mochte es nicht schneiden
lassen, obgleich viele sich oft daran stießen; ich sagte ihnen, ich
sei ja nicht stolz darauf, und ich lasse es ja nicht selber wachsen.
Zuletzt übergab man uns einer Wache, die so grob gegen uns
war, wie der Richter selber; dessen ungeachtet verkündeten wir
die Wahrheit unter den Leuten. 


Am folgenden Morgen sandte
man uns unter Bewachung mehrerer Berittener, die mit Schwertern
und Pistolen bewaffnet waren, nach Redruth [...] und von da
wurden wir nach Laimceston\ort{Laimceston} gebracht [...].
Es waren noch 9 Wochen, bis wir vor Gericht erscheinen
mussten, wozu dann viel Volk herbei strömte, um das Verhör der
Quäker zu hören. Hauptmann Bradden war in Launeeston mit
seiner Reiterei, und seine Leute geleiteten uns durch die 
Volksmenge, welche die Straßen füllte, und es war kein Geringes, uns
hindurchzubringen; auch an allen Fenstern und Türen standen
Leute, die uns sehen wollten. 

\section{Gerichtsverhandlungen mit falschen Anschuldigungen}

Im Gerichtshof angekommen,
warteten wir eine Weile, den Hut auf dem Kopfe; niemand 
bekümmerte sich um uns, zuletzt trieb es mich zu sagen: 
\zitat{Friede sei mit Euch.} Da fragte Richter Glynne, 
damals Ober-Richter
% \picinclude{./100-109/p_s109.jpg} 
von England, den Wörter: \zitat{was sind das für Leute, die
ihr in den Gerichtshof gebracht habt?} \zitat{Gefangene, Herr,} 
antwortete dieser. \zitat{Warum nehmt ihr eure Hüte nicht 
ab?}\index{Hut!abnehmen} fragte
uns der Richter; wir antworteten nichts. \zitat{Nehmt eure Hüte
ab!} wiederholte der Richter, wir sagten wieder nichts; der
Richter sagte: \zitat{der Rat befiehlt euch, die Hüte abzunehmen.}
Nun redete ich und sagte: \zitat{Wann hat je ein Richter, König oder
sonst eine obrigkeitliche Person von Moses bis Daniel, bei den
Juden, dem Volke Gottes, oder bei den Heiden, je befohlen, das
man den Hut abnehme, wenn man vor Gericht erscheint? und
wenn das Gesetz von England irgend etwas derartiges befiehlt,
so zeiget uns dieses Gesetz irgendwo geschrieben oder gedruckt.}
Da wurde der Richter sehr zornig und sagte: \zitat{ich trage mein
Gesetzbuch nicht auf dem Rücken!} \zitat{So nenne mir irgend ein
Buch, welches Statuten darüber enthält, das ich es lesen kann,}
sagte ich. Da gebot der Richter: \zitat{führt den Kerl weg! ich will
ihn züchtigen}, und sie führten uns fort, zu den Dieben hinunter.
Doch gleich darauf rief er den Gefangenwärter wieder, und
gebot ihm, uns wieder zu bringen. Dann sagte er: \zitat{hatten sie
denn etwa Hüte zur Zeit des Moses und Daniels? antwortet
mir! nicht wahr, nun habe ich euch erwischt!} Ich erwiderte:
\zitat{Du kannst Daniel 3\bibel{Daniel 3} lesen, das die 
drei Männer auf Befehl des Nebukadnezar »in Rock, Hosen und Hut« 
in den Feuerofen geworfen wurden\footnote{Anmerkung: von 
Hüten steht in den gängigen Übersetzungen nichts}}. Dieses 
einfache Beispiel machte ihn verstummen,
so das er, weil er nichts mehr zu sagen wusste, rief: 
\zitat{ führet sie wieder fort!} so wurden wir denn 
wieder zu den Dieben hinuntergebracht. 


Am Nachmittag wurden wir wieder vor Gericht
gebracht [...]. Als wir dort warteten, bis wir an die Reihe
kamen, und ich die Menge derer, die hier schwörten, sah, betrübte
es mich, das so viele, die sich für Christen ausgaben, so offen
dem Gebot Christi ungehorsam waren, und der Herr trieb mich,
ein Blatt auszuteilen gegen das Schwören, welches ich bei mir
trug [...].\index{Flugblatt}
Dieses Blatt machte die Runde bei den Gerichtspersonen, und
sie gaben es zuletzt dem Richter, und als wir nun vor ihn 
gerufen wurden, fragte er mich, ob dieses verführerische Blatt mein
sei? Ich antwortete: wenn sie es vor dem ganzen Hofe vorlesen
wollen, so höre ich, ob es mein sei, und dann wolle ich auch 
dazu stehen. Er wollte, das ich es nehme und für mich durchlese.
% \picinclude{./110-119/p_s110.jpg} 
Ich wiederholte, man solle es vorlesen, damit alle urteilen 
könnten,\index{Transparenz}
ob etwas Verführerisches darin sei, in dem Falle wolle ich dafür
leiden. Schließlich lass es der Angestellte mit lauter Stimme, das
alle es hören konnten; als er fertig war, sagte ich: \zitat{ja, es ist
mein Blatt, ich stehe dazu, und ihr müsst auch dazu stehen, wenn
ihr nicht die Schrift verleugnen wollt, denn ist es denn nicht, was
die Schrift sagt, und Christus und die Apostel, denen alle wahren
Christen gehorchen müssen?} Nun ließen sie den Gegenstand
fallen, und der Richter kam wieder auf unsere Hüte zurück und
hieß den Kerkermeister sie uns abnehmen, dieser tat es; aber wir
setzten sie wieder auf [...]. Der Richter hielt nun eine lange
Rede über den Lord Protektor, wie er ihn zum obersten Richter
in England gesetzt, und ihn hierhergeschickt, und dergleichen mehr.
Wir baten ihn, er solle uns Gerechtigkeit erzeigen nach unserer
ungerechten Gefangenschaft diese neun Wochen; statt dessen aber
brachten sie eine Anklage vor, die sie gegen uns zusammengesetzt
hatten, so voll Lügen, das ich meinte, sie richte sich gegen einen
Dieb: wir seien nur mit Waffengewalt und nach großem Widerstand 
hierher gebracht worden! und doch waren wir, wie oben
gemeldet, gekommen. Ich sagte ihnen, das sei falsch und wir
wiederholten unser Gesuch um Gerechtigkeit; die Gefangennahme,
sagte ich, sei ungerecht, denn ich sei auf der Reise von Major
Ceely festgenommen worden. Nun redete Peter Ceely mit dem
Richter und sagte, auf mich zeigend: \zitat{Erlaubt mein Herr, dieser
Mann nahm mich bei Seite und sagte, er könne in einer Stunde
vierzigtausend Mann stellen und das Land in Blut stürzen und
König Karl\person{König Karl} zurückbringen, und ich könne 
ihm dabei behilflich sein.
Ich wollte ihm aus dem Lande helfen, aber er wollte nicht
gehen; ich habe Zeugen, dies zu beschwören}, und er rief den
Zeugen auf. Aber der Richter war nicht gewillt, ihn anzuhören,
und so bat ich, man möchte meine Anklage, auf Grund deren
ich verhaftet sei, vorlesen; der Richter sagte: \zitat{nein, 
sie soll nicht vorgelesen werden}, [...] als ich sah, 
das man sie nicht lesen
wollte, sagte ich zu einem meiner Mitgefangenen, \zitat{du hast eine
Abschrift davon, lies die vor.} \zitat{Kerkermeister}, sagte hierauf der
Richter, \zitat{führ ihn sort! wir wollen doch sehen, wer hier Meister
ist, er oder ich!} und so wurde ich hinweg geführt. Als ich
wieder gerufen wurde, bestand ich wiederum darauf, das mein
Verhaftbefehl vorgelesen werde, denn davon hing meine Gefangenschaft
% \picinclude{./110-119/p_s111.jpg} 
ab. Ich hieß abermals meinen Mitgefangenen ihn lesen,
und er tat es:

\brief{Haftbefehlabschrift}
{
    Peter Ceely, einer der Friedensrichter der Grafschaft, an
    den Kerkermeister von Seiner Hoheit Gefängnis zu Launceston:

    \bigskip

    \zitat{Ich sende Euch hiermit durch den Überbringer dieser 
    die Personen Edward Pyot\person{Pyot, Edward} von Bristol 
    und George Fox aus 
    Drayton-in-the-Elay\ort{Drayton in the Elay} in 
    Leicestershire, und William Salt\person{Salt, William} 
    von London, [...] die als Quäker bekannt sind und sich 
    selber als solche bekennen;
    sie haben Verschiedene Blätter verbreitet, die den öffentlichen
    Frieden gefährden, und können keinen gesetzlichen Grund für ihr
    Erscheinen in dieser Gegend angeben, sie sind gänzlich unbekannt
    in dieser Gegend, haben keinen Pass, weigern sich, irgendwelche
    Beweise ihres guten Wandels zu geben, die das Gesetz verlangt,
    und weigern den Abschwörungseid zu leisten. Wir befehlen euch
    darum im Namen seiner Hoheit des Lord Protektor, diese Personen, 
    [...] wenn sie kommen, in Gewahrsam zu bringen und darin 
    zu lassen, bis sie gesetzlich frei gelassen werden. Versäumet
    nicht, solches zu tun, wo anders es euch gefährlich werden könnte.}

    \bigskip

    Ausgegeben mit meiner Unterschrift und Siegel, St. Joes, den
    18. Januar 1655\jahr{1655}. P. Ceely.
}


Als dies vorgelesen worden war, sagte ich zu den Richtern, [...]
mich an Major Ceely wendend: \zitat{Wo und wann habe ich dich
beiseite genommen? [...] und wenn du mein Ankläger bist, 
warum sitzest du auf der Richterbank? du solltest herunter kommen
und mir ins Gesicht sehen. Übrigens möchte ich fragen, ob nicht
Major Ceely sich des Verrats schuldig machte, dessen er mich 
anklagt, durch sein langes Schweigen? Kennt er seinen Platz als
Soldat wie als Friedentzrichter? denn [...] wenn ich ihn beiseite
genommen, um ihm zu sagen, ich könne vierzigtausend Mann stellen,
und so weiter, [...] so sehet ihr deutlich, das er ja in dieser
Verschwörung beteiligt gewesen wäre, indem er mich [...] aus
der Gegend forthaben wollte [...] und den Verrat nicht früher
entdeckte. Aber ich leugne seine Aussagen und bin unschuldig an
diesem teuflischen Plan.} 


Die Richter ließen nun die Sache fallen,
denn sie sahen das, anstatt das sie mich in eine Falle gelockt hatten,
ich selber ihnen eine gestellt hatte. Major Ceely behauptete nun,
ich habe ihm ins Gesicht geschlagen [...]. Ich fragte ihn, ob er
sich als Richter und Soldat nicht schäme, solches zu sagen [...].
Schließlich, als die Richter sahen, das diese Fallen nichts nutzten,
% \picinclude{./110-119/p_s112.jpg} 
ließen sie uns wieder ins Gefängnis führen und forderten von
jedem zwanzig Goldstücke, weil wir den Hut aufbehalten [...].

\section{Misshandlung in der Haft}

Als das Urteil so lautete, das keine baldige Freilassung
zu erwarten war, hörten wir auf, dem Wörter wöchentlich
7 Schilling für unsere Pferde und 7 für uns selber zu geben;
daraufhin wurde er böse und ganz teuflisch und brachte uns nach
Doomsdale\ort{Doomsdale} hinunter, einen greulichen, stinkenden 
Ort wohin die
Mörder nach der Verurteilung gebracht wurden. Der Ort war
sehr ungesund, so das wenige, die sich hier aufhalten mussten,
wieder gesund heraus kamen; es war kein Abtritt da, und der
Unrat der Gefangenen war seit Jahren nie hinausgeschafft worden.
Es war ein förmlicher Sumpf darin, stellenweise bis über die
Schuhe, von dem Unrat; und man erlaubte uns nicht, rein zu
machen oder uns Betten oder Stroh zum drauf liegen zu 
verschaffen. Am Abend brachten uns einige Bekannte aus der Stadt
ein Licht und etwas Stroh, um drauf zu liegen; wovon wir einiges
verbrannten, um den Gestank zu vertreiben. Die Diebe schliefen
gerade über uns und der Wärter in einem Zimmer daneben.
Scheints drang der Rauch ins Zimmer des Wärters; er geriet
in einen solchen Zorn, das er die 
Nachtgeschirre\footnote{Anmerkung: Nicht klar, was gemeint 
ist} der Diebe nahm,
und sie durch ein Loch gerade auf unsere Köpfe ausleerte; wir
waren so beschmiert davon, das wir weder einander noch uns selber
anrühren konnten; und der Gestank war so arg, das wir fast
darin erstickten. Vorher hatten wir den Gestank zu unsern Füßen
gehabt, jetzt hatten wir ihn auch auf den Köpfen und am Rücken;
und da unser Stroh von dem heruntergeworfenen Dreck beschmutzt
war, so verbreitete es einen greulichen Dunst. Zudem fluchte der
Wärter grässlich über uns und nannte uns \zitat{hackengesichtige 
Hunde}; und andere merkwürdige Namen, die wir noch nie gehört. 

In
diesem Zustand gingen wir fast zu Grunde während der Nacht,
denn wir konnten nicht einmal absitzen, alles war so voll Unrat.
Wir mussten lange in dem Zustand ausharren, bis uns gestattet
wurde, reinzumachen und uns andere Lebensmittel zu verschaffen als
das, was durchs Gitter kam. 

Einmal brachte ein Mädchen uns etwas
Essen; der Wärter arretierte es und führte es vor Gericht, weil
es ins Gefängnis eingedrungen sei, und es geriet in große Not;
dadurch wurden viele andere entmutigt, so das es uns schwer
wurde, uns Wasser oder Lebensmittel zu verschaffen. Wir ließen
nun eine junge Frau aus Londen kommen, Anna 
Downer\person{Downer, Anna}, damit
% \picinclude{./110-119/p_s113.jpg} 
sie uns das Essen kaufe und zubereite; sie war dazu bereit, denn
es war über sie gekommen, zu uns zu kommen in der Liebe
Gottes, und sie war sehr dienstfertig gegen uns [...].


Die Gefangenen und einige andere verschrobene Leute berichteten 
von Gespenstern, die in Doomsdale\ort{Doomsdale} umgingen, und von
den vielen, die hier gestorben seien, um einen damit angst zu
machen. Aber ich sagte, das, wenn auch alle Geister und Teufel
der Hölle dort seien, ich darüber stehe, durch die Kraft Gottes,
und nichts dergleichen fürchte; denn Christus unser Priester werde
uns das Haus und die Mauern heiligen, er, der dem Teufel den
Kopf zerbrochen habe [...].

\section{Durch die Verfolgung bekommen die Quaker Aufmerksamkeit}

Es war nun bald die Zeit der allgemeinen vierteljährlichen
Getichtssitzung, und da der Kerkermeister sich immer noch schlecht
gegen uns benahm, setzten wir einen Bericht über unsere Leiden
auf und schickten ihn zur Gerichtssitzung nach Bodmin.
\index[brief]{Protesprief an das Gericht in Bodmin} Als die
Richter ihn gelesen, gaben sie den Befehl, das die Türen von
Doomsdale geöffnet werden sollten und man uns erlaube, rein
zu machen und unsere Nahrung in der Stadt zu kaufen. Wir
sandten eine Abschrift unseres Leidensberichts an den Protektor
\index[brief]{Protesprief an den Protektor}
und erzählten ihm, wie wir von Major Ceely verhaftet und 
verurteilt worden waren, und wie uns der Kerkermeister misshandelt
hatte. Der Protektor schickte einen Befehl an Hauptmann Fox,
\person{Hauptmann Fox, }
den Befehlshaber von Schloss Pendennis, das er untersuche, wie
es sich mit den Soldaten, die uns misshandelten, verhalte [...].


Solches war der Sache des Herrn sehr förderlich; denn
nachher konnten die Freunde in jedem Turmhaus oder Marktplatz 
reden und es tat ihnen niemand etwas. Ich hörte, das
Hugh Peters\person{Peters, Hugh}, einer der Kapläne des 
Protektor, diesem gesagt habe,
man könne dem George Fox keinen größeren Dienst zur 
Ausbreitung seiner Ansichten in Cornwall tun, 
als ihn in Cornwall\ort{Cornwall}
einzusperren. Und wirklich kam meine Gefangennahme in Cornwall
vom Herrn zur Förderung seiner Sache in dieser Gegend; denn
als es nach der Gerichtssitzung hieß, wir würden gefangen bleiben,
kamen Freunde aus allen Teilen des Landes, um uns zu besuchen.
\index{Gefangenenbesuche}


Diese westlichen Gegenden waren damals sehr in Finsternis\index{Finsternis}, aber
das Licht und die Kraft und die Wahrheit des Herrn brachen
nun hervor und leuchteten über allen, und viele bekehrten sich
von der Finsternis zum Licht und von der Macht 
des Satans\index{Satan}
zu Gott. Es trieb viele, in die Turmhäuser zu gehen, und viele
% \picinclude{./110-119/p_s114.jpg} 
besuchten uns, denn wir durften nun umher gehen im Schlosshof,
und an den Ersten Tagen kamen viele zu uns, denen wir das
Wort des Lebens brachten [...].

In Cornwall, Devonshire, Dorfetshire und Somersetshire
fing die Wahrheit an mächtig zu sprießen, und viele bekehrten sich
zur Christus; viele Freunde fühlten sich getrieben, die Wahrheit
in diesen Gegenden zu verkünden, was die Priester und die
\textit{Frommen} sehr aufbrachte, so das sie die Behörden anstifteten,
den Freunden Fallen zu stellen. Sie stellten Wachen auf den
Landstraßen, unter dem Vorwand, alle verdächtigen Personen
abzufassen, sie ergriffen nun daraufhin Freunde, die vorbeikamen,
um uns im Gefängnis zu besuchen [...]. Aber gerade das,
was sie taten, um der Wahrheit Einhalt zu tun, diente dazu, sie
auszubreiten; denn dadurch wurden die Freunde oft getrieben, zu
den Konstablern oder den Behörden, vor die sie gebracht wurden,
zu reden, was viel dazu beitrug, das die Wahrheit sich in allen
Distrikten ausbreitete. Oft wenn Freunde in die Hände der
Wachen gerieten, ging es zwei oder drei Wochen, ehe sie wieder
frei wurden.


Als Thomas Rawlinson\person{Rawlinson, Thomas} aus dem 
Norden her kam, um uns
zu besuchen, ergriff ihn ein Konstabler in Devonshire und nahm
ihm nachts zwanzig Schilling aus der Tasche, und darauf wurde
er zu Exeter ins Gefängnis geworfen. 
Henry Pollexfen\person{Pollexfen, Henry} warfen
sie auch ins Gefängnis, weil er ein Jesuit sei [...]. Viele
Freunde wurden von ihnen misshandelt; ja Leute, die an ihrer
Arbeit waren, wurden von ihnen gepeitscht und ergriffen, und
es waren doch solche darunter, die eine Einnahme von mehr
als achtzig und hundert Pfund im Jahr hatten; und zwar
geschah ihnen solches, wenn sie kaum vier oder fünf Meilen von
zu Hause weg waren. Unter dem Eindruck all des Bösen, das
mit dem Aufstellen der Wachen und dem Gefangennehmen der
Freunde beabsichtigt war, kam es über mich folgendes zu schreiben:

\brief{Mahnung und Warnung an die Behörden}
{
    Eine Mahnung und Warnung an die Behörden.

    \bigskip

    Ihr Mächte der Erde, Christus ist gekommen um zu regieren,
    \index{Endzeiterwartung}\person{Fox!Endzeiterwartung}
    und er ist unter euch, und ihr kennet ihn nicht; er erleuchtet einen
    jeden unter euch, damit ihr alle an ihn glauben möchtet, an
    das Licht; an den \zitat{der die Kelter allein tritt}. Darum prüfet
    alle in diesem Lichte, ob ihr reif seid, denn die Kelter ist bereit.
    (Offb. 14:19\bibel{Offb. 14:19}) [...].
    % \picinclude{./110-119/p_s115.jpg} 
    Ihr verkündet Gewissensfreiheit\index{Gewissensfreiheit}; 
    und doch darf man seinen
    Freunden keine Briefe bringen, oder seine Freunde oder die
    Gefangenen besuchen, oder ihnen Bücher bringen, ohne das ihr
    Wachen ausstellt, um sie anzuhalten und zu greifen; und sogar
    bewaffnet müssen diese sein gegen die guten Leute, die kaum
    einen Stock mit sich tragen, und die ihr aus Groll Quäker nennt.
    Und die, welche diese Wachen aufstellen, die verkünden 
    Gewissensfreiheit und nehmen solche gefangen, die 
    ihr Gewissen gegen Gott
    und gegen die Menschen rein erhalten wollen, die Gott im Geist
    und in der Wahrheit anbeten, was die, welche nicht im Licht sind,
    Ketzerei nennen! [...] Ist je solch ein Geschlecht gewesen, das
    so wahnsinnig schlecht und verfolgungssüchtig war und Bewaffnete
    ausstellte gegen die Wahrheit und sie verfolgte, wie Grafschaften
    und Städte es jetzt tun? das klingt wie Sodom und Gomorrha!
    \bigskip
    \begin{flushright}G. F.\end{flushright}

}

Es kam mir eine Abschrift eines von der Sitzung von Exeter
ausgehenden Verhaftbefehls in die Hände, der in starken 
Ausdrücken verlangte, \zitat{alle Quäker zu verhassten}, 
und der die Wahrheit und die Freunde schlecht machte; 
da trieb es mich, eine
Antwort zu schreiben und zu verbreiten, um die Freunde und
die Wahrheit gegen solche Verleumdungen\index{Verleumdungen} 
zu verteidigen, und die Schlechtigkeit und Bosheit 
des Verleumdungsgeistes zu zeigen [...].

\section{Erneute Gerichtsverhandlungen}

Wir blieben im Gefängnis bis zur nächsten Sitzung; viele
Freunde, Männer und Frauen, die von der Wache ergriffen
worden waren, waren ins Gefängnis gebracht worden. Viele
von ihnen wurden nach Eröffnung der Sitzung vor die Richter
gebracht und beschuldigt, sie hätten sich gesträubt zu kommen
und waren doch von den Gefängniswärtern gebracht worden.
Der Richter legte ihnen Busen auf, weil sie den Hut nicht 
abnehmen. Wir hingegen mussten nicht mehr vor den Richter.


Während dieser ganzen Zeit und während der Sitzungen war
Unser Wirken für den Herrn reich gesegnet, denn es kamen viele
zu uns, \textit{Fromme} und andere, mit uns zu reden. Elisabeth
Trelawny\person{Trelawny, Elisabeth} von Plymouth, die 
Tochter eines Barons, wurde bekehrt,
worüber Priester und \textit{Fromme} und viele angesehene Personen
außer sich waren und ihr Briefe deswegen schrieben. Da sie eine
weise und gottselige Frau war und denen, die ihr geschrieben,
nicht wollte etwas in die Hand geben, das sie dann hätten können
gegen sie gebrauchen, so schickte sie mir die Briefe; und ich schrieb
% \picinclude{./110-119/p_s116.jpg} 
ihr darüber, und sie beantwortete sie dann. Sie nahm zu in der
Kraft und der Weisheit Gottes, so das sie zuletzt imstande war,
den wichtigsten Priestern und \textit{Frommen} zu antworten; sie hatte
die Herrschaft über sie in der Wahrheit durch die Kraft Gottes,
der sie treu blieb bis in den Tod.


Während ich hier in der Gefangenschaft war, prophezeiten
die Baptisten und Fifthmonarchyleute\index{Fifthmonarchyleute}, 
in diesem Jahre werde
Christus kommen und tausend Jahre auf Erden regieren. Sie
erwarteten dieses Reich als ein äußeres, während er doch in die
Herzen der Menschen gekommen war, um darinnen zu regieren;
aber so wollten ihn diese \textit{Frommen} nicht aufnehmen, darum
misslang ihnen ihr Prophezeien. Christus ist ja schon gekommen 
\index{Wiederkunft Christi}\index{Endzeit} und wohnet in 
den Herzen der Menschen und regieret
darin. Tausende, bei denen er anklopfte, haben ihm aufgetan;
und er ist bei ihnen eingekehrt und hat das 
Abendmahl\index{Abendmahl} mit ihnen
gehalten (Offb. 3:20\bibel{Offb. 03:20@Offb. 3:20}). 
Viele dieser Baptisten\index{Baptisten} und 
Fisthmonarchhileute\index{Fisthmonarchhileute} sind 
die ärgsten Feinde derer, die sich zu Christus hielten,
geworden; aber er regieret in den Herzen seiner Heiligen.



Während der Gerichtssitzung kamen mehrere der Richter zu
uns und waren ziemlich höflich und redeten vernünftig über
göttliche Dinge mit uns und bezeugten uns Teilnahme. 
Hauptmann Fox\person{Hauptmann Fox}, der Gouverneur 
von Schloss Pendennis, trat zu mir
und sah mir ins Gesicht, sagte aber nichts; aber als er wieder
zu seinen Begleitern zurück kam, sagte er, er habe noch nie in
seinem Leben einen einfältigeren Menschen gesehen. Ich rief ihm
nach: \zitat{Wir wollen sehen, wer der Einfältigere ist!} Aber er
ging seines Weges, der hochmütige Tropf.\index{Arroganz}


Thomas Lower\footnote{Thomas Lower, später Schwiegersohn 
von G. F.}\person{Lower, Thomas} besuchte uns 
ebenfalls [...]. Er stellte uns
viele Fragen darüber, das wir behaupteten, die Schrift sei nicht
das Wort Gottes,\index{Bibel!Wort Gottes} und über die 
Sakramente\index{Sakramente} und anderes, und
wir konnten über alles Aufschluss geben. Ich redete auch noch
allein mit ihm, und er bekannte nachher, meine Worte hätten
ihn wie ein Blitzstrahl durchzuckt.\person{Fox!Wirkung auf Andere} 
Er habe noch nie Leute, wie
wir seien, getroffen, die seine innersten Gedanken errieten. Er
wurde nachher bekehrt und ist ein Freund geblieben bis auf diesen
Tag [...] und hat viel um der Wahrheit willen gelitten. 


Es trieb 
mich zu dieser Zeit, die folgende Mahnung an die
Freunde, die Prediger waren, zu richten:

% \picinclude{./110-119/p_s117.jpg} 
\brief{An die Freunde, die Prediger waren}
{
    Freunde!

    \bigskip

    Bleibet in der Kraft des Lebens und der Weisheit und in
    der Furcht des Herrn Himmels und der Erde, damit ihr in der
    Weisheit Gottes bewahret bleiben möget und seinen Gegnern ein
    Schrecken werdet, indem ihr die Wahrheit verbreitet, Zeugen für
    sie erwecket, die Betrügerei stürzet, von der Übertretung zum
    Leben bringt, in den Bund des Lichts und den Frieden Gottes.
    Lasset alle Welt diese Stimme hören, durch Wort oder Schrift.
    Schonet keinen Ort, noch Sprache, noch Feder; tuet das Werk
    in Gehorsam gegen Gott; kämpfet tapfer für die Wahrheit auf
    Erden und zertretet alles, was ihr entgegen ist.\index{Mission} 

    Ihr habt die
    Kraft, missbraucht sie nicht [...]. Regieret mit Christus, dessen
    Thron und Zepter nun ausgerichtet sind, und der herrscht bis an
    die Enden der Erden [...]. Es soll nun das Heil ausgehen
    von Zion, zu richten den Berg Esaus, und das Gesetz soll von 
    Jerusalem ausgehen (Obad), damit es rede zu dem Göttlichen, 
    das in einem
    Jeden ist, und alle Erfindungen und Erfinder überwältige. Alle
    Fürsten der Welt sind Luft vor der Macht Gottes, die ihr 
    geschmecket habt; darum lebet in ihr [...].
    Führet alle zur Anbetung Gottes; pflüget den brachliegenden
    Acker, dreschet das Korn, damit der Same, der Weizen, in die
    Scheunen gesammelt werden könne und Alle zum Ursprung,
    zu Christus kommen der war, ehe der Welt Grund gelegt ward.
    Die Spreu ist durch die Übertretung unter den Weizen gekommen;
    der, welcher ihn ausdrescht, hat die übertretung verlassen und
    erkennt sie, und unterscheidet zwischen dem Wertvollen und dem
    Unwerten, er kann den Weizen vom Unkraut unterscheiden und
    ihn in den Speicher sammeln und bringet so die unsterbliche
    Seele zu Gott, von dem sie kam [...]. Die Prediger des Geistes
    müssen dem gefangenen Geist predigen, damit durch den Geist
    Christi die Menschen zu Gott, dem Vater alles Geistes, geführt
    werden, ihm zu dienen und eins zu sein mit ihm, mit der Schrift
    und untereinander. Dies ist das Wort des Herrn an euch alle.
    \textbf{Seid ein Vorbild und Beispiel in allen Ländern, Ortschaften, 
    Inseln und Völkern, zu denen ihr kommt, damit euer
    Wandel allen Menschen predige [...].} Darin werdet ihr dem
    Herrn angenehm sein und ein Segen werden.


    Schonet den Betrug nicht; greifet ihn mit dem Schwert an;
    bekämpfet ihn; trachtet nicht nach Blut, weder in Wort noch
% \picinclude{./110-119/p_s118.jpg}
    Schrift [...]. Verkündet allen den lebendigen Gott; denn alle
    Lehre, Kirche und Gottesdienst, die durch menschlichen Willen und
    Verstand eingesetzt sind, werden von der Kraft Gottes vernichtet.
    .[...] Verkündet den großen Tag des Feuers und des Schwertes,
    den Tag des Herrn, der im Geist und in der Wahrheit will
    angebetet sein, und bleibt in der Kraft Gottes, damit die Bewohner
    der Erde vor euch erzittern; und damit die Kraft und Herrlichkeit
    des Herrn unter den Heiden Hund den Heuchlern gepriesen
    werde, und ihr in der Weisheit und Furcht, im Leben, im Schrecken
    und in der Herrlichkeit bewahrt bleibt zu seiner Ehre. Es gehet
    ein Ruf, das man die Übertretung verlasse, und der Geist ruft:
    \zitat{kommet}. Es ergehet jetzt ein Ruf, die falschen 
    Gottesdienste\index{Gottesdienst!falscher}
    zu verlassen und dem wahren Gott zu dienen; ein Ruf zur
    Buße [...] damit die Gerechtigkeit hervorbreche; und sie wird
    über die ganze Erde sich ausbreiten. Darum tut treu in der
    Kraft des Herrn euer Werk, ihr, die ihr auserwählt seid [...]
    Gehorchet der Kraft, sie wird euch erretten aus der Hand der
    Unvernünftigen und von der Welt. Durch sie werdet ihr das
    Reich haben, das kein Ende hat und in welchem Herrlichkeit und
    Leben ist [...]. 

\bigskip

\begin{flushright}G. F.\end{flushright}

}


Nach der Sitzung hatten wir manche Unterredungen mit dem
Sheriff und einigen Soldaten, die eine zum Tode verurteilte Frau
bis zur Hinrichtung überwachen mussten. Einer von ihnen sagte:
\zitat{Christus war einer der heftigsten Menschen, die je gelebt}; wir
verwiesen ihm dies. Ein andermal fragten wir den Kerkermeister,
was bei den Gerichtsverhandlungen vorkomme. Er antwortete:
\zitat{O, nur Kleinigkeiten; nur etwa dreißig, die wegen 
Bastardschaft\index{Bastardschaft}\index{Uneheliche Kinder}
verurteilt sind.} Wir wunderten uns sehr, das solche, die doch
Christen zu sein meinten, derartiges eine Kleinigkeit fanden.
Aber dieser Kerkermeister war selber ein sehr schlechter Mensch.
Ich ermahnte ihn oft zur Rechtschaffenheit, aber er behandelte
die Leute, die uns besuchen wollten, schlecht. 
Edward Pyot\person{Pyot, Edward} bekam
einen Käse von seiner Frau geschickt; der Kerkermeister nahm ihn
ihm weg und brachte ihn dem Major, angeblich um ihn auf
verräterische Briefe hin zu durchsuchen; aber obwohl sie nichts
von Briefen fanden, so behielten sie ihn doch. Es hätte diesem
Kerkermeister ganz gut gehen können, wenn er sich anständig
betragen hätte, aber er suchte selber sein Verderben, welches auch
bald über ihn kam; denn im darauf folgenden Jahre wurde er
% \picinclude{./110-119/p_s119.jpg} 
von seiner Stelle abgesetzt und kam selber ins Gefängnis und
bettelte dort bei den Freunden. Und wegen irgend eines 
Vergehens brachte ihn sein Kerkermeister nach Doomsdale und legte
ihn in Ketten und schlug ihn, und erinnerte ihn daran, wie er
jene guten Leute misshandelt habe, die er ohne jeden Grund in
diesen greulichen Kerker getan, und das er nun die verdiente
Strafe für seine Bosheit leiden müsse, und ihm nun mit dem
Maße gemessen werde, mit dem er gemessen habe. Es ging ihm
sehr schlecht und er starb in der Gefangenschaft, und sein Weib
und seine Kinder kamen ins Elend.

\section{Entlassung aus der Gefangenschaft}

Während ich zu Launceston gefangen war, ging ein Freund
zu Oliver Cromwell\person{Cromwell, Oliver} und erbot 
sich, an meiner Statt in 
Doomsdale gefangen zu sein, wenn er es annehmen und mich dafür
in Freiheit setzen wolle. Dies erstaunte Cromwell dermaßen,
das er zu seinen Räten sagte: \zitat{Welcher unter Euch 
würde so viel für mich tun, wenn ich in dieser Lage 
wäre?} Und obgleich er
das Anerbieten des Freundes nicht annahm, sondern sagte, er
könne es nicht tun, weil es gegen das Gesetz sei, so ergriff ihn
doch die Wahrheit mächtig. Einige Zeit darauf schickte er den
Generalmajor Desborough\person{Generalmajor Desborough} 
in der Absicht, uns frei zu lassen;
dieser kam und bot uns die Freilassung an unter der Bedingung,
das wir versprechen, heim zu gehen und nicht mehr zu predigen,
aber wir wollten ihm nichts versprechen; daraufhin schlug er uns
vor zu versprechen, nach Haus zu gehen, \zitat{wenn der Herr es 
zulasse}, worauf Edward Pyot\person{Pyot, Edward} ihm 
einen abschlägigen Brief schrieb.
Als einige Zeit verstrichen war, seitdem dieses Schreiben
abgegeben worden war, schrieb ich ebenfalls an ihn, 
folgendermaßen:

\brief{Brief an Oliver Cromwell}{
Freund,

\bigskip

    Wir, die wir in der Kraft Gottes des Herrschers aller Dinge
    sind, die wir seine Kraft kennen und in ihr wohnen, müssen ihr
    auch gehorchen; und darum müssen wir uns frei halten von
    allem, was Menschenwille befiehlt. Wenn es sich darum handelt,
    etwas zu kaufen oder zu verkaufen, so mag es etwa angehen zu
    sagen: wir wollen, so der Herr es zulässt; aber da wir in der
    Kraft Gottes stehen, unter keines Menschen Willen, so können wir
    solches nicht mit Wahrhaftigkeit sagen, wo es sich um unsere
    Befreiung aus der Gefangenschaft handelt [...].

\bigskip
\begin{flushright}13. des 6. Monats 1656\jahr{1656}. G. F.\end{flushright}

}

% \picinclude{./120-129/p_s120.jpg} 
Bald darauf kam Major Desborough\person{Major Desborough} 
nach Castle-Green\ort{Castle-Green} und
spielte Kegel\index{Spielen!Kegel} mit einigen 
Richtern und anderen. Einige Freunde
wurden getrieben zu ihnen zu gehen, und sie zu ermahnen, ihre
Zeit nicht mit solch eitlen Dingen zuzubringen, und sich ihren
unnützen Vergnügen\index{Vergnügen!unnütze} hinzugeben, 
und ihnen zu bedenken zu geben,
das, wenn sie, die sich doch für Christen ausgeben, dennoch solchen
Vergnügen nachgehen, während sie die Diener Gottes gefangen
halten, Gott sie wegen solchen Treibens heimsuchen werde. Aber
trotz allem, muss man ihnen sagte oder schrieb, lies er uns im
Gefängnis. Wir hörten später, das er die Sache Hauptmann
Vennet übergeben habe, der uns freigelassen hätte, wenn wir
dem Kerkermeister Geld gegeben hätten. Aber wir erklärten ihm,
das wir das nicht tun könnten, da wir ja unschuldigerweise 
gefangen waren [...]. Schließlich kam die Kraft des Herrn so
über ihn, das er uns freiwillig die Freiheit schenkte, am 13. Tage
des 7. Monats 1656\jahr{1656}. [...] Wir waren seit dem Frühling
gefangen gewesen.