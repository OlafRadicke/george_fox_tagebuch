%%%%%%%%%%%%%%%%%%% Kapitel 3. %%%%%%%%%%%%%%%%%%%%%%%%%%%%%%
\chapter[Tumult in Nottingham]{Tumult in Nottingham}

\begin{center}
\textbf{Der Tumult in Nottingham. Wachsender Widerstand, bis zum
Gefängnis in Derby.}
\end{center}


A16 ich einmal am Morgen einetz Ersten Tageß in der Nähe
von Nottingham von einem Hügel auß die Stadt überblickte, da
gewahrte ich daß riesige Turmhauß, und der Herr sagte zu mir:
,,Du mußt hingehen und gegen jene großen Götzen schreien und
gegen die, welche drinnen anbeten«. Jch sagte den ,,Freunden«,
die mit mir waren, nichtß davon, sondern ging mit ihnen hin in
die Versammlung, wo die mächtige Kraft der- Herrn mit uns
war; hier ließ ich sie und ging zum Turmhauß. Die Menge,
die ich hier sah, kam mir vor wie ein Brachfeld und der Priester
wie ein großer Erdklumpen, der oben aus seiner Kanzel stand.
Gr hatte zum Text die Worte des Petruß: »Wir haben ein festes-
prophetischeö Wort und ihr tut wohl, daß ihr daraus achtet, alß
auf ein Licht, das da scheinet an einem dunkeln Ort, biß der Tag
anbreche und der Morgenstern ausgehe in eueren Herzen«
(2. Petr. 1, 191. Er sagte den Leuten, nach dem, waß hier ge-
schrieben stehe, sollten sie alle Lehren, Bekenntnisse und Meinungen
prüfen. Da kam die Kraft dez Herrn so mächtig über mich und
war so stark in mir, daß ich nicht an mich halten konnte, sondern
rusen mußte: ,,O nein, nicht nach dem, was geschrieben stehet!«
und ich sagte ihnen, nach maß: nämlich nach dem heiligen Geist,
durch den die heiligen Männer Gotteö die Schrift geschrieben
haben. Durch diesen, sagte ich, müssen alle Lehren, Bekenntnisse
und Meinungen geprüft werden. Dieser Geist leitet in alle


% \picinclude{./020-029/p_s027.jpg} 
Der Tuniult in Nottingham. Wachsender Widerstand usw. 27
Wahrheit und zur Erkenntniß aller Wahrheit. Die Juden haben
die Schrift gehabt und widerstanden dem heiligen Geist doch und ver-
warfen Christuö, den schönen Morgenstern ; sie verfolgten Ehriftuö
und seine Apostel und wollten ihre Lehren nach der Schrift prüfen;
aber sie irrten in ihrem Urteil und prüften sie nicht richtig, weil
sie ohne den heiligen Geist pritften. Da ich nun so zu ihnen
redete, kamen die Wachen und führten mich weg und brachten
mich in einen wüsten, stinkenden Kerker; der Geruch stieg mir so
in die Nase und den Halß, daß etz- eine Qual war, aber die
Kraft deß Herrn schallte an dem Tage so in ihren Ohren, daß
sie ganz von dem Schall betäubt waren, und ihre Ohren wurden
noch eine zeitlang nicht frei davon, so waren sie im Turmhause
von der Kraft dez Herrn ergriffen worden. Am Abend brachten
sie mich vor die Behörden der Stadt; als ich vor sie trat, war
der Bürgermeister in verdrießlicher, mürrischer Laune, aber die
Kraft deß Herrn beschwichtigte ihn. Sie verhörten mich ausführ-
lich und ich berichtete ihnen, wie der Herr mich getrieben hatte
zu kommen. Nach einigem Hin- und Herreden schickten sie mich
in?-’ Gefängniß zurück. Aber bald darauf ließ mich der Oberscheriff,
John Neckleß, zu sich in sein Haus holen. Als ich eintrat, be-
gegnete mir sein Weib im Flur und sagte: ,,Unserm Hause ist
Heil widerfahren.« Sie reichte mir die Hand und war mächtig
ergriffen von der Kraft Gotte;-’, und ihr Mann und ihre Kinder
und Dienstboten wurden ganz umgewandelt, denn die Kraft des
Herrn war mächtig in ihnen. Jch wohnte bei ihnen und wir
hatten große Versammlungen in ihrem Hause; eß kamen auch
etliche angesehene Standeß-personen, und deß Herrn Kraft tat sich
mächtig kund unter ihnen; John Reckleß ließ dann einen andern
Scherifs holen und eine Frau, mit der sie in Geschäften zu tun
gehabt hatten, und erklärte in Anwesenheit des andern Scheriff,
daß sie beide diese Frau bei einem Handel geschädigt hätten und
sie entschädigen müßten. Er sagte es sehr freundlich, aber der
andere Scherifs leugnete, und die Frau sagte, sie wisse nichtö da-
von. Aber der gerechte Scheriff sagte, ez sei so, und der andere
wisse das ganz gut; nachdem er die Sache aufgedeckt nnd daß
Unrecht, daß sie getan, eingestanden hatte, entschädigte er die
Frau und ermahnte den andern ein gleiches zu tun; die Kraft
Gotteö war mit diesem guten Scherifs und wirkte eine große
Wandlung in ihm und er hatte große Offenbarungen. Alö er


% \picinclude{./020-029/p_s028.jpg} 
am darauffolgenden Martttage in den Pantoffeln in seinem
Zimmer auf- und abging, sagte er: »Jch muß auf den Markt
gehen und den Leuten Buße predigen,« und er ging auf den Markt
und in mehrere Straßen und predigte den Leuten Buße; und auch
noch andere aus der Stadt trieb ez, zu den Behörden zu gehen
und die Leute zur Buße zu erinahnen. Die Räte wurden sehr
böse über mich und ließen mich auß dem Hause deö Scheriff
holen und verurteilten mich zum Gefängniß. Alk; die Gerichtß-
fitzung stattfand, fühlte einer sich getrieben, sich statt meiner an-
zubieten, ,,Leib um Leib, Leben um Leben«. Alö ich vor den
Richter gebracht werden sollte, ging es ziemlich lang, bis mich
der Diener, der mich hinbringen sollte, abholte, und alß ich kam,
hatte sich der Richter schon erhoben, worauß ich sah, daß er er-
zürnt war; er sagte, er wolle dem Jüngling schon einen Verweis?.
geben, wenn er vor ihn gebracht werde; ich war damals unter
dem Namen ,,J«itngling« eingesperrt. Jch wurde denn wieder
inö Gefängnis gebracht. Die Kraft detz Herrn war mächtig
unter den ,,Freunden«, aber das?2 Volk fing an, tätlich zu
werden, so daß der Schloßkommandant Soldaten hinau?-schickte,
um die Leute au?-einander zu treiben, worauf es ruhig wurde;
alle, Priester und Volk, erstaunten ob der herrlichen Kraft, welche
heroorbrach, und etliche der Priester wurden empfänglich gemacht
und einige von ihnen bekannten sich zur Krast Gottes-.
Nachdem ich auö dem Gefängniß von Nottingham, wo ich
einige Zeit gefangen gewesen war, entlassen worden, zog ich
umher, wie vorher im Dienst des- Herm. Als ich nach Man?-field
Woodhouse kam, war dort eine verrückte Frau; daß Haar hing
ihr wirr über die Ohren und der Arzt war gerade bei ihr. Er
war daran, ihr zu Ader zu lassen, nachdem man sie zuvor ge-
bunden hatte; viele Leute waren um sie und hielten sie mit Ge-
walt fest, aber man konnte ihr kein Blut entziehen. Ich befahl,
daß man sie frei mache und ruhig lasse, denn sie konnten dem
Geiste, der sie plagte, nicht beikommen; sie machten sie srei und
ez trieb mich, zu ihr zu reden und sie im Namen dez Herrn still
und ruhig sein zu heißen, und sie war etz; die Krast dez Herrn
beruhigte ihr Gemüt und sie genaö, und sie nahm die Wahrheit
auf und blieb darin biß zu ihrem Tod. Des Herrn Name wurde
oerherrlichet, ihm gebührt die Ehre aller seiner Werke ....
Während ich in Manßfield Woodhouse war, trieb es mich,


% \picinclude{./020-029/p_s029.jpg} 
Der Tumnli in Nottingham. Wachsender Widerstand usn-. 29
ins Turmhaus zu gehen, um den Leuten die Wahrheit zu ver-
künden, aber das Volk fiel in großem Zorn über mich her, sie
schlugen mich zu Boden und erstickten mich fast; ich war arg zer-
schlagen und zerquetscht von ihren Händen, Bibeln und Stöcken.
Dann schleppten sie mich hinaus, wie wohl ich kaum fähig war
zu stehen, und taten mich in den Stock, wo ich einige Stunden
saß. Sie brachten Hundepeitschen und Pserdepeitschen und drohten
mir damit. Dann mußte ich vor die Behörden im Hause eines
Adligen, wo viele angesehene Leute zugegen waren. Als diese
sahen, wie ich mißhandelt worden war, gaben sie mir nach
Vielen Drohungen die Freiheit. Aber der Pöbel trieb mich
zur Stadt hinaus zum Dank dafür, daß ich ihnen das Wort des
Lebens verkündet hatte. Jch war kaum imstande zu stehen und
zu gehen, so übel hatten ste mich zugerichtet. Mit großer An-
strengung ging ich etwa eine Meile weit vor die Stadt, wo ich
Leute traf, die mir etwas zur Grquickung gaben, denn ich war
innerlich ganz auseinander, aber die Kraft des Herm heilte mich
bald wieder. Gs waren aber an dem Tage etliche von der Wahr-
heit des Herrn überzeugt worden, worüber ich mich freute. . . .
An einem Ersten Tage kamen wir nach Bagworth und gingen
ins Turmhaus, wohin einige der Freunde gebracht worden waren;
das Volk schloß sie darin ein und sich selbst mitsamt ihrem
Priester. Als der Priester fertig geredet hatte, machten sie die
Türe auf und wir gingen auch hinein und hatten einen Gottes-
dienst mit ihnen, und hernach hatten wir eine Versammlung in
der Stadt, mit manchen angesehenen Leuten. Als ich weiter zog,
hörte ich von solchen, die in Coventry um ihres Glaubens willen
gefangen waren. Aber als ich unterwegs zu ihrem Gefängnis war,
geschah das Wort des Herrn zu mir: ,,Meine Liebe war immer
mit dir und du bist in meiner Liebe«. Und ich fühlte mich ge-
hoben in der Liebe Gottes und sehr gestärkt an meinem innern
Menschen. Als ich in den Kerker zu den Gefangenen kam, über-
kam mich eine große Finsternis; ich hielt stille, denn mein Geist
ruhte in der Liebe Gottes. Schließlich singen die Gefangenen
an zu prahlen, und lärmten und lästerten, worüber meine
Seele sehr betrübt wurde. Sie sagten, daß sie Gott seien, aber
wir konnten solches nicht ertragen. Als sie ruhig geworden
waren, stand ich auf und fragte sie, ob sie solches aus innerem
Trieb oder auf Grund der Schrift täten? Sie sagten: ,,auf


% \picinclude{./030-039/p_s030.jpg} 
Grund der Schrift.'' Da eine Bibel zur Hand war, hieß ich sie,
mir die betreffende Stelle zu zeigen, und sie zeigten mir die Stelle,
wo daß Tuch vor Petrus herabgelassen wurde und die Stimme
sagte: ,,WaS Gott gereiniget hat, daö mache du nicht gemein''
(Act. 10, 15). A15 ich ihnen zeigte, daß diese Stelle nichtß für
sie beweise, brachten sie eine.andere vor, die davon handelte, wie
Gott alle mit sich selbst versöhnt im Himmel und auf Erden
(Col. 1, 20). Jch sagte ihnen, daß ich diese Stelle ebenfallö an-
erkenne, daß sie aber ebensowenig sür sie passe. Alß ich nun
vernahm, wie sie sagten, sie seien Gott, fragte ich sie, ob sie
wissen, ob ez morgen regnen werde? Sie antworteten, daß sie
daß nicht sagen könnten. Jch erwiderte ihnen: Gott könne das
sagen. Darauf fragte ich sie, ob sie immer so bleiben würden,
wie sie jetzt seien, oder ob sie sich ändern würden? Sie ant-
worteten: sie wüßten ez nicht.'' Jch erwiderte: ,,Gott kann es-
sagen und Gott verändert sich nicht. Jhr sagt, ihr seid Gott und
wißt nicht, ob ihr euch verändert oder nicht?« Sie wurden ver-
wirrt und für den Augenblick fast überwunden. Nachdem ich sie
wegen ihrer Gotteßlästerungen zurecht gewiesen hatte, ging ich
fort, demr ich merkte, daß sie Ranter 1) waren. Jch war nie
mit solchen zuvor zusammengetrosfen und ich priez die Güte des
Herrn, daß sie mir erschienen war, ehe ich zu ihnen gekommen
war. Nicht lange nachher schrieb einer dieser Ranterö, namenß
Joseph Salmon, ein Buch, in dem er widerrief, worauf sie die
Freiheit erhielten ....
Bei meinem Herumziehen auf den Jahrmiirkten und Märkten
und in den Städten, sah ich Tod und Finsterniö in allen, welche
die Kraft dez Herrn nicht ergriffen hatte. Alß ich durch Leieestershire
zog, kam ich nach Twy-Croß; daselbst waren Steuereinnehmer.
Der Herr trieb mich zu ihnen zu gehen und sie zu ermahnen,
sich vor Unterdrückung der Armen zu hüten. Das machte den
Leuten einen großen Eindruck. GS war in jener Stadt ein ange—
sehener Mann, welcher lange krank gewesen war und von den
Arzten aufgegeben wurde; und etliche Freunde aus- der Stadt
wünschten, daß ich zu ihm gehe. Jch ging zu ihm hinauf in sein
Zimmer und sagte ihm das Wort deß Lebens, und es trieb mich,
1) Runter, eine Sekte von mystischen Schwürmern, die sich rtihmten,
daß Christus in ihnen wohne, aus ihnen rede und sie selbst Christuö seien;
daher der Spottname ,,Ranter«——Prahler.


% \picinclude{./030-039/p_s031.jpg} 
Der Turnnlt in Nottingham. Wachsender Widersnmd usw. 31
mit ihm zu beten. Und der Herr erhörte uns und machte ihn
gesund. Als ich aber darauf in einem untern Raum des Hauses
zu der Dienerschaft und einigen andern Anwesenden redete, stürzte
einer aus einem Nebengemach herein mit dem nackten Degen in
der Hand, gerade auf mich los?-. Jch sah ihn unerschrocken an
und sagte: ,,Wehe Dir, arme Kreatur, was willst Du tun mit
Deiner fleischlichen Waffe? mir ist sie nicht mehr als ein Stroh-
halm.« Die Anwesenden waren sehr bestürzt und er entfernte
sich in Zorn und Wut. Alß sein Herr davon hörte, entließ er
ihn auö feinem Dienst. Also beschützte mich der Herr und half
diesem Schwachen und er wurde später den ,,Freunden« sehr
zugetan; und alö ich wieder in jene Stadt kam, besuchte er mich
mit seinem Weibe .....
Als ich nach Derbi) kam, wohnte ich im Haufe eineß Arztes?-;
eine Frau wurde gewonnen und noch Viele andere. Alß ich in
mein Zimmer ging, läutete die Glocke deß Turmhausej-’; nur schon
sie zu hören, ging mir durch Mark und Bein; ich fragte warum
die Glocke läute? man sagte mir, daß an dem Tage eine große
gotteödienstliche Versammlung stattfinde, dazu viele auß dem
Heer, sowie Priester und Prediger kommen werden. Da trieb
eß mich, auch hin zu gehen; und alö sie fertig waren, redete ich
zu ihnen, wa?. mir der Herr eingab. Sie waren ziemlich ruhig;
aber eine Wache kam, nahm mich bei der Hand und sagte, ich
müsse vor den Rat sowie auch die andern beiden, die mit mir
waren. Um die erste Nachmittags:-’stunde wurde ich-vorgenommen.
Jch wurde gefragt, warum ich hingegangen sei. Jch sagte, Gott
habe mich getrieben, es zu tun, und weiter sagte ich. ,,Gott
wohnet nicht in Tempeln mit Händen gemacht.« Jch sagte ihnen
ferner, all ihr Predigen, ihr Taufen und ihr Opfern werde sie
nie heiligen, und ermahnte sie, auf Ehristum in ihnen zu schauen
und nicht aus Menschen; denn Christuö sei es, welcher sie heilige.
Darauf ergingen sie sich in Vielen Worten, aber ich sagte ihnen,
sie sollten sich nicht über Gott und Christus streiten, sondern
ihm gehorchen. Die Kraft Gotteß donnerte unter ihnen und
sie zerstoben davor wie Spreu. Sie hießen mich mehrmalö
aus- dem Zimmer gehen und dann wieder hereinkommen und
trieben mich hin und her; von ein Uhr an bis abends neun ver-
hörten sie mich. Zuweilen sagten sie mir mit höhnischen Worten,
ich sei nicht bei Sinnen. Zuletzt fragten sie mich, ob ich geheiligt


% \picinclude{./030-039/p_s032.jpg}
sei; ich antwortete: »Ja, denn ich war im Paradies- Gotte?-,«
(2. Cor. 12, 4). Dann fragten sie mich, ob ich keine Sünde habe.
Jch antwortete: ,,ChristuZ, mein Erlöser, hat die Sünde von mir
genommen und in ihm ist keine Sünde.« Sie fragten, wie ich
wüßte, daß Christus in uns wohne? Jch sagte: »Durch seinen
Geist, den er unctz gegeben.« Um mich zu versuchen, fragten sie,
ob einer oon uns Christus sei? Jch antwortete: »Nein, wir
sind nichts, Christuö ist alle?-.« Sie sagten: wenn ein Mann
stehle, ob daß keine Sünde sei? Jch antwortete: »AlleS Unrecht
ist Stinde.« Alß sie ez nun müde geworden, mich zu verhören,
verurteilten sie mich zu sechß Monaten im Korrektionßhauö in
Derby alö Gotteölästerer, wie auS folgendem Verhastbesehl zu
ersehen ist:
An den Oberaufseher des Korrektionöhauseß in Derby.
,,Hiemit senden wir euch die Personen George Fox, oormalß
in Manßfield in der Grafschaft Nottingham, und John Fretwell,
Landwirt, vormalß in Stanießby in der Grafschaft Derby, vor
uns gebracht am heutigen Tag und beschuldigt eingestandener
Äußerungen verschiedener gotteölästerlicher Ansichten, die einem
jüngst verfaßten Parlament?-beschluß1) zuwider sind; sie sollen daher
sogleich nach Einsicht Dieses aufgenommen werden, besagter
George Fox und Johann Fretwell, in euern Gewahrsam und
darin sicher verwahrt werden, für die Dauer oon 6 Monaten,
ohne Möglichkeit einer Bürgschaft oder Abkürzung, es wäre denn,
daß sie sich htulttnglich durch ein gutes Betragen auöweisen, oder
durch unsere eigene Verordnung frei würden. Solcheö zutun
möget ihr nicht versäumen.
Mit unsrer Hand und Siegel gegeben am heutigen Tage
:30. Oktober 1650. Ger. Bennet. «
Nath. Barton.« .....
Während ich im Gefängnis war, kamen oft ,,Fromme«, um eine
Unterredung mit mir zu haben; noch ehe sie etwas sagten, merkte
ich immer, daß sie kamen, um für die bleibende Sündhaftigkeit und
Unoollkommenheit einzutreten. Jch fragte sie, ob sie glänbig seien und
11 Partamentöbeschluß vom 2. Mai 1648 gegen GotteHläster11ng und
Ketzerei. Ein Beschluß, der von der unglaublichen Härte der damals regierenden
Pre-zbyterianer zeugt.


% \picinclude{./030-039/p_s033.jpg} 
Der Tnmult in Nottingham. Wachsender Widerstand usw. Z3
Glauben hätten? Sie sagten: ,,Ja.« Jeh fragte sie: in wen?
Sie sagten: ,,J«n Christu-3.« Jch erwiderte: Wenn ihr wahre
an Christus- Glaubende seid, so seid ihr vom Tode zum Leben
eingegangen, und wenn ihr vom Tode frei seid, dann seid ihr ez
auch von der Sünde, die den Tod bringt. Und wenn euer
Glaube wahr ist, so wird er euch den Sieg geben über Sünde
und—Tc-ufel und eure Herzen und Gewissen reinigen — denn der
wahre Glaube ist in reinen Gewissen (1 Tim. 3) und er wird
machen, daß ihr Gott gefallet und euch wieder Zugang zu ihm
Verschaffen.« Aber sie wollten nicht von Reinheit und von Sieg
über Sünde und Teufel hören; denn sie sagten, sie können nicht
glaubeny daß jemand könne frei von Sünde sein schon dießseitß
des Grabe-3. Jch hieß sie, da-Z Schwatzen über die Schrift, die
das Wort heiliger Männer sei, aufgeben, wenn sie für Unheiligkeit
eintreten wollten. Einmal kam auch eine Anzahl solcher ,,Frommer«
zu mir und fingen an, die Sündhaftigkeit zu befürworten. Jch
fragte sie: ob sie Hoffnung hätten? ,,Ja, ja! daß wäre, wenn
wir keine Hoffnung hötten!« Jch fragte sie: ,,Waö für eine
Hoffnung ist etz, die ihr habt? Jst Christus in euch die Hoffnung
eurer Herrlichkeit? (Col. 1, 27.) Reinigt sie euch, gleich wie er
rein ist?« Aber sie wollten nichtö davon hören, «daß sie selber
hienieden schon rein werden sollten. Darauf gebot ich ihnen,
nicht mehr über die Schrift zu reden, welche daß Wort heiliger
Männer sei. Denn die heiligen Männer, welche die Schrift ge-
schrieben haben, seien für Heiligkeit in Herz, Leben und Wandel
hienieden eingetreten. ,,Jhr aber«, sagte ich, ,,tretet für Unreinheit
und Sünde ein, die vom Teufel sind, wa-3 habt ihr zu schaffen
mit den Worten heiliger Männer?«
Der Kerkermeister, ein großer ,,Frommer«, hatte eine schreck-
liche Wut auf mich und redete sehr schlecht von mir. Aber etz
gefiel dem Herm, ihn eines Tageö so mächtig zu ergreifen, daß
er in großer Angst und innerer Not war. Alö ich in meinem
Zimmer umherlief, hörte ich klägliche Laute und hörte, wie er zu
seiner Frau sagte: ,,Frau, ich habe den Tag des Gerichts gesehen,
und George Fox war da, und ich hatte Angst vor ihm, weil ich
ihm so viel böseß zugefügt hatte und so vieles wider ihn zu den
Vorgesetzten und ,,Frommen« gesagt hatte und zu den Richtern
und in den Wirt?-häusern.« Hierauf kam er gegen Abend zu mir
ins Zimmer und sagte: »Jch bin gegen euch gewesen wie ein
George Fox- 3


% \picinclude{./030-039/p_s034.jpg} 
Löwe; nun aber komme ich wie ein Lamm und wie der Kerker-
meister, der zitternd zu Pauluß und Silaß kam.« Und er bat,
daß er bei mir bleiben dürfe. Jch sagte, ich sei in seiner Macht
und er könne mit mir machen, waß er wolle; aber er sagte:
nein, er wolle meine Grlaubniß haben, und er möchte, daß er
immer mit mir sein könnte, aber nicht mich alß Gefangenen haben;
er und sein Hauß seien meinetwegen geplagt gewesen. Jch erlaubte
ihm denn, bei mir zu sein, und er öfsnete mir sein Herz rmd
sagte, er glaube, daß daß, waß ich vom wahren Glauben und
von der wahren Hoffnung sage, wahr sei, und er wunderte sich,
daß der andere, der mit mir gefangen war, nicht dabei bleibe.
Gr sagte: ,,Jener andere tat unrecht, ihr aber seid ein Gerechter.«
Gr gestand mir auch, daß oft, wenn ich ihn gebeten hatte, mich
unter daß Volk gehen zu lassen, um ihnen daß Wort deß Herrn
zu verkünden, und er eß mir verweigert habe, habe er sich damit
eine große Last auferlegt; denn er sei in große Angst geraten
und einige Zeit ganz verstört und niedergedriickt gewesen, so daß
er gar keine Kraft mehr gehabt habe. Am Morgen ging er fort
und ging zu den Richtern und sagte ihnen, wie er und sein Hauß
meinetwegen geplagt gewesen seien, und einer der Richter erwiderte
ihm, daß auch sie geplagt seien, darum daß sie mich sesthielten.
Eß war Richter Bennet zu Derby, welcher un-3 zuerst Quäkers)
genannt hatte, weil ich ihnen gesagt hatte, gsie müßten erzittem
vor dem Wort Gotteß. Solcheß geschah im Jahre 1650.
Hierauf erlaubten mir die Richter, eine Meile weit zu gehen.
Joh sah, wo sie hinauß wollten und sagte dem Kerkermeister,
wenn sie mir zeigen wollten, wie weit eine Meile sei, so wolle
ich manchmal so weit gehen; denn ich glaube, sie dachten, ich
würde davon laufen. Und der Kerkermeister gestand nachher,
daß sie eß in dieser Absicht gestattet hätten, damit ich entkomme
und sie von ihrer Angst befreit würden; aber ich sagte ihm, daß
ich nicht diesen Geist habe. «
Dieser Kerkermeister hatte eine Schwester, ein kränklicheß
jungeß Weib. Sie kam zu mir, um mich zu besuchen; und nach-
dem sie einige Zeit bei mir gewesen war, und ich Worte der
Wahrheit zu ihr geredet hatte, ging sie hinunter und sagte den
1) Quüker, das heißt ,,Zitterer«, der Spottname, den die Gegner den
Freunden anhängten, wegin der in ihren ersten Versammlungen sich einstellendea
Konvulsionen.


% \picinclude{./030-039/p_s035.jpg} 
Erlebnisse im Gefängnis zu Derbi) usw. Z5
andern, wir seien unschuldige Leute und täten niemand nichts zu
leide, sondern allen nur Gutes, sogar solchen, die uns haßten,
und bat sie, freundlich gegen mich zu sein. ....
Während ich im Korrektionshaus war, besuchten mich meine
Verwandten, und da sie über meine Gefangenschaft bekümmert
waren, gingen sie zu den Richtern und baten sie, daß ich mit
ihnen heim gehen dürfe. Sie erboten sich, sich mit hundert Pfund
zu verbiirgen und einige andere aus Derby, die mit ihnen waren,
je mit siinszig Psund, daß ich nicht mehr dorthin komme, um
gegen die Ptiestet zu reden. So wurde ich vor die Richter
gebracht, und weil ich nicht einwilligen wollte, daß irgendjemand
sich meinetwegen verpftichte, — denn ich war ja keines Vergehens
schuldig und hatte das Wort des Lebens und der Wahrheit ge-
redet, — erhob sich Richter Bennet zornig, und als ich niederkniete, um
Gott zu bitten, ihm zu vergeben, rannte er auf mich los und schlug
mich mit beiden Händen und schrie: »Fort mit ihm! Kerkermeister,
nimm ihn fort!« Hierauf wurde ich wieder in den Kerker gebracht
und mußte dort bleiben, bis meine Zeit von sechs Monaten um
war. Aber ich durfte nun eine Meile weit allein gehen, was ich
tat, als ich fühlte, daß ich es durfte. Ost ging ich auf den Markt
und in die Straßen und ermahnte die Leute, sich von ihrer
Schlechtigkeit zu bekehren, und ging dann wieder ins Gefängnis.
Und da Leute von allerleiislteligionen mit mir im Gefängnis
waren, ging ich hie und da zu ihnen und wohnte ihren Versamm-
lungen an den Ersten Tagen bei ....