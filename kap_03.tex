%%%%%%%%%%%%%%%%%%% Kapitel 3. %%%%%%%%%%%%%%%%%%%%%%%%%%%%%%
\chapter[Tumult in Nottingham]{Tumult in Nottingham}

\begin{center}
\textbf{Der Tumult in Nottingham. Wachsender Widerstand, bis zum
Gefängnis in Derby.}
\end{center}

\section{Tumulte bei dem Gottesdienst in Nottingham}
Als  ich einmal am Morgen eines Ersten Tages in der Nähe
von Nottingham\ort{Nottingham} von einem Hügel aus die 
Stadt überblickte, da
gewahrte ich das riesige Turmhaus, und der Herr sagte zu mir:
\zitat{Du musst hingehen und gegen jene großen Götzen schreien und
gegen die, welche drinnen anbeten}. Ich sagte den \textit{Freunden},
die mit mir waren, nichts davon, sondern ging mit ihnen hin in
die Versammlung, wo die mächtige Kraft des Herrn mit uns
war; hier lies ich sie und ging zum Turmhaus. Die Menge,
die ich hier sah, kam mir vor wie ein Brachfeld und der Priester
wie ein großer Erdklumpen, der oben auf seiner Kanzel stand.
Er hatte zum Text die Worte des Petrus: \zitat{Wir haben ein festes
prophetisches Wort und ihr tut wohl, das ihr darauf achtet, als
auf ein Licht, das da scheinet an einem dunkeln Ort, bis der Tag
anbreche und der Morgenstern aufgehe in eueren Herzen}
(2. Petr. 1:19\bibel{Petr. 2. 01:19@2. Petr. 1:19}). 
Er sagte den Leuten, nach dem, was hier geschrieben 
stehe, sollten sie alle Lehren, Bekenntnisse und Meinungen
prüfen. Da kam die Kraft des Herrn so mächtig über mich und
war so stark in mir, das ich nicht an mich halten konnte, sondern
rufen\index{Gottesdienst!Störung} musste: \zitat{O 
nein, nicht nach dem, was geschrieben stehet!}\index{Exegese}
und ich sagte ihnen, nach was: nämlich nach dem heiligen Geist,
durch den die heiligen Männer Gottes die Schrift geschrieben
haben. Durch diesen, sagte ich, müssen alle Lehren, Bekenntnisse
und Meinungen geprüft werden. Dieser Geist leitet in alle
% \picinclude{./020-029/p_s027.jpg} 
Wahrheit und zur Erkenntnis aller Wahrheit. Die 
Juden\index{Juden} haben
die Schrift gehabt und widerstanden dem heiligen Geist doch und 
verwarfen Christus, den schönen Morgenstern ; sie verfolgten Christus
und seine Apostel und wollten ihre Lehren nach der Schrift prüfen;
aber sie irrten in ihrem Urteil und prüften sie nicht richtig, weil
sie ohne den heiligen Geist prüften. Da ich nun so zu ihnen
redete, kamen die Wachen und führten mich weg und brachten
mich in einen wüsten, stinkenden Kerker; der Geruch stieg mir so
in die Nase und den Hals, das es eine Qual war, aber die
Kraft des Herrn schallte an dem Tage so in ihren Ohren, das
sie ganz von dem Schall betäubt waren, und ihre Ohren wurden
noch eine zeitlang nicht frei davon, so waren sie im Turmhause
von der Kraft des Herrn ergriffen worden. 


\section{Bekehrung des Sheriff John Neckles}

Am Abend brachten
sie mich vor die Behörden der Stadt; als ich vor sie trat, war
der Bürgermeister in verdrieslicher, mürrischer Laune, aber die
Kraft des Herrn beschwichtigte ihn. Sie verhörten mich 
ausführlich und ich berichtete ihnen, wie der Herr mich 
getrieben hatte
zu kommen. Nach einigem Hin- und Herreden schickten sie mich
ins Gefängnis zurück. Aber bald darauf lies mich der Ober-Sheriff,
John Neckles\person{Neckles, John (Sheriff)}, zu sich in 
sein Haus holen. Als ich eintrat, begegnete mir sein Weib 
im Flur und sagte: \zitat{Unserm Hause ist
Heil widerfahren.} Sie reichte mir die Hand und war mächtig
ergriffen von der Kraft Gottes, und ihr Mann und ihre Kinder
und Dienstboten wurden ganz umgewandelt, denn die Kraft des
Herrn war mächtig in ihnen. Ich wohnte bei ihnen und wir
hatten große Versammlungen in ihrem Hause; es kamen auch
etliche angesehene Standespersonen, und des Herrn Kraft tat sich
mächtig kund unter ihnen; John 
Reckles\person{Reckles, John} lies dann einen andern
Sheriff holen und eine Frau, mit der sie in Geschäften zu tun
gehabt hatten, und erklärte in Anwesenheit des andern Sheriff,
das sie beide diese Frau bei einem Handel geschädigt hätten und
sie entschädigen müssten. Er sagte es sehr freundlich, aber der
andere Sheriff leugnete, und die Frau sagte, sie wisse nichts 
davon. Aber der gerechte Sheriff sagte, es sei so, und der andere
wisse das ganz gut; nachdem er die Sache aufgedeckt und das
Unrecht, das sie getan, eingestanden hatte, entschädigte er die
Frau und ermahnte den andern ein gleiches zu tun; die Kraft
Gottes war mit diesem guten Sheriff und wirkte eine große
Wandlung in ihm und er hatte große Offenbarungen. Als er
% \picinclude{./020-029/p_s028.jpg} 
am darauf folgenden Marktag in den Pantoffeln in seinem
Zimmer auf- und abging, sagte er: \zitat{Ich muss auf den Markt
gehen und den Leuten Buße predigen,} und er ging auf den Markt
und in mehrere Straßen und predigte den Leuten Buße; und auch
noch andere aus der Stadt trieb es, zu den Behörden zu gehen
und die Leute zur Buße zu ermahnen. Die Räte wurden sehr
böse über mich und ließen mich aus dem Hause des Sheriff
holen und verurteilten mich zum Gefängnis. Als die 
Gerichtssitzung stattfand, fühlte einer sich 
getrieben, sich statt meiner anzubieten, \zitat{Leib 
um Leib, Leben um Leben}. Als ich vor den
Richter gebracht werden sollte, ging es ziemlich lang, bis mich
der Diener, der mich hinbringen sollte, abholte, und als ich kam,
hatte sich der Richter schon erhoben, woraus ich sah, das er 
erzürnt war; er sagte, er wolle dem Jüngling schon einen Verweis
geben, wenn er vor ihn gebracht werde; ich war damals unter
dem Namen \zitat{Jüngling} eingesperrt. Ich wurde denn wieder
ins Gefängnis gebracht. Die Kraft des Herrn war mächtig
unter den \textit{Freunden}, aber das Volk fing an, tätlich zu
werden, so das der Schloskommandant Soldaten hinaus schickte,
um die Leute auseinander zu treiben, worauf es ruhig wurde;
alle, Priester und Volk, erstaunten ob der herrlichen Kraft, welche
hervorbrach, und etliche der Priester wurden empfänglich gemacht
und einige von ihnen bekannten sich zur Kraft Gottes.

\section{Krankenheilung, Predigt und Misshandlung in Woodhouse}

Nachdem ich aus dem Gefängnis von Nottingham, wo ich
einige Zeit gefangen gewesen war, entlassen worden, zog ich
umher, wie vorher im Dienst des Herrn. Als ich 
nach Mansfield\ort{Mansfield}
Woodhouse\ort{Woodhouse} kam, war dort eine verrückte 
Frau\index{Wahnsinn}; das Haar hing
ihr wirr über die Ohren und der Arzt war gerade bei ihr. Er
war daran, ihr zu Ader zu lassen, nachdem man sie zuvor 
gebunden hatte; viele Leute waren um sie und hielten sie mit 
Gewalt fest, aber man konnte ihr kein Blut entziehen. Ich befahl,
das man sie frei mache und ruhig lasse, denn sie konnten dem
Geiste, der sie plagte, nicht beikommen; sie machten sie frei und
es trieb mich, zu ihr zu reden und sie im Namen des Herrn still
und ruhig sein zu heißen, und sie war es; die Kraft des Herrn
beruhigte ihr Gemüt und sie genas\index{Heilung}, und sie nahm die Wahrheit
auf und blieb darin bis zu ihrem Tod. Des Herrn Name wurde
verherrlichet, ihm gebührt die Ehre aller seiner Werke [...].


Während ich in Mansfield Woodhouse war, trieb es mich,
% \picinclude{./020-029/p_s029.jpg} 
ins Turmhaus zu gehen, um den Leuten die Wahrheit zu 
verkünden, aber das Volk fiel in großem Zorn über mich her, sie
schlugen mich zu Boden und erstickten mich fast; ich war arg 
zerschlagen und zerquetscht von ihren Händen, Bibeln und 
Stöcken.\index{Misshandlung}
Dann schleppten sie mich hinaus, wie wohl ich kaum fähig war
zu stehen, und taten mich in den Stock, wo ich einige Stunden
saß. Sie brachten Hundepeitschen und Pferdepeitschen und drohten
mir damit. Dann musste ich vor die Behörden im Hause eines
Adligen, wo viele angesehene Leute zugegen waren. Als diese
sahen, wie ich misshandelt worden war, gaben sie mir nach
Vielen Drohungen die Freiheit. Aber der Pöbel trieb mich
zur Stadt hinaus zum Dank dafür, das ich ihnen das Wort des
Lebens verkündet hatte. Ich war kaum imstande zu stehen und
zu gehen, so übel hatten sie mich zugerichtet. Mit großer 
Anstrengung ging ich etwa eine Meile weit vor die Stadt, wo ich
Leute traf, die mir etwas zur Erquickung gaben, denn ich war
innerlich ganz auseinander, aber die Kraft des Herrn heilte mich
bald wieder. Es waren aber an dem Tage etliche von der 
Wahrheit des Herrn überzeugt worden, worüber ich mich freute [...].

\section{Gefängnisbesuch in Coventry und erste Begegnung mit Ranters}

An einem Ersten Tage kamen wir nach Bagworth\ort{} und gingen
ins Turmhaus, wohin einige der Freunde gebracht worden waren;
das Volk schloss sie darin ein und sich selbst mitsamt ihrem
Priester. Als der Priester fertig geredet hatte, machten sie die
Türe auf und wir gingen auch hinein und hatten einen 
Gottesdienst mit ihnen, und hernach hatten wir eine Versammlung in
der Stadt, mit manchen angesehenen Leuten. Als ich weiter zog,
hörte ich von solchen, die in Coventry\ort{Coventry} um ihres Glaubens willen
gefangen waren. Aber als ich unterwegs zu ihrem Gefängnis war,
geschah das Wort des Herrn zu mir: \zitat{Meine Liebe war immer
mit dir und du bist in meiner Liebe}. Und ich fühlte mich 
gehoben in der Liebe Gottes und sehr gestärkt an meinem innern
Menschen. Als ich in den Kerker zu den Gefangenen kam, über-
kam mich eine große Finsternis; ich hielt stille, denn mein Geist
ruhte in der Liebe Gottes. Schließlich fingen die Gefangenen
an zu prahlen, und lärmten und lästerten, worüber meine
Seele sehr betrübt wurde. Sie sagten, das sie Gott seien, aber
wir konnten solches nicht ertragen. Als sie ruhig geworden
waren, stand ich auf und fragte sie, ob sie solches aus innerem
Trieb oder auf Grund der Schrift täten? Sie sagten: \zitat{auf
% \picinclude{./030-039/p_s030.jpg} 
Grund der Schrift.} Da eine Bibel zur Hand war, hieß ich sie,
mir die betreffende Stelle zu zeigen, und sie zeigten mir die Stelle,
wo das Tuch vor Petrus herabgelassen wurde und die Stimme
sagte: ,\zitat{Was Gott gereinigt hat, das mache du nicht gemein}
(Act. 10:15\bibel{Act. 10:15}). Als ich ihnen zeigte, das diese Stelle nichts für
sie beweise, brachten sie eine andere vor, die davon handelte, wie
Gott alle mit sich selbst versöhnt im Himmel und auf Erden
(Col. 1:20\bibel{Col. 01:20@Col. 1:20}). Ich sagte ihnen, 
das ich diese Stelle ebenfalls anerkenne, das sie aber 
ebensowenig für sie passe. Als ich nun
vernahm, wie sie sagten, sie seien Gott, fragte ich sie, ob sie
wissen, ob es morgen regnen werde? Sie antworteten, das sie
das nicht sagen könnten. Ich erwiderte ihnen: Gott könne das
sagen. Darauf fragte ich sie, ob sie immer so bleiben würden,
wie sie jetzt seien, oder ob sie sich ändern würden? Sie 
antworteten: sie wüsten es nicht. Ich erwiderte: \zitat{Gott 
kann es sagen und Gott verändert sich nicht. Ihr sagt, ihr seid Gott und
wisst nicht, ob ihr euch verändert oder nicht?} Sie wurden 
verwirrt und für den Augenblick fast überwunden. Nachdem ich sie
wegen ihrer Gotteslästerungen zurecht gewiesen hatte, ging ich
fort, denn ich merkte, das sie 
Ranter\index{Ranter}\footnote{Ranter, eine Sekte von mystischen 
Schwärmern, die sich rühmten, das Christus in ihnen wohne, 
aus ihnen rede und sie selbst Christus seien; daher der 
Spottname \zitat{Ranter} -- Prahler.} waren. Ich war nie
mit solchen zuvor zusammengetroffen und ich pries die Güte des
Herrn, das sie mir erschienen war, ehe ich zu ihnen gekommen
war. Nicht lange nachher schrieb einer dieser Ranters, namens
Joseph Salmon\person{Salmon, Joseph}, ein Buch, in dem er 
widerrief, worauf sie die Freiheit erhielten ]...].

\section{Ermahnung der Steuereinnehmer in Leicestershire}

Bei meinem Herumziehen auf den Jahrmärkten\index{Jahrmarkt} und Märkten
und in den Städten, sah ich Tod und Finsternis in allen, welche
die Kraft des Herrn nicht ergriffen hatte. Als ich 
durch Leicestershire\ort{Leicestershire}
zog, kam ich nach Twycross\ort{Twycross}; daselbst waren Steuereinnehmer.
Der Herr trieb mich zu ihnen zu gehen und sie zu ermahnen,
sich vor Unterdrückung der Armen zu hüten. Das machte den
Leuten einen großen Eindruck. Es war in jener Stadt ein 
angesehener Mann, welcher lange krank gewesen war und von den
Ärzten aufgegeben wurde; und etliche Freunde aus der Stadt
wünschten, das ich zu ihm gehe. Ich ging zu ihm hinauf in sein
Zimmer und sagte ihm das Wort des Lebens, und es trieb mich,
% \picinclude{./030-039/p_s031.jpg} 
mit ihm zu beten. Und der Herr erhörte uns und machte ihn
gesund\index{Krankenheilung}. Als ich aber darauf 
in einem unteren Raum des Hauses
zu der Dienerschaft und einigen andern Anwesenden redete, stürzte
einer aus einem Nebengemach herein mit dem nackten Degen in
der Hand, gerade auf mich los. Ich sah ihn unerschrocken an
und sagte: \zitat{Wehe Dir, arme Kreatur, was willst Du tun mit
Deiner fleischlichen Waffe? mir ist sie nicht mehr als ein 
Strohhalm.} Die Anwesenden waren sehr bestürzt und er entfernte
sich in Zorn und Wut. Als sein Herr davon hörte, entließ er
ihn aus feinem Dienst. Also beschützte mich der Herr und half
diesem Schwachen und er wurde später den \textit{Freunden} sehr
zugetan; und als ich wieder in jene Stadt kam, besuchte er mich
mit seinem Weibe [...].

\section{Gefangennahme in Derby}

Als ich nach Derby\ort{Derby} kam, wohnte ich im Hause eines Arztes.
Eine Frau wurde gewonnen und noch Viele andere. Als ich in
mein Zimmer ging, läutete die Glocke des Turmhauses; nur schon
sie zu hören, ging mir durch Mark und Bein; ich fragte warum
die Glocke läute? man sagte mir, das an dem Tage eine große
gottesdienstliche Versammlung stattfinde, dazu viele aus dem
Heer, sowie Priester und Prediger kommen werden. Da trieb
es mich, auch hin zu gehen; und als sie fertig waren, redete ich
zu ihnen, was. mir der Herr eingab. Sie waren ziemlich ruhig;
aber eine Wache kam, nahm mich bei der Hand und sagte, ich
müsse vor den Rat sowie auch die andern beiden, die mit mir
waren. Um die erste Nachmittagsstunde wurde ich vorgenommen.
Ich wurde gefragt, warum ich hingegangen sei. Ich sagte, Gott
habe mich getrieben, es zu tun, und weiter sagte ich. \zitat{Gott
wohnet nicht in Tempeln mit Händen gemacht.} Ich sagte ihnen
ferner, all ihr Predigen, ihr Taufen\index{Taufe} und ihr Opfern werde sie
nie heiligen, und ermahnte sie, auf Christum in ihnen zu schauen
und nicht aus Menschen; denn Christus sei es, welcher sie heilige.
Darauf ergingen sie sich in Vielen Worten, aber ich sagte ihnen,
sie sollten sich nicht über Gott und Christus streiten, sondern
ihm gehorchen\index{Ökumene}. Die Kraft Gottes donnerte unter ihnen und
sie zerstoben davor wie Spreu. Sie hießen mich mehrmals
aus dem Zimmer gehen und dann wieder hereinkommen und
trieben mich hin und her; von ein Uhr an bis abends neun 
verhörten sie mich. Zuweilen sagten sie mir mit höhnischen Worten,
ich sei nicht bei Sinnen. Zuletzt fragten sie mich, ob ich 
geheiligt\index{Heiligung}
% \picinclude{./030-039/p_s032.jpg}
sei; ich antwortete: \zitat{Ja, denn ich war im Paradies\index{Paradies} Gottes}
(2. Cor. 12:4\bibel{Cor. 2. 12:04@2. Cor. 12:4}). Dann fragten 
sie mich, ob ich keine Sünde\index{Sünde} habe.
Ich antwortete: \zitat{Christus, mein Erlöser, hat die Sünde von mir
genommen und in ihm ist keine Sünde.} Sie fragten, wie ich
wüste, das Christus in uns wohne? Ich sagte: \zitat{Durch seinen
Geist, den er uns gegeben.} Um mich zu versuchen, fragten sie,
ob einer von uns Christus sei? Ich antwortete: \zitat{Nein, wir
sind nichts, Christus ist alles.} Sie sagten: wenn ein Mann
stehle, ob das keine Sünde sei? Ich antwortete: \zitat{Alles Unrecht
ist Sünde.} Als sie es nun müde geworden, mich zu verhören,
verurteilten sie mich zu sechs Monaten im Korrektionshaus in
Derby als Gotteslästerer, wie aus folgendem Verhaftbefehl zu
ersehen ist:

\grosszitat{
    An den Oberaufseher des Korrektionöhauses in Derby.
    \bigskip

    Hiermit senden wir euch die Personen George Fox, vormals
    in Mansfield in der Grafschaft Nottingham, und John Fretwell,
    Landwirt, vormals in Staniesby in der Grafschaft Derby, vor
    uns gebracht am heutigen Tag und beschuldigt eingestandener
    Äußerungen verschiedener gotteslästerlicher Ansichten, die einem
    jüngst verfassten Parlamentsbeschluss\footnote{Partamentsbeschlus 
    vom 2. Mai 1648\jahr{1648} gegen Gotteslästerung und Ketzerei. 
    Ein Beschluss, 
    der von der unglaublichen Härte der damals regierenden
    Presbyterianer\index{Presbyterianer} zeugt.} zuwider sind; 
    sie sollen daher
    sogleich nach Einsicht Dieses aufgenommen werden, besagter
    George Fox und Johann Fretwell\person{Fretwell, Johann}, in 
    euern Gewahrsam und
    darin sicher verwahrt werden, für die Dauer von 6 Monaten,
    ohne Möglichkeit einer Bürgschaft oder Abkürzung, es wäre denn,
    das sie sich hinlänglich durch ein gutes Betragen ausweisen, oder
    durch unsere eigene Verordnung frei würden. Solches zu tun
    möget ihr nicht versäumen.
    \bigskip
    \begin{flushright}
    Mit unsrer Hand und Siegel gegeben am heutigen Tage
    30.~Oktober~1650. Ger. Bennet. Nath. Barton.\end{flushright}
}

Während ich im Gefängnis war, kamen oft \textit{Fromme}, um eine
Unterredung mit mir zu haben; noch ehe sie etwas sagten, merkte
ich immer, das sie kamen, um für die bleibende 
Sündhaftigkeit\index{Sümdhaftigkeit} und 
Unvollkommenheit\index{Unvollkommenheit} einzutreten. Ich 
fragte sie, ob sie gläubig seien und
% \picinclude{./030-039/p_s033.jpg} 
Glauben hätten? Sie sagten: \zitat{Ja.} Ich fragte sie: in wen?
Sie sagten: \zitat{In Christus.} Ich erwiderte: \zitat{Wenn ihr wahre
an Christus Glaubende seid, so seid ihr vom Tode zum Leben
eingegangen, und wenn ihr vom Tode frei seid, dann seid ihr es
auch von der Sünde, die den Tod bringt. Und wenn euer
Glaube wahr ist, so wird er euch den Sieg geben über Sünde
und Teufel\index{Teufel} und eure Herzen und Gewissen 
reinigen -- denn der
wahre Glaube ist in reinen Gewissen 
(1 Tim. 3\bibel{Tim. 1. 03@1 Tim. 3}) und er wird
machen, das ihr Gott gefallet und euch wieder Zugang zu ihm
Verschaffen.} Aber sie wollten nicht von Reinheit und von Sieg
über Sünde und Teufel hören; denn sie sagten, sie können nicht
glauben das jemand könne frei von Sünde sein schon diesseits
des Grabes. Ich hieß sie, das Schwatzen über die Schrift, die
das Wort heiliger Männer sei, aufgeben, wenn sie für Unheiligkeit
eintreten wollten. Einmal kam auch eine Anzahl solcher \textit{Frommer}
zu mir und fingen an, die Sündhaftigkeit zu befürworten. Ich
fragte sie: ob sie Hoffnung hätten? \zitat{Ja, ja! das wäre, wenn
wir keine Hoffnung hätten!} Ich fragte sie: \zitat{Was für eine
Hoffnung ist es, die ihr habt? Ist Christus in euch die Hoffnung
eurer Herrlichkeit? (Col. 1:27\bibel{Col. 01:27@Col. 1:27}) 
Reinigt sie euch, gleich wie er
rein ist?} Aber sie wollten nichts davon hören, das sie selber
hienieden schon rein werden sollten. Darauf gebot ich ihnen,
nicht mehr über die Schrift\index{Bibel} zu reden, welche das Wort heiliger
Männer sei. Denn die heiligen Männer, welche die Schrift 
geschrieben haben, seien für Heiligkeit in Herz, Leben und Wandel
hienieden eingetreten. \zitat{Ihr aber}, sagte ich, \zitat{tretet für Unreinheit
und Sünde ein, die vom Teufel sind, was habt ihr zu schaffen
mit den Worten heiliger Männer?}


\section{Bekehrung des Kerkermeister und Besuch anderer Gottesdienste}

Der Kerkermeister, ein großer \textit{Frommer}, hatte eine 
schreckliche Wut auf mich und redete sehr schlecht von mir. Aber es
gefiel dem Herrn, ihn eines Tages so mächtig zu ergreifen, das
er in großer Angst und innerer Not war. Als ich in meinem
Zimmer umherlief, hörte ich klägliche Laute und hörte, wie er zu
seiner Frau sagte: \zitat{Frau, ich habe den Tag des Gerichts gesehen,
und George Fox war da, und ich hatte Angst vor ihm, weil ich
ihm so viel böses zugefügt hatte und so vieles wider ihn zu den
Vorgesetzten und Frommen gesagt hatte und zu den Richtern
und in den Wirtshäusern.} Hierauf kam er gegen Abend zu mir
ins Zimmer und sagte: \zitat{Ich bin gegen euch gewesen wie ein
% \picinclude{./030-039/p_s034.jpg} 
Löwe; nun aber komme ich wie ein Lamm und wie der 
Kerkermeister, der zitternd zu Paulus und Silas kam.} Und er bat,
das er bei mir bleiben dürfe. Ich sagte, ich sei in seiner Macht
und er könne mit mir machen, was er wolle; aber er sagte:
nein, er wolle meine Erlaubnis haben, und er möchte, das er
immer mit mir sein könnte, aber nicht mich als Gefangenen haben;
er und sein Haus seien meinetwegen geplagt gewesen. Ich erlaubte
ihm denn, bei mir zu sein, und er öffnete mir sein Herz und
sagte, er glaube, das das, was ich vom wahren Glauben und
von der wahren Hoffnung sage, wahr sei, und er wunderte sich,
das der andere, der mit mir gefangen war, nicht dabei bleibe.
Er sagte: \zitat{Jener andere tat unrecht, ihr aber seid ein Gerechter.}
Er gestand mir auch, das oft, wenn ich ihn gebeten hatte, mich
unter das Volk gehen zu lassen, um ihnen das Wort des Herrn
zu verkünden, und er es mir verweigert habe, habe er sich damit
eine große Last auferlegt; denn er sei in große Angst geraten
und einige Zeit ganz verstört und niedergedrückt gewesen, so das
er gar keine Kraft mehr gehabt habe. 

Am Morgen ging er fort
und ging zu den Richtern und sagte ihnen, wie er und sein Haus
meinetwegen geplagt gewesen seien, und einer der Richter erwiderte
ihm, das auch sie geplagt seien, darum das sie mich festhielten.
Es war Richter Bennet\person{Richter Bennet} zu Derby, 
welcher uns zuerst 
Quäkers\index{Quaker!nahmensentstehung}\footnote{Quüker, 
das heist \zitat{Zitterer}, der Spottname, den die Gegner den
Freunden abhängten, wegen der in ihren ersten Versammlungen 
sich einstellenden Konvulsionen. \textbf{Anmerkung Olaf Radicke:} Das 
ist eine original Fußnote, und offensichtlicher Unsinn! Da hier
G. Fox selbst den Grund des Namens nennt, der plausibler scheint.}
genannt hatte, weil ich ihnen gesagt hatte, sie müssten erzittern
vor dem Wort Gottes. Solches geschah im Jahre 1650\jahr{1650}.

Hierauf erlaubten mir die Richter, eine Meile weit zu gehen.
Ich sah, wo sie hinaus wollten und sagte dem Kerkermeister,
wenn sie mir zeigen wollten, wie weit eine Meile sei, so wolle
ich manchmal so weit gehen; denn ich glaube, sie dachten, ich
würde davon laufen. Und der Kerkermeister gestand nachher,
das sie es in dieser Absicht gestattet hätten, damit ich entkomme
und sie von ihrer Angst befreit würden; aber ich sagte ihm, das
ich nicht diesen Geist habe. 

Dieser Kerkermeister hatte eine Schwester, ein kränkliches
junges Weib. Sie kam zu mir, um mich zu besuchen; und nach
dem sie einige Zeit bei mir gewesen war, und ich Worte der
Wahrheit zu ihr geredet hatte, ging sie hinunter und sagte den
% \picinclude{./030-039/p_s035.jpg} 
andern, wir seien unschuldige Leute und täten niemand nichts zu
leide, sondern allen nur Gutes, sogar solchen, die uns hasten,
und bat sie, freundlich gegen mich zu sein. [...].

Während ich im Korrektionshaus war, besuchten mich meine
Verwandten\index{Verwandten}, und da sie über meine Gefangenschaft bekümmert
waren, gingen sie zu den Richtern und baten sie, das ich mit
ihnen heim gehen dürfe. Sie erboten sich, sich mit hundert Pfund
zu verbürgen und einige andere aus Derby, die mit ihnen waren,
je mit fünfzig Pfund, das ich nicht mehr dorthin komme, um
gegen die Priester zu reden. So wurde ich vor die Richter
gebracht, und weil ich nicht einwilligen wollte, das irgendjemand
sich meinetwegen verpflichte, -- denn ich war ja keines Vergehens
schuldig und hatte das Wort des Lebens und der Wahrheit geredet, --  
erhob sich Richter Bennet zornig, und als ich niederkniete, um
Gott zu bitten, ihm zu vergeben, rannte er auf mich los und schlug
mich mit beiden Händen\index{Misshandlung} und schrie: \zitat{Fort mit 
ihm! Kerkermeister, nimm ihn fort!} Hierauf wurde ich wieder in den Kerker gebracht
und musste dort bleiben, bis meine Zeit von sechs Monaten um
war. Aber ich durfte nun eine Meile weit allein gehen, was ich
tat, als ich fühlte, das ich es durfte. Oft ging ich auf den Markt
und in die Straßen und ermahnte die Leute, sich von ihrer
Schlechtigkeit zu bekehren, und ging dann wieder ins Gefängnis.
Und da Leute von allerlei Religionen mit mir im Gefängnis
waren, ging ich hie und da zu ihnen und wohnte ihren 
Versammlungen an den Ersten Tagen bei\index{Ökumene} [...].