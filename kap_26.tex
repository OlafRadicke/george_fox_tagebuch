
%%%%%%%%%%%%%%%%%%% Kapitel 26. %%%%%%%%%%%%%%%%%%%%%%%%%%%%%%
\chapter[Jakob II. Amnestie]{Jakob II. Amnestie}

\begin{center}
\textbf{Kampf für die Ordnung im Quäkertum. Jakob II. Amnestie.}
\end{center}


Nachdem ich einige Wochen in South-Street\ort{South-Street} gewesen war
und dort manche Versammlungen für die Freunde gehalten hatte,
kehrte ich nach London\ort{London} zurück. Hier half 
ich unter andrem den
Freunden ein Zeugnis\index{Zeugnis} aufsetzen, um sie 
von dem Verdachte zu
reinigen, sie hätten sich am letzten Aufstand im Westen oder an
irgend andern Ausständen oder Verschwörungen gegen die Regierung
beteiligt. Und dieses Zeugnis wurde dann dem obersten Richter
eingereicht, der im Begriff war, nach dem Westen zu gehen, um
die Gefangenen zu verhören.

Ich blieb einige Zeit in London und arbeitete im Dienst der
Wahrheit. Dann ging ich für etwa eine Woche wieder aufs
Land, weil meine Gesundheit unter dem Mangel an frischer Lust
sehr litt [...] und kehrte dann wieder in die Stadt zurück, wo
ich während zwei Monaten die Versammlungen besuchte und mein
Möglichstes tat, um für die Freunde, die in allen Teilen des
Landes viel zu leiden hatten, Erleichterungen zu erwirken. Auch
schrieb ich mehrere Schriften zur Förderung der Wahrheit. Die
eine handelte von der Ordnung in der Kirche Gottes, der sich
etliche unter den Freunden stark 
widersetzt\index{Konflikt!unter Quakern} hatten. Sie lautete:

\grosszitat{
Überall in der Welt besteht für Familie, Gesellschaft oder Stadt
irgend eine Ordnung. Im alten Testament war es die Ordnung\index{Ordnung!der Zusammenlebens}
Arons und Melchisedeks (Hebr. 7:11\bibel{Hebr. 7:11}) und 
danach die Ordnung Jesu
Christi, und er verachtete diese Ordnung nicht. Gott ist ein Gott
der Ordnung in seiner ganzen Schöpfung, so auch in seiner Kirche.
Und alle, die an das Licht glauben, an das Leben in Christus,
durch das man vom Tod ins Leben eingeht, sind in der Ordnung
des heiligen Geistes, und im Licht und Leben, der Kraft und dem
Reich Jesu Christi\index{Reich Gottes}, deren Wachstum 
kein Ende nimmt. Aber
solches ist verborgen den Geistern der Unordnung, die so viel
% \picinclude{./290-299/p_s299.jpg} 
schreiben und drucken gegen die Ordnung, die der Herr durch
seinen Geist und seine Kraft unter seinem Volke aufgerichtet hat.

Ihr, die ihr so viel gegen die Ordnung schreit, ihr seid ja
in ein \zitat{Land der Finsternis\index{Finsternis} und 
des Dunkels geraten; ein Land,
da es stockfinster ist und da keine Ordnung ist, und wo es ist
wie Finsternis, wenn es hell wird} 
(Hiob 10:21\bibel{Hiob 10:21}). Ist nicht
dies euer Zustand, wie alle, die in der Wahrheit und nach dem
Evangelium des Lebens und des Heils wandeln, sehen können? [...]
\bigskip
\begin{flushright}
G. F.\end{flushright}
}


Ich konnte abermals nicht lange in London bleiben, da ich
die Eingeschlossenheit in der Stadt nicht lange hintereinander
ertragen konnte [...]\index{Stadtleben}. Ehe ich die Stadt 
verließ, hörte ich von
einem berühmten Gelehrten aus Polen, der kürzlich hergekommen
war; ich lud ihn in meine Wohnung ein und hatte eine lange
Unterredung mit ihm. Nachdem ich mich über alles, was ich zu
wissen wünschte, erkundigt hatte, schrieb ich einen Brief an den
König von Polen, wegen der Freunde in Danzig, die lange
schwer zu leiden gehabt hatten. Es folgt hier eine Abschrift
davon:

\grosszitat{
    An den König von Polen.
    \bigskip
    An Johann den Dritten\person{Johann III.}, König 
    von Polen\person{König von Polen}, Großherzog von
    Litauen\ort{Litauen}, Russland\ort{Russland} und Preußen\ort{Preußen}, 
    Beschützer der Stadt Danzig, [...]
    wegen der heimgesuchten und unschuldigen Leute, die man im
    Groll Quäker nennt, die jetzt bei Wasser und Brot in oben
    genannter Stadt sind, in strenger Gefangenschaft, wo man ihren
    Frauen und Kindern kaum erlaubt, sie zu besuchen.
    \bigskip
    O König!
    \bigskip
    Die Behörden der Stadt Danzig\ort{Danzig} sagen, es sei dein Wille,
    das dieses unschuldige, heimgesuchte Volk solche Unterdrückung zu
    erleiden habe. Nun ist die Strafe nur darum über sie verhängt
    worden, weil sie zusammenkommen im Namen Jesu Christi ihres
    Erlösers und Heilands, der für ihre Sünden starb und zu ihrer
    Rechtfertigung von den Toten auferstanden ist, der ihr Prophet
    ist, welchen Gott erweckt hat, wie Moses
    Und nun, in diesen Tagen des neuen Evangeliums und des
    neuen Bundes, sollten alle auf ihn hören, die \zitat{gewesen wie die
    irrenden Schafe, nun aber sich bekehrt haben zum Hirten und
    Bischof ihrer Seelen. Er hat sein Leben gegeben für seine
    Schafe, und sie hören seine Stimme und folgen ihm und er führt
    sie auf seine Weide} (Joh. 10:9\bibel{Joh. 10:09@Joh. 10:9}).
    % \picinclude{./300-309/p_s300.jpg} 
    Ich hörte, O König, du bekennest dich öffentlich zum Christen-
    tum und zum mächtigen Namen Jesu Christi, deö Königö der
    Könige, deß Herrn der Herren, dem alle Gewalt im Himmel und
    auf E-rden gegeben ist, der alle Völker mit eisernem Szepter
    regiert. GZ scheint uns darum hart, o König, das; jemand, der
    offen Christuö bekennt, solche Strafen über ein harmloseß und
    unschuldigeß Volk verhängt, nur weil sie zarte Gewissen haben
    und zusammen kommen, um den ewigen Gott, der sie gemacht
    hat, im Geist und in der Wahrheit anzubeten, wie Christuß etz vor
    1600 Jahren eingesetzt hat nach Joh. 4, 23. 24.
    Jch bitte nun den König, darüber nachzudenken, ob Christus
    im neuen Testament je seinen Aposteln ein Gebot gegeben, sie
    sollten jemand ins- Gefängniö werfen bei Wasser und Brot, der
    sich nicht in allen Stücken ihrer Religion, ihrem Glauben und
    ihrer Art der Anbetung anschloß? Wo haben die Apostel nach
    der Himmelfahrt in der wahren Kirche solcheß getan,? Lehren
    nicht Christue und die Apostel, seine Nachfolger sollen die Feinde
    lieben, und bitten für die, welche sie hassen, verfolgen und ver-
    leumden? (Matth. 5.)
    Jst etz nicht eine Schande für das Christentum den Türken
    und andern gegenüber, daß ein Christ den andern verfolgt um
    der Glarfenölehre, der Art der Anbetung und der Religion
    willen? Sie können nicht beweisen, daß Christuz, den sie ihren
    Herrn und Meister nennen, je ein solches Gebot gegeben. Christu-3
    sagt, seine Nachsolger sollen sich unter einander lieben, daran
    werde man erkennen, daß sie seine Jünger seien (Joh. 13). Und
    hat nicht Christuö jene getadelt, die wollten Feuer vom Himmel
    regnen lassen, um alle zu verderben, die ihn nicht aufnehmen
    wollten?« (Luc. 9). Gr sagte zu ihnen: »Wisset ihr nicht, weö
    Geistes Kinder ihr seid? Wissen je die, welche die Menschen
    verfolgen oder töten, weil sie eine Religion nicht annehmen
    wollen, weß Geisteö Kinder sie sind? Wäre ez nicht gut,
    wenn alle durch den Geist Christi wüßten, weß Geistes
    Kinder sie sind? Denn der Apostel sagt, Römer 8, 9: ,,Wer
    Christi Geist nicht hat, der ist nicht sein«; und 2. Cor. 10, 4:
    ,,Die Waffen unsrer Riiterschaft sind nicht sleischlich sondern geist-
    lich«; und Epl). 6, 12: ,,Wir haben nicht mit Fleisch und Blut zu
    % \picinclude{./300-309/p_s301.jpg} 
    Kampf für die Ordnung im Quäkertum. Jakobyll. Amnesiie. 301
    kämpfen, sondern mit den bösen Geistern unter dem Himmel.«
    Daraus- ersehen wir, daß der Kampf der ersten Christen und
    ihre Waffen geistiger Art waren. Würde eß dem König und den
    Behörden von Danzig nicht gegen daß Gewissen gehen, wenn die
    Türken sie zu ihrer Religion zwingen würden? oder wenn die
    Behörden von Danzig zur Religion det-’ Königö von Polen ge-
    zwungen würden? oder würde etz der König von Polen nicht
    grausam und gegen sein Gewissen finden, wenn er zur Religion
    der Behörden von Danzig gezwungen würde? und im Fall sie
    sich derselben nicht unterwerfen wollten, von Weib und Kindern
    getrennt und auz dem Lande verbannt, oder bei Wasser und Brot
    inö Gefängniß geworfen würden ?
    Wir bitten darum den König und die Behörden in aller
    christlichen Demut, daß sie in dieser Angelegenheit nach dem
    königlichen Gesetz Gotteß vorgehen möchten, nämlich ,,andern zu
    tun, waö sie möchten, daß man ihnen tue« (Jak. 2, 8), ,,rmd
    ihren Nächsten zu lieben als- sich selbst« (Matth. 22, 39). Denn
    wir hoffen und glauben denn doch, daß sowohl der König von
    Polen und seine Leute, altz auch die Behörden von Danzig, die
    Schriften deß neuen Testamentß sowie deö alten, kennen; und wir
    bitten darum den König und seine Räte, darauf zu achten, daß
    sie nicht dem königlichen Gesetz Gotteß und dem herrlichen und
    ewigen Evangelium der Wahrheit entgegen, ein unschuldigeß Volk
    gefangen nehmen, nur weil etz zusammen kommt mit zarten Ge-
    wissen, um Gott seinem Schöpfer zu dienen und ihn anzubeten.
    Wir bitten den König in christlicher Liebe, dieseö alles ernst-
    lich und eingehend zu bedenken, und Befehl zu geben, daß die
    unschuldigen Gefangenen, unsere Freunde, die sogenannten Quäker,
    frei gelassen werden autz ihrer harten Gefangenschaft in Danzig,
    daß sie frei sein mögen, den lebendigen Gott im Geist und in der
    Wahrheit anzubeten und ihm zu dienen, und heimzugehen um
    ihr Handwerk weiter zu treiben und ihre Familien zu erhalten.
    Wir glauben, daß der König, wenn er solch ein edles, ruhm-
    oolleß, ja christlicheö Werk tut, nicht unbelohnt bleiben wird
    von dem großen Gott, dem wir dienen, der die Herzen der
    Könige und ihr Leben und die Länge ihrer Tage in seiner
    Hand hat.
    Von Einem, der möchte, daß der König und alle seine
    Räte in der Furcht de?. Herrn bewahrt bleiben mögen und
    % \picinclude{./300-309/p_s302.jpg} 
    sein Wort der Weisheit annehmen, durch welches alle Dinge ge-
    schaffen wurden. [...]
    \bigskip
    \begin{flushright}
    London, 10. des 3. Monats, den man pflegt Mai zu nennen, 1684.
    G. F.\end{flushright}
}


Jch schrieb in dieser Zeit noch manches andere im Dienst
der Wahrheit; etwas »über das Richten,« denn etliche, die von
der Wahrheit abgesallen waren, hatten eine solche Angst, von ihr
gerichtet zu werden, daß sie sich eifrig bemühten, gegen das Richten
zu schreien. Darum erließ ich ein Schreiben, um aus der Schrift
der Wahrheit zu beweisen, daß die Kirche Christi Macht hat, alle,
die vorgeben dazu zu gehören, nicht nur in Dingen dieser Welt,
sondern auch in religiösen Dingen, zu richten. ....
Bei den Verhören in den Gerichtssitzungen, im 2. Monat 1686
in Hicks Hall, wurden viele Freunde vorgenommen; ich war
täglich bei ihnen, um zu raten und zu helfen, damit nichts ver-
säumt und kein Vorteil unbenutzt bleibe; und gewöhnlich hatten
sie guten Erfolg. Bald daraus gefiel es dem König, nachdem
wir ihm immer wieder Klagen über unsere Leiden vorgelegt hatten,
zu befehlen, daß man: »alle die um des Gewissens willen gefangen
waren, sreilasfe, soweit er Macht habe es zu bestimmen.« Die
Türen der Gesängniss e taten sich denn auch aus, und Viele hundert
Frermde, von denen manche lange gefangen gewesen waren, er-
hielten die Freiheit. Viele oon ihnen kamen zur Jahresversamnn
lung, zur großen Freude der Freunde. ....
Jch brachte den größten Teil des Jahres 1686 in London
zu, außer wenn ich nach Bethnal-Green oder Enfield ging, oder
nach Chiswick, wo ein Freund eine Schule errichtet hatte, in der
Kinder von Freunden erzogen wurden.
.... Auch schrieb ich noch allerlei in diesem Jahr, unter anderm
eine Ermahnung » an die Freunde, in der Einigkeit der Wahrheit zu
bleiben, in welcher keine Trennung noch Gntzweiung ist.« ....
Bald daraus, als ich merkte, daß etliche Abtrünnige, welche
der Feind zur Trennung und Spaltung von den Freunden geführt
hatte, fortsuhren in ihrem Schreien und ihrem Widerstand gegen
unsre Monats-, Vierteljahres- und Jahresversammlungen, so trieb
es mich, einen kurzen Brief an die Freunde zu schreiben, um sie zu
erinnern, daß sie durch den Geist des Herrn in ihrem Innern
die Bestätigung und Besiegelung empfangen hatten, daß diese
Versammlungen vom Herrn seien und von ihm angenommen


% \picinclude{./300-309/p_s303.jpg} 
Wirken in London unter dem Zeichen der Toleranz. 303
werden, und daß sie darum nicht von den Gegnern erschüttert
werden können:
,,Meine lieben Freunde im Herrn Jesus Christus,
Jhr alle, die ihr in seinem heiligen Namen versammelt seid,
wisset, daß eure Versammlungen, die oierteljährlichen wie die
andern, durch die Kraft und den Geist Gottes eingesetzt sind.
Sie sind in euern Herzen von dieser Kraft und diesem Geist be-
zeugt, durch die Kraft und den Geist Gottes sind sie in euch gegründet
und ihr in ihnen. Gott der Herr hat es euch durch seinen Geist
besiegelt, daß eure Versammlungen nach seiner Ordnung und
Einberufung geschehen, und er hat sie anerkannt, indem er euch
mit seiner gesegneten Gegenwart in denselben begnadete; ihr
habt es reichlich erfahren, wie er euch mit seinem Leben, seiner
Weisheit und Kraft und mit himmlischen Gütern aus seinen
Schätzen und Quellen ausriistete, und aus denselben sind wiederum
viele Danksagungen und Lobpreisungen in euern Versammlungen
zu seinem heiligen Namen zurückgekehrt. Er hat euch eure Ver-
sammlungen durch seinen Geist besiegelt, und daß euer Zusammen-
kommen im Herrn geschehen ist, in Christus seinem Sohn und in
seinem Namen und nicht durch Menschen. Darum gebühret dem
Herrn, daß er durch sie und in ihnen gepriesen werde, der euch
und sie beschützt hat mit seinem mächtigen Arm gegen alle Gegner
und Feinde und ihre verleumderischen Zungen und Bücher. Denn
des Herrn Macht und Same regieret über alles; er erhält seine
Kinder zu seiner Ehre, als die so »Obmacht und Hoheit haben
am Ehrentage« (Ps. 110, 3) .... aus daß alle dem Herrn in
Jesus Christus dienen von Geschlecht zu Geschlecht.«
London, den 3. des 11. Monats 1686. G. F.
