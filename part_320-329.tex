% \picinclude{./320-329/p_s320.jpg} 
die Sau, die, nachdem sie gewaschen ist, sich wieder im Kot
wälzt (Spr. 26, 11). Dietz ist der Zustand vieler gewesen, wie
Gott und sein Volk weiß. .
Darum bleibet alle fest in Jesus Christus- euerm Haupt, in
welchem ihr alle einß seid, Männer und Weiber, und seine Herr-
schaft kennet. Seine Herrschaft und sein Friede werden nicht
aufhören zu wachsen; aber die Herrschaft deö Teuselö und mit
ihr alle, die nicht in Christo sind, sich ihm und seiner Herrschaft
widersetzten, werden ein Ende haben, ihr Gericht bleibt nicht aus-
und ihre Verdammniß schlummert nicht. Darum lebet und
wandelt alle in Liebe, Unschuld und Reinheit, in Licht, Leben,
Geist und Kraft auß Gott und Ehristue, die über allem sind.
Bleibet in der Rechtschassenheit und Heiligkeit, in der Kraft und
dem heiligen Geist Gotteß, in welchem daß Reich Gottez steht.
Alle, die ihr Kinder des neuen, himmlischen Jerusalem auö der
Höhe seid, richtet eure Blicke dorthin.
Der Geist der Auflehnung und des Widerstandes, der früher
und auch kürzlich wieder sich erhoben hat, stammt nicht auö dem
Reich Gotteß und ist ferne vom Reich Gottes:-’ und vom himmlischen
Jerusalem und fällt dem Gericht und der Verdammniß anheim
mit allen seinen Büchern, Worten und Werken. Darum sollen
die Freunde in der Kraft und dem Geist Gotteß leben und wandeln,
die über jenem Geist sind und im Samen, der ihn vernichten und
in Stücke schlagen wird (1. Mos. 3). Jn diesem Samen habt ihr
Frieden und Freude in Gott, und die Macht, jenen Geist der
Auflehnung zu richten, und eure Einigkeit ist in der Kraft und
dem Geist Gotteß.
Lasset keinen ihm selber leben, sondern alle sollen Gott
leben, wie sie auch ihm sterben sollen (Röm. 14); und suchet den
Frieden der Kirche Christi und den Frieden aller Menschen in
ihm, denn ,,selig sind die Friedfertigen.« Bleibet in der reinen
friedlichen, himmlischen Weiöheit Gotteß, welche friedsam, gelinde
’ und voll Barmherzigkeit ist. Trachtet alle, einerlei Sinnes und
Herzens- zu sein, eine Seele und eine Meinung in Christus,
und habet seinen Sinn und Geist in euch wohnen, ermuntert
euch untereinander in der Liebe Gotteö, welcher den Leib Christi,
seine Kirche, erbauet, deren heiligeß Haupt er ist. Ghre sei Gott
durch Ehristuß, jetzt und immerdar; er ist der Felß und Grund,
der Emmanuel Gott mit und daß Amen in allem, der Anfang


% \picinclude{./320-329/p_s321.jpg} 
rmd daß Ende. Jn ihm lebet und wandelt, in welchem ihr
ewigeß Leben habet; in ihm werdet ihr mich spüren und ich euch.
Alle Kinder dez neuen Jerusalem auß der Höhe, der heiligen
Stadt, deren Licht der Herr und daß Lamm sind, und welche der
Tempel ist (Offb. 2 1), in ihr sind sie wiedergeboren auö dem
Geist; also ist daß; Jerusalem aus- der Höhe die Mutter derer,
die auS dem Geist geboren sind. Die, welche inß himmlische
Jerusalem gekommen sind und noch kommen, nehmen Christuin
auf, und er gibt ihnen Macht, Gottes Kinder zu sein, und sie sind
wiedergeboren auö dem Geist, und so ist daß Jerusalem auß der
Höhe ihre Mutter (Gal. 4). Solche kommen zum himmlischen
Berge Zion, zu der Menge vieler tausend Engel, zu den Geistern
der vollkommenen Gerechten, zu der Stadt deö lebendigen Gottetz,
zu der Gemeinde der Erstgeborenen, die im Himmel angeschrieben
sind und den Namen Gottes tragen (Ebr. 12). Hier ist eine
neue Mutter, von der ein neues himmlische?-, geistigeö Geschlecht
abstammen wird. ES gibt keine Spaltung, keinen Streit, keine
Entzweiung im himmlischen Jerusalem, noch im Leib Christi,
· welcher autz lebendigen Steinen erbaut ist (l.Petr. 2), ein geistiges
Hauß. Bei Christuö ist keine Spaltung; denn in ihm ist Friede.
Christutz sagt: ,,Jn mir habt ihr Friede« (Joh. 16). Und er ist
aus- der Höhe und nicht von dieser Erde. Jn dieser Welt und
in ihrem Geist ist Angst; darum bleibet in Christuö und wandelt
in ihm.
Jerusalem war die Piutter aller wahren Christen, vor dem
Abfall. Seitdem die äußerlichen Christen sich in viele Sekten
gespalten haben, haben sie oiele Mütter; alle aber, die durch
Christi Kraft und Geist vom Abfall zurück gekommen sind, haben
Jerusalem auß der Höhe zur Mutter und keine sonst; sie ernährt
alle ihre geistigen Kinder.« G. F.
(Dieser Brief wurde an der Jahreöoersanunlung, in London,
im Jahre 1691 gelesen.)
George Fox. 21


% \picinclude{./320-329/p_s321z01.jpg} 

% \picinclude{./320-329/p_s321z03.jpg} 

% \picinclude{./320-329/p_s321z04.jpg} 