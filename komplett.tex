% Für das erstellen von PDFs, benutze das bash-skript make_pdf.sh
% Für das Erstellen von eBooks benutze den Befehl:
% pandoc -s komplett.tex -o komplett.epub
% Zum lesen benutze das Programm fbreader

% \documentclass[a4paper,10pt]{article}
\documentclass[a4paper,12pt,twoside]{book} %article}
\usepackage[utf8]{inputenc}

\usepackage{ngerman}
% \usepackage[a4paper,landscape]{geometry}
%\usepackage[a4paper]{geometry}
\usepackage{multicol}
\usepackage{graphicx}
\usepackage{floatflt}
\usepackage{float}
\usepackage{url}
\usepackage[pdftex,unicode]{hyperref} 

\usepackage{index}
  \newindex{default}{idx}{ind}{Schlagwortverzeichnis}
  \newindex{bibel}{bidx}{bind}{Bibelstellen}
  \newindex{brief}{brdx}{brnd}{Briefverzeichnis}
  \newindex{buch}{budx}{bund}{Bücherverzeichnis}
  \newindex{ort}{odx}{ond}{Ortsverzeichnis}
  \newindex{person}{pdx}{pnd}{Personenverzeichnis}


\pagestyle{plain}

\newcommand{\picinclude}[1]{\includegraphics[height=1.0\textheight]{#1}}
\newcommand{\zitat}[1]{"`\textit{#1}"'}
\newcommand{\grosszitat}[1]{\bigskip \begin{quote} #1\end{quote} \bigskip}

% Index-Relevante Befehle.
\newcommand{\bibel}[1]{\index[bibel]{#1}}
\newcommand{\brief}[2]{\bigskip \begin{quote}\index[brief]{#1} #2\end{quote} \bigskip}
\newcommand{\buch}[1]{\index[buch]{#1}}
\newcommand{\buchtitel}[1]{"`\textit{#1}"'\index[buch]{#1}}
\newcommand{\ort}[1]{\index[ort]{#1}}
\newcommand{\person}[1]{\index[person]{#1}}

%opening
\title{George Fox -- Aufzeichnungen und Briefe des ersten Quäkers}
\author{Übersetzerin: Margrit Stähelin \\ 
Originalerscheinung: Tübingen 1908 \\ 
Verlag I.C.B. Mohr (Paul Siebek) \\
Nachbearbeitung: Olaf Radicke
}

\date{Erstellungsdatum: \today}

% \makeindex

\begin{document}

\maketitle


\begin{figure}[h!]
 \centering
 \includegraphics{./cc-lizenz-by.png}
 % cc-lizenz-by.png: 32x32 pixel, 72dpi, 1.13x1.13 cm, bb=0 0 32 32
\end{figure}

\newpage 

\tableofcontents

\newpage

\frontmatter 

\chapter{Vorwort}
% \section{Vorwort}
\label{sec:vorwort}

\section{Urheber und Autoren}

Diese Dokument basiert auf ein Scann des Werkes \zitat{George Fox -- 
Aufzeichnungen und Briefe des ersten Quäkers}, einer 
Übersetzung von Margrit Stähelin, erschienen Tübingen 1908, 
im Verlag I.C.B. Mohr (Paul Siebek). 


Trotz intensiver 
Bemühungen war es mir nicht möglich zu ermitteln ob und wer das 
Urheberrecht an dem Werk besitzt. Sollte jemand zur
Klärung beitragen können, bitte ich um Hinweise an mich!

\begin{center}
Olaf Radicke \\
Ludwig-Richter-Str. 28 \\
80687 München \\
briefkasten@olaf-radicke.de \\
\end{center}

Ich glaube das der Text für das Quakertum in Deutschland so 
unverzichtbar ist, das ich mich entschlossen habe trotzdem den Text 
(wieder) zu veröffentlichen. Meine Text-satz und -bearbeitung 
stelle ich unter 
einer \textit{Creative Commons-Lizenz lizenziert}: 

\bigskip

\begin{center}
\textbf{Namensnennung-Weitergabe unter gleichen Bedingungen 3.0 
Deutschland (CC BY-SA 3.0).}
\end{center}

\section{Lizenz}
\subsection*{Sie dürfen}
%\addcontentsline{toc}{section}{Sie dürfen}

\begin{itemize}
 \item das Werk bzw. den Inhalt vervielfältigen, verbreiten und öffentlich zugänglich machen
 \item Abwandlungen und Bearbeitungen des Werkes bzw. Inhaltes anfertigen
 \item das Werk kommerziell nutzen
\end{itemize}

% \subsection{Zu den folgenden Bedingungen}
\subsection*{Zu den folgenden Bedingungen}
%\addcontentsline{toc}{section}{Zu den folgenden Bedingungen}

\begin{description}
 \item[Namensnennung] Sie müssen den Namen des Autors/Rechteinhabers in der von ihm 
festgelegten Weise nennen.

 \item[Weitergabe unter gleichen Bedingungen] Wenn Sie das lizenzierte Werk bzw. den 
lizenzierten Inhalt bearbeiten oder in anderer Weise erkennbar als Grundlage für 
eigenes Schaffen verwenden, dürfen Sie die daraufhin neu entstandenen Werke bzw. 
Inhalte nur unter Verwendung von Lizenzbedingungen weitergeben, die mit denen dieses 
Lizenzvertrages identisch oder vergleichbar sind.
 \end{description}

% \subsection{Wobei gilt}
\subsection*{Wobei gilt}
%\addcontentsline{toc}{section}{Wobei gilt}

\begin{description}
    \item[Verzichtserklärung] Jede der vorgenannten Bedingungen kann aufgehoben werden, 
	  sofern Sie die ausdrückliche Einwilligung des Rechteinhabers dazu erhalten.
    \item[Public Domain (gemeinfreie oder nicht-schützbare Inhalte)] Soweit das Werk, 
	  der Inhalt oder irgendein Teil davon zur Public Domain der jeweiligen Rechtsordnung 
	  gehört, wird dieser Status von der Lizenz in keiner Weise berührt.
    \item[Sonstige Rechte] Die Lizenz hat keinerlei Einfluss auf die folgenden Rechte:
      \begin{itemize}
          \item Die Rechte, die jedermann wegen der Schranken des Urheberrechts oder 
		aufgrund gesetzlicher Erlaubnisse zustehen (in einigen Ländern als 
		grundsätzliche Doktrin des fair use etabliert);
          \item Das Urheberpersönlichkeitsrecht des Rechteinhabers;
          \item Rechte anderer Personen, entweder am Lizenzgegenstand selber oder bezüglich 
		seiner Verwendung, zum Beispiel Persönlichkeitsrechte abgebildeter Personen.
      \end{itemize}
    \item[Hinweis] Im Falle einer Verbreitung müssen Sie anderen alle Lizenzbedingungen 
	  mitteilen, die für dieses Werk gelten. Am einfachsten ist es, an entsprechender 
	  Stelle einen Link auf diese Seite einzubinden.

 \end{description}

Diese "Commons Deed" ist lediglich eine vereinfachte Zusammenfassung des rechtsverbindlichen 
Lizenzvertrages in allgemeinverständlicher Sprache. Deteils, Erleuterungen und vollständigen
Lizenz-Text erhalten Sie unter \url{http://creativecommons.org/licenses/by-sa/3.0/de/}

\section{Anmerkungen zu Änderungen}

Der Text basiert zwar auf der Übersetzung von Stähelin, doch wurden 
zahlreiche Änderungen gemacht. Die wichtigsten 
seien hier genannt.

\begin{itemize}
 \item Die Kapitelnummerierung wurde nicht statisch übernommen. Die 
  Reihenfolge ist aber geblieben.
 \item Die Schreibweise \zitat{Quäker} wurde durch den englische 
  \zitat{Quaker} ersetzt.
 \item Die Schreibweise \zitat{Renter} wurde durch den englische  
  \zitat{Ranter} ersetzt.
 \item Die Kapitel-Überschriften wurden z.T. gekürzt oder geändert.
 \item Es wurden zur besseren Gliederung Unterabschitte mit Überschriften eingefügt.
 \item Es wurde ein Index erstellt mit Orten, Personen und Anderen
  Dingen.
 \item Es wurden Bilder hinzugefügt.
 \item Lange Zitate (z.B. bei Briefen) wurden durch Einrückung besser  
  kenntlich gemacht.
 \item Namen von Bibel-Büchern geändert, so wie sie Heute in der Regel
  verwendet werden.
 \item Die Altdeutscheschrift durch moderne Schrift ersetzt.
 \item An einigen Stellen Kommentare eingefügt.
 \item Absätze anders gesetzt.
\end{itemize}

Die Arbeiten an dem
Text wurden mit dem Versionskontrollsystem 
Git\footnote{\url{http://de.wikipedia.org/wiki/Git}} protokolliert.
Das Git-Versionsarchiv kann unter dieser URL heruntergeladen werden:
\url{http://www.fkbk.de/git/george_fox_tagebuch/}  

\newpage 

\mainmatter 

% % \picinclude{./vorwort/p_v01.jpg}
George Fox.
Aufzeichnungen und Briefe dez ersten Quäkerz.
« Jn Außwahlüberfetztvon
Marg. stähelin.
Mit einer Einführung von
V Professor 1). Paul Wernle.
M
P.-Iizgiöze czezzelizcbczsss (ier l-reuncso
(HDMI-LCS-:)
S S s Z i si N W 7
I’rin:—i.Oui5—’s:c-iscjZcjoncsssir. Z
Tübing en.
Verlag von J. E. B. Mohr (Paul Siebeck).
1908.


% \picinclude{./vorwort/p_v02.jpg}

Alle Rechte vorbehalten.


% \picinclude{./vorwort/p_v03.jpg}

Jnhaltzverzetchntz.
I: Seite
Zur Einführung. Von Professor 1). Paul Wernle ....... 7
Kap. 1. Erweckung und Krisis bis zum Durchbruch ..... 1
Kap. ll. Erste Versammlungen und Proteste ........ 16
Kap. lll. Der. Tumult in Nottingham. Wachsender Widerstand,
bis szum Gefängnis in Derby .......... 26
Kap. 17. Erlebnisse im Gefängnis zu Derby. Ein ,,Wehe« über
die Stadt Lichfield. Erste Missionsgenossen. Antikirch-
liche Agitation und Kampf gegen die Ranter .... 35
Kap. 7. Christus in uns. Erkenntnis der Quäkerischen Welt-
mission. Das Haus Richter Fells in Swarthmore. Der
Pöbel von Ulverstone. Rechtfertigung vor dem Gericht
in Lancaster ................ 56
Kap. 71. Fox der Hexerei verdächtigt. Falsche Osfenbarungen bei
Freunden. Gefangenschaft in Earlisle ....... 71
Kap. 711. Kämpfe mit schwärmerischen Rantern und zehntengierigen
Priestern. Fox in Wetstone verhaftet und vor Cromwell
geschickt .........   ......... 83
Kap. 7111. Brief an den Papst. Die Studenten von Cambrigde.
Die Quäker in der Bibel. Wachsende Entfremdung von
Cromwell ................. 96
Kap. llc. Angriffe der Jndependenten und Presbyterianer. Ahnungen,
Heilungen, Bekehrungen. Dispute über Taufe und Er-
wählung. Gefangennahme auf Grund angeblicher Ver-
schwörung. Wirken während der Gefangenschaft . . . 104
Kap. 1. Warnung an die Kegelspieler. Naylors Fall. Dis-put
mit Paul Gwin. Besuch bei Eromwell. Herumreisen bei
den gefangenen Freunden. Reise in Wales ..... 120
Kap. I1. Reise nach Schottland. Kampf gegen die Prädestinations-
lehre und Widerstand der schottischen Geistlichkeit . . . 128
Kap. Ill. Erste Jahresversammlung. Warnung an Cromwell vor
der Königskrone. Trostbrief an dessen Tochter. Gesichte
vom Tode Cromwells und der kommenden Reaktion . . 134
Kap. 1111. Ein Gottesgericht. Ermahnung zur Barmherzigkeit bei
Schissbrüchen. Quäkersreundlicher Erlaß des General
Monk. Fox als Königsfeind gefangen und schließlich .
auf Befehl Karls ll. befreit .......... 145


% \picinclude{./vorwort/p_v04.jpg}

17 Jnhaltsverzeichnis.
Kap. 117. Beginn neuer Quäkerverfolgungen bei Anlaß der Ver-
schwörung der Fifthmonarchy -Leute. Des Q-uäkers
John Perots Verirrungen. O-uäker mißhandelt in Neu-
England und Malta .... . ........ 154
Kap. 17. Ein Gottesgericht. Verhaftung wegen .angeblicher Ver-
schwörung und schreckliche Gefangenschaft in Lancaster und .
Scarbro. Dispute im Gefängnis mit Baptisten und andern.
Fox sieht den Brand von London voraus ...... 169
Kap. 171. Einrichtung der Monatsversammlungen. Regelung der
Osuäkerehen. Gründung von Knaben- und Mädchenschulen.
Reformation des Quäkertums .......... 188
Kap. 1711. Reise nach Jrland. Rückkehr, und Heirat mit Margaret
Fell. Jhre abermalige Gefangennahme. Schwere innere
Anfechtungen ................ 201
Kap. )c71l1. Reise nach Amerika. Barbadoes. Jamaika ..... 215
Kap. X11. Arbeit in Nordamerika unter Engländern und Jndianern 224
Kap. IX. Ankunft in Bristol. Zusammentreffen mit William Penn
und andern. Verteidigung der Frauenversammlungen.
Vorgeahnte Gefangenschaft in Worcefter. Brief an den
König über die Grundsätze der Quäker. Krankheit. Be-
freiung. Während der Gefangenschaft verfaßte Schriften 235
Kap. 111. Fox sammelt und ordnet die Bücher und Schriften, die er
geschrieben, und tritt für die Frauenversammlungen ein . 243
Kap. JTI11. Reife nach Holland. Einrichtung der kirchlichen Ordnung
für Holland und Deutschland. Briefwechsel mit Prinzessin
Elisabeth. Reise nach Deutschland bis Oldenburg. Briefe
an verschiedene Behörden von Holland und Deutschland 251
Kap. Jllllll. Rückkehr nach England. Kampf der Ordnungspartei gegen
die unbotmiißigen Quäker. Briefe über Toleranz an den
König von Polen, den Großmogul und andere .... 266
Kap. ZR17. Allerlei Mahn- und Troftschreiben ........ 281
Kap. D17. Zweite Reise nach Holland. Brief an den Herzog von *
Holstein zur Verteidigung des öffentlichen Redens der
Frauen ...... . ........... 285
Kap. II71. Kampf für die Ordnung im Quäkertum. Jakobsll.Amneftie 298
Kap. )T?(71l. Wirken in London unter dem Zeichen der Toleranz . . 303
Kap. II7111. Ahnung kommender Revolutionen. Christus König. Letzte
—— Arbeiten. Krankheit und Tod ......... 306
Anhang. Ein Brief von George Fox nach seinem Tode vorgefunden mit
der Aufschrift: Nicht vor der Zeit zu öffnen ..... 319
Zeittafel ......... . ............. 322
Berichtigungen .................... 324



% \picinclude{./vorwort/p_v05.jpg} 

zur Einführung.
Der Mann, der aus den folgenden Auszeichnungen zu uns
redet, ist schon von seinen Zeitgenossen als ein Rätsel und Ge-
heimnis angestaunt worden. Den einen erschien er als ein Mann,
der gewaltig predigte und nicht wie die Schriftgelehrten, den andern
als ein Verrückter, der selber andere verhexen könne. Man er-
zählte von ihm, er schlafe in keinem Bett, er könne fliegen, er
könne nicht ertrinken und man könne ihn nicht bluten machen,
weil er ein Zauberer sei. Seine magische Wirkung auf die
Zuhörer erklärten sich die einen daraus, daß er Flaschen bei sich
trage und den Leuten daraus zu trinken gebe, damit sie ihm
nachfolgten, andere meinten, er lege den Leuten Bänder um den Arm.
Selbst auf die Tiere gehe eine Kraft von ihm aus: die Hunde
mucksen nicht gegen die Quäker. Den Menschen sehe er den
Teufel im Gesicht geschrieben; ,,durchbohre mich nicht so mit
deinen Augen, wende deine Augen ab von mir«, rief ihm ein
streitsüchtiger Täufer zu. Anders wirkte er auf eine Frau in
Beverly durch eine kurze Ansprache in der Kirche daselbst; sie
erzählte nachher, ein Engel oder ein Geist sei in die Kirche ge-
kommen und habe herrliche Dinge von Gott geredet zur Ver-
wunderung aller Anwesenden, und als er geendet habe, sei er
verschwunden, sie wisse nicht, woher er gekommen noch, wohin er
gegangen sei.
Gin Jahrhundert später hat ihn Voltaire mit Jesus ver-
glichen. Der Vergleich war im Sinne einer Herabsetzung Jesu
gemeint. Voltaire glaubte, daß der englische Kanzelredner Tillotson
unendlich geschmackvoller als Jesus gepredigt habe; um nun die
richtige Analogie für das Bildungs-niveau Jesu zu finden, verglich
er ihn mit einem ungebildeten Schwärmer und Narren aus der
neuern Zeit, mit George Fox.



% \picinclude{./vorwort/p_v06.jpg} 


71 Zur Einführung.
Unter allem, was Voltaire von Jesnö zu sagen weiß, ist doch
dieser Vergleich mit Fox fast daö Beste. Allerdingz wird sich bei
jedem genaueren Zusammenschauen beider die handgreisliche
Uberlegenheit Jesu aufdrängen müssen, aber die Analogien sind
zahlreich und überraschend genug. E8 sind beideö Laien, die auf
Grund einer unmittelbaren inneren Berufung und Erleuchtung
sich getrieben fühlen, eine neue Weise, wie man Gott dienen soll,
zu verkünden, im Gegensatz zu allem, maß gerade von den Frommen
ihrer Zeit als göttlich autzgegeben wurde, heißen sie nun Pharisäer
oder Puritaner. Auch die wunderbaren Begleiterscheinungen haben
sie gemein; Heilkräfte gehen von ihnen aus, selbst auf Sterbende,
die von den Arzten ausgegeben sind, Vorahnungen und Gesichte
scheinen sie über den Zeitverlauf zu erheben, manche Antworten
und Weisungen gibt ihnen direkt der Geist, während sie bei an-
deren Gelegenheiten durch die Selbsteoidenz ihreß gesunden
Menschenoerstandetz überraschen. Nächst den Evangelien ist es
besonderß die Erzählung der Apostelgeschichte, an die man durch
Fox erinnert wird. Das beim Gebet erbebende Versammlungs-
hauö in Jerusalem, die Steinigung und Wiederbelebung des
Paulus in Lystra. die Gesängniöszene in Philippi mit dem
Kerkermeister, die Seefahrt nach Rom mit der Angst der Schiffs-
leute und der göttlichen Zuoersicht deö Apostel?-, alleö daß
wiederholt sich im Leben zdes Fox mit wenig veränderten Um-
ständen. Man glaubt, in die Tage des- UrchristentumZ zurück-
versetzt zu sein, nur mit dem Unterschied, daß, maß dort in der
Regel erst nach Jahrzehnten durch sekundäre Berichterstatter
schriftlich aufgezeichnet wurde, hier in einer eigenhändigen
Niederschrift des Manne?-, der all daß erlebt hat, unö entgegen-
tritt. Man darf hoffen, daß die neuteftamentlichen Gxegeten sich
künftig diesen Laienkommentar zu den Erlebnissen Jesu und der
Apostel nicht entgehen lassen, nicht im Jnteresse einer klein-
gläubigen Apologetik, sondern um ihre Einsicht in das, was in
einer enthusiastischen Zeit bei einem »Mann Gottes'' möglich ist,
zu erweitern und mehr Leben und Farbe der Wirklichkeit in ihre
oft so erstaunlich dürftige Auzlegerphantasie zu bekommen.
Aber nicht nur sür den biblischen Auöleger, für jeden Re-
ligion-Jforscher muß diese Quäkerselbstbiographie von höchster
Anziehungßkrast sein. Die noch junge Wissenschaft der Religions-
psychologie findet hier eineö ihrer allerinstruktiosten Dokumente.


% \picinclude{./vorwort/p_v07.jpg} 
Zur Einsührung. 711
Was unsere heutige Religionsforschung vor den früheren Zeiten
voraus hat, das ist ja eben die Wendung zu den ursprünglichen
religiösen Erlebnissen, während die frühere Forschung allzulange
sich bei der nachträglichen Verarbeitung dieser Erlebnisse in Dog-
men und Systemen aufgehalten hatte. Wir Theologen erkennen
heute, daß es für uns nichts Wichtigeres gibt, als aus die Personen
in der Geschichte zu lauschen, die Gott gehört und gesehen haben,
in denen also, wie der technische Ausdruck heißt, Religion aus
erster Hand uns vorliegt. Die schönste, sruchtbarste Religions-
psychologie der Gegenwart, William James ,,Religiöse Gr-
fahrung in ihrer Mannigsaltigkeitch hat ihren Wert darin, daß
sie den Zeugnissen aus erster Hand möglichst unvoreingenommen
nachgegangen ist. Zu ihnen gehört als eines der merkwürdigsten
eben das ,,Journal« des George Fox.
Man kann hier studieren, «wie die Bekehrung bei einem
solchen Mann Gottes vorgegangen ist. Alle ihre Vorbedingungen
läßt er uns erkennen, die Abstammung, das Milieu, den moralischen
Habitus vor der religiösen Krisis; nur eins, nicht das Umvichtigste,
sehlt in seinen Grinnerungen: die enthusiastische Zeit mit ihren
unerhörten weltgeschichtlichen und kirchlichen Umwälzungen, in die
Weingartens ,,Revolutionskirchen Englands-« immer noch die
klassische Einführung sind. Dann verfolge man den ,,Durchbruch«
selbst mit seiner ganze Jahre ausfüllenden Langsamkeit, dem
Wechsel der Seligkeitsgefühle mit den surchtbarsten anhaltendsten
Depressionen, den vielen pathologischen Begleiterscheinungen bis
zu dem Höhepunkt der Krisis, da Fox 14 Tage lang wie tot
daliegt, so verändert in Aussehen und Gestalt, als ob sein Körper
neu gebildet oder verwandelt wäre. Und dann als Folge das
souveräne Bewußtsein göttlicher Grwählung und Sendung, das
ihn keinen Augenblick in der Seligkeit der Gottesliebe ausruhen
läßt, sondern sofort ihn zu den Brüdern treibt, nicht um sie zu
bekehren, sondern um das schlummernde Bewußtsein des Gottes-
aeistes und seiner Kraft auch in ihnen zu wecken. Sein ganzes
Leben lang geht ihm dies Pneumatische nach, das- die Psychis
aker so gern in ihre Domäne ziehen möchten, während es für
ihn selber der Geist Gottes gewesen ist: plötzliche Stimmen, Ge-
stchte und Gefühle, Heilungen und Bewahrungen der mannig-
sachsten Art. Die Berichte darüber sind erstklassig wegen der er-
staunlichen Schlichtheit und Ausrichtigkeit dieses Berichterstatters,


% \picinclude{./vorwort/p_v08.jpg} 
7lll Zur Einführung.
der seine Wunder so natürlich erzählt, daß wir sie zu verstehen
glauben, und seine Gefichte und ihre Deutung, resp. Erfüllung
so auseinanderhält, daß er uns oft die Mittel der Kritik selber
in die Hand gibt. Nur davor darf vielleicht gewarnt werden,
sich zu einseitig auf die Sammlung dieser außerordentlichen
pneumatischen Erlebnisse zu beschränken. Das königliche Gott-
oertrauen in all den rasenden E-xzessen des englischen Pöbels,
in den schauerlichen Kerkern des damaligen Englands, in See-
sturm und Seeräubergefahr, in den Wäldern und Sümpsen
Nordamerikas breitet das Wunder über sein alltägliches Leben
aus. Dieser Mann scheint aus anderem Stoss zu sein und andere
Kräfte in sich zu tragen, als wir andere Menschen, wir verstehen,
daß man ihn für einen Zauberer hielt, wenn nicht so manche
Krankheiten, Hemmungen und Versuchungen rms wieder daran
erinnern würden, daß auch er ein Mensch gewesen ist.
Aber mir ist, als sehe ich ihn schon lange mit merkbarem Grimm
seine Erregung darüber bemeistern, daß er für uns eine historische
Merkwürdigkeit, ein religionspsychologisches Objekt geworden sei.
Soll das der ganze Wert meines göttlichen Auftrags gewesen
sein, euch interessanten Stoss für eure sogenannte Wissenschaft zu
geben? dazu mein Wahrheitszeugnis, meine Kämpfe und namen-
losen Leiden, meine Sammlung der Kinder Gottes in aller
Welt, damit ihr subtile psychologische und psychiatrische Unter-
suchungen an mir anstellen könnt? An das Licht und an den
Samen Gottes in euch appelliere ich: behandelt den lebendigen
Geist Gottes nicht wie einen Toten!
Welches ist der Platz des George Fox und seiner Quäker
in der Geschichte gewesen? Jndem wir das in Kürze feststellen,
wird deutlich, ob der Mami uns noch heute etwas zu sagen hat.
Weingarten hat einmal treffend das Quäkertum die geist-
liche Nachhut des Enthusiasmus der englischen Revolutionszeit
genannt. Ungeheure Bewegungen sind ihm vorangegangen, auf
denen es fußt, deren Gewinn es voraussetzt. Fox hat gut die
unpolitische neutestamentliche Ethik der Wehrlosigkeit und absoluten
Friedlichkeit predigen, nachdem zuvor der alttestamentliche Pari-
tanismus in einer gewaltigen kriegerischen Erhebung England
zur Vormacht des Protestantismus erhob; ohne das Heldentum
des Schwertes kein Raum für sein stilles, friedliches Heldentum.
Und ebenso hat Fox gut am Beispiel der puritanischen Revo-


% \picinclude{./vorwort/p_v09.jpg} 
Zur Einführung. 11
lutionßkirchen seine Kirchenkritik zu Ende denken, nachdem zuvor
der puritanische Kirchensturm daö prunkvolle Gebäude der anglis
kanischen Staatökirche mit ihrem ss- katholischen Apparat hinweg-
gefegt hatte; ohne die gewaltigen kirchlichen Reformationen und
Reduktionen der Puritaner keine Möglichkeit seine-3 antikirchlichen
Radikalißmuö. Auch der Gnthustaömuö, das Lauschen aus die
Stimmen des gegenwärtigen Gottes-geisteß, ist vor ihm in England
aufgetreten und hat seinen eigenen Enthusiaömuß angesteckt. Man
lernt auö seinen Auszeichnungen die Puritaner fast nur nach ihren
schlechten Seiten kennen; und doch ist der ganze Fox und sein
Quäkertum nur denkbar auf der Grundlage des puritanischen
Befreiung?-kampfß.
ES bleibt darum doch denkwürdig, daß es zu einem so
scharfen Gegensatz zwischen Fox und den Puritanerkirchen der
Preöbyterianer, Jndependenten und Täuser gekommen ist. Was
ist der Grund dieseß Kampfeß?
Die Puritanerkirchen erhohen sich alle auf objektiver, historischer
Grundlage. Daß historische Erlösung?-werk Christi war für sie
alle der Grund der Seligkeit und darum stand der Glaube, das
Bekenntniö, an der Spitze ihreß Christentum?-. Jin Ernstmachen
mit der absoluten Autorität der Bibel suchte jede Gemeinschaft
die andere zu überbieten, jede Kirche wollte reiner nach Gotteß
Wort geordnet sein. So wichtig ihnen auch die Reformation dez
Lebens war, der Nachdruck beim Einzelnen wie in der Offent-
lichkeit ruhte auf einem prononcierten Zur-Schau-stellen des Ve-
kenntnisseß, der Bibel, der kirchlichen Ordnung. EZ ist vielleicht
nie in der Geschichte so viel in der Bibel gelesen, so eifrig gebetet,
so lang und viel gepredigt worden, wie unter der Herrschaft des
Puritanertumß. Und da von der Reformation her die Lehre von
der auch im Ehristenstande bleibenden Sündhaftigkeit sich diesen
Frommen eingeprägt hatte, so lag ez allerdingtz nahe, zu meinen,
daß elementare wie feinere sittliche Gebrechen durch den geistlichen
Habituö, das Bekenntniö, genugsam aufgehoben würden; darin
liegt die Verwandtschaft des Puritanißmuö und jedes Pietißmuß
mit dem Pharisäertum.
An diesem Punkt setzt die Kritik, der Protest, der Gotteßzom
unseres- Quäkerß ein. Ich sehe seine Eigenttimlichkeit gar nicht
in seinem Gnthusiaömuß, sondern in seiner moralischen Gesundheit
und gründlichen Ehrlichkeit. Er scheint mir der ausrichtigste,


% \picinclude{./vorwort/p_v10.jpg} 
K Zur Einsühtung.
lauterste Mann seines Zeitalters zu sein und darin allerdings im
Sinn Thomas Carlyles ein ganzer Held. Er hatte einen
einzigen Sinn, den Sinn für Recht und Unrecht, vorausgesetzt,
daß man das Wort ,,Recht« in seinem weiten Sinn nimmt,
da es alle Liebe und Menschlichkeit in sich schließt. Wenn man
alles übersieht was er in seinem ganzen Leben angreift, wofür
er kämpft, es ist — ein paar Außerlichkeiten, die bei ihm sehr
innerlich gemeint waren, abgerechnet — immer die schlichte
natürliche Moral, für die er eintritt, Recht und Liebe und Treue
(Mt. 23): nicht lügen, nicht Unrecht tun, nicht schwören, sluchen,
stehlen, nicht Gottes Namen mißbrauchen, für alle Menschen den
Frieden und das Gute suchen und friedlich leben mit ihnen. Gr
ist jedesmal empört, wenn er sieht, daß Dienstboten am Lohn
verkürzt werden, daß Wirte ihre Gäste betrunken machen, daß
Steuereinnehmer die Armen bedrücken, daß arme Reisende lieblos
behandelt werden, daß bei einem Schifsbruch die benachbarte Ve-
völkerung sich auf den Raubstürzt, statt sich der Schissbrüchigen
anzunehmen. Aus eigener Anschauung lernte er die Sünden des
englischen Rechts und Strafwesens kennen: die lange Ver-
schleppung der Prozesse, die vorschnelle Füllung von Todes-
urteilen wegen unwichtiger Vergehen in Geldsachen oder das
Vieh betreffend, den Mißbrauch des Eides, die schauerlichen
Gefängnisse Tmit den barbarischen Kerkermeistern, die Ansteckung
der Gefangenen durch die schlechte Gesellschaft, die Unterschlagung
der für die Gefangenen bestimmten Speisen, die gänzliche Ver-
wahrlosung der Familien der Gefangenen während ihrer Ge-
fangenschaft. Wahr ist, daß sich ihm dabei zuweilen Unwichtiges
als wichtig saufdrängte und er aus dem Duzen aller Menschen
und dem Aufbehalten des Hutes selbst vor den Richtern mit
einem Gigensinn bestand, den wir bei Jesus und selbst seinen
Jüngern nicht finden. Aber diese Außerlichkeiten der Konvention
hat er eben anders betrachtet; er verstand nicht und konnte nicht
verstehen, wie Menschen, die doch alle Brüder sind, künstlicher
Formen unter einander bedürfen. Daß er dann in seinem Recht-
sinn schlechterdings keinen Unterschied zwischen Frauen und Mün-
nern, zwischen Engländern und Negern oder Jndianern machen
kann, daß er sie alle schlechtweg als Menschen nimmt mit dem
vollen Anspruch aus menschliche Behandlung, braucht kaum hin-
zugefügt zu werden. Gr nimmt das alle-3 auch gar nicht als


% \picinclude{./vorwort/p_v11.jpg} 
Zur Einführung. Il
christlich in Anspruch; das Licht, das einen jeden Menschen er
leuchtet, das Gewissen, wird in allen Menschen dasselbe Recht
und Unrecht erkennen müssen.
Aber da drängte sich ihm nun die entsetzliche Frage auf;
wo ist bei den Christen, bei diesen Puritanern und Frommen
(Bekennern), diese Wahrhaftigkeit, Menschlichkeit und Liebe? Als
er selbst noch Trost bei einzelnen ihrer Führer suchte, erlebte er
nichts als Enttäuschungen; der eine schwatzte seine Leiden und
Bekümmernisse den Dienstboten aus, ein anderer geriet mitten
im Gespräch, als Fox aus Versehen auf den Rand eines Garten-
beetes trat, in solche Wut, als ob sein Haus in Flammen stünde.
Sie besaßen das gar nicht, was sie bekannten. Und später machte
er die Beobachtung, daß ihm und seinen ,,Freunden« aus diesen
Puritanerkirchen die roheste, gemeinste Verfolgung erwuchs, die
sich in Amerika, in den puritanischen Musterländern, bis zur
Hinrichtrmg einzelner Quäker steigerte. Diese Kirchen, die eben
aus jahrzehntelanger Verfolgungszeit zur Freiheit gelangt waren,
zeigten sich genau so tmduldsam, iso heerschsüchtig- so pfäfsisch
wie ihre früheren Verfolger, Katholiken und Anglikaner. Man
kann dies Urteil als einseitig beanstanden, als einen Ausf-luß des
sektenhaften Richtgeistes, der sich der Quäker bemächtigte; so wie
es hier aussieht, haben sich Licht und Finsternis nicht verteilt,
man denke nur an Richard Baxter und an Oliver Cromwell.
Aber daß Fox auch guten Grund zu dieser Beurteilung hatte, wer
wird das leugnen? Eine so ausgesprochen sromme Bewegung
wie der Puritanistnus fordert den strengsten Maßstab heraus.
Von da aus stellte sich ihm das scharfe Entweder — Oder
einer doppelten Frömmigkeit aus: die der frommen Formen und
Worte, und die der Kraft. Auf der einen Seite standen ihm alle
vorhandenen Religionen, Puritaner und Anglikaner und Katholiken
allesamt, deren Unterschiede doch nur in den Formen bestehen,
aus der andern Seite das, was Gott will, wozu Jesus in die
Welt gekommen ist. Fromm sein, das heißt die Kraft Gottes
besitzen, von ihr allein beherrscht werden, dies, und dies allein.
Gr nannte diese Kraft den Geist oder mit Vorliebe den Samen
Gottes, glaubte kühn, daß in jedem Menschen dieser Same oet-
borgen sei, eben als das Zeugnis seines Gewissens, und daß der
ein Christ sei, in dem der Same durch Gottes Wunder lebendig
und mächtig über alles geworden sei. Nen waren diese Gedanken


% \picinclude{./vorwort/p_v12.jpg} 
Xll Zur Einführung.
nicht, wir finden sie in der Resormationszeit besonders bei
Sebastian Franck und später bei Jakob Böhme, der in der
Zeit des Fox auch in England englisch gelesen wurde. E3
kommt aber nicht aus die Priorität an, sondern darauf, daß sie
hier bei Fox mehr als Gedanken waren, daß sie die Lebenskraft
einer ganzen Gemeinschaft wurden. Und sie traten hier nicht wie
in Deutschland zu einem toten Kirchentum und einer starren
Orthvdoxie in Gegensatz, sondern zku den lebendigsten, jugend-
srischesten Puritanergemeinschaften. tzhnen galt der für fromme
Ohren wahrhaft entsetzliche Kriegsrus: nicht Bibel, nicht Ve—
kenntnis, nicht Kirchen, sondern allein der Geist, der Gott, der in
uns selber als Lehrer tmd als Kraft lebendig ist!
Es war eine höchst gefährliche Losung, die Fox damit auf-
nahm, die Losung aller Schwarmgeister und Fanatiker, aus der
von Jahrhundert zu Jahrhundert die unheinilichsten und grau-
sigsten Exzesse der Religionsgeschichte geboren worden sind. Wie
leicht verbergen sich dunkles Triebleben und oerworrene mensch-
liche Einbildung unter dem hohen Titel des Geistes Gottes! Als
Fox auftrat, wimmelte es in England von Enthusiasten aller
Art, entfesselt durch die allgemeine Emanzipation des Revolutions-
zeitalters. Sie traten mit Träumen, Gesichten und Stimmen aus,
gaben sich selbst für Christus aus und erklärten, sündlos zu sein.
Alle Geschichte war ihnen bloßes Symbol ihrer eigenen Erlebnisse,
man hörte geradezu die Leugnung, daß Jesu Tod eine geschichtliche
Tatsache sei; sein Leiden sei ja in uns. Diese Gott- und Christus-
trunkenen Schwärmer wurden von den Kirchlichen Runter,
d. h. Prahler, genannt. Und bevor der Quäkername sich allgemein
verbreitete, sind auch Fox und seine ersten »Freunde« als Ranter
angeschrien und verfolgt worden. Obschon Fox von Anfang an
dagegen protestierte, es ist Tatsache, daß die ersten Quäker und
die Ranter sich wie ein Haar vom andern unterschieden, daß
einige der hervorragendsten Quäker den Rantergeist nicht los
geworden sind. Der messianische Einzug des James Naylor
in Bristol ist ein echtes Ranterstück. Und äußerlich betrachtet, i
wer will die Ossenbarungen des Fox von den Eingebungen der
Runter unterscheiden?
Aber während diese Ranter spurlos und namenlos in der
Geschichte der religiösen Schwtirmerei wieder untergegangen sind,
bilden die »Freunde« bis heut eine blühende religiöse Gemein-


% \picinclude{./vorwort/p_v13.jpg} 
Zur Einsührung. Xlll
schaft von charakteristischer Eigenart. Daß kommt daher, daß sie
die Niichternen, die moralisch Gesunden in dem enthusiastischen
Wirbelsturm waren. Der Gnthusiaßmus ist bei ihnen nur die
Form, die Grstlingöform, in der ihre neue moralische Kraft sich
manifestiert, nicht anders, als eß beim Urchristentum der Fall war.
Man erkennt an diesem Enthusiasmus die absolute Energie, mit
der sie von ihrer Wahrheit erfaßt waren, mochte die ganze Welt
widersprechen. Sobald man aber aus den Jnhalt achtet, stößt man
aus jene schlichte Menschlichkeit, das- Einfachste und Nüchternste, was
jemalß Grweckungßprediger gefordert haben. Fox ist kein Schwärmer
gewesen, obschon ihn die von ihm erkannte Wahrheit beherrschte
wie eine tiefe Schwärmerei. In keinem Augenblick seineß Lebenß
hat er seinen klaren Sinn für Recht und Unrecht, gut und böse
verloren. Daß rettete ihn an schwindelnden Abgrtinden vorbei
und durch alle Verlorkungen des religiösen Wahnsinnß, dem manche
seiner Genossen erlagen. Er war vielleicht nicht immer Herr über
die von ihm entsachte Bewegung, wie ihm überhaupt das Herrscher-
talent abging, aber er war immer Herr über sich selbst. Daß
ist der eine Grund, daß daß Geistprinzip ihm nichts geschadet
hat: seine gründliche moralische Festigkeit und Gesundheit. Dazu
kommt aber, daß der Gegensatz zum historischen Christentum nicht
von ferne so tief war, wie ihn die Kampslosung deß Fox: ,,nicht
die Bibel, sondern der Geist« könnte erscheinen lassen. Kein
Mensch seiner Zeit hat mehr in seiner Bibel und auö seiner
Bibel gelebt alß eben Fox; seine individuelle, kernige Sprache ist
ihm sogar durch den biblischen Dialekt ganz abhanden gekommen.
Selbst in seinen eigenen Liebling;-’gedanken steht er aus dem festen
Grund dez Reformationöeoangeliuniß von der den Menschen
durch Gottes Gnade geschenkten und durch gar kein Eigenwerk
von ferne zu oerdienenden Erlösung. Jn allem, wat; er redet und
tut, beruft er sich aus das- Wort Jesu und der Apostel, und daß
cthische Jdeal, daß er darauß ableitet, berührt sich ausß engste
mit dem täuferischen Ideal der Friedsamkeit und Wehrlosigkeit,
von dem die Täufer selbst sich in den Reoolutionökriegen hatten
abdräugen lassen. So oerleugnet er nirgends die Kontinuität
mit dem Christentum der von ihm so radikal verworsenen Kirchen;
er ist auch schon viel zu bescheiden und ehrlich, um den Anspruch
zu erheben, autz dem Geist Gotteö heraus ein neuer Religionß-
stifter zu sein. Was er eigentlich will, ist nur die radikale Re-


% \picinclude{./vorwort/p_v14.jpg} 
W1 Zur Einführung.
formation in Tat und Leben, das Emftmachen mit Jesu Wort
und Geist in rücksichtslosem Kampf mit allem, was Welt und
Tradition dariiberlegten. Stellt er also Geist und Bibel in so
scharfen Gegensatz, so meint er letztlich zwei Dinge: das Recht
des Laienverständnisses der Bibel im Gegensatz zum theologischen
Privileg — nicht Gelehrsamkeit, sondern allein Frömmigkeit kann
Gott recht verstehen — und die Notwendigkeit, die Kraft der
biblischen Religion im Leben zu beweisen, statt im Besitz und
Lesen des Bibelbuchs. So sbetrachtet, ist er keine Jnstanz, die
sich gegen den Wert der Bibel und des historischen Christen-
tums anführen läßt, sondern ein Zeugnis der Lebenskraft,
welche aus der Berührung der Geschichte — Gottes in der Ge-
schichte — mit einem aufrichtigen, tapferen Menschenherzen quillt.
So ist auch, was wir heute brauchen, keine neue Offenbarung
verborgener Seiten Gottes, sondern ein ganz anderes Grnstmachen
mit der uns in der Geschichte geschenkten Erkenntnis Gottes und
unserer Pflicht, als das Namenchristentum es kennt.
Damit scheint mir das Wesentliche zum Verständnis des Fox
angedeutet zu sein. Höchstens fehlt noch ein ganz auffallender
Punkt: sein massiver Vergeltungsglaube. Jn seiner sittlichen
Forderung iiberschritt er bewußt die alttestamentliche Stufe und
lebte die Ethik der Bergpredigt wie wenig Christen vor ihm.
Aber in seinem Vergeltungsglauben, der ihn mit größtem Eifer
und innerer Zufriedenheit die jeweiligen Strafen der Quäker-
Verfolger, besonders die auffallenden, plötzlichen Gottesgerichte
notieren ließ, kommt er uns seltsam alttestamentlich zurückgeblieben
oor. Für ihn war das die notwendige Ergänzung seines extremen
Spiritualismus. Der Gott, der ihn innerlich bewegte, war der-
selbe, der die Vorfälle des äußeren Lebens in seiner Hand hielt
und durch sichtbare Strafen oder Segnungen kundgab, auf welcher
Seite das Recht war. Genau so haben es Cromwell und die
Jndependenten geglaubt. Deshalb ist es Fox doch keinen Augen-
blick eingefallen, etwa nun auch seine und der ,,Freunde« Leiden
als Strafen Gottes aufzunehmen und nach der sie verdienenden
menschlichen Verschuldung zu fragen. Die entsetzlichsten Miß-
handlungen und die empörendsten Ungerechtigkeiten der Justiz
nahm er als Kind Gottes gelassen und sogar fröhlich auf, ohne
an eine Strafe Gottes dabei zudenken. Das sind Jnkonsequenzen.
Man kann es doch auch wohl verstehen, daß ein Mann mit


% \picinclude{./vorwort/p_v15.jpg} 
Zur Einführung. 17
solchem Rechtsfmn so fest sich an den Glauben an eine moralische
Weltordnung auch im Außern geklammert hat.
Die Geschichte des Fox und der Freunde, die diese Auf-
zeichnungen uns vorführen, zerfällt in zwei deutlich unterschiedene
Perioden. Zuerst die Sturm und Drangzeit, die Periode des
extremen Enthusiasmus und der extremen antikirchlichen Agitation.
Es ist die Zeit der ersten Liebe, des größten Heroismus, aber
auch einer rohen Unreife, die sich später in den langen Leidens-
jahren korrigiert. Fox ist damals ein Kirchenstürmer der wildesten
Art gewesen, er ging darauf aus, die Leute aus den Puritaner-
kirchen und von den Puritanerpfarrern weg zu reißen mit allen
Mitteln der Agitation. Wenn sich dann der ganze Haß der
Pfarrer und ihrer Anhänger in der brutalsten Weise über ihm
entlud, so ist ihm das sicher nicht unnerschuldet begegnet, wenn
er auch mit Recht darin eine wunderliche Manifestation des puri-
tanischen Christentums, dieser Gxtrafrömmigkeit, sah. Es bestand
für ihn selbst vorübergehend die Gefahr, daß das Nein in der
Agitation das Ja, das Recht und die Liebe, übertöne. Es bildete
sich damals ein Richtgeist unter den Freunden aus, der sie in
allen andern Gemeinschaften Babylon und den Antichrist erblicken
ließ, während sie allein Jerusalem vertraten. Mit Cromwells
Protektorat standen sie auf gespanntestem Fuß; es ist nur mensch-
lich, daß dem toleranten Mann ihnen gegenüber wiederholt die
Geduld ausging. Dazu die gräßliche Schwärmerei des James
Naylor, eines der Führer der Bewegung, und auf der andern
Seite die Phantastik der Weltmission, die gleich den Papst in
Rom und den Sultan in Konstantinopel zu bekehren strebte und
Schreiben an alle Potentaten der Welt ergehen ließ, damit sie
dem ,,Kommen des Herrn« nichts in den Weg stellten. Man
kann diese Zeit in ihrer Weise mit der Sichtungszeit der
Brüdergemeinde vergleichen, es ist fast ein Wunder, daß die
Gemeinschaft diese Schwärmerei überstand. Nun, sie haben auch
dafür gelitten, und unter den Blättern aus der Geschichte religiösen
Heldentums ragen zweifellos die Erzählungen des Fox aus dieser
ersten Zeit immer hervor.
Die zweite Periode, welche ungefähr mit der Restauration
des Königtums einsetzt, ist dann die Periode der Ernüchterung
und der Organisation. Erst jetzt beginnt im vollen Sinn ihr
unschuldiges Leiden, nachdem das frühere so vielfach durch eigene


% \picinclude{./vorwort/p_v16.jpg} 
171 Zur Einführung.
Schuld provoziert worden war. Nicht nur hielten sie sich inder
politischen Neuordnung durchaus neutral, ja grundsätzlich un-
politisch, sodaß auch nicht der Schein einer Gefahr für den Staat
bestand, ihre Kirchenstürmerei hatten sie längst in dem Maß auf-
gegeben, als sie sich zu selbständigen Gemeinschaften mit eigenem
,,Gottesdienste« zusammenschloss en; sie taten schlechterdings niemand
etwas zu leide. Dennoch hat die Verfolgung seitens des neu-
befestigten anglikanischen Staatskirchentums gerade sie mit beson-
derer Härte getroffen, z. T. wegen ihrer absoluten Eidverweigerung,
die ihnen schlimm aus-gedeutet werden konnte, und hat dadurch
dazu beigetragen, daß auch der letzte revolutionäre Gedanke sich
verlor, und gar nichts anderes übrig blieb als die gänzliche
Wehrlosigkeit und Leidsamkeit, durch die einst das alte Täufertum
sich ein bleibendes Andenken in der Geschichte erwarb. Der
stürmische Enthusiasmus war verflogen, aber der stille Enthusias-
mus, mit dem das Gotteskind alles Leid, das Menschen ihm
antun können, friedlich, unverbittert, ja selig im Grunde, hinnimmt,
blieb als die Frucht der großen Zeit. Zugleich aber ist diese
Leidenszeit die Zeit des Bauens, der Organisation. Aus der
kirchenstürmerischen Bewegung geht selbst eine neue Kirche — Fox
selbst braucht den Ausdruck Kirche dafür — hervor, und das
Erstaunlichste ist, daß nicht etwa Epigonen im Gegensatz zur
ursprünglichen Tendenz des Stifters diese Verkirchlichung durch-
setzen, sondern daß der Begründer des quäkerischen Enthusiasmus
auch der kirchliche Organisator ist. Zuerst hatten sich die Jahres-
versammlung und die Vierteljahrsversammlungen eingebürgert zu
wichtigeren Beratungen, während man am ,,Ersten Tag'' (dem
Sonntag) zu zwanglosen Aussprachen aus der Eingebung des
Geistes zusammenkam. Das war ein Anfang von Ordnung, aber
dem einzelnen blieb eine ungebundene Freiheit während des
ganzen Vierteljahrs. Da tat Fox im Jahr 1666 den entschei-
denden Schritt zur geschlossenen kirchlichen Organisation mit der
Einrichtung von Monatsversammlungen sowohl fiir die Männer
als für die Frauen, vornehmlich zur Durchführung der Kirchen-
zucht gegen unordentliche Mitglieder, auch zur Regelung der
Quäkerehen 2c. Er hat damals ganz England im Jnteresse
dieser Organisation bereist und alle Quäker im Ausland, auf
dem Kontinent, in Jrland, Schottland und Amerika, zur Nach-
ahmung dieser Organisation aufgefordert. Eine große Reformation


% \picinclude{./vorwort/p_v17.jpg} 
Zur Einführung. 1711
des Quäkertumß leitet er selbst von dieser neuen Verfassung her.
Allein es fehlte nicht an ganz energischem Widerstand aus den
Kreisen der »Freunde«, welche die christliche Freiheit durch eine
neue Menschensatzung bedroht glaubten, in der Kirchenzucht ein
uneoangelischeß Richten aufkommen sahen und speziell von der
den Frauen in dieser Organisation gewährten Stellung nichtö
wissen wollten. EZ ist kein Zufall, daß in diesem Zusammenhang
wieder von den Rantern die Rede ist. Der alte Rantergeist, der
extreme Subjektiviömuß und Jndividualiömuß, sah sich durch diese
kirchliche Organisation in-8 Herz getroffen. GS berührt in der
. Tat seltsam, wenn man den alten Fox jetzt den göttlichen Ursprung
dieser »eoangelischen Ordnungen« verfechten sieht; war das noch
der Prophet und Enthusiast von ehemal-Z? Und doch ist er nicht
von sich abgefallen, als er für seine Gemeinschaft die für ihren
Bestand notwendigen Formen schuf. Jndioidualist im extremen
Sinn war er nie gewesen, sondern von Anfang an Gemeinschaftß-
mann, und darum Mann der Liebe und Ordnung. Er hat einfach
gelernt, was- jeder gereifte Mensch einmal lernen muß, daß der
Geist zerfließt und zerflattert, wenn er nicht durch Organisationen
sich einen dauerhaften Körper geben kann. Waß müßten wir
heute vom Quäkertum ohne diese kirchliche »Reformation«! Zudem
ist die Quäkerorganisation unter allen mir bekannten kirchlichen
Ordnungen die freieste, formloseste geblieben. Gar kein Glaubens-
bekenntnis, und daß Grftaunlichste — keine Sakramente! Der
Gottesdienst so, daß man zusammenkommt, auf den Geist wartet,
und wenn einmal der Geist niemand zum Reden treibt, sich nach
stiller Versammlung die Hand gibt und friedlich autzeinandergeht.
Bekenntnis und Verfassung und Kultuö sind hier nichtß, das
Leben ist allez, und will und soll sein, wa-3 es- von Anfang war:
Recht und Liebe und Treue, schlichte Menschlichkeit.
Man soll eß nie vergessen, daß —— nächst dem einzelnen
Roger Williams- —— die Quäker die ersten waren, die mit einer
unterschiedölosen religiösen Toleranz praktisch ernst machten, nicht
aus Jndifferenz sondern aus Glauben. Andere, wie die Puritaner,
hatten nur, solang sie selbst verfolgt waren, Toleranz begehrt,
und, zur Herrschaft gelangt, sie schmählich verleugnet. In Penn-
stlvanien hat tatsächlich jeder seineö Glaubenö frei gelebt. Hier
bei den Quäkern zuerst ist den Frauen die volle kirchliche Gleich-
stellung mit dem Mann gegeben worden; ez gibt kein doppeltes


% \picinclude{./vorwort/p_v18.jpg} 
Illlll Zur Einführung.
Recht vor Gott. Von Fox erging während seiner amerikanischen
Reise die Mahnung, die Negersklaven mild und freundlich zu
behandeln und sie frei zu lassen, nachdem sie einige Jahre gedient.
Die Quäker sind damit im Kampf gegen die Sklaverei Voran-
gegangen. Und wenn mehr alz ein Jahrhundert später Elisabeth
Fry alz Resomiatorin deö Gesängnißwesenß England und den
Kontinent durchzogen hat, so sehen wir auß den Aufzeichnungen
des Fox, daß sie nur seine Aufgabe zu Ende führte. EZ gibt kein
großes Werk der Menschlichkeit und Barmherzigkeit, an dem nicht
die Quäker beteiligt sind, und daß nicht letztlich in dem wurzelt,
waz Fox ale die Kraft des Samenö Gotteö erkannte.
Daß ist doch mehr als eine historische oder religionöpsycho-
logische Merkwiirdigkeit, ee ist die beste Kraft dez Evangelium-3,
es- ist Jesuß selbst, der hier wieder einmal die Umhiillungeu, in
denen ihn menschliche Schwachheit und Kleinglaube konferoieren,
konseroieren mtiss en, frei herau?-tritt, um die Menschen einen ganzen
großen Schritt auswärtz zu führen in der Richtung auf sein
Gotteßreich. Wir in der Schweiz und in Deutschland sind nicht
Quäker und werden auch nach unserer Eigenart keine Quäker-
gemeinschaften bilden, aber wir sind Jünger dez Evangeliumß
nur dann, wenn wir ganz allein daß wollen, wa-3* die Quäker
wollten, ein Leben in der Kraft Gotteß statt in den Formen
und Worten, und die Richtung der Kraft: Recht und Liebe
und Treue, Menschlichkeit. Viele haben gemeint, daß die Quäker
doch keine rechten Christen seien, weil sie gar keine Sakra-
mente haben. Aber dem steht daß Wort Jesu entgegen: an
den Früchten sollt ihr sie erkennen! Hätte eine unserer Kirchen
solche Früchte wie die Quäker!
Daß nach dem Tode dee George Fox von William Penn
heraußgegebene Journal, auß dem im folgenden eine Auswahl
gegeben wird, ist nicht, wie der Titel vermuten ließe, ein wirk-
liches Tagebuch, sondern eine zusammenhängend geschriebene
Selbstbiographie, die allerdingß tagebuchartige Notizen voraußsetzt.
GZ ist ein Werk deß Alter;-’, geschrieben in der Restaurationözeit,
wie ich vermute, etwa im Jahre 1677, als Fox sich überhaupt
an die Sammlung und Ordnung seiner älteren Dokumente machte,
und dann in den folgenden Jahren noch ergänzt. Für seine
relativ einheitliche Abfassung spricht einmal der durchweg einheit-
liche Stil, der gar keine Wandlungen aufweist, sodann die Hinzu-


% \picinclude{./vorwort/p_v19.jpg} 
Zur Einführung. 111
fügung einer ganzen Reihe späterer Notizen bei viel früheren
Jahrgängen, vor allem der religiöse und kirchliche Geist des
Ganzen. Die Anerkennung Karl Stuarts als des von Gott be-
stimmten rechtmäßigen Königs beherrscht das ganze Buch, von
dem ursprünglichen Enthusiasmus ist wohl die Erinnerung be-
halten, aber aus ihm heraus geschrieben ist keine Seite, ja es
läßt sich eine gewisse apologetische Absicht nicht verkenneu, auch
das Quäkertum der Vergangenheit als politisch harmlos und in
jeder Weise ungefährlich hinzustellen. Die Verirrungen einzelner
Quäker, z. B. des James Naylor sind behutsam angedeutet, aber
man gewinnt keinen Eindruck, wie kritisch sie damals siir das
Quäkertum gewesen sind. Die Vermengung der Quäker mit den
Rantern wird von Anfang an säuberlich abgewehrt, die wirkliche
Entstehung des Quäkernamens aus den krankhasten Konvulsionen
der ,,Freunde« (Zitterer) wird verdeckt. Etwas Unwahres möchte ich
in dieser Darstellung nicht sehen, wohl aber da und dort eine
unwillkiirliche Verschiebung, veranlaßt durch die eigene Ernüchterung
und den Zwang, sich der Anklagen und Verleumdungen zu er-
wehren. Jch glaube, daß das, was erzählt ist, immer historische
Wahrheit ist; aber ob immer alles, zumal aus der Sturm- und
Drangzeit, erzählt ist, was man später noch wußte, muß für uns
dahin gestellt bleiben; über James Naylor wußte Fox jedenfalls
noch mehr. Aus dem Gedächtnis aber kann er diesen unermeßlich
reichen, im einzelnen so detaillierten Stoss nicht niedergeschrieben
haben. Eine eigene Aufzeichnung der Stationen muß ihm vor-
gelegen haben und zugleich wohl kurze Notizen über die merk-
wiirdigsten Erlebnisse an den einzelnen Orten. Leider hatte er
gar kein Jnteresse an der Chronologie; Jahreszahlen finden sich
im ganzen Buch nur in den mitgeteilten Dokumenten, eigenen
Briefen, Hastbefehlen 2c. Dazwischen erwähnt er jedoch eine
Reihe weltgeschichtlicher Begebenheiten, die der Quäkergeschichte
doch ein gewisses chronologisches Gerippe geben und deshalb in
der Übersetzung mit Fleiß gesammelt sind.
Die Beschränkung aus eine Auswahl ergab sich uns statt
einer ganzen Ubersetzung mit Notwendigkeit, weil das Ganze so
gut wie keine Leser gesunden hätte. Nicht nur des Umfangs
wegen. Die Erzählung wiederholt sich unendlich, und die ein-
gelegten Briefe sind von einer ermiidenden Breite und Monotonie.
Fox ist doch sicher ursprünglich ein origineller Laie gewesen mit



% \picinclude{./vorwort/p_v20.jpg}
II Zur Einführung.
realistischem Ausdruck und oft ungewöhnlichen: Mutterwitz.
Allein die Notwendigkeit, 40 Jahre lang unaufhörlich reden zu
müssen, und eigentlich doch immer dasselbe, hat seine Originalität
stark vermindert und ihm in Sprache und Schrift die Monotonie
verliehen, die im allgemeinen den Gemeinschaft-'spredigern nach-
zugehen pflegt. Einen wesentlichen Vorzug wird man ihm gerade
deöhalb doch zugestehen müssen: er erzählt durchaus schlicht, un-
gesucht, und daz gibt der Sache, die er erzählt, eine um so ge-
waltigere Wirkung, eZ steckt ’auch nicht ein Schimmer Eitelkeit
darin. Die Ubersetzerin — es ist die Tochter des Basler Kirchen-
historikertz und Zwinglibiographen Rud. Stähelin — hat sich
bemüht, so schmuckloö schlicht zu erzählen wie er selbst und die
Sache durch sich selbst reden zu lassen. Aufgenommen haben wir
alles, waß unß sür die Charakteristik deö Fox und die Geschichte
des Quäkertumö wesentlich schien, speziell auch möglichst alle
religiösen Merkwürdigkeiten und die zerstreuten welt- und kirchen-
geschichtlichen Notizen, welche die Verbindung mit der allgemeinen
Geschichte ermöglichen. In dieser beschränkten Außwahl, die im
Grunde doch alletz Wesentliche wiedergibt, wird, wie wir nicht —
zweifeln, die Lektüre unsereö Bucheß Vielen Genuß bringen, und,
hoffen wir, etwaö Besseres alß Genuß. Carlyle hat einmal im
Sartor Resartuz das merkwürdigste Greigniö der neuem Geschichte
den George Fox genannt, der sich einen Anzug von Leder machte.
Wer diese Aufzeichnungen lesen wird, der wird seine Paradoxsie
verstehen. P. Wernle.


\picinclude{./000-009/p_s001.jpg}
\section{Kapitel 1}

\begin{center}
\textbf{Erweckung und Krisiz bis zum Durchbruch.}
\end{center}


Auf daß Jedermann wisse, was der Herr an mir getan, und
sehe, wie Er mich durch mancherlei Prüfungen, Versuchungen und
Trübsale führte, um mich für daß Werk, für daß; Er mich bestimmt 
hatte, vorzubereiten und auszurüsten, und dadurch getrieben 
werde, seine unendliche Güte und Weisheit anzubeten und
zu preisen — so will ich kurz berichten, wie es in meiner Jugend
um mich stand, und wie das Werk des Herrn in mir angefangen
und fortgesetzt wurde seit meiner Kindheit.


Ich wurde geboren im Monat den man Juli nennt\footnote{Fox 
verwarf die üblichen Monatsbezeichnungen als heidnisch.} 1624,
zu Drayton in-the-Clay, in Leicestershire. Mein Vater hieß
Christoph Fox; er war Weber von Beruf, ein ehrbarer Mann,
und es war ein "`Same von Gott"` in ihm. Die Nachbarn
nannten ihn: den "'gerechten Chrtster"'. Meine Mutter war eine
rechtschaffene Frau; ihr Mädchenname war Mary Lago, aus der
Familie der Lago und aus dem Geschlecht der Märtyrer.

In meiner frühesten Kindheit war ich so ernsten und gesetzten
Gemütez, wie es bei Kindern selten ist, so daß, wenn ich Erwachsene 
leichtfertig und ausgelassen mit einander tun sah, ich
einen Abscheu davor in meinem Herzen verspürte und zu mir
sagte: "`Wenn ich einmal ein Mann sein werde, sicherlich werde
ich nicht so leichtfertig tun."'

A1s ich elf Jahre alt war, wußte ich schon was rein und
recht ist; denn ich war als Kind gelehrt worden, wie man rein
bleibt. Der Herr lehrte mich, treu zu sein in allen Dingen, sowohl
innerlich gegen Gott als äußerlich gegen die Menschen; und daß
ich mich in allen Dingen an "`ja"` und "`nein"` halten solle; nicht
wie die Kinder der Welt, die ihren Mund voll List und gleißnerischer
Worte haben, sondern meine Worte sollen: wenig sein, "'lieblich


\picinclude{./000-009/p_s002.jpg}


und mit Salz gewürzet"` (Col. 4, 6); und daß ich nicht essen
und trinken solle, um mich wollüstig zu machen, sondern um der
Gesundheit willen, jeder Ding dazu gebrauchend, wozu es be-
stimmt ist, zur Ehre dessen, der alleß geschaffen hat ....
Alß ich dann heranwuchß, wollten meine Angehörigen einen
Priester 1) aus mir machen. Aber andere rieten zu anderm; so
kam ich zu einem, der seineö Zeichenz ein Lederhändler war, aber
mit Wolle handelte und Vieh züchtete und verkaufte; und es ging
mancherlei durch meine Hände. . Während ich bei ihm war,
war er gesegnet; aber nachdem ich ihn Verlassen, ging ez ihm
schlecht und er getiet in Verfall. Während dieser ganzen Zeit
tat ich weder gegen einen Mann noch gegen eine Frau etwas-
Unrechteö; denn die Kraft dez Herrn war mit mir und bewahrte
mich. Während ich in diesem Dienste stand, gebrauchte ich im
Verkehr daö Wort "'wahrlich"', und es war eine übliche Redenöart
bei meinen Bekannten: wenn George sagt "`wahrlich"`, so kann
ihn nichts umstimmen. Wenn die Buben oder rohe Leute über
mich lachten, kümmerte ich mich nicht um sie, sondern ging meiner
Wege; aber gewöhnlich hatten mich die Leute gem wegen meiner
Geradheit und Ehrlichkeit.
Alß ich, noch nicht ganz neunzehnjährig, in Geschäften an
einem Jahrmarkt war, kam mein Vetter, namenö Bradford, ein
"'Frommer"' cpt0keS801·) und mit ihm noch ein anderer "`Frommer"'
und forderten mich auf, mit ihnen einen Krug Bier zu trinken,
und da ich durstig war, ging ich mit ihnen hinein; denn ich
liebte jeden, der Sinn für daß Gute hatte und den Herrn
suchte. A13 jeder ein Glas getrunken hatte, fingen sie an, sich
zuzutrinken und verlangten noch mehr, indem sie aus-machten,
daß der, welcher nicht trinken würde, alletz bezahlen sollte. GS
betrübte mich, daß jemand, der sich für religiöß außgab, solchetz
tat; sie taten mir sehr weh, denn ee; war mir dergleichen noch
nie vorgekommen bei keiner Art von Menschen; darum stand ich
aus um zu gehen, indem ich meine Hand in die Tasche steckte,
einen Groschen vor sie aus den Tisch legte und sagte: "`wenn ez
so ist, will ich euch Verlassen."' So kehrte ich nach Hause zurück,
aber ich ging in jener Nacht nicht zu Bett, denn ich konnte nicht
schlafen; bald ging ich im Zimmer auf und ab, bald betete und
1) Fox bezeichnet mit priest die ordinierteu Geistlichen.


\picinclude{./000-009/p_s003.jpg}


schrie ich zum Herrn, welcher also zu mir redete: "`Tu siehst, wie
junge Leute zusammengehen in Eitelkeit und alte Leute in die
Erde. Du mußt dich von ihnen abwenden und dich von ihnen,
den jungen wie den alten, fern halten und ihnen allen ein
Fremdling werden."'
Darauf, am 9. Tage dee; 7. Monatz 1643, verließ ich nach
Gottes Befehl meine Verwandschaft und brach allen Umgang und
alle Kameradschaft mit jung und alt ab. Ich begab mich nach
Lutterworth, wo ich einige Zeit blieb und von da ging ich nach
Northampton, wo ich mich ebensallß aufhielt; darauf nach New-
port Pagnell, von wo ich nach einiger Zeit weiter nach Barnet
ging, im 4. Monat 1644. Als ich nun so da; Land durchzog,
wurden die "`Frommen"` (prokeesore) auf mich aufmerksam und
wollten mich kennen lernen. Aber ich mied fie; denn ich spürte,
daß sie nicht besaßen, was sie bekannten (proteezeä). Während
der Zeit, da ich in Barnet war, kam eine große Anfechtung zu
verzweifeln über mich. Ich sah, wie Chriftuö versucht worden
war, und war in großer Not; bald ging ich nicht aus meinem
Zimmer, und bald wanderte ich einsam durch die Fluten, um auf
den Herrn zu warten.
Ich fragte mich, warum mir solcheö widersahren müsse? Ich
prüfte mich und sagte zu mir selber: "'War ich je zuvor so ge-
wesen?"` Ich dachte, ich hätte mich vielleicht gegen meine An-
gehörigen verfehlt, weil ich sie verlassen hatte. Ich mußte immer-
während darüber nachdenken, daß ich solches getan hatte, und
mich fragen, ob ich einem von ihnen ein Unrecht getan hätte;
aber die Anfechtung wurde schwerer und schwerer, und ich wurde
bis zur Verzweiflung versucht. Und weil Satan sein Vorhaben
auf diese Weise nicht erreichte, so legte er mir Fallstricke und
Lockungen, damit ich eine Sünde begehen möchte, die er auß-
nützen könnte, um mich zur Verzweifluug zu bringen. Ich war
etwa 20 Iahre alt, als diese Prüfungen über mich kamen, und
die Angst dauerte mehrere Jahre und ich hätte mich gerne davon
frei gemacht. Ich ging zu manchem Priester, um Trost zu suchen,
aber ich fand keinen bei ihnen.
Von Barnet ging ich nach London, wo ich eine Wohnung nahm,
und dort war ich in großem Elend und Iammer; denn ich sah
daß die großen "'Frommen"' der Stadt alle in den Banden der
Finfterniß waren. Ich hatte einen Oheim dort, einen Baptisten,



\picinclude{./000-009/p_s004.jpg}

die waren damals gottselig (temier); dennoch konnte ich ihm meine
Stimmung nicht kundtun, noch mich ihm anschließen, denn ich
dUWhsch11Uke celle, jung und alt und wie ez um sie stand. Etliche
gottselige Leute (tenrler people) hätten mich gern dort behalten,
aber ich getraute mich nicht und wandte mich wieder gegen
Leicestershire; der Gedanke, ich könnte meinen Eltern und Ange-
hörigen weh tun, bedrückte mich; denn sie waren, wie ich merkte,
betrübt über meine Abwesenheit.
A15 ich nach Leieesterfhire kam, wollten meine Leute, daß ich
mich oerheirate; aber ich sagte ihnen, ich sei noch ein Knabe und
müsse weise werden. Andre hätten mich gerne bei der Hilfßtruppe
im Militär 1) gesehen, aber ich weigerte mich; und eß betrübte mich,
daß sie mir solche Dinge vorschlugen, dennsich war ein gottseliger
(temier) Jüngling. Darauf ging ich nach Eoventry, wo ich auf
einige Zeit ein Zimmer im Hause eineö "`Frommen"` hatte, bis
die Leute ansingen mich zu kennen; denn ez waren viele gott-
selige Leute in jener Stadt. Nach einiger Zeit ging ich wieder
in meine Heimat und blieb etwa ein Jahr dort, in großer Trüb-
sal; während mancher Nacht irrte ich einsam umher.
Dornach kam der Priester von Drayton, Nathanael Steoenz,
oft zu mir und ich ging oft zu ihm; und ein anderer Priester
kam oft mit ihm und sie verschmiihten nicht, mich anzuhören; ich
stellte ihnen Fragen und di?-kutierte mit ihnen. Dieser Priester
Stevenö stellte mir folgende Frage: warum Christuß am Kreuz
gerufen habe: "'mein Gott, mein Gott, warum hast du mich oer-
lassen?"` und warum er gesagt habe: "'wenn etz möglich, so gehe
dieser Kelch an mir vorüber, aber nicht wie ich will, sondern wie
du willst."` Jch erwiderte ihm, daß zu der Zeit die Sünde der
ganzen Menschheit auf ihm gelegen habe und er ihre Missetat
und Übertrettmg tragen und für sie geopfert und verwundet
werden mußte, sofern er Mensch war; aber er starb nicht, sofern
er Gott war; und weil er so für alle starb und den Tod schmeckte
für jeden Menschen, wurde er zum Opfer für die Sünden der
ganzen Welt. So sprach ich, weil ich zu jener Zeit gewisser-
maßen die Leiden Christi, und maß er durchgemacht, an mir
nachempfand. Der Priester sagte auch, es sei eine sehr treffende
Antwort, eine, wie er sie noch nie gehört habe. Zu jener Zeit
1) E3 war det Anfang der Bürgerkeiege.


\picinclude{./000-009/p_s005.jpg}

pflegte er mich zu loben und anerkennend von mir zu andern zu
sprechen; und das, war- ich ihm während der Woche im Gespräch
mitteilte, predigte er dann am \textit{Ersten Tage} 1); deöwegen mochte
ich ihn nicht leiden. Später wurde dieser Priester ein großer
Verfolger.
Darauf ging ich zu einem andern Priester in Mancetter in
Warwickshire und diökutierte mit ihm über den Grund der Ver-
suchungen und der Verzweiflung, aber er verstand meinen Zustand
nicht; er riet mir, zu rauchen und Psalmen zu singen; nun mochte
ich aber den Tabak nicht und zum Psalmensingen war ich nicht
aufgelegt; ich konnte nicht singen. Er lud mich ein, wieder zu
kommen; dann wolle er mir manches sagen; aber als ich kam,
war er ärgerlich und verdrießlich, weil meine früheren Worte ihm
mißfallen hatten. Gr redete mit seinen Dienstboten über meine
Leiden und Bekümmernisse, und ich bereute, einem solchen meine
Gesinnung aufgedeckt zu haben. Jch sah, daß sie alle leidige
Tröster (Hiob 16, 2) waren, und sie machten meine Unruhe noch
größer. Darauf hörte ich von einem Priester, der in der Nähe
von Tamworth lebte und für einen erfahrenen Mann galt. Ich
ging sieben Meilen weit zu ihm, aber ich fand, daß er nur ein
leereö, hohletz Gefäß war. Auch von einem Or. Cradock in Eoventry
hörte ich und ging zu ihm. Ich befragte ihn über Versuchung
und Verzweiflung und wie die Ansechtungen wohl über den
Menschen kommen, Gr fragte mich, wer Jesu Mutter und Vater
gewesen seien? Jch entgegnete, Maria sei seine Mutter gewesen
und er gelte als der Sohn Joseph?-, aber er sei der Sohn Gotteö.
Wir gingen gerade auf einem schmalen Weg in seinem Garten
und beim Umdrehen trat ich mit dem Fuß auf den Rand einetz
Veeteß, worüber der Mann in Wut geriet, als- ob sein Haut; in
Flammen stünde, und unsere ganze Unterredung war gestört und
ich ging in Vekümmerniz hinweg, bekümmeiter alß ich gekommen
war. Ich sah, daß sie alle leidige Tröster waren und so viel
wie nichtö für mich, denn sie konnten sich nicht in meinen Zustand
versetzen. Daraufhin ging ich zu einem, namens Macham, einem
Priester von hohem Ansehen. Gr verordnete mir Arznei und ich
mußte zu Ader lassen. Aber man konnte mir keinen Tropfen Blut
entziehen, weder am Arm noch am Kopf, trotz aller Mühe, die
Iii Fox hat den Grundsatz, statt Sonntag, Erster Tag zu sagen, da etz für
. ihn keine heiligen Tage gibt.


\picinclude{./000-009/p_s006.jpg}

man sich gab, weil mein Körper wie ausgetrocknet war
durch Kummer, Unruhe und Jammer, die so schwer auf mir
lagen, daß ich hätte wünschen können, gar nicht oder blind ge-
boren zu sein, damit ich nie die Schlechtigkeit und Eitelkeit der
Welt gesehen hätte, oder taub, daß ich nie eitle und böse Worte
gehört hätte, und wie der Name des Herrn gelästert wurde. Als
die Zeit, die man Weihnacht nennt, kam, ging ich, während andere
sich belustigten und sichs wohl sein ließen, von Haus zu Haus zu
armen Witwen und gab ihnen Geld. Wenn ich zu Hochzeiten einge-
laden war, wie zuweilen geschah, ging ich nie hin, sondern machte
erst am folgenden Tage oder bald darauf einen Besuch, und wenn
die Leute arm waren, gab ich ihnen Geld; ich besaß davon ge-
rade so viel, daß ich niemanden zur Last zu fallen brauchte und
noch dem Dürftigen etwas spenden konnte.
Zu Anfang des Jahres 1646 als-—ich, auf dem Wege nach
Cooentry, mich den Toren der Stadt näherte, stieg die Frage in
mir auf, wie man sagen könne: alle Christen seien Gläubige, so-
wohl Papisten als Protestanten; und der Herr Offenbarte mir,
daß, wenn alle Gläubige wären, so wären sie alle aus Gott ge-
boren und oom Tode zum Leben durchgedrungen (1. Joh. 3,:;
nur solche seien wahre Gläubige; und wenn auch andere sagen,
sie seien auch wahre Gläubige, so seien sie es doch nicht.
Gin andermal, als ich am Morgen eines Ersten Tages über
ein Feld ging, offenbarte mir der Herr, daß in Oxford oder Cam-
bridge erzogen sein noch nicht genüge, um tüchtig nnd fähig zum
Dienst Christi zu machen; ich verwunderte mich darüber, denn
das war die allgemeine Meinung der Leute. Aber ich sah es
vollständig ein, als der Herr es mir offenbarte und war über-
zeugt davon nnd pries die Güte Gottes, die mir solches an diesem
Morgen geoffenbart hatte. EH griff das Amt des Priesters Stevens
an, daß: ,,in Oxford oder Cambridge erzogen zu sein noch nicht
genüge, um tüchtig und fähig zum Dienst Christi zu machen"'; es
wurde mir klar, daß da-3, was mir geoffenbart worden war, das
priesterliche Amt angreife. Meine Angehörigen waren sehr betrübt,
daß ich nicht mit ihnen kommen wollte, um den Priester zu hören.
Jch ging eben lieber allein ins Freie mit einer Bibel. Ich fragte sie,
ob nicht der Apostel zu den Gläubigen sage, "'sie bedürfen nicht, daß
sie jemand lehre, die Salbung lehre sie"` (1. Joh. 2). Aber wie-
wohl sie wußten, daß solches in der Schrift steht, und daß es


\picinclude{./000-009/p_s007.jpg}

wahr ist, waren sie doch betrübt, daß ich mich in diesem Punkte
nicht unterwerfen und mit ihnen den Priester anhören konnte.
Ich sah ein, daß es ein ander Ding ist, ein wahrer Gläubiger
zu sein, als das worauf es diesen ankommt ...... Warum sollte
ich also diesen anhängen? Weder diesen noch irgendwelchen
Dissentern konnte ich mich anschließen, sondern war allein, ein
Fremdling, und hielt mich einzig an den Herrn Jesus Christus.
Ein andermal hatte ich die Offenbarung, daß Gott, der die
Welt gemacht hat, nicht in Tempeln mit Händen gemacht wohne.
Dies schien mir zuerst ein seltsames Wort, denn sowohl die
Priester als auch das Volk pflegten ihre Tempel oder Kirchen
\textit{Stätten der Ghrsurcht}, \textit{heiliger Boden} und \textit{Tempel Gottes}
zu nennen. Aber der Herr zeigte mir deutlich, daß er nicht
in diesen Tempeln wohne, die von Menschen verordnet und
ausgerichtet waren, sondern in den Herzen der Menschen. Denn
sowohl Stephanus als der Apostel Paulus gaben Zeugnis,
daß er nicht in Tempeln mit Händen gemacht wohne (Art. 7, 48),
nicht einmal in demjenigen, den er einst zu bauen befohlen hatte,
sintemal er ihm ein Ende gemacht hatte, sondern sein Volk sei
sein Tempel, und da wohne er. Solches wurde mir geoffenbart,
während ich durchs Feld zu den Meinigen ging. Als ich kam,
sagten sie mir, Priester Stevens sei dagewesen und habe gesagt,
er sei besorgt um mich, weil ich neuen Lichtern nathgehe. Jch
lächelte bei mir selber, im Gedanken was der Herr mir über ihn
und seinesgleichen geoffenbart hatte. Aber ich sagte meinen Ver-
wandten nichts davon. Denn obgleich sie den Priester durch-
schauten, gingen sie doch, ihn zu hören, und waren betrübt, daß
ich nicht auch ging. Aber ich kam ihnen mit Schriststellen und
zeigte ihnen, daß es eine Salbung gibt im Menschen, die ihn
lehrt, und daß der Herr sein Volk selber lehren will. Ich hatte
auch große Ossenbarungen über das, was in der Mokalypse steht;
wenn ich davon redete, so sagten die "'Frommen"' und die Priester,
sie sei ein versiegeltes Buch, und wollten mich davon abbringen;
aber ich sagte ihnen, Christus könne die Siegel öffnen und sie
sei das, was uns am nächsten angehe; denn die Briefe seien an
die Heiligen früherer Zeiten gerichtet, aber die Apokalypse handle
von den künftigen Dingen.
Ich traf mit Leuten zusammen, welche die Ansicht hatten,
die Frauen hätten keine Seelen, "`nicht mehr als eine Gans"',


\picinclude{./000-009/p_s008.jpg}

wahr ist, waren sie doch betrübt, daß ich mich in diesem Punkte
nicht unterwerfen und mit ihnen den Priester anhören konnte.
Ich sah ein, daß es ein ander Ding ist, ein wahrer Gläubiger
zu sein, als das worauf es diesen ankommt ...... Warum sollte
ich also diesen anhängen? Weder diesen noch irgendwelchen
Dissentern konnte ich mich anschließen, sondern war allein, ein
Fremdling, und hielt mich einzig an den Herrn Jesus Christus.
Ein andermal hatte ich die Offenbarung, daß Gott, der die
Welt gemacht hat, nicht in Tempeln mit Händen gemacht wohne.
Dies schien mir zuerst ein seltsames Wort, denn sowohl die
Priester als auch das Volk pflegten ihre Tempel oder Kirchen
\textit{Stätten der Ghrsurcht}, \textit{heiliger Boden} und \textit{Tempel Gottes}
zu nennen. Aber der Herr zeigte mir deutlich, daß er nicht
in diesen Tempeln wohne, die von Menschen verordnet und
ausgerichtet waren, sondern in den Herzen der Menschen. Denn
sowohl Stephanus als der Apostel Paulus gaben Zeugnis,
daß er nicht in Tempeln mit Händen gemacht wohne (Art. 7, 48),
nicht einmal in demjenigen, den er einst zu bauen befohlen hatte,
sintemal er ihm ein Ende gemacht hatte, sondern sein Volk sei
sein Tempel, und da wohne er. Solches wurde mir geoffenbart,
während ich durchs Feld zu den Meinigen ging. Als ich kam,
sagten sie mir, Priester Stevens sei dagewesen und habe gesagt,
er sei besorgt um mich, weil ich neuen Lichtern nathgehe. Jch
lächelte bei mir selber, im Gedanken was der Herr mir über ihn
und seinesgleichen geoffenbart hatte. Aber ich sagte meinen Ver-
wandten nichts davon. Denn obgleich sie den Priester durch-
schauten, gingen sie doch, ihn zu hören, und waren betrübt, daß
ich nicht auch ging. Aber ich kam ihnen mit Schriststellen und
zeigte ihnen, daß es eine Salbung gibt im Menschen, die ihn
lehrt, und daß der Herr sein Volk selber lehren will. Ich hatte
auch große Ossenbarungen über das, was in der Mokalypse steht;
wenn ich davon redete, so sagten die "`Frommen"` und die Priester,
sie sei ein versiegeltes Buch, und wollten mich davon abbringen;
aber ich sagte ihnen, Christus könne die Siegel öffnen und sie
sei das, was uns am nächsten angehe; denn die Briefe seien an
die Heiligen früherer Zeiten gerichtet, aber die Apokalypse handle
von den künftigen Dingen.
Jch traf mit Leuten zusammen, welche die Ansicht hatten,
die Frauen hätten keine Seelen, "'nicht mehr als eine Gans"`,


\picinclude{./000-009/p_s009.jpg} 

ein ßiziann der Schmerzen, in den Zeiten, da der Herr sein Werk
in mir anfing.
Während dieser ganzen Zeit hatte ich mich nie mit irgend
jemand zu irgend einer religiösen Richtung Verbunden, sondern
gab mich ganz dem Herrn hin; von aller schlechten Gesellschaft
hatte ich mich losgemacht, hatte Abschied genommen von Vater
und Mutter und allen andern Angehörigen und zog als ein
Fremdling umher, wohin der Herr mein Herz lenkte; ich mietete
ein Zimmer jeweilen in der Stadt, in die ich kam und weilte oft
etwa einen Monat an einem Orte; denn ich wagte nie lange an
einem Orte zu bleiben, da ich fürchtete, als gottseliger Jüngling
sowohl bei den "'Frommen"' als auch bei den Ungläubigen Schaden
zu nehmen, wenn ich viel mit den einen oder den anderen umging;
darum oerhielt ich mich meist wie ein Fremdling; ich suchte hinun-
lische Weisheit, und Erkenntnis kam mir einzig vom Herrn. Jrh
wurde losgelöst von den äußeren Dingen, um mich allein auf
den Herrn zu verlassen. Meine Prüfungen und Trübsale waren
sehr schwer; aber wenn es mir zwischen hinein etwas leichter
wurde, so geriet ich ost in solch eine himmlische Freude, daß ich
wiihnte, in Abrahams Schoß gewesen zu sein. Wie ich das Elend,
in dem ich war, nicht schildern kann, ebensowenig kann ich die
Barmherzigkeit beschreiben, die Gott in diesem Elend an mir getan
hat ....
Nachdem ich die Offenbarung vom Herm empfangen hatte,
"'daß in Oxford oder Cambridge erzogen zu sein noch nicht zum
Dienst des Herrn besähige"`, achtete ich die Priester weniger und
sah mehr auf die Dissenter; ich sah, daß unter diesen einige
Gottseligkeit sei, und viele von ihnen kamen auch später, zu einer
festen Uberzeugung, weil sie Offenbarungen hatten. Aber wie
ich die Priester aufgegeben hatte, so ließ ich auch die Separa-
ristenprediger und solche, welche als die Erfahrensten angesehen
wurden; denn ich sah, daß keiner unter ihnen allen war, derzu meinem
Zustand sprechen konnte. Als alle meine Hoffnungen auf sie und alle
Menschen dahin waren, so daß ich nichts hatte, das mir von außen
hals, und ich nicht wußte, was tun — da! o da hörte ich eine
Stimme: "'es ist Einer, der zu deinem Zustand sprechen kann,
nämlich Jesus Christus Und als ich das hörte, hüpfte mein
Herz vor Freude. Dann zeigte mir der Herr, warum niemand
auf der Welt mir in meinem damaligen Zustand helfen konnte,


\picinclude{./010-019/p_s010.jpg} 
\picinclude{./010-019/p_s011.jpg} 
\picinclude{./010-019/p_s012.jpg} 
\picinclude{./010-019/p_s013.jpg} 
\picinclude{./010-019/p_s014.jpg} 
\picinclude{./010-019/p_s015.jpg} 
\picinclude{./010-019/p_s016.jpg} 
\picinclude{./010-019/p_s017.jpg} 
\picinclude{./010-019/p_s018.jpg} 
\picinclude{./010-019/p_s019.jpg} 


% \picinclude{./020-029/p_s020.jpg} 
20 Kapitel 11.
Lohn zu verkürzen, sondern ihnen zu geben, was recht und billig
sei, und die Dienstboten ermahnte ich, ihre Pflicht zu tun und
ehrlich zu dienen; sie nahmen meine Mahnungen freundlich auf,
denn ich wurde vom Herm dazu getrieben.
Ferner trieb etz mich, an verschiedene Gerichtöhöfe und in ver-
schiedene Turmhäuser in Manöfield und an andern Orten zu gehen,
um alle zu ermahnen vom Unterdrücken und vom Schwören abzu-
lassen und sich von der Ungerechtigkeit zum Herrn zu bekehren und
recht zu tun. Jnßbesondere trieb es mich, nach einer Gerichtßver-
handlung in Manöfield zu einem zu gehen, der einer der schlech-
testen Menschen der dortigen Gegend war, und mit ihm zu reden;
er war ein Säufer und berüchtigte: Mädchenhändler; ich warnte ihn
beim allmächtigen Gott wegen s eines schlechten Wandels; als ich auß-
geredet hatte und ihn Verlassen wollte, lies er mir nach und sagte
mir, während ich mit ihm gesprochen habe, sei er so ergriffen worden,
daß ihn seine Kräfte ganz verließen. So wurde dieser Mann be-
kehrt, und er ließ ab von seiner Schlechtigkeit und blieb rechtschaffen
und nüchtern zum Erstaunen aller, die ihn vorher gekannt hatten.
Und das Werk des Herrn nahm zu und viele kamen von der Finster- ,
nie zum Licht, im Laufe dieser drei Jahre 1646, 1647 und 1648.
GS wurden in dieser Zeit mehrere Versammlungen für Freunde ein-
gerichtet, damit Gott sich kund tue durch sein Licht, seinen Geist
und seine Kraft; denn dee Herrn Kraft brach immer herrlicher hervor.
Nun war ich ini Geiste bei Idem stammenden Schwert vorbei
inö Paradieß Gotteö eingedrungen. Alle Dinge waren wie um-
gewandelt ftir mich und die ganze Schöpfung hatte einen andern
Geruch für mich, über alles waß Worte außdrücken können. Ich
wußte nur noch von Reinheit, Unschuld und Rechtschaffenheit, denn
ich war erneuert zum Ebenbild Gotteß (Col. 3, 10) durch Christus,
in den Zustand, in dem Adam vor dem Fall gewesen war. Die
ganze Schöpfung wurde mir offenbar und es- wurde mir gezeigt,
wie alle Dinge mit dem Namen genannt wurden, der ihrem
Wesen und ihren Kräften entsprach. Jch war unschliisstg, ob ich
nicht sollte Heilkunde treiben zum Nutzen der Menschheit, als ich
sah, wie die Natur und die Kräfte aller Dinge mir so geoffenbart
wurden vom Herrn. Aber alsbald wurde ich ergriffen im Geist
und erkannte einen andern, sicherem Zustand als die Sitndlosig-
keit Adams, den Zustand Jesu Christi, der nicht fallen konnte.
Und der Herr zeigte mir, daß die, so ihm treu bleiben im Licht


% \picinclude{./020-029/p_s021.jpg} 
Erste Versammlungen und Proteste. 21
und in der Kraft Christi, erhoben werden in den Zustand, darin
Adam vor dem Fall gewesen war, in welchem die bewundernß-
werten Werke der Schöpfung und ihre Kräfte erkannt werden
können durch die Offenbarung deß göttlichen Worteß der Weiß-
heit und der Kraft, durch welche sie gemacht waren. Der Herr
führte mich in große Dinge ein, und wunderbare Tiefen wurden
mir geoffenbart, die alleß iibertrafen, waß Worte beschreiben
können. Aber wer sich dem Geist Gotteß unierwirst und hinein-
wächst in daß Gbenbild und die Kraft deß Allmächtigen, der wird
daß Wort der Weißheit empfangen, daß alle Dinge offenbar macht,
und wird dazu gelangen, die verborgene Einheit in dem ewigen
Wesen zu erkennen.
So reiste ich umher im Dienste deß Herrn, wie mich der
Herr führte. Alß ich nach Nottingham kam, war Gotteß mächtige
Kraft mit den Freunden. Von da ging ich nach Elawson in
Leieestershire im Tale Veavor, und auch dort wirkte die Kraft
Gotteß in Verschiedenen Städten und Dörfern, in denen Freunde
beisammen waren. Während ich dort war, offenbarte mir der
Herr drei Dinge, die sich auf die drei großen Berufßarten in der
Welt — Heilkunde, sogenannte Gotteßgelehrtheit und Recht?--H
wissenschast bezogen. Er zeigte mir, daß die Ärzte nicht die
Wei?-heit Gotteß haben, durch die alle Kreatur geschaffen ist, und
daß sie darum ihre Kräfte nicht kennen, weil sie nicht im Worte der
Weiß-heit sind, durch daß alleß gemacht ist. Gr zeigte mir, daß
die Priester nicht den wahren Glauben haben, dessen Ursprung
Christus ist; den Glauben, der reinigt und den Sieg gibt und
durch des man Gott gefällt, welcheß Geheimniß deß Glaubenß
in reinem Gewissen ist (1. Tim. 3, 9). Gr zeigte mir ferner, daß
die Rechtßgelehrten nicht die wahre Villigkeit und Gerechtigkeit
besitzen und nicht daß Gesetz Gotteß haben, nach welchem schon
die erste Ubertretung und alle weiteren Sünden gerichtet worden
sind und welcheß dem Geiste Gotteß entspricht, den die Menschen
in sich betrüben und gegen den sie sündigen (Eph. 4, 30).
Und daß diese drei, die Ärzte, die Priester und die Rechtßgelehrten,
die Welt ohne Weißheit regieren, ohne Glauben, ohne Billigkeit,
ohne Recht und ohne daß Gesetz Gotteß; die einen, indem sie
vorgeben, den Leib zu heilen, die andern die Seele und die dritten
daß Eigentum der Leute zu schützen. Aber ich sah, daß sie alle
die Weißheit, den Glauben, die Gerechtigkeit und daß GesetzZGotteS


% \picinclude{./020-029/p_s022.jpg} 
22 Kapitel 11.
nicht hatten. Und als der Herr mir diese Dinge osfenbarte, fühlte
ich, daß seine Kraft sich über alle ergoß und daß sie durch die-
selbe alle umgewandelt werden könnten, wenn sie sie aufnehmen und
sich ihr beugen würden. Die Priester würden umgewandelt werden
und zum wahren Glauben kommen, welcher eine Gabe Gottes
ist. Die Rechtsgelehrten würden umgewandelt werden und zum
Gesetz Gottes (Jar. 2, 2) kommen, welches dem göttlichen im
Herzen entspricht und es möglich macht, seinen Nächsten wie sich
selbst zu lieben. Dieses Gesetz läßt den Menschen erkennen, daß
wenn er seinem Nächsten schadet, so schadet er sich selber, und
es lehret ihn, andern zu tun, wie er möchte, daß die andern ihm
tun. Die Ärzte können umgewandelt werden und zur Weisheit
Gottes kommen, durch die alle Dinge geschaffen sind, und so
eine rechte Erkenntnis über diese Dinge erlangen und ihre Kräfte
erkennen an den Namen, die die Weisheit, die sie gemacht, ihnen
gab ....
Der Herr offenbaite mir durch seine unsichtbare Kraft, daß
ein jeder erleuchtet werde durch das heilige Licht Christi (Joh. 1, 9).
Und ich erkannte, daß es in allen leuchtet, und daß alle, die
daran glaubten, aus der Verdammnis zum Licht des Lebens
kamen und Kinder des Lichts wurden (Joh. 12, 36). Aber die,
welche es haßten und nicht daran glaubten, die verdammte es, wie-
wohl sie schienen Christum zu bekennen.,« Solches sah ich in der
reinen Offenbarung des Lichts, ohne jegliche menschliche Hilfe;
auch wußte ich damals nicht, wo es in der Schrift zu sinden
war; doch später, als ich in der Schrift forschte, fand ich es.
Damals aber hatte ich jenes Licht und jenen Geist geschaut, welche
gewesen, ehe die Schrift gegeben worden war, und welche die
heiligen Männer Gottes getrieben hatten, die Schrift zu schreiben;
und ich erkannte, daß alle, welche Gott, Christus oder die Schrift
recht kennen wollen, zu diesem Geist gelangen müssen. Aber ich
merkte eine Trägheit und faule Schläfrigkeit in den Leuten, die
mich erstaunte; oftmals, wenn ich einschlafen wollte, schweifte
mein Geist über alles hinaus zu dem, der von Ewigkeit zu Ewig-
keit ist. Jch sah, daß der Tod über diesen schltisrigen und faulen
Zustand kommen mußte, und ich sagte den Leuten, sie müßten
dazu kommen, dieses schläfrige, träge Wesen zu töten und zu
kreuzigen durch die Kraft Gottes, damit ihre Herzen und Sinne
droben seien.


% \picinclude{./020-029/p_s023.jpg} 
Erste Versammlungen und Proteste. 23
Einmal alß ich durchs Feld wanderte, sagte der Herr zu mir:
,,Dein Name ist geschrieben im Lebenßbuche deß Lammeö, welcheö
gewesen vor der Erschaffung der Welt«. Alk- der Herr dietz sagte,
da glaubte ich e3 und erkannte es, kraft der neuen Geburt. Einige
Zeit darauf befahl mir der Herr, in die Welt hinaus zu gehen,
die wie eine dornige Wildniß war; und alß ich in der Kraft
Gottes mit dem Wort des Lebenö in die Welt hinauß kam, lehnte
sich die Welt dagegen auf und tobte wie die großen tobenden
Wogen der See; Priester wie ,,Fromme«, die Obrigkeit wie das
Volk, alle waren wie die See, als ich kam, den Tag deß Herrn
unter ihnen zu verkünden und ihnen Buße zu predigen ......
Als mich Gott und sein Sohn Jesuß Christuß außsandten
in die Welt, um sein ewigeö Evangelium und Reich zu predigen,
freute ich mich, daß ich den Befehl hatte, die Leute jenem innern
Licht, Geist und Gnade zuzuführen, durch die alle ihr Heil und
den Weg zu Gott erkennen können; ja, jenem heiligen Geist,
der in alle Wahrheit führt und von welchem ich bestimmt wußte,
daß er nie jemanden trtigt.
Durch diese göttliche Kraft und den Geist Gottes und daß
Licht Jesu sollte ich nun die Menschen von ihren eigenen Wegen
ab zu Christus?-, dem neuen, lebendigen Weg bringen; ab von
ihren Kirchen von Menschen gemacht, zur Kirche in Gott, zur
Gemeinde derHeiligen, die imHi1nmel angeschrieben ist (Gbr. 12, 23),
deren Haupt Ehristuß ist; ab von den Lehrern dieser Welt, die
von Menschen eingesetzt sind, damit sie von Ehristus:3 lernen, der
der Weg, die Wahrheit und daß Leben ist (Joh. 14, 6), von welchem
der Vater sagt: ,,dieS ist mein lisber Sohn, den höret« (Luc. 9, 35);
ab von allem weltlichen Gottezdienst, damit sie den Geist der Wahr-
heit in ihrem Jnnern erkennen und sich von demselben führen
lassen; daß sie in demselben den Vater der Geister anbeten, dem
solcheß anbeten angenehm ist; die, welche nicht in diesem Geiste
anbeten, wissen nicht, maß sie anbeten. Jch sollte die Menschen
abbringen von all den Gottesdiensten dieser Welt, welche eitel
sind, damit sie zu dem wahren Gotteßdienst kommen, welcher die
Witwen und Waisen in ihrer Trübsal tröstet (Jar. 1, 27) und be-
wahret von der Befleckung der Welt; dann gäbe es:) nicht so viele
Bettler, deren Anblick so ost mein Herz betrübt, weil er von so
viel Hartherzigkeit zeugt unter denen, die vorgeben, C-hristus3 zu
bekennen. Ich sollte sie von allen Gemeinschaften, Singereien


% \picinclude{./020-029/p_s024.jpg} 
24 Kapitel ll.
und Betereien dieser Welt abbringen, welche Formen ohne Kraft
sind, auf daß ihre Gemeinschaft im heiligen Geist sei, im ewigen
Geist Gotteö, und sie darin anbeten und singen, durch die Gnade,
die von Christus kommt; und so dem Herm in ihren Herzen
singen’und spielen, der seinen geliebten Sohn gesandt hat, um
ihr Retter zu sein; der seine himmlische Sonne über und in allen
scheinen läßt und seinen himmlischen Regen über Gerechte und
Ungerechte außgießt (Matth. 5), wie der äußere Regen über alle
fällt und die äußere Sonne fiir alle scheint; dietz ist Gotteö un-
außsprechliche Liebe zur Welt. Jch sollte die Leute von den
jüdischen Zeremonien abbringen und von den heidnischen Fabeln
und den menschlichen Einrichtungen und weltlichen Lehren, durch
welche die Leute hin und her von einer Sekte zur andern ge-
trieben werden, und von allen ihren bettelhaften Lehranstalten
und ihren Schulen und Hochschulen, in denen sie Prediger Christi
machen wollen, die aber wahrlich Prediger ihrer eigenen Machen-
schaft sind und nicht Christi; von allen ihren Bildern und Kreuzen
und Besprengen von Kindern; allen ihren sogenannten heiligen
Tagen und nichtigen Traditionen, die sie seit den Tagen der
Apostel eingerichtet haben und gegen welche die Kraft Gottes
sich richtet; vermöge dieser Kraft wurde ich getrieben, gegen
alleß daß aufzutreten und gegen alle, die nicht umsonst pre-
digten und doch solche waren, die umsonst vom Herrn empfangen
hatten. s
Ferner verbot mir der Herr, als er mich in die Welt hinauö
sandte, meinen Hut abzunehmen vor irgendjemand, hoch oder
niedrig; und ich hatte den Befehl, zu allen, Männern und Frauen,
,,Du« zu sagen, ohne irgend einen Unterschied zu machen zwischen
reich oder arm, groß oder klein; und ich sollte unterwegs- auf
meinen Reisen den Leuten nicht guten Morgen oder guten Abend
sagen, noch mich vor irgendjemand neigen oder daß Knie beugen.
Solcheö machte die Sekten und Gemeinschaften zornig. Aber die
Kraft des Herrn half mir durch alleß hindurch, zu seiner Ehre,
und viele kehrten sich in kurzer Zeit zu Gott, denn der große
Tag des- Herrn ging auf auß der Höhe und brach eilendö an,
und in seinem Lichte gingen vielen die Augen über ihren Zu-
stand auf.
Aber o, die Wut, in welcher damals- Priester, Obrigkeit,
»Fromme« und andere waren! Aber hauptsächlich die Priester


% \picinclude{./020-029/p_s025.jpg} 
Erste Versammlungen und Proteste. 25
und die ,,Frommen«; denn obgleich das- ,,Du« gegen eine ein-
zelne Person ihrer eigenen Grammatik und Formenlehre, sowie
auch der Bibel entsprach, so konnten sie sich doch nicht drein
finden, es zu hören; und maß die Hut-Ehre anbetraf, daß ich
den Hut nicht vor ihnen abnehmen konnte, das machte sie ganz
wütend ....
In jener Zeit fühlte ich mich, zu meiner schweren Prüfung,
auch berufen, in die Gerichtßhöfe zu gehen, um nach Gerechtigkeit
zu schreien und die Richter und Behörden in Wort und Schrift
zur Gerechtigkeit zu mahnen; ich mußte solche, die öffentliche Gast-
häuser hielten, ermahnen, den Leuten nicht mehr zu trinken zu
geben, als ihnen gut sei; ich mußte auftreten gegen ihre Feste
und Gelage, Spiele, Späße und Belustigungen aller Art, durch
die die Leute zur Eitelkeit und Liederlichkeit verleitet und von
der Gotteßsurcht abgebracht wurden; am häufigsten schändeten
sie Gott (Röm. 2, 23) in dieser Weise an den Tagen, die sie als-
heilige bezeichneten. Auch an Jahrmärkten und Märkten mußte
ich mich gegen ihr trügerischeö Handeln wenden, ihren Schwindel
und Betrug; ich mußte sie mahnen, die Wahrheit zu sagen, ihr
ja—ja und ihr neinsnein sein zu lassen, und andern zu tun, wie
sie wollten, daß man ihnen tue, alleß indem ich sie an den großen
Tag dez Herrn erinnerte, der über sie alle kommen werde. Auch
gegen allerlei Musizieren und gegen die Schwindler, die in den
Vuden ihr Wesen trieben, mußte ich auftreten, denn sie gefähr-
deten die Unschuld und reizten den Sinn der Leute zur Eitelkeit.
Jch mußte auch manchen schweren Gang zu Lehrern und Lehrerinnen
tun, um sie zu erinahnen, die Kinder in der Furcht deö Herm zu
erziehen, damit sie nicht in Eitelkeit, Leichisinn und Schlechtigkeit
aufwachsen. Ebenso mußte ich Lehrer und Lehrerinnen, sowie die
Väter und Mütter ermahnen, darauf zu achten, daß man die
Kinder und die Dienstboten daheim im Hanse zur Gotteßstircht an-
halte, damit sie Vorbilder der Tugend und Mäßigkeit werden.
Die irdische Gesinnung der Priester tat mir weh, und wenn
ich die Glocken läuten hörte, welche die Leute inö Turnthauß
rufen sollten, ging es mir durch Mark gund Bein, denn eS war
gerade wie eine Marktglocke, welche die Leute zusammenruft, daß I
der Priester seine Ware Izum Verkauf außbieteu kann. O, die
großen Geldsummen, die zusammenkamen durch ihr Handeln mit
Bibeln und durch ihr Predigen, vom höchsten Bischof biz zum


% \picinclude{./020-029/p_s026.jpg} 
26 Kapitel lll.
einfachsten Priester! Wa;-’ für ein Handel in der Welt kommt
diesem gleich! Und doch wurde die Schrift gegeben umsonst! Und
Christus hatte seinen Jüngern befohlen, umsonst zu predigen;
und die Propheten und Apostel verkündeten allen geizigen Miet-
lingen und allen, die für Geld iveiösagten, daß Gericht. Jch
aber wurde au?-gesandt, in diesem freien Geist daß Wort vom
Leben und der Versöhnung umsonst zu predigen, auf daß alle zu
Christus kommen, welcher umsonst gibt und in daß E-benbild
Gotteß erneuert, nach dem Mann und Weib geschaffen waren
vor dem Fall, auf daß sie himmlische Güter in Jesuß Ehristuß
haben möchten.
Kapitel lll.
Der Tumult in Nottingham. Wachsender Widerstand, bis zum
Gefängnis in Derby.
A16 ich einmal am Morgen einetz Ersten Tageß in der Nähe
von Nottingham von einem Hügel auß die Stadt überblickte, da
gewahrte ich daß riesige Turmhauß, und der Herr sagte zu mir:
,,Du mußt hingehen und gegen jene großen Götzen schreien und
gegen die, welche drinnen anbeten«. Jch sagte den ,,Freunden«,
die mit mir waren, nichtß davon, sondern ging mit ihnen hin in
die Versammlung, wo die mächtige Kraft der- Herrn mit uns
war; hier ließ ich sie und ging zum Turmhauß. Die Menge,
die ich hier sah, kam mir vor wie ein Brachfeld und der Priester
wie ein großer Erdklumpen, der oben aus seiner Kanzel stand.
Gr hatte zum Text die Worte des Petruß: »Wir haben ein festes-
prophetischeö Wort und ihr tut wohl, daß ihr daraus achtet, alß
auf ein Licht, das da scheinet an einem dunkeln Ort, biß der Tag
anbreche und der Morgenstern ausgehe in eueren Herzen«
(2. Petr. 1, 191. Er sagte den Leuten, nach dem, waß hier ge-
schrieben stehe, sollten sie alle Lehren, Bekenntnisse und Meinungen
prüfen. Da kam die Kraft dez Herrn so mächtig über mich und
war so stark in mir, daß ich nicht an mich halten konnte, sondern
rusen mußte: ,,O nein, nicht nach dem, was geschrieben stehet!«
und ich sagte ihnen, nach maß: nämlich nach dem heiligen Geist,
durch den die heiligen Männer Gotteö die Schrift geschrieben
haben. Durch diesen, sagte ich, müssen alle Lehren, Bekenntnisse
und Meinungen geprüft werden. Dieser Geist leitet in alle


% \picinclude{./020-029/p_s027.jpg} 
Der Tuniult in Nottingham. Wachsender Widerstand usw. 27
Wahrheit und zur Erkenntniß aller Wahrheit. Die Juden haben
die Schrift gehabt und widerstanden dem heiligen Geist doch und ver-
warfen Christuö, den schönen Morgenstern ; sie verfolgten Ehriftuö
und seine Apostel und wollten ihre Lehren nach der Schrift prüfen;
aber sie irrten in ihrem Urteil und prüften sie nicht richtig, weil
sie ohne den heiligen Geist pritften. Da ich nun so zu ihnen
redete, kamen die Wachen und führten mich weg und brachten
mich in einen wüsten, stinkenden Kerker; der Geruch stieg mir so
in die Nase und den Halß, daß etz- eine Qual war, aber die
Kraft deß Herrn schallte an dem Tage so in ihren Ohren, daß
sie ganz von dem Schall betäubt waren, und ihre Ohren wurden
noch eine zeitlang nicht frei davon, so waren sie im Turmhause
von der Kraft dez Herrn ergriffen worden. Am Abend brachten
sie mich vor die Behörden der Stadt; als ich vor sie trat, war
der Bürgermeister in verdrießlicher, mürrischer Laune, aber die
Kraft deß Herrn beschwichtigte ihn. Sie verhörten mich ausführ-
lich und ich berichtete ihnen, wie der Herr mich getrieben hatte
zu kommen. Nach einigem Hin- und Herreden schickten sie mich
in?-’ Gefängniß zurück. Aber bald darauf ließ mich der Oberscheriff,
John Neckleß, zu sich in sein Haus holen. Als ich eintrat, be-
gegnete mir sein Weib im Flur und sagte: ,,Unserm Hause ist
Heil widerfahren.« Sie reichte mir die Hand und war mächtig
ergriffen von der Kraft Gotte;-’, und ihr Mann und ihre Kinder
und Dienstboten wurden ganz umgewandelt, denn die Kraft des
Herrn war mächtig in ihnen. Jch wohnte bei ihnen und wir
hatten große Versammlungen in ihrem Hause; eß kamen auch
etliche angesehene Standeß-personen, und deß Herrn Kraft tat sich
mächtig kund unter ihnen; John Reckleß ließ dann einen andern
Scherifs holen und eine Frau, mit der sie in Geschäften zu tun
gehabt hatten, und erklärte in Anwesenheit des andern Scheriff,
daß sie beide diese Frau bei einem Handel geschädigt hätten und
sie entschädigen müßten. Er sagte es sehr freundlich, aber der
andere Scherifs leugnete, und die Frau sagte, sie wisse nichtö da-
von. Aber der gerechte Scheriff sagte, ez sei so, und der andere
wisse das ganz gut; nachdem er die Sache aufgedeckt nnd daß
Unrecht, daß sie getan, eingestanden hatte, entschädigte er die
Frau und ermahnte den andern ein gleiches zu tun; die Kraft
Gotteö war mit diesem guten Scherifs und wirkte eine große
Wandlung in ihm und er hatte große Offenbarungen. Alö er


% \picinclude{./020-029/p_s028.jpg} 
28 Kapitel 111.
am darauffolgenden Martttage in den Pantoffeln in seinem
Zimmer auf- und abging, sagte er: »Jch muß auf den Markt
gehen und den Leuten Buße predigen,« und er ging auf den Markt
und in mehrere Straßen und predigte den Leuten Buße; und auch
noch andere aus der Stadt trieb ez, zu den Behörden zu gehen
und die Leute zur Buße zu erinahnen. Die Räte wurden sehr
böse über mich und ließen mich auß dem Hause deö Scheriff
holen und verurteilten mich zum Gefängniß. Alk; die Gerichtß-
fitzung stattfand, fühlte einer sich getrieben, sich statt meiner an-
zubieten, ,,Leib um Leib, Leben um Leben«. Alö ich vor den
Richter gebracht werden sollte, ging es ziemlich lang, bis mich
der Diener, der mich hinbringen sollte, abholte, und alß ich kam,
hatte sich der Richter schon erhoben, worauß ich sah, daß er er-
zürnt war; er sagte, er wolle dem Jüngling schon einen Verweis?.
geben, wenn er vor ihn gebracht werde; ich war damals unter
dem Namen ,,J«itngling« eingesperrt. Jch wurde denn wieder
inö Gefängnis gebracht. Die Kraft detz Herrn war mächtig
unter den ,,Freunden«, aber das?2 Volk fing an, tätlich zu
werden, so daß der Schloßkommandant Soldaten hinau?-schickte,
um die Leute au?-einander zu treiben, worauf es ruhig wurde;
alle, Priester und Volk, erstaunten ob der herrlichen Kraft, welche
heroorbrach, und etliche der Priester wurden empfänglich gemacht
und einige von ihnen bekannten sich zur Krast Gottes-.
Nachdem ich auö dem Gefängniß von Nottingham, wo ich
einige Zeit gefangen gewesen war, entlassen worden, zog ich
umher, wie vorher im Dienst des- Herm. Als ich nach Man?-field
Woodhouse kam, war dort eine verrückte Frau; daß Haar hing
ihr wirr über die Ohren und der Arzt war gerade bei ihr. Er
war daran, ihr zu Ader zu lassen, nachdem man sie zuvor ge-
bunden hatte; viele Leute waren um sie und hielten sie mit Ge-
walt fest, aber man konnte ihr kein Blut entziehen. Ich befahl,
daß man sie frei mache und ruhig lasse, denn sie konnten dem
Geiste, der sie plagte, nicht beikommen; sie machten sie srei und
ez trieb mich, zu ihr zu reden und sie im Namen dez Herrn still
und ruhig sein zu heißen, und sie war etz; die Krast dez Herrn
beruhigte ihr Gemüt und sie genaö, und sie nahm die Wahrheit
auf und blieb darin biß zu ihrem Tod. Des Herrn Name wurde
oerherrlichet, ihm gebührt die Ehre aller seiner Werke ....
Während ich in Manßfield Woodhouse war, trieb es mich,


% \picinclude{./020-029/p_s029.jpg} 
Der Tumnli in Nottingham. Wachsender Widerstand usn-. 29
ins Turmhaus zu gehen, um den Leuten die Wahrheit zu ver-
künden, aber das Volk fiel in großem Zorn über mich her, sie
schlugen mich zu Boden und erstickten mich fast; ich war arg zer-
schlagen und zerquetscht von ihren Händen, Bibeln und Stöcken.
Dann schleppten sie mich hinaus, wie wohl ich kaum fähig war
zu stehen, und taten mich in den Stock, wo ich einige Stunden
saß. Sie brachten Hundepeitschen und Pserdepeitschen und drohten
mir damit. Dann mußte ich vor die Behörden im Hause eines
Adligen, wo viele angesehene Leute zugegen waren. Als diese
sahen, wie ich mißhandelt worden war, gaben sie mir nach
Vielen Drohungen die Freiheit. Aber der Pöbel trieb mich
zur Stadt hinaus zum Dank dafür, daß ich ihnen das Wort des
Lebens verkündet hatte. Jch war kaum imstande zu stehen und
zu gehen, so übel hatten ste mich zugerichtet. Mit großer An-
strengung ging ich etwa eine Meile weit vor die Stadt, wo ich
Leute traf, die mir etwas zur Grquickung gaben, denn ich war
innerlich ganz auseinander, aber die Kraft des Herm heilte mich
bald wieder. Gs waren aber an dem Tage etliche von der Wahr-
heit des Herrn überzeugt worden, worüber ich mich freute. . . .
An einem Ersten Tage kamen wir nach Bagworth und gingen
ins Turmhaus, wohin einige der Freunde gebracht worden waren;
das Volk schloß sie darin ein und sich selbst mitsamt ihrem
Priester. Als der Priester fertig geredet hatte, machten sie die
Türe auf und wir gingen auch hinein und hatten einen Gottes-
dienst mit ihnen, und hernach hatten wir eine Versammlung in
der Stadt, mit manchen angesehenen Leuten. Als ich weiter zog,
hörte ich von solchen, die in Coventry um ihres Glaubens willen
gefangen waren. Aber als ich unterwegs zu ihrem Gefängnis war,
geschah das Wort des Herrn zu mir: ,,Meine Liebe war immer
mit dir und du bist in meiner Liebe«. Und ich fühlte mich ge-
hoben in der Liebe Gottes und sehr gestärkt an meinem innern
Menschen. Als ich in den Kerker zu den Gefangenen kam, über-
kam mich eine große Finsternis; ich hielt stille, denn mein Geist
ruhte in der Liebe Gottes. Schließlich singen die Gefangenen
an zu prahlen, und lärmten und lästerten, worüber meine
Seele sehr betrübt wurde. Sie sagten, daß sie Gott seien, aber
wir konnten solches nicht ertragen. Als sie ruhig geworden
waren, stand ich auf und fragte sie, ob sie solches aus innerem
Trieb oder auf Grund der Schrift täten? Sie sagten: ,,auf



% \picinclude{./030-039/p_s030.jpg} 
Grund der Schrift.'' Da eine Bibel zur Hand war, hieß ich sie,
mir die betreffende Stelle zu zeigen, und sie zeigten mir die Stelle,
wo daß Tuch vor Petrus herabgelassen wurde und die Stimme
sagte: ,,WaS Gott gereiniget hat, daö mache du nicht gemein''
(Act. 10, 15). A15 ich ihnen zeigte, daß diese Stelle nichtß für
sie beweise, brachten sie eine.andere vor, die davon handelte, wie
Gott alle mit sich selbst versöhnt im Himmel und auf Erden
(Col. 1, 20). Jch sagte ihnen, daß ich diese Stelle ebenfallö an-
erkenne, daß sie aber ebensowenig sür sie passe. Alß ich nun
vernahm, wie sie sagten, sie seien Gott, fragte ich sie, ob sie
wissen, ob ez morgen regnen werde? Sie antworteten, daß sie
daß nicht sagen könnten. Jch erwiderte ihnen: Gott könne das
sagen. Darauf fragte ich sie, ob sie immer so bleiben würden,
wie sie jetzt seien, oder ob sie sich ändern würden? Sie ant-
worteten: sie wüßten ez nicht.'' Jch erwiderte: ,,Gott kann es-
sagen und Gott verändert sich nicht. Jhr sagt, ihr seid Gott und
wißt nicht, ob ihr euch verändert oder nicht?« Sie wurden ver-
wirrt und für den Augenblick fast überwunden. Nachdem ich sie
wegen ihrer Gotteßlästerungen zurecht gewiesen hatte, ging ich
fort, demr ich merkte, daß sie Ranter 1) waren. Jch war nie
mit solchen zuvor zusammengetrosfen und ich priez die Güte des
Herrn, daß sie mir erschienen war, ehe ich zu ihnen gekommen
war. Nicht lange nachher schrieb einer dieser Ranterö, namenß
Joseph Salmon, ein Buch, in dem er widerrief, worauf sie die
Freiheit erhielten ....
Bei meinem Herumziehen auf den Jahrmiirkten und Märkten
und in den Städten, sah ich Tod und Finsterniö in allen, welche
die Kraft dez Herrn nicht ergriffen hatte. Alß ich durch Leieestershire
zog, kam ich nach Twy-Croß; daselbst waren Steuereinnehmer.
Der Herr trieb mich zu ihnen zu gehen und sie zu ermahnen,
sich vor Unterdrückung der Armen zu hüten. Das machte den
Leuten einen großen Eindruck. GS war in jener Stadt ein ange—
sehener Mann, welcher lange krank gewesen war und von den
Arzten aufgegeben wurde; und etliche Freunde aus- der Stadt
wünschten, daß ich zu ihm gehe. Jch ging zu ihm hinauf in sein
Zimmer und sagte ihm das Wort deß Lebens, und es trieb mich,
1) Runter, eine Sekte von mystischen Schwürmern, die sich rtihmten,
daß Christus in ihnen wohne, aus ihnen rede und sie selbst Christuö seien;
daher der Spottname ,,Ranter«——Prahler.


% \picinclude{./030-039/p_s031.jpg} 
Der Turnnlt in Nottingham. Wachsender Widersnmd usw. 31
mit ihm zu beten. Und der Herr erhörte uns und machte ihn
gesund. Als ich aber darauf in einem untern Raum des Hauses
zu der Dienerschaft und einigen andern Anwesenden redete, stürzte
einer aus einem Nebengemach herein mit dem nackten Degen in
der Hand, gerade auf mich los?-. Jch sah ihn unerschrocken an
und sagte: ,,Wehe Dir, arme Kreatur, was willst Du tun mit
Deiner fleischlichen Waffe? mir ist sie nicht mehr als ein Stroh-
halm.« Die Anwesenden waren sehr bestürzt und er entfernte
sich in Zorn und Wut. Alß sein Herr davon hörte, entließ er
ihn auö feinem Dienst. Also beschützte mich der Herr und half
diesem Schwachen und er wurde später den ,,Freunden« sehr
zugetan; und alö ich wieder in jene Stadt kam, besuchte er mich
mit seinem Weibe .....
Als ich nach Derbi) kam, wohnte ich im Haufe eineß Arztes?-;
eine Frau wurde gewonnen und noch Viele andere. Alß ich in
mein Zimmer ging, läutete die Glocke deß Turmhausej-’; nur schon
sie zu hören, ging mir durch Mark und Bein; ich fragte warum
die Glocke läute? man sagte mir, daß an dem Tage eine große
gotteödienstliche Versammlung stattfinde, dazu viele auß dem
Heer, sowie Priester und Prediger kommen werden. Da trieb
eß mich, auch hin zu gehen; und alö sie fertig waren, redete ich
zu ihnen, wa?. mir der Herr eingab. Sie waren ziemlich ruhig;
aber eine Wache kam, nahm mich bei der Hand und sagte, ich
müsse vor den Rat sowie auch die andern beiden, die mit mir
waren. Um die erste Nachmittags:-’stunde wurde ich-vorgenommen.
Jch wurde gefragt, warum ich hingegangen sei. Jch sagte, Gott
habe mich getrieben, es zu tun, und weiter sagte ich. ,,Gott
wohnet nicht in Tempeln mit Händen gemacht.« Jch sagte ihnen
ferner, all ihr Predigen, ihr Taufen und ihr Opfern werde sie
nie heiligen, und ermahnte sie, auf Ehristum in ihnen zu schauen
und nicht aus Menschen; denn Christuö sei es, welcher sie heilige.
Darauf ergingen sie sich in Vielen Worten, aber ich sagte ihnen,
sie sollten sich nicht über Gott und Christus streiten, sondern
ihm gehorchen. Die Kraft Gotteß donnerte unter ihnen und
sie zerstoben davor wie Spreu. Sie hießen mich mehrmalö
aus- dem Zimmer gehen und dann wieder hereinkommen und
trieben mich hin und her; von ein Uhr an bis abends neun ver-
hörten sie mich. Zuweilen sagten sie mir mit höhnischen Worten,
ich sei nicht bei Sinnen. Zuletzt fragten sie mich, ob ich geheiligt


% \picinclude{./030-039/p_s032.jpg}
sei; ich antwortete: »Ja, denn ich war im Paradies- Gotte?-,«
(2. Cor. 12, 4). Dann fragten sie mich, ob ich keine Sünde habe.
Jch antwortete: ,,ChristuZ, mein Erlöser, hat die Sünde von mir
genommen und in ihm ist keine Sünde.« Sie fragten, wie ich
wüßte, daß Christus in uns wohne? Jch sagte: »Durch seinen
Geist, den er unctz gegeben.« Um mich zu versuchen, fragten sie,
ob einer oon uns Christus sei? Jch antwortete: »Nein, wir
sind nichts, Christuö ist alle?-.« Sie sagten: wenn ein Mann
stehle, ob daß keine Sünde sei? Jch antwortete: »AlleS Unrecht
ist Stinde.« Alß sie ez nun müde geworden, mich zu verhören,
verurteilten sie mich zu sechß Monaten im Korrektionßhauö in
Derby alö Gotteölästerer, wie auS folgendem Verhastbesehl zu
ersehen ist:
An den Oberaufseher des Korrektionöhauseß in Derby.
,,Hiemit senden wir euch die Personen George Fox, oormalß
in Manßfield in der Grafschaft Nottingham, und John Fretwell,
Landwirt, vormalß in Stanießby in der Grafschaft Derby, vor
uns gebracht am heutigen Tag und beschuldigt eingestandener
Äußerungen verschiedener gotteölästerlicher Ansichten, die einem
jüngst verfaßten Parlament?-beschluß1) zuwider sind; sie sollen daher
sogleich nach Einsicht Dieses aufgenommen werden, besagter
George Fox und Johann Fretwell, in euern Gewahrsam und
darin sicher verwahrt werden, für die Dauer oon 6 Monaten,
ohne Möglichkeit einer Bürgschaft oder Abkürzung, es wäre denn,
daß sie sich htulttnglich durch ein gutes Betragen auöweisen, oder
durch unsere eigene Verordnung frei würden. Solcheö zutun
möget ihr nicht versäumen.
Mit unsrer Hand und Siegel gegeben am heutigen Tage
:30. Oktober 1650. Ger. Bennet. «
Nath. Barton.« .....
Während ich im Gefängnis war, kamen oft ,,Fromme«, um eine
Unterredung mit mir zu haben; noch ehe sie etwas sagten, merkte
ich immer, daß sie kamen, um für die bleibende Sündhaftigkeit und
Unoollkommenheit einzutreten. Jch fragte sie, ob sie glänbig seien und
11 Partamentöbeschluß vom 2. Mai 1648 gegen GotteHläster11ng und
Ketzerei. Ein Beschluß, der von der unglaublichen Härte der damals regierenden
Pre-zbyterianer zeugt.


% \picinclude{./030-039/p_s033.jpg} 
Der Tnmult in Nottingham. Wachsender Widerstand usw. Z3
Glauben hätten? Sie sagten: ,,Ja.« Jeh fragte sie: in wen?
Sie sagten: ,,J«n Christu-3.« Jch erwiderte: Wenn ihr wahre
an Christus- Glaubende seid, so seid ihr vom Tode zum Leben
eingegangen, und wenn ihr vom Tode frei seid, dann seid ihr ez
auch von der Sünde, die den Tod bringt. Und wenn euer
Glaube wahr ist, so wird er euch den Sieg geben über Sünde
und—Tc-ufel und eure Herzen und Gewissen reinigen — denn der
wahre Glaube ist in reinen Gewissen (1 Tim. 3) und er wird
machen, daß ihr Gott gefallet und euch wieder Zugang zu ihm
Verschaffen.« Aber sie wollten nicht von Reinheit und von Sieg
über Sünde und Teufel hören; denn sie sagten, sie können nicht
glaubeny daß jemand könne frei von Sünde sein schon dießseitß
des Grabe-3. Jch hieß sie, da-Z Schwatzen über die Schrift, die
das Wort heiliger Männer sei, aufgeben, wenn sie für Unheiligkeit
eintreten wollten. Einmal kam auch eine Anzahl solcher ,,Frommer«
zu mir und fingen an, die Sündhaftigkeit zu befürworten. Jch
fragte sie: ob sie Hoffnung hätten? ,,Ja, ja! daß wäre, wenn
wir keine Hoffnung hötten!« Jch fragte sie: ,,Waö für eine
Hoffnung ist etz, die ihr habt? Jst Christus in euch die Hoffnung
eurer Herrlichkeit? (Col. 1, 27.) Reinigt sie euch, gleich wie er
rein ist?« Aber sie wollten nichtö davon hören, «daß sie selber
hienieden schon rein werden sollten. Darauf gebot ich ihnen,
nicht mehr über die Schrift zu reden, welche daß Wort heiliger
Männer sei. Denn die heiligen Männer, welche die Schrift ge-
schrieben haben, seien für Heiligkeit in Herz, Leben und Wandel
hienieden eingetreten. ,,Jhr aber«, sagte ich, ,,tretet für Unreinheit
und Sünde ein, die vom Teufel sind, wa-3 habt ihr zu schaffen
mit den Worten heiliger Männer?«
Der Kerkermeister, ein großer ,,Frommer«, hatte eine schreck-
liche Wut auf mich und redete sehr schlecht von mir. Aber etz
gefiel dem Herm, ihn eines Tageö so mächtig zu ergreifen, daß
er in großer Angst und innerer Not war. Alö ich in meinem
Zimmer umherlief, hörte ich klägliche Laute und hörte, wie er zu
seiner Frau sagte: ,,Frau, ich habe den Tag des Gerichts gesehen,
und George Fox war da, und ich hatte Angst vor ihm, weil ich
ihm so viel böseß zugefügt hatte und so vieles wider ihn zu den
Vorgesetzten und ,,Frommen« gesagt hatte und zu den Richtern
und in den Wirt?-häusern.« Hierauf kam er gegen Abend zu mir
ins Zimmer und sagte: »Jch bin gegen euch gewesen wie ein
George Fox- 3


% \picinclude{./030-039/p_s034.jpg} 
Löwe; nun aber komme ich wie ein Lamm und wie der Kerker-
meister, der zitternd zu Pauluß und Silaß kam.« Und er bat,
daß er bei mir bleiben dürfe. Jch sagte, ich sei in seiner Macht
und er könne mit mir machen, waß er wolle; aber er sagte:
nein, er wolle meine Grlaubniß haben, und er möchte, daß er
immer mit mir sein könnte, aber nicht mich alß Gefangenen haben;
er und sein Hauß seien meinetwegen geplagt gewesen. Jch erlaubte
ihm denn, bei mir zu sein, und er öfsnete mir sein Herz rmd
sagte, er glaube, daß daß, waß ich vom wahren Glauben und
von der wahren Hoffnung sage, wahr sei, und er wunderte sich,
daß der andere, der mit mir gefangen war, nicht dabei bleibe.
Gr sagte: ,,Jener andere tat unrecht, ihr aber seid ein Gerechter.«
Gr gestand mir auch, daß oft, wenn ich ihn gebeten hatte, mich
unter daß Volk gehen zu lassen, um ihnen daß Wort deß Herrn
zu verkünden, und er eß mir verweigert habe, habe er sich damit
eine große Last auferlegt; denn er sei in große Angst geraten
und einige Zeit ganz verstört und niedergedriickt gewesen, so daß
er gar keine Kraft mehr gehabt habe. Am Morgen ging er fort
und ging zu den Richtern und sagte ihnen, wie er und sein Hauß
meinetwegen geplagt gewesen seien, und einer der Richter erwiderte
ihm, daß auch sie geplagt seien, darum daß sie mich sesthielten.
Eß war Richter Bennet zu Derby, welcher un-3 zuerst Quäkers)
genannt hatte, weil ich ihnen gesagt hatte, gsie müßten erzittem
vor dem Wort Gotteß. Solcheß geschah im Jahre 1650.
Hierauf erlaubten mir die Richter, eine Meile weit zu gehen.
Joh sah, wo sie hinauß wollten und sagte dem Kerkermeister,
wenn sie mir zeigen wollten, wie weit eine Meile sei, so wolle
ich manchmal so weit gehen; denn ich glaube, sie dachten, ich
würde davon laufen. Und der Kerkermeister gestand nachher,
daß sie eß in dieser Absicht gestattet hätten, damit ich entkomme
und sie von ihrer Angst befreit würden; aber ich sagte ihm, daß
ich nicht diesen Geist habe. «
Dieser Kerkermeister hatte eine Schwester, ein kränklicheß
jungeß Weib. Sie kam zu mir, um mich zu besuchen; und nach-
dem sie einige Zeit bei mir gewesen war, und ich Worte der
Wahrheit zu ihr geredet hatte, ging sie hinunter und sagte den
1) Quüker, das heißt ,,Zitterer«, der Spottname, den die Gegner den
Freunden anhängten, wegin der in ihren ersten Versammlungen sich einstellendea
Konvulsionen.


% \picinclude{./030-039/p_s035.jpg} 
Erlebnisse im Gefängnis zu Derbi) usw. Z5
andern, wir seien unschuldige Leute und täten niemand nichts zu
leide, sondern allen nur Gutes, sogar solchen, die uns haßten,
und bat sie, freundlich gegen mich zu sein. ....
Während ich im Korrektionshaus war, besuchten mich meine
Verwandten, und da sie über meine Gefangenschaft bekümmert
waren, gingen sie zu den Richtern und baten sie, daß ich mit
ihnen heim gehen dürfe. Sie erboten sich, sich mit hundert Pfund
zu verbiirgen und einige andere aus Derby, die mit ihnen waren,
je mit siinszig Psund, daß ich nicht mehr dorthin komme, um
gegen die Ptiestet zu reden. So wurde ich vor die Richter
gebracht, und weil ich nicht einwilligen wollte, daß irgendjemand
sich meinetwegen verpftichte, — denn ich war ja keines Vergehens
schuldig und hatte das Wort des Lebens und der Wahrheit ge-
redet, — erhob sich Richter Bennet zornig, und als ich niederkniete, um
Gott zu bitten, ihm zu vergeben, rannte er auf mich los und schlug
mich mit beiden Händen und schrie: »Fort mit ihm! Kerkermeister,
nimm ihn fort!« Hierauf wurde ich wieder in den Kerker gebracht
und mußte dort bleiben, bis meine Zeit von sechs Monaten um
war. Aber ich durfte nun eine Meile weit allein gehen, was ich
tat, als ich fühlte, daß ich es durfte. Ost ging ich auf den Markt
und in die Straßen und ermahnte die Leute, sich von ihrer
Schlechtigkeit zu bekehren, und ging dann wieder ins Gefängnis.
Und da Leute von allerleiislteligionen mit mir im Gefängnis
waren, ging ich hie und da zu ihnen und wohnte ihren Versamm-
lungen an den Ersten Tagen bei ....

%%%%%%%%%%%%%%%%%%% Kapitel 4. %%%%%%%%%%%%%%%%%%%%%%%%%%%%%%
\chapter[Kampf gegen die Ranter]{Kampf gegen die Ranter}

\begin{center}
\textbf{Erlebnisse im Gefängnis zu Derby. Ein \zitat{Wehe} 
über die Stadt
Lichfield\ort{Lichfield}. Erste Missionsgenossen. Antikirchliche 
Agitation und Kampf gegen die Ranter\index{Ranter}.}
\end{center}


Während ich noch im Gefängnis war, kam ein Soldat zu
mir und erzählte hmir, wie er im Turmhause gewesen sei und dem
Priester zugehört habe, und wie dann auf einmal eine grofze Angst
über ihn gekommen sei und die Stimme des Herrn also zu ihm
geschehen sei: ,,Weißt du nicht, daß mein Diener im Gefängnis
ift? zu ihm gehe und frage ihn um Rat«. Jch redete mit ihm
wie es sein gegenwärtiger Zustand erheischte, und sein Verständnis
zbt


% \picinclude{./030-039/p_s036.jpg} 
wurde geöffnet. Jch sagte ihm, daß der, welcher ihm seine Sünden
ausdecke und ihn um ihretwillen ängstige, ihm auch die Rettung
zeigen werde; denn der dem Menschen die Sünden aufdeckt, ist
derselbe, der sie auch hinwegnimmt. Während ich mit ihm redete,
offenbarte sich ihm der Herr, so daß er anfing, die Wahrheit dez
Herrn und Gorteß Gnade zu erkennen; er fing an, unerschrocken
in seinem Regiment unter den Soldaten von der Wahrheit zu
reden; denn die Schrift wurde ihm mehr und mehr offenbar, und
er ging soweit zu sagen: sein Oberst sei blind wie Nebukadnezar,
daß er den Diener deß Herrn inß Gesängnis werfe. Von da an
hegte sein Oberst einen Groll gegen ihn. Alß im darauffolgenden
Jahre in der Schlacht von Worcester die beiden Armeen neben-
einander lagen, kamen zwei auß der Armee des Könige und
forderten, daß zwei anß der Armee des Parlamentß sich mit ihnen
schlagen sollten; da wählte der Oberst ihn und noch einen, um
der Forderung Folge zu leisten. A13 sein Kamerad im Kampfe
gefallen war, trieb er seine beiden Gegner zur Stadt hinauö, ohne
einen Schuß auf sie abzuseuern; dies erzählte er mir nach seiner
Rückkehr mit eigenem Munde. Nach Beendigung der Schlacht
sah er die Betrügerei und Heuchelei der Offiziere ein, und im
Gedanken daran, wie wunderbar der Herr ihn bewahrt hatte und
waß etz eigentlich um den Krieg sei, legte er die Waffen nieder.
Die Zeit meiner Gefangenschaft war nun fast zu Ende und
da viel neue Soldaten aus-gehoben wurden, so wollten mich die
Kommifsäre zu ihrem Hauptmann machen, und die Soldaten
erklärten, sie wollten keinen andern als mich haben. Der Kerken
meister erhielt den Befehl, mich vor die Soldaten und ihre Vor-
gesetzten aus den Marktplatz zu siihren; dort boten sie mir dieseß
Ehrenamt, wie sie eß nannten, an und fragten mich, ob ich nicht
wolle die Waffen ergreifen für den Commonwealth gegen Karl
Stuart.1) Ich erwiderte ihnen, ich wisse wohl, woher aller
Krieg komme: auö der Begierde, wie schon Jakobus- lehre
(Jak. 4); ich aber stehe in jener Kraft und jenem Leben,
die von vornherein allen Krieg ausschließen. Sie wollten mich
überreden, ihr Anerbieten anzunehmen; sie meinten, ich weigere
mich nur aus Bescheidenheit. Aber ich erklärte ihnen, ich sei in
den Bund des- Friedenö eingetreten, welcher bestanden, ehe es
1) 1651 Schlacht von W1-rceöter zwischen Cron1toell(Con1n1onwealth) und
Karl ll.


% \picinclude{./030-039/p_s037.jpg} 
Erlebnisse im Gefängnis zu Derby usw. 37
Krieg und Zank gab. Sie sagten, sie bieten es mir in Liebe und
Zuneigung an wegen meiner Tugend, und ähnliche Schmeicheleien
mehr. Aber ich sagte ihnen, wenn solches ihre Liebe sei, so trete
ich sie mit Füßen. Da wurden sie zornig und sagten: ,,Nimm
ihn hinweg, Kerkermeister, und wirs ihn in den untersten Kerker
zu den Schelmen und Verbrechern.« Jch wurde weggefiihrt und
an einen wiisten, stinkenden Ort 1) gebracht, wo kein Bett war,
mit 30 Verbrechern, wo ich beinahe ein halbes Jahr gefangen
war, außer, wenn sie mich dann und wann ein wenig in den
Garten ließen, weil sie sicher waren, daß ich nicht davon laufe.
Es hatte damals, als man mich in diesen Kerker gebracht hatte,
geheißen, ich werde wohl nicht mehr heraus kommen. Aber ich
glaubte an Gott und daß ich zu seiner Zeit daraus befreit werde.
Denn der Herr hatte es mir vorausgesagt, daß ich nicht bald
von diesem Ort wegkomme, da ich dort eine Ausgabe für ihn zu
erfüllen habe.
Als es bekannt wurde, daß ich im Kerker von Derby sei, kamen
meineAngehörigen, um mich wieder zu besuchen; denn sie betrachteten
es als eine große Schande für sie, daß ich um der Religion
willen gefangen war; und etliche hielten mich für verrückt, weil
ich für die Reinheit, Gerechtigkeit und Vollkommenheit eintrat.
Unter denen, die zu mir kamen, war einer aus Nottingham,
ein Soldat, der früher Baptist gewesen war. Jin Laufe des Ge-
sprächs sagte er zu mir: »Dein Glaube gründet sich auf einen
Mann, der in Jerusalem gestorben sein soll; solches ist aber nie
geschehen«. Gs betrübte mich sehr, ihn so reden zu hören, und
ich sagte: ,,Wie! hat nicht Christus gelitten vor den Toren Jeru-
salems durch die Juden, die ,,Frommen«, die Hohenpriester und
durch Pilatus?« Aber er leugnete, daß Christus je äußerlich
gelitten habe. Jch fragte ihn, ob denn keine Hohenpriester, keine
Juden, kein Pilatus äußerlich dort gewesen sei? und als er das
nicht bestreiten konnte, sagte ich: ,,So gewiß ein Hohepriester,
ein Pilatus und Juden äußerlich dort gewesen sind, so gewiß ist
Christus äußerlich verfolgt worden von ihnen und hat durch sie
1) Die Zustände der Gefängnisse und Korrektionshäuset im 17. Jahrh.
waren überaus traurig. Überall herrschte große Unreinlichkeit; die Verwaltung
war der Willkür des Gefängnisvotstehers anheim gegeben, der nicht besoldet
war, sondern von den Gefangenen bezahlt wurde, die die Kosten ihres Aufent-
haltes selbst tragen mußten. Vgl. Aschrott, Englisches Gesängniswesen.


% \picinclude{./030-039/p_s038.jpg} 
gelitten«.' Die Reden dieses Menschen veranlaßten eine Ver-
leumdung gegen uns, als ob die Quäker bestritten, daß Christus
gelitten habe und in Jerusalem gestorben sei. Gs war dies ganz
falsch; nie war der leiseste Gedanke daoon in unsern Herzen ge
wesen; es war eine bloße Verleumdung, die uns traf, und die
aus dem Gerede dieses Menschen entstanden war. Derselbe
Mässch behauptete auch, niemals habe irgend ein Apostel oder
Prophet, oder Heiliger oder Mann Gottes äußerlich gelitten; alle
ihre Leiden seien innerlich gewesen; aber ich bewies ihm anBei-
spielen, wie viele unter ihnen gelitten und durch wen sie gelitten;
und so widerlegte die Kraft des Herrn seine oerkehrten Ansichten.
Eine andere Sorte kam zu mir, die behaupteten, sie könnten
Geister unterscheiden. Jch fragte sie, welches der erste Schritt
zum Frieden sei? und in was der Mensch seine Rettung suchen
müsse? Sie fuhren auf und sagten in ihrem Hochmut, ich sei
verrückt; und solche wollten Geister unterscheiden können und
kannten nicht einmal ihren eigenen Geist!
Während dieser Zeit meiner Gefangenschaft geriet ich in
große Bekümmernis über das Vorgehen der Richter und Beamten
in ihren Gerichtshösen. Gs trieb mich, an die Richter zu schreiben,
darum daß sie das Todesurteil fällten wegen allerlei unwichtiger
Vergehen, in Geldsachen oder das Vieh betreffend. Jch mußte
ihnen zeigen, wie solches von jeher dem Gesetz Gottes zuwider
war; ich war deswegen in meinem Geiste sehr betrübt bis inden
Tod, aber da ich mich unter den Willen Gottes stellte, so er-
wachte ein himmlisches Sehnen nach dem Herrn in meinem Herzen,
ich sah den Himmel offen und freute mich und gab Gott die
Ehre ....
Jn diesem Zustande trieb es mich, an die Richter zu schreiben,
wie schädlich es für die Gefangenen sei, so lange im Kerker zu
sein, wie sie da schlechtes von einander lernten, wenn sie mit-
einander über ihre bösen Taten reden. Darum sollten die Urteile
rasch gesprochen werden. Denn ich war ein gottseliger Jüngling
rmd wandelte in der Furcht des Herrn; es betrübte mich, ihre
schlechten Reden zu hören, ich mußte ihnen oft Vorstellungen über
ihre bösen Worte machen und über ihr häßliches Betragen unter-
einander. Die Leute wunderten sich, wie ich bewahrt und behiitet
blieb; denn nie konnten sie mir ein Wort oder eine Tat nach-
weisen, die sie hätten zu meinen Ungunsten auslegen können


% \picinclude{./030-039/p_s039.jpg} 
Erlebnisse im Gefängnis zu Derby usw. 39
während der ganzen Zeit, die ich dort war; denn die unendliche
Kraft des Herrn hielt mich aufrecht und bewahrte mich während
der ganzen Zeit; ihm sei Lob und Ehre immerdar.
Es war eine junge Person mit mir im Gefängnis, die ihrem
Herrn Geld gestohlen hatte. Llls sie zum Tode verurteilt werden
sollte, schrieb ich an den Richter und ans Schwurgericht und
stellte ihnen vor, wie es immer gegen das Gesetz Gottes gewesen
sei, die Leute wegen Diebstahls zum Tode zu verurteilen, und
bat um Gnade. Sie wurde aber doch verurteilt, und man grub
ihr ein Grab und führte sie zur Hinrichtung. D,a schrieb ich noch
einmal ein paar Worte und warnte alle, sich vsor Raubgier und
Habsucht zu hüten, da sie von Gott wegführe, und ermahnte alle
den Herrn zu fürchten, allen irdischen Begierden zu entsagen und
die Zeit zu nützen, dieweil sie da ist; solches hieß ich sie unter
dem Galgen vorlesen. Und obgleich sie sie schon auf der Leiter
hatten, bereit gehenkt zu werden, mit einem Tuch über den Augen,
so wurde sie nun nicht hingerichtet, sondern sie führten sie wieder
zurück ins Gefängnis, und im Gefängnis kam sie nachher dazu,
Gottes ewige Wahrheit zu erkennen.
Es war noch ein anderer Gefangener mit mir, ein schlechter,
gottloser Mensch, ein bertichtigter Schwarzkünstler und Zauberer.
Er drohte, was er alles zu mir sagen und mir tun wolle, aber
er hatte keine Macht, den Mund gegen mich aufzutun. Einmal
gerieten der Kerkermeister und er aneinander und er drohte, er
wolle den Teufel rufen und das Haus niederreißen, so daß der
Kerkermeister Angst bekam. Da trieb mich der Herr hinzugehen
und ihm Einhalt zu gebieten und zu sagen: ,,Komm, laß sehen
was du kannst, tue dein Außerstes«. Jch sagte ihm, der Teufel
sei schon in ihm selber bei uns, die Kraft des Herrn binde ihn
aber. Da schlich er sich davon.
Als nun die Zeit der Schlacht von Worcester kam, sandte
der Richter Vennet Konstabler, um mich zu zwingen, Soldat zu
werden, da er gesehen hatte, daß ich kein Kommando übernehmen
würde. Jch sagte ihnen, ich sei ganz gegen allen äußeren Krieg.
Sie kamen wieder, um mir Werbegeld zu geben, aber ich nahm
es nicht. Daraus wurde ich vor den Wachtmeister Holes gebracht,
der mich eine Weile behielt und dann wieder zurückschickte. Nach
einiger Zeit wurde ich wieder heraufgeholt und vor den Kommifsär
gebracht, welcher erklärte, ich müsse als Soldat gehen, aber ich



% \picinclude{./040-049/p_s040.jpg} 
40 Kapitel 17.
sagte ihnen, ich sei hiestir tot. Sie sagten, ich sei ja am Leben.
Jch sagte ihnen, wo Neid und Zank sei, da sei Verderben (Jak. 3, 16).
Sie boten mir zweimal Geld an, aber ich wollte nichtß an-
nehmen; daraus wurden sie böse und verurteilten mich zum Ge-
fängnis- ....
Jch war tief betrübt und bearbeitet in meinem Geist während
meiner Gefangenschaft wegen der Schlechtigkeit, die in der Stadt
herrschte; denn obgleich etliche gewonnen waren, so war doch die
Mehrzahl sehr oerhärtet. Jch sah, wie sich das- Außgießen der
Liebe Gotteß von ihnen wegwandte. Ich trauerte über sie, und
es kam über mich, folgende Klage über sie zu verbreiten:
,,O Derby! Wie die Wasser abfließen, wenn die Schleusen
sich öffnen, also fließet die Liebe Gotteß von dir ab, o Derby.
Darum siehe zu, wo du stehest und auf welchem Grund du bist,
ehe du gänzlich verlassen wirst. Der Herr hat mich zweimal ge-
rufen, ehe ich zu dir kam, um gegen deine Eitelkeit und Schlech-
tigkeit aufzutreten und alle zu ermahnen, auf den Herm und
nicht auf Menschen zu sehen. ,,Wehe der prächtigen Krone der
Trunkenen! der welken Blume ihrer Herrlichkeit« (Jes. 28, ll.
Wehe denen, die mit Worten ihren Glauben zur Schau tragen und
doch hochmütig und hochfahrend sind und Unterdrückung und Haß
üben. O Derby! Deine Frömmigkeit und dein Predigen stinken
gen Himmel! Jhr feiert einen Sabbat in Worten und versammelt
euch, um euch schön zu kleiden, ihr frönet der Eitelkeit. Die
Weiber gehen mit aufgerichtetem Halse und geschminkten Ge-
sichtern, wie ez die alten Propheten verurteilt haben (Jes. 3, 16).
Eure Versammlungen sind dem Herrn ein Greuel; ihr erhebet
die Eitelkeit und beuget euch davor; das Laster gedeiht und da-Z
Böse wird geehrt; daö Schlechte wird von den Schlechten ge-
duldet und doch bekennen sie alle Christus mit Worten. O über
die Schlechtigkeit unter euch! EH bricht mir fast das Herz, zu
sehen, wie Gott unter euch verachtet ist, o Derby!«
A18 ich gesehen, wie Gottez Liebe sich von diesem Orte ab-
wandte, wußte ich, daß meine Gefangenschaft hier nun nicht mehr
lange andauern werde, aber ich sah, daß, wenn der Herr mich
srei machen werde, so werde eß sein, wie wenn man einen
Löwen auß seiner Höhle auf die wilden Tiere dee Waldes ab-
läßt. Denn alle »Frommen« hatten eine tierische Gesinnung, die
der Sünde huldigte, so lange sie lebten. Sie waren alle dem


% \picinclude{./040-049/p_s041.jpg} 
Erlebnisse im Gefängnis zu Derby usw. 41
Geist und dem Leben seiud, der in der Schrift gegeben ift und
den sie in Worten bekannten. So geschah ez, wie man hernach
sehen wird.
Ez stand ein Gericht über der Stadt, und den Behörden war
ez unbehaglich meinetwegen; aber sie wußten nicht, waz sie mit
mtr machen sollten. Einmal wollten sie mich vorz Parlament
schicken, ein andermal mich nach Jrland oerbannen. Zuerst
nannten sie mich einen Betrüger und Verfiihrer und Gottes-
lästerer; dann, alz Gott seine Strafe über sie schickte, sagten fie,
ich sei ein ehrlicher, tugendhafter Mensch. Aber ob sie eine gute
oder schlechte Meinung von mir hatten, war mir gleichgültig;
denn weder richtete mich daz eine auf, noch warf mich das andere
nieder, dem Herrn sei Lob. Schließlich mußten sie mich frei
lassen, zu Anfang des Winterz 1651, nachdem ich fast etn Jahr
in Derby gefangen gewesen war, sechz Monate im Zuchthauz
und die übrigen im Kerker.
Alz ich nun wieder meine Freiheit hatte, fuhr ich fort wie
zuvor in der Arbeit für den Herrn und zog im Lande umher,
zuerst in der Gegend meiner Heimat, Leicestershire; ich hielt unter-
wegz Versammlungen, und dez Herrn Geist und Kraft war mit
nur ....
Einmal alz ich mit einigen Freunden unterwegz war und
eine Turmhauzspitze erblickte, ging ez mir durch Mark und Bein;
ich fragte, waz daz für eine Ortschaft sei? ez hieß: Lichfield.
Alsobald erging daz Wort dez Herrn an mich, daß ich dorthin
gehen müsse. Alz wir bei dem Hause angelangt waren, in daz
wir gehen wollten, bat ich die Freunde, die mit mir waren, hinein-
zugehen; ich sagte ihnen aber nicht, wohin ich zu gehen hatte.
Sobald sie im Hause waren, entfernte ich mich und lief über
Hecken und Gräben, biz ich eine Meile weit von Lichsield ent-s
fernt war; da waren auf einem weiten Felde Schäfer, die ihre
Schafe hüteten. Hier befahl mir der Herr, meine Schuhe auzzu-
ziehen; ich zögerte, denn ez war Winter; doch daß Wort dez
Herrn war wie Feuer in mir. So zog ich denn meine Schuhe
aus und ließ sie bei den Schäfern, und die armen Schäfer zitterten
und waren ganz bestürzt. Darauf lief ich wieder eine Meile,
und sobald ich wieder in der Stadt war, erging daz Wort dez
Herrn an mich: ,,Rufe: wehe der blutigen Stadt Lichfield!« Ich
ging also die Straße auf und ab und rief: ,,Wehe der blutigen


% \picinclude{./040-049/p_s042.jpg} 
42 Kapitel 17.
Stadt Lichfield!« Da ez Markttag war, ging ich aus den Markt-
platz, lies aus demselben umher und rief von Zeit zu Zeit: ,,W-ehe
der blutigen Stadt Lichfield!« Und niemand tat mir etwaß.
Während ich rufend durch die Straßen ging, schien es mir, 11lS
ob ein Bach von Blut durch die Straße fließe, und der Markt-
platz kam mir vor wie ein Teich von Blut. Alö ich mich der
mir aufgetragenen Verkündigung entledigt hatte, verließ ich im
Frieden die Stadt. Jch kehrte zu den Hirten zurück, gab ihnen
Geld tmd erhielt meine Schuhe von ihnen zurück. Aber das
Feuer dez Herrn war so in meinen Füßen und in meinem ganzen
Körper, daß mir nichts daran lag, meine Schuhe überhaupt wieder
anzuziehen; und ich wußte nicht recht, ob ich ez tun sollte oder
nicht, bis ich die Grlaubniß dazu vom Herrn fühlte; nachdem ich
meine Füße gewaschen, zog ich meine Schuhe wieder an. Darauf
versiel ich in tiefeß Nachstnnen, warum und aus welchem Grunde
ich wohl gesandt worden sei, gegen diese Stadt zu reden und sie
die ,,blutige Stadt« zu nennen; denn obwohl eine Zeitlang daß
Parlament und eine Zeitlang der König die Herrschaft über diesen
Kirchenspengel gehabt hatte und viel Blut in der Stadt vergossen
worden war während des- Krieges zwischen beiden, so war ez
doch nicht schlimmer gewesen alß an vielen anderen Orten auch.
Nach und nach aber fiel es mir ein, wie zur Zeit dez Kaiserß
Diocletian tausend Christen in Lichsield gemartert worden waren;
darum hatte ich ohne Schuhe durch den Bach ihreß Bluteß gehen
müssen, damit die Erinnerung an das Blut jener Märtyrer, daß
vor mehr als tausend Jahren vergossen worden und in ihren
Straßen erkaltet war, wach werde. Die Nachwirkung jenes Bluteß
war über mich gekommen, so daß ich dem Herrn hatte gehorchen
müssen. Man weiß auö alten Uberlieserungen, wie viel christ-
liche Vriten dort gelitten haben. Ich könnte noch viel berichten
über alle:-’, waß sich mir offenbarte über daß hier während der
zehn Verfolgungen und später vergossene Märtyrerblut, aber ich
überlasse es dem Herrn und seinem Buch, au?7 welchem alleß
gerichtet werden wird; denn sein Buch und fein Geist sind sichere
Uberlieserer.
Darauf zog ich im Lande umher und hatte vielerorts Ver-
sammlungen unter den freundlich Gesinnten. Aber meine Ange-
hörigen waren böse über mich. Nach einiger Zeit kehrte ich nach
Nottinghamshire zurück und ging dann nach Derbshire, um dort


% \picinclude{./040-049/p_s043.jpg} 
Erlebnisse im Gefängnis zu Derby usw. 43
die freundlich Gesinnten aufzusuchen. Jn Yorkshire und an einigen
andern Orten predigte ich Buße: darauf kam ich nach Balby,
wo Richard Famöworth 1) und einige andere gewonnen wurden.
So reiste ich im Lande umher, Buße predigend und daö Wort
dez Herm verkündigend, bis-3 ich in die Gegend von Wakefield
kam, wo James Naylor lebte; er und Thomaß Goodyear
kamen zu mir; beide wurden gewonnen und nahmen die Wahrheit
auf. Auch William Dem?-bury und seine Frau und viele andere
kamen zu mir, wurden gewonnen und nahmen die Wahrheit auf.
Von dort begab ich mich nach Hauptmann PurZloe’S Hauß in
die Nähe von Selby, und besuchte John Leek, der inß Gefängniß
zu mir gekommen war, und er wurde gewonnen. Ich besaß ein
Pferd, mußte mich aber leider davon trennen, da ich nicht wußte,
maß damit anfangen, weil mich der Herr trieb in manches- an-
gesehene Hautz zu gehen, um die Leute zu ermahnen, sich zum
Herrn zu bekehren. Unter anderm trieb mich der Herr auch inß
Turmhauß von Beverly zu gehen, daß damalß eine Stätte beson-
derer Frömmigkeit war; da ich vom Regen ganz durchnäßt war,
ging ich zuerst nach der Herberge. Jn der Türe kam ein junge-3
Weib auf mich zu und sagte: ,,Wie! seid ihr ez? Kommt herein«,
wie wenn sie mich schon gekannt hätte; denn die Kraft dez Herrn
hatte ihr Herz vorbereitet. Ich nahm etwaß zu mir und ging
inß Bett. Am Morgen zog ich meine noch nassen Kleider an
und bezahlte meine Zeche und begab mich ins Turmhauö, wo
einer predigte. A15 er geendet, trieb mich die mächtige Kraft Gotteß,
zu ihnen zu reden, und ich wies sie aus Chrisiuz, ihren Lehrer, hin.
Die Kraft dez Herrn war so mächtig, daß alle von großer Furcht
ergriffen wurden. Der Bürgermeister kam und sprach ein paar
Worte mit mir, aber niemand hatte Macht, mir etwaß zu tun.
Jch verließ die Stadt und ging am Nachmittag in ein anderez
Turmhauß, etwa zwei Meilen weit entfernt. A13 der Priester
geendet, trieb ez mich, eingehend zu ihm und den Leuten über
den Weg deö Lebens und der Wahrheit und den Grund der Gr-
wählung und Verdammung zu reden. Der Priester sagte, er sei
1) Richard Farnsworth, William Dewßbury und James Naylor waren
die ersten bedeutenden Missionsprediger der Quätet. (Näheres s. Weingarten,
Revolutionskirchen Englandtz. S. 218ss.) James Naylor ist in der Geschichte
betiichtigt geworden durch seinen Messiatzeinzug in Bristol, dem Höhepunkt der
saft zum Wahnsinn gesteigerten Schwärmerei des älteren Quälertumö.


% \picinclude{./040-049/p_s044.jpg} 
44 Kapitel lll.
zu kindlich, um mit mir zu dißputieren; ich erklärte ihm, ich sei
nicht gekommen, um zu di?-putieren, sondern um daß Wort dez
Lebenß und der Wahrheit zu verkünden, und damit sie alle den
Samen kennen lernen möchten, den Gott allen verheißeu, den
Männern wie den Frauen. Die Leute waren hier sehr empfänglich
und wünschten, daß ich wiederkäme an einem Wochentag, um
ihnen zu predigen, aber ich wie:3 sie an ihren Lehrer Jesuö
Christus und verließ sie. Am folgenden Tage ging ich nach
Cranstick zu Hauptmann Pnrßloe, der mich zu Richter Hotham
begleitete. Dieser war ein gottseliger Mann, der auch Gottes
Wirken schon in seinem Herzen verspürt hatte. Nachdem wir eine
Zeitlang über göttliche Dinge geredet hatten, nahm er mich mit
in sein Zimmer und bekannte mir, daß ihm diese Ansichten
schon seit zehn Jahren vertraut seien, und wie er sich freue, daß
der Herr sie nun auch verkünden lasse unter den Leuten. Nach-
her kam noch ein Priester zu ihm, mit dem ich auch über die
Wahrheit redete. Aber der war bald zum Schweigen gebracht,
denn er war ein bloßer Phantaft, der sich daß, wovon er redete,
innerlich nicht angeeignet hatte.
Während ich da war, kam eine angesehene Frau auß Beverly,
um Richter Hotham in irgend einer wichtigen Angelegenheit zu
sprechen. Jin Laufe dez Gesprächeß erzählte sie ihm, daß am
vergangenen Sabbat, wie sie diesen Tag nannten, ein Engel oder
ein Geist in die Kirche von Beverly gekommen sei und herrliche
Dinge von Gott geredet habe zur Verwunderung aller Anwesenden,
und alß er geendet habe, sei er verschwunden; sie wisse nicht, woher
er gekommen, noch wohin er gegangen sei, alle haben sich ge-
wundert, die Priester, die »Frommen« und die Behörden der Stadt.
Richter Hotham erzählte mir das- nachher wieder, woraus ich ihm
mitteilte, daß ich e3 gewesen, der an jenem Tage im Turmhauß
gewesen und die Wahrheit verkündet hatte ....
Am Nachmittag ging ich in ein andereß Turmhauö, wo ein
großer, angesehener Priester, ein Doktor, wie sie ihn nannten,
redete, einer von denen, die Richter Hotham wollte kommen lassen.
Jch ging hin und wartete, biS der Priester geendet hatte. Die
Worte, die er alö Text genommen hatte, waren: ,,Wohlan alle,
die ihr dursiig seid, kommet her zum Wasser, und die ihr nicht
Geld habt, kommt her, kauset und esset, kommt her und kauset
ohne Geld, beide:-3 Wein und Milch (Jes. 55, 1).*- Und der Herr


% \picinclude{./040-049/p_s045.jpg} 
Erlebnisse im Gefängnis zu Derby usw. 45
trieb mich zu sagen: ,,Komm herunter, du Verführer; heißest du
die Leute umsonst kommen und umsonst vom Wasser dez Lebenz
nehmen, und nimmst jährlich dreihundert Pfund dafür, daß du die
Schrift oerkündest? Errötest du nicht vor Scham? Tat der
Prophet Jesaiaß und Christuß, die diese Worte umsonst geredet
und mitgeteilt hatten, auch also? Sagte nicht Christus- zu seinen
Jüngern, alö er sie auösandte zu predigen: umsonst habt ihr ez
empfangen, umsonst gebet etz auch?« Der Priester machte sich
ganz bestürzt davon; nachdem er seine Herde verlassen hatte,
hatte ich so oiel Zeit, alß ich wollte, um zu den Leuten zu sprechen;
ich wieö sie von der Finsternis zum Licht und zur Gnade Gotteß,
die sie lehren und ihnen Rettung bringen werde, und zum Geist
Gotteß in ihrem Jnnern, der sie umsonst lehre.
Dann kehrte ich zu Richter Hothamö Hauß zurück; alß ich
eintrat, schloß er mich in seine Arme und sagte, sein Haus sei
mein Hau-3. Denn er freute sich sehr über daß Werk dez Herrn
und daß seine Kraft kund geworden. Dann erzählte er mir,
warum er am Morgen nicht mit mir zum Turmhauö gegangen
war, und was für Gründe er gehabt hatte; er hatte sich gesagt,
wenn er mit mir in-3 Turmhauß gehe, so würden die Wachen
mich ihm übergeben und da werde er so in die Sache verwickelt;
dann wisse er nicht, maß machen. Darum sei er froh gewesen,
alß Hauptmann Pur?-loe gekommen; aber keiner von ihnen war
in Amts-kleidung gewesen oder hatte den Kragen um den Halß I
gehabt. GZ war damalt-3 etwaß ganz Ungewöhnliche?-, daß einer
ohne Kragen inß Turmhauß kam; aber Hauptmann Purßloe
war ohne einen solchen mit mir ins Turmhauö gekommen, so hatte
die Kraft des Herrn ihn übernommen, daß er gar nicht daran
dachte.
Jch zog weiter und kam an einen Abend zu einer Herberge.
Jch bat die Wirtin, mir etwaß Fleisch zu bringen, wenn sie solches-
habe; aber weil ich »du« und ,,dich« zu ihr sagte, sah sie mich
besremdet an; ich fragte sie, ob sie Milch habe. Sie sagte: nein.
Jch merkte, daß sie nicht die Wahrheit sagte, und um sie noch
weiter zu prüfen, fragte ich sie, ob sie Rahm habe; sie verneinte eö
ebenfalls. Nun stand ein Butterfaß im Zimmer und ein kleiner
Knabe, der daneben spielte, steckte seine Hand hinein und stieß eß
um und oerschiittete allen Rahm vor meinen Augen auf den
Boden; da zeigte es sich, daß die Frau eine Lügnerin war. Sie


% \picinclude{./040-049/p_s046.jpg} 
46 Kapitel 17.
erschrak, stieß eine Verwünschung aus, hob das Kind auf und
schlug es tüchtig; aber ich machte ihr Vorwürfe wegen ihrer
Lüge und ihres Betrügens. Nachdem der Herr solcherweise ihre
Betrügerei und Bosheit aufgedeckt hatte, verließ ich das Haus
und ging weiter, bis ich zu einem Heuschober kam und brachte
nun die Nacht darin zu im Regen und Schnee, denn es war
drei Tage vor dem Tag, den sie Ehristfest nennen.
f Am folgenden Tage kam ich nach York, wo etliche sehr gott-
selige Leute waren. Am Ersten Tage der darauffolgenden
Woche hieß mich der Herr in das große Münster gehen und zum
Priester Bowles und seinen Zuhörern reden tn ihrer großen
Kathedrale. Jch ging hin und als der Ptiestet geendet, sagte ich,
ich habe ihm und der Gemeinde eine Botschaft von Gott dem
Herrn zu bringen. ,,Dann sage sie schnell!« sagte einer der
,,Frotnmen« aus der Versammlung; denn es war gefroren und
schneite und war sehr kaltes Wetter. Jch sagte ihnen, solches
seiüdas Wort des Herrn an sie: ,,Jhr lebet in Worten, aber der
Herr der Allmächtige verlangt Früchte von euch.« Kaum waren
die Worte aus meinem Munde, so stießen sie mich hinaus und
warfen mich die Stufen hinunter; aber ich stand aus, ohne verletzt
zu sein und ging in meine Wohnung. Etliche wurden überzeugt;
denn schon die Seufzer, die ich ausstieß unter dem Druck und
dem Zwang des Geistes Gottes in mir, genügten, um vieler
i Herzen zu öffnen und zu ergreifen, sodaß sie bekannten, die Seufzer,
die ich ausstoße, machen ihnen Eindruck.; mein ganzes Wesen
war bedrückt davon, daß sie bekannten und nicht besaßen, Worte
machten und keine Früchte brachten.
Nachdem ich für den Augenblick meinen Dienst in York getan
hatte und etliche dort gewonnen worden waren und die Wahrheit
Gottes angenommen und sich zu seiner Lehre bekannt hatten,
verließ ich York und wandte mich nach Cleveland und fand dort
Leute, welche die Kraft Gottes geschmeckt hatten. Ich sah, daß
ein Same in jener Gegend war, und daß Gott dort ein demiitiges
Volk hatte. Unterwegs holte mich, gegen Abend, ein Päpstlicher
ein und redete mit mir über seine Religion und über ihre Gottes-
dienste, und ich ließ ihn alles sagen, was er aus dem Herzen
hatte. Ich brachte die Nacht in einer Schänke zu; am folgenden
Morgen trieb mich der Herr, zu diesem Päpstlichen zu reden. Jch
begab mich in seine Wohnung und zeugte gegen seine Religion


% \picinclude{./040-049/p_s047.jpg} 
Erlebnisse im Gefängnis zu Derbi; usw. 47
und alle ihre abergläubischen Gebräuche und sagte ihm, Gott sei
gekommen, sein Volk selbst zu lehren; das brachte den Papisten
dergestalt auf, daß es ihn aus seinem eigenen Hause trieb ....
Obgleich zu der Zeit der Schnee sehr tief war, fuhr ich fort
herutnzureisen und kam zu einem Marktflecken, wo ich viele
,,Fromme« traf, mit denen ich lange Unterredungen hatte. Ich
stellte ihnen viele Fragen, die sie nicht beantworten konnten, weil
sie sagten, man habe sie noch nie in ihrem Leben so schwere
Dinge gefragt. Von da ging ich nach Stath, wo ich ebenfalls
viele ,,Fromme« und einige Ranter traf. Ich hatte große Ver-
sammlungen unter ihnen undsviele Bekehrungen. Viele nahmen:die
Wahrheit aus, worunter einer, der hundert Jahre alt war; ein
anderer war ein Oberkonstabler und einer war ein Priester, namens
Philipp Scafe. Diesen machte der Herr später durch seinen Geist
zu einem freien Verkündiger seines freien Evangeliums.
Der Priester dieses Ortes war sehr hochfahrend und bedrückte
die Leute sehr mit seinen Abgaben. Wenn sie aus den Fischfang
gingen, so machte er sie Abgaben vom Erlös bezahlen, obgleich
sie dieselben so weit her hatten und sie bis nach Yarmouth zum
verkaufen brachten. Es trieb mich, dort ins Turmhaus zu gehen,
um die Wahrheit zu verkünden und den Priester bloß zu stellen.
Als ich mit ihm geredet hatte und ihm die Unterdrückung des
Volkes vorgestellt hatte, lief er davon. Die Ältesten der Gemeinde
waren sehr hochmiitig und leichtfertig; darum verließ ich sie, nach-
dem ich das Wort des Lebens verkündet hatte, weil sie dasselbe
nicht aufnehmen wollten. Aber das Wort des Lebens, das ich
unter ihnen verkündet hatte, blieb bei etlichen von ihnen, so daß
etliche der Ersten aus der Gemeinde des Nachts zu mir kamen,
und die meisten wurden gewonnen und bekannten sich zur Wahrheit;
so begann die Wahrheit sich in dieser Gegend auszubreiten, und
wir hatten große Versammlungen; dadurch wurden die Priester
zornig und die Ranter fingen an, unruhig zu werden und ließen
mir sagen, sie wollten eine Unterredung mit mir haben, die Priester,
welche Unterdrückung übten, und die Runter. Es wurde ein Tag
festgesetzt und die Ranter erichienen; es kam auch noch ein anderer
Priester, ein Schotte, aber der Priester, welcher sich der Unter-
drückung schuldig gemacht hatte, nicht. Philipp Scafe, der be
kehrte Priester, war bei mir und es erschienen viele Leute. Als
wir uns gesetzt hatten, erklärte ein Ranter, namens T. Bushel,


% \picinclude{./040-049/p_s048.jpg} 
48 Kapitel 17.
er habe ein Gesicht von mir gehabt; ich sei an einem großen Pult
gesessen und er habe kommen müssen und seinen Hut vor mir
abnehmen und sich tief vor mir verbeugen, und er habe eß getan;
und noch viele andere Schmeicheleien sagte er mir. Jch sagte
zu ihm, er habe daß nur erfunden und er solle zu sich selber
sagen: ,,Schäme dich, du Hund«. Er sagte, eß sei nur Neid von
mir, so zu sagen. Darauf fragte ich ihn, waß der Neid eigentlich
sei rmd wie er im Menschen entstehe und waß daß Htindische sei
und wie eß im Menschen entstehe. Denn ich sah genau, daß er
etwaß Hündisrheß hatte, und darum wollte ich von ihm wissen,
wie dieseß Hündische in ihm entstanden sei. ,,Denn«, sagte ich
ihm, ,,mir müssen zuerst von dem reden, maß in unserm Leib
geschieht, ehe wir von dem reden können, roaß außer dem Leibe
ist.« Damit stopfte ich ihm daß Maul und allen seinen Runter-
genossen, denn er war ihr Haupt. Dann ries ich den Priester,
welcher die Leute unterdrückte, aber er kam nicht; nur der schottische
Priester erschien, der mit wenig Worten zum Schweigen gebracht
war; denn eß war innerlich kein Leben in ihm von dem, waß er-
bekannte. Nun war die Gelegenheit da, mit den Leuten zu reden.
Jch zeigte klar, wie die Ranter waren und verglich sie mit den
Prahlern in Sodom. Jch zeigte, wie ihre Priester die gleiche
Sorte von Mietlingen seien, wie die falschen Propheten früherer
Zeiten, und wie die Priester damals daß Volk auch in dieser
Weise regierten, indem sie ihren Gewinn im Auge hatten und
um Geld ihr Amt besorgten und um schnöden Gewinnß willen
lehrten. Jch stellte Christuß und die wahren Propheten und die
Apostel den Priestern gegenüber und zeigte, wie Ehristuß, die
Propheten und die Apostel sie schon lange an ihren Früchten
erkannt hätten. Dann wieß ich sie aus den Lehrer in ihrem
Jnnern hin, Jesus Christuß, ihren Heiland. Und ich predigte
Christuß in den Herzen, nachdem ich alle diese Höhen geebnet
hatte. Die Leute waren alle ruhig und die Widersacher zum
Schweigen gebracht. Denn obgleich eß innerlich in ihnen kochte,
so hielt die Kraft sie doch gebunden, so daß sie nicht loßbrechen
konnten .....
Ein anderer Priester ließ mich holen, um mit mir zu reden,
und etliche ,,Freunde« gingen mit mir nach seinem Hauö. Alß
er hörte, daß wir gekommen seien, entwischte er auß dem Hause
und versteckte sich unter einer Hecke. Die Leute gingen, ihn zu


% \picinclude{./040-049/p_s049.jpg} 
Erlebnisse im Gefängnis zu Derby usw. 49
zu suchen und fanden ihn, aber sie brachten ihn nicht dazu, zu
uns zu kommen. Daraus ging ich in ein nahegelegenes Turm-
haus, wo der Priester und das Volk in großer Erregung waren,
denn eben dieser Priester hatte den Freunden mit allem Ntöglichen,
das er tun werde, gedroht; als ich aber kam, machte er sich davon,
denn die Kraft des Herrn kam über ihn und über die andern.
Ja, des Herrn ewige Kraft kam über die Erde und drang zu
den Herzen der Menschen und machte die Priester und die
,,Frommen« zittern. Sie machte die Geister der Erde und der
Lust erbeben, zu welchen sie Vorgaben zu beten, sodaß sie einen
Schreck bekamen, wenn es hieß: »Der Mann in den ledernen
Kleidern kommi!«1) An vielen Orten machten sich die Priester,
wenn sie das hörten, davon, so waren sie von Furcht vor der
ewigen Kraft Gottes ergriffen .....
Von hier gingen wir über Scarbvrough .... nach Malton .....
Am Ersten Tag kam eine Frau, eine der angesehensten ,,Frvmmen«
unter den Jndependenten, welche ein solches Vorurteil gegen mich
hatte, daß sie sagte, ehe sie kam, sie würde sich freuen, mich er-
hängt zu sehen; aber als sie kam, wurde sie gewonnen und ge-
hört seither zu den »Freunden«.
Daraus hatte ich hier große Versammlungen; es hätten noch
mehr Leute daran teil genommen, aber sie wagten es nicht, aus
Furcht vor ihren Angehörigen. Es wurde damals als etwas
Unerhörtes angesehen, daß man in Häusern predigte statt in der
,,Kirche«, wie sie es nannten; darum wurde sehr gewünscht, daß
ich ins Turmhaus gehe und ddrt rede. Einer der Priester schrieb
mir und lud mich ein, im Turmhaus zu predigen, und nannte
mich seinen Bruder. Ein anderer Priester, eine bekannte Persön-
lichkeit, hielt dort eine Stunde. Nun hatte mir der Herr während
meiner Gefangenschaft in Derby kund getan, ich solle in den
Turmhäusern predigen, um die Leute von denselben abzubringen,
und es kamen mir auch zuweilen Bedenken wegen der Kanzeln,
in denen die Priester herumsaulenzten. Die Turmhäuser und
Kanzeln verletzten mein Gefühl, weil sowohl die Priester als auch
das Volk sie Gotteshäuser nannten und im Wahne waren, daß
Gott da in äußern sichtbaren Häusern wohne, statt im Gegenteil
1) Fox trug immer Kleider aus Leder, die et wegen ihrer Einfachheit und
Dauerhaftigkeit allen andern Kleidungsstiicken vorzog. (Vgl. Carlyles, Surtor
Resartus: Ein Ereignis in der neuen Geschichte.)
George Fox. 4



% \picinclude{./050-059/p_s050.jpg} 
Verlangen zu tragen, daß Gott und Christus in ihren Herzen
und Leibern wohne, aus daß sie Tempel Gottes? würden. Denn
der Wostel sagt: ,,Gott wohnet nicht in Tempeln mit Händen
gemacht'' (Act. 7, 48). Weil man aber diese Stätten nun einmal
heilig hielt, so fand man ez- schrecklich, wenn man etwas dagegen
sagte. A18 ich inZ Turmhauß kam, waren nicht mehr altz 11 Zu-
hörer dort, und der Priester hielt ihnen die Predigt. Alß nun
in der Stadt bekannt wurde, ich sei im Turmhause, so füllte sich
daßselbe bald mit Menschen. Alk- der Priester, der an dem Tage
zu predigen hatte, geendet hatte, hieß er den andern Priester, der
mich aufgefordert hatte zu kommen, mich auf die Kanzel führen,
aber ich ließ ihm sagen, ich brauche nicht auf eine Kanzel zu
steigen. Darauf ließ, er mir wieder sagen, er wünsche aber, daß
ich sie befteige, weil dort ein besserer Platz sei, an dem mich die
Leute sehen könnten. Ich ließ ihm darauf sagen, man sehe mich
gut genug, da wo ich sei, ich sei nicht gekommen, solche Stätten
noch aufrecht zu erhalten und ihr Bestehen und den Handel, der
damit getrieben wird. Alk- ich dieö gesagt hatte, fingen sie an,
böse zu werden und sagten: ,,Da8 sind die falschen Propheten
der letzten Zeiten«. Diese Rede Verletzte etliche und sie murrten
darüber; nun stand ich auf und hieß alle ruhig sein; ich stieg
auf einen hohen Stuhl und erklärte ihnen, woran man die falschen
Propheten erkenne, und daß sie schon gekommen seien; und dann
zeigte ich ihnen im Gegensatz dazu die wahren Propheten, Christus-
und die Apostel. Ich wieß sie alle an ihren inneren Lehrer,
Christue, der sie von der Finsterniö zum Lichte führen könne.
Nachdem ich ihnen verschiedene Schriftstellen erklärt hatte, wies
ich sie auf den Geist Gottes in ihren Herzen hin, durch welchen
sie zu ihm kommen könnten und erkennen, wer die falschen
Propheten seien. Nachdem ich so ein reiches Wirken unter ihnen
gehabt hatte, zog ich im Frieden von dannen ....
Hierauf kam ich nach Pickering, wo die Richter im Turm-
hauö ihre Sitzungen hielten; Ftiedenztichtet Robinson war Vor-
sitzender. Ich hatte zur gleichen Zeit eine Versammlung im
Schulhaus und viele »Fromme« und Priester wohnten ihr bei
und stellten allerlei Fragen, die zu ihrer Zufriedenheit beantwortet
wurden. ES war gerade die Zeit der Gerichtösit-zungen, und da
wurden auch vier Oberkonstabler bekehrt. GS kam Richter Robin-
son zu Ohren, daß der Priester, den er allen andern Priestern


% \picinclude{./050-059/p_s051.jpg} 
Erlebnisse im Gefängnis zu Terby usw. 51
oorzog, besiegt und überzeugt worden war. Wir gingen nach
der Versammlung in eine Herberge; Richter Robinson’s Priester
war sehr bescheiden und lieb und wollte sogar durchaus mein
Essen bezahlen, was ich aber nicht zuließ. Dann bot er mir sein
Turmhaus an, um darin zu predigen, aber ich lehnte es ab und
erklärte ihm und den andern, daß ich eben gekommen sei, um die
Leute oon diesen Dingen ab und zu Christus zu bringen.
Am folgenden Morgen ging ich mit den vier Konstablern und
andern, um Richter Robinson zu besuchen, der mir unter der
Türe seines Zimmers entgegenkam. Jch sagte ihm, ich könne ihm
keine menschliche Ehre erweisen; er sagte, er sehe nicht aus das.
Jch ging nun mit ihm ins Zimmer und tat ihm den Unterschied
zwischenswahren und falschen Propheten dar, und wie die wahren
höher stehen als die falschen, und richtete seinen Sinn aus
Christum seinen Lehrer. Jch deutete ihm die Gleichnisse, und
wie es sich mit der Grwählung und Verwersung verhalte, wie
man in der ersten Geburt in der Verwersung sei und in der
zweiten in der Grwählung. Ich zeigte ihm, wer die Verheiß-ungen
Gottes habe und wen sein Gericht verdamme. Gr gab alles zu
und war so offen für die Wahrheit, daß, wenn ein anderer an-«
wesender Richter eine kleine Ginwendung machen wollte, er ihn
belehrte. Beim Fortgehen sagte er, ich tue sehr gut, diese mir
von Gott verliehene Gabe zu gebrauchen. Gr nahm den obersten
Konstabler beiseite und wollte ihm etwas Geld siir mich geben,
weil er nicht wollte, daß ich in ihrer Gegend irgend welche Aus-
gaben habe; aber sie sagten ihm, daß ich nicht dazu zu bringen
sei, etwas anzunehmen. Jch schätzte seine Freundlichkeit, das
Geld jedoch lehnte ich ab.
Jch zog im Lande umher und der Priester, der mich Bruder
genannt hatte, zog mit mir. Als wir in eine Stadt kamen, wo
wir im Sinne hatten etwas zu essen, läuteten die Glocken.
Jch fragte, warum sie läuten; man sagte mtr, sie läuten für mich,
damit ich im Turmhaus predige. Bald daraus trieb es mich
dorthin. Als ich kam, sah ich die Leute auf dem Turmhausplatze
versammelt; der alte Priester wollte, daß ich ins Turmhaus gehe,
  ich sagte, es sei nicht nötig. Ge besremdete die Leute, daß
ich nicht in das gehen wollte, das sie ,,Goiteshaus« nannten. Jch
stellte mich auf den Platz des Turmhauses und erklärte den Leuten,
ich sei nicht. gekommen, ihre göizendienerischen Tempel ausrecht
 


% \picinclude{./050-059/p_s052.jpg}
zu erhalten, noch die Priester mit ihren Zehnten, Zulagen, Ab-
gaben und Pfrtinden, noch ihre jüdischen und heidnischen Zere-
monien und Traditionen; denn die gelten mir alle nicht?-. Jch er-
klärte ihnen, dieses Stück Boden sei nicht heiliger, ale irgend ein
anderes Stück Land. Jch zeigte ihnen, daß die Apostel, wenn
sie in die Synagogen und die Tempel der Juden gegangen seien,
die ja Gott selber sogar vorgeschrieben habe, so sei ez nur ge-
schehen, um die Leute davon Iabzubringen und von den Opfern
und Zehnten und den habsüchtigen Ptiestem jener Zeit. Und
die, welche zur Wahrheit belehrt wurden und an den von den
Aposteln gepredigten C-hristuö Hglaubten, hätten sich nachher in
den Wohnhäusern versammelt. Ich sagte ihnen, daß alle, welche
Christus, daß Wort deö Lebenß, predigen, eö umsonst tun sollen
wie die Apostel, und wie Christus eß geboten habe. So war ich
gesandt worden von Gott dem Herrn Himmels und der Erden
umsonst zu predigen und die Leute von diesen äußeren Tempeln
mit Händen gemacht, worin Gott nicht wohnt, abzubringen, damit
sie erkennen, daß ihre Leiber Tempel Gottes werden sollen. Jch
mußte die Leute abbringen von ihren jüdischen Zeremonien, aber-
gläubischen und heidnis chen Gebräuchen, Traditionen und Menschen-
satzruigen, von der Lehre all der Mietlinge, die Zehnten nehmen
und große Psründen, die um Bestechung predigen und für Geld
weiösagen, die gar nicht von Gott und von Christus gesandt
sind, wie sie ja selber bekennen, wenn sie sagen, sie haben nie
die Stimme Gotteß noch Christi vernommen. So ermahnte ich
denn die—Leute, abzulassen von alle dem, und wieß sie auf den
Geist und die Gnade Gottetz hin, welche inwendig in ihnen sind,
und auf daß Licht Jesu in ihren Herzen, aus daß sie dazu kommen
möchten, C-hristum zu kennen, der sie umsonst lehre und ihnen
Rettung bringe und ihnen die Schrift öffne. Alleö war ruhig
und viele wurden gewonnen, der Herr sei gepriesen.
Jch kam daraus in eine andere Stadt, wo wieder eine große
Versammlung war; der vorhin erwähnte Priester begleitete mich
und allerlei ,,Fromme« kamen dazu herbei. Ich saß mehrere
Stunden auf einem Heuschober und sagte nichte, denn sie sollten
nach Worten hungern. Die ,,Frommen« kamen immer wieder
zu dem alten Priester und fragten ihn, wann ich beginnen werde
zu reden. Er hieß sie warten und sagte ihnen, das Volk habe
immer lange gewartet, bis Christus gesprochen habe. Schließlich


% \picinclude{./050-059/p_s053.jpg} 
Erlebnisse im Gefängnis zu Derby usw. 53
trieb mich der Herr zu reden, und sie wurden von der Kraft deö
Herrn erfaßt; das Wort detz Lebens erreichte sie und es- geschah
eine allgemeine Bekehrung unter ihnen.
Ich zog weiter; der alte Priester und einige andere waren
mit mir. Unterwegß riesen ihn ein paar Leute an: ,,Mr. Bones,
wir sind euch Geld schuldig für Zehnten; kommt doch und nehmt
eZ!« Aber er wehrte mit der Hand ab und sagte, er habe genug,
er wolle nichtß davon, sie sollten ez nur behalten; und er prieß
Gott, daß er solcheö sagen konnte. Schließlich kamen wir zu dem
Turmhauß dieses- alten Priesters im Nioor; alß wir eingetreten
waren, ging er vorausz und öffnete die Kanzeltür, aber ich sagte ihm,
ich würde nicht hineingehen. Das Turmhauß war stark bemalt;
ich sagte ihm und den Leuten, die dabei waren, daß gemalte Tier
(Offb. 17, 3.) habe ein gemalteß Haus. Dann erklärte ich ihnen die
Entstehung aller dieser Häuser und ihre abergläubischen Gebräuche;
ich zeigte ihnen, daß die Apostel nicht in die Tempel gegangen
seien, um diese aufrecht zu erhalten, sondern um die Leute zu
Christuö, dem wahren Gut, zu führen; ich zeigte ihnen den wahren
Gotte?-dienst, den Christus gegründet hat; ich zeigte den Unter-
schied zwischen Ehristuß dem wahren Weg und allen verkehrten
Wegen, indem ich ihnen die Gleichnisse deutete und sie von der
Finsternitz zum wahren Lichte wieö; damit sie durch dasselbe sich
selbst erkennen möchten und ihre Sünden und ihren Erlöser und
durch den Glauben an ihn erlöst würden von ihren Sünden .....
Nun kam ich nach Eranstick, zu Hauptmann Purßloe und
Friedenßrichter Hotham, die mich beide freundlich empsingen, weil
sie sich freuten, daß die Kraft des Herrn erschienen war und daß
die Wahrheit sich auzbreitete und so viele sie aufnahmen, und daß
Frichen?-richter Robinson so freundlich gewesen war. Hotham
sagte, wenn Gott nicht diese Anschauungen von Licht und Leben
hätte kund werden lassen, so wäre daß ganze Land von den
Rantern überschwennnt worden und alle Richter des- Landeß mit
allen ihren Gesetzen hätten ihnen nicht zu wehren vermocht. ,,Denn«,
sagte er, ,,wenn sie auch gesagt und getan hätten, was- wir ihnen
befehlen, so hätten sie doch nicht von ihren Wisichten gelassen.
Aber eure Grundsätze der Wahrheit werfen alle ihre Grundsätze
und daß, worauf sie die ihrigen gründen, über den Hausen«.
Darum war er so froh, daß Gott diese Grundsätze des Leben-3
und der Wahrheit hatte durch mich kund werden lassen ....


% \picinclude{./050-059/p_s054.jpg} 
Als am folgenden Tage die Freunde mich verlassen hatten,
reiste ich allein weiter und verkündete den Tag des Herrn überall,
wohin ich kam, und ermahnte zur Buße. Eines Abends kam ich
in die Stadt Patrington, und während ich durch die Stadt ging,
ermahnte ich sowohl die Priester als das Volk Buße zu tun und
sich zum Herrn zu bekehren. Gs wurde finster, ehe ich ans Ende
der Stadt kam, und eine große Menge hatte sich um mich ver-
sammelt, während ich das Wort des Lebens verkündete. -- Als
ich meine Ausgabe erfüllt hatte, ging ich in eine Herberge und
verlangte Unterkunft für die Nacht, aber sie wurde mir verweigert.
Daraus bat ich um etwas Fleisch und Milch, ich wolle es bezahlen;
aber auch das wollte man mir nicht geben. So verließ ich die
Stadt; einige junge Leute kamen hinter mir drein und fragten
mich, was es neues gebe. Jch hieß sie Buße tun und Gott
fürchten. Als ich eine Strecke weiter gegangen war, kam ich wieder
an ein Haus und bat, man solle mir etwas Fleisch und Milch
geben und Nachtherberge, gegen Bezahlung; aber sie schlugen es
mir ab; dann ging ich zu einem andern Haus und verlangte das-
selbe; aber sie wiesen mich ebenfalls ab. Jnzwischen war es so
dunkel geworden, daß ich die Landstraße nicht mehr sehen konnte;
ich endeckte einen Wassergraben und schöpfte etwas Wasser um
mich zu erfrischen; dann überschritt ich den Graben und da ich
von der Reise müde war, setzte ich mich unter einen Ginsterstrauch
und wartete bis es Tag war. Mit Tagesanbruch erhob ich mich
und ging weiter. Hinter mir drein kam ein Mann mit einer
Heugabel, der schritt neben mir her bis zu einer Stadt, und
noch ehe die Sonne ausgegangen war, hatte er diese Stadt und
die Polizei gegen mich ausgehetzt; ich verkündete Gottes ewige
Wahrheit unter ihnen und warnte sie vor dem Tag des Herrn,
der kommen würde über alle Sünde und Ungerechtigkeit, und er-
mahnte sie, Buße zu tun. Mer sie.griffen mich und brachten
mich nach Patrington zurück, etwa drei Meilen weit, und be-
wachten mich mit Stöcken, Heugabeln und Hellebarden. Als ich
nach Patrington kam, war die ganze Stadt in Aufruhr. Die
Priester und das Volk berieten sich zusammen; so konnte ich
ihnen abermals das Wort des Lebens verkünden und sie zur Buße
ermahnen. Endlich nahm mich einer der »Frommen«, ein guter
Mann, mit in sein Haus, wo ich mich an etwas Brot und Milch
erlabte, denn ich hatte seit mehreren Tagen nicht-3 gegessen. Dann


% \picinclude{./050-059/p_s055.jpg} 
Erlebnisse ini Gefängnis zu Derby usw. 55
schleppten sie mich etwa neun Meilen weit zu einem Richter.
Als wir nahe bei dessen Haus waren, kam einer hinter uns her
geritten und fragte mich, ob ich der sei, der verhaftet worden
war. Ich fragte, warum er es wissen wolle; er sagte, es ge-
schehe in keiner bösen Absicht; da sagte ich ihm, daß ich es
sei; darauf ritt er voraus zum Richter. Meine Begleiter sagten,
hoffentlich sei der Richter nicht betrunken, wenn wir zu ihm
kämen; denn er pflegte schon frtihmorgens betrunken zu sein.
Als ich vor ihn trat und meinen Hut nicht abnahm und ihn mit
Du anredete, fragte er den, welcher uns oorgeritten war, ob ich
verrückt sei, aber er sagte ihm, nein, es sei mein Grundsatz. Jch
ermahnte den Richter, Buße zu tun und sich zum Licht zu be-
kehren, mit dem Christus ihn erleuchtet, damit er durch dasselbe
alle seine bösen Worte und Taten erkennen möge, und zu Christus
zurückzukehren, solange es noch Zeit sei. ,,Ja, ja«, sagte er, ,,das
Licht von dem im dritten Kapitel des Johannes gesprochen wird-«.
Joh bat ihn, er möge doch auf dieses Licht achten und ihm ge-
horchen. Während ich ihn ermahnte, legte ich ihm die Hand auf,
und er ward übernommen von der Kraft des Herm und die
Wächter waren bestürzt. Er führte mich nun in ein kleines Gemach,
um zu untersuchen, was ich von Briefen und Schriften in der
Tasche habe; ich wies ihm meine Kleider und zeigte ihm, daß
ich keine Briefe bei mir hatte; er sagte, man sehe an meiner
Wäsche, daß ich kein Landstreicher sei, und ließ mich frei. Jch
ging mit dem Mann, der vor uns hergeriiten, nach Patrington
zurück, denn er lebte daselbst. Als wir ankamen, wünschte er, ich
solle eine Versammlung auf dem Hauptplatz halten, aber ich sagte
es sei nicht nötig, sein Haus genüge. Gr wollte, daß ich zu Bett
gehe oder mich doch aufs Bett lege; dies wünschte er namentlich,
damit er sagen könne, man habe mich in oder doch wenigstens
auf einem Bett gesehen; denn es ging das Gerücht, ich wolle
in keinem Bett schlafen, weil ich damals oft im Freien über-
nachtete. Als der Erste Tag kam, ging ich ins Turmhans
und verkündete dem Priester und dem Volk die Wahrheit; und
die Leute taten mir nichts, denn die Kraft Gottes war über sie
gekommen. Gleich nachher hatte ich eine große Versammlung in
dem Hause des Mannes, der mich beherbergte, und viele wurden
von Gottes ewiger Wahrheit überzeugt und sind derselben treu
geblieben bis aus den heutigen Tag. Sie bereuten es sehr, daß


% \picinclude{./050-059/p_s056.jpg} 
sie mich nicht aufgenommen und beherbergt hatten, als ich zuerst
bei ihnen gewesen war .......


%%%%%%%%%%%%%%%%%%% Kapitel 5. %%%%%%%%%%%%%%%%%%%%%%%%%%%%%%

\chapter[Quäkerischen Weltmission]{Quäkerischen Weltmission}

\begin{center}
\textbf{Christus in uns. Erkenntnis der Quäkerischen Weltmission. Das
Haus Richter Fells in Swarthmore; der Pöbel von Ulverstone.
Rechtfertigung vor dem Gericht in Lancastre.}
\end{center}



\begin{floatingfigure}[3]{4cm}
\includegraphics[width=0.20\textwidth]{./pics/swarthmore_hall.png}
\label{bild:swarthmoor} 
\end{floatingfigure}



Wir zogen durch Nottinghamshire nach Lineolnshire .....
Hier kam zu einer unserer Versammlungen ein Mann und erhob eine
falsche Anklage gegen mich; er verbreitete überall daö Gerücht, ich
habe gesagt, ich sei Christus-, was gänzlich falsch war. Al-? ich dann
nach Gainßborough kam, wo einer der Freunde auf dem Markt-
platz die Wahrheit verkündet hatte, fand ich die ganze Stadt und
alle Marktleute in Aufruhr. Jch ging inß Haus eineö Freunde?-,
und daß Volk drängte sich hinter mir drein, biz das Haus ganz
voll war von »Frommen«, Giferern und Pöbel; da kam jener
falsche Verleumder herein und klagte mich öffentlich vor allen an,
ich hätte gesagt, ich sei Christuß, und er habe Zeugen, es zu be-
weisen. Das brachte die Leute so in Wut, daß man Miihe hatte,
mich vor ihnen zu schützen. Da trieb mich der Geist dez Herrn
aus einen Tisch zu stehen und in der ewigen Kraft dez Herrn
den Leuten zu verkünden, daß Christuö in ihnen sei, etz sei denn,
daß sie Verdammte seien; und daß eß Christuö, die ewige Kraft
Gottes sei, welche jetzt auß mir zu ihnen rede, nicht ich sei Christus;
die Leute waren im allgemeinen befriedigt außer jenem »Frommen«
und einigen falschen Zeugen. Jch nannte diesen Ankläger Judaö,
und es trieb mich, ihm zu sagen, daß das Ende deö Judaß auch
das seine sein werde; solches- lasse ihm der Herr durch mich sagen.
Dez Herrn Macht kam über alle und beruhigte die Gemüter der
Leute und sie gingen in Frieden fort. Jener Judaß aber machte
sich davon und erhenkte sich und man steckte einen Pfahl in sein
Grab. Daraufhin erhoben die bösen Priester eine Verleumdung
gegen unö und streuten aus, ein Quäker habe sich erhenkt in
Lineolnshire. Diese Lüge ließen sie drucken und verbreiten und
hänften so Sünde auf Sünde. Mer wir und die Wahrheit wurden
nicht davon getroffen; denn jener war so wenig ein Quäker als
der Priester, der solcheß gedruckt hatte; vielmehr war eögeiner


% \picinclude{./050-059/p_s057.jpg} 
Christus in uns. Erkenntniz der Quükerischen Weltmisfion usw. 57
ihrer eigenen Leute. Aber trotz dieser argen Lüge, mit welcher der
Gegner beabsichtigthatte, unß zu verleumden und die Leute von
der von unß verkiindeten Wahrheit abzukehren, nahmen doch viele
in Lineolnshire daß Evangelium an, da sie von der ewigen Wahr-
heit überzeugt waren und sich zu Füßen deß himmlischen Herm
setzten ......
Wir zogen nun wieder .... über Warmßworth . . . Bably,
Doneaster .... nach Tickhill, wo an einem Ersten Tage die
Freunde der Gegend sich versammelten, und eß herrschte durch
Gottes Macht eine tiefe Zerknirschung in der Versammlung.
Jch verließ die Versammlung, da Gott mich trieb inß Turmhauß
zu gehen. Alß ich dorthin kam, fand ich den Priester und saft
alle Gemeindeältesten im Chor beisammen. Jch ging zu ihnen
und hub an zu ihnen zu reden, aber sie sielen sogleich über mich
her, und ein Priester nahm seine Bibel und schlug mich damit
inß Gesicht, so daß ich heftig blutete im Turmhauß; daß Volk
schrie: ,,Hinauß mit ihm auß der Kirche!« Und alß sie mich hinauß
gebracht hatten, prügelten sie mich und warfen mich zu Boden
und über eine Hecke; hernach schleppten sie mich durch ein Hauß
aus die Straße; sie warfen mich mit Steinen und schlugen mich,
während sie mich Vorwärtß tissect, so daß ich über und über mit
Kot beschmiert war. Sie nahmen mir den Hut, den ich nicht
mehr wieder bekam. Alß ich jedoch wieder auf den Füßen war,
verkündete ich ihnen daß Wort deß Lebenß und zeigte ihnen, wo-
hin ihre Lehre sie führe und wie sie daß Christentum entehrten. Nach
einer Weile ging ich wieder in die Versammlung zurück zu den
Freunden. Und alß die Priester und die Leute am Hause vorbei
kamen, ging ich mit einigen Freunden hinauß in den Hof und
redete zum Priester rmd den Leuten. Der Priester verhöhnte
unß und nannte unß ,,Quäker«. Aber die Macht deß Herrn
kam dermaßen über sie und daß Wort deß Lebenß wurde ihnen
so überzeugend und eindringlich verkündet, daß der Priester selber
zu zittern begann und einer sagte: »seht wie der Priester zittert
und bebt, er wird auch ein Quäker«. Alß die Versammlung zu
Ende war, gingen die Freunde sort, und ich ging, ohne Hut,
nach Balby, etwa sieben bis acht Meilen weit. Die Freunde
wurden an dem Tage dergestalt von dem Priester und seinen
Anhängern mißhandelt, daß einige Friedenßrichter, alß sie davon
hörten, kamen und ein Verhör in dieser Stadt anstellten, um


% \picinclude{./050-059/p_s058.jpg} 
58 Kapitel lt.
die Sache zu untersuchen. Der, welcher mich blutig geschlagen
hatte, fürchtete, man haue ihm die Hand ab; aber ich vergab
ihm und klagte nicht gegen ihn.
Zu Anfang deö Jahres- 1652 regte sich heftiger Widerstand
gegen die Wahrheit und die Freunde, bei Priestern und Volk
und bei etlichen der Behörden in Yorkshire, so daß der Priester
von Warmöworth sich einen Verhaftbefehl gegen mich und Thomaß
Aldam verschaffte, der in allen Teilen im westlichen Bezirk York-
shireö auögeführt werden konnte. Zu dieser Zeit hatte ich ein
Gesicht von einem Bären und zwei großen, riesigen Hunden, und
wie ich bei ihnen vorbei mußte, ohne daß sie mir ein-aß tun
konnten. Und so geschah es; denn der Konstabler ergriff Thomaß
Aldam und brachte ihn nach York; und ich ging ein großeö Stück
Wegß mit ihm. Der Kanstabler hatte auch einen Verhaftbesehl
gegen mich und sagte zu mir: er sehe mich schon, aber er möge
nicht einen der ihm fremd sei, behelligen; Thomaß Aldam sei
eben sein Nachbar. Also hielt ihn die Kraft des Herrn, daß er
mich in Ruhe ließ. Wir kamen in die Wohnung deö Leutnant
Roper, wo wir eine große Versammlung hatten, worunter viele
angesehene Leute waren; die Wahrheit wurde mächtig kund
unter ihnen und die Schrift herrlich erklärt, und die Gleichnisse
und Reden Jesu wurden außgelegt und die Kirche, wie sie in den
Tagen der Apostel war, und der Abfall von derselben. Die
Wahrheit gelangte zur Herrschaft an jenem Tage, so daß jene
angesehenen Leute alle zugestanden: ,,diese Anschauungen werden
sich über die ganze Erde aus-breiten«. Dieser Versammlung
wohnten auch Jametz Naylor, Thomas Goodyear und William
Dewßbury, die das Jahr vorher gewonnen worden waren, sowie
Richard Farne-worth bei. Der Konstabler blieb mit Thomaß
Aldam, bis die Versammlung auö war, darm ging er mit ihm
nach dem Gefängnis in York; mich aber ließ er in Ruhe ....
Darnach kam ich nach Hightown, wo eine Fran wohnte, die
kurz vorher bekehrt worden war. Wir gingen in ihr Haus und.
hielten eine Versammlung, und die Leute versammelten sich, und
wir verkiindeten ihnen die Wahrheit und wirkten für den Herrn
unter ihnen, und sie gingen in Frieden wieder von dannen.
Aber ez war dort eine Witwe, namens Green, von böser Ge-
sinnung; diese ging zu einem sogenannten ,,Herrn« (cieutleman)
und verklagte unö bei ihm, obwohl er kein Beamter war. Am


% \picinclude{./050-059/p_s059.jpg} 
Christus in uns. Erkenntnis der Quäkerischen Weltmission usw. 59
nächsten Morgen sandten wir dem Priester einige Fragen. Alö
wir gerade fort gehen wollten, kamen einige, die sich zu une
hielten, gerannt und sagten, dieser Mörder habe sein Schwert
für unß geschärst und komme mit demselben gegen unß. Da wir
gerade fort gingen, oerfehlten wir ihn. Aber kaum waren wir
sort, so kam er in das Haus, in dem wir gewesen waren, und
es hieß allgemein, wenn wir nicht fort gewesen wären, so wären
wir ermordet worden. Wir brachten die Nacht im Walde zu
und wurden ganz durchnäßt, denn es regnete stark. Am Morgen
trieb es- mich roicher in die Stadt zurück, wo sie unß außfiihrlich
über jenen Bösewicht berichteten.
Von da gingen wir nach Vradsord, wo wir Richard Farnß-
worth trafen, von dem wir unö kurz vorher getrennt hatten.
A18 wir in sein Haus kamen, setzte man unß Fleisch nor, aber alß
. ich anfangen wollte, geschah daß Wort deö Herrn an mich: »Jß nicht
Brot bei einem Neidischen« (Spr. 23, 6). Sogleich stand ich
vom Tische auf und aß nicht;3. Die Frau war eine Baptistin.
Nachdem ich die ganze Familie ermahnt hatte, sich zum Herrn
zu bekehren und auf seine Lehre in ihren Herzen zu merken,
gingen wir von dannen ......
Unterwegs; kamen wir zu einem großen Hügel, genannt Pend-
lehill; und der Herr trieb mich, aus denselben hinauf zu gehen,
maß ich mit großer Anstrengung tat, denn er war sehr steil und
hoch. Alö ich oben ankam, blickte ich auf daß Meer, das Lan-
cashire umspült. Von diesem Hügel aut:-’ zeigte mir der Herr
die Orte, wo ihm ein großetz Volk sollte gesammelt werden.
Beim Hinuntergehen sand ich eine Wasserquelle am Abhang
dez Hiigelß, auß der ich mich ersrischte, denn ich hatte in den
letzten Tagen nur wenig gegessen und getrunken. Am Abend
kamen wir zu einer Herberge .... und hier ließ mich der
Herr ein Gesicht sehen: eine große Schar in weißen Kleidern
am Ufer eines Flusses, die zum Herrn kamen, und der Ort, den
ich sah, war bei Wen?-leydale und Sedbergh. . .
Wir zogen durch die Daleö . . . nach Dent .... Hier ging
ich zu Richard Robinson und redete von der Wahrheit zu ihm:
Ju einer Versammlung bei Frieden?-richter Benson traf ich Leute,
die sich vom öffentlichen Gotteödienst lozgesagt hatten. Dies
war der Ort, den ich gesehen, wo eine Schar in weißen Kleidern
daher kam. GS war eine große Versammlung, und die meisten



% \picinclude{./060-069/p_s060.jpg} 
wurden gewonnen und haben noch jetzt große Versammlungen von
Freunden in der Nähe von Sedbergh, die ich damals zuerst zu-
sammen sammelte im Namen Jesu.
GS fand ein großer Jahrmarkt statt, an welchem man pflegte
Dienstboten zu dingen; ich verkündete den Tag dee Herrn. Nach-
dem ich dietz getan, ging ich auf den Platz des Turmhausetz, und
viele Leute kamen vom Jahrmarkt zu mir und eine Menge Priester
und »Fwmme«. Da verkündete ich die ewige Wahrheit dez
Herrn und das Wort dez Lebenö während mehrerer Stunden
und zeigte, daß der Herr gekommen sei, sein Volk selbst zu lehren
und es abzubringen von den Wegen dieser Welt und ihren Lehrern,
zu Ehristuö dem wahren Lehrer und wahren Weg. Jch machte
ihnen klar, wie ihre Lehrer denen gleich seien, die von jeher
von den Propheten, von Ehristuß und den Aposteln verdammt.
worden sind. Jch ermahnte alle von ihren mit Händen gemachten
Tempeln abzulassen und auf den Empfang dez Geistes zu warten,
damit sie erkennen könnten, daß sie der Tempel Gottes seien.
Nicht ein einziger von den Priestern hatte Macht, seinen Mund
auszutun gegen daß, waß ich verkündete; zuletzt sagte einer von
der Wache: ,,Warum geht ihr nicht in die ,,Kirche«? hier ist
kein geeigneter Platz zum Predigen«. Jch sagte ihm, ich leugne
ihre Kirche. Da erhob sich Francis Howgill,1) Prediger einer
Gemeinschaft. Er hatte mich nie vorher gesehen, aber er unternahm
es, diesem Hauptmann zu antworten und brachte ihn bald zum
Schweigen; und von mir sagte er: ,,dieser predigt gewaltig und
nicht wie die Schriftgelehrten (Matth. 7, 29)«. Jch erklärte da-
rauf den Leuten, daß dieser Boden hier nicht heiliger sei alö an
einem andern Ort und daß nicht dieses Haus die Kirche sei,
sondern die Gemeinde, deren Haupt Christutz ist. Bald nachher
kamen dann einige Priester zu mir und ich ermahnte sie, Buße zu
tun. Einer von ihnen sagte, ichsei verrückt, und wandte sich von
mir ab; aber manche wurden gewonnen an dem Tage und freuten
sich über die Verkündigung der Wahrheit und nahmen sie mit
Freuden auf. Einer unter ihnen, Hauptmann Ward, nahm die
Wahrheit in Liebe auf und lebte darin bis zu seinem Tode.. . .
Von da ging ich nach Unterbarrow, zu einem namenß Miles
Bateman ..... Am Morgen ging ich auß . . und als ich in
1) Franeiö Hotogill, später ein eisriger Qnäkerprediger (s.Wein-
garten a. a. O.)


% \picinclude{./060-069/p_s061.jpg} 
Christns in uns. Erkenntnis der Quäkerischen Weltmission usw. 61
der Nähe auf einem Hügel hin und her ging, sah ich einige
Reisende, welche um Unterstützung baten, und ich sah, daß sie
eß nötig hatten; aber man gab ihnen nichts und sagte ihnen, sie
seien—Strolche. GS betrübte mich solche Hartherzigkeit unter den
,,Frommen« zu sehen, und ale sie alle beim Frühstück saßen, lies
ich den Reisenden etwa eine Viertelmeile nach und gab ihnen
etwaß Geld. A16 nun einige von den andern aus dem Hause
kamen und sahen, daß ich eine Viertelmeile weg war, sagten sie, ich
hätte nicht so weit kommen können, wenn ich nicht Flügel hätte.
Daraufhin war es nahe daran, daß man die Versammlung ab-
sagte; denn man hatte eine so merkwürdige Meinung von mir
bekommen, daß viele nicht eine Versammlung mit mir haben
wollten. Jch sagte ihnen, ich sei jenen armen Reisenden nachge-
laufen, um ihnen etwaß Geld zu geben, weil mich die Hart-
herzigkeit, mit der man sie fortgeschickt, betrübt habe ....
Von da tzging ich nach Ulverstone und Swarthmore zu
Richter Fell; es kam auch einer, Priester Lampitt, der behauptete,
Eingebungen zu haben- Jch redete lange mit ihm, denn er sprach
von wichtigen Eingebungen und von Vollkommenheit und blen-
dete die Leute dadurch. Gr hätte mich gerne gewähren lassen,
aber ich konnte ihn nicht gewähren lassen, weil er so unlauter
war. Gr sagte, er sei mehr als Johanneß, und tat, alz ob er
alle Dinge wüßte. Jch sagte ihm, der Tod habe von Adam biz
Moses regiert (Röm. 5, 14); und weil er tot sei, kenne er Moses
nicht, denn Moseß habe daß Paradies- Gotteß gesehen; er aber
kenne weder Moseß noch die Propheten noch Johannes. Denn
die höckerichte und rauhe Natur war noch in ihm, und der Berg
der Sünde und deö Verderbens, und der Weg für den Herrn
war nicht bereitet in ihm (Jes. 40). Er bekannte, er sei in großer
Trübsal gewesen, beteuerte aber, nun könne er Psalmen fmgen
und alleö machen, maß; man von ihm verlange. Ich sagte ihm,
er gehöre zum Diebögesindel, aber Moses und die Propheten
und Christuö predigen, das- könne er nicht; dazu müßte er den
gleichen Geist haben wie jene. Margaret Fell war den ganzen
Tag nicht zu Hause gewesen; am Abend erzählten ihr ihre Kinder,
daß Priester Lampitt und ich gestritten hätten; die-J betrübte sie,
weil er dem gleichen Bekenntnis angehörte wie sie; aber er
oerbarg sein schmutzigeß Treiben vor ihnen.
Wir sprachen noch lange miteinander am Abend, und ich ver-


% \picinclude{./060-069/p_s062.jpg} 
kiindete ihr und ihrer Familie die Wahrheit. Am folgenden Tage
kam Lampitt wieder, und ich redete lange mit ihm, und Margaret
Fell, die ihn jetzt ganz durchschaute, war dabei. Eine Über-
zeugung der Wahrheit kam über sie und die Jhrigen. Als bald
darauf ein allgemeiner Bußtag abgehalten werden sollte, bat sie
mich, mit ihr ine Turmhaus von Ulverstone zu kommen, denn
sie hatte sich noch nicht gänzlich davon lo?-gemacht. Jch erwiderte
ihr: ,,Jch muß tun, wie mich der Herr heißt.« Jch verließ sie
und ging int? Freie und daß- Wort dee Herrn geschah also zu
mir: ,,Gehe ihnen nach inß Turmhau?-.« Als ich kam, sang
Lampitt gerade mit den Leuten; aber sein Geist war so unlanter,
und waß sie sangen, paßte so wenig für ihr Bedürfnis, daß, als-
sie fertig gesungen hatten, der Herr mich trieb, also zu ihnen zu
reden: ,,Der ist nicht ein Jude, der etz äußerlich ist, sondern der
ist ein Jude, der innerlich einer ist, in seinem Leben, daß er nicht
vor den Menschen, sondern vor Gott führt« (Röm. 2, 28.29).
Dann zeigte ich ihnen nach des Herrn weiterer Offenbarung, daß
Gott gekommen sei, sein Volk zu lehren (1.Joh.2, 27), (Joh.1-i,26).
durch seinen Geist und sie abzubringen von allen ihren früheren
Gebräuchen, ihren Bekenntnissen, Kirchen und Gotte?-diensten;
denn daß alleö seien nur Menschensatzungen; daß Leben und den
Geist, auz dem diese Satzungen entstanden, die hätten sie doch
nicht. Da rief Friedenörichter Sawrey: ,,Fort mit ihm!« Aber
Richter Fell?. Frau sagte zu den Beamten: ,,Laßt ihn gehen;
warum soll er nicht so gut reden wie ein anderer?« Auch Lampitt,
der Betrüger, sagte, man solle mich reden lassen. Aber alö ich eine
Zeitlang geredet hatte, ließ mich Friedenßrichter Sawrey hinaus
bringen durch die Konstabler; da redete ich auf dem Kirchhof
weiter . . . Jch ging nun nach Becliff . . . und andere Orte . . .
Bald darauf, alß Richter Fell nach Hause kam, ließ Margaret
Fell mich holen und ließ mir sagen, ich solle doch zu ihnen
kommen; ich fühlte die Freiheit vom Herrn, etz zu tun und ging
hin. Ich sah, daß die Priester und die ,,Frommen« und der
Frieden?-richter Sawrey, Richter Fell und Hauptmann Sande durch
ihre Lügen gegen die Wahrheit eingenommen hatten, aber alß ich
kam und mit ihnen redete, gelang ez mir, alle ihre Einwände zu
widerlegen, und ich überzeugte Hauptmann Sands an Hand der
Schrift so völlig, daß er ganz befestigt war in seiner Uberzeugung.
Nach einigem Hin- und Herreden war Richter Fell ebenfalltz zu-


% \picinclude{./060-069/p_s063.jpg} 
Chrisiuß in unß. Erkenntnis der Quitkerischen Weltmission usko. 63
frieden gestellt und gelangte dazu, durch daß, waß ihm der Geist
Gotteß eröffnet hatte, etwaß Höhereß zu erkennen, alß waß die
weltlichen Priester und Lehrer lehrten, und ging nicht mehr hin,
sie zu hören alle die Jahre biß zu seinem Tode; denn er wußte
nun, daß daß, waß ich lehrte, die Wahrheit sei, und daß Ehristuß
der Lehrer seineß Volkeß ist und sein Heiland . . . Während ich
in dieser Gegend war, kamen Richard Farnßworth und Jameß
Naylor, mich und die anderen zu sehen, und weil Richter Fell
nun darüber beruhigt war, daß eß die Wahrheit sei, die ich ver-
kündige, so erlaubte er mir, Versammlungen in seinem Hause zu
haben trotz aller Einwände; und eß wurde eine große Versamm-
lung eingerichtet, die fast vierzig Jahre, biß 1690, bestand, so
daß ein neueß Versammlungßhauß in der Nähe gebaut wurde . . .
Jch hörte von einer großen Versammlung, die in Uloerstone
stattfinden sollte, und ging darum dorthin und begab mich inß
Turmhauß, in der Furcht und der Kraft Gottes-. Alß der Priester
geendet hatte, redete ich daß Wort deß Herrn zu ihnen, daß wie ein
Hammer und ein Feuer unter ihnen wirkte (Jer. 23, 29). Lampitt,
der Priester des Orteß, war mit den meisten andern Priestern
uneinß gewesen vorher, nun aber taten sie sich alle zusammen
gegen die Wahrheit. Aber die mächtige Kraft deß Herrn war
über allem und tat sich so herrlich kund, daß Priester Bennett
sagte: »Die Kirche erbebt!« und sich fiirchtete und zitterte. Und
nachdem er einige unverständliche Worte geredet, eilte er hinauß,
auß Furcht, sie möchte über seinem Kopf zusammenstürzen. Viele
Priester versammelten sich, aber sie hatten noch keine Macht, Ver-
folgungen zu veranstalten.
Alß ich nun hier fertig war, ging ich wieder nach Swarth-
more, wohin vier oder siins Priester kamen; im Gespräch mit
ihnen fragte ich, ob einer unter ihnen sei, der sagen könne, daß
Wort deß Herm: ,,Gehe hin und rede zu den oder jenen«, sei je
einmal an ihn ergangen? Keiner wagte, eß von sich zu be-
haupten. Aber einer von ihnen wurde zornig und sagte, er
könne von Erfahrungen so gut berichten wie ich. Ich erwiderte
ihm, Erfahrungen seien allerdingß etwas, aber eine Botschaft
erhalten und damit außziehen, ein Wort vom Herm haben und
verkünden wie die Apostel und Propheten und wie ich, wenn
ich unter ihnen predige, daß sei noch etwaß andereß. Und ich
fragte sie darum noch einmal, ob einer unter ihnen sagen könne,


% \picinclude{./060-069/p_s064.jpg} 
er habe irgend einmal einen Vefehl unmittelbar vom Herrn
empfangen; aber eß konnte eß keiner. Da erklärte ich ihnen, daß
seien falsche Propheten und falsche Apostel und Antichristen,,die
die Worte der wahren Propheten und wahren Apostel und Christi
gebrauchen und die Erfahrungen anderer verwenden und selber
nie eine Stimme Gotteß oder Christi vernommen haben. Solche
wie sie könnten eben bloß die Erfahrungen un.d Worte anderer
vernehmen. Daß verwirrte sie sehr und stellte sie bloß. Ein
andermal im Gespräch mit einigen Priestern im Hause Richter
Fellö und in dessen Anwesenheit stellte ich die gleiche Frage und
siigte hinzu: daß könne eben jeder, der lesen könne, die Erfah-
rungen der Propheten rmd Apostel verkünden, die in der Schrift
aufgezeichnet seien. Hierauf bekannte ein alter Priester, Thomas
Taylor, dem Richter Fell ehrlich: er habe nie die Stimme Gotte-5
oder Christi vernommen, die ihn irgendwohin gesandt habe; er
rede von seinen eigenen Erfahrungen und den Erfahrungen der
Heiligen früherer Zeiten, und dies predige er. Solcheß bestärlte
Richter Fell in der Überzeugung, daß die Priester im Jrrtum
seien. Denn er hatte vorher, wie die meisten Leute damaltz,
geglaubt, sie seien von Gott gesandt.
Zu dieser Zeit wurde Thoma-8 Taylor 1) bekehrt und durch-
reiste init mir Westmorland ..... Die Priester wurden immer
aufgebrachter gegen unß und verfolgten un?-, wo sie nur konnten.
Jameß Naylor und Franciß Howgill wurden inß Gefängnis ge-
worfen . . . Aber dem Herrn sei Lob, die Wahrheit breitete sich
immer mehr aus. Denn um diese Zeit spürten sich John Aud-
land, John Camm, Edward Burroughi), Richard Hubberthorn 9)
1) Thomar Taylor hatte in Oxford studiert und war Puritanerprediget
geworden. Dann, weil er nicht mehr wollte ,,11m Lohn predigen--, schloß er
sich den Quakern an und wirkte eifrig als Prediger und durch Schristen.
2) Edward Burtough, ursprünglich Prediger der Epiekopalkirche, war
aus dieser ausgetreten und hatte sich den Preöbyterianetn angeschlossen; nach
einigen Untertedungen mit Fox bekehrte er sich sodann zum Quäkertnm, dessen
eifriges tätigeö Glied er blieb, bis er 1662 für seinen Glauben im Kerker, wohin
man ihn auz einer Versammlung gebracht hatte, starb.
3) Richard Hubberthorn, eine bescheidene, friedliche Natur, kriinklich und
mit einer schwachen Stimme, arbeitete dennoch Großez als Prediger. 1662 wurde
er ebenfalls aus einer Versammlung in den Kerker geschleppt, wo sein schwacher
Körper bald den Entbehrungen erlag; doch pries er noch auf seinem Toteubeit
die Güte Gottes.


% \picinclude{./060-069/p_s065.jpg} 
Christus in uns. Erkenntnis der Quüterischen Weltmission usw. 65
und andere mit der Kraft von oben ausgerüstet und traten auf
als Prediger und erwiesen sich alß treue Arbeiter, die umher-
zogen und daß Evangelium umsonst predigten, wodurch Tausende
bekehrt wurden und von nun an dem Herrn gehörten ....
Nachdem ich die Freunde in Westmorland besucht hatte, ging
ich wieder nach Uloerstone, wo Priester Lampitt war. Dieser
hatte selber gepredigt, man müsse sich von Gott lehren lassen;
und alle Menschen, Männer und Frauen, können dazu kommen,
daß Evangelium zu predigen. A18 ez sich aber darum handelte,
dieZ mit der Tat zu beweisen, so verfolgte er sowohl diese Lehre
als die Lehrer .... Als nun die Versammlungen ihren Anfang
nahmen und wir in einer Privatwohnung zusammenkamen, wurde
Lampitt sehr aufgebracht und sagte, wir verlassen den Tempel
und gehen in den Götzentempel Jerobeamö, so daß viele der
,,Frommen« sahen, wie wenig er mit dem, maß er gepredigt hatte,
Ernst machte. Nun legte man ihnen die Sache mit dem Götzen-
tempel Jerobeamz C1. Könige 13), auz und zeigte ihnen, daß
eher ihre Häuser, die sie Kirchen nennen, die Götzentempel Jero-
beamß seien und die alten Meßhäuser, die daß finstere Papsttum
eingesetzt und die jene, die sich Protestanten nennen und meinen,
sie seien ausgeklärter als- die Päpstlichen, auch noch festhalten,
obgleich doch Gott sie nie angeordnet. Und der Tempel, den
Gott in Jerusalem eingesetzt, dem habe Ehristuö. eine Ende ge-
macht, und die, welche ihn aufnahmen und an ihn glaubten, deren
Leiber wurden Tempel Gottes, in denen Christus- und der heilige
Geist wohnen (1. Cor. 6, 19T Diese versammelten sich ....
und kamen zusammen in ihren Wohnhäusern, die dann nicht
Tempel genannt wurden oder Kirchen, sondern ihre Leiber waren
i die Tempel und die Gläubigen die Kirche, deren Haupt Ehristutz
ist .... Daraus trieb ee mich ins Turmhauß zu gehen, two Priester
und ,,Fromme« und viel Volk versammelt waren. Jch stand in der
Nähe von Priester Lampitt, der daraus loß wütete in seiner
Predigt. Nachdem der Herr meinen Mund aufgetan, daß ich
reden sollte, kam Friedenßrichter John Sawrey zu mir und sagte,
wenn ich mich an das halten wolle, waz in der Schrift stehe,
so könne ich reden. Jch wunderte mich über diese Rede und sagte,
ich werde mich sicher an die Schrift halten und sie vorweisen,
um daß Gesagte zu begründen; denn ich hätte ihm und Lampitt
etwas zu sagen. Hierauf sagte er wieder, ich solle nicht reden,
George Fox. Z


% \picinclude{./060-069/p_s066.jpg} 
und widersprach sich damit selber, nachdem er ja eben gesagt
hatte, ich solle reden, wenn ich mich an die Schrift halten wolle.
Die Leute waren ruhig und hörten mir gerne zu, bis Friedens-
tichtet Sawrey, der Hauptanstifter der grausamen Verfolgung
im Norden, sie gegen mich aufhetzte, und sie ansingen, mich zu
stoßen, schlagen und quälen. Sie gerieten alsbald in Wut und
fielen über mich her, im Turmhau-8, und schlugen mich vor seinen
Augen zu Boden, stießen mich rmd traten mich mit Füßen. Der
Aufruhr war groß, so daß etliche über ihre Stühle fielen im
Gedränge. Schließlich kam Sawrey und befreite mich aus ihren
Händen und führte mich hinauö und übergab mich den Konstablern
und hieß sie mich peitschen und zur Stadt hinauß führen. Sie
führten mich etwa eine Meile weit, etliche hielten mich am Kragen,
etliche an den Armen und den Schultern und schleppten und
zerrten mich vorwärtö. Von den Freunden, die auf den Markt
und ins Turmhauß gekommen waren, um mich zu hören, wurden
viele auch zu Boden geworfen und dermaßen geschlagen, daß
manche von Blut überströmt waren. Richter Fellö Sohn, der
mir nachrannte, um zu sehen, maß mit Mir geschehe, warfen sie
in einen Wassergraben und einige schrien: ,,schlagt ihm die Zähne
aus dem Kopf.« A16 sie mich nun biz insz Moor hinauzgeschleppt
hatten, gefolgt von einem großen Haufen, gaben mir die Kon-
stabler mit ihren Weidenruten ein paar Schläge über den
Rücken und überließen mich dem Pöbel, der mit Stöcken, Hecken-
pfählen, Stechpalmen und Gichenzweigen versehen über mich
herfiel und mich auf Kopf und Glieder schlug, bis mir die
Sinne vergingen und ich auf den nassen Boden hinfiel. Als-
ich wieder zu mir kam und merkte, daß ich aus der nassen
Erde lag und die Leute mich umstanden, blieb ich einige
Zeit unbeweglich; und die Kraft des Herrn durchzuckte mich und
die ewige Erquickung erquickte mich, so daß ich wieder ausstehen
konnte in der stärkenden Kraft des Herrn; und die Arme auß-
streckend, sagte ich mit lauter Stimme: ,,Schlaget wieder, hier
sind meine Arme, mein Kopf und meine Wangen.« Einer auß
dem Haufen, ein ,,Frommer«, aber ein roher Kerl, schlug mir
mit seinem Stab genade auf die ausgestreckte Hand; meine Hand
wurde von diesem Schlag so zerquetscht und mein Arm so ge-
lähmt, daß ich ihn nicht wieder zurück ziehen konnte, und etliche
riefen: ,,Seine Hand ist für immer verstiimmelt; er wird sie nie


% \picinclude{./060-069/p_s067.jpg} 
Christus in uns. Erkenutniz der Quäkerischen Weltmission usw. 67
mehr gebrauchen können!« Aber ich betrachtete die Hand in der
Liebe zu Gott; denn ich stand zu allen, die mich verfolgt hatten,
in der Liebe Gottes; und nach einer Weile durchzuckte mich die
Liebe Gotte-3 und zuckte durch meinen Arm und meine Hand,
so daß ich augenblicklich die Kraft darin wieder spürte, vor aller
Augen. Daraufhin gerieten sie selber untereinander in Streit und
sagten mir, wenn ich ihnen Geld gebe, so wollten sie mich vor
den andern schützen. Aber der Herr trieb mich, ihnen das Wort
des Lebens zu verkünden, und ich zeigte ihnen, maß sie für ein
oerkehrtetz Christentum haben, und waß für Früchte die Predigten
ihrer Priester brächten; ich sagte ihnen, sie seien eher Juden oder
Heiden al-3 Christen. Darauf trieb mich der Herr, wieder durch
das Volk hindurch aus den Markt von Ulverstone zu gehen. Auf
dem Wege begegnete mir ein Soldat mit dem Schwert an der
Seite: ,,Herr,« sagte er, ,,ich sehe, daß Jhr ein Mann seid, und
es tut mir leid, daß Jhr so mißhandelt werdet;« und er bot mir
an, mir nach Kräften zu helfen. Aber ich sagte ihm: ,,Die Kraft
des Herrn ist über AlleZ,« und ging durch das Volk hindurch auf
den Markt, und keiner hatte Macht, mich anzurühren. Und als
auf dem Markte einige Freunde mißhandelt wurden, und ich jenen
Soldaten mit seinem nackten Schwerte mitten darunter sah, da
sprang ich hinzu, ergriff seinen Arm und befahl ihm, das Schwert
wieder einzustecken, wenn er eß mit mir halten wolle, und mit
mir auß dem Haufen herauö zu kommen; denn ich wolle nicht,
daß durch ihn ein Unheil geschehe. Einige Tage darauf wurde
dieser Soldat von sieben Männern ergriffen und durchgeprligelt,
weil er ez mit mir und den Freunden gehalten habe. ES war
in jenen Tagen die Art der Verfolger in diesen Gegenden,
ihrer 20 oder 40 aus einen einzigen loßzugehen. An Vielen Orten «
wurden die Freunde in der Weise überfallen, daß sie schier nicht
auf die Straße konnten; man warf ihnen Steine an und miß-
handelte sie. Al?. ich nach Swarthmore kam, kam ich gerade
dazu, wie die dortigen Freunde den Freunden die von Lampittö
Zuhörern mißhandelt worden waren die gebrochenen und ver-
letzten Glieder verbunden. Mein ganzer Körper war gelb, schwarz
und blau von den Schlägen, die ich an jenem Tage erhalten
hatte. Und die Priester singen wieder an zu prophezeihen, daß
wir in einem halben Jahr alle vernichtet sein würden.
Etwa zwei Wochen später ging ich auf die Jnsel Walney und
 


% \picinclude{./060-069/p_s068.jpg} 
James Naylor mit mir. An einem Morgen ging ich mit einem
Boot zu James Lancaster. Sowie ich ans Land stieg, stürzten vierzig
Männer mit Stöcken, Fischangeln und Knütteln hervor, überfielen
mich, schlugen und zerrten mich und versuchten, mich ins Wasser zurück
zu stoßen. Tlls sie mich beinahe in die See zurück geworfen hatten
und ich sah, daß sie mich umbringen wollten, lief ich mitten unter sie
zurück; aber sie fielen wieder über mich her und schlugen mich,
bis ich betäubt war. Als ich wieder zu mir kam, sah ich wie
James Lancasters Frau Steine nach meinem Gesicht wars, und
ihr Mann beugte sich über mich, um die Steine von mir abzu-
halten. Die Leute hatten James Lancasters Frau glauben
machen, ich hätte ihren Mann verhext, und hatten ihr ver-
sprochen, wenn sie ihnen melde, wann ich kommen werde, so
wollten sie mich töten. Und als bekannt wurde, daß ich komme,
hatten viele aus der Stadt sich aufgemacht, mit Knütteln und
Stöcken, um mich zu töten. Aber die Kraft des Herrn
schützte mich, daß sie mir nichts antun konnten. Zuletzt gelang
es mir, wieder aufzustehen; aber sie warfen mich sogleich wieder
ins Boot zurück. Als James Lancaster es sah, kam er sogleich
und brachte mich übers Wasser, daß ich vor ihnen sicher war;
aber so lange wir noch erreichbar waren auf dem Wasser, warfen
sie uns Steine nach. Als wir am anderen User ankamen, sahen
wir, wie sie James Naylor schlugen. Während ich noch drüben
gewesen war, hatte er sich abseits gehalten, so daß sie ihn erst sahen,
als ich fort war; da übersielen sie ihn und schrieen: ,,Tötet ihn!-«
Llls ich die Stadt, am anderen User, erreichte, kamen die
Leute mit Drefchflegeln und Stöcken, um mich zu verhindern, in
die Stadt zu kommen und schrieen: ,,Tötet ihn, tötet ihn! Schlagt
ihn auf den Kopf- bringt den Karren und führt ihn aus den
Kirchhoflss Nachdem sie mich mtßhandelt hatten, schleppten sie
mich aus der Stadt und ließen mich liegen. James Lancaster
ging nun zurück, um nach James Naylor zu sehen; als ich nun
allein da war, ging ich zu einem Wassergraben, und nachdem ich
mich gewaschen hatte, ging ich drei Meilen weit zu Thomas
Hutton, wo Lawson, der bekehrte Priester war. Als ich eintrat,
konnte ich fast nicht reden, so war ich zugerichtet; ich sagte ihnen nur,
wie ich James Naylor verlassen; da nahm jeder von ihnen ein
Pferd und holten ihn noch in jener Nacht zu sich. Als Margaret
Fell am nächsten Tag davon hörte, schickte sie ein Pferd und lleß


% \picinclude{./060-069/p_s069.jpg} 
Chrisiutz in unö. Erkenntnis der Quiikerischen Weltmission usw. 69
mich holen; aber ich war so verwundet, daß ich daß Schütteln
dez Pferdes nur mit großen Schmerzen ertragen konnte. Al-3 ich
nach Swarthmore kam, erließen die Friedenörichter Sawrey und
Thompson von Lancaster einen Verhastbefehl gegen mich; aber
als Richter Fell zurückkam, wurde er nicht auzgeführts Richter
Fell war nämlich die ganze Zeit meiner Mißhandlung nicht in
der Stadt gewesen. A13 er zurück kam, schickte er Verhaftbefehle
nach der Jnsel Walney, um alle jene Aufrührer festzunehmen,
worauf viele von ihnen entftohen; James Lanrasterß Frau be-
kehrte sich später zur Wahrheit und bereute, waö sie mir ange-
tan hatte, sowie auch andere jener grausamen Verfolger; aber
viele von ihnen traf daß Gericht Gottes-, und ez sind etliche unter
ihnen seither zu Grunde gegangen. Richter Fell verlangte einen
Bericht meiner Verfolgung; aber ich sagte ihm, sie hätten ja nicht
anderß handeln können in dem Geiste, in dem sie seien; es seien
die Früchte von dem, waö ihre Priester predigten, und beweise,
daß ihre Frömmigkeit und Religion falsch sei; er berichtete seiner
Frau, ich nehme die Sache leicht, wie einer, den sie nichts angehe;
und in der Tat hatte mich deö Herrn Kraft wieder geheilt ....
Jch ging mit Richter Fell zur Gerichtösitzung nach Lancaster;
er gestand mir unterwegs-, daß ihm noch nie eine solche Ange-
legenheit vorgekommen sei, und daß er nicht recht wisse, wie
er sich dabei verhalten solle. Ich sagte ihm, daß Pauluö, als-
er vor die Obersten der Schule trat und die Juden und Priester
viele falsche Anklagen gegen ihn vorbrachten, die ganze Zeit stille
schwieg. Dann, alß Festuö und Agrippa ihn hießen, für sichselber
reden, tat er ek- und reinigte sich von allen jenen falschen An-
schuldigungen; so sollte er ez mit mir machen. Vor dem Gericht
in Lancaster traten etwa vierzig Priester gegen mich aus; zuxihrem
Redner hatten sie einen namens Marshall gewählt und als
Zeugen einen jungen Priester und zwei Priestersöhne, die schon
vorher beschworen, daß ich eine Gotteölästerung ausgesprochen.
Die Richter hörten alleö an, maß die Priester und ihre Zeugen
gegen mich Vorbringen konnten .... aber die Zeugen waren
so verwirrt, daß sie sich bald alö falsche Zeugen verrieten ....
ES waren mehrere Leute anwesend, die auch in jener Ver-
sammlung gewesen waren, in der ich die Gotteßlästerung auöge-
sprochen haben sollte, alles Leute, die geachtet und angesehen waren
in dieser Gegend; sie erklärten, vor offenem Gerichtöhof, daß die



% \picinclude{./070-079/p_s070.jpg} 
Gide der Zeugen gänzlich falsch seien, und daß ich nichts der-
gleichen geäußert habe .... Oberst West, der ale Friedenörichter
der Gegend hier war, schenkte dieser Auzsage Gehör; und
nachdem er vorher lange krank gewesen war, bekannte er nun,
heute habe ihn der Herr geheilt, und fügte bei, er habe noch nie
so viele gute Menschen und liebe Gesichter beisammen gesehen in
seinem ganzen Leben. Und darauf wandte er sich zu mir und
sagte vor allen: ,,George, wenn du irgend etwas zu den Leuten
zu sagen hast, so tue ee ungehindert«. GS trieb mich zu reden,
worauf der Priester, der gegen mich geredet hatte, sich davon
machte. Jch sühlte mich getrieben zu erklären, daß: ,,die heilige
Schrift vom Geist Gotteß eingegeben sei, und daß alle zuerst
den Geist Gottes in ihrem Jnnern erkennen müssen, durch den
sie Gott und Ehristuö, von denen die Propheten und Apostel
lernten, erkennen können; und durch diesen selben Geist werden
sie dann auch die heilige Schrift verstehen. Denn wie der Geist
Gottes in denen war, die die Schrift geschrieben, so muß derselbe
Geist Gottes auch in denen sein, die die Schrift verstehen wollen;
durch diesen Geist haben sie allein Gemeinschaft mit dem Vater
und dem Sohne und mit der Schrift und untereinander; ohne
diesen Geist aber kann man weder Gott noch Christuß, noch die
Schrift kennen, noch Gemeinschaft untereinander haben«. Kaum
hatte ich solches gesagt, so brachen eine Anzahl Priester hinter
mir los und einer, Jackuö, behauptete unter anderm, der
Buchstabe und der Geist seien unzertrennlich. Darauf erwiderte
ich: ,,dann hat also jeder, der den Buchstaben hat, auch den
Geist, und kann also den Geist mit dem Buchstaben der Schrift
kausen«. Richter Fell und Hauptmann West machten den Pnestcttt
Vorstellungen über diesen ofsenkundigen Irrtum und sagten ihnen,
daß sie ja dann ihrer Ansicht nach den Geist in der Tasche herum
tragen könnten, wie den Buchstaben. Als die Priester sich besiegt
sahen, kehrten sie ihre Wut gegen die Friedenörichter, weil sie
ihre Rache gegen mich nicht stillen konnten. A18 die Friedens-
richter sahen, daß die Zeugen nicht mit einander übereinstimmten,
und daß sie eigentlich nur gewonnen worden waren, um der
Bosheit der Priester zu dienen, und daß alle ihre Anklagen nicht
gültig waren vor dem Gesetz, sprachen sie mich frei .... Jch
war also vor offenem Gerichtshof von allen falschen Anschul-
digungen gereinigt, und viele priesen Gott darüber; denn es war


% \picinclude{./070-079/p_s071.jpg} 
Fox der hexerei verdächtigt. Falsche Offenbarungen usw. 71.
ein Tag der Freude für viele. Friedenßrichter Benson von West-
morland 1) und Major Ripan von Lancaster wurden gewonnen.
GS war ein Tag deS Heilö für Hunderte; denn der Herr Jesus
Christus, der ,,Weg zum Vater«, und der ,,Lehrer der umsonst
lehrt--, wurden gepriesen, und sein ewige-8 Evangelium wurde ge-
Hpredigt und das Leben wurde verkündet, trotz allen diesen Priestern
und gewinnsüchtigen Predigern. Der Herr öffnete an dem Tage
vielen den Mund, daß sie den Priestern Vorstellungen machten
in den Herbergen und in den Straßen, so daß sie wie ein altes
morscheö Gebäude zerfielen; und es hieß allgemein, die Quäker
hätten gesiegt, und die Priester seien unterlegen. Unter andern
war auch Thoma-3 Briggs an diesem Tage gewonnen worden. Gr
war ein Gegner der Freunde gewesen, und als er einmal mit
John Lawson, einem Freund, über die Vollkommenheit geredet hatte,
rief er: ,,waZ, du glaubst an V-ollkommenheit?« und gab ihm dabei
eine Ohrfeige. Dieser Thomaß Briggß 2) wurde an diesem Tage
gewonnen und trat gegen seinen eigenenen Priester Jackuz auf; er
wurde nachher ein treuer Diener dez Evangeliums und blieb es
biz anö Ende seiner Tage ......


%%%%%%%%%%%%%%%%%%% Kapitel 6. %%%%%%%%%%%%%%%%%%%%%%%%%%%%%%

\chapter[Falsche Offenbarungen]{Falsche Offenbarungen}

\begin{center}
\textbf{Fox der Hexerei verdächtigt. Falsche Offenbarungen 
bei Freunden. Gefangenschaft in Carlisle.}
\end{center}

.... Von Lancaster ging ich zu Friedenßrichter West; Richard
Hubberthorn begleitete mich. Da wir den Weg und die Gefahr
der Sandbänke nicht kannten, ritten wir über eine Stelle, über
die, wie wir nachher erfuhren, noch nie jemand zuvor geritten
war. Wir ließen unsre Pferde über sehr gefährliche Stellen
schwimmen. Alß wir ankamen, fragte untz Friedenzrichter West,
ob wir nicht zwei Männer hätten über die Sandbänke reiten
sehen. »Jch werde«, fügte er bei, ,,über kurzem ihre Kleider
1) Getoase Benson war früher Oberst in der Armee gewesen und nun
Friedenstichter in Kendal.
2) Thoma-3 Briggs, der bisher ein eisriger Verfolger der Freunde ge-
wesen, wurde nun ihr Anhiiuger und ein bedeutender Prediger. E: hatte eine
große Gabe der Überzeugung. Er begleitete Fox aus vielen Reisen.


% \picinclude{./070-079/p_s072.jpg} 
haben, denn sie sind sicher ertrunken, und ich bin der Leichen-
schauer«. A15 wir ihm nun sagten, daß wir diese Männer seien,
da wunderte er sich sehr und wollte kaum glauben, daß wir nicht
ertrunken seien. Und die Priester und »Fromen« benützten es, um
daß Gerücht über mich zu verbreiten, ich könne nicht ertrinken mid
man könne mich nicht bluten machen, also sei ich ein Zauberer.
EZ war in der Tat oft vorgekommen, daß ich kaum blutete, wenn
sie mich mit ihren Stöcken schlugen und meinen Leib arg miß-
handelten. Alle diese Verleumdungen kitmmerten mich nicht um
meiner selbst willen; nur um die Wahrheit war mir bange,
gegen die sie mit solchen Mitteln die Leute einzunehmen suchten;
denn ich dachte daran, wie ihre verräterischen Vorfahren den
Hau?-herrn Beelzebub genannt hatten (Matth. 10, 25), und so
konnten ja diese von dem Leben und der Kraft Gotteß abge-
fallenen Christen mit seinem Samen nicht anderßz verfahren. Aber
die Kraft dez Herrn erhob mich über ihre Verläumderischen Zungen
und ihre blutige, mörderische Gesinnung; sie waren selber behext
und darum konnten sie nicht zu Gott und Christuö kommen.
Von Frieden?-richter West ging ich nach Swarthmore, wo
die Kraft deß Herrn die Verfolger niederhielt. ES trieb mich,
verschiedene Briefe von hier aus an die Magistrate, Priester und
,,Frommen« der Umgegend, die sich früher an den Verfolgtmgen
beteiligt hatten, zu schreiben. . . und hernach trieb es mich, an
die Leute in Uloerstone im allgemeinen einen Mahnbrief zu
schreiben ....
Unter den eifrigsten Zuhörern und Nachsolgern dez Priester-?
Lampitt vonU10erstone war ein Adam Sands, ein sehr schlechter,
verdorbener Mensch, der gerne die Wahrheit und ihre Anhänger
vernichtet hätte, wenn er gekonnt hätte. GZ trieb mich, an diesen
also zu schreiben:
,,Adam Sands!
Jch wende mich an daß Licht in deinem Gewissen, du Kind
des Teuselö, du Feind der Gerechtigkeit. Der Herr wird dich
darniederwerfen, wenn du schon eine Zeitlang jetzt herrschest. Die
Strafe Gottes muß dich treffen, der du dich in deiner Bosheit gegen
die reine Wahrheit Gotteß verhärtest. Durch die reine Wahrheit
Gotteö, die du verfolgest und der du widerstrebst, wirst du ver-
nichtet werden; sie ist ewig und schließt auch dich ein; du wirst
in dem Lichte, daß du verachteft, gesehen und in demselben ver-


% \picinclude{./070-079/p_s073.jpg} 
Fox der Hexerei verdächtigt. Falsche Osseubarungen nsw. 73
dümmt, du in deinem tierischen Wesen und dein Weib in seiner
Heuchelei; euer Morden der Gerechtigkeit wird erkannt werden;
das Licht in deinem Gewissen wird dir das, was ich dir hier
schreibe, bezeugen und wird dich erkennen lassen, daß du nicht
aus Gott geboren bist, sondern daß du sem von der Wahrheit
noch in einem tierischen Wesen bist. Wenn je einmal deine Augen
die ausgehen werden und du bereust, so wirst du sehen, daß ich
ein Freund deiner Seele bin und dein ewiges Heil will.
G. F.«
Dieser Adam Sands kam später elendiglich um .....
Ich ging nach Swarthmore zurück. Ich hatte große Offen-
banmgen vom Herm, nicht nur uber göttliche Dinge, sondern
auch über äußere, die die Regierung betrafen. Eines Tages,
als ich im Gerichtssaal Richter Fell und Friedensrichter Benson
über die jüngsten Ereignisse sprechen hörte und oom Parlament,
dasidamals tagte, und das man das ,,lange Parlament« nannte,
trieb es mich, ihnen zu sagen, daß, ehe zwei Wochen um seien,
das Parlament aufgelöst und der Redner von seinem Stuhl herunter
gerissen sein werde. Und als nach zwei Wochen Friedensrichter
Benson wieder kam, sagte er zu Richter Fell, jetzt sehe er, daß
George Fox ein wahrer Prophet sei: Oliver Cromwell habe das
Parlament ausgelöst! (20. April 1653.)
Um diese Zeit fastete ich etwa 10 Tage lang, weil mein
Geist um der Wahrheit willen schwer heimgesucht war; denn
James Milner und Richard Näher hatten Einbildungen und viele
machten es ihnen nach. Dieser James Milner und einige seiner
Anhänger hatten zuerst wahre Offenbarungen; aber da sie in
Hochmut und Selbstiiberhebung gerieten, irrten sie von der
Wahrheit ab. Der Herr trieb mich, zu ihnen zu gehen und ihnen
Ihre Verirrungen vorzustellen; und sie kamen dazu, ihre Torheit
emzusehen, und gaben sie auf und kamen aus den Weg der
Wahrheit zurück. Darauf begab ich mich in eine Versammlung
M Arn-Side, der Richard Myer beiwohnte; er hatte lange einen
lshmen Arm gehabt. Der Herr trieb mich, ihm vor allen An-
wesenden zu sagen: ,,Stehe auf!« und er stand aus und streckte
seinen Arm, der so lange lahm gewesen war, aus und sagte:
»Wisset, alle ihr Leute, daß ich heute geheilt worden bin.« Seine
Eltern wollten es kaum glauben, und als die Versammlung
vorbei war, nahmen sie ihn aus die Seite und zogen ihm sein


% \picinclude{./070-079/p_s074.jpg} 
Wamß au?-; da sahen sie, daß ez wahr sei. Er kam bald darauf
in eine Versammlung in Swarthmore und berichtete da, wie
der Herr ihn geheilt habe. Dornach befahl ihm der Herr, nach
York zu gehen in seinem Auftrag; aber er gehorchte dem Herrn
nicht; und der Herr schlug ihn abermals, daß er etwa dreiviertel
Jahr daraus starb ....
Um diese Zeit wurde Anthony Pearson X), der ein Gegner der
Freunde gewesen war, gewonnen. Er kam nach Swarthmore,
und da ich gerade dort bei Oberst West war, holte man mich.
Oberst West sagte: ,,Geht, Fox, denn Jhr könnt dem Mann zu
großem Nutzen gereichen«—. Also ging ich, und die Kraft deß
Herrn ergriff ihn.
Um diese Zeit tat der Herr auch etlichen den Mund auf,
daß sie den Priestern und dem Volk die Wahrheit verkündeten,
und viele wurden de-zwegen inö Gefängniö geworfen. Jch ging
nun nach Cumberland, wo Anthony Pearson, seine Frau und
mehrere Freunde mich nach Bootle begleiteten; Anthony Pearson
verließ uns dann, um zur Gerichtßsitzung nach Carlißle zu gehen;
denn er war Frieden?-richter in drei Grafschasten. An einem
Ersten Tage ging ich ins Turmhauö von Vootle, und als der
Priester fertig war, sing ich an zu reden. Aber die Leute waren
sehr unverschämt und prügelten mich im Hofe. Einer gab mir
einen starken Schlag aus daß Handgelenk, sodaß man allgemein
glaubte, er hätte meine Hand in Stücke geschlagen. Der Kon-
stabler hätte gem den Frieden wieder hergestellt und einige, die
mich geschlagen, eingesteckt; aber ich ließ es nicht zu. Nachdem
ich zu ihnen geredet, ging ich nach der Wohnung dez Joseph
Nicolson und der Konstabler begleitete mich, um mich vor der
Menge zu schützen.
Am Nachmittag hatte der Priester einen andern Priester
kommen lassen, einen sehr angesehenen Mann auß London. Ehe
ich inö Turmhautz eintrat, saß ich eine Weile auf dem Platz davor
und einige Freunde mit mir; aber die Freunde wurden getrieben, ins
Turmhauz zu gehen, und ich ging ihnen nach. Der Londoner
Priester brachte in seiner Predigt alle erdenklichen Schriftftellen
von falschen Propheten rmd Antichristen und wandte sie auf unö
an. Aber alß er geendet, nahm ich alle die Schriftstellen noch
1) Friedenörichtev Pearson wurde ,,bekehrt, als er auf dem Richterstuhl saß«.


% \picinclude{./070-079/p_s075.jpg} 
Fox der Hexerei verdächtigt. Falsche Qssenbarungen usw. 75
einmal durch und kehrte sie gegen ihn. Darauf überfielen mich
die Anwesenden, aber der Konstabler befahl ihnen Ruhe. Nun
wurde der Priester zornig und erklärte, ich dürfe nicht an diesem
Ort reden. Jch erklärte ihm, er habe auch seine Stunde zum
Predigen gehabt, nun sei seine Zeit um, und nun dürfe ich so
gut die meine reden wie er, denn er sei auch nur ein Fremder
hier. Und ich öffnete ihnen die Schrift und zeigte ihnen, daß
diese Stellen, die von falschen Propheten, Betrügern und Anti-
christen reden, sie und ihreögleichen betrefse und alle, die in ihren
Fußstapfen gehen und die gleichen Früchte hervorbringen wie sie;
und nicht uns-, denn man könne uns solche Dinge nicht nach-
sagen. Jch zeigte ihnen, wie sie nicht in den Fußstapfen der
wahren Propheten und Apostel seien und wies ihnen an den
Früchten, die sie hervorbringen, nach, daß sie etz seien, von denen
die Schriftstellen handeln und nicht wir. Und ich verkündete ihnen
die Wahrheit und daß Wort deß Lebenß und wies sie aus Christ-.13,
ihren Lehrer. Alletz war ruhig während ich redete; aber alß ich
geeudet hatte und hinaus kam, waren die Priester in einer solchen
Wut, daß ihr Mund gegen mich schäumte. Der Priester des
Orts redete auf dem Turmplatz zu den Leuten und sagte ihnen:
,,Dieser Mensch hat in Laneashire alle rechtfchassenen Männer
und Frauen für sich zu gewinnen gewußt, und nun will er hier
da?-selbe tun.« Jch erwiderte ihm: ,,WaS bleibt dann für die
Priester übrig, außer solchen wie sie selber sind? Denn wenn es
die Rechtschafsenen sind, die sich zur Wahrheit bekehren und sie
aufnehmen und sich zu Christuß bekehren, so sind es die Schlechten,
die dir und deineßgleichen folgen! Etliche suchten für ihren Priester
einzutreten, und für das Zehntenwesen; aber ich sagte ihnen, sie
täten besser, für Christus einzutreten, der den Zehntenpriestern
und dem Zehntenwesen ein Ende machte und der seine Jünger
aussandte mit der Weisung: ,,untsonst zu geben, wa-3 sie umsonst
empfangen hatten«. Und des-Z Herrn Macht kam über alle und brachte
sie zum Schweigen und hielt die Schreier zurück, daß sie den Unfug,
den sie planten, nicht ausführen konnten. A13 ich zu Joseph
Nieolson zurück kam, entdeckte ich ein großeß Loch in meinem
Rock, daß von einem großen Messerstich herrührte; aber ez war
nicht tiefer alö der Rock gegangen, denn der Herr hatte ihre
Ubeltat vereitelt .....
Darnach ging ich in ein Dorf, und eine große Schar be-


% \picinclude{./070-079/p_s076.jpg} 
gleitete mich. Während ich in einem mit Leuten ganz gefüllten
Hauö das- Wort des Lebenö verkündete, gewahrte ich eine Frau,
die, wie ich gleich merkte, einen unsauberen Geist hatte. Der
Herr trieb mich, ernstlich mit ihr zu reden und ihr zu sagen, sie
sei unter dem Einfluß eineö unsauberen Geisteö; hierauf verließ
sie daß Zimmer. Weil ich fremd war an diesem Orte und die
äußeren Verhältnisse der Frau nicht kannte, wanderten sich die
Leute sehr und sagten mir nachher, ich hätte etwas Merkwiirdigeß
entdeckt; denn diese Frau sei wirklich lange als eine schlechte
Person bekannt gewesen. Der Herr hatte mir die Gabe der
Unterscheidung gegeben, durch welche ich den Zustand und die
Verfassung der Leute ost erkannte und die Geister prüfen konnte
denn nicht lange vorher, alö ich in eine Versammlung ging, sah
ich auf dem Felde einige Frauen, bei denen ich einen unsauberen
Geist erkannte; und etz trieb mich, von meinem Wege ab zu ihnen
zu gehen und ihnen ihren Zustand aufzudecken. Ein andermal
kam eine in die Versammlung in Swarthmore, und etz trieb mich,
ernstlich mit ihr zu reden und zu sagen, sie stehe unter der Macht
eines bösen Geisteö; und die Leute sagten nachher, etz sei daß
allgemein von ihr bekannt. Gin andermal kam eine andere Frau
und stand in einiger Entfernung von mir, und es trieb mich zu
ihr zu gehen und zu sagen: »Du bist eine Hure gewesen«; denn
ich erkannte den Zustand und daß Leben dieser Frau; sie antwortete
mir, es gebe viele, die ihr ihre äußern Sünden nennen können,
aber ihre inwendigen habe ihr noch niemand sagen können; daraus
sagte ich ihr, ihr Herz tue nicht recht vor dem Herrn, rmd
aus dem inwendigen komme das auöwendige; diese Frau wurde
nachher von der Wahrheit dez Herrn überzeugt und schloß sich
den Freunden an .....
Wir gingen nun nach Earliöle ..... An einem Markttage
ging ich aus den Markt. Die Magistrate hatten Drohungen er-
gehen lassen und ihre Leute geschickt; und ihre Frauen hatten
gesagt, wenn ich komme, so reißen sie mir die Haare auö, und
die Schutzleute sollten mich nur sestnehmen. Aber ich ging dennoch
auf den Platz, im Gehorsam gegen den Herrn, und verkündete
ihnen dort, daß der Tag des Herm über all ihr betrügerischeö
Tun und ihre betriigerische Ware komme; und sie sollten sich alle
abwenden von ihrem Beträgen und Überlisten und sich an Ja
und Nein halten und einander die Wahrheit sagen; dann komme


% \picinclude{./070-079/p_s077.jpg} 
Fox der Hexerei verdächtigt. Falsche Offenbarungen usw. 77
die Kraft und die Wahrheit des Herrn zu ihnen. Nachdem ich
ihnen so das Wort des Lebens verkündet hatte, in einem Ge-
dränge, das zu groß gewesen war, als daß die Schutzleute und
die Weiber der Magistrate zu mir hätten gelangen können, zog
ich ruhig weiter. Viele Soldaten und andere kamen zu mir und
einige Baptisten, die heftige Streiter waren; unter diesen war
auch ein Helfer, ein böser Mann, der, als er die Kraft des Herrn
verspürte, aufschrie vor Zorn, worauf ich meine Augen auf ihn
heftete und ernstlich zu ihm redete in der Kraft des Herm; und
er schrie: ,,Durchbohre mich nicht so mit deinen Augen! wende
deine Augen ab von mir«.
Am folgenden Ersten Tage ging ich ins Turmhaus, und
nachdem der Priester geendigt hatte, predigte ich den Leuten
die Wahrheit und das Wort des Lebens. Der Priester entfernte
sich und man wollte mich aus dem Turmhaus jagen. Aber ich
verkündete den Weg des Herrn weiter unter ihnen und sagte:
,,ich komme, euch das Wort des Lebens und der Seligkeit zu ver-
künden«. Die Macht des Herrn tat sich mächtig kund unter
ihnen, so daß sie zitterten und bebten, und meinten, das Turm-
haus schwanke, und einige meinten, es werde auf ihre Köpfe fallen;
die Weiber der Magistrate rasten und suchten mit aller Gewalt,
an mich heran zu kommen; aber die Soldaten und die Freunde
umringten mich. Zuletzt kam der ganze Pöbel der Stadt ins
Turmhaus, mit Stöcken und Steinen und schrie: ,,nieder mit
diesen rundköpfigen Schuften!« und warfen mir Steine an.
Hierauf schickte der Statthalter Soldaten ins Turmhaus, um
Ruhe zu schaffen unter den Leuten; mich nahmen sie freundlich
bei der Hand und hießen mich mit ihnen kommen. Als wir
auf die Straße kamen, war die Stadt in Aufrrchr, und einige
dieser Soldaten kamen ins Gefängnis, weil sie sich meiner ange-
nommen hatten, gegen die Leute aus der Stadt. Gin Leutnant,
der belehrt worden war, nahm mich in sein Haus, wo eine Vap-
tistenversammlung war; auch Freunde kamen dazu, und wir hatten
eine sehr ruhige Versammlung; sie hörten das Wort des Lebens
gerne, und viele nahmen es auf. Am folgenden Tage, als die
Magistrate im Stadthaus versammelt waren, ließen sie mich vor
sie bringen. Jch war eben im Haus eines Baptisten; als ich
von dem Befehl hörte, ging ich nach dem Stadthaus hinauf, wo
viel Pöbel versammelt war, der allerlei falsche Dinge über mich


% \picinclude{./070-079/p_s078.jpg} 
auögesagt hatte. Ich hatte eine lange Unterredung mit den
Magistraten, worin ich auseinandersetzte, waz für Früchte die
Predigten ihrer Priester bringen, und wie wenig Christentum darin
sei; und ich sagte ihnen, daß sie zwar alß große »Fromme«
gelten, — sie waren Preßbhterianer und Jndependenten — aber
eben nicht im Besitz ihrer Frömmigkeit seien. Nach einem langen
Verhör verurteilten sie mich zum Gefängniö, alß Gottes-lästerer,
Ketzer und Verführer, obgleich sie mich gerechter Weise keines
dieser Dinge beschuldigen konnten. EZ waren zwei Kerkermeister
im Kerker von Carliöle, ein oberer und ein unterer, die auß-
sahen wie zwei große Värensührer. A15 ich gebracht wurde,
führte mich der Oberkerkermeister in ein großeß Zimmer und sagte
mir, ich könne hier haben, was ich wolle; aber ich erwiderte
ihm, er solle kein Geld von mir erwarten, denn ich werde weder
in einem seiner Betten schlafen, noch von seinen Speisen essen,
woraus er mich in ein anderes Gemach führte, wo ich nach einiger
Zeit etwas zum drauf liegen erhielt. Hier lag ich gefangen biß-
zur Zeit der Gerichtösitzung, wo ich, wie es- allgemein hieß, er-
henkt werde. Der Oberscherifs Wilfrid Lawson, hetzte sie auf,
mich zu töten, und sagte, er wolle mich selbst biß zu meiner Hin-
richtung bewachen. Sie waren sehr streng und setzten drei Muske-
tiere zu meiner Wache, einen vor meine Türe, einen anderen
unten an die Treppe und einen dritten vor die Haustüre, und
sie ließen niemand zu mir, außer um mir das nötigste zu bringen.
Dez Nachts brachten sie Priester zu mir, oft erst um zehn Uhr,
die schrecklich roh und teuflisch waren. GS gab eine Rotte von
schottischen Priestern, Preßbyterianer, zusammengesetzt auö Neid
tmd Bo?-heit, die nicht »geschickt waren, göttliche Dinge zu reden«
und sehr schmutzige Reden führten. Aber der Herr verlieh mir
durch seine Kraft die Herrschaft über sie alle, so daß sie erkannten,
in welchem Geist sie waren und maß sie für Früchte brachten.
Auch angesehene sogenannte ,,Damen« (lmtjez) kamen, um den
Mann zu sehen, von dem es hieß, er müsse sterben. Während
die Richter und Räte miteinander berieten, auf welche Art ich
sterben solle, vereitelte der Herr in Unerwarteter Weise ihren
Anschlag, indem der Anwalt einen Einwand verbrachte, der
alle ihre Absichten über den Haufen warf, so daß sie keine
Macht mehr hatten, mich vor Gericht zu bringen .....
Nachdem die Richter die Stadt verlassen hatten, erhielt der


% \picinclude{./070-079/p_s079.jpg} 
Fox der Hexerei verdächtigt. Falsche Ossenbarungen usw. 79
Kerkermeister Befehl, mich in den untersten Kerker zu den Straßen-
räubern, Dieben und Mördern zu werfen, obgleich ich schon vor-
her in sehr strengem Gewahrsam gewesen war. Jch kam nun
an einen gräulichen, schmutzigen Ort, wo nicht einmal ein Abtritt
war, Amd Frauen und Männer in unziemlicher Weise zusammen-
gesperrt waren, und die Gefangenen waren voll Läuse, so daß eine
Frau fast davon aufgesressen wurde; aber so schlecht auch der
Ort war, so kamen doch die Gefangenen alle dazu, mir zugetan
und ganz nachgiebig zu werden, und etliche wurden von der Wahr-
heit bekehrt, wie dieß bei Zöllnern und Huren zu allen Zeiten
geschehen, so daß sie jeden Priester, der anß Gitter kam, um mit
ihnen zu diöputieren, zu Schanden machen konnten. Der Kerker-
meister war sehr hart und der Unterkerkermeister roh gegen mich und
gegen die Freunde, die zu mir kamen. Gr schlug oft Freunde,
die nur anß Gitter kamen, um mich zu sehen, mit einem großen
Knüttel. Jch konnte am Gitter hinaus steigen, um zuweilen etwas
Fleisch herein zu langen, was ihn schrecklich böß machte. Einmal
überkam ihn ein solcher Zorn, daß er mich mit einem Kniittel
durchprügelte und dazu schrie: ,,komm vom Fenster weg!« obschon
ich gerade damals nicht dran war. Während er mich schlug,
—kam es in der Kraft des Herrn über mich, zu singen, maß ihn
noch wütender machte. Esr holte einen Geigenspieler und ließ
ihn vor mir spielen, weil er meinte, mich damit zu verdrießen.
Aber während seinem Spiel kam eZ über mich, in Gottes ewiger
Kraft zu singen, und meine Stimme übertäubte den Lärm des
Geigerö, waö ihn so oerwirrte, daß er das Spielen aufgab und
sich daoonmachte.
Richter Vensontz Frau fühlte sich getrieben, mich zu besuchen,
und kein anderes- Fleisch zu essen, al-3 von dem, daö man mir
an die Kerkertür brachte. Später wurde sie selbst in York ins
Gesängniß getan, während sie schwanger war, weil sie einem
Priester widersprochen hatte, und man gestattete ihr nicht, aus
dem Gefängniö zu gehen zur Zeit ihrer Niederkunft; so gebar
sie im Kerker ein Kind. Sie war eine gläubige, gottselige Frau,
und blieb etz biö zu ihrem Tode.
Während meiner Gefangenschaft im Kerker zu Carliöle ver-
breitete sich daß Gerücht von meiner wahrscheinlicher! Hinrichttmg
überall hin. Alß sie im Parlament — ich glaube es wurde das
kleine Parlament genannt — hörten, es sollte in Carliöle ein



% \picinclude{./080-089/p_s080.jpg} 
80 Kapitel V1.
junger Mann um seines Glaubens willen hingerichtet werden,
schrieben sie deshalb an die Magistrate.
Ungefähr um die gleiche Zeit schrieb ich an die Behörden
von Earlisle, die mich ins Gefängnis geworfen und die die
Freunde auf Anstiften der zehntengierigen Priester verfolgten:
,,Freunde! Thomas Eraston und Cuthbert Stadholm,
Guer Tun ist in London bei den Gutgesinnten bekannt ge-
worden. Was habt ihr alles geleistet an Gesangennehmen,
Güterschändungen, Metzeleien und anderen Scheußlichkeiten in den
letzten paar Jahren! ganz menschenunwiirdig, wie wenn ihr
noch nie die Schrift gelesen und zu Herzen genommen hättet!
Jst das das Ziel der Religion Earlisles und seiner Kirche und
seiner Ehristlichkeit? ihr habt es zu schanden gemacht mit eurer
Blindheit, eurem tollen Treiben und eurem verkehrten Gtfern.
War es nicht immer die Art der blinden Leiter und der falschen
Propheten zu zanken (Jes. 56), mit denen, die ihnen den Mund
nicht füllen wollen? Seid ihr nicht die Lasttiere und Diener der
Priester gewesen? Wenn sie euch anspornen, das Schwert gegen
den Unschuldigen zu gebrauchen, so rennt ihr auf solche, die nach 3
den Befehlen der Schrift die Waffe nicht gebrauchen dürfen, loss
Und doch wollt ihr eure unheiligen Hände und gemeinen Lippen
zu Gott erheben, und gebet oor, zu fasten und seid doch voll
Hader und Zank (Jes. 58, 4). Brannte nie euer Herz in euch?
habt ihr nie über euren Zustand nachgedacht? Seid ihr ganz
der Lust des Teufels, dem Verfolgen, anheimgefallen? Wo ist
eure Feindes-liebe? (Matth. 5). Wo ist euer Beherbergen der
Fremdlinge? (Matth. 25, 35). Wie überwindet ihr Böses mit
Gutem? (Röm. 12, 21). Wo sind eure Lehrer, die ,,durch heil-
same Lehre die Widersprecher strafen?« (Tit. 1, 9) .... Leset die
Schrift und sehet, wie unähnlich ihr den Aposteln und Propheten
seid; und wie ihr denen gleichet, die die Propheten, die Apostel
und Christus verfolgten. Jhr gehet in ihren Fußstapfen und
kämpfet mit Fleisch und Blut, nicht mit den Fürsten der Welt,
die in der Finsternis dieser Welt herrschen, und mit den bösen
Geistern unter dem Himmel« (Gph. 6, 12). Jn keinem anderen
Lande geschehen solche Greuel, daß man den Leuten ihr Gut
raubt, ihnen ihre Ochsen und Rinder nimmt, ihre Schafe, ihr
Getreide und ihr Hausgeräte und gibt es den Priestern, die doch
nichts für sie gearbeitet haben. Jhr seid eher Straßenräuber


% \picinclude{./080-089/p_s081.jpg} 
Fox der Hexerei verdächtigt. Falsche Qsfenbarungen usw. 81
alß Diener Gottes gegen die Freunde; ihr verklagt sie bei euren
Gerichten und legt ihnen Bußen auf, weil sie die Gebote Christi
nicht übertreten, also nicht schwören wollen« ..... G
Anthony Pearson and Gervase Benson dursten mich nicht im
Gesängniö besuchen, obwohl sie Frieden?-richter waren. Sie
schrieben darum an die Magistrate und Priester von Carlißle:
,,Wir bezeugen, daß dieser George Fox, der von den Magi-
straten, von den Friedenörichtern, den Priestern und dem Volt ver-
folgt wird und gegenwärtig alt?. Gotteßlästerer und Verführer
gefangen gesetzt ist, ein Prediger dez Worteß Gottes ist und daß
ewige Evangelium verkündet; durch sein mächtiges Predigen hat
der große Vater der Heiligen den Blinden die Augen geöffnet,
den Tauben die Ohren aufgetan, die Gefangenen erlöst und die
Toten auferweckt (Jes. 35, 5). Christuß wird jetzt gepredigt unter
den Seinen, wie er war und ist; und weil er mm, in der Gestalt
seineö getreuen Dienerß, wieder erscheint, sv verfolgen ihn die
Abgefallenen, Fürsten, Herrscher, Priester und Volk. Nicht alß
ein Übeltäter leidet er von euch, ihr Magistrate, sondern weil er
nicht abgefallen ist und gegen daß Treiben der Welt und daß
Böse auftritt. EZ ist immer so gewesen, daß, wo die oerderbte
Natur den Samen Gottes unterdrückte, die Verderbten suchen die,
in denen dieser Same ausging, gefangen zu nehmen .... Wie
Christuö daö, maß man einem der Geringsten erweist, als ihm
getan ansieht (Matth. 5, 25), also siehet er auch daß, wa?. man
ihnen nicht tut, als ihm nicht getan an. Wenn ihr nun soweit
geht, daß ihr nicht einmal anderen gestatten wollt, einen gefangenen
Bruder in seinen Leiden zu besuchen, so werdet ihr in den feurigen
Pfuhl, der mit Schwefel brennt, geworfen (Offb. 19, 20). Der
Herr ist gekommen, die Berge zu stürzen und zu Staub zu zer-
nialmen (Jes. 41, 15), und er wird rächen die Unterdrückung der
Gewissen seineö Volkes- an allen ungerechten Herrschern, Beamten
und Gesetzen. Er wird seinem Volke sein Gesetz geben nicht nach
dem, wat?. vor Augen ist, sondern nach Recht und Gerechtigkeit.
Man hat nun gesehen, wie eure Herzen voll Haß sind gegen die
Wahrheit Gotteö, die er durch sein von der Welt oerachteteö und
zum Spott ,,Quäker« genannteö Volk verkünden läßt. Jhr seid
ärger als die Heiden, die Pauluö inö Gefängniß warfen; denn
niemand hat damals seinen Freunden verboten, ihn zu besuchen,
Gkotge Fox. 6


% \picinclude{./080-089/p_s082.jpg} 
82 Kapitel Ill.
darum treten sie gegen euch als Zeugen auf. M ist offenbar
geworden, daß ihr denen gleich seid, die Christus töteten und die
Apostel gefangen nahmen unter dem gleichen Vorwand, nämlich
daß sie den Jrrtum Wahrheit und die Diener Gottes Gottes-
lästerer nannten. Aber das Gericht, das über euch kommen wird,
ist schrecklich, ihr ungerechten Magistrate und Priester und ihr
alle, die ihr mit Worten die Wahrheit bekennet, und doch die
Kraft der Wahrheit und die, die in der Wahrheit sind und für die
Wahrheit einstehen, verfobget. Gehet in euch, dieweil es Zeit
ist, und bedenket, was Jesaias 17 geschrieben steht!«
Geroase Benson
Anthony Pearson.
Bald darauf kam die Macht des Herrn über die Richter
und sie setzten mich frei. Kurz vorher war Anthony Pearson
mit dem Gouverneur in meinen Kerker gekommen um zu sehen,
wie ich behandelt werde. Sie fanden den Ort so gräulich und
den Geruch so schlecht, daß sie sich über die Magistrate entsetzten,
die solches von dem Kerkermeister geschehen ließen. Sie ließen
die Wärter in den Kerker kommen und sich für ihr Betragen
rechtfertigen. Den Unterkerkermeister, der so grob gewesen war,
sperrten sie darauf zu uns ins Gefängnis unter die Räuber.
Nachdem ich nun frei war, ging ich zu Thomas Bewley . . .
Dann ging ich auss Land und hatte viele große Versammlungen . . .
und tausende bekehrten sich zum Herrn Jesus Christus-.
Dann ging ich nach Westmorland . . . Durham, Hexhain . . .
Gilsland . . . nach Eumberland ..... Hier überall, sowie in
Northumberland, Laneashire und Yorkshire fanden große Be-
kehrungen statt, und was Gott gepflanzt hatte, wuchs und gedieh
unter dem Himmelsregen von oben und Gottes leuchtender Herr-
lichkeit, sodaß sich vieler Mund öffnete zum Lobe Gottes; ja: ,,aus
dem Munde der Unmitndigen und Säuglinge richtete er sich eine
Macht zu« (Psalm 8, Z).


% \picinclude{./080-089/p_s083.jpg} 
Kämpfe mit schwärmerischen Ranters und zehntengierigen Pticstertt usw. 83
Kapitel 711.
Kämpfe mit schwärmerischen Ranters und zelzntengietigen Priestern.
Fox in Wetstone verhaftet und vor Crounvell geschickt.
Die Priester und Frommen traten aufs neue mit ihren
Prophezeihungen gegen uns auf. Schon lange hatten sie vor-
ausgesagt, daß wir binnen eines Monats vernichtet sein werden;
hernach verlängerten sie die Frist auf ein halbes Jahr; als aber
auch diese Zeit längst um war, und wir im Gegenteil an Zahl
zunahmen, streuten sie aus, wir werden einander gegenseitig ver-
zehren. Es kam nämlich ost vor, daß nach den Versammlungen
manche, die einen weiten Heimweg hatten, bei Freunden blieben,
es waren ost mehr Leute als Betten vorhanden, so daß Viele auf
dem Heu übernachten mußten. Da wurden die »Frommen« von
der Furcht Eains gepackt; sie hatten Angst, daß, wenn wir ein-
ander zu Grunde gerichtet hätten, wir dann der Gemeinde zur
Last fallen und uns von ihr unterhalten lassen werden. Llls sie
aber sahen, wie der Herr den Freunden Segen und Gedeihen
gab, wie dem Abraham, ,,beim Acker und beim Korb, beim Gin-
gehen und beim Ausgehen, beim Aufstehen und beim Niederliegen«
(5. Mose 28), da erkannten sie die Ungerechtigkeit ihrer Prophe-
zeihungen, und daß man ,,umsonst flucht, wo der Herr segnet«
(4. Mose, 23). Als nach den ersten Bekehrungen die Freundes
den Hut nicht vor den Leuten abnahmen, einer einzelnen Person
nicht mit ihr, sondern mit ,,du« und ,,dich« antworteten, sich nicht
verneigten und nicht bei der Begrüßung schmeichelhafte Worte
gebrauchten und nicht die Art und Weise der Welt mitmachten, da ver-
loren viele von ihnen in ihren Geschäften die Kundschaft; man
scheute sich vor ihnen und wollte keine Geschäfte mit ihnen machen,
so daß eine Zeitlang die Freunde kaum ihr Brot verdienten. Aber
als die Leute sahen, wie treu und ehrlich die Freunde waren,
und daß ihr ja — ja und ihr nein — nein war; daß sie Wort hielten
im Verkehr und niemanden hintergingen noch betrogen, und wie
der Herr ihnen Segen und Gedeihen gab; wie ein Kind, das sie
schickten, um einen Einkauf zu machen, gerade so gut bedient
wurde wie sie selbst, da predigte das Leben und der Wandel der
Fretmde, und es traf das, was von Gott kam, in ihren Gewissen.
Nun wandelten sich die Dinge dermaßen, daß man beständig
fragen hörte: ,,Wo ist ein Krämer, ein Tuchhändler, ein Schneider,
 


% \picinclude{./080-089/p_s084.jpg} 
84 Kapitel 711.
ein Schuster, ein Handwerker, der Quäker ist?« Die Freunde
bekamen mehr Arbeit als manche andere Handwerker und betei-
ligten sich reger am geschäftlichen Verkehr. Nun schlugen die
gehässigen »Frommen« einen anderen Ton an und fingen an zu
murren: ,,Wenn wir diese Quäker gewähren lassen, so werden sie
unß den Handel deß ganzen Landeß an sich reißen.« Also tat
der Herr an seinem Volke, und es ist mein ernstlichster Wunsch-
daß alle, die seine heilige Wahrheit bekennen, in der E-rkenntnitz
bewahrt und durch den Geist und die Kraft in der Treue erhalten
bleiben mögen, erstlich gegen Gott, im Gehorsam in allen Dingen,
und dann gegen die Menschen, in Rechtschaffenheit und Gerech-
tigkeit in allem Verkehr; damit Gott der Herr verherrlicht werde
durch einen Wandel in Wahrheit und Heiligkeit, Gerechtigkeit und
Gottseligkeit .....
Die Priester in Newcastle, Kendal und anderen nördlichen
Gegenden waren sehr aufgebracht gegen unß. Einer, uamenß
Gilpin, der manchmal zu unß nach Kendal gekommen war, war
bald von der Wahrheit abgefallen und aus allerlei einfältige
Gedanken gekommen, und die Priester gebrauchten nun daß gegen
unö, wo sie nur konnten; aber die Kraft des- Herrn wars sie
alle darnieder. Der Herr vernichtete zwei der Verfolgung?-süchtigen
Richter von Carliöle und der dritte wurde einige Zeit darauf
seineß Amts entsetzt und verließ die Stadt.
Um diese Zeit wurde den Soldaten der Eid, den sie Oliwer
Eromwell schwören sollten, vorgelegt, und viele wurden entlassen,
weil sie im Gehorsam gegen Christus nicht schwören konnten.
Einer von diesen war John Stubbö, der bekehrt worden war
während meiner Gefangenschaft in Carliöle, und ein guter Soldat
im Kampfe de-3 Lamnieö und ein treuer Jünger Jesu geworden
ist. Er reiste Viel umher im Dienste dez Herm, in Holland,
Schottland, Jtalien, J-rland, Ägypten, Amerika. Und die Kraft
Gotteö bewährte ihn vor den Händen der Papisten, obgleich er
oft in großer Gefahr vor der Jnquisition war. Andere unter
den Soldaten jedoch, die wohl ihrer überzeugung nach bekehrt
worden waren, aber nicht zum Gehorsam gegen die Wahrheit
gelangten, schwuren den Eid Cromwellö: als diese später in Schott-
land waren, kamen sie in die Nähe einer Garnison; die dortige
Mannschaft glaubte eß seien Feinde und töteten sie ....
Der Herr trieb viele von denen, die er auöerlesen hatte, in


% \picinclude{./080-089/p_s085.jpg} 
Kämpfe mit schwärmerischen Ramcrs und zehntengierigeu Priestern usw. 85
seinem Weinberg zu arbeiten, nach Süden zu gehen und sich im
Dienste des Evangeliums nach den südlichen und westlichen Teilen
des Landes zu verteilen; so gingen Francis Howgill und Edward
Burrough nach London, John Camm und John Audland nach
Bristol, Richard Hubberthorn und George Whiteheads) gegen
Norwich, Thomas Holmes 1) nach Wales und andere nach anderen
Richtungen; etwa sechzig Diener hatte der Herr ausersehen und
aus dem Norden in die Verschiedenen Teile des Landes gesandt.
Um die Zeit singen Rice Jones oon Nottingham, ein früherer
Baptist und jetzt Ranter, und seine Anhänger an, gegen mich
zu prophezeien; sie sagten, ich hätte jetzt meinen Höhepunkt
erreicht und werde nun bald tief fallen .... Aber seine und
der Seinen Weissagungen erflillte sich an ihnen selber; denn bald
daraus fielen sie ganz auseinander und viele von ihnen wurden
Freunde und blieben es; und durch des Herm mächtige Macht und
Wahrheit vermehrten sich die Freunde .... Rice Jones da-
gegen leistete den Eid und war also dem Gebot Christi unge-
horsam. Viele falsche Propheten haben sich gegen mich erhoben,
aber der Herr hat sie alle vernichtet und wird auch ferner alle
vernichten, die sich gegen seinen gesegneten Samen erheben ....
Ju der Nähe von Kidsley-Park stieß ich aus eine Schar
Ranter; aber die Kraft des Herrn hielt sie drunten. Von da
ging ich in die Gegend des Peak zu Thomas Hammersley,
wohin die Ranter dieser Gegend kamen und Viele angesehene
,,Fromme«. Die Ranter traten gegen mich auf und fingen an
zu schwören; als ich ihnen deswegen Vorstellungen machte, oer-
suchten sie, Schriftstellen zu bringen und sagten, Abraham, Jakob
und Joseph haben geschworen und die Priester und Moses und
die Engel. Jch erwiderte: ,,ich gebe zu, daß alle diese es taten,
wie die Schrift es berichtet; Christus aber sagt: ,,schwöret nicht!«
Und Christus ist das Ende der Propheten und des alten Priester-
tums und des Gesetzes Moses und regiert über das Haus Jakobs
und Josephs, und er sagt: ,,ihr sollt nicht schwören.« Und als
Gott den Erstgeborenen in die Welt sandte, sagte er: »alle Engel
sollen ihn anbeten« (Gbr. 1, 6), also diesen Christus, der sagte, ihr
sollt nicht schwören. Und was die Begründung anbelangt, welche
1) George Whitehead und Thomas Holmes, zwei eisrige Quäkerprediger.
(Näheres s. Weingarten a. a. O.)


% \picinclude{./080-089/p_s086.jpg} 
86 Kapitel 711.
die Menschen für daß Schwören geltend machen, um ihre Strei-
tigkeiten zu Ende zu bringen, so hat Ehristu-5, der gesagt hat,
ihr sollt nicht schwören, den Teufel und seine Werke, deren eines
eben daö Streiten ist, vernichtet. Und Gott sagt: »dieS ist mein
lieber Sohn, an dem ich Wohlgesallen habe, ihn sollt ihr hören.«
(Mark. 9, 7). Also soll man den Sohn hören, der daß Schwören
verbietet. Und der Apostel Jakobus, welcher den Sohn hörte,
und ihm folgte und ihn verkündete, verbietet daß Schwören,
Jakobuz 5, 12.« Die Kraft des Herm erfaßte sie und sein
Sohn und seine Lehre beherrschten sie. Das Wort des Lebenß
wurde reich und herrlich unter ihnen verkündet an dem Tage,
und viele wurden bekehrt.
Diesem Thomas Hammersley wurde einmal gestattet, an
einem Geschworenengericht als- Geschworener zu amtieren ohne
einen Eid abzulegen; als- er dann, als- Vorsitzender, sein Gut-
achten abgab, erklärte der Richter, er sei nun doch schon seit Vielen
Jahren Richter, aber er habe noch nie ein so redlicheß Gutachten
gehört, als daß von diesem Quäker! ES ließe sich noch viel
derartiges berichten, wenn die Zeit reichen würde. Die herrliche
Wahrheit des Herrn goß sich auß; ihr gebühret Preiß und Ehre
ewiglich!
Aus der Durchreise durch Derbyshire besuchte ich überall
Freunde, bis ich nach Swannington kam; hier war eine große
Versammlung, zu der Baptisten, Ranter und viele andere ,,Fromme«
kamen. ES hatte viele Zusammenstöße mit ihnen und den Priestern
der Stadt gegeben. Von überallher kamen Freunde zu dieser
Versammlung, so John Audland, Franciö Howgill, Edward Pyot
von Bristol und Edward Vurrough aus London und es wurden
viele bekehrt. Die Ranter machten Störungen und benahmen sich
sehr unverschämt; aber schließlich kam die Macht dez Herm über
sie und sie unterlagen. Am darauffolgenden Tage kam Jacob
Bottomley, ein großer Ranter von Leieester; aber die Kraft des
Herrn überwältigte ihn. So auch einen Priester. Wir ließen
den Rantern sagen, sie sollten kommen und ez mit ihrem Gott
versuchen; sie kamen in Haufen und waren sehr wild und sangen
und pfifsen und tanzten; aber die Kraft dez Herrn überwältigte
sie so, daß viele von ihnen bekehrt wurden.
Von hier ging ich nach Twycroß, wohin auch Ranter kamen
und vor mir sangen und tanzten; aber in der Furcht deß Herrn


% \picinclude{./080-089/p_s087.jpg} 
Kämpfe mit schwärmerischen Ranters und zehntengierigen Priestern usw. 87
trieb es mich, sie zu tadeln; und die Kraft des Herrn kam über sie,
so daß einige von ihnen bekehrt wurden und den Geist Gottes
aufnahmen. Sie sind tüchtige Leute geworden, die rechtschaffen
in der Wahrheit Christi leben und wandeln. Jch ging zu Anthony
Brickleh in Warwickshire, wo eine große Versammlung war;
mehrere Baptisten und andere kamen und lärmten; aber die Kraft
des Herrn kam über sie.
Hierauf ging ich nach Drayton in Leicestershire, um meine
Verwandten zu besuchen. Kaum war ich angekommen, so ließ
der Priester Nathanael Stephens, der noch einen andern Priester
hatte kommen lassen und die Umgegend von meinem Kommen
benachrichtigt hatte, mich zu sich holen, denn sie konnten nichts
machen, ehe ich kam. Da ich drei Jahre meine Angehörigen
nicht gesehen hatte, so wußte ich nichts von ihren Absichten. Jch
ging nun aus den Platz des Turmhauses, wo die beiden
Priester waren, und wo sich eine Menge Leute versammelt hatten.
Als ich kam, wollten die Leute, daß ich ins Turmhaus gehe; ich
fragte sie, was ich dort tun solle; sie erwiderten, Stephens
könne die Kälte nicht ertragen; ich sagte, er könne sie so gut
ertragen wie ich. Zuletzt begaben wir uns in einen großen Saal;
Richard Farnsworth war auch dabei; wir hatten einen großen
Disput mit den Priestern über ihren Wandel, und daß sie so
sehr das Gegenteil von dem seien, was Christus und die Apostel
gewesen. Die Priester wollten wissen, wo die Zehnten verboten
oder aufgehoben seien; ich wies es ihnen nach im 7. Kap. des
Hebräerbrieses, wo nicht nur die Zehnten, sondern das ganze
Priestertum, das Zehnten annahm, aufgehoben war und das
Gesetz, nach welchem das Priestertum eingesetzt und die Zehnten
erhoben wurden. Hierauf hetzten die Priester das Volk zur Frech-
heit und Roheit gegen uns auf. Jeh hatte Stephens seit seiner
Kindheit gekannt und konnte ihnen darum aufdecken, was für
eine Art von Mensch er sei und was hinter seinen Predigten
stecke, und wie er, wie alle Priester, die Verheißungen aus den
alten Menschen, der sterben muß, bezog; dann zeigte ich ihnen,
daß die Verheißungen vielmehr dem Samen galten, nicht den
vielen Samen, sondern dem einen Samen, Christus, der derselbe
ist in Mann und Weib; denn alle müssen wiedergeboren werden,
ehe sie ins Reich Gottes eingehen können. Er erwiderte mir
daraus, ich sollte nicht in der Weise richten; ich entgegnete ihm,


% \picinclude{./080-089/p_s088.jpg} 
88 Kapitel 711.
,,der Geistliche richtet alle?-« (1. Cor. 2, 15); er gab zu, daß dietz
genau der Schrift gemäß sei; dann aber fuhr er fort: ,,ihr Nach-
barn, das ist die Sache: George Fox ist zum Lichte der Sonne
gekommen und nun möchte er mein Sternenlicht au8löschen.« Jch
erwiderte: ,,ich will nicht das- kleinste Maß von dem, was einer
von Gott hat, in jemand unterdrücken, noch viel weniger sein
Sternenlicht auölöschen, wenn eS ein wirkliches Sternenlicht ist,
ein Licht vom Mvrgenstern.« Dann erklärte ich ihm, daß, wenn
er etwaß von Gott oder Ehristuß empfangen habe, er umsonst
predigen müsse und nicht Zehnten nehmen von den Leuten sür
seine Predigten, da er ja gesehen habe, wie Christu;3 seinen
Jüngern befohlen habe, umsonst zu geben, wie sie es- umsonst
empfangen hätten. Jch schärste ihm also ein, nicht mehr für
Zehnten und Lohn zu predigen. Aber er sagte, dem werde er
sich nicht fügen. Die Leute fingen an, unverschämt zu werden,
und wir brachen darum auf. Dennoch waren etliche an dem
Tage der Wahrheit zugetan worden. Ghe ich fort ging, sagte
ich ihnen, daß ich im Sinn habe, nächste Woche, so Gott wolle,
wieder in der Stadt zu sein. Jn der Zwischenzeit ging ich in
die Umgegend und hielt Versammlungen, und nach acht Tagen
kam ich wieder zurück. Der Priester hatte für diese Zeit 7 Priester
kommen lassen, um ihm zu helfen; und Stephenß hatte in einem
Gottesdienst am Markttage in Adderßton angezeigt, daß an dem
und dem Tage ein Diöput mit mir stattfinden werde. Jch wußte
nichts davon und hatte nur gesagt, ich werde über acht Tage
wieder in der Stadt sein. Die acht Priester hatten etliche hundert
Leute versammelt, meist aua der Umgegend und wollten, ich sollte
ins Turmhauö gehen; aber ich wollte nicht hingehen, sondem
ich ging aus einen Hügel und redete von dort zum Volk .....
GS kamen einige Unverschämte und nahmen mich aus die
Arme und trugen mich unter die Türe dee Turmhauseß, in der
Absicht, mich mit Gewalt inß Turmhauß zu bringen; da aber
die Tür geschlossen war, purzelten sie alle übereinander; und ich
lag zu unterst. So bald ich konnte, kroch ich hervor und ging wieder
auf den Hügel; nun schleppten sie mich biö an die Mauer
dez Turmhauseö und setzten mich auf eine Art Steinbank; alle
Priester waren auch herbeigelaufen und standen mitten unter dem
Volk herum, und alle schrieen: »Beweise, beweise!« Jch sagte,
ich hörte nicht auf ihre Stimmen, denn eß seien die Stimmen von


% \picinclude{./080-089/p_s089.jpg} 
Kämpfe mit schwiirmetischen Ramerz und zehntengierigen Priestern usw. 89
Mietlingen und Fremdlingen. Sie schrieen wieder: ,,Beweise,
beweise!« Ich wies aus Johannes, wo sie sehen können, wac-
Christuß zu ihreßgleichen sage, nämlich: »Jch bin der gute Hirte,
der sein Leben gibt für seine Schafe, der Mietling aber flieht,
wenn der Wolf kommt.« Jch schlug ihnen vor, ihnen zu beweisen,
daß sie solche Mietlinge seien; darauf rissen die Priester mich
wieder herunter und stiegen selber alle auf Steinbänke
an der Mauer des Turmhauseß. Da fühlte ich, wie Gottes
mächtige Kraft über alle kam und sprach zu ihnen: ,,Wenn ihr
mir Gehör schenken wollt und mich ruhig anhören, so will ich
euch (muß der Schrift zeigen, warum ich die acht Priester oder
Lehrer, die vor mir stehen, nicht anerkenne und überhaupt keine
Mietlingßlehrer der Welt.« Priester und Volk erklärten sich bereit
zu hören. Da zeigte, ich ihnen auß den Propheten Jesaja,
Jeremia, Ezechiel, Micha, Maleachi und anderen, daß sie in den
Fußstapfen derer wandeln, gegen die Gott seine Propheten ge-
sandt hatte ....
Dann als ich an das neue Testament kam, zeigte ich ihnen,
daß sie wie die Hohenpriester und Schriftgelehrten seien, und
wie die Pharisäer, gegen die Christuö wehe! schrie (Matth. 23).
Jsndem ich in dieser Weise außführlich aus der Schrift bewiesen
hatte, warum sie den Pharisäern gleichen, . . . und sie vor allem
Volk unter die Pharisäer, falschen Propheten und Verführer
gerechnet hatte und gezeigt, wie ihresgleichen von den wahren
Propheten und Christus verdammt werden, wieß ich sie aus daß
Licht Jesu Christi hin, daß einen jeden, der in die Welt kommt,
erleuchtet (Joh. 1, 9), und durch dieseß Licht könnten sie erkennen,
, ob daß Gesagte wahr sei. Sie mochten nichtß davon hören, daß
ich sie aus das- Göttliche in ihnen, auf daß Licht Jesu Christi
hinwieö. Bis dahin waren sie alle ruhig gewesen, nun aber
rief einer der »From1nen«: ,,Wirst du denn nie fertig, Fox?«
Ich erwiderte, ich sei nun bald fertig; ich fuhr noch eine Weile
fort, biß ich fühlte, daß ich an ihnen getan hatte, maß ich mußte
in der Kraft dez Herrn ...... A13 ich fertig war, flüsterten
die Priester untereinander, und Priester Stephenö kam zu mir
und verlangte, daß mein Vater und mein Bruder und ich mit
ihm beseite kommen, damit er mit uns reden könne; und die
anderen Priester mußten daß Volk davon abhalten unß nachzu-
kommen. Jch ging sehr ungern mit ihm, aber da daö Volk schrie:



% \picinclude{./090-099/p_s090.jpg} 
90 Kapitel 711.
,,Geh, George, geh nur,« so fürchtete ich, daß, wenn ich nicht
ginge, man sage, ich sei meinen Eltern ungehorsam; so ging ich,
und die übrigen Priester wollten daß Volk abhalten, aber es
gelang ihnen nicht, denn da alle uns- hören wollten, wurden wir
ganz umringt. Jch fragte den Priester, maß er zu sagen habe?
er antwortete, wenn er nicht aus dem rechten Wege sei, so sollte
ich für ihn beten; und wenn ich nicht auf dem rechten Wege sei,
so wollte er für mich beten; und er wolle mir oorsagen, was ich
für ihn beten solle. Ich erwiderte ihm: ,,eS scheint, daß du nicht
einmal weißt, ob du auf dem rechten Wege bist; ich aber weiß,
daß ich auf dem rechten Wege bin, Jesus Christus, in welchem
du nicht bist, und du wolltest mir vorsagen, wie ich zu beten
habe, und verwirfst doch daß Common-Prayerbook so gut wie ich,
und ich verwerfe dein Geplapper ebenfalls. So du willst, daß ich
nach etwas Hergesagtem für dich bete, heißt daß nicht, die Lehre
der Apostel mißachten und ihr Beten im Geist, der die Worte
eingibt?« Hier fingen die Leute an zu lachen; mich aber trieb
ez, weiter zu ihm zu reden. Nachdem ich ihm gesagt, was
mir zu sagen oblag, und daß ich, so Gott wolle, über acht Tage wieder
in der Stadt sein werde, gingen wir fort. Die Priester machten,
daß sie fort kamen und viele wurden gewonnen, denn die Kraft dez
Herrn kam über alle. Wenn sie schon meinten an diesem Tage
der Wahrheit geschadet zu haben, war doch mancher gewonnen
worden, und viele, die schon früher gewonnen worden, wurden durch
daß, was an jenem Tage geschehen, bestärkt, und etz gab den
Priestern einen Stoß. Mein Vater, obgleich er ein Anhänger der
Priester war, war so befriedigt, daß er mit seinem Stock auf die
Erde schlug und sagte: »wahrlich, ich sehe, daß wer willenß ist,
bei der Wahrheit zu bleiben, dem wird sie durchhelsen« .....
Darauf zog ich wieder umher und hielt Versammlungen und
kam nach Swannington, wohin auch wieder Soldaten kamen;
aber die Versammlung war ruhig, die Macht Gotteö war
über allen, und die Soldaten störten mich nicht. Darauf ging
ich nach Leicester und Whetstone. Dahin kamen siebzehn Soldaten
auö Oberst Hackerß Regiment, mit ihrem Anführer, und führten
mich, gerade vor Beginn der Versammlung, hinweg, obgleich die
Freunde, die von allen möglichen Orten hergekommen waren,
schon anfingen sich zu versammeln. Ich sagte dem Vorgesetzten,
er solle wenigstenß die Freunde in Ruhe lassen, ich wolle für sie


% \picinclude{./090-099/p_s091.jpg} 
Kämpfe mit schwärmerischen Ranters und zehutengierigen Priestern usw. 91
alle haften; so nahmen sie denn mich und ließen die andern in
Ruhe, ausgenommen Alexander Parken!) der mit mir kam. Am
Abend brachten sie mich vor Oberst Hacker; sein Major, seine
Hauptleute und viele seiner Leute waren zugegen und wir gaben
auöfiihrlich Auskunft über die Priester und über die Versammlungen,
denn ez ging damals gerade daß Gerücht von einer Verschwörung
gegen Oliver Eromwell. Jch hatte lange Grörterungen über das
Licht Christi, daß einen jeden, der in die Welt kommt, erleuchtet
(Joh. 1, 9). Oberst Hacker fragte, ob e-3 dieses Licht auß Ehristuß
gewesen sei, daß den Judaö dazu geführt habe, seinen Herrn zu
verraten und sich darnach zu erhängen? Jch sagte ihm: ,,nein,
das war der Geist der Finsterniß, der Christuz und sein Licht
haßte.« Darauf sagte Hacker, ich solle nach Hause gehen und dort
bleiben, und nicht überall zu den Versammlungen gehen. Jch sagte
ihm, ich sei ein ganz harmloser Mensch und habe nichts mit Ver-
schwörungen zu tun, vielmehr verabscheue ich solcheß. Sein Sohn
Needham sagte: »Vater, dieser Mensch hat nun schon lange ge-
herrscht, eö ist Zeit, daß man ihn unschädlich mache.« q Jch fragte
ihn, ,,warum, was habe ich getan? oder wem habe ich je etwas
zu leide getan? ich bin in dieser Gegend geboren und aufge-
wachsen, wer kann mir irgend etwaß Böses nachsagen seit meiner
Kindheit?« Darauf fragte mich Oberst Hacker nochmals-, ob ich
nach Hause gehen wolle und dort bleiben? Ich antwortete ihm,
ich würde mich ja mit einem solchen Versprechen schuldig bekennen,
wenn ich nach Hause ginge und auö meinem Hause ein Gefängniß
machen wollte; und ginge ich dann doch zu den Versammlungen, so
würde ez heißen, ich sei dem Befehl ungehorsam. Jch erklärte
ihnen, ich gehe auf dee- Herrn Geheiß zu den Versammlungen,
darum könne ich mich ihren Vorschriften nicht fügen; aber wir
seien ein sriedlichez Volk. ,,Gut denn,« sagte Oberst Hacker, »ich will
euch zum Lord Protektor schicken, durch Hauptmann Drury, einen
aus seiner Leibgarde.« Die Nacht über wurde ich alß Gefangener
gehalten und am folgenden Morgen um sechö Uhr dem Haupt-
mann Drury übergeben. Ich wünschte vor dem Fortgehen noch
mit Oberst Hacker zu reden, er ließ mich vor sein Bett kommen und
drang sogleich wieder in mich, nach Hause zu gehen und keine
Versammlungen zu halten; ich erklärte ihm, ich könne mich dem
1) Alexander Parker, ein Mann von vornehmer Herkunft, reiste viel im
Dienst des Quäkertnmö und schrieb viele Bücher und Briefe zu seiner Verbreitung. NT


% \picinclude{./090-099/p_s092.jpg} 
92 Kapitel 711.
nicht fügen, sondern müsse meine Freiheit haben. ,,Dann,« sagte
er, ,,müßt ihr vor den Protektorcks Hierauf kniete ich an feinem
Bett nieder und betete zum Herm, ihm zu vergeben, denn er war
ein Pilatus, auch wenn er seine Hände gewaschen hätte; und ich
flehte zum Herrn, daß, wenn der Tag seiner Prüfung und Heim-
suchung komme, er sich dessen, was ich ihm gesagt, erinnern möge.
Gr war eben aufgehetzt von Priester Stephens und den andern
Priestern und ,,From1nen«, die darin ihre Bosheit ausließen, weil
sie mich durch ihr Argument nicht hatten überwinden können
und dem Geiste Gottes in mir nicht hatten widerstehen können;
darum hatten sie nun die Soldaten geschickt, um mich zu greifen.
Als später dieser Oberst Hacker im Gefängnis in London
war, wurde es ihm ein oder zwei Tage vor seiner Hinrichtung
in Erinnerung gebracht, wie er an den Unschuldigen gehandelt
hatte, und er gedachte daran und bekannte es Margaret Fell;
und es bedriickte ihn. Nun konnte sein Sohn, der damals
gesagt hatte, ich habe genug geherrscht, es sei Zeit, mich fort zu
schaffen, zusehen, wie sein Vater sottgeschasft wurde, als man ihn
erhängte in Tyburn.
Jch wurde nun von Hauptmann Drury als Gefangener von
Leieester fortgebracht. Als wir nach Harborough kamen, fragte
er mich, ob ich heimgehen wolle und 14 Tage dort bleiben? Er
versprach mir die Freiheit, wenn ich weder Versammlungen halten
noch zu solchen gehen wolle. Jch erwiderte ihm, ich könne nichts
dergleichen versprechen; er fragte und versuchte mich wiederholt
auf dem Wege in derselben Weise, und immer gab ich ihm die-
selbe Antwort. So brachte er mich nach London und quartierte
mich in Mermaid ein; unterwegs trieb es mich, die Leute zu J
warnen vor dem Tag des Herrn, der über sie kommen werde.,
Nachdem Hauptmann Drury mich untergebracht, verließ er mich
und ging zum Protektor, um Bericht über mich zu erstatten. Als
er zurückkam, sagte er, der Protektor verlange, daß ich kein
mörderisches Schwert gegen ihn oder die Regierung gebrauche,
und daß ich dies in beliebigen Worten schristlich erklären und mit
meiner Unterschrift versehen solle. Jch antwortete Hauptmann
Drury nur wenig; aber am nächsten Morgen trieb mich der Herr,«
ein Schreiben an den Protektor auszusetzen, in dem ich vor dem«
Angesicht Gottes des Herrn erklärte, daß ich das Tragen eines
mörderischen Schwertes oder irgend einer anderen äußeren Waffe


% \picinclude{./090-099/p_s093.jpg} 
Kämpfe mit schwiirmerischen Routers und zehntengierigen Priestern usw. 93
verabscheue, und daß ich von Gott gesandt sei, Zeugnis abzu-
legen gegen jegliche Gewaltttitigkeit und gegen die Werke der
Finsternis; und um die Leute von der Finsternis zum Licht zu
bringen und vom Kriegen und Streiten zum Evangelium des
Friedens. Nachdem ich geschrieben, was der Herr mir eingegeben
hatte, setzte ich meinen Namen darunter und übergab es Haupt-
mann Drury, damit er es Oliver Eromwell gebe, was er auch
tat. Rath einiger Zeit brachte mich Hauptmann Drury vor den
Protektor in Whitehall; es war an einem Morgen, ehe er ange-
kleidet war, und einer, namens Harvey, der sich auch eine Zeit
lang zu den Freunden gehalten hatte aber ungehorsam geworden
war, bediente ihn. Als ich eintrat, trieb es mich zu sagen:
,,Friede sei mit diesem Hause,« und ich ermahnte ihn, in der Furcht
Gottes zu bleiben, damit er Weisheit von ihm empfangen möge,
daß sie ihn leite; und daß er alle Dinge, die in seiner Hand
seien, zu Gottes Ehre regiere. Jch redete lange mit ihm über
die Wahrheit und über die Religion, er zeigte sich sehr verständig;
aber er sagte, wir zankten mit den Priestern, die er Diener Gottes
nannte. Jch entgegnete ihm, ich zanke nicht mit ihnen, sondern
sie mit mir und mit meinen Freunden. ,,Aber«, sagte ich, ,,wenn
wir die Propheten und Apostel anerkennen, so können wir solche
Lehrer, Propheten und Hirten, gegen welche die Propheten und
Christus auftraten, nicht gut heißen, sondem wir müssen auch
gegen sie auftreten, durch denselben Geist und dieselbe Krast.«
Ferner zeigte ich ihm, daß die Propheten, Christus und die
Apostel umsonst predigten und gegen die auftraten, welche es
nicht umsonst taten, sondern um schändlichen Gewinnes willen
und die um Geld wahrsagten und um Lohn lehrten (Micha 3, 11),
gierig und geizig waren und nie genug bekamen; und daß die,
welche den Geist Christi und der Apostel und Propheten haben,
auch jetzt noch gegen das alles austreten müssen, wie jene damals.
Während ich sprach, sagte er mehrmals, es sei sehr gut, es sei
wahr. Ich sagte ihm, daß alle, die sich Christen nennen, die
H Schrift haben, aber nicht alle die Kraft und den Geist, welche
die hatten, die die Schrift geschrieben, und dies sei der Grund,
warum sie nicht in der Gemeinschaft mit dem Vater und dem
Sohne seien, noch mit der Schrift, noch unter einander. Jch
redete noch über vieles andere mit ihm; da aber Leute herein
kamen, zog ich mich ein wenig zurück; als ich mich anschickte fort


% \picinclude{./090-099/p_s094.jpg} 
94 Kapitel VU.
zu gehen, faßte er mich bei der Hand, und sagte mit Tränen in
den Augen: ,,Komm wieder zu mir, denn wenn du und ich nur
eine Stunde im Tage beisammen wären, so würden wir einander
näher kommen«; und er fügte bei, er wünsche mir so wenig etwaß
Böseö als seiner eigenen Seele. Jch sagte ihm, wenn er etz tun
würde, so würde er damit seiner eigenen Seele schaden; und ich
bat ihn, aus die Stimme Gottes zu hören, auf daß er in seiner
Wei?-heit bleiben möge und ihm gehorchen; wenn er ez tue, so
werde er vor Hartherzigkeit bewahrt bleiben; wenn er aber
nicht auf Gottes Stimme höre, so werde sein Herz verhärtet
werden. E-r sagte, dietz sei wahr; daraus ging ich hinauß, und
Hauptmann Drury kam hinter mir drein und teilte mir mit, sein
Lord Protektor sage, ich sei frei und könne gehen, wohin ich
wolle. Darauf wurde ich in einen großen Saal geführt, wo
die Kammerherrn des Lord Protektor zu speisen pflegten; ich fragte,
warum ich hierher geführt werde? sie sagten, es geschehe auf
Befehl des Protektor, damit ich mit ihnen speise. Jch hieß sie,
dem Protektor sagen, daß ich nicht von seinem Brote esse, noch
von seinen Getränken trinke. Al?-’ er dies hörte, sagte er: ,,nun
sehe ich, daß ein Volk entstanden und heroorgetreten ist, welcheß
ich nicht zu gewinnen vermag, weder durch Gaben, noch durch
Ehren, noch Stellen, während mir dies bei allen anderen Sekten
und Menschen gelingt«, worauf man ihm entgegnete, daß wir
ja daß Eigene hingeben und darum kaum nach dem Seinigen
trachten würden ....
Jch begab mich nach London, wo wir große und mächtige
Versammlungen hatten. Der Zudrang war so groß, daß ich fast
nicht hinein konnte, und die Wahrheit breitete sich ungeheuer auß.
Thomaö Aldam und Robert Craven und viele Freunde kamen
nach London, um nach mir zu sehen; aber Alexander Parker
blieb bei mir.
Nach einiger Zeit ging ich wieder nach Whitehall und ez
trieb mich, den Tag de,8 Herrn unter ihnen zu verkünden und
daß der Herr gekommen sei, sein Volk selbst zu lehren, und ich
predigte sowohl den Osfizieren alö denen von der Garde Oliver?-.
Aber ein Priester widersprach, alß ich daß Wort des Herm
verkündete; denn Oliver hatte verschiedene Priester um sich, und
dieser war ein Neuigkeitökrämer, ein häßlicher Priester, ein hinter-
listiger, mißgünstiger Mann; ich sagte ihm, er solle Buße tun


% \picinclude{./090-099/p_s095.jpg} 
Kämpfe mit schwärmerischen Ranterö und zehntengierigen Priestern usw. 95
und er setzte in der darauf folgenden Woche in seine Zeitung,
ich sei in Whitehall gewesen und habe dort einem Diener Gotteß
gesagt, er solle Buße tun. Als ich wieder dorthin kam, traf ich
ihn wieder, und viele Leute schatten sich um unö. Ich bewiez
dem Priester, daß er in verschiedenen Dingen gelogen habe, und
er mußte schweigen. Gr schrieb in der Zeitung, ich habe silberne
Knöpfe; was: falsch war, denn sie waren bloß auß Blech. Ferner
schrieb er, ich lege den Leuten Bänder um die Arme, damit sie
mir folgen; daß war wieder gelogen, denn ich hatte in meinem
ganzen Leben nie Bänder getragen oder gebraucht. Drei Freunde
gingen hin, um den Priester zur Rede zu stellen und ihn zu
fragen, woher er diese Dinge habe; er sagte, eine Frau habe es
ihm gesagt; und wenn sie wieder kommen, so wolle er ihnen ihren
Namen sagen. Alz sie wieder kamen, sagte er, ez sei ein Mami
gewesen, aber er sage den Namen nicht, wenn sie wieder kommen,
wolle er ihn. dann sagen. Alß sie daß drittemal kamen, sagte
er ihn wieder nicht, behauptete aber, wenn ich erkläre, daß alleß
nicht wahr sei, so wolle er etz in die Zeitung setzen. Als darauf
die Freunde ihm diese Erklärung brachten, so wollte er sie doch
nicht aufnehmen, sondern wurde zornig. So handelte dieser
infame Lügenschmied, um der Wahrheit zu schaden und um die
Leute gegen die Freunde und die Wahrheit einzunehmen, wovon
ein außführlicher Bericht in einem Buche, daß bald darauf ge-
druckt wurde, kann ersehen werden. Diese liignerischen Priester
waren Jndependenten, wie die zu Leieester; aber deß Herrn
Kraft oernichtete alle ihre Lügen, und viele kamen dazu, die
Schlechtigkeit der Priester einzusehen. Der Herr dez Himmelö
brachte mich durch seine Kraft durch alles- hindurch, und seine
herrliche Kraft tat sich kund im Lande, so daß in dieser Zeit
viele Freunde getrieben wurden, umher zu ziehen, um das ewige
Evangelium zu verkünden, in allen Teilen detß Landes und auch
in Schottland; und die Herrlichkeit des Herrn erschien allen zu
seiner ewigen Ehre ..... ES fanden große Bekehrungen in
London statt und auch mehrere im Hause deZ Protektorß uf in
seiner Famlie; ich versuchte zu ihm zu gehen, aber ich be am
keinen Zutritt, die Wachen waren so unfreundlich.
Die Preßbyterianer, Jndependenten und Baptisten waren sehr
erzürnt, denn viele bekehrten sich zum Herrn Jesuß Christus und
hörten seine Lehre. Sie empfingen seine Kraft und fpürten sie


% \picinclude{./090-099/p_s096.jpg} 
96 Kapitel 7111.
in ihren Herzen, und das trieb sie, gegen die übrigen auf-
zutreten.
Kapitel Vlll.
Brief an den Papst. Die Studenten von Cambridge. Die Qniiter
in der Bibel. Wachsende Entfremdung von Cromwell.
Es kam über mich vom Herrn, ein kurzes Schreiben auszu-
setzen und zu verbreiten, als Grmahnung an den Papst und alle
Könige und Herrscher von ganz Europa:
,,Freunde,
Jhr Häupter und Obersten, ihr Könige und Fürsten alle,
verfolget nicht in Erbitterung und Eifer die Lämmer Christi; wendet
euch nicht ab, wenn Gottes Stimme, seine innige Liebe und Barm-
herzigkeit aus der Höhe euch ruft, auf daß nicht sein Ann und
seine Macht, die jetzt die Welt ergriffen haben, euch unversehens
erfassen. Sie kehrt sich gegen die Könige, und die Weisen werden
weichen müssen, und ihre Krone wird zu Staub werden; und sie
werden erniedrigt und dem Erdboden gleich gemacht werden. Der
Herr wird König sein und wird die Krone dem geben, der seinen
Willen tut. Die Zeit ist gekommen, daß Gott der Herr Himmels
und der Erde die Stolzen entlarven wird und ihren Ruhm stürzen.
Jhr, die ihr Christus bekennet und liebet doch eure Feinde nicht,
sondern nehmet im Gegenteil seine Freunde gefangen, ihr zeiget
damit, daß ihr nicht in dem Leben seid, das aus ihm kommt, ihr.
liebet Christus nicht, wenn ihr nicht seine Gebote haltet. Des
Herrn Zorn fängt an zu brennen, und sein Feuer verbreitet
sich, um die Böfewichter zu zerstören, und es wird kein Zweig
noch Reis übrig lassen. Die so ihren Wandel nicht mehr in
Gott haben, sind nicht mehr in jenem Geist, der die Schrift ein-
gegeben hat, und nicht mehr im Lichte, damit Christus sie alle
erleuchtet hat .... Darum seid schnell zuhören, schnell zu reden,
aber langsam zu verfolgen (Jak. 1, 19); denn der Herr führt nun
sein Volk aus den Wegen der Welt zu Christus dem wahren
Weg, und von allen weltlichen Kirchen zu der Kirche, die in ihm,
dem Vater Jesu Christi, ist, und von allen Lehrern der Welt,
um selber ihr Lehrer zu sein durch seinen Geist; von den irdischen
Bildnissen, zum Gbenbilde seiner selbst; und von den irdischen


% \picinclude{./090-099/p_s097.jpg} 
Vries an den Papst. Die Studenten von Cambridge usw. 97
Kreuzen aus Holz und Stein, zu der Kraft des Kreuzes Christi.
Denn alle diese Bilder und Kreuze sind ein Absall von Gott und
seiner Kraft und dem Kreuz Christi, welches nun die Welt richten
wird und alles niederwersen, was ihm entgegen ist; seine Macht
hat kein Ende.
Lasset solches die Könige von Frankreich und von Spanien
und den Papst wissen, damit sie alles prüfen und das Gute behalten;
1md sie sollen vor allem prüfen, ob sie nicht den Geist dämpsten
(1. Thess. 5, 19), denn der große Tag des Herrn ist über die
Bosheit und Gottlosigkeit und Ungerechtigkeit der Menschen
gekommen, und der Herr wird ,,durchs Feuer richten und durch
sein Schwert alles Fleisch« (Jes. 66, 18). Und die Wahrheit
und die Krone der Ehren und das Szepter der Gerechtigkeit
werden erhöht werden; und das Göttliche, das in einem jeden
ist, auch wenn er davon abgesallen ist, wird hiervon Zeugnis
geben. Christus ist als Licht in die Welt gekommen und erleuchtet
einen jeden, der in die Welt kommt, damit dadurch alle zum
Glauben kommen. Und wer das Licht, womit Christus ihn
erleuchtet, spürt, der spüret Christus in seinem Jnnern und das
Kreuz Christi, diese Kraft Gottes; der brauchet kein hölzernes oder
steinernes Kreuz, um an Christus und sein Kreuz gemahnt zu
werden; denn es ist selber die Kraft Gottes, welche sich ihm
innerlich kund tut.« G. F.
Ferner trieb es mich, einen Brief an den Protektor zu
schreiben, um ihn zu ermahnen, aus das große Werk zu achten,
das der Herr unter allen Völkern zu tun im Begriffe ist, und
aus das Beben, das sie alle erzittern macht, damit er auf der
Hut sei, daß er nicht mit seinem scharfen Verstand, seiner Geschick-
lichkeit und seiner Klugheit selbstische Nebenzwecke verfolge.
Es wurde zu der Zeit eine Verordnung zur Prüfung der
sogenannten Geistlichen erlassen, ob man sie bestätigen oder ihrer
Amter und Besoldungen entsetzen solle, und es trieb mich, den
betreffenden Vorgesetzten darum zu schreiben.
,,Freunde,
.... Christus zeigt seinen Jüngern und dem Volk, wie
man solche wie diese zu prüfen hat Sie werden von den Menschen
Herr genatmt. Sie sitzen aus den ersten Plätzen der Versamm-
lung; sie sind Hörer aber nicht Täter. Gr rief siebenmal Wehe!
über ste und verurteilte sie (Matth. 23) .... Gs gab in alten Zeiten
George Fox. 7


% \picinclude{./090-099/p_s098.jpg} 
98 Kapitel 7111.
ein Kornhauö, wo die Waisen, die Fremdlinge und die Witwen
hinkamen und zu essen bekamen, und die, welche ihre Zehnten
nicht inö Kornhauö brachten, gediehen nicht (Maleachi 3); hat
aber Ehristuö nicht allen Zehnten und Priestern und Tempeln
ein Ende gemacht? .... Sind je die Priester, die Zehnten nach
Menschensatzungen nahmen, gediehen? .... Warfen die Apostel
je jemanden in den Kerker wegen der Zehnten, wie ihr es jetzt tut?
Zum Beispiel: Ralph Hollingworth, Priester von Phillingham,
hat zu Lincoln einen armen Dachdecker namenß Thomaö Bromby
wegen einer kleinen Abgabe, nicht mehr als sechö Schilling, inß
Gefängniß geworfen, wo er nun schon seit achtunddreißig Tagen
ist; und der Priester ersuchte den Richter, daß man dem Mann
nicht erlaube, etwaS zu seinem Unterhalt im Gefängnis in der
Stadt zu verdienen. Jst dieses eine Empfehlung für euch, die ihr
die Aufgabe habt, die Priester zu wählen? . . .» Ehristutz hieß
seine Jünger, alß er sie au?-sandte, umsonst zu geben, wie sie
umsonst empfangen hatten; und in den Städten, durch die sie
zogen, mußten sie sehen, wer würdig war, und dort bleiben und
essen, waö man ihnen oorsetzte; und alß sie zu Christus zurück
kamen und er sie fragte, ob sie Mangel gelitten hätten, so sagten
sie: ,,nein«. Sie gingen nicht in die Stadt und fragten die
Leute, wie oiel sie im Jahre bekommen, wie dietz jetzt geschieht
von denen, die abgesallen sind. Der Apostel sagt, ,,habe ich
nicht zu essen und zu trinken?« aber er sagt nicht: ,,habe ich nicht
Osterpfriinde, Aufbesserungen und Geldsummencks .... ,,EZ soll
dem Ochsen, der da drischt, nicht daß Maul verbunden werden«
(5. Mos. 25, 4), aber sehet zu, ob ihr auch gedroschen habt und
ob daß- Korn in den Scheimen ist! Dies sagt einer, der eure
Seelen lieb hat und euer ewigeß Heil will.« G. F.
Nachdem ich einige Zeit in London gewesen und dort gewirkt
hatte, trieb etz mich nach Bedsordshire zu John Crook 1) zu gehen,
wo eine große Versammlung war und viele die Wahrheit annahmen.
John Crook sagte mir, daß am folgenden Tage mehrere Herren
der Umgegend mit ihm speisen werden, um mit ihm zu diskutieren.
Sie kamen und ich redete von der ewigen Wahrheit Gotteö zu
ihnen. Mehrere Freunde gingen an jenem Tage inz Turmhauß.
1) John Crook, früher ein angesehener Friedens-richter der Grafschaft
Bedford, wurde ein in vielen Verfolgungen standhafter Quäker.


% \picinclude{./090-099/p_s099.jpg} 
Brief an den Papst. Die Studenten von Cambridge usw. 99
Und in der Umgegend war auch eine Versammlung und es trieb
mich hin zu gehen, obwohl es mehrere Meilen weit weg war.
John Crook ging mit mir. Es war einer dort, Gritton, der
Baptist gewesen, aber jetzt höher hinaus wollte und sich ein Prüfer
der Geister nannte. Gr sagte den Leuten, wie viel Vermögen
sie haben, und behauptete, ihnen sagen zu können, wenn ihnen
etwas gestohlen oder verbrannt wurde, wer es getan. Dadurch
hatte er die Gunst vieler erworben. Dieser Mann redete gerade
laut, als ich kam. Er hieß Alexander Parker seine Hoffnung
begründen. Alexander erwiderte: ,,C-hristus ist unsere Hofstiung;«
weil diese Antwort nicht so schnell gegeben wurde, wie er sie
erwartete, so schrie er: »sein Mund ist gestopst!« Daraus richtete
er sich an mich, denn ich stand schweigend dabei, weil er vieles
sagte, das sich nicht mit der Schrift vertrug. Jch fragte ihn,
ob er sich aus die Schrift berufen könne? er sagte: ,,ja;« ich hieß
die Leute ihre Bibeln nehmen und die Stellen aussuchen, die
er angeben würde, aber er konnte es nicht. So war er beschämt
und ging fort und seine Anhänger wurden meistens gewonnen ....
John Erook blieb in der Kraft Gottes, aber er wurde seines
Amtes als Richter entsetzt ....
Jch ging nach Romney, wo die Leute von meinem Kommen
gehört, und es war darum eine sehr große Versammlung. Zu
dieser kam Samuel Fischer 1), ein großer Baptistenprediger. E-r
hatte eine Pfarrei sgehabt, die ihm etwa zweihundert Pfund im
Jahre eingebracht hatte und die er um des Gewissens willen auf-
gegeben hatte. Der Pfarrer der Baptisten war auch dabei und
viele ihrer Leute. Die Kraft des Herrn ward so mächtig kund,
daß viele ergriffen wurden ..... Als die Versammlung vorüber
war, sagte Samuel Fischers Frau: »so, nun laßt uns darüber
reden, was geistig und was fleischlich ist, damit wir die Lehre
des Geistes von der Lehre des Fleisches [unterscheiden könne«n.«
Samuel Fischer und manche andere traten für das Wort des
Lebens ein, das an diesem Tage ihnen war erklärt worden. Der
andere Pfarrer unds seine Anhänger redeten dagegen .....
Samuel Fischer nahm die Wahrheit an und wurde ein getreuer
Prediger; er predigte umsonst und arbeitete viel für den Herrn;
1) Samuel Fischer und John Stubbs gingen später u. a. nach Rom und
traten dort mutig gegen papistischen Aberglauben auf. Fischer starb 1665 im
Gefängnis in London an der Pest.
 


% \picinclude{./100-109/p_s100.jpg} 
% \picinclude{./100-109/p_s101.jpg} 
% \picinclude{./100-109/p_s102.jpg} 
% \picinclude{./100-109/p_s103.jpg} 
% \picinclude{./100-109/p_s104.jpg} 
% \picinclude{./100-109/p_s105.jpg} 
% \picinclude{./100-109/p_s106.jpg} 
% \picinclude{./100-109/p_s107.jpg} 
% \picinclude{./100-109/p_s108.jpg} 
% \picinclude{./100-109/p_s109.jpg} 

% \picinclude{./110-119/p_s110.jpg} 
110 Kapitel 11.
Jch wiederholte, man solle etz vorlesen, damit alle urteilen könnten,
ob etwas Verführerischeß darin sei, in dem Falle wolle ich dafür
leiden. Schließlich laö ez der Angestellte mit lauter Stimme, daß
alle es hören konnten; alö er fertig war, sagte ich: »ja, es ist
mein Blatt, ich stehe dazu, und ihr müßt auch dazu stehen, wenn
ihr nicht die Schrift verleugnen wollt, denn ist es denn nicht, maß
die Schrift sagt, und Christuö und die Apostel, denen alle wahren
Christen gehorchen müßen?« Nun ließen sie den Gegenstand
fallen, und der Richter kam wieder auf unsere Hüte zurück und
hieß den Kerkermeister sie une abnehmen, dieser tat es; aber wir
setzten sie wieder auf ..... Der Richter hielt nun eine lange
Rede über den Lord Protektor, wie er ihn zum obersten Richter
in England gesetzt, und ihn hierhergeschickt, und dergleichen mehr.
Wir baten ihn, er solle uns Gerechtigkeit erzeigen nach unserer
ungerechten Gefangenschaft diese nenn Wochen; statt dessen aber
brachten sie eine Anklage vor, die sie gegen unz zusammengesetzt
hatten, so voll Lügen, daß ich meinte, sie richte sich gegen einen
Dieb: wir seien nur mit Wasfengewalt und nach großem Wider-
stand hierher gebracht worden! und doch waren wir, wie oben
gemeldet, gekommen. Ich sagte ihnen, daß sei falsch und wir
wiederholten unser Gesuch um Gerechtigkeit; die Gefangennahme,
sagte ich, sei ungerecht, denn ich sei auf der Reise von Major
Ceely festgenommen worden. Ntm redete Peter Eeely mit dem
Richter und sagte, auf mich zeigend: »Grlaubt mein Herr, dieser
Mann nahm mich bei Seite und sagte, er könne in einer Stunde
vierzigtaufend Mann stellen und da-? Land in Blut stürzen und
König Karl zurückbringen, und ich könne ihm dabei behilflich sein.
Jch wollte ihm auö dem Lande helfen, aber er wollte nicht
gehen; ich habe Zeugen, die?7 zu beschwören«, und er rief den
Zeugen auf. Aber der Richter war nicht gewillt, ihn anzuhören,
und so bat ich, man möchte meine Anklage, auf Grund deren
ich verhaftet sei, vorlesen; der Richter sagte: ,,nein, sie soll nicht
vorgelesen werden«, .... als ich sah, daß man sie nicht lesen
wollte, sagte ich zu einem meiner Mitgefangenen, »du hast eine
Abschrift davon, lies die vor.« ,,Kerkermeister«, sagte hieraus der
Richter, »führ ihn sort! wir wollen doch sehen, wer hier Meister
ist, er oder ich!« und so wurde ich hinweg geführt. A16 ich
wieder gerufen wurde, bestand ich wiederum darauf, daß mein
Verhastbesehl vorgelesen werde, denn davon hing meine Gefangen-


% \picinclude{./110-119/p_s111.jpg} 
Angriffe der Jndependenien und Ptesbyterianer usw. 111
schaft ab. Jch hieß abermalö meinen—Mitgesangenen ihn lesen,
und er tat ez:
,,Peter Ceely, einer der Friedenörichter der Grafschaft, an
den Kerkermeister von Seiner Hoheit Gefängniö zu Launeeston:
»Jch sende Euch hiermit durch den Überbringer dieser- die
Personen Edward Pyot von Bristol und George Fox auz Dray-
ton-in-the-Elay in Leicestershire, und William Salt von London, . .
die alt; Quäker bekannt sind und sich selber als.3 solche bekennen;
sie haben Verschiedene Blätter verbreitet, die den öffentlichen
Frieden gefährden, und können keinen gesetzlichen Grund für ihr
Erscheinen in dieser Gegend angeben, sie sind gänzlich unbekannt
in dieser Gegend, haben keinen Paß, weigern sich, irgendwelche
Beweise ihreö guten Wandels zu geben, die da; Gesetz verlangt,
und weigern den Abschwörungtzeid zu leisten. Wir befehlen euch
darum im Namen seiner Hoheit dez? Lord Protektor, diese Per-
sonen, . . . wenn sie kommen, in Gewahrsam zu bringen und da-
rin zu lassen, bis sie gesetzlich srei gelassen werden. Versäumet
nicht, solcheö zu tun, wo andere'- eö euch gefährlich werden könnte.
Auögegeben mit meiner Unterschrift und Siegel, St. Joes, den
18. Januar 1655. P. Eeely.«
Als dietz vorgelesen worden war, sagte ich zu den Richtern, . . .
mich an Major Ceely wendend: »Wo und wann habe ich dich
beiseite genommen? .... und wenn du mein Ankläger bist, wa-
rum sitzest du auf der Richterbank? du solltest herunter kommen
und mir ins Gesicht sehn. Übrigens möchte ich fragen, ob nicht
Major Ceely sich der- Verrat-3 schuldig machte, dessen er mich an-
klagt, durch sein langeö Schweigen? Kennt er seinen Platz als
Soldat wie alö Friedentzrichter? denn .... wenn ich ihn beiseite
genommen, um ihm zu sagen, ich könne oierzigtaus end Mann stellen,
und so weiter, .... so sehet ihr deutlich, daß er ja in dieser
Verschwörung beteiligt gewesen wäre, indem er mich .... aus
der Gegend forthaben wollte .... und den Verrat nicht früher
entdeckte. Aber ich leugne seine Autzsagen und bin unschuldig an
diesem teuflischen Plan.« Die Richter ließen nun die Sache fallen,
denn sie sahen daß, anstatt daß sie mich in eine Falle gelockt hatten,
ich selber ihnen eine gestellt hatte. Major Eeely behauptete nun,
ich habe ihm ins Gesicht geschlagen ..... Jch fragte ihn, ob er
sich alö Richter und Soldat nicht schäme, solcheö zu sagen ....
Schließlich, als die Richter sahen, daß diese Fallen nichts nutzten,


% \picinclude{./110-119/p_s112.jpg} 
112 Kapitel llc.
ließen sie uns wieder ins Gefängnis führen und forderten von
jedem zwanzig Goldstücke, weil wir den Hut ausbehalten .....
Als das Urteil so lautete, daß keine baldige Freilassung
zu erwarten war, hörten wir aus, dem Wörter wöchentlich
7 Schilling für unsere Pferde und 7 für uns selber zu geben;
daraufhin wurde er böse und ganz teuflisch und brachte uns nach
Doomsdale hinunter, einen greulichen, stinkenden Ort wohin die
Mörder nach der Verurteilung gebracht wurden. Der Ort war
sehr ungesund, so daß wenige, die sich hier aushalten mußten,
wieder gesund heraus kamen; es war kein Abtritt da, und der
Unrat der Gefangenen war seit Jahren nie hinausgeschasst worden.
Es war ein förmlicher Sumpf darin, stellenweise bis über die
Schuhe, von dem Unrat; und man erlaubte uns nicht, rein zu
machen oder uns Betten oder Stroh zum drauf liegen zu ver-
schaffen. Am Abend brachten uns einige Bekannte aus der Stadt
ein Licht und etwas Stroh, um draus zu liegen; wovon wir einiges
verbrannten, um den Gestank zu vertreiben. Die Diebe schliefen
gerade über uns und der Wörter in einem Zimmer daneben.
Scheints drang der Rauch ins Zimmer des Wärters; er geriet
in einen solchen Zorn, daß er die Nachtgeschirre der Diebe nahm,
und sie durch ein Loch gerade auf unsere Köpfe ausleerte; wir
waren so beschmiert davon, daß wir weder einander noch uns selber
anrühren konnten; und der Gestank war so arg, daß wir fast
darin erstickten. Vorher hatten wir den Gestank zu unsern Füßen
gehabt, jetzt hatten wir ihn auch aus den Köpfen und am Rücken;
und da unser Stroh von dem heruntergeworsenen Dreck beschmutzt
war, so verbreitete es einen greulichen Dunst. Zudem fluchte der
Wörter gräszlich über uns und nannte uns »hackengesichtige Hunde«;
und andere merkwürdige Namen, die wir noch nie gehört. Jn
diesem Zustand gingen wir fast zu Grunde während der Nacht,
denn wir konnten nicht einmal absitzen, alles war so voll Unrat.
Wir mußten lange in dem Zustand ausharren, bis uns gestattet
wurde, reinzumachen und uns andere Lebensmittel zu verschaffen als
das, was durchs Gitter kam. Einmal brachte einMädchen uns etwas
Essen; der Wärter arretierte es und führte es vor Gericht, weil
es ins Gefängnis eingedrungen sei, und es geriet in große Not;
dadurch wurden viele andere entmutigt, so daß es uns schwer
wurde, uns Wasser oder Lebensmittel zu verschaffen. Wir ließen
nun eine junge Frau aus Londen kommen, Anna Downer, damit


% \picinclude{./110-119/p_s113.jpg} 
Angriffe der Jndependenten und Presbhterianer usw. 113
sie uns das Essen kaufe und zubereite; sie war dazu bereit, denn
es war über sie gekommen, zu uns zu kommen in der Liebe
Gottes, und sie war sehr dienstfertig gegen uns .....
Die Gefangenen und einige andere verschrobene Leute be-
richteten von Gespenstern, die in Doomsdale umgingen, und von
den vielen, die hier gestorben seien, um einen damit angst zu
machen. Aber ich sagte, daß, wenn auch alle Geister und Teufel
der Hölle dort seien, ich darüber stehe, durch die Kraft Gottes,
und nichts dergleichen fürchte; denn Christus unser Priester werde
uns das Haus und die Mauern heiligen, er, der dem Teufel den
Kopf zerbrochen habe .....
Es war gtun bald die Zeit der allgemeinen vierteljährlichen
Getichtssiizuttg, und da der Kerkermeister sich immer noch schlecht
gegen uns benahm, setzten wir einen Bericht über unsere Leiden
auf und schickten ihn zur Gerichtsfitzung nach Bodmin. Als die
Richter ihn gelesen, gaben sie den Befehl, daß die Türen von
Doomsdale geöffnet werden sollten und man uns erlaube, rein
zu machen und unsere Nahrung in der Stadt zu kaufen. Wir
sandten eine Abschrift unseres Leidensberichts an den Protektor
und erzählten ihm, wie wir von Major Ceely verhaftet und ver-
urteilt worden waren, und wie uns der Kerkermeister mißhandelt
hatte. Der Protektor schickte einen Befehl an Hauptmann Fox,
den Befehlshaber von Schloß Pendennis, daß er untersuche, wie
es sich mit den Soldaten, die uns mißhandelten, verhalte .....
Solches war der Sache des Herrn sehr förderlich; denn
nachher konnten die Freunde in jedem Turmhaus oder Markt-
platz reden und es tat ihnen niemand etwas. Jch hörte, daß
Hugh Peters, einer der Kapläne des Protektor, diesem gesagt habe,
man könne dem George Fox keinen größern Dienst zur Ausbrei-
’ tung seiner Ansichten in Cornwall tun, als ihn in Cornwall ein-
zusperren. Und wirklich kam meine Gesangennahme in Cornwall
vom Herrn zur Förderung seiner Sache in dieser Gegend; denn
als es nach der Gerichtssitzung hieß, wir würden gefangen bleiben,
kamen Freunde aus allen Teilen des Landes, um uns zu besuchen.
Diese westlichen Gegenden waren damals sehr in Finsternis, aber
das Licht und die Kraft und die Wahrheit des Herrn brachen
nun hervor und leuchteten über allen, und viele bekehrten sich
von der Finsternis zum Licht und von der Macht des Satans
zu Gott. Ge trieb viele, in die Turmhäuser zu gehen, und viele
George Fox. 8


% \picinclude{./110-119/p_s114.jpg} 
114 Kapitel llc.
besuchten unö, denn wir durften nun umher gehen im Schloßhof,
und an den Ersten Tagen kamen oiele zu unß, denen wir daß
Wort des Lebenö brachten .....
Ju Cornwall, Deoonfhire, Dorsetshire und Somersetshire
fing die Wahrheit an mächtig zu sprießen, und viele bekehrten sich
zuzEhristuS; viele Freunde fühlten sich getrieben, die Wahrheit
in diesen Gegenden zu verkünden, waS die Priester und die
,,Frommen« sehr aufbrachte, so daß sie die Behörden anstifteten,
den Freunden Fallen zu stellen. Sie stellten Wachen auf den
Landstraßeu, unter dem Vorwand, alle verdächtigen Personen
abzufassen, sie ergriffen nun daraufhin Freunde, die vorbeikamen,
um unß im Gefängnis zu besuchen ..... Aber gerade daß,
waö sie taten, um der Wahrheit Einhalt zu tun, diente dazu, sie
auözubreiten; denn dadurch wurden die Freunde oft getrieben, zu
den Konstablern oder den Behörden, vor die sie gebracht wurden,
zu reden, was viel dazu beitrug, daß die Wahrheit sich in allen
Distrikten aus-breitete. Oft wenn Freunde in die Hände der
Wachen gerieten, ging es zwei oder drei Wochen, ehe sie wieder
frei wurden.
Alß Thomaß Rawlinson au-J dem Norden her kam, um unz
zu besuchen, ergriff ihn ein Konstabler in Devonshire und nahm
ihm nachts zwanzig Schilling aus der Tasche, und darauf wurde
er zu Exeter ins Gefängniß geworfen. Henry Pollexfen warfen
sie auch inß Gefängniz, weil er ein Jesuit sei ..... Viele
Freunde wurden von ihnen mißhandelt; ja Leute, die an ihrer
Arbeit waren, wurden von ihnen gepeitscht und ergriffen, und
ez waren doch solche darunter, die eine Einnahme von mehr
als achtzig und hundert Pfund im Jahr hatten; und zwar
geschah ihnen solcheö, wenn sie kaum vier oder fünf Meilen von
zu Hause weg waren. Unter dem Eindruck all des Bösen, daß «
mit dem Aufstellen der Wachen und dem Gefangennehmen der
Freunde beabsichtigt war, kam ez über mich folgendeß zu schreiben:
,,Eine Mahnung und Warnung an die Behörden.
»Jhr Mächte der Erde, Christus ist gekommen um zu regieren,
und er ist unter euch, und ihr kennetihn nicht; er erleuchtet einen
jeden unter euch, damit ihr alle an ihn glauben möchtet, an
das Licht; an den »der die Kelter allein tritt«. Darum prüfet
alle in diesem Lichte, ob ihr reif seid, denn die Kelter ist bereit.
(Offb. 14, 19) ....


% \picinclude{./110-119/p_s115.jpg} 
Angriffe der Jndependenten und Presbhterianet usw. 115
»Jht verkündet Gewissensfreiheit; und doch darf man seinen
Freunden keine Briefe bringen, oder seine Freunde oder die
Gefangenen besuchen, oder ihnen Bücher bringen, ohne daß ihr
Wachen ausstellt, um sie anzuhalten und zu greifen; und sogar
bewaffnet müssen diese sein gegen die guten Leute, die kaum
einen Stock mit sich tragen, und die ihr aus Groll Quäker nennt.
Und die, welche diese Wachen ausstellen, die verkünden Gewissens-
freiheit und nehmen solche gefangen, die ihr Gewissen gegen Gott
und gegen die Menschen rein erhalten wollen, die Gott im Geist
und in der Wahrheit anbeten, was die, welche nicht im Licht sind,
Ketzerei nennen! . . . Jst je solch ein Geschlecht gewesen, das
so wahnsinnig schlecht und verfolgungssüchtig war und Bewasfnete
ausstellte gegen die Wahrheit und sie verfolgte, wie Grafschasten
und Städte es jetzt tun? das klingt wie Sodom und Gomorrah!«
G. F.
Gs kam mir eine Abschrift eines von der Sitzung von E-xeter
ausgehenden Verhastbesehls in die Hände, der in starken Aus-
drücken verlangte, ,,alle Quäker zu verhasten«, und der die Wahr-
heit und die Freunde schlecht machte; da trieb es mich, eine
Antwort zu schreiben und zu verbreiten, um die Freunde und
die Wahrheit gegen solche Verleumdungen zu verteidigen, und
die Schlechtigkeit und Bosheit des Verleumdungsgeistes zu zeigen. . .
Wir blieben im Gefängnis bis zur nächsten Sitzung; viele
Freunde, Männer und Frauen, die von der Wache ergriffen
worden waren, waren ins Gefängnis gebracht worden. Viele
von ihnen wurden nach Eröffnung der Sitzung vor die Richter
gebracht und beschuldigt, sie hätten sich gesträubt zu kommen
und waren doch von den Gefängniswärtern gebracht worden.
Der Richter legte ihnen Bußen auf, weil sie den Hut nicht ab-
nehmen. Wir hingegen mußten nicht mehr oor den Richter.
Während dieser ganzen Zeit und während der Sitzungen war
Unser Wirken für den Herrn reich gesegnet, denn es kamen viele
zu uns, »Fromrne« und andere, mit uns zu reden. Elisabeth
Trelawm) von Plymouth, die Tochter eines Barons, wurde bekehrt,
worüber Priester und ,,Fromme« und viele angesehene Personen
außer sich waren und ihr Briefe deswegen schrieben. Da sie eine
weise und gottselige Frau war und denen, die ihr geschrieben,
nicht wollte etwas in die Hand geben, das sie dann hätten können
gegen sie gebrauchen, so schickte sie mir die Briefe; und ich schrieb
 


% \picinclude{./110-119/p_s116.jpg} 
116 Kapitel 1:.
ihr darüber, rmd sie beantwortete sie dann. Sie nahm zu in der
Kraft und der Weiöheit Gotrtesz, so daß sie zuletzt imstande war,
den msichtigsten Priestern und ,,Frommen« zu antworten; sie hatte
die Herrschaft über sie in der Wahrheit durch die Kraft Gotteö,
der sie treu blieb biz in den Tod.
Während ich hier in der Gefangenschaft war, prophezeiten
die Baptisten und Fifthmonarchyleute, in diesem Jahre werde
Christus- kommen und tausend Jahre auf Grden regieren. Sie
erwarteten dieseö Reich als ein äußereö, während er doch in die
Herzen der Menschen gekommen war, um darinnen zu regieren;
aber so wollten ihn diese ,,Frommen« nicht aufnehmen, darum
mißlang ihnen ihr Prophezeien. A, Ehristuö ist ja schon ge-
kommen tmd wohnet in den Herzen der Menschen und regieret
darin. Tausende, bei denen er anklopfte, haben ihm aufgetan;
und er ist bei ihnen eingekehrt und hat das Abendmahl mit ihnen
gehalten (Offb. 3, 20). Z Viele dieser Baptisten und Fisthmonarchh-
leute sind die ärgsten Feinde derer, die sich zu Ehristuß hielten,
geworden; aber er regieret in den Herzen seiner Heiligen.
Während der Gerichtßsitzung kamen mehrere der Richter zu
unö und waren ziemlich höflich und redeten vernünftig über
göttliche Dinge mit uns und bezeugten unß Teilnahme. Haupt-
mann Fox, der Gouverneur von Schloß Pendenniö, trat zu mir
mid sah mir inß Gesicht, sagte aber nichtß; aber alß er wieder
zu seinen Begleitern zurück kam, sagte er, er habe noch nie in
seinem Leben einen einfältigeren Menschen gesehen. Jch rief ihm
nach: ,,Wir wollen sehen, wer der Ginfältigere ist!« Aber er
ging seineß Wegeß, der hochmütige Tropf.
Thomas Lower 1) besuchte unß ebenfallß .... Er stellte uns
viele Fragen darüber, daß wir behaupteten, die Schrift sei nicht
das- Wort Gotteö, und über die Sakramente und andere?-, und
wir konnten über alleß Ausschluß geben. Jch redete auch noch
allein mit ihm, und er bekannte nachher, meine Worte hätten
ihn wie ein Blitzstrahl durchzuckt. Er habe noch nie Leute, wie
wir seien, getroffen, die seine innersten Gedanken errieten. Gr
wurde nachher bekehrt und ist ein Freund geblieben biß auf diesen
Tag .... und hat viel um der Wahrheit willen gelitten. R
EZ trieb mich zu dieser Zeit, die folgende Mahnung an die
Freunde, die Prediger waren, zu richten:
1) Thomas Lower, später Schwiegersohn von G. F.


% \picinclude{./110-119/p_s117.jpg} 
Angriffe der Jndependenten und Presbyteriuner usw. 117
,,Freunde!
Bleibet in der Kraft des Lebenö und der Weisheit und in
der Furcht des Herrn Himmels und der Erde, damit ihr in der
Weiöheit Gottes bewahret bleiben möget und seinen Gegnern ein
Schrecken werdet, indem ihr die Wahrheit verbreitet, Zeugen für
sie erwecket, die Betrügerei stürzet, von der Übertretung zum
Leben bringt, in den Bund deß Lichtß und den Frieden Gotteö.
Lasset alle Welt diese Stimme hören, durch Wort oder Schrift.
Schonet keinen Ort, noch Sprache, noch Feder; tuet das Werk
in Gehorsam gegen Gott; kämpfet tapfer für die Wahrheit auf
Erden und zertretet alles, waß ihr entgegen ist. Jhr habt die
Kraft, mißbraucht sie nicht ..... Regieret mit Ehristuß, dessen
Thron und Szepter nun ausgerichtet sind, und der herrscht bis an
die Enden der Erden ..... GH soll nun daß Heil außgehen
von Zion, zu richten den Berg Esauß, und daß Gesetz soll von Jeru-
salem au?-gehen (Obad), damit es redezu dem Göttlichen, daßin einem
Jeden ist, und alle Erfindungen und Erfinder überwältige. Alle
Fürsten der Welt sind Luft vor der Macht Gotteß, die ihr ge-
schmecket habt; darum lebet in ihr .....
Ftthret alle zur Anbetung Gotteö; pflüget den brachliegenden
Acker, dreschet daß Korn, damit der Same, der Weizen, in die
Scheunen gesammelt werden könne und Alle zum Ursprung,
. zu Ehristuß, kommens der war, ehe der Welt Grund gelegt ward.
Die Spreu ist durch die Übertretung unter den Weizen gekommen;
der, welcher ihn auödrescht, hat die llbertretung verlassen und
erkennt sie, und unterscheidet zwischen dem Wertoollen und dem
Unwerten, er kann den Weizen vom Unkraut unterscheiden und
ihn in den Speicher sammeln und bringet so die unsterbliche
Seele zu Gott, von dem sie kamsf. . . Die Prediger des Geisteß
müssen dem gefangenen Geist predigen, damit durch den Geist
Christi die Menschen zu Gott, dem Vater alleö Geistes, geführt
werden, ihm zu dienen und eins zu sein mit ihm, mit der Schrift
und untereinander. Dieß ist das Wort deß Herrn an euch alle.
. .   Seid ein Vorbild und Beispiel in allen Ländern, Ort-
schaften, Jnseln und Völkern, zu denen ihr kommt, damit euer
Wandel allen Menschen predige .... Darin werdet ihr dem
Herrn angenehm sein und ein Segen werden.
Schonet den Betrug nicht; greiset ihn mit dem Schwert an;
bekämpfet ihn; trachtet nicht nach Blut, weder in Wort noch


% \picinclude{./110-119/p_s118.jpg} 
118 L Kapitel 11.
Schrist ..... Verktindet allen den lebendigen Gott; denn alle
Lehre, Kirche und Gottesdienst, die durch menschlichen Willen und
Verstand eingesetzt sind, werden von der Kraft Gottes vernichtet.
.... Verkiindet den großen Tag des Feuerß und des Schwerteß,
den Tag deö Herrn, der im Geist und in der Wahrheit will
angebetet sein, und bleibt in der Kraft Gottes, damit die Bewohner
der Erde vor euch erzittern; und damit die Kraft und Herrlich-
keit deö Herm unter den Heiden Hund den Heuchlern gepriesen
werde, und ihr in der Weisheit und Furcht, im Leben, im Schrecken
und in der Herrlichkeit bewahrt bleibt zu seiner Ehre. EZ gehet
ein Ruf, daß man die Übertretung verlasse, und der Geist ruft:
,,kommet«. EZ ergehet jetzt ein Ruf, die falschen Gotteödienste
zu verlassen und dem wahren Gott zu dienen; ein Ruf zur
Buße .... damit die Gerechtigkeit hervorbreche; und sie wird
über die ganze Erde sich ausbreiten. Darum tut treu in der
Kraft dee- Herrn euer Werk, ihr, die ihr au?-erwählt seid .....
Gehorchet der Kraft, sie wird euch erretten auß der Hand der
Unoernünftigen und von der Welt. Durch sie werdet ihr daß
Reich haben, dasz kein Ende hat und in welchem Herrlichkeit und
Leben isi.« .... G. F.
Nach der Sitzung hatten wir manche Unterredungen mit dem
Scheriff und einigen Soldaten, die eine zum Tode verurteilte Frau
bis zur Hintichnmg überwachen mußten. Einer von ihnen sagte:
,,Ehristuö war einer der heftigsten Menschen, die je gelebt«; wir
verwiesen ihm dies. Ein andermal fragten wir den Kerkermeister,
was- bei den Gericht?-Verhandlungen vorkomme. Er antwortete:
,,O, nur Kleinigkeiten; nur etwa dreißig, die wegen Bastardschaft
verurteilt sind.« Wir wunderten untz sehr, daß solche, die doch
Christen zu sein meinten, derartigeö eine Kleinigkeit fanden.
Aber dieser Kerkermeister war selber ein sehr schlechter Mensch.
Jch ermahnte ihn oft zur Rechtschaffenheit, aber er behandelte
die Leute, die unz besuchen wollten, schlecht. Edward Pyot bekam
einen Käse von seiner Frau geschickt; der Kerkermeifter nahm ihn
ihm weg und brachte ihn dem Major, angeblich um ihn auf
verräterische Briefe hin zu durchsuchen; aber obwohl sie nichts
von Vriefett fanden, so behielten sie ihn doch. EZ hätte diesem
Kerkermeister ganz gut gehen können, wenn er sich anständig
betragen hätte, aber er suchte selber sein Verderben, welcheö auch
bald über ihn kam; denn im darauf folgenden Jahre wurde er


% \picinclude{./110-119/p_s119.jpg} 
Angriffe der Jndependenten und Presbyterianer usw. 119
von seiner Stelle abgesetzt und kam selber ins Gefängnis und
bettelte dort bei den Freunden. Und wegen irgend eines Ver-
gehens brachte ihn sein Kerkermeister nach Doomsdale und legte
ihn in Ketten und schlug ihn, und erinnerte ihn daran, wie er
jene guten Leute mißhandelt habe, die er ohne jeden Grund in
diesen greulichen Kerker getan, und daß er nun die verdiente
Strafe für seine Bosheit leiden müsse, und ihm nun mit dem
Maße gemessen werde, mit dem er gemessen habe. Es ging ihm
sehr schlecht und er starb in der Gefangenschaft, und sein Weib
und seine Kinder kamen ins Elend.
Während ich zu Launceston gefangen war, ging ein Freund
zu Oliver Cromwell und erbot sich, an meiner Statt in Dooms-
dale gefangen zu sein, wenn er es annehmen und mich dafür
in Freiheit setzen wolle. Dies erstaunte Eromwell dermaßen,
daß er zu seinen Räten sagte: ,,Welcher unter Euch würde so oiel
für mich tun, wenn ich in dieser Lage wäre?« Und obgleich er
das Anerbieten des Freundes nicht annahm, sondern sagte, er
könne es nicht tun, weil es gegen das Gesetz sei, so ergriff ihn
doch die Wahrheit mächtig. Einige Zeit darauf schickte er den
Generalmajor Desborough in der Absicht, uns frei zu lassen;
dieser kam und bot uns die Freilassung an unter der Bedingung,
daß wir oersprechen, heim zu gehen und nicht mehr zu predigen,
, aber wir wollten ihm nichts versprechen; daraufhin schlug er uns
vor zu versprechen, nach Haus zu gehen, ,,wenn der Herr es zu-
lasse«, worauf Edward Phot ihm einen abschlägigen Brief schrieb.
Als einige Zeit verstrichen war, seitdem dieses Schreiben
abgegeben worden war, schrieb ich ebenfalls an ihn, folgender-
maßen:
,,Freund,
Wir, die wir in der Kraft Gottes des Herrschers aller Dinge
sind, die wir seine Kraft kennen und in ihr wohnen, müssen ihr
auch gehorchen; und darum müssen wir uns frei halten von
allem, was Menschenwille befiehlt. Wenn es sich darum handelt,
etwas zu kaufen oder zu verkaufen, so mag es etwa angehen zu
sagen: wir wollen, so der Herr es zuläßt; aber da wir in der
Kraft Gottes stehen, unter keines Menschen Willen, so können wir
solches nicht mit Wahrhaftigkeit sagen, wo es sich um unsere
Befreiung aus der Gefangenschaft handelt .....
13. des 6. Monats 1656. G. F.


\picinclude{./120-129/p_s120.jpg} 
\picinclude{./120-129/p_s121.jpg} 
\picinclude{./120-129/p_s122.jpg} 
\picinclude{./120-129/p_s123.jpg} 
\picinclude{./120-129/p_s124.jpg} 
\picinclude{./120-129/p_s125.jpg} 
\picinclude{./120-129/p_s126.jpg} 
\picinclude{./120-129/p_s127.jpg} 
\picinclude{./120-129/p_s128.jpg} 
\picinclude{./120-129/p_s129.jpg} 

\picinclude{./130-139/p_s130.jpg} 
\picinclude{./130-139/p_s131.jpg} 
\picinclude{./130-139/p_s132.jpg} 
\picinclude{./130-139/p_s133.jpg} 
\picinclude{./130-139/p_s134.jpg} 
\picinclude{./130-139/p_s135.jpg} 
\picinclude{./130-139/p_s136.jpg} 
\picinclude{./130-139/p_s137.jpg} 
\picinclude{./130-139/p_s138.jpg} 
\picinclude{./130-139/p_s139.jpg} 

\picinclude{./140-149/p_s140.jpg} 
\picinclude{./140-149/p_s141.jpg} 
\picinclude{./140-149/p_s142.jpg} 
\picinclude{./140-149/p_s143.jpg} 
\picinclude{./140-149/p_s144.jpg} 
\picinclude{./140-149/p_s145.jpg} 
\picinclude{./140-149/p_s146.jpg} 
\picinclude{./140-149/p_s147.jpg} 
\picinclude{./140-149/p_s148.jpg} 
\picinclude{./140-149/p_s149.jpg} 

% \picinclude{./150-159/p_s150.jpg} 
% \picinclude{./150-159/p_s151.jpg} 
% \picinclude{./150-159/p_s152.jpg} 
% \picinclude{./150-159/p_s153.jpg} 
% \picinclude{./150-159/p_s154.jpg} 
% \picinclude{./150-159/p_s155.jpg} 
% \picinclude{./150-159/p_s156.jpg} 
% \picinclude{./150-159/p_s157.jpg} 
% \picinclude{./150-159/p_s158.jpg} 
% \picinclude{./150-159/p_s159.jpg} 

% \picinclude{./160-169/p_s160.jpg} 
Edward Burrough sagte: \glqq So tue es eilends, denn wir können nicht
wissen, wie viele in Bälde noch hingerichtet werden.\grqq Der König
sagte: \glqq So bald ihr wollt,\glqq und befahl einem der Anwesenden:
\grqq holt den Sekretär, so will ich es sogleich tun.\glqq Als der Sekretär
kam, wurde sofort ein Erlass zugesagt. Ein paar Tage darauf
ging Edward Burrough wieder zum König, um ihn zu bitten,
den Erlass abzuschicken; der König antwortete, er habe jetzt keine
Gelegenheit ein Schiff dorthin zu schicken; wenn wir es aber tun
wollten, so stehe uns das frei, so bald wir wollten. Darauf fragte
Edward den König, ob er allenfalls auch einen sogenannter:
\grqq Quäker\grqq mit seiner Sendung betrauen würde? Der König 
antwortete: \glqq Ja, es kann gehen, wer will.\grqq Hierauf nannte 
Edward ihm Samuel Chattok, der aus Neu-England, seiner Heimat
verbannt worden war und nicht zurückkehren durfte, es sei denn
mit dieser Sendung. Dann ließ er Ralph Goldsmith kommen,
den Besitzer eines guten Schiffes, und einigte sich mit ihm auf
300 Pfund, in Waren oder bar, und Abfahrt in zehn Tagen. Er
rüstete sich alsbald, unter Segel zu gehen, und, vom Winde 
begünstigt, kam er nach etwa sechs Wochen, am Morgen eines
Ersten Tages, in Boston in Neu-England an. Es reisten viele
mit ihm, aus Alt- und Neu-England, Freunde, die der Herr
trieb, mitzugehen und aufzutreten gegen die blutigen Verfolger,
welche alle Übrigen an Grausamkeit übertrafen.

Als die Bewohner von Boston ein Schiff mit englischen
Farben in den Hafen von Boston fahren sahen, kamen sie
gleich aufs Schiff und fragten nach dem Kapitän, und Ralph
Goldsmith sagte ihnen, das er es sei. Sie fragten ihn, ob er
Briefe habe? Er sagte: \glqq ja\grqq. Sie fragten, ob er sie 
ausliefern wolle? er antwortete: \glqq nein, heute nicht.\grqq 
Darauf begaben sie sich ans Ufer und berichteten, es sei 
ein ganzes Schiff voll Quäker angekommen, und Samuel Shattock 
sei darunter, der nach den Gesetzen hingerichtet werden müsse, 
wenn er aus der Verbannung zurückkomme! denn sie wussten nichts 
von seiner Sendung. Den ganzen Tag wurden alle streng abgesperrt, 
und keiner von der Schiffsmannschaft durfte landen. Am 
folgenden Morgen begaben sich die Gesandten des Königs, 
Samuel Shattock und der Befehlshaber des Schiffs, Ralph 
Goldsmith ans Ufer, und nachdem sie die Männer, die sie ans 
Land geführt hatten, zurückgeschickt hatten,
gingen sie durch die Stadt zum Haus des Gouverneurs, John
% \picinclude{./160-169/p_s161.jpg} 
Endicott, und klopften. Der Gouverneur schickte jemand heraus,
um sie nach ihrem Begehren zu fragen. Sie ließen ihm sagen,
sie kämen vom König von England und werden ihre Botschaft
niemand übergeben, als dem Gouverneur selbst. Darauf wurden
sie vorgelassen. Der Gouverneur erschien und nachdem er ihre
Botschaft vernommen und ihren Auftrag, nahm er seinen Hut ab
und betrachtete sie. Dann verließ er sie und begab sich zum
Untergouverneur, und nach einer kurzen Unterredung mit diesem
kam er zu den Freunden zurück und sagte ihnen: \glqq Wir
werden Seiner Majestät Befehl gehorchen.\grqq Hierauf erhielten
die Reisenden die Erlaubnis zu landen, und rasch verbreitete sich
die Kunde von dem Vorgefallenen in der Stadt, und die Freunde
aus der Stadt vereinigten sich mit den Reisenden des Schiffes,
um Gott zu loben und zu danken, das er sie so wunderbar aus
den Zähnen derer, die sie umbringen wollten, befreite. Während
sie beisammen waren, kam ein Freund herein, der von ihrem
blutigen Gesetz zum Tode verurteilt worden war und lange Zeit
in Fesseln gelegen und auf seine Hinrichtung gewartet hatte. Da
wurde die Freude noch größer, und alle erhoben ihre Herzen in
inbrünstigem Loben Gottes, welcher würdig ist zu nehmem Preis,
Ruhm und Ehre; denn er allein kann frei machen und erretten
und helfen allen denen, die ihr Vertrauen auf ihn setzen [...]

Vorher, als ich noch im Gefängnis zu Lancaster war, war
ein Buch von mir (\textit{The Battledore}) veröffentlicht 
worden, das zeigen sollte, wie in allen Sprachen \glqq du\grqq 
und \glqq dich\grqq die eigentliche Anrede an eine einzelne 
Person sei und \glqq ihr\grqq nur an mehrere.
Ich hatte es an Beispielen aus der Schrift und aus Lehrbüchern
in etwa dreißig Sprachen nachgewiesen. J. Stubbs und Benjamin
Furly hatten sich auf meine Veranlassung sehr Mühe gegeben,
das Material zu sammeln, und ich fügte dann noch einiges bei.
Als es fertig war, erhielten der König und die Räte, die Bischöfe
von Canterbury und London und die beiden Universitäten eine
Abschrift und es wurde viel gekauft. Der König sagte, es sei
richtig, das diese Völker so sprechen; und als man den Bischof
von Canterbury fragte, was er davon halte, so wusste er nicht,
was er sagen sollte; denn es wirkte so überzeugend auf die Leute,
das viele daraufhin sich kaum mehr ärgerten, wenn wir \glqq du\grqq
und \glqq dir\grqq zu ihnen sagten, während man uns das vorher sehr
übel genommen hatte [...]
% \picinclude{./160-169/p_s162.jpg} 

Da die Priester und Bischöfe gerade eifrig am Werk waren,
ihre Gottesdienste einzurichten und alle zu zwingen, daran teil
zu nehmen, trieb es mich, folgendes zu schreiben, um die Art der
wahren Gottesdienste, die Christus eingesetzt hat und die Gott
annimmt, zu zeigen:

\glqq Der wahre Gottesdienst Christ geschieht im Geist und steht
allen Menschen offen. Die im Geist und in der Wahrheit anbeten, 
die sind Gott angenehm (Join). Es gibt dem Volk Odem
und den Geist denen, die auf der Erde sind (Jes. 42, 5), und er
gibt ihnen eine unsterbliche Seele; sie sind sie Tempel, in denen
er wohnen will (1. Cor. 3, 16). Die, welche äußerlich
Juden waren, mussten nach Jerusalem gehen, um anzubeten, so
lange sie dort ihren äußeren Tempel hatten; [...] nun aber
sollen alle \glqq Gott im Geist und in der Wahrheit anbeten\grqq. Dies
ist ein Gottesdienst der Freiheit, denn \glqq wo der Geist ist, da ist
Freiheit\grqq (2. Cor. 3, 17). Die Früchte des Geistes werden
offenbar werden; und man soll der Geist keine Schranken setzen,
sondern in ihm wandeln und lebens, damit man seine Früchte
hervorbringen kann. [...] Denn ,an ihren Früchten sollt ihr
sie erkennen\grqq (Matth. 7, 16) [...] 
 \begin{flushright}G. F. \end{flushright}

Viele Papisten und Jesuiten fingen damals an, den Freunden
zu schmeicheln und zu sagen, so oft sie einen von ihnen sahen,
von allen Sekten haben die Quäkeren meisten Selbstverleugnung,
und es sei schade, das sie nicht in die heilige Mutterkirche 
zurückkehrten. In dieser Weise schwatzten sie den Leuten vor und 
behaupteten, sie würden gern mit den Freunden unterhandeln; aber
die Freunde verabscheuten es, sich mit ihnen einzulassen und
hielten es für gefährlich und sogar anstößig, weil es Jesuiten
waren. Als ich aber davon hörte, sagte ich: \glqq lasst uns mit ihnen
unterhandeln, seien sie, wer sie wollen.\grqq Somit wurde die Zeit
festgesetzt, zu der zwei, die wie Höflinge aussahen, kamen; sie
fragten nach unsern Namen, die wir ihnen nannten; wir aber
fragten nicht nach ihren Namen, denn wir wussten ja, das sie
Papisten waren und wir Quäker. Ich fragte sie das selbe, was
ich schon früher einen Jesuiten gefragt hatte, nämlich, ob die
Römische Kirche nicht abgefallen sei von der Kraft, dem Geist
und den Grundsätzen der apostolischen Zeiten? [...] Als sie
sahen, das wir es genau nahmen, wichen sie aus, indem sie sagten,
es sei eine Anmaßung zu behauptet, irgend jemand habe den
% \picinclude{./160-169/p_s163.jpg} 
Geist und die Kraft, den die Apostel hatten. Aber ich sagte, es
sei eine Anmaßung von ihnen, die Worte Christi, der Apostel und
Propheten zu benützen und die Leute glauben zu machen, sie seien
Nachfolger der Apostel und Propheten, da sie doch zugeben müssen,
sie haben nicht den Geist und die Kraft der Apostel. Ich zeigte
ihnen, wie verschieden ihr Tun und ihre Früchte von denen der
Apostel seien. Darauf erwiderte mir einer von ihnen: \glqq Ihr seid
eine Gesellschaft von Träumern.\grqq \glqq Nein,\grqq erwiderte ich, 
\glqq sondern ihr seid widerwärtige Träumer, die ihr euch als die 
Nachfolger der Apostel träumt, während ihr doch zugebt, das ihr nicht ihren
Geist und ihre Kraft habt. Und ist es nicht Befleckung des Fleischer?
zu sagen, es sei Anmaßung zu behaupten, man habe den Geist
und die Kraft der Apostel? Und wenn ihr nun zugebt, das ihr
nicht den Geist und die Kraft der Apostel habt,\grqq sagte ich, \glqq 
so ist es klar, das ihr von einem anderen Geist und einer anderen
Kraft geleitet werdet als die erste Kirche und die Apostel.\grqq Ich
erklärte ihnen, das es ein böser Geist sei, der sie leite und sie zu
dem Beten mit Rosenkränzen und zu Bildern geführt habe und
zum errichten von Klöstern \index{Kloster} und zum Töten um des Glaubens
willen. Ich wies sie darauf hin, wie solches Tun gesetzlich und
nicht nach dem Evangelium der Freiheit sei. Sie waren dieser
Reden bald überdrüssig und gingen fort, und wir Vernahmen,
das sie den Papisten rieten, nicht mit uns zu disputieren, noch
von unsern Büchern zu lesen; somit waren wir sie los. Aber
wir setzten uns mit allen andern Sekten auseinander, mit den
Presbyterianern\index{Presbyterianern}, 
den Independenten\index{Independenten}, 
den Seekers\index{Independenten}, 
den Baptisten\index{Baptisten},
den Episkopalen\index{Episkopalen}, 
den Socinianern\index{Socinianern}, 
den Brownisten\index{Brownisten}, 
den Lutheranern\index{Lutheranern},
den Calvinisten\index{Calvinisten}, 
den Arminianern\index{Arminianern}, 
den Fifthmonarchyleuten\index{Fifthmonarchyleuten}, 
den Feministen\index{Feministen}, 
den Rantern\index{Rantern}. 
Von diesen allen behauptete niemand,
den gleichen Geist und die gleiche Kraft wie die Apostel zu haben.
In diesem Geist und dieser Kraft verlieh unö also der Herr den
Sieg über sie alle. Was die Fisthmonarchyleute betrifft, so trieb
es mich, eine Schrift zu schreiben, um ihren Jrrtum aufzudecken.
Sie erwarteten Christi persönliche Wiederkunft in äußerer Form
und Weise und setzten dazu das Jahr 1666 fest, und viele, wenn
eß um diese Zeit donnerte und regnete, machten sich bereit, weil
sie meinten, nun komme Ehristuö, um sein Reich auszurichten,
und bildeten sich ein, sie müßten nun die Hure draußen in der
Welt töten (Offb. 17). Aber ich sagte ihnen, die Hure sei lebendig
11*



% \picinclude{./160-169/p_s164.jpg} 

164 Kapitel 117.
in ihnen und noch nicht verzehrt vom Feuer Gottetz und von
ihnen im Geist und der Kraft des Herm vernichtet. Und ihre
Erwartungen, daß C-hristuö äußerlich wiederkomme, um sein Reich
auszurichten, sei wie daß ,,siehe hier, siehe da« (Luc. 17, 28) der
Pharisäer. Aber Christuö sei vor mehr alö 1600 Jahren ge-
kommen, um sein Reich auszurichten, wie Nebukadnezar geträumt
und Daniel prophezeit, und habe die vier Reiche zertrümmert und
daß große Vild mit dem goldenen Kopf und den Armen und
Beinen auß Silber, alleß habe der Wind Gotteö weggeblasen
wie im Sommer die Spreu beim Dreschen (Dan. 2, 32).
Christus habe gesagt, als er auf Grden war: ,,Mein Reich ist
nicht von dieser Welt.« Wenn es von dieser Welt gewesen wäre,
so hätten seine Diener gekämpft (Joh. 18); aber es war nicht
von dieser Welt, darum kämpften sie nicht. Alle diese Fifth-
monarchyleute, die mit fleischlichen Waffen kämpfen, sind keine
Diener Christi, sondern Diener deß Tierez und der Hure; Christus
sagt: ,,Mir ist gegeben alle Gewalt im Himmel und auf E-rden«
(Matth. 28,18), und sein Reich, daß vor 1600 Jahren aufgerichtet
wurde, herrschet noch. Und der Apostel sagt: ,,Wir sehen Ehristuö
regieren, und er wird fort regieren, biz daß alle Dinge ihm unter-
tan find« (1. Cor. 15).
In diesem Jahre, 1661, trieb es viele Freunde überß Meer.,
zu gehen, um die Wahrheit in fremden Ländern zu verkünden.
John Stubbe-, Henry Fell und Richard Costrob trieb ez nach
China und Priester Johannes Gegend zu gehen; aber kein Schiff
wollte sie nehmen. Mit vieler Mühe erhielten sie eine Vollmacht
vom König; aber die Ostindische Gesellschaft fand Mittel und
Wege, sie zu umgehen, und die Schiff?-herren wollten sie nicht
nehmen. Sie begaben sich nun nach Holland, in der Hoffnung,
dort überfahren zu können; aber auch dort wollte sie niemand
nehmen. Nun nahm Henry Fell und John Stubbß ein Schiff,
daß nach Alexandrien in Ägypten ging, in der Absicht, von dort
aus sich einer Karawane anzuschließen. Doch da kam Daniel
Baker und veranlaßte Richard Costrop gegen seine innere Frei-
heit, mit ihm nach Smyrna zu gehen. Auf der Überfahrt wurde
Richard krank, da kümmerte sich Daniel Baker gar nicht um ihn
und er starb. Aber der hartherzige Mann verlor später seine Stelle.
John Stubbß und Henry Fell erreichten Alexandrien, aber
sie waren kaum dort, alö der engliche Konsul sie schon verbannte;


% \picinclude{./160-169/p_s165.jpg} 

Beginn neuer Quäkerversolgnugen bei Anlaß der Verschwötungen usw. 165
doch verbreiteten sie, ehe sie sort gingen, viele Bücher und Schriften,
um den Türken und Griechen den Weg der Wahrheit zu zeigen;
daß Buch betitelt: »Die Gewalt des Papsteß gebrochen«, gaben
sie einem alten Mönch, damit er eß dem Papst bringe oder
schicke; alß der Mönch ez durchgelesen, legte er die Hand aufs
Herz und sagte: ,,Waß hier geschrieben steht, ist Wahrheit; wenn
ich eß aber öffentlich bekennen würde, so würden sie mich oer-
brennen.« John Stubbß und Henry Fell kehrten nach Eng-
land zurück, weil eß ihnen nicht erlaubt wurde, weiter zu gehen,
und kamen wieder nach London. Stubbß hatte eine Vision, daß
die Engländer und Holländer, die sich verbündet hatten, sie nicht
überzuschisfen, sich untereinander entzweien werden, und so kam
eß auch .....
Wir hatten aber nicht nur Schwereß von außen zu erdulden,
sondern auch unter unß durch John Perrot und seine Anhänger.
Einem trügerischen Geiste nachgebend, suchte er unter den Freunden
den schlechten, unziemlichen Brauch einzuführen, daß man während
deß allgemeinen Gebetß den Hut aufbehalten solle. Viele Freunde
hatten mit ihm und seinen Anhängern darüber gesprochen, und
ich hatte einigen deßwegen geschrieben, aber er und andere taten
sich nur noch mehr gegen unß zusammen .....
Eine der Sorgen, die die Freunde von außen trafen, war,
daß man die Art, wie sie sich verheirateten, beanstandete. So
kam zum Beispiel folgender Fall vor daß Gericht von Notting-
ham: etwa zwei Jahre vorher hatten sich zwei auß der Gemein-
schaft der Freunde geehelicht; da starb der Mann und hinterließ
der Frau, die guter Hoffnung war, einen Besitz an Land und Zinß-
lehen. Alß daß Kind geboren war, erklärte eß daß Gericht alß
Erbe seineß Vater-Z, und eß wurde alß solcher anerkannt. Später
heiratete ein anderer Freund die Witwe. Daraufhin kam ein
naher Verwandter deß ersten Manneß und verklagte den Freund,
der die Witwe geheiratet, und suchte ihm seinen Besitz zu entreißen
und daß Kind seineß Grbeß zu berauben und alleß an sich zu
bringen, alß nächster Erbe deß ersten Manneß. Um dieß zu
begründen, suchte er die illegitime Geburt deß Kindeß zu beweisen
mit der Behauptung, die Ghe sei nicht nach dem Gesetz gewesen.
Bei den Verhandlungen gebrauchte der Kläger ungebührliche Auß-
drücke gegen die Freunde und sagte, sie täten sich zusammen wie
daß Vieh; und andere scheußliche Dinge. Nachdem die Anwälte


% \picinclude{./160-169/p_s166.jpg} 

166 Kapitel R17.
beider Parteien gesprochen hatten, nahm der Richter die Sache
in die Hand und sagte, eö sei eine Ehe im Paradies geschlossen
worden, ale- der Adam die E-oa und die Eva den Adam genommen
hatte; es sei eben die Zustimmung der beiden Teile, maß eine
Ehe au?-mache. Was die Quäker anbelange, so kenne er ihre
Ansichten nicht, aber er glaube nicht, daß sie sich zusammentun
wie das unvernünftige Vieh, wie man von ihnen behaupte, sondern
wie Christen, und darum glaube er, die Ehe sei gesetzlich gewesen-
und daß Kind legitimer Erbe. Um daß E-nicht zu überzeugen,
brachte er einen anderen Fall: Gin Mann, der schwach und bett-
lägerig war, hatte in diesem Zustande den Wunsch, sich zu ver-
ehelichen und erklärte vor Zeugen, daß er diese Frau zum
Weibe nehme und die Frau erklärte, daß sie diesen zum Mann
nehme; diese Ehe wurde später angefochten, aber alle Bischöfe
erklärten damalö die Ehe für gültig. Daraufhin entschied daß
Gericht auch zu Gunsten des Quäkerkindeß, gegen den Mann,
der etz um sein Erbe bringen wollte.
Um diese Zeit wurde der Suprematß- und Huldigungßeid
von den Freunden gefordert alö eine Falle, denn man wußte,
daß wir nicht schwören konnten, und ez wurden in der Folge
viele gefangen gesetzt. Bei dieser Gelegenheit veröffentlichten
die Freunde die Schrift: »Die Gründe und Ursachen, warum wir
nicht schwören« und es trieb mich, derselben einige Linien bei-
zufügen, damit man sie dem Magistrate gebe:
»Die Welt sagt: ,,Küsse das Vuch« 1); daß Buch aber sagt:
,,Küsse den Sohn, daß er nicht zürne« (Ps. 2, 12). Der Sohn sagt:
»Bleibet bei Ja und Nein in euren Reden, denn maß darüber ist,
daß ist vom Ubel« (Matth. 5, 37). Wiederum sagt die Welt: »Leget
die Hand auf daß Buch«; aber daß Buch sagt: ,,WaS unsre Hände
betastet haben vom Worte dez Leben-i-« (1. Joh. 1, 1) .... Und
Gott sagt: »Dieö ist mein lieber Sohn, den sollt ihr hören«;
er ist daß- Leben, die Wahrheit, daß Licht und der Weg zu Gott.«
G. F.
Weil so viele Freunde gefangen waren, verfaßten Richard
Hubberthorn und ich eine Schrift und ließen sie dem König überreichen,
damit er erfahre, wie wir von seinen Beamten behandelt wurden
sie lautete:
1) Aus der Formel beim Schwören des Eides.


% \picinclude{./160-169/p_s167.jpg} 
Beginn neuer Quäketversolgungen bei Anlaß der Verschwörungen usw. 167
An den König:
»Fre1md,
Der du der Herrscher dieses Reiches bist! Hier ist eine Auf-
zählung eines Teiles der Leiden, die das Volk Gottes, das man
im Ärger Quäker nennt, zu erdulden hat. Unter dem Wechsel
der Mächte, die deiner Regierung vorangingen, haben sie viel
gelitten; 3170 wurden gefangen genommen um des Gewissens
millen, und weil sie Zeugnis ablegten für die Wahrheit, die in
Christus ist; und noch jetzt sind 73 Personen im Namen des Common-
wealth gefangen; 32 Personen starben im Gefängnis während
der Zeit des Commonwealth und unter Oliver und Richard, in
harter, grausamer Gefangenschaft, aus schmutzigem Stroh und in
gräulichen Löchern. Und 3068 Personen sind seit deiner Rückkehr
gefangen genommen worden durch solche, die sich damit bei dir
einzuschmeicheln suchten. Zudem werden unsre Versammlungen
täglich gestört durch Männer mit Waffen und Kntitteln, obwohl
mir friedlich zusammenkommen, nach der Art des Volkes Gottes
der ersten Zeiten; unsre Freunde werden ins Wasser geworfen
und werden blutig geschlagen; ja, es können gar nicht alle die
Gräueltaten aufgezählt werden. Nun möchten wir gerne von dir
erbitten, daß du alle, die im Namen des Commonwealth und im
Namen der beiden Protektoren und in deinem eigenen Namen
um des Gewissens und der Wahrheit willen gefangen sind, frei
gebest; haben sie doch nie die Hand erhoben gegen dich oder
irgend sonst jemand, und daß, wenn sich die Freunde friedlich
versammeln, um Gott anzubeten, sie nicht mehr durch rohe Be-
wassnete gestört werden. Gin Hauptgrund dieser frühern Gefangen-
nahme war der, daß wir den Protektoren und den verschiedenen
Regierungen keine Eide leisten konnten; und nun tut man uns
ins,Gefängnis, weil wir den Huldigrmgseid nicht leisten können.
Z Wenn nun dir oder irgend einem Menschen gegenüber unser
ja nicht ja und unser nein nicht nein sein sollte, dann laß uns
dafür das leiden, was andere leiden müssen, wenn sie einen Eid
brechen.! Wir haben alle diese Jahre viel gelitten an unserm
eigenen Leib und an unserer Habe, unter mancherlei Regierungen,
weil wir nicht schwören, sondern Christi Gebot folgen, das sagt,
,,ihr sollt überhaupt nicht schwören«; dieses besiegeln wir mit Leib
und Gut, mit unserm ja und nein, wie Christus es befiehlt.
Bedenke das in der Weisheit, die aus Gott ist, damit du in



% \picinclude{./160-169/p_s168.jpg} 

168 Kapitel R17.
derselben solchem Tun Einhalt gebietest, du, der du die Herrschaft
hast und solches oermagst. Wir möchten, daß alle, die jetzt im
Gefängni-8 sind, frei werden und nicht wieder um der Wahrheit
und des Gewissenß willen gefangen genommen werden. Und wenn
du untersuchst, ob sie unschuldig leiden, so laß ihre Ankläger vor
dich kommen, und wir wollen, wenn nötig, au?-führlich Bericht
über ihre Leiden erstatten.« G. F. und R. H .....
Zwei Freunde, beides Frauen, waren auf Malta bei der
Jnquisition gefangen, Katharine Gvanß und Sarah Chevertz; da
es hieß, ein Lord D'Aubenh, ein römisch-katholischer Priester,
könne ihnen die Freiheit verschaffen, so ging ich zu ihm. Nach-
dem ich ihn über alletz, was ihre Gefangennahme betraf, unter-
richtet hatte, bat ich ihn, an die dortigen Behörden um ihre
Freilassung zu schreiben. Gr versprach bereitwilligst, ez zu tun
und daß, wenn ich in einem Monat wieder komme, man mir
ihre Freisprechung mitteilen wolle. Als ich zur bestimmten Zeit
wieder hinkam, sagte er, sein Brief sei scheintz nicht angekommen,
denn er habe keine Antwort erhalten, aber er versprach, nochmalß
zu schreiben, und tat es auch, und sie wurden beide frei.
Mit diesem hohen Herm redete ich viel über Religion, und
er gab zu, daß Christus jeden, der in die Welt kommt, erleuchtet
mit seinem geistigen Licht, und daß er den Tod für einen jeden
gekostet hat, und daß die heilfame Gnade Gottes allen Menschen
erschienen ist und sie lehrt und ihnen daß Heil bringt, wenn sie
ihr gehorchen. Jch fragte ihn darauf, wozu denn die Papisten
alle ihre Bilder und Reliquien brauchen, wenn sie an dieseö Licht
glauben und die Gnade, die sie lehrt und ihnen daß Heil bringt,
annehmen? Er antwortete, das seien nur Mittel, um daß
Volk in Unterwürsigkeit zu erhalten. Gr zeigte sich in dieser
Unterredung sehr weitherzig; ich hörte nie einen Papisten soviel
zugeben wie diesen .....
Jm gleichen Jahre, alö ich in Eambridgeshire war, hörte ich,
daß Edward Burrough gestorben war; und da ich wußte, wie
schwer und traurig dieser Verlust für die Freunde war, schrieb
ich folgende Zeilen zur Aufrichtung und Beruhigung ihrer Ge-
müter:
,,Freunde
Seid stille und ergeben und gefaßt im Samen Gottes, der
sich nicht ändert, damit ihr den lieben Edward Burrough unter


% \picinclude{./160-169/p_s169.jpg} 

Ein Gottesgerichi. Verhaftung wegen angeblicher Verschwörung usw. 169
euch spüren möget in diesem Samen, durch den er euch bei Gott,
bei dem er jetzt ist, vertreten wird; durch diesen Samen könnet
ihr ihn alle sehen und fühlen, denn in diesem ist Einigkeit und
Leben; freuet euch seiner im unoergänglichen Leben, das unsicht-
bar ist.« .... G. F.
Kapitel Ill
Ein Gottesgericht. Verhaftung wegen angeblicher Verschwörung
und schreckliche Gefangenschaft in Lancaster und Searlvo. Disput
im Gefängnis mit Baptisten und andern. Fox steht den Brand
von London voraus.
Wir gingen nach Tenterden und hatten dort eine Versamm-
lung, zu der viele Freunde aus der Umgegend kamen. Nach der
Versammlung ging ich ein wenig mit Thomas Briggs ins Freie,
während man unsre Pferde bereit machte. Als wir uns um-
wandten, sahen wir einen Hauptmann und einen Hausen Soldaten
mit geladenen Gewehren aus uns zukommen; einige von ihnen
hießen uns zu ihrem Hauptmann kommen. Als wir vor ihn
traten, fragte er: ,,Welcher ist George Fr-x?« Jch erwiderte:
»Jch bin es«. Da trat er auf mich zu und sagte: ,,Jch werde
dafür sorgen, daß dir nichts geschieht bei diesen Soldaten.« Dar-
auf tief er sie und hieß sie, mich festnehmen; auch Thomas
Briggs und unsern Hauswirt nahmen sie fest, aber die Kraft des
Herm war mächtig über ihnen. Nun kam der Hauptmann
wieder zu mir und sagte, ich müsse mit ihm in die Stadt;
er war ganz höflich mit mir und hieß die Soldaten mit den
andern nachkommen. Jch fragte ihn unterwegs, warum er
eigentlich solches tue, denn es war mir schon lange nichts der-
artiges mehr vorgekommen, und ich ermahnteihn, s eine Mitmenschen,
wenn sie ruhig leben, doch auch in Ruhe zu lassen. Als wir in
die Stadt kamen, brachten sie uns in eine Herberge, die zugleich
das Haus des Kerkermeisters war. Und bald darauf kam der
Bürgermeister und jener Hauptmann und einer seiner Leute, die
auch Friedensrichter waren, und fragten mich, warum ich herge-
kommen sei um Unruhe zu stiften? Jch erwiderte, ich sei nicht
gekommen, um Unruhe zu stiften und habe das auch nicht getan.
Sie sagten, es gebe aber ein Gesetz speziell gegen Quäkerver-
sammlungen. Jch antwortete, ich wisse von keinem derartigen


% \picinclude{./170-179/p_s170.jpg} 
A
170 Kapitel Ill.
Gesetz. Darauf brachten sie die Verordnungen gegen Quäker und
andere. Jch sagte, das gehe ja gegen solche, welche die Unter-
tanen des Königs gefährden und Grundsätze haben, welche der
Qbrigkeit gefährlich seien; also gehe es nicht gegen uns, denn wir
hätten keine der Obrigkeit gefährlichen Grundsätze und unsere
Versammlungen seien friedliche. Sie behaupteten, ich sei ein
Feind des Königs. Jch antwortete: ,,Wir lieben jedermann und
sind niemands Feind; was mich betrifft, so bin ich ins Ge-
fängnis zu Derby gebracht worden, weil ich nicht wollte die
Waffen gegen den König nehmen, und nachher bin ich von Oberst
Hacker nach London gebracht worden als ein Mitoerschworener
für die Rückkehr König Karls und dort gefangen gewesen, bis
Oliver mir die Freiheit schenkte«. Sie fragten mich, ob ich während
des Aufstandes gefangen gewesen sei? Ich sagte: »Ja, ich war
damals gefangen und seither wieder und erhielt die Freiheit auf
des Königs Vesehl«. Ich erklärte ihnen die Verordnung und
machte sie aus die letzte Kundmachung des Königs aufmerksam und
brachte ihnen Beispiele von andern Friedensrichtern und was
das Oberhaus darüber gesagt hatte. Ich redete auch mit
ihnen über ihren Seelenzustand und ermahnte sie, in der Furcht
Gottes zu wandeln und gegen ihre gottesfürchtigen Mitmenschen
mild zu sein und auf Gottes Weisheit zu achten, durch welche alle
Dinge geschaffen seien, damit diese Weisheit ihnen zu teil werde
und sie leite, sodaß sie in derselben alles zu Gottes Ehre
regieren möchten. Sie verlangten, daß wir uns verpflichten
sollten, bei der nächsten Gerichtssitzung zu erscheinen, aber wir oer-
weigerten jegliche Verpflichtung auf Grund unserer Unschuld. Dar-
aus wollten sie uns versprechen machen, nie mehr hierher zu
kommen, aber wir ließen uns auch darauf nicht ein. Als sie sahen,
daß sie nichts erreichten, sagten sie, sie wollten uns zeigen, daß
sie gewillt seien, uns höflich zu behandeln; der Bürgermeister habe
nämlich die Güte, uns die Freiheit zu schenken. Ich erwiderte,
ihr hösliches Benehmen bekunde eine anständige Gesinnung, und
so gingen wir von dannen .....
Joseph Hellen und G. Vewley waren im Loo gewesen, um
Blanch Pope, eine Rantersrau, zu besuchen, angeblich um sie zu
bekehren; aber ehe sie sie wieder verließen, waren sie so verstrickt
in ihre Ansichten, daß sie fast im Begriffe schienen, eher ihre An-
hänger zu werden, besonders Joseph Hellen. Sie hatte sie unter


% \picinclude{./170-179/p_s171.jpg} 
Ein Gotteßgericht. Verhaftung wegen angeblicher Verschwörung usw. 171
anderm gefragt: ,,Wer machte den Teufel? war es- nicht Gott?«
Diese einfältige Frage verblüffte die Beiden so, daß sie nicht ant-
worten konnten. Sie legten mir nachher die Frage vor, ich ver-
neinte sie, ,,denn« sagte ich, ,,alleS waß Gott machte, war gut,
und der Teufel ist nicht gut; er hieß Schlange, ehe er Teufel und
Feind hieß, und darnach wurde er Teufel genannt. Später
wurde er Drache genannt, weil er ein Zerstörer war. Der Teufel
blieb nicht in der Wahrheit (Joh. 8,44) und als er die Wahrheit
verließ, wurde er der Teufel. Von den Juden hieß eß, alß sie
die Wahrheit verließen, sie seien vom Teufel, und man nannte
sie Schlangen (Matth. 23). Für den Teufel gibt ez keine Ver-
heißung, daß er je wieder zur Wahrheit zurückkehren werde, aber
für die Menschen, die von ihm verführt werden, steht die Ver-
heißung, daß der Same de-J Weibeß der Schlange den Kopf zer-
treten und ihre Macht zertrümmern werde (1. Mos. 3). Nachdem
diese Fragen auöführlich zur Beruhigung der Freunde erörtert
worden waren, sahen sich die Beiden, die den Geist der Runterz-
frau hatten aufkommen lassen, von der Wahrheit gerichtet; der
eine, Joseph Heilen, wandte sich ganz von unz ab und die Freunde
erkannten ihn nicht mehr alö zu ihnen gehörend; der andre da-
gegen, George Bewley, wurde wieder zurückgewonnen und wurde
später recht brauchbar .....
Ich hörte von einem Oberst Robinson in Cornwall, einem
bösen Menschen, der bei der Rückkehr dez Königs- zum Friedenß-
richter gemacht worden war, daß er die Freunde grausam ver-
folge nnd oiele von ihnen inS Gefängniß getan habe; alß er hörte,
daß ihnen durch die Gunst dez Kerkermeiftertz einige kleine Frei-
heiten zugestanden wurden und sie auögehen durften, um Weib
und Kinder zu sehen, erhob er dezwegen beim Gericht eine An-
klage gegen den Kerkermeister, und dieser mußte eine Buße von
20 Pfund bezahlen, und die Freunde wurden einige Zeit sehr knapp
gehalten. Nach der Gerichtßsitzung schickte dann Robinson zu
einem benachbarten Friedenörichter und ließ ihm sagen, er solle
ihm helfen, auf diese Fanatiker Jagd zu machen. An dem Tage,
als sie MM ihr Vorhaben au?-führen wollten, schickte, er feinen
Knecht mit den Pferden Vorauß und ging zu Fuß von seiner
Wohnung nach einer Farm, auf der er seine Kühe und seine
Milchwirtschaft hatte und wo seine Knechte und Mägde gerade
am Melken waren. Ab?. er kam, fragte er nach dem Stier; die


% \picinclude{./170-179/p_s172.jpg} 
172 Kapitel ZI?.
Mägde sagten, sie hätten ihn auf dem Felde eingesperrt, weil er
störrig sei bei den Kühen und sie am Melken hindere. Da ging
er ins Feld und begann nach seiner Gewohnheit seinen Stock gegen
den Stier zu schwingen, der Stier schnaubte nach ihm und holte
nach rückwärts aus, dann kehrte er sich und rannte wütend auf
ihn los und bohrte ihm die Hörner in die Seite, nahm ihn auf
die Hörner, schleuderte ihn über sich hinweg und riß ihm die Seite
aus bis zum Bauch, dann wiihlte er mit den Hörnern im Boden
und brüllte und leckte seines Herrn Blut aus. Als eine der
Mägde den Herrn schreien hörte, rannte sie ins Feld, packte den
Stier bei den Hörnern und riß ihn von ihrem Meister weg.
Der Stier stieß sie ganz sanft mit seinen Hörnern zur Seite, ohne
ihr weh zu tun, und ließ nicht ab, sein Opfer zu durchstechen und
sein Blut auszulecken. Nun rannte sie davon und holte ein paar
Männer, die in einiger Entfernung arbeiteten, um ihrem Meister
zu helfen. Aber es gelang ihnen erst den Stier wegzubringen, als
sie die Kettenhunde auf ihn hetzten, da rannte er wutschnaubend
davon. Als die Schwester Robinsons hörte, was geschehen, kam
sie heraus und sagte: »Ach, Bruder, welch schweres Gericht hat
dich betrosfen!« Er antwortete: »Ja wahrlich ein schweres Ge-
richt! laß den Stier töten und sein Fleisch den Armen geben.««
Sie brachten ihn nach Hause, aber er starb bald darauf. Der
Stier war so wild geworden, daß sie ihn erschießen mußten, denn
niemand konnte sich ihm nähern, um ihn zu töten. So gibt der
Herr ost Beweise seines gerechten Gerichts über die Verfolger
seines Volkes, aus daß man sich fürchte und sich in acht nehme. . .
Jch kam nach Swarthmore, wo man mir sagte, Oberst Kirby
habe seine Leute geschickt, um mich festzunehmen. Während der
Nacht, als ich in meinem Bett lag, trieb mich der Herr, am
nächsten Tage nach Kirbyhall zu Oberst Kirby zu gehen, fast zwei
Stunden weit, um mit ihm zu reden; ich ging denn auch ....
und sagte ihm, ich hätte gehört, er wolle etwas von mir, ob er
irgend etwas gegen mich habe? Er sagte vor allen Anwesenden,
daß er ein Gentleman sei und darum nichts gegen mich habe,
hingegen solle Mistreß Fell keine Versammlungen in ihrem Hause
haben, das sei gegen die Verordnungen. Ich erklärte ihm, diese
Verordnungen tressen nicht uns, sondern die, welche sich ver-
sammeln, um Komplotte und Verschwörungen zu machen; ....
die, welche sich bei Margaret Fell versammelten, seien friedliche


% \picinclude{./170-179/p_s173.jpg} 
Ein Gotteögericht. Verhaftung wegen angeblicher Verschwörung usw. 173
Leute. Nachdem wir längere Zeit miteinander geredet, gab er
mir die Hand und wiederholte, daß er nichtß gegen mich habe.
Sv kehrte ich nach Swarthmore zurück ..... Bald darauf
ging Oberst Kirby nach London in eine Privatsitzung der Richter
in Holkerhall, und dort wurde ein Verhaftbefehl gegen mich auf-
gesetzt . . . Jch hörte davon und hätte gut entwischen können, . . .
aber da das Gerücht ging von einer Verschwörung, so fürchtete
ich, sie würden, wenn ich mich davon machte, über die Freunde
herfallen, wenn ich aber bleibe, so würden sie mich nehmen, und
die Freunde könnten sich eher davon machen, und ich blieb also . , .
Am folgenden Tage kam ein Beamter mit Pistole und Schwert.
Jch sagte ihm, ich wisse, warum er komme, und sei dageblieben,
um mich festnehmen zu lassen; . . . ich verlangte, daß er mir den
Befehl zeige, aber er weigerte sich. So ging ich mit ihm, und
Margaret Fell begleitete unß nach Holkerhall ..... Dort wurde
mir unter anderem der Supremat?-eid vorgelegt; alß ich ihn nicht
schwören wollte, verlangten einige, daß ich inß Gefängniß von
Lancaster geschickt werde, andere wollten nur, daß ich verspreche
an der Gericht?-sttzung zu erscheinen, worauf ich entlassen wurde,
und ich kehrte also wieder mit Margaret Fell nach Swarthmore
zurück.
Am Gerichtßtage ging ich wie verabredet war, nach Lancaster . .
Der alte Richter Ratvlinson, der Vorsitzende, fragte mich, ob ich
um die Verschwörung wisse? Jch sagte, ich habe in Yorkshire
davon gehört. Gr fragte mich, ob ich etz den Behörden ange-
zeigt? Jch erwiderte, ich hätte ja Schriften gegen Verschwörungen
geschrieben ..... Sie legten mir den Suprematö- und Huldi-
gungßeid vor; ich sagte ihnen, daß ich nicht schwören könne, weil
Christuß und seine Apostel es- verboten hätten, und sie hätten ja
schon genugsam erfahren, wie ez bei solchen gehe, welche schwören,
ich aber habe noch nie in meinem Leben einen Eid geleistet. Hierauf
fragte mich Rawlinson, ob ich es für gesetzwidrig halte, zu
schwören? Diese Frage stellte er absichtlich, um mich zu fangen;
denn es war eine Verordnung gemacht worden, daß alle, die
sagen, ez sei gesetzwidrig zu schwören, verbannt oder hart bestraft
würden. Aber weil ich die Falle merkte, vermied ich sie und er-
klärte ihm, daß in den Tagen des Gesetzeß, bevor Christuö ge-
kommen sei, daß Gesetz den Juden geboten habe, zu schwören
(3. Mos. 19); Christuz aber, der in den Tagen dez Evangeliums


% \picinclude{./170-179/p_s174.jpg} 
174 Kapitel IV.
das Gesetz erfüllte, befehle, überhaupt nicht zu schwören (Matth. 5),
und der Apostel Jakobuö verbiete daß Schwören selbst denen, die
Juden waren und daß Gesetz Gotteß hatten. Nach vielem Hin-
und Herreden riefen sie den Gesangenwärter und verurteilten mich
zum Gefängniö. Ich trug die Schrift bei mir, die ich gegen
Verschwörungen geschrieben hatte, und bat, daß man sie vor dem
ganzen Gericht?-hofe vorlese oder lesen lasse, aber sie wollten nicht.
Als ich nun solchermaßen eingesperrt war, dafür, daß ich mich
geweigert hatte zu schwören, war mir daran gelegen, daß sie und
alle Leute wissen möchten, daß ich um der Lehre Christi willen
leide und darum, daß ich seine Gebote gehalten. Ich hörte später,
daß die Richter sagten, sie hätten besondere Befehle vom Oberst
Kirby gehabt, mich zu verfolgen, trotz seinem schönen Benehmen
und seiner anscheinenden Freundlichkeit damalß, als er vor allen
Anwesenden erklärt hatte, er habe nichts- gegen mich ....
Ich wurde biß zur Gerichtöverhandlung gefangen gehalten, und
da Richter Turner und Richter Twiszden gerade an der Reihe waren,
wurde ich vor Richter Twiötden gebracht, am 14. Tage desk-
Monatö, den man März nennt, im Iahre 1663. A15 ich vor-
geführt wurde, sagte ich: ,,Friede sei mit euch allen«. Der
Richter sah mich an und fragte: ,,Warum kommst du hier vor
Gericht mit dem Hut aus dem Kopf?« A13 der Kerkenneister mir
ihn hieraus wegnahm, sagte ich: ,,Da3 Hutabnehmen ist doch nicht
eine Ehre, die vor Gott gilt!« Daraus fragte mich der Richter:
,,Wollet ihr den Huldigungßeid leisten, George Fox?« Ich er-
widerte: ,,Jch habe nie in meinem Leben einen Eid geleistet, noch
mich zu irgend einem Vertrag oerpfiichtet«; darauf fragte er:
»Wollt ihr schwören oder nicht?« Ich erwiderte: »Ich bin ein
Christ, und Ehristuö befiehlt, nicht zu schwören, ebenso der Apostel
Iakobuß, und ob ich Gott oder Menschen gehorchen soll, darüber
urteile du selbst«. Er sagte: ,,Ich frage euch nochmals, ob ihr
schwören wollt oder nicht?« Ich antwortete abermaltz: ,,Ich bin
weder Türke, noch Jude, noch Heide, sondem ein Christ und
werde mich zum Christentum bekennen«. Und darauf fragte ich
ihn, ob er nicht wisse, daß die Christen der ersten Zeiten unter
den 10 Verfolgungen, sowie auch einige Märtyrer in den Tagen
der Königin Maria sich weigerten zu schwören, weil Christsuö
und die Apostel es verboten hätten; ferner sagte ich ihm, sie
hätten ja genugsam die Erfahrung gemacht, wie viele zuerst dem


% \picinclude{./170-179/p_s175.jpg} 
Ein Gotteögericht. Verhaftung wegen angeblicher Verschwörung usw. 175
König geschworen hatten und nachher gegen ihn; waz mich be-
treffe, so habe ich nie in meinem Leben einen Eid geleistet, und
meine Huldigung bestehe nicht im Leisten eineß Eideö, sondern
darin, daß ich Wahrheit und Treue halte,. denn, sagte ich, ich
ehre jedermann, wieoielmehr denn den König. Ehristuß aber,
der große Prophet und der König aller Könige und Heiland der
Welt, der große Richter der ganzen Erde, hat gesagt, daß man
nicht schwören soll, soll ich nun Christu?7 oder dir gehorchen?
Denn etz geschiehet au-H Gewissens:-’zartheit und auz Gehorsam gegen
Christi Gebote, daß ich nicht schwöre, und wir haben ja ein
König?-wort für zarte Gewissen. Daraus fragte ich den
Richter, ob er den König anerkenne: ,,Ja«, sagt er, ,,ich aner-
kenne den König«. ,,Warum«, fragte ich, ,,befolgst du denn dann
nicht seinen Erlaß von Breda und seine Versprechen, die er bei
seiner Rückkehr machte, daß niemand um der Religion willen ver-
folgt werde, solange er ruhig lebe? Wenn du den König aner-
kennst, warum verfolgst du mich, verhöhnst mich und treibst mich
dazu, einen Eid zu leisten, maß doch Sache des Glaubenß ist, und
siehst doch, daß weder du noch sonst jemand mich eineß unfried-
lichen Lebens zeihen kann''. Hierauf wurde er sehr gereizt und
sagte: ,,Kerl, wollt ihr schwören!« Jch sagte darauf, ich sei keiner
seiner ,,Kerl=Z«, sondern ein Christ und es stehe einem alten Richter nicht
an, hier zu sitzen und den Gefangenen Spottnamen zu geben,
weder seinen grauen Haaren noch seinem Amt. Darauf sagte er:
»J—ch bin auch ein Ehrist«. ,,So handle auch christlich«, sagte ich;
»Kerl«, sagte er, »willst du mir mit deinen Reden Angst machen?
Aber«, fügte er Verlegen hinzu, ,,jetzt brauche ich ja dieses- Wort
wieder!« und er bezwang sich. Ich sagte: ,,Jch rede in Liebe so
mit dir, weil eine solche Sprache dir als; Richter nicht ansteht.
Du solltest deinem Gefangenen das- Gesetz erklären, wenn er un-
wissend ist und einen oerkehrten Weg geht«. ,,Jch rede ebenfalls-
in Liebe mit dir«, sagte er: ,,aber«, erwiderte ich, ,,die Liebe ge-
braucht keine Spottnamen«. Daraus erhob er sich und sagte: »Jch
lasse mich nicht von dir einschüchtem, du sprichst so laut, deine
Stimme iibertäubt die meinige uud alle andern, ich müßte drei
oder oier Au?-rufer kommen lassen, um dich zu übertönen, du hast
gute Lungen«. Ich erwiderte: ,,Jch bin hier gefangen um Jesu
willen, um seinetwillen leide ich und stehe ich heute hier, und
wenn meine Stimme fünfmal so laut wäre, so würde ich sie er-


% \picinclude{./170-179/p_s176.jpg} 
176 Kapitel RV.
heben und erschallen lassen sür Ehrisiu?-, für dessen Sache ich
heute vor dem Richtstuhl stehe im Gehorsam gegen Christus,
welcher gebietet, nicht zu schwören, vor dessen Richtstuhl ihr alle
stehen und Rechenschaft ablegen müßt«. ,,So antworte mir nun
George For«, sagte er, »ob du den Eid leisten willst oder nicht«.
Ich erwiderte: ,,Jch frage dich nochmalß, ob ich Gott oder den
Menschen gehorchen soll? beurteile du daß selber. Wenn ich
überhaupt einen Eid leisten wollte, so wäre es dieser; aber ich
leugne überhaupt alle Eide, nicht nur den oder jenen, nach der
Lehre Christi, der seinen Nachfolgern gebot, überhaupt nicht zu
schwören. Wenn nun du oder sonst jemand von euch, oder eure
Prediger oder Priester mir beweisen wollen, daß Christud oder
seine Apostel irgend einmal, nachdem sie alleß Schwören verboten
hatten, ez den Christen wieder geboten, so will ich schwören«.
Ich sah, daß verschiedene Priester zugegen waren, aber nicht ein
einziger wollte reden. »Nun denn«, sagte der Richter, »ich bin
ein Diener des König;3 und der König hat mich nicht geschickt,
um mit dir zu dißputieren, sondern daß Gesetz an dir auözuüben;
legt ihm also den Huldigungßeid oor«. »Wenn du den König
lieb hast«, sagte ich, »warum hälst du dich nicht an daß, watz er
sagt? und an seine Erklärung, in der er unß Gewissenßsreiheit zu-
gesagt hatte? Ich bin ein Mann mit einem zarten Gewissen und
kann auß Gehorsam gegen Christi Gebot nicht schwören«. »Wenn
er also nicht schwören will'', sagte der Richter, »so führet ihn
in den Kerker«. Ich sagte, eß sei um Christi willen, daß ich
nicht schwören könne, ihm müsse ich gehorchen; aber der Herr
möge ihnen allen vergeben. So führte mich der Kerkermeister
hinweg, aber ich sühlte, daß deß Herrn mächtige Kraft über ihnen
allen war .....
Während ich nun hier im Kerker war, trieb setz mich, an
Richter Flemming, einen der hestigsten Verfolger der Freunde,
folgendermaßen zu schreiben: »O, Richter Flemming! Barmherzig-
keit, Milde und Güte zieret die Menschen und auch die Behörden.
O, hörest du nicht daß Schreien derer, die durch die Versolgungen
Witwen und Waisen geworden sind? Sind sie nicht wie
Schafe von Konstabler zu Konstabler getrieben worden, wie wenn
sie die größten Übeltäter und Bösewichter im Lande wären? GS
betrübt die Herzen oieler einsichtiger Leute, zu sehen, wie man
ihre ehrlichen Mitmenschen, die ein friedsameß, stilleß Leben ge-


% \picinclude{./170-179/p_s177.jpg} 
Ein Gottesgerirht. Verhaftung wegen angeblicher Verschwörung usw. 177
ssührt, behandelt hat. Wieder ist einer gestorben, den ihr ins Ge-
fängnis geworfen; er hat fünf Kinder hinterlassen, die nun ver-
waist sind. Solltest du nun nicht für diese vaterlosen Kinder
sorgen, sowie auch für die Weiber und Hinterlassenen der andern?
Jst es nicht deine Pflicht? Denke an Hiob, Kap. 29: »Gr war
ein Vater der Armen; er errettete den Armen, der da schrie und
die Waisen, die keinen Helfer hatten, er brach die Kinnbaeken des
Ungerechten und riß den Raub aus seinen Zähnen.« Und nun
vergleiche dein Leben mit dem seinen und hüte dich vor dem Tage
des Gerichts, welcher kommen wird, und vor dem Urteil Christi,
wenn ein jeder muß Rechenschaft ablegen und den Lohn empfangen
für seine Taten. Als-dann wird es heißen; o, wo sind die ver-
lorenen Tage! — Als John Stubbs vor dich gebracht wurde, der
ein Weib und vier kleine Kinder hatte, und mit seiner Hände
Arbeit nur den dürstigsten Lebensunterhalt verdiente, da riesest
du: fordert diesem Menschen den Eid ab! und als er dir vor-
stellte, daß er ein armer Mann sei, ließest du kein Mitleid auf-
kommen und wolltest ihn nicht hören, und nun ist er im Ge-
fängnis, weil er nicht schwören konnte, also nicht das Gebot
Christi und der Apostel übertreten konnte. Hoffentlich wirst du für
seine Familie sorgen, damit seine Kinder nicht Hungers sterben.
Jst denn das dem König gehuldigt, wenn man tut, wovon Christus
und die Apostel sagen, es sei Unrecht und führe in die Verdamm-
nis? Jhr würdet wohl auch Christus und die Apostel, welche
das Schwören verboten, ins Gefängnis geworfen haben, wenn sie
zu eurer Zeit gelebt hätten.
Denke auch an deinen armen Mitmenschen William Wilson,
der allgemein als ein fleißiger Mann bekannt war, und der sein
Weib und seine Kinder ehrlich durchbrachte, obgleich er nichts be-
saß, als was er durch seiner Hände Arbeit erwarb. Sogar auf
den Märkten wird über den Tod dieser Beiden geredet; man hört
das Schreien derer, die um der Gerechtigkeit willen Witwen und
Waisen geworden sind. Wenn John Stubbs und William Wil-
son geschworen hätten, so hätten sie damit ihre Freiheit wieder
erlangt, wenn sie auch daneben es mit den Marktschreiern und
Schnurranten gehalten hätten. O gehet in euch! es ist solches
nicht nach des Herrn Sinn. Und auch der König hat erklärt, es
solle gegen keinen seiner Untertanen, der friedlich lebe, eine Grau-
samkeit ausgeübt werden. Sodann sind einigen sehr rechtschafsenen
George Fc;. 12


% \picinclude{./170-179/p_s178.jpg} 
178 Kapitel K7.
Leuten Bußen auferlegt worden, obgleich sie selber nichts hatten,
und es eher am Platze gewesen wäre, ihnen etwas zu geben, als
ihnen noch etwas zu nehmen. Weil du weißt, daß sie um ihrer zarten
Gewissen willen keinen Eid schwören können, so stellst du ihnen da-
mit eine Falle. Wie denkst du, daß das Volk über ein derartiges
Tun redet? Sie sagen: Wir wissen, daß die Quäker sich an ihr
ja und nein halten, andere dagegen sehen wir schwören und wieder
abschwören! Jch weise dich an den Geist Gottes in deinem Ge-
wissen, Richter Fleming, der du so eifrig die Gefangennahme
des George Fox betriebeft und so böse warst über die, die ihn
nicht gefangen nahmen. Wo ist dein Erbarmen mit den armen,
oerwaiften Kindern? Hüte dich vor der Grausamkeit des Herodes,
der kein Mitleid kannte; Esau hat es also gemacht und nicht
Jakob! Thomas Walters von Bolton ist auch hier im Gefäng-
nis und wird darin festgehalten, weil er sich nach Christi Gebot
weigert zu schwören, und dabei hat er fünf kleine Kinder und seine
Frau ist ihrer Niederkunst nahe; du solltest dich doch seiner an-
nehmen und dafür sorgen, daß seine Frau und seine Kinder nicht
Mangel leiden, da sie durch deine Schuld verwaist dastehn.
Klingt dir das Schreien der Verwaisten nicht in den Ohren, und
siehest du das Blut derer, die durch dich umgekommen sind, nicht
vor dir? Es wird dich am Tage des Gerichts ein schweres Ur-
teil treffen, wie willst du dich verantworten, wenn du nach deinen
Werken gerichtet werden wirst und vor den Richterstuhl des
Allmächtigen treten mußt? .... Aber trotz alledem sagen wir
Quäker: der Herr vergebe dir und rechne dir diese Dinge nicht
an, wenn es sein heiliger Wille ist.« G. F.
Bald darnach starb Richter Flemings Weib, und hinterließ
ihm dreizehn oder vierzehn mutterlose Kinder .....
Einige Zeit vorher war Margaret Fell auch von Richter
Fleming als Gefangene nach Lancaster geschickt worden, und als
ste, an der Gerichtssitzung, den Eid nicht schwören wollte, wurde
sie weiter zum Gefängnis verurteilt. ....
Während ich im Gefängnis zu Lancaster war, hieß es, der
Türke werde über die Christenheit herfallen, und viele kamen in
große Angst. Eines Tages, als ich in meiner Zelle auf und
nieder ging, kam es über mich vom Herrn, daß ich sah, wie die
Krast des Herrn sich gegen den Türken kehrte, sodaß er wieder
umkehren mußte, und ich teilte einigen mit, was der Herr mich


% \picinclude{./170-179/p_s179.jpg} 
Ein Gotteögericht. Verhaftung wegen angeblicher Verschwörung usw. 179
hatte sehen lassen, und binnen eineß Monatö kam die Nachricht,
daß er geschlagen worden war.!)
Ein andermal alö ich in meiner Zelle auf- und niederging
und zum Herrn aufschaute, sah ich den Engel dee- Herrn, wie er
mit einem leuchtenden Schwert gen Süden wieß, und daß ganze
Schloß schien in Feuer zu stehn. Nicht lange darauf brach der
Krieg in Holland aus, 2) und dann eine große Seuche und dann daß
Feuer in London ; 9-) da war wahrlich daß Schwert des Herm gezogen.
Durch die lange Gefangenschaft an diesem ungesunden Orte
war ich sehr angegriffen in meiner Gesundheit, aber die Kraft
des Herm war stärker alß alles, sie half mir hindurch und hielt
mich aufrecht und half mir für den Herrn wirken, so viel der
Ort es erlaubte. Ich antwortete denn auch während dieser Zeit
auf mehrere Bücher, wie: ,,die Messe«, ,,da;3 Eommon Prayer
Buch«, »daß Direetorrs«, ,,daH Kirchenbekenntni:-tz««, welcheß die
vier mächtigsten Religionen 4) sind, die sich seit den Tagen der
Apostel erhoben.
Nach der Gerichtßoerhandlung war es- einigen der Richter
etwas ungemütlich, daß ich in Lancaster war, denn ich hatte sie
bei den Verhandlungen tüchtig geärgert, und sie bemühten sich
sehr darum, daß man mich anderöwohin bringe .... Etwa sechß
Wochen nach der Gerichtsverhandlung erhielten sie denn auch den
Befehl vom König und dem Rat, mich von Lancaster fortzubringen,
und zugleich kam ein Brief vom Earl von Anglesea, worin ee;
hieß, daß, wenn alle-Z, dessen man mich beschuldigt hatte, wahr
sei, so verdiene ich keinerlei Nachsicht noch Milde. Und doch war
daß Ärgste, waß sie gegen mich vorgebracht hatten daß-’, daß ich
einem Gebot Christi nicht ungehorsam sein konnte .....
Sie brachten mich nun nach Schloß Scarbro, wo sie mich in
ein Gemach führten und mir einen zur Wache setzten. Da ich seht
schwach war und öfters ohnmächtig wurde, so ließen sie mich
« manchmal mit der Wache an die frische Luft gehen; nach einiger
Zeit brachten sie mich in ein andereß Gelaß, daß offen war, so
1) 1664 Sieg der Abendländet (Deutschland und Frankreich) über die
Türken bei der Abtei St. Gotthardt.
2) 1665 Krieg zwischen England und den Niederlanden.
3) Oktober 1665 die große Pest in London, September 1666 der große
Brand in London.
4) Römische, Bischässliche, Preßbhterianer und Jndependenten.
12*


% \picinclude{./180-189/p_s180.jpg} 
180 Kapitel R7.
daß etz herein regnete, und wo es schrecklich rauchte, waö mir
sehr schadete. Eines Tageß besuchte mich der Gouverneur Sir
John Crossland mit Sir Franeißt Cobb. Jch bat den Gouverneur,
mich in mein Zimmer zu begleiten, um zu sehen, waß daß für
ein Ort sei. Jch hatte ein kleineß Feuer darin angezündet, welches
nun derart rauchte, daß man seinen Weg schier nicht fand. Da
der Gouverneur ein Papist war, so sagte ich ihm, ez sei sein
Fegefeuer, daß sie mir zum Aufenthalt gegeben hätten. Jch mußte
etwa 50 Schilling ausgeben 1), um den Regen abzuhalten und zu
machen, daß eö nicht so stark rauchte. Und alß ich diese AUS-
gaben gemacht hatte, und es- etwaß erträglicher geworden, gaben
sie mir ein noch schlechtereö Gelaß, wo ich weder ein Kamin noch
irgend eine andere Vorrichtung, um Feuer zu machen, hatte. Da
ez gegen die See gelegen und sehr offen war, so trieb der Wind
den Regen ungehindert herein, sodaß dass Wasser biz zu meinem
Bett kam und im Zimmer herumlief, und ich ez mit einem Ge-
fäß ausschöpfen mußte. Und wenn meine Kleider naß waren, so
hatte ich kein Feuer, um sie zu trocknen, sodaß mein Körper ganz
erstarrt war vor Kälte, und meine Finger so geschwollen waren,
daß einer so groß war wie sonst zwei. Obgleich ich in diesem
Raum auch zu bezahlen hatte, so gelang es- mir doch nicht, Wind
und Regen abzuhalten .....
ES wurde den Freunden nicht gestattet, mich zu besuchen;
aber sonst führten sie hier und da jemanden zu mir, entweder
um mich anzusehen, oder um sich mit mir zu unterreden. Einmal
kam eine Schar Papisten, um mit mir zu dißputieren; sie behaup-
teten, der Papst sei unfehlbar und sei immer unsehlbar gewesen
seit Petrus- Zeit, aber ich bewietz ihnen das Gegenteil aus der
Geschichte: ein Bischof von Rom, Marcellinuß mit Namen, habe
den Glauben abgeschworen und den Götzenbildern gehuldigt, dieser
sei also nicht unfehlbar gewesen. IJch sagte ihnen, wenn sie den
tmfehlbaren Geist hätten, so bedürsten sie keiner Kerker, Schwerter,
Foltern, Scheiterhaufen, Geißeln und Galgen, um ihre Religion
aufrecht zu erhalten, denn wenn sie den unfehlbaren Geist hätten,
so würden sie die Leben der Menschen schützen, statt sie umzu-
bringen, und würden in Sachen der Religion nur geistliche Waffen
1) Die Gefangenen hatten zu der Zeit die Kosten ihres- Aufenthaltes in
den Gefängnissen selbst zu tragen (s. Aschrott, Engl. Gesüngniswesen).


% \picinclude{./180-189/p_s181.jpg} 
Ein Gottezgericht. Verhaftung wegen angeblicher Verschwörung usw. 181
brauchen. Jch erzählte ihnen auch, waö einer der Jhrigen mir
berichtet hatte: eine in Kent lebende Frau war nicht nur selber
Papistin gewesen, sondern hatte auch viele andere für ihren
Glauben gewonnen. Aber alß sie zur Wahrheit dez Herm be-
kehrt wurde und durch sie zu Jesuö Christuß ihrem Heiland kam,
ermahnte sie die Papisten, ein gleiches- zu tun, unter anderm
auch einen Schneider, der bei ihr in Arbeit war; sie zeigte ihm
die Verkehrtheit der pästlichen Religion und suchte ihn für die
Wahrheit zu gewinnen, da zog er sein Messer und stellte sich
zwischen sie und die Türe, aber sie trat ihm mutig entgegen und
ermahnte ihn, sein Messer weg zu tun, denn sie kannte setne
Grundsätze; auf die Frage, waß er wohl mit dem Messer ge-
macht hätte, antwortete die Frau: ,,er hätte mich erstochen«, und
aus die weitere Frage, ob er dieö wegen ihrer Religion getan
hätte, erwiderte sie: »ja, denn es- ist der Grundsatz der Papisten,
jeden, der von ihrer Religion abtrünnig wird, womöglich zu töten«.
Dieseö erzählte ich nun den Papisten und fügte bei, ich hätte e-5’
von jemand, der früher zu ihnen gehört, sich jedoch von ihnen
gewandt habe, weil er hinter ihre Handlung?-weise gekommen war.
Sie leugneten nicht, daß sie solche Grundsätze hätten, fragten
aber, ob ich nun solches?. weitererzählen werde? Jch erwiderte:
,,ja, denn solche Dinge müssen weitererzählt werden, damit man
ersährt, wie sehr eure Religion vom wahren Christentum abweicht«.
Darauf gingen sie sehr zornig fort. Gin anderer Papist, welcher
kam, um mit mir zu diöputieren, behauptete, alle Patriarchen seien
in der Hölle gewesen, biö Christus zur Hölle hinabgesahren sei,
da habe der Teufel gesagt: »waS kommst du hierher, unsere sichere
Burg zu sprengen?« und Christuß habe geantwortet, er komme,
um alle diese zu befreien, und sei drei Tage und drei Nächte in
der Hölle gewesen, um sie alle zu befreien. Jch erwiderte ihm,
daß sei unrichtig, denn Christuß habe ja zum Schächer gesagt:
,,heute noch sollst du mit mir im Paradiese sein.« Und Henoch
und Elias seien in den Himmel gekommen, auch Abraham, denn
ez heiße, Lazaruß sei in Abrahamtz Schoß gewesen, und Moseß
und Eliaö seien mit Jesus auf dem Berg gewesen, ehe er leiden
mußte. Diese Beispiele stopften dem Papisten den Mund und
brachten ihn in Verlegenheit.
Ein andermal kam Doktor Witty, ein berühmter Arzt, mit
Lord Falconbridge; mit ihnen kam auch der Gouverneur der


% \picinclude{./180-189/p_s182.jpg} 
182 Kapitel IV.
Festung Tynemouth und mehrere Adlige. Als ich zu ihnen ge-
rufen wurde, sing Witt:) ein Gespräch mit mir an und fragte mich,
warum ich im Gefängnis?. sei. Jch antwortete: »weil ich den Ge-
boten Ehristi nicht ungehorsam sein will«. Gr sagte, ich hätte
dem König den Treueid leisten sollen. Da er ein eifriger Preß-
byterianer war, so fragte ich ihn, ob er denn nicht zuerst gegen
den König und daß Unterhauß geschworen und sich zum
schottischen Covenant bekannt habe und seither wieder zum König
geschworen habe? watz denn dann daß Schwören niitze? Mein
Huldigungseid, fügte ich bei, bestehe eben nicht im Schwören,
sondern in Wahrheit und Treue. Nach einigem Hin- und Her-
reden wurde ich wieder in meine Zelle zurückgeschickt; nachher
prahlte dieser Arzt bei seinen Patienten in der Stadt herum, er
habe mich besiegt. Alö ich von seinem Prahlen hörte, sagte ich
dem Gouverneur, ez sei ein geringer Ruhm zu sagen, man habe
einen Gefangenen besiegt. Jch bat, man solle ihm sagen seinen
Besuch zu wiederholen, wenn er wieder ine; Schloß komme. Er
kam nach einiger Zeit wieder mit sechözehn oder siebzehn ange-
sehenen Leuten und erlitt eine noch größere Niederlage als das
erstemal; er behauptete nämlich, Ehristuß habe nicht alle, die in
die Welt kommen, erleuchtet, und die heilsame Gnade Gotteß sei
nicht allen Menschen erschienen, und Christuß sei nicht für alle
Menschen gestorben. Ich fragte ihn, was daß für Menschen
seien, die Ehristuö nicht erleuchtet habe, denen die heilsame Gnade
nicht erschienen sei, und für die er nicht gestorben sei? Er sagte,
die Ghebrecher, die Götzendiener, die Gottlosen. Jch fragte ihn,
ob die Ghebrecher und Gottlosen keine Sünder seien? Gr sagte,
doch. ,,Und starb nicht Ehristuz eben für die Sünder?« fragte ich,
,,kam er nicht, die Sünder zur Buße zu rufen?« Er sagte: »doch«.
,,Dann hast du dir selber daß Maul gestopft«, sagte ich. Hiermit
hatte ich bewiesen, daß die Gnade Gotteß allen Menschen er-
schienen ist, obgleich viele sie in Mutwillen kehren und ihr wider-
streben, und daß Christuß alle Menschen erleuchtet hat, wenn
schon Viele dasz Licht hassen. Manche der Anwesenden gaben zu,
daß dietz wahr sei, der Doktor aber ging fort und kam nie mehr
zu mir.
Ein andermal brachte der Gouverneur einen Priester zu
mir, aber sein Mund war bald gestopft. z’Bald daraus brachte
er zwei Parlamentsmitglieder, die mich fragten, Ob ich Prediger


% \picinclude{./180-189/p_s183.jpg} 
Ein Gottesgericht. Verhaftung wegen angeblicher Verschwörung usw. 183
und Bischöfe gelten lasse. Jrh erwiderte: ,,ja, solche die Christuß
sendet, die umsonst empfangen und umsonst geben, die dazu be-
stimmt sind und den Geist und die Kraft haben, welche auch die
Apostel hatten. Solche Bischöfe und Prediger aber wie eure, die
nichts tun, alö maß; ihnen ein guteß Einkommen bringt, die
lasse ich nicht gelten, denn sie sind nicht den Aposteln gleich.
Christus sagte zu seinen Jüngern: ,,Gehet hin in alle Welt und
predigt daß Evangelium umsonst.« Jhr Parlamentßmitglieder,
die ihr euren Bischöfen und Predigern so große Pfründen gebt,
ihr habt sie verdorben. Meinet ihr etwa, diese gehen zu allen
Völkern? oder überhaupt über ihre fetten Pfründen hinauß, um
zu predigen? Urteilt selber, ob sie daß tun oder nicht.«.
Ein andermal kam die Witwe von Lord Fairfax und viele
mit ihr, unter anderm auch ein Priester. ES trieb mich, ihnen
die Wahrheit zu verkünden; der Priester fragte mich, warum wir
,,du« und ,,dich« zu den Leuten sagen? denn er hielt uns für
Narren und Dummköpfe deßwegen. Ich fragte ihn, ob er finde,
die, welche die Schrift übersetzten rmd die Grammatik und Sprach-
lehre machten, seien Narren und Dummköpfe gewesen, weil sie
sie so übersetzten und lehrten, daß ,,du« für eine Person und
,,ihr« für mehrere gilt? wenn denn diese Narren und Dummköpfe
gewesen seien, warum denn dann nicht er und die, welche seine
Ansicht teilen und sich für weise halten, die Grammatik und Sprach-
lehre und die Bibel verbessern, und die Mehrzahl statt der Ein-
zahl setzen? Wenn es aber weise Männer gewesen seien, die die
Bibel übersetzten und die Sptachlehre und Grammatik machten,
so sollen sie sich fragen, ob nicht etwa sie die Narren und Dumm-
köpse seien, die nicht reden wie die Bibel und die Grammatik
lehre, sondern uns:-’, die ez tun, darum schelten? So war dem
Priester der Mund gestopft, und viele wurden von der Wahr-
heit überzeugt und waren recht empfänglich und zugänglich. Einige
boten mir Geld an, aber ich nahm es nicht.
Hierauf kam Doktor Cradock mit drei weiteren Priestern,
dem Gouverneur und seiner Frau, einer »Dame« (lachs) wie
man zu sagen pflegt, und einer andern ,,Dame« und eine ganze
Schar mit ihnen. Doktor Eradock fragte mich, warum ich im
Gefängnis sei; ich antwortete: ,,Weil ich den Geboten Christi und
der Apostel, nicht zu schwören, gehorche.« Wenn aber er, ein
Doktor und Friedenörichter, mir beweisen könne, daß Christuz


% \picinclude{./180-189/p_s184.jpg} 
184 Kapitel 17.
oder der Apostel den Christen, nachdem er ihnen verboten hatte,
zu schwören, es ihnen nachher wieder zu tun befahl, so wolle
auch ich es tun. Ich Tgab ihm die Bibel, damit er mir irgend
ein solches Gebot zeige, wenn er könne. Er sagte: »Jhr sollt
ohne Heuchelei und heiliglich schwören (Jer. 4, 2). ,,Ja, ja, sagte
ich, »so hieß es zu Jeremias Zeiten, aber das war lange bevor
Christus befahl: ihr sollt überhaupt nicht schwören (Matth. 5, 34).
Aus dem alten Testament könnte ich ebensoviele Beispiele oder vielleicht
noch mehr bringen, aber was nützen sie für den Beweis, daß das
Schwören auch im neuen Testament erlaubt war, nachdem Christus
und die Apostel es verboten? Übrigens: zu wem wird dort ge-
sagt, sie sollten nicht schwören ? zu den Heiden oder zu den Juden?«
Hierauf gab er keine Antwort. Aber einer der Priester sagte:
,,zu den Juden,« und Doktor Cradock gab es zu. ,,Gut,« sagte
ich, ,,aber wo hat Gott je den Heiden ein Gebot gegeben zu
schwören? und ihr wisset ja, daß wir von Natur Heiden sind.«
,,Allerdings,« sagte Doktor Eradock; ,,zwar zur Zeit des Evan-
geliums mußte alles aus zweier oder dreier Zeugen Mund bestätigt
werden, aber geschworen wurde nicht.« Warum also,« fragte ich,
,,zwingst du den Christen Gide ab gegen dein besseres Wissen?
und warum exkommunizierst du die Freunde?« ser hatte nämlich
viele sowohl in York als auch in Lancashire exkommuniziert). Gr
sagte: ,,weil sie nicht in die Kirche kamen.« »So!« sagte ich,
,,vor mehr als zwanzig Jahren, als wir noch Knaben und Mädchen
waren, da überließet ihr uns den Presbyterianern, den Jndepen-
denten und Baptisten, und viele von diesen nahmen uns Hab und
Gut und verfolgten uns, weil wir uns ihnen nicht anschließen
wollten; damals waren wir noch jung und wußten wenig
von euren Ansichten; hättet ihr nun die alten Leute, denen sie
bekannt waren, bei euch behalten und eure Ansichten in Kraft
erhalten wollen, so hättet ihr sollen entweder euch nicht von uns
wenden, wie ihr getan, oder ihr hättet uns sollen eure E-pisreln,
Kollekten, Homilien und Abendliiurgien senden, wie Paulus ja
auch den Heiligen geschrieben hatte, als er in der Gefangenschaft
von ihnen getrennt gewesen war. Wir hätten allesamt können
Türken oder Juden werden, was das, was wir in dieser Zeit von
euch empfmgen, anbelangt; und nun habt ihr uns, alt und jung,
exkommuniziert, also aus eurer Kirche ausgestoßen, ehe ihr uns für
dieselbe gewonnen habt. Jst es nicht ein Unsinn, uns auszu-


% \picinclude{./180-189/p_s185.jpg} 
Ein Gottetzgeeicht. Verhaftung wegen angeblicher Verschwörung usw. 185
weisen, ehe wir drin waren? Ja, wenn ihr unö für eure Kirche
gewonnen hättet und wir ihr angehört hätten und dann etwa
Unrechteß getan hätten, so wäre etz einigermaßen begründet ge-
wesen. Watz nennst du iibrigenß ,,Kirche?« ,,Nun,« sagte er,
,,daZ was du ,,TurmhauS« nennst.« Darauf fragte ich ihn, ob
denn Ehristuß sein Blut für daß Turmhaue-’ vergossen habe. ,,Und,«
sagte ich, ,,wenn nun die Kirche die Braut Christi und Christus-
datz Haupt der Kirche genannt wird, glaubst du denn, daß- Turm-
haus sei die Braut Christi und er das Haupt dieseö alten Gebäude?-?
ist er nicht vielmehr daß- Haupt der Gemeinde?« (Gph. 5). ,,Er
ist daö Haupt der Gemeinde,« erwiderte er, ,,und sie ist die
Kirche.« ,,Jhr habt also den Namen Kirche, welcher der Gemeinde
zukommt, einem alten Hause gegeben,« sagte ich, ,,und habt die
Leute gelehrt, solcheö zu glauben!« Weiter fragte ich ihn, warum
die Freunde verfolgt werden darum, daß sie den Zehnten nicht
geben? Ob Gott je den Heiden geboten habe, den Zehnten zu
bezahlen? Ob Christuö nicht die Zehnten aufgehoben habe, alß
er daß Levitische Priestettum, daß Zehnten nahm, aufhob? Und
ob Christus, alö er seine Jünger außsandte zu predigen, ihnen
nicht geboten habe, umsonst zu predigen? und ob nicht alle Diener
Christi verpflichtet seien, dieseß Gebot zu halten? Gr sagte, er
wolle hierüber nicht streiten; er schien überhaupt nicht gern bei
diesem Gegenstand zu verharren, sondern ging bald zu einem
andern über und sagte: ,,Jhr verheiratet euch, aber man weiß
nicht, wie ihr dabei verfahrt.« Jch riet ihm, zu kommen und
selbst zu sehen. Er drohte, unö seine Macht fühlen zu lassen; ich
riet ihm, zu bedenken, daß er ein alter Mann sei, und fragte ihn,
wo er oon der Genesiß biz zur Offenbarung irgendwo lese, daß
ein Priester jemand getraut habe; er solle mir ein solcheö Beispiel
zeigen, wenn er wolle, daß wir zu ihnen kommen sollten, um
um? trauen zu lassen. ,,Du hast ja,« sagte ich, ,,einen der Freunde
zwei Jahre nach seinem Tode noch erkommuniziert wegen seiner
Ehe; warum exkommunizierst du nicht auch Jsaak, Jakob, Boaö und
Ruth? Warum machst du deine Macht nicht Lauch an diesen
geltend? Denn ez steht nirgends, daß sie von einem Priester
getraut worden seien, sondern sie nahmen einander in der Ver-
sammlung in Gegenwart Gotteß und seiner Gemeinde; und so
tun wir. Wir haben also die heiligen Männer und Frauen der
Schrift auf unsrer Seite in dieser Sache.« Wir redeten lange


% \picinclude{./180-189/p_s186.jpg} 
186 Kapitel ZW.
hin und her; als er aber sah, daß er nichtß über mich vermochte,
ging er fort mit feinen Begleitern .....
Jn diesem und dem vorhergehenden Jahre waren viele
Freunde gefangen genommen worden. Viele waren in London, in
Newgate und andernGesängnissen, wo die Krankheit (Pest) herrschte,
und starben dort. Viele wurden auch verbannt und auf dez
Königz Befehl auf Schiffe gebracht. Oft wollten die Schiffßherren
sie nicht aufnehmen und setzten sie wieder ank; Land; doch gelangten
viele nach Barbadoeß, Jamaika und Neoiß, und der Herr segnete
sie dort .....
Nachdem ich mehr altz ein Jahr im Schloß zu Searbro ge-
fangen gewesen war, schickte ich einen Brief an den König, in
dem ich ihm von meiner Gefangenschaft berichtete und von der
schlechten Behandlung, die ich während derselben zu erdulden
hatte, und daß man mir gesagt habe, niemand alö er könne mich
frei machen. Und John Whitehead begab sich zu Gßquire Marsh,
mit dem er befreundet war, um ihm von mir zu reden, und dieser
versprach, daß, wenn John Whitehead einen Bericht über meine
Angelegenheit verfassen wolle, er denselben John Birkenhead, der
über die Begnadigung-zgesuche zu entscheiden hatte, einhändigen
und sich um meine Freisprechung bemühen wolle. John
Whitehead und Glliß Hookeß oerfaßten nun einen Bericht über
meine Gefangennahme und meine Leiden während der Gefangen-
schaft und brachten ihn Marsh, der ihn John Birkenhead über-
brachte und einen Befehl zu meiner Freisprechung erwirkte. Jn
demselben hieß e-3, daß der König von glaubwürdiger Seite er-
fahren habe, ich sei stets gegen alles Komplottieren und Streiten
gewesen, und habe etwaige Verschwörungen eher entdecken helfen,
alß daß ich mich selber daran beteiligt hätte, und so sei ez Sein
königlicheö Wohlgefallen, daß ich auö meiner Gefangenschaft befreit
werde. Sobald dieser Befehl bekannt war, kam John Whitehead
damit nach Scarbro und übergab ihn dem Gouverneur, der nun
die betreffenden Behörden zusammen berief und ohne weitere
Bürgschaft für mein friedsameß Leben, zufrieden mit der Erklärung,
daß ich ein stiller Bürger sei, mich frei ließ .....
Gleich am Tage nach meiner Freilassung brach das Feuer
in London auß, und daß Gerücht davon verbreitete sich rasch im
Lande. Da sah ich, daß Gott der Herr sein Wort wahr gemacht
hatte, daß int Gefängniö zu Lancaster zu mir geschehen war, alö


% \picinclude{./180-189/p_s187.jpg} 
Ein Gottezgericht. Verhaftung wegen angeblicher Verschwörung usw. 187
ich den Engel dez Herrn gesehen hatte, wie er mit einem leuch-
tenden Schwerte gen Süden zeigte, wie ich schon berichtet habe.
Die Bewohner waren vor diesem Feuer gewarnt worden; aber
wenige hatten esJ geglaubt oder zu Herzen genommen, vielmehr
wurden sie noch schlechter und hochmütiger. Ein Freund war nämlich
getrieben worden, von Huntingdonshire herunter zu kommen, kurz
vor der Feuerßbrunst, und sein Geld herum zu streuen, sein Pferd
frei in den Straßen herum zuführen, die Kniebänder aufzulösen,
die Strümpfe herunter hängen zu lassen, daß Wamß aufzuknöpfen
und den Leuten zu sagen: »so werdet ihr herum laufen und euer
Hab und Gut utnherstreuen, halb nackt, wie Wahnsinnige; und so
geschah es-, alö die Stadt brannte. So machte der Herr seine
Propheten und Diener zu Werkzeugen seiner Kraft und gab
ihnen Zeichen seineß Gerichtß und sandte sie, daß Volk zu warnen;
aber statt Buße zu tun, haben sie sie mißhandelt und etliche
gefangen genommen, unter der früheren Regierung sowohl als?
jetzt; aber der Herr ist gerecht, wohl dem, der seinen Worten
gehorcht! Etliche trieb e-J, nackt in den Straßen umher zu laufen,
um zu zeigen, wie Gott ihnen ihre heuchlerisehe Frömmigkeit
abreißen werde und sie nackt und bloß machen werde. Aber
das Volk hatte, statt in sich zu gehen, diese ost gegeißelt oder
sonst mißhandelt oder gar gefangen genommen. Andere trieb ez,
in Sticken umhetzugehen und die Rache und Strafe Gotteö wegen
des großen Hochmuteö zu verkünden; aber wenige gaben darauf
acht. In den Tagen der früheren Regierung machten die falschen,
frömmlerischen Priester mehrere Petitionen gegen unß an Oliver
und Richard, die sogenannten Protektoren, und an das Parlament
und die Richter und Räte, voller Lügen, Verleumdungen und
Schmühungen; aber wir oerschafsten unö Abschriften davon, und
mit Gotteß Hilfe antworteten wir auf alle, und wuschen die
Wahrheit und une-’ rein. Aber o, welcheMächte der Finsternis erhoben
sich in denen, die zum Lügen ihre Zuflucht nahmen! aber der
Herr stürzte sie alle und schützte seine Lämmer durch seine Kraft
und Wahrheit, sein Licht und sein Leben, und deckte sie, wie mit
Adlerß Flügeln. Solcheß gab uns Mut, aus ihn zu vertrauen,
der alle, die sich im Finstern gegen seine Wahrheit und sein Volk
verbünden, stürzt und vernichtet, und der durch diese Wahr-
heit seinem Volk Macht gibt, ihm in der Wahrheit zu dienen. ..


% \picinclude{./180-189/p_s188.jpg} 
188 Kapitel IR-’1.
Kapitel Zyl.
Einrichtung der Monatöversommlungm. Regelung
der Qnäletehen. ldriindnng von Knaben- u. Mädchenschulen.
Reformation des Qnäkertuus.
Nachdem ich nun wieder frei war, zog ich wieder umher,
nach Whitbt) . . . nach Oran .... und zuletzt nach Marmaduke
Storrz? ..,. wo ich eine große Versammlung hatte ..... Am
Tage nach derselben sollten zweie von der Freunden sich zur
Ehe nehmen, und es- war deshalb eine sehr zahlreiche Versamm-
lung, der ich beiwohnte. ES trieb mich, den Leuten unsern Stand-
punkt über die Eheschließung auseinanderzusetzen, indem ich
ihnen zeigte, wie man im Volke Gotteß einander zur Ehe ge-
nommen hatte in der Versammlung der Ältesten und wie es Gott
gewesen, der Mann und Weib zusammentigte vor dem Fall.
Nachher hätten dann zwar die Menschen sich selber zusammen-
getan; im Stand der Erlösung aber werde daß Zusammenfügen
durch Gott alö die richtige und ehrenhafte Verbindung angesehen,
und nie lesen wir von irgend einem Priester, von der Genesis
biz zur Offenbarung, daß er je zweie zusammengab. Hierauf
redete ich ihnen von den Pflichten der Eheleute, wie sie beide Gott
dienen sollten, als gleichermaßen Erben deö Lebens und der Gnade
(1. Petr. 3, 7) ..... Dann besuchte ich die Freunde im Lande
umher, biß ich nach York kam, wo ich eine große Versammlung
harte. Nach derselben besuchte ich Richter Robinson, einen früheren
Friedenörichter, der von Anfang an mir und den Freunden sehr
wohlgesinnt gewesen war. EZ war ein Priester bei ihm, welcher
mir sagte, es heiße von un?-, wir liebten niemanden alsz uns selber.
Ich erwiderte ihm, daß wir alle Menscken lieben, als Gotteß
Geschöpfe, die ja alle von Adam und Eve abstammen, und daß
wir die Brüder lieben, durch den Heiligen Geist. Dies brachte
ihn zum Schweigen, und wir gingen schließlich in Frieden aus--
einander; darnach reiste ich weiter.
Jch schrieb um diese Zeit ein Buch, betitelt: Fürchte Gott
und ehre den König. Ich zeigte darin, daß niemand wahrhaft
Gott fürchten und den König ehren könne, der nicht mit der Sünde
und dem Bösen breche. Dieses Buch machte großen Eindruck
aus die Soldaten und auf viele andere Leute .....
Nachdem ich viele Grafschaften durchzogen hatte, wo ich
Freunde besuchte und mit ihnen oiele große und gesegnete Ver-


% \picinclude{./180-189/p_s189.jpg} 
Einrichtung der Monatöversammlungen. Regelung der Quäkerehen usw. 189
sammlungen hatte, kam ich nach London. Aber ich fühlte mich
sehr schwach nach der beinahe dreijährigen harten und grausamen
Gesangenschaft; alle meine Gelenke und mein ganzer Körper waren
so steif und lahm, daß ich fast mein Pferd nicht besteigen noch
mich bewegen konnte; auch konnte ich schier die Nähe eineß Feuerß-
nicht ertragen oder den Genuß von warmem Fleisch, nachdem ich
so lange beides entbehrt hatte. In London besuchte ich öfters?3
die Brandstätten und sah mich aufmerksam darin um. Jch sah,
daß die Stadt so aus-sah, wie mir der Herr einige Jahre zuoor
geoffenbart hatte .....
Um diese Zeit erreichte die Kraft dez Herrn etliche, welche
die Wahrheit verlassen und die Freunde angegriffen hatten; sie
strömte so herrlich hernieder, daß sie ihre Schmähschriften ver-
dammten und zum Teil zerrissen. Wir hatten etliche Versamm-
lungen mit ihnen, und des Herm Krast war über allen und richtete
die Abtriinnigen. Jn diesen Versammlungen, welche ganze Tage
dauerten, kamen manche, welche mit John Perrot und andern
abgeirrt waren, zurück und Verurteilten den Geist, der sie verführt
hatte, den Hut auszubehalten während der Gebete der Freunde
und ihrer eigenen. Etliche von ihnen bekannten, die Freunde
seien besser als sie, und wenn die Freunde nicht gewesen wären,
so wären sie inß Verderben geraten. So ward deö Herm Kraft
herrlich offenbar und goß sich aus über alle.
g,;Daraus trieb mich der Herr, das Einrichten von fünf Monntö-
Versammlungen zu beantragen, für Männer und Frauen der
Stadt London außer den schon bestehenden Versammlungen fiir
Frauen und den Vierteljahre?-Versammlungen, damit die Herrlich-
keit Gotteß hoch gehalten werde und die, welche einen unordent=
lichen und leichtsinnigen Wandel führten und nicht nach der Wahr-
heit lebten, ermahnt und szurechtgewiesen würden. Denn weil
die Freunde nur oierteljährliche Versammlungen gehabt hatten, so
trieb es mich, nun, da die Wahrheit sich so ausgebreitet hatte,
und die Freunde zahlreicher geworden waren, das Eimsichtett von
monatlichen Versammlungen im ganzen Lande zu beantragen.
Und der Herr ofsenbarte mir, maß ich tun müsse und wie die
monatlichen und oierteljährlichen Versammlungen siir Männer
und Frauen in diesen und andern Ländern eingerichtet werden
müssen, und daß ich denen, zu welchen ich nicht gehen könne,
schreiben solle, daß sie etz auch so machen. Nachdem die Sache


% \picinclude{./190-199/p_s190.jpg} 
in London eingerichtet war, [...] ging ich nach Essex\ort{Essex}. Nachdem
die Monatsversammlungen hier eingerichtet waren, ging ich nach
Suffolk\ort{Suffolk} und Norfolk\ort{Norfolk} [...] Als 
auch hier die Monatsversammlungen eingerichtet waren, ging ich 
nach Huntingdonshire\ort{Huntingdonshire}, wo sie
ebenfalls eingerichtet wurden. [...] Ebenso in Bedfordshire,
und Nottinghamshire\ort{Nottinghamshire}, [...] 
Leieestershire\ort{Leieestershire}, [...] Warwickshire\ort{Warwickshire},
[...] In Staffordshire\ort{Staffordshire} hatten wir eine allgemeine 
Männerversammlung und richteten dort ebenfalls eine allgemeine 
Monatsversammlung ein [...] In Chefhire\ort{Chefhire} hatten wir ebenfalls eine
allgemeine Männerversammlung, in der die Monatsversammlung
für diese Grafschaft eingerichtet wurde [...] Auch in Laneashire
wurden die Monateversammlungen für diese Grafschaft eingerichtet, 
nach dem Evangelium [...] Von hier aus sandte ich
Schreiben nach Wesimorland, Durham, Cleveland, Northumberland, 
Eumberland und Schottland, um die Freunde zu ermahnen,
die Monatsversammlungen an diesen Orten einzurichten, maß sie
auch taten. So kam die Kraft des Herrn über alle, und ihre
Erben nahmen von ihr Besitz. Denn unsere Versammlungen
sind von der Kraft Gottes eingesetzt nach dem Evangelium, das
Leben und unvergängliches Wesen ans Licht bringt (2. Tim. 1,10)\bibel{Tim. 2. 01,10@2. Tim. 1,10},
damit alle, die der Teufel\index{Teufel} in Finsternis gebracht hat, wieder
sehend werden, und alle, die Erben des Evangeliums sind, auch in
diesem Evangelium wandeln, und Gott preisen mit Seele, Leib und
Geist, welche sind Gottes. Denn die Ordnungen des herrlichen
Evangeliums sind nicht von Menschen gemacht [...]
Durch Denbigshire und Montgomeryshire kamen wir nach
Merionetshire (Wales). Nachdem wir hier die Monatsversammlungen
eingerichtet, verließen wir Waleß und kehrten nach Shropshire
zurück [...] Dann gingen wir nach Woreestershire, wo wir
eine allgemeine Männerversammlung hatten, in Pafhur, wo ebenfalls
die Monatsversammlungen eingerichtet wurden [...]
In Herefordshire hatten wir mehrere gesegnete Zusammenkünfte. 
Auch hielten wir eine allgemeine Männerversammlung,
in der alle Monatsversammlungen festgesetzt wurden. Es war
gerade eine Verordnung erschienen gegen das Abhalten von Versammlungen. 
A1s wir nun nach Herefordshire kamen, berichtete
man uns von einer großen Versammlung der dortigen Prezbyterianer, 
welche entschlossen waren, alles eher zu ertragen und
auszugeben, als von ihren Versammlungen zu lassen. A1s nun
% \picinclude{./190-199/p_s191.jpg} 
diese Verordnung bekannt geworden sei, so seien die Leute gekommen, 
aber der Priester habe sich davon gemacht und habe sie
im Stich gelassen. Daraufhin kamen sie heimlich in Leominster
zusammen, hielten Brot, Käse und Getränke in Bereitschaft, damit,
wenn die Wachen kommen würden, sie ihre Bibeln bei Seite legen
könnten und sich ans Essen machen. Der Gerichtsdiener kam
ihnen aber auf die Spur, trat unter sie und sagte: "`euer Brot
und Wein hilft euch nicht; gebt eure Redner herans"' Sie 
antworteten: "`was würde dann aus ihren Frauen und Kindern
werden?"' Aber er nahm ihre Redner gefangen und behielt sie
eine Weile. Er erzählte es Peter Young und sagte, dies seien
die ärgsten Heuchler, die je für eine Religion Bekenntnis 
abzulegen suchten.

Ähnliches bewerkstelligten sie an andern Orten. In London\ort{London}
war einer namens Pocock,\person{Pocock} welcher 
Abigail Daray\person{Daray, Abigail} heiratete, eine
sogenannte Dame, und da sie eine Bekännerin der Wahrheit war,
so ging ich in sein Haus, um sie zu besuchen. Dieser Pocock
war ein Erz-Presbyterianer\index{Presbyterianer} und sehr übel gesinnt gegen uns
und pflegte unsre Leute "`Hauskriecher"' (housecreeper\index{housecreeper}) zu nennen.
Als er nun einmal fort war, sagte seine Frau zu mir: "`Ich muss
dir etwas über meinen Mann sagen."' "`Nein,"' sagte ich, "`du
sollst nicht über deinen Mann reden."' "`Doch,"' erwiderte sie,
"`in diesem Falle muss ich es. Am vorigen Ersten Tag hatte
er mit seinen Priestern und Genossen eine Versammlung; sie hatten
Lichter, Tabakspfeifen, Brot und Käse und kaltes Fleisch vor sich
auf dem Tisch, und sie hatten sich verabredet, falls die Beamten
sie überraschen sollten, aufzuhören mit Predigen und Beten und
sich ans- Essen zu machen."' Als ich ihn wieder sah, sagte ich:
"`Ihr, die ihr uns verfolgt und gefangen genommen habt und
unsrer Habe beraubt, weil wir uns eurer Religion nicht anschließen 
wollten, und uns \textit{Kriecher} nanntet, ihr schämt euch nicht,
das ihr nun nicht einmal zu eurer Religion sieht? Habt ihr je
gesehen, das wir uns bei unsern Versammlungen mit Brot und
Käse versahen? oder habt ihr irgendwo in der Schrift gelesen,
das die Heiligen dergleichen taten?"' ,"`Ei,"' sagte der Alte, ,"`wir
sollen ja klug sein wie die Schlangen"' Ich erwiderte: "`Dies
ist allerdings Schlangenklugheit! Wer hätte aber gedacht, das
ihr Presbyterianer und Independenten,\index{Independenten} nachdem ihr solche, die
sich eurem Glauben nicht anschließen wollten, Verfolgtet, gefangennahmt,
% \picinclude{./190-199/p_s192.jpg} 
peitschtet und beraubtet, nun selber zurückweicht und nicht
wagt, zu eurem Glauben zu stehen, sondern denselben mit Hilfe
von Tabakpfeifen, Flaschen, Brot und Käse zu verbergen sucht?"'
Aber ich vernahm später, das solche Heucheleien nur allzuhäufig
betrieben wurden in den Zeiten der Verfolgung.
Als wir in Heresordshire alles die Versammlung Betreffende
geordnet hatten, gingen wir nach Monmouthshire, wo wir mehrere
gesegnete Versammlungen hatten, und bei Walter Jenkins,\person{Jenkins, Walter} einem
früheren Friedensrichter, hatten wir eine große Zusammenkunft
und es wurden mehrere gewonnen. Es war eine ruhige Versammlung; 
in einer früheren hingegen war ein halb betrunkener Gerichtsdiener\index{Andachtsstörung}
erschienen und hatte behauptet, er müsse die Redner abfassen; 
aber die Kraft Gottes war so mächtig gewesen in jener
Versammlung, das sie ihn trotz seines Wütens bannte und er sich
ihr nicht entziehen konnte. Als die Versammlung aus war, war
ich noch ein wenig geblieben und er ebenfalls, ich redete ein
wenig mit ihm und ging dann ruhig weg. In der Nacht kamen
ein paar und schossen mit einer Flinte gegen das Haus, verletzten
aber niemand. So kam die Kraft des Herrn über alle und
band die widerspenstigen Geister, so das wir keinen Schaden
nahmen.

Nun gingen wir nach Gloucestershire\ort{Gloucestershire}, und hatten dort viele
gesegnete Versammlungen in der ganzen Grafschaft herum, und
zuletzt gingen wir weiter nach Bristol,\ort{Bristol} wo nach einer sehr 
ersprießlichen Zeit die Männer und Frauenversammlungen ebenfalls eingerichtet wurden.
Einmal als ich in Bristol in meinem Bett war, geschah das
Wort des Herrn zu mir, ich solle wieder nach London zurückgehen.
Am folgenden Morgen kam Alexander Parker\person{Parker, Alexander} und einige andere
zu mir. Ich fragte sie, was ihnen sei? und ebenso fragten sie
mich, was mir sei? Ich sagte ihnen, ich fühle, das ich nach
London zurückkehren müsse. Sie sagten, gerade so sei es auch
ihnen. So ergaben wir uns drein, nach London zu gehen; denn
welchen Weg auch der Herr uns führte, wir gingen ihn in seiner
Kraft. Wir gingen über Wiltshire\ort{Wiltshire} und ordneten dort die 
Monatsversammlung für Männer, in der Kraft des Herrn, und besuchten
die Freunde, bis wir nach London kamen.

Nachdem wir die Freunde in der Stadt besucht hatten, trieb
es mich, sie zu ermahnen, alle ihre Eheschließungen\index{Heirat} vor die 
% \picinclude{./190-199/p_s193.jpg} 
Versammlungen der Männer und Frauen zu bringen, um sie den
Gläubigen vorzulegen. Diese Vorsorge möge getroffen werden,
um Unordnungen zu verhüten, wie solche von etlichen begangen
worden waren. Denn viele hatten sich gegen den Willen der
Ihrigen Verheiratet, und einige junge Leute, die sich zu uns hielten,
hatten sich mit solchen, die der Welt angehörten,\index{Die der Welt angehörten} verbunden;
Witwen hatten sich wieder verheiratet, ohne Fürsorge zu treffen
für ihre Kinder, trotz meiner Schrift über das Heiraten, die ich
im Jahre 1653\index{Jahr!1653} veröffentlicht hatte, als die Wahrheit noch wenig
verbreitet war. Ich hatte darin die Freunde, für die es in
Betracht kam, ermahnt, die Sache doch ja immer den Gläubigen
vorzulegen, ehe sie etwas abmachten, und erst danach bekannt
zu machen, auf dem Markte oder in der Versammlung, je nachdem 
es sie triebe. Und wenn dann alles ins Reine gebracht
worden sei, wenn sie frei seien von jeder anderweitigen 
Verpflichtung, und ihre Angehörigen einverstanden, so sollten sie eine 
Versammlung bestimmen, in der sie sich dann, in Gegenwart von
mindestens zwölf Zeugen, zur Ehe nehmen. Da nun diese Vorschriften
nicht befolgt wurden, und die Wahrheit sich weiter im Lande
ausgebreitet hatte, so wurde in der Kraft und dem Geist des
Herrn verordnet, das die Eheschließungen den vierteljährlichen
und den monatlichen Versammlungen der Männer vorgelegt werden
sollten. Die Freunde sollten dafür sorgen, das die Angehörigen
beider Teile einverstanden seien, und das die Witwen Bestimmungen
getroffen haben für die Kinder aus erster Ehe, ehe sie wieder
heiraten, und was es sonst noch zu ordnen gibt, damit alles 
geschehe in Reinheit und Gerechtigkeit, zur Ehre Gottes. Später
wurde verordnet durch die Weißheit Gottes, das, wenn der eine
Teil aus einer andern Gegend oder aus einem andern Land komme
oder einer anderen Monatsversammlung zugehörte, so solle er eine
Bescheinigung bringen von der Versammlung, der er zugehörte,
als Gewähr bei der Monatsversammlung, der sie ihre Absicht,
sich zu heiraten, vorlegen.

Nachdem diese Angelegenheit, sowie viele andere Dienste für
Gott, in Ordnung gebracht und geregelt waren in den 
verschiedenen Stadtgemeinden, verließ ich London und ging, wie mich
die Kraft des Herrn führte, nach Hertfordshire\ort{Hertfordshire}. Nachdem ich
viele Freunde dort besucht hatte, und die Monatsversammlungen
für Männer dort geordnet waren, hatte ich eine große 
% \picinclude{./190-199/p_s194.jpg} 
Versammlung in Baldock\ort{Baldock} mit allen möglichen Leuten. Darauf kehrte ich
nach London zurück über Waltham,\ort{Waltham} wo ich ihnen riet, eine Schule\index{Schule}
für Knaben einzurichten, sowie auch in Shacklewell\ort{Shacklewell} eine Schule\index{Quakerschule}
für Mädchen, um sie in allem Guten und Nützlichen zu unterrichten [...]

Wir zogen durch Gloucestershire\ort{Gloucestershire} und besuchten die Freunde,
dann kamen wir nach Monmouthshire,\ort{Monmouthshire} wo wir mit Vertretern
aller Versammlungen des ganzen Bezirks zusammentrafen und in
der Kraft des Herrn auch hier die Monatsversammlungen 
einrichteten, damit alle die Herrlichkeit Gottes feiern möchten und
die, welche nicht nach dem Evangelium wandeln, ermahnt und
zurechtgewiesen würden. Und wirklich bewirkten diese 
Versammlungen eine große Besserung unter den Leuten, so das die 
Obrigkeit ihren Nutzen einsah.\index{Monatsversammlungen, Sinn und Nutzen der }

Wir kamen an einen Ort in der Nähe von Minehead\ort{Minehead}, wo
wir eine große allgemeine Männerversammlung für alle Freunde
von Somersetshire\ort{Somersetshire} hatten. Es war auch einer dabei, ein
Schwindler, von dem einige gutmütige Leute gemeint hatten, ich
solle ihn bleibend zu mir nehmen; aber ich sah, das er ein
Schwindler war, und hieß sie darum, ihn zu mir bringen, damit
ich sehe, ob er mir ins Gesicht sehen könne. Er konnte es
nicht, sondern blickte unruhig hin und her. Er hatte einen
Priester betrogen, indem er ihn glauben machte, er sei ein 
Prediger, hatte sich sein Priesterkleid verschafft und sich in demselben
davon gemacht.

Nach der Versammlung gingen wir weiter nach Minehead,\ort{Minehead}
wo wir rasteten. In der Nacht musste ich ringen mit einem
Geist der Finsternis, der sich gegen die Kirche Christi erheben
wollte, um sie in Verwirrung zu bringen.\index{Anfechtung/Versuchung} Am folgenden Morgen
trieb es mich, einige Zeilen an die Freunde zu schreiben, um sie
zu warnen.

\grosszitat{
  Liebe Freunde,
  \medskip 
  Lebet in der Kraft Gottes des Herrn, und in seinem Samen,
  der größer ist als alle Versuchungen, die der Geist der Finsternis
  euch anhaben kann, welcher euch ihm Untertan machen und sich
  unter euch erheben möchte; er ist noch nicht gekommen, aber in
  der Kraft Gottes und seines Samens haltet euch über demselben
  und verdammet ihn. Denn ich fühlte einen Geist der Finsternis in
  der vergangenen Nacht, der suchte sich zu erheben und unter euch
  % \picinclude{./190-199/p_s195.jpg} 
  aufzustehen; aber ihr könnet ihn bezwingen mit Gottes Kraft und
  sein Treiben verdammen, ehe er irgendwo Eingang gefunden
  hat. Mehr will ich nicht sagen; meine Liebe im Samen Gottes,
  in welchem kein Wechsel ist."'

  \begin{flushright}
  Minehead in Somersetshire\ort{Somersetshire}, 22. des 4. Monats, 1668.\index{Jahr!1668}

  G. F.
  \end{flushright}
}

Nachdem wir die meisten Versammlungen in Somersetshire
besucht hatten, gingen wir weiter nach Dorsetshire\ort{Dorsetshire} 
zu Georg Harris,\person{Harris, Georg}
in dessen Haus wir eine große Versammlung für Männer hatten.
Hier wurden nun alle Monatsversammlungen für Männer für
den ganzen Bezirk geordnet nach den herrlichen Geboten des
Evangeliums, auf das alle möchten in der Kraft Gottes "`das
Verlorene suchen und das Verirrte wieder holen"' (Hes. 34,4\bibel{Hes. 34:04@Hes. 34:4}),
das Gute ehren und das Böse strafen.

Hierauf kamen wir nach Southampton,\ort{Southampton} wo wir am Ersten
Tage eine große Versammlung hatten. Von da gingen wir zu
Hauptmann Reeves,\person{Hauptmann Reeves} 
wo die allgemeine Versammlung für Männer
für Hampshire stattfand; Es waren viele aus der ganzen Grafschaft 
gekommen, und wir hatten eine gesegnete Zeit. Die Monatsversammlungen 
der Männer für diese Grafschaft wurden geordnet
nach den Vorschriften des Evangeliums, welches Leben und
unsterbliches Wesen in ihnen ans Licht gebracht hatte. Da erschien
eine Bande Ranter,\index{Ranter} die unsre Versammlung recht störten und
sich derselben widersetzten.\index{Versammlung!Störung}

Es war eine Frau dabei, die bei einem Mann gelegen hatte;
dieser erzählte es nun auf dem Marktplatz und rühmte sich
seiner Schlechtigkeit; eine Anzahl dieser liederlichen Leute wohnte
zusammen in einem Haus, ganz nahe bei dem Ort, wo wir unsere
Versammlungen hatten. Ich ging zu ihnen und hielt ihnen ihre
Schlechtigkeit vor. Der Herr des Hauses sagte: nun! warum mich
denn das so sehr erstaune? Ein andrer sagte, es werde mich
wohl auch straucheln machen! Ich erwiderte ihnen, ihre
Schlechtigkeit werde mich nicht zum Straucheln bringen, denn ich stehe
über derselben. Und der Herr trieb mich, ihnen zu sagen, das
die Strafen und das Gericht Gottes über sie kommen werden.
Sie zogen später im Land herum, bis sie schließlich ins Gefängnis
in Winchester\ort{Winchester} geworfen wurden, wo der Mann, der bei der Frau
gelegen hatte, nach dem Kerkermeister stach, ihn jedoch nicht tötete.
Als sie dann aus dem Gefängnis entlassen waren, erhängte sich
% \picinclude{./190-199/p_s196.jpg} 
der Mann, der den Kerkermeister erstechen wollte; die Frau hätte
auch fast einem Kinde den Hals abgeschnitten, wie wir hörten.
Diese Leute hatten früher in der Nähe von London gelebt, und als
die Stadt brannte, prophezeiten sie, das dass ganze übrige London
innerhalb vierzehn Tagen verbrennen würde und flohen aus der
Stadt. Diese Ranter nun, große Gegner der Freunde, und Störer
unsrer Versammlungen, wurden zuweilen in der Gegend, wo die
Leute sie nicht kannten, für Quäker gehalten. Darum trieb mich
der Herr, ein Schreiben zu verfassen, das unter den Behörden
und dem Volk in Hampshire\ort{Hampshire} verbreitet werden sollte, damit man
sehe, das die Wahrheit und die Freunde mit diesen liederlichen
Leuten nichts zu tun haben [...]

So waren nun im ganzen Lande die monatlichen Männerversammlungen 
geordnet. Denn in Berkshire war ich früher
gewesen, damals, als die meisten der ersten Freunde im
Gefängnis waren; ich hatte ihnen den Nutzen dieser 
Monatsversammlungen auseinandergesetzt, und sie hatten sie daraufhin auch
eingerichtet. Auch nach Irland\ort{Irland} und 
Schottland,\ort{Schottland} nach Holland,\ort{Holland}
Barbadoes\ort{Barbadoes} und mehrere Orte in Amerika, sandte ich 
durch zuverlässige Freunde Schreiben, um die Freunde zu ermahnen, ihre
monatlichen Männerversammlungen überall zu ordnen. Vierteljährliche 
Versammlungen hatten sie schon vorher gehabt; aber
jetzt, da die Wahrheit sich unter ihnen verbreitet hatte, sollten sie
auch Monatsversammlungen einrichten in der Kraft Gottes, durch
die sie bekehrt worden waren. Seit diese Versammlungen eingerichtet 
worden sind, und die Getreuen des Herrn, die Erben des
Evangeliums sich versammeln, in der Kraft ihres Meisters, aus
die sich diese Versammlungen gründen, haben viele ihren Mund
aufgetan in Dank und Lobpreisung, und viele haben dem Herrn
mit Tränen gedankt, das er mich in seinem Dienst ausgesandt
hatte. Alle, denen Gottes Ehre und Herrlichkeit am Herzen liegt,
alle, denen es ein Anliegen ist, das sein Name, den sie bekennen,
nicht gelästert werde und das, wer die Wahrheit bekennt, auch
in der Wahrheit, Gerechtigkeit und Heiligkeit wandelt, können nun
das Reich Christi,\index{Reich Christi} dessen Wachstum kein Ende hat, 
kennen und sehen, besitzen und daran teil haben. Der ewige Ruhm und Preis
Gottes ist in jedem Herzen, das treu ist, eingepflanzt; wir
dürfen sagen, das die Ordnung des Evangeliums unter uns
nicht von Menschen, noch durch Menschen, sondern von und
% \picinclude{./190-199/p_s197.jpg} 
durch Jesus Christus; und durch den heiligen Geist ausgerichtet
wurde [...]

Nach London\ort{London} zurückgekehrt, blieb ich einige Zeit dort, um die
Freunde in der Stadt und der Umgegend zu besuchen. Einmal
ging ich zu Esquire Marsh,\person{Marsh, Esquire} der mir und den Freunden viel
Freundlichkeit erwiesen hatte; es traf sich, das er gerade am
Mittagessen war, als ich kam. Kaum hatte er meinen Namen
gehört, so ließ er mich herauf holen und wollte, das ich mich mit
ihm zu Tisch setze; aber ich hatte nicht die Freiheit, es zu tun.
Es waren mehrere hochgestellte Personen mit ihm bei Tisch, und
er sagte zu einem von ihnen, einem angesehenen Papisten:\index{Papisten} "`Hier
ist ein Quäker, den ihr noch nie gesehen habt."' Der Papist fragte
mich, ob ich die Kindertaufe anerkenne?\index{Kindertaufe} Ich erwiderte ihm, es
stehe nichts in der Bibel davon. "`Wie,"' sagte er, "`nichts über
die Kindertaufe?"' Ich sagte: "`nein."' Ich sagte ihm: "`Wir
anerkennen die Eine Taufe durch den Einen Geist in dem Einen
Leib (Kor. 12);\bibel{Kor. 12} jedoch dafür, das man ein wenig Wasser einem
Kinde übers Gesicht schüttet und sagt, das sei nun das Kind taufen
und zu einem Christen machen, gibt es kein Bibelwort."', Ferner
fragte er mich, ob ich den katholischen Glauben anerkenne? Ich
antwortete "`ja,"' fügte aber hinzu, "`weder der Papst noch die
Papisten haben den katholischen Glauben; denn der wahre Glaube
wirket in der Liebe und reinigt das Herz; wenn ihr den Glauben
hättet, welcher den Sieg gibt, und durch den man den Zugang
zu Gott hat, so würdet ihr den Leuten nicht von einem Fegefeuer\index{Fegefeuer}
nach dem Tode reden."'\index{Katholischer Glaube} Ich suchte 
nun zu beweisen, das kein Papst und kein Papist, welcher ein 
Fegefeuer nach dem Tod
annehme, den wahren Glauben habe; denn der wahre, herrliche,
göttliche Glaube, dessen Anfänger Christus ist, gibt den Sieg über
Teufel und Sünde, die den Menschen von Gott getrennt haben.
Wenn sie, die Papisten, den wahren Glauben hätten, so würden
sie nicht solche, die einen andern Glauben haben, verfolgen und\index{Verfolgung}
mit Foltern, Gefängnissen und Geldbußen ihnen ihren Glauben
aufzwingen. Das sei nicht die Art der Apostel\person{Apostel} und ersten Christen
gewesen, die den wahren Glauben Christi besaßen und bezeugten;
sondern die irgläubigen Juden\indexname{Juden} und Heiden machten es so. "`Wenn
Du,"' sagte ich, "`ein Haupt und Führer der Papisten, 
aufgewachsen und erzogen in der Lehre des Papstes, sagst, es gebe kein
Heil außer in eurer Kirche, so möchte ich gerne wissen, was denn
% \picinclude{./190-199/p_s198.jpg} 
in eurer Kirche das Heil bringt?"' Er antwortete: "`Ein gutes
Leben."' "`Sonst nichts?"' sagte ich. "`Doch,"' sagte er, "`gute
Werke."' "`So, daß bringt eurer Kirche Heil,"' sagte ich, "`ein
gutes Leben und gute Werke! das ist also eure Lehre und euer
Grundsatz! Dann wissen weder du, noch der Papst, noch irgend
ein Papist, woher das Heil kommt."' Darauf fragte er mich,
woher denn das Heil in unsrer Kirche komme? Ich sagte ihm:
"`nichts anderes, als was in den Tagen der Apostel das Heil der
Kirche war, ist es auch für unsre Kirche, nämlich, "`die heilsame
Gnade Gottes, die allen Menschen erschienen ist"' (Tit. 2,11).\bibel{Tit. 02:11@Tit. 2:11}
Wie sie einst die Heiligen lehrte, so lehrt sie jetzt uns: "`zu 
verleugnen das ungöttliche Wesen und die weltlichen Lüste, und
gottselig, gerecht und züchtig zu leben in der Welt"' (Tit.2,12).\bibel{Tit. 02:12@Tit.2:12} 
Es sind also weder die guten Werke noch ein gutes Leben, die daß
Heil bringen, sondern die Gnade."'\index{Rechtfertigung} "`Und diese heilsame Gnade
erscheint allen Menschen, sagt ihr?"' rief der Priester. "`Ja,"'
erwiderte ich. "`Das gebe ich euch nicht zu!"' rief er. Ich 
antwortete: "`Alle, die es nicht zugeben, sind Sektierer und haben
nicht den allumfassenden Glauben der Apostel."'\index{Sektierer}

Darauf redete er über die Mutter-Kirche. Ich sagte ihm,
alle die verschiedenen Sekten im Christentum hätten uns 
vorgeworfen, wir verließen die Mutterkirche. Die Papisten warfen
uns den Abfall von der Mutterkirche vor, mit der Behauptung,
Rom sei diese einzige Mutterkirche. Die Bischöflichen beschuldigten
uns des Abfalls vom alten protestantischen Glauben, indem sie
geltend machten, sie hätten die reformierte Mutterkirche. Die
Presbnterianer und Independenten schalten uns, das wir sie 
verlassen, indem beide behaupteten, sie hätten die wahre reformierte
Mutterkirche. "`Allein,"' sagte ich, "`wenn wir irgend einen äußeren
Ort als Mutterkirche anerkennen würden, so wäre es Jerusalem,\ort{Jerusalem}
wo das Evangelium zuerst verkündet wurde durch Christus selbst
und seine Apostel, wo Christus litt, wo die große Vekehrung
zum Christentum durch Petrus stattfand, der Ort der prophetischen
Zeichen und Wunder, die in Christus ihre Erfüllung hatten, und
wo er seinen Jüngern befahl, "`zu warten, bis das sie angetan
würden mit der Kraft aus der Höhe"' (Luk. 24,49)\bibel{Luk. 24:49}. 
Wenn irgend ein äußerer Ort verdiene, eine Mutterkirche genannt zu werden, 
so sei es derjenige, an dem die erste große Bekehrumg zum Christentum 
stattfand. Aber der Apostel sagt, 
Gal. 4,25—27\bibel{Gal. 04:25—27@Gal. 4:25—27}: "`Das
% \picinclude{./190-199/p_s199.jpg} 
Jerusalem, das zu dieser Zeit ist, ist dienstbar mit seinen Kindern.
Aber das Jerusalem, das droben ist, ist die Freie, die ist unser
aller Mutter. Sei fröhlich, du Unfruchtbare, und juble die du
nicht schwanger bist; denn die Einsame hat mehr Kinder als
die den Mann hat."' Der Apostel sagt nicht, das sichtbare
Jerusalem sei die Mutter, obgleich die erste und große Bekehrung
zum Christentum dort stattfand. Und noch weniger berechtigt ist, das
Rom\ort{Rom} oder sonst ein Ort oder eine Stadt so bezeichnet werde von
den Kindern des freien, oberen Jerusalem; auch sind solche nicht
Kinder des freien, oberen Jerusalem, welche das sichtbare Jerusalem 
oder Rom oder irgend einen andern Ort oder eine Sekte
ihre Mutter nennen. Und obgleich von entarteten Christen vielen
Orten und Sekten dieser Titel gegeben wurde, so sagen wir
dennoch wie einst der Apostel: "`Das Jerusalem, das droben ist,
das ist die Freie, die ist unser aller Mutter."' Und wir können
kein anderes Jerusalem, noch ein Rom, noch irgend eine Sekte
als unsre Mutter anerkennen, sondern allein das Jerusalem, das
droben ist, die Freie, die Mutter aller derer, die wiedergeboren\index{wiedergeboren}
sind und wahrhaft an das Licht glauben und eingepflanzt sind
in Christus, den himmlischen Weinstock. Denn alle, welche 
wiedergeboren sind aus dem unvergänglichen Samen durch das Wort
Gottes, welches ewiglich bleibet, nähren sich von der Milch des
Wortes, an den Brüsten des Lebens und wachsen und nehmen
zu durch dieselbe und können keine andere Mutter anerkennen,
als das Jerusalem, welches droben ist. "`O,"' sagte Esquire
Marsh zu dem Papisten, "`Ihr wisset nicht, was für ein Mann
der ist; wenn er nur hier und da in die Kirche kommen wollte,
so wäre er ein ausgezeichneter Mensch."'

Ich nahm Marsh beiseite, um wegen der Freunde mit ihm
zu reden; er war Friedensrichter von Middlesex,\ort{Middlesex} und da er an den
Hof kam, übertragen ihm die andern Richter die Leitung mancher
Geschäftes. Er sagte mir, das er in Verlegenheit sei, wie er zu
unterscheiden habe zwischen uns und einigen Dissentern.\index{Dissentern} "`Denn,"'
sagte er, "`ihr könnt nicht schwören und die Independenten, die
Baptisten und die Fifth-Monarchy-Leute sagen ebenfalls, sie können
nicht schwören; wie soll ich denn nun zwischen euch und ihnen 
unterscheiden, da ihr allesamt sagt, ihr könnt um des Gewissens willen
nicht schwören?"'\index{Schwören}\index{Quaker von Anderen unterscheiden.} 
Ich antwortete ihm: "`Ich will dir zeigen, wie
du uns unterscheiden kannst. Jene, wenigstens die meisten, der
\picinclude{./200-209/p_s200.jpg} 
\picinclude{./200-209/p_s201.jpg} 
\picinclude{./200-209/p_s202.jpg} 
\picinclude{./200-209/p_s203.jpg} 
\picinclude{./200-209/p_s204.jpg} 
\picinclude{./200-209/p_s205.jpg} 
\picinclude{./200-209/p_s206.jpg} 
\picinclude{./200-209/p_s207.jpg} 
\picinclude{./200-209/p_s208.jpg} 
\picinclude{./200-209/p_s209.jpg} 

% \picinclude{./210-219/p_s210.jpg} 
unermüdliches Anhalten; der König gab Sir John Otway den
Befehl, dem Sheriff in einem Brief seinen diesbezüglichen Willen
kund zu tun, sowie in betrefs anderer aus der Gegend. Diesen
Brief nahm Sarah Fell mit, als sie mit ihrem Bruder und ihrer
Schwester Rouß nach Lancaster ging; und durch sie schrieb ich
folgendes an meine Frau:

\grosszitat{
  Mein liebes Herz in der Wahrheit und dem Leben, welches
  sich nimmermehr verändert.

  Es kam über mich, Mary Lower und Sarah sollten zum
  König gehen und zu Kirby, daß die Kraft des Herrn sich an
  ihnen allen kund tun möge zu deiner Befreiung. Sie gingen,
  wollten aber dann wieder zurückkommen; aber es kam über mich,
  sie noch ein wenig länger zu halten, damit sie die Lossprechung
  zu Ende bringen; dies ist nun geschehen, und wie du siehst, sende
  ich sie dir hier. Meine letzte Erklärung ist sehr förderlich gewesen,
  man war im ganzen damit zufrieden. Soviel für heute, meine
  Liebe im heiligen Samen.

  \begin{flushright}
  G. F.\end{flushright}
}

Die erwähnte Erklärung war ein gedrucktes Blatt, das ich
bei Anlass einer neuen Verfolgung geschrieben. Zu der Zeit
nämlich, da ich von Leicester nach London zurückkehrte, hatte
sich ein neuer Sturm erhoben infolge einer sehr stürmischen
Versammlung im Turmhauß in Gloucestershire;\index{Gloucestershire} es hieß, einige
Parlamentsmitglieder hätten sie dazu benutzt, um ein Gesetz gegen
verführerische Konventikel\index{Konventikel} durchzusetzen. Das selbe wurde bald
darauf Veröffentlicht und wurde auf uns angewandt, die wir doch
Vor allen andern frei waren von Verführung und Tumult. Darauf
schrieb ich eine Erklärung und zeigte an Hand der Ausdrücke
dieses Gesetzes das wir nicht derartige Leute seien noch
unsere Versammlungen derart, wie sie in dem Gesetz beschrieben
seien. [...]

Wir machten uns auf den Weg nach Rochester. Unterwegs,
als ich einen Hügel hinunterstieg, wurde meine Seele von einer
schweren Last bedrückt; ich bestieg mein Pferd, aber der Druck
blieb dermaßen, das ich kaum fähig war, weiter zu reiten. Endlich
kamen wir nach Rochester; aber ich war sehr erschöpft, weil die
Geister der Welt mich so schwer bedrückten und so schwer auf
mir lasteten. Mit Mühe erreichte ich Gravesend und lag dort
in einer Herberge, aber ich konnte kaum essen noch schlafen. Am
% \picinclude{./210-219/p_s211.jpg} 
folgenden Tage machten sich John Routz und Alexander Parker,
auf nach London; ich ging mit John Stubbs, welcher zu mir
gekommen war, mit der Fähre nach Essex. Wir kamen nach
Hornchurch, wo am Ersten Tage eine Versammlung war. Darauf
ritt ich unter großen Beschwerden nach Stratford zu einem
Freunde namens; Williamß, der früher Hauptmann gewesen war.
Hier lag ich in großer Schwachheit und verlor schließlich Gehör
und Gesicht.\index{Erkrankung} Mehrere Freunde kamen von London, um mich zu
besuchen, und ich sagte ihnen, ich müsse ein Zeichen sein für die,
welche die Wahrheit nicht sehen und hören wollten. Ich blieb
einige Zeit in diesem Zustand. Es kamen etliche zu mir, und
obgleich ich sie nicht sehen konnte, so durchschaute ich doch ihr
Inneres, welche aufrichtig waren und welche nicht. Verschiedene
Freunde, die Arzneikunde trieben, kamen zu mir und wollten mir
Medizin\index{Medizin} geben, aber ich durfte mich mit keinem einlassen, denn
ich spürte, das ich durch eine Heimsuchung hindurch müsse, und
darum wollte ich nur zuverlässige ernste Freunde um mich haben.
Unter großen Leiden und Beschwerden, in großer Gedrücktheit und
Niedergeschlagenheit, lag ich mehrere Wochen krank, und ich kam
so herunter und wurde so schwach, das die wenigsten glaubten,
ich würde am Leben bleiben. Einige, die bei mir waren, gingen
weg und sagten, sie wollten mich nicht sterben sehen; es hieß in
London und in der Umgegend, ich sei gestorben, aber ich fühlte,
das mich innerlich die Kraft des Herrn aufrecht erhielt. Als die,
welche um mich waren, mich aufgegeben hatten, hieß ich sie, mir
einen Wagen holen, um mich zu Gerrard Robert\index{Personen!Robert, Gerrard} zu bringen,
etwa zwölf Meilen weit weg, denn ich erkannte, das es das
richtige sei, dorthin zu gehen. Ich hatte wieder einen Schimmer,
so das ich die Leute und die Gegend erkennen konnte, aber das
war alles [...]

Ich litt zu dieser Zeit mehr, als sich mit Worten sagen lässt,\index{Vision}
denn ich musste in die Tiefe, und ich sah alle Religionen der Welt
und die Menschen, die darin leben, und die Priester, die sie vertraten, 
und die wie eine Bande von Menschenfressern waren; sie
fraßen die Menschen auf wie Brot und nagten das Fleisch von
ihren Knochen. Wahrer Glaube aber und Anbetung, wahre
Diener Gottes, ach! da waren keiner unter denen, die sich dafür
ausgaben! Denn die, welche behaupteten, eine Kirche zu sein,
waren nur eine Gesellschaft von Menschenfressern, Menschen mit
% \picinclude{./210-219/p_s212.jpg} 
harten Gesichtern und langen Zähnen; und wenn sie gleich über
die Menschenfresser in Amerika geschrieen, ich sah, das sie ganz
gleich waren. Den großen Frommen unter den Juden\index{Juden} sind sie
gleich, die \zitat{Gottes-Volk fressen wie Brot} (Micha 3,3),\index{Bibel!Micha 03:03@Micha 3:3} den
falschen Propheten und Priestern, die dem Volk Frieden predigten,
so lange als es ihnen zu fressen gab; wo man ihnen aber
nichts in das Maul gibt, da predigen sie, es müsse ein Krieg
kommen, \zitat{sie fressen das Fleisch meines Volkes und zerlegen es
wie Fleisch in einem Kessel} (Micha 3,5).\index{Bibel!Micha 03:05@Micha 3:5}

So sind sie, die sich jetzt als Christen ausgeben, sowohl
Priester als Fromme; sie sind nicht in der Kraft und dem Geist,
in welchem Christus, die heiligen Propheten\index{Prophet} und die Apostel\index{Apostel}
waren; sie sind von der gleichen Art wie die alten jüdischen
Frommen und sind Menschenfresser so gut wie jene. Sie haben
die Verfolgungen angezettelt und haben die bösen Angeber 
aufgestiftet, so das ein Freund kaum ruhig im engsten Familienkreise
sich aussprechen kann, wenn er sich zum Essen setzt, ohne daß
nicht ein paar andere schon bereit wären, ihn zu verklagen .....
Obgleich es eine Zeit grausamer Verfolgungen war, so war
doch Gottes Kraft über allen, und sein ewiger Same trug den Sieg
davon; und es wurde den Freunden gegeben, festzustehen und treu
zu bleiben in der Kraft des Herrn. Einige einsichtßvolle Leute
von anderen Glaubenßrichtungen bekannten, wenn die Freunde
nicht außharrten, so würde dat? Land dem Laster verfallen.
Obgleich ich durch meine Schwachheit verhindert war, wie
gewohnt bei den Freunden herumzureisen, so sandte ich doch nach
einem inneren Antrieb folgende Zeilen alö Aufmunterung:

\grosszitat{
  Liebe Freunde,

  Der Same über allen. Wandelt darin, in ihm habt ihr
  alle das Leben, lasset euch nicht irre machen durch die böse Zeit,
  denn der Gerechte hatte je und je vom Ungerechten zu leiden,
  aber der Gerechte trug den Sieg davon zu jeder Zeit. GZ ist
  allezeit so gewesen; durch den Glauben wurden Berge bezwungen,
  und wurden der Zorn der Gottlosen und die feurigen Pfeile des
  Bösewichtß ausgelöscht. Wenn gleich die Wellen und Stürme
  hoch gehen, so wird euer Glaube euch doch darüber halten, denn
  jene sind zeitlich, die Wahrheit aber ist ewig. Darum bleibet auf
  dem heiligen Berge, wo kein Unheil euch tressen wird. Denket
  nicht, daß irgend etwas die Wahrheit überdauern werde; sie
  % \picinclude{./210-219/p_s213.jpg} 
  Reise nach Jrland. Rückkehr und Heirat mit Margaret Fell usw. 213
  stehet fest und ist über allem, das nicht auß der Wahrheit ist;
  das Gute wird daß Böse überwinden, das Licht die Finsterniz,
  die Tugend das Laster, die Gerechtigkeit daß Unrecht. Der falsche
  Prophet kann den wahren nicht überwinden, aber der wahre
  Prophet, Christus, wird alle falschen überwinden. Darum bleibet
  treu und harret auß in Geduld.

  \begin{flushright}G. F.\end{flushright}

} 

Einige Zeit darauf gesiel eß dem Herrn, die Hitze dieser
grausamen Verfolgung zu dämpfen, und ich fühlte in meiner
Seele trotz meiner äußern Schwachheit den Sieg über die Geister
jener Menschensresser, welche sie angestiftet und bis zu solcher
Grausamkeit weiter geführt. Und ich fühlte deutlich, und die
Freunde, die bei mir waren und zu mir kamen, sahen dies, daß,
mit dem Aufhören der Verfolgung, ich frei wurde von dem Druck
und den Leiden, die so schwer auf mir gelegen hatten, so daß ich
gegen den Frühling anfing, mich zu erholen und umherzugehen
über alle Erwartung vieler, die nie gedacht hätten, daß ich je
wieder herumreisen würde.
Während ich unter dieser seelischen Anfechtung war, ward
mir der Zustand des Neuen Jerusalem, das vom Himmel herunter
kommt, geoffenbart, daß einige fleischlich Gesinnte sich als eine
sichtbare, auß greifbaren Stoffen gemachte Stadt vorgestellt hatten.
Jch sah seine Schönheit und Herrlichkeit, seine Länge, Breite und
Höhe, alleß in schönem Verhältniß. Ich sah, daß alle, die im
Lichte Christi und im Glauben an ihn sind und im heiligen
Geiste, in welchem Christuö und seine Apostel und Propheten
waren, und in der Gnade, der Wahrheit und der Kraft Gotteß,
in dieser Stadt sind, Glieder derselben sind und daß Recht haben,
vom Baume de-8 Lebens- zu essen, welcher jeden Monat seine Frucht
gibt, und dessen Blätter den Völkern Heilung bringen (Ofstz. 22).
Die aber nicht in der Gnade Gottes, der Wahrheit, dem Licht,
dem Geist und der Kraft Gottes- sind, und die ,,dem heiligen Geist
widerstehen und die Gnade Gotteß aus Mutwillen ziehen« (Jud. 4),
die vom Glauben abgeirrt sind und die ,,Verheißungen, Offen-
barungen und Gingebrmgen verachten«, dieseö fmd die Hunde
und die Ungläubigen, die draußen sind (Offb. 22). Diese bilden
die große Stadt Babylon, die Verwirrung, und ihr Behältniö ist
die Macht der Finsterniö, und der böse Geist dez Jrrtumß umgibt
und bedeckt sie. Jn dieser großen Stadt Babylon sind die


% \picinclude{./210-219/p_s214.jpg} 
214 Kapitel 3711.
falschen Propheten, die in einem oerkehrten Geist und einer
falschen Krast stehen, daß Tier, das in der Gewalt des Drachen
ist, die Hure, die den heiligen Geist und Chriftuin ihren Gemahl
verlassen hat (Hos.-1). Aber dez Herrn Macht ist größer ale
alle Macht der Finsternis, alle falschen Propheten und ihre An-
beter: diese ,,gehören in den feurigen Pfuhl« (Oss. 19, 20). ....
Jch sah noch viele Dinge über daß himmlische Jerusalem, welche
aber schwer zu beschreiben und noch schwerer zu Verstehen wären ....
Während ich in Enfield wirkte, spürte ich einen Schaden,
der öfters- unter den Bekennern der Wahrheit vorkam; nämlich,
wenn sie in ein anderes Land zogen, heirateten sie unter Freunden,
bei denen sie stemd waren, und von denen man nicht wußte, ob sie
makellos: und ordentlich seien oder nicht. Und eine innere Stimme
hieß mich ihnen folgende-H Verfahren zur Verhütung derartiger
Mißstände zu empfehlen:
,,Alle Freunde, die sich verheiraten, Männer wie Frauen,
sollen, wenn sie aus einem anderen Land, einer andern Gegend
oder Jnsel kommen, der Männeroersammlung, der sie ihre Absicht
zu heiraten, vorlegen, eine Bescheinigung von der Männewer-
sammlung aus dem Ort ihrer Herkunft bringen. Denn da die
Männewersammlung aus Gläubigen besteht, so werden die
herumschwärmenden bösen Geister gebannt. Kommt nun einer
mit einer Bescheinigung oder einem Empfehlung?-schreiben einer
Männeroersammlung zu einer andern, so wird diese durch jene
erquickt, und sie kann die Sache getrost unternehmen. Dies wird
viel Verdruß ersparen. Und waö ihr ihnen dann in der Kraft
Gottes- zu sagen habt in Ermahnung und Lehre, daß tut in der
Kraft und dem Geist Gotteß; lasset sie die Pflichten und die Be-
deutung der Ehe wissen. Die Einigkeit im Geist und Kraft,
Licht und Weißheit von Gott möge unter allen Männeroersamm-
kungen in der ganzen Welt herrschen in dem Einen, dem Leben.
Laßt hiervon Abschriften in jedeö Land, jede Gegend und
Jnsel, wo Freunde sind, senden, damit alle Dinge heilig, rein
und gerecht bewahret bleiben in Einigkeit und Füedctt, und daß
Gott über alleß gepriesen werde unter euch, seinen Auöerwählten,
seinem Volk und Erbe, die ihr seine et-wählten Söhne und Töchter
und Erben seines Lebens seid. Soviel davon; meine Liebe in
dem, daß nicht ändert.
Den 1-1.deS 1. Monats 1671. G. F.


% \picinclude{./210-219/p_s215.jpg} 
Reise nach Amerika. Varbadoez. Jamaika. 215
Zu dieser Zeit trieb eö mich, den Herm also anzurufen:
,,Herr, Gott, Allmächtiger!
Fördere die Arbeit und schütze die Gerechtigkeit und Billig-
keit im Land. Steure der Boßheit und Ungerechtigkeit, Bedrückung
und Falschheit, Grausamkeit und Unbarmherzigkeit, auf daß
Barmherzigkeit und Gerechtigkeit möge überhand nehmen.
O, Herr, Gott! Nichte die Wahrheit im Lande aus und
schütze sie. Tilge auß alleß Laster, Hurerei, Abgötterei, den Geist
der Unzucht, welcher macht, daß das Volk dich nicht ehrt, noch
ihre Seelen, noch ihren Leib, noch das Christentum, noch Zucht,
noch Menschenwürde.
O, Herr, gib der Obrigkeit ins Herz, all diesem ungöttlichen
Wesen, dieser Gewalttätigkeit und Grausamkeit, der Gottlosigkeit und
dem Fluchen zu wehren, und alle schlechten Häuser und Spiel-
häuser au-Jzurotten, welche die Jugend und daß Volk verderben,
und sie deinem Reich entführen, in welcheß nichtß Unreineö je
eingehen kann. Solcheö Treiben fiihret die Leute in die Hölle.
Herr, reinige daß Land von allen diesen Dingen nach Deiner
Barmherzigkeit, daß Dein Zorn gestillet werde, o Gott, und nicht
über daß Land hereinbreche.
17. des 2. Monatß 1671. G. F.
Kapitel Islll.
Reise nach Amerika. Barbadoes. Jamaika.
Wie schon erwähnt, hatte ich zwei Töchter meiner Frau
zum König geschickt, um ihre Freisprechung zu erwirken, und sie
hatten auch seinen dießbezüglichen Befehl dem Besehlßhaber in
Laneashire gebracht; . . . aber der Sturm der Verfolgung war
gerade so mächtig geworden, daß man Mittel fand, sie weiter
gefangen zu halten. Als nun aber die Verfolgungen etwas
nachließen, trieb eS mich, Martha Fischer und eine andere Frau
aus- dem Kreise der Freunde zu veranlassen, abermalß zum König
zu gehen, um ihre Freilassung zu erbitten. Sie gingen im Glauben
an die Kraft dez Herrn, welcher sie Gnade finden ließ vor dem
König, so daß er einen befiegelten Freilassungöbefehl bewilligte,
nachdem sie fast zehn Jahre gefangen gewesen war, und ihre Gitter
mit Beschlag belegt, dergleichen kaum je in England war erhört
worden. Ich schickte die Freisprechang sofort zu ihr durch einen


% \picinclude{./210-219/p_s216.jpg} 
216
F Kapitel :07111.
reund, und zu ls . . »
befehl müsse deis essiceschsecse-Kleb ich she- wie sie den Frestassungs-
mit, daß eg über m. zu emmen lassen und teilte ihr auch
gehen nach Amerikaschsieesessssässsxsszn sel v1smHerM« überß Meer zu
Ich London eilen: da dag Sghrtzxm,. sobald es:-’ ihr möglich sei,
em. dee Zwischenzeit Sing ich nach fdisschstschon Re Abreise süsse
meme Frau kam, und dann rütt «g on-zu sohn Nous, biz
weil die Jahreßversammlung IMO etetgch mich zur Reise. Doch
zu derselben .... Dann, als unseraSsand, so blieb ich noch bis
mlch zu begleiten beabsichtigten bereit wchsss uns dee Freunde, die
6. Monate- 1671 nach Gravesend und elren, gmg sch, am 12« dez
 Z2 hegteszeten mich ck-xr ds2«’WtH?’e UI-meh reee de
. esse mit mir machten waren. T ' sse Freunde,
Idmundson, John RMS Joh;1StubbH Hofnaß Briggö, William
aueuskek, John Eartwr« t R s s o omon Eeeleee- Jameß
John Hull, Elisabeth HUIZVL u::exl·WHdderS,.George Pattison,
war eine Jacht und hieß ,,IndufMe,,ssT eeh ME ' Unser Schiss
Forster, und wir waren etwa 50 Passa er Kapttan hieß Thomaz
1 A15 wir etwa drei Wochen auf demgßxzzv. . . .
Wlr etwa yjer Seemeilm hinter UW ein S ter waren, bemerkten
sßzgse, es E ein maursscheß Piratensehifs djsdnsunxnser Kapitän
eme. ,, ommt,« sagte er - « — zu Verfolgen
Mgmt ez dunkel geworden ist sßkjxrxollen zum Abendessen Sehen,
Dietz sa te e · Z en sse unsere Spur Verlie «
ich .9. r, um die Reisenden zu beruhi en d ssrens
UI Mge an sich zu äugstigen. Die Fseutsdcesnis eg singen
Heek: b dtezt weil sie Gott vertrauten und keinerleseßzsechchsssaren
e ru e. Als d' , uk ihr
meiner Kajüte aus ie Sonne untergegangen war, sah ich om:
dunkel wurde, ände1:tersVs:1irds;IseKske1ss aus uns zukam. Alß ez
aber es änderte die seine auch, Untzchtßlxßtz   ihm auzzuweichen;
kamen der Kapitän und andere zu mir in o. e Unßj » Km der Nacht
mich, was sie mn sollten. Ich Mw t:neme.Kaiute und fragten
mann, und fragte sie, Wag sie für das HejI:,hFes?tsetz kein Schiffß-
ZK-lgabe nur zwei Wege: entweder wir müßtee dns; Fw fagtem
O en, oder hin d tkt . . u a Schiff über-
wie vorher. Gun he euzen und die gleiche Richtun ein al
uch sagte wenn etz Rss - g h ten
sicherlich auch hin- und derkreuzen U gubergseien, so werden sie
tzelange, fo sei daran gar nicht zu dxsnkxxlasddaz Uberholen an-
te viel schneller fahren sz - . . s a man ja sehe wie
a wir. Sie fragten mich wieder, maß


% \picinclude{./210-219/p_s217.jpg} 
Reise nach Amerika. Bardadoczß. Jamaika. 217
sie denn tun sollten: ,,denn,« sagten sie, ,,wenn die Schifföleute da-
malß den Rat deß Paulutz befolgt hätten, so wäre es ihnen
nicht so schlimm ergangen.« Jch erwiderte: ,,EZ ist eine Glaubens-
prüfung, und darum muß man auf den Herrn warten und auf
seinen Rat.« Während ich mich nun innerlich sammelte, zeigte
mir der Herr, daß er mit seinem Leben und mit seiner Kraft
zwischen uns und dem Schiff, das unö verfolgte, stehe. Ich
teilte diez dem Kapitän und den anderen mit, und daß ez nun
daß Beste sei, zu kreuzen und den rechten Kurz einzuschlagen.
Jch hieß sie auch alle Lichter au?-löschen außer dem einen, das
sie beim Steuer brauchten, und den Reisenden sagen, sie sollten
sich still und ruhig verhalten. Jn der Nacht etwa um 11 Uhr,
kam die Wache und sagte, sie seien ganz nahe hinter unö. Daß
beunruhigte einige der Reisenden. Jch richtete mich in meiner
Kajüte auf, und da der Mond noch nicht untergegangen war. sah
ich durch die Luke, daß sie ganz nahe waren. Jch wollte auf-
stehen und hinaußgehen; aber ich erinnerte mich der Worte des
Herrn, ,,daß er mit seinem Leben und seiner Kraft zwischen uns-
und ihnen stehe,« und legte mich wieder nieder. Der Kapitän
und einige der Schiff?-leute kamen abermals- und fragten, ob sie
nicht nach dieser oder jener Richtung steuern sollten? Jch sagte
ihnen, sie sollten machen, wie sie wollten. Da ging der Mond
vollends unter, ein neuer Wind erhob sich, und der Herr oerbarg
nnß vor ihnen; wir segelten rasch und sahen sie nicht mehr. Am
folgenden Tag, einem Ersten Tag, hatten wir eine öffentliche
Versammlung auf dem Schiffe, wie wir sie gewöhnlich während
der ganzen Reise an diesem Tage zu halten pflegten; und deö
Herrn Gegenwart war mächtig unter uns. Und ich ermahnte die
Leute, an Gottes Barmherzigkeit zu denken, die sie errettet; denn
sie wären jetzt vielleicht alle in den Händen der Türken, wenn
des Herrn Hand sie nicht errettet hätte. Etwa eine Woche
darauf suchten der Kapitän und einige der Schiffs-leute den
Reisenden einzureden, etz seien nicht türkische Seeräuber gewesen,
die uns verfolgten, sondern ein Kaufmamisßschiff, das nach den
Kanarischen Jnseln ging. A16 ich das hörte, fragte ich sie,
warum sie denn dann solcheß zu mir gesagt hätten? warum sie
die Reisenden beunruhigt hätten? und warum sie, um ihnen
davon zu fahren, den Kurz geändert hätten? Sie sollten sich
hüten, Gotteß Barmherzigkeit zu verachten. Später, als wir in


% \picinclude{./210-219/p_s218.jpg} 
218 Kapitel Kslll.
Varbadoes waren, kam ein maurischer Kaufmann und erzählte den
Leuten, die Mannschaft eines maurischen Piratenschiffs habe auf
dem Meer ein ungeheures Jachtschiss gesehen, das größte, das
sie je gesehen hätten, sie hätten es verfolgt, und seien schon ganz
nahe gewesen, aber es sei ein Geist darin gewesen, so daß sie es
nicht erobern konnten. Dies bestätigte uns in unserer Uberzeugung,
daß es ein maurisches Piratenschiff war, das uns verfolgte, und
daß es der Herr gewesen, der uns befreit hatte.
Ich war nicht seekrank gewesen auf der Reise, wie so viele
der Freunde und andere Reisende; aber alle die Wunden und
Schläge, die ich früher erlitten, die Krankheiten, die ich mir durch
die Kälte und die Gntbehrungen während meiner Gesangenschasten
zugezogen hatte, machten sich nun während der Reise wieder
geltend, so daß mein Magen sehr angegriffen war, und ich heftige
Schmerzen in allen Gliedern hatte. Es fing an, nachdem ich
etwa einen Monat auf der See war; zuerst schwitzte ich stark,
und an Kopf und Leib zeigten sich überall Pusteln, nnd meine
Hände und Füße wurden so geschwollen, daß ich nur mit Mühe
und unter großen Schmerzen meine Strümpfe und Pantoffeln
anziehen konnte; auf einmal hörte das Schwitzen auf, und
als ich in das heiße Klima kam, wo die anderen tüchtig schwitzten,
konnte ich gar nicht schwitzen, sondern mein Körper war heiß
und trocken und brennend, und was vorher in Pusteln aus-
gebrochen war, schlug jetzt nach innen auf Herz und Magen, so
daß ich sehr krank war und über alle Maßen schwach; dies
dauerte während der ganzen übrigen Zeit der Reise, während
der etwa vier Wochen, die wir noch auf dem Wasser waren.
Am frühen Morgen des 3. Tages des 8. Monats erblickten wir
die Jnsel Barbadoes, aber es dauerte noch bis zwischen neun und
zehn des Abends, ehe wir in den Hafen der Carlisle=Bay ein-
fuhren. Wir gingen sobald wie möglich ans Land, und ich be-
gab mich mit einigen Freunden in das Haus eines Freundes,
eines Kaufmanns namens Richard Forstall, der etwa zehn
Minuten von der Landungsbrücke wohnte. Aber ich war so
krank und schwach, daß ich sehr müde wurde von diesem kurzen
Gang, und Vollständig erschöpft ankam. Jch lag dort mehrere
Tage krank, und obgleich man mir mehrmals Mittel gab, um
mich schwitzen zu machen, so kam es doch nie zu einem rechten
Schweiß. Was sie mir gaben, oertrocknete eher meinen Körper


% \picinclude{./210-219/p_s219.jpg} 
Reise nach Amerika. Barbadoes. Jamaika. 219
noch mehr, und machte mich noch kränker, als ich sonst gewesen
wäre. Diese Schmerzen in allen Gliedern dauerten etwa drei
Wochen, und ich litt sehr, so daß ich kaum je Ruhe finden
konnte, aber ich war ziemlich getrost und der Geist ward Herr
über alle?-. Auch hinderte mich meine Krankheit nicht am Dienst
für die Wahrheit, sondern sowohl auf der See als in Barbadoeß,
ehe ich herum reisen konnte, gab ich verschiedene Schriften
heraus, die ein Freund für mich schrieb, und von denen ich einige
mit der ersten Gelegenheit nach England schickte, um gedruckt
zu werden .....
Weil ich so schwach war, daß ich nicht an die verschiedenen
Versammlungen reisen konnte, nahmen sich die andern Freunde
dez Werkes dez Herrn an; schon am Tage nach unserer Ankunft
hatten sie eine große Versammlung an der Landungzbrücke, und
nach derselben noch mehrere in verschiedenen Teilen der Jnsel,
maß die Bevölkerung sehr in Aufregung brachte, so daß viele zu
den Versammlungen kamen, worunter mehrere von hohem Rang;
denn sie hatten gehört, daß ich aus der Jnsel angekommen sei
und erwarteten, mich bei den Versammlungen zu sehen, da sie
nicht wußten, daß ich zu schwach war, um zu kommen. Meine
Schwachheit wich darum so lange nicht von mir, weil mein Ge-
müt zuerst sehr niedergedriickt war von der Schmutzigkeit und
Ungerechtigkeit und Gemeinheit der Leute, maß wie eine schwere
Last auf mir lag. Aber nachdem ich etwa einen Monat auf der
Jnsel gewesen war, wurde es mir etwaß leichter zu Mut, und
ich fühlte mich wieder etwaß kräftiger, so daß ich wieder umher
gehen konnte zu den Freunden .....
Weil ich aber doch nicht gut viel umher reisen konnte, so
kamen die Freunde auf der Jnsel überein, die Männer- und
Frauen-Versammlungen zur Ordnung der kirchlichen Angelegen-
heiten im Hause Thomaß Rouz, bei dem ich wohnte, abzuhalten,
so daß ich bei allen Versammlungen dabei war und recht für den
Herrn wirken konnte. Denn sie hatten in manchen Dingen Be-
lehrung nötig, weil sich auö Mangel an Vorsicht und Wachsam-
keit allerlei Unordnungen eingeschlichen hatten. Jch ermahnte sie,
besonderß in der Männeroersammlung, recht vorsichtig und wach-
sam in bezug auf das Heiraten zu sein und die Freunde zu
verhindern, in die Verwandtschaft zu heiraten, sowie auch zu
hastig vorzugehen bei Wiederoerheiratung nach dem Tode des


% \picinclude{./220-229/p_s220.jpg} 
220 Kapitel Illlll.
Manneö oder der Frau. Jch ermahnte sie, daß in solchen Fällen
dem verstorbenen Teile die geziemende Ehrerbietung sollte bezeugt
werden. Jch wieß sie auch daraus hin, wie unziemlich eö sei,
ihre Kinder so früh einander zu verheiraten, mit dreizehn und
vierzehn Jahren, und waß für Schäden und Nachteile auö solchen
frühen Heiraten entstehen. Jch ermahnte sie ferner, ihre Fuß-
böden gründlich zu reinigen, ihre Häuser rein zu halten und auch
außerhalb der Versammlungen einander nicht mit verleum-
derischen Reden zu schaden. Jch ermahnte sie, genaue Ver-
zeichnisse zu führen über Geburten, Heiraten und Beerdigungen,
in eigenö dazu bestimmten Büchern; auch sollten sie ein be-
sonderez Buch führen über die Bestrafungen solcher, die von der
Wahrheit abweichen und einen unordentlichen Wandel führen,
und über Buße und Wiederaufnahme solcher, die wieder zurück
kommen. Jch empfahl ihnen an, sür geeignete Begräbniöplätze
zu sorgen, die an etlichen Orten noch fehlten. Jch gab ihnen
auch einige Räte inbetreff der Vermäehtnisse, welche Freunde zu
beliebigem Gebrauch hinterlassen hatten, und wie sie darüber
verfügen sollten, und über allerlei andere kirchliche Angelegenheiten.
 Jnbetreff der Schwarzen oder Neger, hieß ich versuchen, die-
selben in der Furcht Gotteö zu unterweisen, sowohl die gekauften
alö die, welche in der Familie geboren wurden, damit alle dazu
kommen möchten, den Herm zu kennen, so daß jeder Hau?-vater
mit Josua sagen könne: ,,ich aber und mein Haus wollen dem
Herrn dienen«. Jch ermahnte sie auch ihre Aufseher dazu zu
bringen, mild und freundlich gegen die Neger zu sein, und sie
nicht grausam zu behandeln, wie viele es taten und noch run,
und sie, wenn sie einige Jahre alß Sklaven gedient, freizulasseng
Viele köstliche, herrliche Dinge wurden in diesen Versammlungen
offenbar, durch den Geist und die Kraft Gotteß, zur Erbauung,
Ausrichtung und Stärkung der Freunde im Glauben und der
Heiligen Ordnung dez Evangeliums .....
A18 e3 mir wieder besser ging, machten wir dem Gouverneur
einen Besuch. Er empfing unö sehr höflich und behandelte uns-
sehr freundlich und hieß unß mit ihm zu Mittag essen. Jn der
gleichen Woche ging ich nach Bridge-Town; und da die Behörden,
die militärischen wie die andern, von meinem Besuch beim Gou-
verneur und seiner freundlichen Aufnahme gehört hatten, so kamen
aus allen Teilen det? Lnndetz viele Leute von hohem Rang,


% \picinclude{./220-229/p_s221.jpg} 
Reise nach Amerika. Barbadoeö. Jamaika. 221
Richter, Friedenßrichter, Oberste, Hauptleute zu dieser Versamm-
lung ..... Von den Freunden, die mit mir gekommen waren,
gingen viele nach Jamaika und andere Orte, so daß wenige mit
mir in Barbadoeö blieben. Wir hatten viele große und schöne
Versammlungen .... sie waren friedlich und nicht gestört von
Seiten der Regierung; jedoch gehässige Priester .... und Bap-
tisten .... und Fromme brachten Schmähschriften gegen unß ....
diesen traten wir mit einer Schrift entgegen, die im Namen der
sogenannten Quäker sollte verbreitet werden, um die Wahrheit
und die Freunde von solchen falschen Anschuldigungen zu reinigen.
.... E3 hieß darin unter anderem: .... ,,eine Verleumdung,
die sie gegen unß au?-streuten ist, daß wir die Neger zu Aufftänden
anstiften, und gerade das Verabscheuen wir im Jnnersten; der
Herr, der die Herzen prüft, weiß eß, und kann uns daß Zeugniz
geben, daß dietz eine ganz abscheuliche Unwahrheit ist.  Wir haben
in bezug aus sie gesagt, man solle sie lehren, nüchtern und recht-
schaffen zu sein, Gott zu fürchten und ihre Herren und Herrinnen
zu lieben und treu und fleißig ihren Herren zu dienen; dann
würden ihre Herren und ihre Aufseher sie lieben und gütig und
freundlich behandeln; auch sollten sie ihre Weiber nicht schlagen
noch die Weiber ihre Männer, und die Männer sollten nicht
mehrere Weiber haben; sie sollten nicht stehlen noch sich betrinken,
nicht Ehebruch noch Unzucht treiben, nicht fluchen, nicht schwören,
nicht lügen oder sich unter einander beschimpfen, denn es sei
etwas in ihnen, daß; ihnen sage, sie sollten diese und andere
schlechte Dinge nicht tun. .s’’ Wenn sie sie aber dennoch tun, so
kehrten wir sie, daß es- nur zwei Wege gibt, der eine, der zum
Himmel sührt, den die Gerechten gehen, und der andere, der
zur Hölle führt, den die Gottlosen gehen und die Ghebrecher,
Hurer, Mörder und Lügner. Zu den Einen wird der Herr sagen:
Kommet, ihr Gerechten meinetz Vaterö, erbet daß Reich! Zu
den Andern aber wird er sagen: Gehet hin, ihr Verfluchten in
das ewige Feuer! und so werden die Ungerechten in die ewige
Pein gehen, die Gerechten aber in das ewige Leben (Matth. 25).
Wisset, Freunde eS ist keine Schande für einen Haußvater,
die Seinen selber zu unterweisen oder jemand andertz etz für ihn
tun heißen, vielmehr ist es eine wichtige Pflicht, die ihm zu
tun auferlegt ift. Abraham und Josua haben etz also gemacht.
Vom ersteren heißt ez Genesiö 18, 19.5: ,,er wird befehlen seinen


% \picinclude{./220-229/p_s222.jpg} 
222 Kapitel Il-’1ll.
Kindern und seinem Hauß, daß sie dez Herrn Wege halten .... «,
und der zweite sagt, Josua 24,15: ,,erwählet euch heute welchem
ihr dienen wollt; ich aber und mein Hauß wollen dem Herrn
dienen«. Wir erklären, daß wir eö für unsere Pflicht halten,
mit denen und für die zu beten, die unsrem Hause angehören,
und sie zu lehren und zu ermahnen; denn eß ist dietz ein Befehl
vom Herrn und der Ungehorsam hiegegen wird sein Mißfallen
erregen, wie wir Jeremiaß I, 25 sehen können: ,,Schütte deinen
Zorn über die Heiden, die dich nicht kennen und über die Fa-
milien, die deinen Namen nicht anrufen«. Nun bilden die
Reger, die Rothäuter, die Twanieß, die Jndianer überall einen
großen Teil der Familien hier auf dieser Jnsel, und etz wird Rechen-
schaft über sie gefordert werden von dem, der kommen wird zu
richten die Lebendigen und die Toten am großen Tage deß
Gerichtß: ,,da ein jeder empfangen wird seinen Lohn, nach dem
er gehandelt hat, eö sei gut oder böse«, wenn er ,,wird geofsen-
baret werden mit Feuerflammen, Rache zu geben über die so
Gott nicht erkennen;« .... und ,,eß werden in den letzten Tagen
Spötter kommen, die nach den eigenen Lüften wandeln« .... ,,eZ
wird aber dez Herrn Tag kommen wie ein Dieb in der Nacht«. . . .
wie 2. Thess. 1,8 und 2. Pet. Z zu sehen ist.«
Die Veranlassung zu diesem Gerücht, daß wir versuchten die
Neger aufzuhetzen, hatten unsre Gegner darauö geschöpft, daß
wir Versammlungen mit und unter den Negern gehabt hatten,
denn sowohl ich ale; andere der Freunde hatten mehrere Ver-
sammlungen mit ihnen in verschiedenen Plantagen, in denen wir
sie zu Rechtschafsenheit, zur Keuschheit, zur Nüchternheit und zur
Frömmigkeit ermahnten, und zum Gehorsam gegen ihre Herm
und Meister, also gerade das Gegenteil von dem, wasz unsre
übelwollenden Gegner bößwillig gegen unß au?-streuten. ....
Ehe ich die Jnsel verließ, schrieb ich folgenden Brief an
meine Frau:
,,Mein liebetz Herz,
Welcher meine Liebe gehört, sowie allen Kindern im Samen
dez Lebenß, der sich nicht verändert, sondem größer ist alß alleß,
gelobt sei der Herr ewiglich. Jch habe unauösprechlich an Seele
und Leib zu erdulden gehabt; aber der Gott dez Himmels sei
gelobt, seine Wahrheit geht über alleß. Ich bin jetzt gesund, und
so der Herr will, gehe ich in einigen Tagen von Barbadoeö nach


% \picinclude{./220-229/p_s223.jpg} 
Reise nach Amerika. Varbadoes. Jamaika. 223
Jamaika und gedenke nur kurze Zeit dort zu bleiben. Jrh hoffe,
daß ihr alle im Samen des Lebens, frei von aller Kümmernis,
bewahret bleibet. Die Freunde sind im allgemeinen wohl. Grüße
mir die Freunde, die nach mir fragen. Soviel diesmal. Meine
Liebe im Samen und Leben, die nicht wechseln.«
Varbadoes, 6. des 11. Monats 1671. G. F.
Jch schisste mich am 8. des 11. Monats 1671 in Barbadoes
fiir Jamaika ein ..... Wir hatten eine gute, rasche Überfahrt, ....
und trafen in Jamaika James Lancaster, John Eartwright und
George Pattison wieder, die eifrig im Dienste der Wahrheit ge-
arbeitet hatten, dem wir uns nun auch widmeten, wir reisten
aus der Jnsel hin und her; es ist ein recht schönes Land, doch
sind die Leute zum Teil recht verdorben und ausschweifend. Wir
wirkten viel. Gs war eine große Belehrung und viele nahmen
die Wahrheit aus, worunter manche angesehene Leute. Wir
hatten viele Versammlungen hier, die zahlreich und ganz ruhig
waren. Die Leute begegneten uns sehr anständig und niemand
tat den Mund gegen uns auf. Jch war zweimal beim Gouver-
neur und den Behörden, die sehr freundlich gegen mich waren.
Etwa eine Woche nach meiner Ankunft in Jamaika schied
Elisabeth Hooton, eine sehr alte Frau, die viel im Dienst der
Wahrheit umhergereist war und viel dafür gelitten, aus diesem
Leben. Sie war noch am Tage vor ihrem Tode gesund und
schied in Frieden, und gab noch im Sterben der Wahrheit die
Ehre. Nachdem wir etwa sieben Wochen in Jamaika gewesen,
und unter den dortigen Freunden etwas Ordnung geschaffen und
mehrere Versammlungen unter ihnen eingerichtet hatten, ließen
wir Solomon Gccles dort, und schisften uns für Maryland
ein .....
Ehe ich Jamaika verließ schrieb ich noch einmal einen Brief
an meine Frau:
,,Mein liebes Herz,
Dir und den Kindern meine Liebe in dem, das über allem
ist und sich nicht verändert, und allen Freunden die bei euch sind,
Jch bin nun etwa fünf Wochen in Jamaika gewesen. Den
Freunden geht es im ganzen gut und wir haben große Be-
kehrungen gehabt; aber es würde zu weit führen, über alles zu
schreiben; überall warten Leiden meiner, aber der gesegnete Same
ist über allem. Der Herr sei .-gelobt, welcher Herr ist über Land


% \picinclude{./220-229/p_s224.jpg} 
224 Kapitel X11.
und Meer und alleß waz darinnen ist. Wir haben im Sinn,
etwa anfangß deß nächsten Monatö von hier abzureisen nach
Maryland zu, so der Herr will. Bleibet alle miteinander im
Samen dez Herrn; in seiner Wahrheit bleibe ich in der Liebe zu
euch allen.«
Jamaika, 23. deß 12. Monatß 1671. Gs Fs
Kapitel III.
Arbeit in N-tdcmetita unter Engländern nnd Indianern.
Wir schissten unß am 8. des 1. Monats 1671 ein; und da
wir schlechten Wind hatten, segelten wir eine ganze Woche hin
und her, ehe wir von Jamaika fort kamen. EZ war eine schwie-
rige und gefahroolle Reise, besonderö alö wir den Golf von
Florida passierten, wo wir manche Schwierigkeiten durch Wind
und Sturm zu bestehen hatten. Aber der große Gott, welcher
Herr ist über Meer und Land, welcher aus den Flügeln des
Windeö dahin fährt, bewahrte unß durch seine Kraft vor vielen
großen Gefahren, wenn bei dem Ungestüm deß Wetterö unser
Schiff oft nahe daran war umzuschlagen, und daß Tauwerk großen-
teiltz zerbrochen wurde. Wahrlich, wir merlten, daß der Herr ein
Gott der Nähe ift, und hört auf daß Flehen seines Volkeß. Denn
alö die Winde so stark und heftig tobten, und ez so mächtig
stürmte, daß die Schifföleute sich nicht zu helfen wußten, und
daß Schiff sich selbst überließen, da beteten wir zum Herrn, welcher
unß gnädig erhörte, Wind und Wellen siillte und unö günstiges
Wetter gab, so daß wir unö unsrer Errettung freuen durften.
Gelobt und gepriesen sei der herrliche Name dez Herrn, der
Macht hat über alleß, dem Wind und Wellen gehorchen. ....
Wir waren etwa sechß biß sieben Wochen untemzegß von
Jamaika nach Maryland .... Dort trafen wir John Burnyeat,
der die Absicht hatte, sich bald nach England einzuschifsen, aber
als wir kamen, änderte er seinen Vorsatz und schloß sich uns an
zum Dienst für den Herm .... Er hatte eine Versammlung für
alle Freunde von Maryland veranstaltet, damit er sie alle mit-
einander sehe, um Abschied von ihnen zu nehmen; und nun
fügte etz die Votsehung Gotteö so, daß wir gerade zur rechten
Zeit landeten, um dieser Versammlung beizuwohnen; .... ES
war eine sehr große Versammlung, die vier Tage dauerte ....


% \picinclude{./220-229/p_s225.jpg} 
Arbeit in Nordamerika unter Engländern und Indianern. 225
Nach der allgemeinen Versammlung fingen die Männer- und
Frauen-Versammlungen an ..... Hernach gingen wir nach einem
andern Orte, die Klippen genannt, wo eine andere große Ver-
sammlung stattfinden sollte. Wir gingen einen Teil dez Wegeö
zu Land, den Rest zu Wasser; und da sich ein Sturm erhob, stieß
unser Boot aus und wäre fast zertrümmert worden, und daß
Wasser drang herein. Ich schwitzte stark, da ich sehr warm au?-
der Versammlung gekommen war, und nun wurde ich vom Wasser
ganz durchnäßt; aber weil ich Glauben hatte in die göttliche Kraft,
wurde ich vor Schaden bewahrt, der Herr sei gepriesen .....
Wir hatten hier auch eine Männer- und Frauen-Versammlung,
und in vielen dieser Versammlungen wurde die Angelegenheit der
Kirche geordnet.
Nach diesen Versammlungen trennten wir uns und verteilten
unö auf die verschiedenen Küsten, zum Dienst der Wahrheit.
Jameö Lancaster und John Eartwright gingen zu Wasser nach
Neu-England; William Edmundson und drei andere Freunde
schifften sich für Virginia ein, wo die Dinge sehr in Unordnung
geraten waren; John Burnyeat, Robert Widderö, George Pattison
und ich mit einigen andern Freunden gingen mit einem Boot
nach der Ostküste und hatten dort am Ersten Tag eine Versamm-
lung, wo viele die Wahrheit mit Freude aufnahmen und die —
Freunde reichlich erquickt wurden. ES war eine große, selige
Versammlung, und es waren mehrere Personen von Nang auß-
der Gegend dabei, darunter zwei Zriedensrichter. ES kam über
mich vom Herrn, dem »Kaiser« der Indianer und seinen ,,Königen«
sagen zu lassen, sie sollten zu dieser Versammlung kommen. Der
,,Kaiser« kam und wohnte ihr bei, aber seine »Könige«, welche weiter
weg wohnten, konnten nicht zur rechten Zeit kommen; aber sie
kamen später nach, mit ihren Leuten. Ich hatte am Abend zwei-
mal eine gute Zeit mit ihnen, und sie hörten daß- Wort des Herrn
gerne und bekannten sich dazu. Ich bat sie, daß, wa-3 ich ihnen
sagte, dann auch ihrem Volke zu sagen und ihm zu verkünden,
daß Gott jetzt die Hütte des Zeugnisses in der Wüste aufrichte
und das Panier und segenöreiche Zeichen seiner Gerechtigkeit. Sie
benahmen sich sehr anständig und fragten, wann die nächste Ver-
sammlung sein werde, sie wollten dazu herkommen; aber sie
erzählten uns, sie hätten eine heftige Auszeinandersetzung gehabt
mit ihren Räten, wegen ihreö Kommenö. Am folgenden Tage
Stotz- F.-3. 15


% \picinclude{./220-229/p_s226.jpg} 
226 Kapitel R11.
traten wir unsre Reise nach Neu-England an, eine schwierige
Reise, durch Wälder und Sümpfe und große Flüsse; dann mußten
wir die Wildnis passieren, die jetzt West-Jets:) genannt wird,
die aber damalß nicht von Engländern bewohnt war; einen ganzen
Tag reisten wir, ohne Mann oder Frau, Hauß oder Wohnort
zu tressen. Zuweilen nächtigten wir im Wald bei einem Feuer,
zuweilen in den Hütten und Häusern der Jmdianer. Eines Abends
kamen wir in eine indianische Ortschaft und übernachteten beim
,,König«, der ein sehr achten?-werter Mann war; er sowohl alö sein
Weib nahmen uns sehr liebevoll auf, und seine Dienerschaft be-
handelte unö sehr ehrerbietig. Sie gaben uns Matten, um darauf
zu schlafen, aber zu essen hatten sie wenig, da sie an dem Tage
wenig gefangen hatten. In einer andern indianischen Ortschaft, in
der wir unß aufhielten, kam der »König« zu uns; er sprach ein
wenig englisch. Ich redete viel mit ihm und auch mit seinen
Leuten, und sie waren sehr lieb mit uns. Schließlich kamen
wir nach Middletown, einer englischen Pslanzung in Ost=Jersy;
doch konnten wir nicht zu einer Versammlung bleiben, da e-3 unß
trieb, rechtzeitig zur Halbjahreßversammlung in der Oysterbay,
auf Long-JS-land, zu sein ..... Ein Freund, Richard Hartshorn,
setzte unö in seinem eigenen Boote über nach Long-Jtzland, und
am zweitfolgenden Morgen erreichten wir die Oysterbay, wo wir der
Halbjahreßversammlung beiwohnten ..... Nachdem dann die
Freunde wieder nach Hause gegangen waren, blieben wir noch
einige Tage auf der Jnsel und warteten dann in der Oysterbat)
auf günstigen Wind, um nach Rhode- Jöland zu gehen .....
Dort kamen wir am 13. dez 3. Monatß an und wurden mit
Freuden von den Freunden ausgenommen. Am nächsten
Ersten Tage hatten wir eine große Versammlung, welcher der
Unterstatthalter und mehrere von der Behörde beiwohnten. Sie
wurden mächtig von der Wahrheit ergriffen. Jn der daraus
folgenden Woche sand hier die Jahreöveriammlung für alle Freunde
von Neu-England und den übrigen angrenzenden Kolonien statt;
außer sehr vielen Freunden, die in diesen Gegenden lebten, kamen
noch John Stubbß auß Barbadoet? und Jameö Lancaster und
John Cartwrigth, von verschiedenen Seiten, um ihr beizuwohnen.
Diese Versammlung dauerte sechß Tage; an den vier ersten Tagen
waren allgemeine, gotteödienstliche Versammlungen, zu welchen
eine große Menge Leute kamen; denn sie hatten keinen Priester


% \picinclude{./220-229/p_s227.jpg} 
Arbeit in Nordamerika unter Engländern und Jndianern. 227
in Rhode=J-sland und darum keinerlei bestimmte Form irgend
einer Art von Gottesdienst; und weil der Unterstatthalter mit
mehreren von der Regierung täglich zu den Versammlungen kamen,
wurden die Leute ermutigt, so daß sie von allen Seiten herbei-
strömten. Wir hatten ein gesegnetes Wirken unter ihnen, und
die Wahrheit fand gute Ausnahme. Jch habe selten Leute von
dieser Art mit mehr Aufmerksamkeit, Fleiß und Liebe zuhören
sehen, als diese es im ganzen während der Vier Tage taten, was
auch andere Freunde beobachteten. Nachdem die öffentlichen
Versammlungen zu Ende waren, begann die Männerversammlung,
welche sehr zahlreich, köstlich und feierlich war; und tags daraus
war die Frauenversamnrlung, die ebenfalls sehr zahlreich und
feierlich war. Da diese beiden Versammlungen zur Ordnung
kirchlicher Angelegenheiten veranstaltet waren, so wurden viele
wichtige Dinge eröffnet und mitgeteilt, durch Rat, Belehrung und
Unterweisung, für das Verhalten in den verschiedenen in Frage
kommenden Verrichtungen, damit alles rein, lieblich und kräftig
unter ihnen erhalten werde. Verschiedene Männer- und Frauen-
Versammlungen für andere Gegenden wurden in diesen beiden
Versammlungen beschlossen und eingerichtet, zur Fürsorge sitr die
Armen und andere kirchliche Angelegenheiten, und damit dafür
gesorgt werde, daß alle, die die Wahrheit bekennen, auch nach
dem Evangelium wandeln. Als diese große Versammlung auf
Rhode-Jsland zu Ende war, wurde den Freunden der Abschied
etwas schwer; denn die herrliche Macht des Herrn, die über allen
war, und seine gesegnete Wahrheit und sein Leben, die sich über sie
ausgossen, hatten sie so untereinander verbunden und vereinigt,
daß sie zwei Tage damit zubrachten, um sich unter einander und
von den Freunden auf der Jnsel zu verabschieden. Darauf gingen
sie fort, mächtig erfüllt von der Gegenwart und der Krast Gottes,
sreudigen Herzens, jeder in seine Heimat, nach den verschiedenen
Kolonien, in denen sie lebten. ....
Während dieser Zeit fand eine Vermählung von zweien von
den Freunden statt, der wir beiwohnten. Gs war im Hause
eines Freundes, der früher Gouverneur dieser Gegend gewesen war;
drei Friedens-richter und viele, die nicht zu uns gehörten, waren
zugegen; aber alle, diese sowohl als die Freunde, sagten, sie hätten
noch nie eine so andächtige Zusammenkunft gesehen bei einem
derartigen Anlaß, noch eine so feierliche Vermählung rmd eine
15’


% \picinclude{./220-229/p_s228.jpg} 
228 Kapitel X11.
so Oorzügliche Ordnung. Dermaßen durchdrang die Wahrheit
alle. Es ist hoffentlich sür viele ein gutes Beispiel gewesen, denn
sie sind aus allen Teilen dez Landes dazu gekommen. Jch hatte
innnerlich viel durchzumachen wegen der Ranter in dieser Gegend,
die eine Versammlung, der ich nicht beigewohnt hatte, gestört
hatten. Jch zeigte darum eine Versammlung unter ihnen an, im
Glauben, daß der Herr mir Macht über sie geben werde, was
er auch tat, zu seiner Ehre und Verherrlichung. Sein Name
sei gelobt ewiglich .....
Darauf hatten wir eine Versammlung in Prooidence ....
ebenso in Narraganset ..... Von dort ging ich nach der Jnsel
Shelter. .... Dort hatten wir am Tage nach unsrer Ankunft,
einem Ersten Tage, eine Versammlung. Jn der gleichen Woche hatte
ich eine unter den Indianern, ihr ,,König« war zugegen und sein
Rat und einige hundert Indianer; sie setzten sich unter uns, ganz
wie die Freunde taten, und hörten aufmerksam zu, während ich
durch einen Dolmetscher, einen Jndianer, der gut englisch sprach,
zu ihnen redete. Sie waren nach der Versammlung sehr lieb
und bekannten, das, was man ihnen gesagt habe, sei Wahrheit
gewesen .....
Wir reisten nun umher und kamen schließlich nach Shrewsbury,
wo sich etwas ereignete, das damals eine wichtige Probe für uns
war. John Jay, ein Freund aus Barbadoes, der mit uns von
Rhode-Jsland gekommen war und uns durch die Wälder von
Maryland begleiten wollte, bestieg ein Pferd, um es zu versuchen;
das Pferd sing an zu gallopieren und warf ihn ab, so daß er
auf den Kopf fiel und den Hals brach, wie die Leute sagten.
Einige hoben ihn auf, in der Meinung, er sei tot, und trugen
ihn ein Stück weit und legten ihn unter einen Baum. Jch ging
sogleich zu ihm, und als ich ihn anriihrte, hielt ich ihn auch für
tot. Während ich neben ihm stand, und ihn und die Seinen
beklagte, fuhr ich ihm durch die Haare, wobei sein Kopf sich hin
und her drehte, so schlaff war sein Hals. Nun nahm ich seinen
Kopf zwischen meine beiden Hände, und, indem ich meine Knie
gegen den Baum stemmte, hob ich seinen Kopf in die Höhe und
sah, daß da nichts zerrissen oder gebrochen war; ich faßte ihn
nun mit einer Hand unter dem Kinn und mit der andern hinten
am Kopf und bewegte seinen Kopf zwei- oder dreimal mit aller
Kraft hin und her und renkte ihn ein. Jch bemerkte bald, wie


% \picinclude{./220-229/p_s229.jpg} 
Arbeit in Nordamerika unter Engländern und Jndianern. 229
sein Hale wieder anfing, Halt zu bekommen; darauf sing er an
zu röcheln und gleich darauf zu atmen. Die Leute waren ent-
setzt, aber ich hieß sie, guten Mutß zu sein, Glauben zu haben
und ihn inß Haus zu tragen. Sie taten eö und legten ihn neben
daß Feuer. Jch hieß sie, ihm etwas Warmeö zu trinken zu geben
und ihn zu Bett zu bringen. Nach einer Weile fing er an zu sprechen,
aber er wußte nicht, was mit ihm geschehen war. Am folgenden
Tage zogen wir weiter, und er mit uns-, ziemlich wohl, etwa
16 Meilen zu einer Versammlung nach Middletown durch Wälder
und Sümpse und über einen Fluß, wo wir unsere Pferde hinüber
schwimmen ließen und selber aus einem hohlen Vaumstanun hinüber
setzten. Er reiste noch viele hundert Meilen mit unß .....
Wir hatten hier eine herrliche Versammlung ..... Nach der-
selben gingen wir nach Middletown-Harbour, etwa siins Meilen weit,
um am nächsten Tage unsre große Reise anzutreten, durch die
Wälder nach Maryland. Unsre Führer waren Jndianer. Jch
beschloß, den Weg durch die Wälder auf der andern Seite der
Delawara-Bay zu nehmen, um die Flüsse und Buchten so viel wie
möglich zu vermeiden. Wir machten unö am 9. des 7. Monat?.
auf den Weg und kamen durch viele indianische Ortschaften und über
mehrere Flüsse und Sümpse; alö wir etwa 40 Meilen weit
geritten waren, machten wir ein Feuer für die Nacht und legten
unß daneben. Wenn wir zu Jndianern kamen, verkündeten
wir ihnen den Tag deß Herrn. Tagß daraus reisten wir 50 Meilen.
Nachtß fanden wir ein alteö Hauß, aus dem die Indianer die
Leute vertrieben hatten; wir machten ein Feuer und blieben dort
am Eingang der Delawara-Bay. Am folgenden Tage ließen wir
unsre Pferde etwa eine Meile weit über den Fluß schwimmen,
nach der Jnsel Upper-Dinidock, und darauf ausß Festland. Wir
selber hatten von den Jndianern ein Canoe gemietet und unö
darin von ihnen übersetzen lassen .....
Dann gingen wir nach Newcastle, jetzt Neu-Amsterdam
genannt; .... am 16. dez 7. Monats zogen wir weiter .... .
Nach einer beschwerlichen Reise erreichten wir daß; Haus Robert
Harwoodz in Mileö River in Maryland und wohnten am
folgenden Tage einer Versammlung bei. .... Von da gingö
nach dem Kentischen User, .... und dann zu Wasser, etwa zwanzig
Meilen weit, zu einer sehr großen Versammlung ..... E3 war
eine gesegnete Versammlung und von großem Nutzen sowohl zur


% \picinclude{./230-239/p_s230.jpg} 
Bekehrung zur Wahrheit als auch zur Befestigung solcher, die
schon bekehrt waren [...] Der Herr sei gelobt, der seine 
Wahrheit sich ausbreiten lässt! Nach der Versammlung kam eine Frau
zu mir; ihr Mann war einer der Friedensrichter der Gegend
und ein Mitglied der Behörde und sagte mir, ihr Mann
sei krank und werde wahrscheinlich sterben; ich solle doch mit ihr
heim kommen, um ihn zu besuchen. Sie wohnte drei Meilen weit
weg, und da ich gerade erhitzt aus der Versammlung kam, so war
es- hart für mich, mit ihr zu gehen; doch im Bewusstsein meines
Berufes nahm ich ein Pferd, ging mit ihr, besuchte ihren Mann
und redete, was der Herr mir eingab. Der Mann wurde sehr
erquickt und erholte sich gänzlich durch die Kraft des Herrn und
kam später zu unsern Versammlungen. Ich kehrte wieder am
gleichen Abend zu den Freunden zurück; und am folgenden Tage
zogen wir von dort weiter, etwa 20 Meilen nach Tredhaven
Creek,\ort{Tredhaven Creek} von wo wir am 3. des 8. Monats 
zur allgemeinen Versammlung aller Freunde von Maryland gingen. Dieselbe dauerte
5 Tage. Die drei ersten Versammlungen waren für öffentlichen
Gottesdienst,\index{Öffentliche Versammlung} zu welchen 
aller Arten Leute kamen; die beiden
andern für Männer- und Frauen-Versammlungen. Zu den öffentlichen 
Versammlungen kamen viele Protestanten\index{Protestanten} von verschiedenen
Richtungen, und einige Papisten;\index{Papisten} es waren mehrere Personen
von der Obrigkeit und ihre Frauen darunter und andere Angesehene 
der Gegend [...] 

Als wir unsern Dienst in Maryland verrichtet hatten, und
da wir die Absicht hatten, nach Virginia\ort{Virginia} zu gehen, hielten wir
eine Versammlung in Paturent am 4. des 9. Monats, um uns
von den Freunden zu verabschieden [...]

Am 5. schifften wir uns ein für Virginia und erreichten nach
drei Tagen Nanceum.\ort{Nanceum} Danach eilten wir 
nach Carolina\ort{Carolina}; doch
hatten wir unterwegs mehrere schöne Versammlungen [...] Am
21. des 9. Monats nach einem beschwerlichen Weg durch die
Wälder, Sümpfe, Moräste, erreichten wir 
Bonners Creek\ort{Bonners Creek}; dort
brachten wir die Nacht am Feuer zu; eine Frau gab uns eine
Matte, um darauf zu schlafen; das war das erste Haus in 
Carolina, das wir erreichten.
Hier ließen wir unsre ermüdeten Pferde und fuhren in einem
Kanu den Fluss hinunter, nach Hugh 
Smiths\person{Smith, Hugh} Haus, wo Leute
aller Richtungen uns besuchten, und viele von ihnen nahmen uns
% \picinclude{./230-239/p_s231.jpg} 
freundlich auf; Freunde gab es in der Gegend nicht. Nathaniel
Batts war darunter, der frühere Gouverneur von Roan Dak\index{Roan Dak}. Er
war bekannt als Hauptmann Batts\person{Hauptmann Batts} 
und war ein rauher, heftiger
Mann gewesen. Er fragte mich nach einer Frau in Cumberland,
die, wie er gehört habe, durch unsre Gebete und Handauflegen
geheilt worden sei, nachdem sie lange krank und von den Ärzten
aufgegeben worden war, und er wollte wissen, ob es wahr sei.
Ich sagte ihm, wir rühmen uns solcher Dinge nicht, aber es seien
viele solche Dinge geschehen in der Kraft Christi. An der 
Connie~Dak~Bay\ort{Connie~Dak~Bay} empfing uns der Gouverneur 
liebevoll; aber ein dortiger
Gelehrter wollte durchaus mit uns disputieren. Und sein Widerstand 
war uns sehr nützlich, da er uns Gelegenheit gab, die Leute
über manches aufzuklären, das Licht und den Geist Gottes betreffend;
er wollte nicht gelten lassen, das sie in einem jeden seien, und 
versicherte, sie seien nicht in den Indianern.\index{Rassismus} 
Hierauf rief ich einen
Indianer herbei und fragte ihn, ob nicht, wenn er lüge oder
jemandem Böses tue, etwas in ihm sei, das ihn dafür strafe? Er
sagte, es sei so etwas in ihm, das ihn darüber strafe, und er
schäme\index{Scharm} sich, wenn er Unrecht getan oder etwas Unrechtes gesagt
habe. Also beschämten wir den Gelehrten vor dem Gouverneur
und dem Volk weil der gute Mann so weit gegangen war, das er
nicht einmal die Schrift gelten\index{Bibel, Bedeutung} lies [...].

Von hier ging ich zu den Indianern und redete durch einen
Dolmetscher zu ihnen; ich zeigte ihnen, das Gott alle Dinge in
sechs Tagen gemacht habe\index{Schöpfungslehre} und nur eine Frau für einen Mann
gemacht habe,\index{Monogamie} und das Gott die alte Welt vernichtete wegen ihrer
Schlechtigkeit. Darauf sprach ich ihnen von Christus und zeigte
ihnen, das er für alle Menschen gestorben sei,\index{Opfertod} für ihre Sünden
so gut wie für die der andern, und das, wenn sie Böses täten, er sie
verbrennen werde, wenn sie aber Gutes tun, sie nicht verbrannt
würden.\index{Hölle} Ihr junger \zitat{König} war unter ihnen und andere ihrer
Häuptlinge und sie schienen, was ich ihnen sagte, gut aufzunehmen.

Nachdem wir die nördlichen Gegenden von Carolina\ort{Nord Carolina} durchreist
und der Wahrheit hier einigermasen den Weg bereitet hatten,
gingen wir in der Richtung von Virginia zurück. Wir hatten
unterwegs mehrere Versammlungen, so auch eine sehr segensreiche
bei Hugh Shmith. Gelobt sei der Herr ewiglich! Die Leute waren
sehr empfänglich und unser Wirken war gesegnet unter ihnen. Es
war ein indianischer Häuptling dabei, der sehr lieb war und
% \picinclude{./230-239/p_s232.jpg} 
bekannte, was wir sagten, sei Wahrheit. Auch ein Indianerpriester,\index{Indianerpriester} 
ein \zitat{Pawaw}, wie sie sagen, war da, der ruhig mit den
andern da saß. Am 9. des 10. Monats gingen wir nach Bonners
Creek zurück, wo wir unsre Pferde gelassen, nachdem wir etwa
18 Tage in Nord-Carolina gewesen waren.

Da unsre Pferde nun ausgeruht hatten, gingen wir weiter
nach Virginia durch Wälder und Sümpfe, so weit wir an diesem
Tage kommen konnten. Am folgenden Tage hatten wir eine
schwierige Reise durch Sumpf und Moor und waren den ganzen
Tag sehr nass und schmutzig; des Nachts trockneten wir uns dann
an einem Feuer. Am nächsten Abend erreichten wir Sommertown\ort{Sommertown}.
Als wir zur Herberge kamen, sah uns die Frau des Hauses und sagte
ihrem Sohn, er solle die Hunde festbinden; sie hatten nämlich
in Virginia und Carolina große Hunde, um ihre Häuser zu 
bewachen, da sie so einsam im Walde leben; aber der Sohn sagte,
es sei nicht nötig, denn die Hunde machten sich nicht an diese
Art Leute. Als wir nun in das Haus kamen, sagte er, wir
seien wie die Kinder Israel, gegen die die Hunde nicht mucksten
(2. Mos. 11,7)\bibel{Mos. 02. 11:07@2. Mos. 11:7} [...]

Wir waren drei Wochen unterwegs durch Virginia [...] Als
wir das Werk, das uns hier obgelegen, beendet hatten, fuhren
wir am 30. des 10. Monats in einer offenen Schaluppe ab, um
nach Maryland zurückzukehren [...] Da ein starker Sturm sich
erhob, waren wir froh, vor Nacht das Ufer zu erreichen und
übernachteten in einem Hause in Willougby Point\index{Willougby Point}. [...] 
Am Morgen kehrten wir zu unserm Boot zurück und segelten so rasch
als möglich weiter. Aber am Abend erhob sich abermals ein
Sturm, so das wir wieder ans Ufer mussten, wo wir die Nacht
an einem Feuer zubrachten, um uns zu trocknen, und um uns
herum heulten die Wölfe [...] Am 3. des 11. Monats
war der Wind ziemlich günstig und wir benützen ihn, um so
schnell wie möglich fort zu kommen; am Abend erreichten
wir Milfordhaven\ort{Milfordhaven}. Darauf passierten wir 
den Rappahannockflus\ort{Rappahannock (Fluss)}, den 
Patomac\ort{Patomac}, und dann fuhren wir in der Richtung
des Paturentflusses\ort{ Paturent (Fluss)} und erreichten in der Frühe des Morgens
James Prestons\person{Preston, James} Haus. Wir waren sehr müde, gingen aber
doch am nächsten Tage, einem Ersten Tage, zu einer 
Versammlung in der Nähe [...] Die Kälte wurde um so 
empfindlicher und der Frost und Schnee so heftig, das es fast unerträglich
% \picinclude{./230-239/p_s233.jpg} 
war. Auch war es gefährlich umherzureisen [...] Am 27. des
11. Monats hatten wir eine köstliche Versammlung in einem
Tabakshaus. Am darauffolgenden Tage kehrten wir wieder zu
James Preston zurück; als wir dort anlangten, war sein Haus
in der vorhergehenden Nacht abgebrannt, so das wir die Nächte
bei sehr kaltem Wetter im Freien am Feuer zubringen mussten.
Wir machten die merkwürdige Beobachtung, das sich eines Tages
mitten in diesem kalten Wetter der Wind gegen Süden drehte
und es ganz unerträglich heiß wurde [...] Am 2. des 12. Monats
hatten wir eine herrliche Versammlung in Paturent\ort{Paturent}. Am 12.
reisten wir weiter in unserm Boot. Den Anamessy und den
Amoroca\ort{Amoroca} umgehend, kamen wir nach Manaoke. Dann passierten
wir den Wicocomaco, wo wir eine gesegnete Versammlung hatten;
dann gings zu Pferde etwa 24 Meilen weit durch Sümpfe und
Wälder zum Hause eines Richters, wo wir ebenfalls eine 
lebendige Versammlung hatten. Zu dieser kam eine Frau aus 
Anamessy, die viele Jahre schwermütig\index{Schwermütigkeit} 
gewesen war und oft während
zwei Monaten nichts redete und sich um nichts kümmerte; als ich
von ihr hörte, trieb mich der Herr, zur ihr zu gehen, und ihr zu
verkünden, das ihrem Hause Heil widerfahren sei. Nachdem ich
Worte des Lebens zu ihr geredet hatte und den Herrn für sie
angefleht, kam sie zurecht und zog mit uns umher zu den 
Versammlungen und ist seither gesund, dem Herrn sei Lob [...]

Nun hatten wir unsre Arbeit in dieser Gegend getan und
verließen Anamessy\ort{Anamessy}. Wir gingen zu Wasser etwa 50 Meilen,
nach dem Hungerflus [...] Danach etwa 40 Meilen nach dem
kleinen Choptankflus [...] Ehe wir dann weiter zogen, hatten
wir eine herrliche Versammlung, zu der viele Leute kamen; unter
anderem auch mehrere von den Behörden der Stadt, mit ihren
Frauen. Von den Indianern kam einer, den sie ihren \zitat{Kaiser}
nannten, ein Indianer-\zitat{König} und sein Dolmetscher, die alle sehr
andächtig zuhörten und sehr lieb waren. Es war eine grundlegende 
Versammlung. Dies war am 23. des 1. Monats. Am
24. gingen wir 10 Meilen weit zu Wasser nach der Indianerstadt, 
wo jener \zitat{Kaiser} wohnte. Ich hatte ihn von meinem Kommen
benachrichtigt und gebeten, seine \zitat{Könige} und Räte zu versammeln.
Am Morgen kam der \zitat{Kaiser} selber und führte mich in die Stadt;
und sie waren alle beisammen mit ihren Dolmetschern und ihren
Leuten, und die alte \zitat{Kaiserin} war auch da. Sie waren sehr ernst
% \picinclude{./230-239/p_s234.jpg} 
und ruhig und sehr aufmerksam, mehr als manche, die sich Christen
nennen. Es waren einige mit mir, die verdolmetschen konnten
und wir hatten eine schöne Versammlung mit ihnen, die großen
Nutzen schaffte, denn sie brachte die Wahrheit und die Freunde
bei ihnen in Ansehen. Gelobt sei der Herr! [...]

Nachdem wir die meisten Gegenden dieses Landes durchzogen
hatten und die meisten der Plantagen besucht und überall, wo wir
hinkamen, Alarm geblasen\index{Alarm blasen} hatten und Gottes Tag des Heils
den Leuten verkündet hatten, spürten wir im Geist, das wir
bald unsere Aufgabe in diesen Weltgegenden erfüllt hätten, und
wandten uns wieder Alt-England zu. Doch es verlangte uns,
und der Herr schenkte uns die innere Freiheit dazu, bis zur
nahen Generalversammlung\index{Generalversammlung} 
für Maryland\ort{Maryland} zu bleiben, damit wir
alle Freunde beisammen sehen könnten, ehe wir abreisten. Die
Zwischenzeit brachten wir damit zu, Freunde und Gemeindemitglieder
zu besuchen, Versammlungen um die Cliffs und Paturent\ort{Paturent} herum
beizuwohnen, und Antworten zu schreiben auf allerlei Kritteleien,
welche etliche Gegner der Wahrheit erhoben und verbreitet hatten,\index{Verteidigungsschrift}
um die Leute zu verhindern, die Wahrheit anzunehmen; so waren
wir also nicht müßig, sondern wirkten für den Herrn, bis zur
allgemeinen Provinzialversammlung, die am 17. des 3. Monats
begann und vier Tage dauerte. Am Ersten Tage hatten die
Männer und Frauen ihre geschäftlichen Versammlungen, in welchen
die Angelegenheiten der Kirche besorgt wurden, und viele Dinge
wurden ihnen geoffenbart zu ihrer Erbauung und Trost. Die
drei übrigen Tage wurden mit öffentlichen 
Versammlungen\index{Öffentliche Versammlung} zugebracht, 
wobei verschiedene angesehene Regierungsbeamte anwesend
waren und viele andere; sie waren alle im allgemeinen befriedigt
und manchen ging es zu Herzen. Denn es war eine wundervolle,
herrliche Versammlung, und die mächtige Gegenwart des Herrn
ward überall fühlbar und gesehen; gelobt und gepriesen werde
sein heiliger Name immerdar, welcher Herrschaft gibt über alles.
[...] Nach dieser Versammlung nahmen wir Abschied von den
Freunden [...] Am nächsten Tage, dem 21. des 3. Monats
1673\index{Jahr!1673} schifften wir uns ein für England und am 28. des 4. Monats
landeten wir im Hafen von Bristol\ort{Bristol}. Dieselbe treue Hand der
Vorsehung, die uns geleitet und glücklich hinüber gebracht hatte,
wachte bei unserer Rückkehr über uns und brachte uns glücklich
zurück. Lob und Dank sei seinem heiligen Namen innnerdar!
% \picinclude{./230-239/p_s235.jpg} 

Wir hatten während der Reise Viele köstliche Versammlungen
an Bord des Schiffes gehabt, gewöhnlich zwei in der Woche, in
denen die gesegnete Gegenwart des Herrn uns mächtig erquickte,
und oft goss sie sich aus über uns und machte die Anwesenden
empfänglich.

%%%%%%%%%%%%%%%%%%% Kapitel 20. %%%%%%%%%%%%%%%%%%%%%%%%%%%%%%

\chapter[Penn, Frauenversammlungen und Grundsätze]{William Penn, Frauenversammlungen und  über die 
Grundsätze der Quäker.}

\begin{center}
\textbf{Ankunft in Bristol. Zusammentreffen mit William Penn und
anderen. Verteidigung der Frauenversammlungen. Vorgeahnte
Gefangenschaft in Worcester. Brief an den König über die 
Grundsätze der Quäker. Krankheit. Befreiung. Während der
Gefangenschaft verfasste Schriften.}
\end{center}



Als wir ans Ufer kamen, begaben wir uns nach Shirehampton\ort{Shirehampton} [...] 
und ritten von da nach Bristol\ort{Bristol} [...]. Am Abend
schrieb ich von hier einen Brief an meine Frau:

\brief{Fell, Margaret}{
Liebes Herz!

\bigskip 

Heute Abend landeten wir in Bristol, Gott dem Herrn sei
Lob immerdar, der unser Schutzherr war und unser Schiff lenkte,
dem Gott der ganzen Erde, der Meere und Winde, der die Wolken
zu seinen Wagen macht [...] Robert Widders\person{Widders, Robert} und James
Lancaster\person{Lancaster, James} sind mit mir und sind gesund. Dem Herrn sei Lob
und Ehre, der uns durch so manche Gefahr geholfen, auf dem
Wasser, in den Stürmen, vor Seeräubern und andern Räubern,
in den Gefahren der Wildnis und unter den falschen Frommen.
Sein Ruhm ist über alles, Amen. Darum werdet das neue
Leben inne, und lebet ganz dem Herrn in demselben. Ich möchte,
so der Herr will, eine Zeit lang hier bleiben, vielleicht bis zum
Jahrmarkt. Genug diesmal. Meine Liebe allen Freunden.

\bigskip 

\begin{flushright}Bristol, 28. des 4. Monats 1673\index{Jahr!1673}. G. F.\end{flushright}
}

Zwischen diesem Tage und dem Jahrmarkt kam meine Frau
aus dem Norden zu mir nach Bristol, und ihr Schwiegersohn,
Thomas Lower\person{Lower, Thomas} und zwei ihrer Töchter kamen mit ihr. Ihr
anderer Schwiegersohn, John Raus\person{Raus, John}, William Penn\person{Penn, William}\footnote{
William Penn, Sohn des großen Admiral Penn, zeigte schon während
seiner Studienzeit in Christchurch College\index{Christchurch College} 
in Oxford ernst religiöse Bedürfnisse.
Als er nach Hause zurückgekehrt, sich den herrschenden Hofsitten nicht fügen wollte,
sagte sein Vater sich von ihm los. Er schloss sich nun den Quäkern an und
wurde ihr bedeutendstes Mitglied. Aus einem ihm von Karl II. überlassenen
und nach ihm Pennsylwanien genannten Landstrich am Delaware\ort{Delaware} gründete er
eine Freistatt für seine verfolgten Glaubensgenossen. (Näheres siehe Weingarten
a.a.O.S. 408.)
} und seine
% \picinclude{./230-239/p_s236.jpg} 
Frau und Gerrard Roberts\person{Roberts, Gerrard} kamen von London; und noch viele
andere Freunde aus verschiedenen Gegenden des Landes kamen
zum Jahrmarkt\index{Jahrmarkt}, und wir hatten herrliche Versammlungen, in
denen des Herrn Kraft über allen war. Nachdem ich meine
Arbeit für den Herrn in dieser Stadt getan, ging ich nach 
Gloucestershire\ort{Gloucestershire}, [...] und von da nach Wiltshire. 
In Slattenford in Wiltshire\ort{Wiltshire} hatten wir eine schöne Versammlung, obgleich wir
manchen Widerstand erfuhren von solchen, die sich den 
Frauenversammlungen\index{Gleichberechtigung} widersetzten; der Herr 
trieb mich, dieselben den
Freunden anzuempfehlen\index{Konflikt} zu Nutz und Frommen der Kirche Christi:
\zitat{Gläubige Frauen, welche zum Glauben an die Wahrheit berufen 
sind und desselben köstlichen Glaubens teilhaftig gemacht
sind, wie die Männer, und Miterben desselben ewigen Evangeliums 
des Lebens und Heils, sollen gleicherweise in den Besitz
und Stand der Ordnungen des Evangeliums kommen und somit
Mitgehilfen der Männer werden bei den Neugestaltungen im
Dienst der Wahrheit in den Angelegenheiten der Kirche, wie sie
es in den zeitlichen Dingen des täglichen Lebens sind, damit
die ganze Hausgemeinde Gottes, sowohl, Männer wie Frauen,
ihre Pflicht und Aufgabe im Hausstand Gottes kennen, einsehen
und ausüben möchten, damit besser für die Armen gesorgt werde,
die Jungen unterrichtet und in Gottes Wegen unterwiesen werden,
die Wankelmütigen und Liederlichen zurechtgewiesen und getadelt
werden in der Furcht Gottes, die Unbescholtenheit solcher, die sich
verheiraten wollen, genauer und strenger untersucht werde in der
Weisheit Gottes, und alle Glieder des geistigen Leibes, der Kirche,
einander bewachen und helfen in der Liebe.} Aber nachdem diese
Gegner sich lange in Hader und Zank ereifert hatten, warf der
Herr einen ihrer Führer darnieder, so das er sich demütigte und
das Unrecht einsah, das er tat, als er sich Gottes himmlischer
Macht widersetzte, er gestand den Freunden seinen Irrtum ein
und veröffentlichte später ein Schreiben, in welchem er erklärte,
er hätte sich eigensinnig widersetzt, trotz meiner häufigen 
Warnungen, bis das Feuer des Herrn in ihm entbrannt sei, und er
den Engel des Herrn mit gezogenem Schwert gesehen habe, im
% \picinclude{./230-239/p_s237.jpg} 
Begriff ihn zu töten [...]. Wir besuchten viele Freunde und
kamen schließlich nach Kingston\ort{Kingston}, wo ich mit 
meiner Frau\person{Fell, Margaret} und
ihrer Tochter Rachel\person{Fell, Rachel} zusammen traf. 
Ich hielt mich nicht lange
hier auf, sondern ging nach London, wo die Baptisten\index{Baptisten} mit einigen
Sozinianern\index{Sozinianern} sehr bösartig geworden waren und Viele Bücher gegen
uns gedruckt hatten, und ich hatte viele Arbeit in der Kraft des
Herrn, ehe ich aus dieser Stadt loskommen konnte [...].\index{Schmähschriften}

Darauf machte ich eine kurze Reise durch einige Gegenden
in Essex\ort{Essex} und kehrte nach London\ort{London} 
zurück, wohin ich mich innerlich
gezogen fühlte, denn ich hatte gehört, das man viele Freunde
vor die Richter gebracht hatte und etliche in London und an andern
Orten gefangen genommen hatte, weil sie ihre Schaufenster an
Festtagen\index{Festtagen} und an sogenannten 
Fasttagen\index{Fasttagen} geöffnet hatten, und
weil sie Zeugnis ablegten gegen alles Feiern solcher Tage. Die
Freunde mussten dies tun, weil sie ja wussten, das die wahren
Christen die Feste der Juden zur Zeit der Apostel nicht hielten,
und so konnten sie auch nicht die sogenannten Feste der Heiden
und Papisten\index{Papisten}, welche unter den sogenannten Christen seit den
Tagen der Apostel eingesetzt wurden, halten. Denn wir sind 
befreit worden vom Halten bestimmter Tage, durch Jesum Christum
und sind in den Tag gebracht worden, der aus der Höhe aufgegangen 
ist, und wir sind jetzt in Ihm, der Herr ist über den
jüdischen Sabbath\index{Jüdischen Sabbath}, und über das Wesen 
der jüdischen Zeichen [...

Bald darauf, als ich, in Adderbury\ort{Adderbury}, am Nachtessen saß, fühlte
ich, das ich gefangen genommen werden würde, ich sagte aber noch
niemand etwas [...]. Am andern Tage hatten wir eine Versammlung 
[...] Nach derselben saß ich mit einigen Freunden in einem
Zimmer im Gespräch, da kam Henry Parker\person{Parker, Henry}, 
ein Friedensrichter, ins Haus, und mit ihm Rowland 
Hains\person{Hains, Rowland}, ein Priester aus Hunniton
in Warwickshire\index{Warwickshire}, [...] und Henry Parker nahm mich gefangen und
Thomas Lower\person{Lower, Thomas} zur Gesellschaft 
mit, und obgleich er nichts gegen
uns Vorbringen konnte, schickte er uns beide ins Gefängnis von
Woreester\ort{Woreester}, durch einen merkwürdigen Verhaftbefehl [...]
Da ich derart zum Gefangenen gemacht worden war, ohne viel
Aussicht, vor der vierteljährlichen Gerichtssitzung frei zu werden,
veranlassten wir einige Freunde, meine Frau\person{Fell, Margaret} und ihre Töchter
nach dem Norden zu begleiten [...] Als ich dachte, das meine
Frau zu Hause angelangt sei, schrieb ich ihr aus der Gefangenschaft 
folgenden Brief:
% \picinclude{./230-239/p_s238.jpg} 


\brief{Fell, Margaret}{
  Liebes Herz!

  \bigskip 

  Du schienst ein wenig betrübt, als ich vom Gefängnis sprach
  und dann gefangen genommen wurde. Ergib dich in den Willen
  des Herrn. Denn als ich im Hause des John Rous in Kingston
  war, hatte ich ein Gesicht, das ich gefangen genommen würde,
  und als ich bei Bray Doily in Oxfordshire war, sah ich, während
  ich am Abendessen saß, das ich gefangen genommen und Leiden
  zu erdulden haben würde. Aber des Herrn Macht ist über allem, sein
  Name sei gepriesen ewiglich [...]

\bigskip 
\begin{flushright}G. F.\end{flushright}

}

Wir wurden erst am letzten Tage der Gerichtssitzung 
vorgenommen, am 21. des 11. Monats 1673\index{Jahr!1673} [...] Ich musste
Bericht über meine Reise geben. Friedensrichter Parker hatte,
um den Fall recht schwer scheinen zu lassen, eine große Geschichte
gemacht, es seien Leute von London, von Cornwall, aus dem
Norden und von Bristol im Hause gewesen, als man mich gefangen 
genommen habe; hieraus erklärte ich ihm, das das alles
gewissermaßen eine einzige Familie gewesen 
sei,\footnote{Der Konventikel-Akt\index{Konventikel-Akt} von 1664\index{Jahr!1664} sagt: 
\zitat{jede Prioatandacht von mehr
als fünf Personen außer der Familie, wobei nicht das Common-Prayer-Boot
zugrunde gelegt wird, wird mit dreimonalichem Gefängnis, zum dritten mal
mit Verbannung bestraft.}} indem niemand von
London dagewesen sei, als ich selber, niemand aus dem Norden als
meine Frau und meine Töchter, niemand von Cornwall als mein
Schwiegersohn.

Als ich fertig geredet, stand der Vorsitzende, Simpson, ein
einstiger Presbyterianer, auf und sagte: \zitat{Was Ihr da sagt, klingt
recht unschuldig}. Darauf flüsterten er und Parker eine Weile
miteinander und daraus stand er wieder auf und sagte: \zitat{M. Fox,
Ihr seid ein ausgezeichneter Mann und alles, was Ihr da sagt,
mag wahr sein, aber, um uns ganz zu befriedigen, wollt Ihr den
Huldigungseid leisten?}\index{Eid} Ich erwiderte ihm, sie hätten versprochen,
uns keine Falle zu stellen, dies sei aber einfach eine Falle, da sie
ja wissen, das wir nicht schwören können. Sie ließen dennoch
den Eid vorlesen. Ich erklärte ihnen daraus: \zitat{Ich habe nie in
meinem Leben einen Eid geleistet, aber ich bin immer der Regierung
gehorsam gewesen; ich war im Kerker von Derby sechs Monate,
weil ich die Waffen nicht gegen König Karl bei Woreester 
erheben wollte, und weil ich in die Versammlungen ging, wurde
ich nach Leicestershire gebracht, vor Oliver 
Cromwell\person{Cromwell, Oliver}, als einer der
% \picinclude{./230-239/p_s239.jpg} 
Mitverschworenen für die Rückkehr des Königs. Ihr wisst es ja
nach euren eigenen Gewissen, das wir, die ihr Quäker nennt
keinen Eid leisten können, weil Christus es verboten hat. Was
aber den Inhalt des Eides anbelangt, so kann ich sagen, und
sage es auch, das ich den König von England als den rechtmäßigen 
Erben und Nachfolger des englischen Reiches anerkenne\index{Monarchie}
und alle Verschwörer und Verschwörungen gegen ihn verabscheue,
und ich hege nur Liebe und Wohlwollen in meinem Herzen gegen
ihn und gegen alle Menschen und wünsche ihm und ihnen allen
nichts als Gutes. Der Herr weiß es, vor welchem ich als ein
unschuldiger Mann stehe. Was den Suprematseid anbelangt, so
verabscheue ich den Papst\index{Papst} und seine Macht und 
seine Religion\index{Katholizismus}
von ganzem Herzen.} [...] Aber sie ließen mich vom 
Gefangenwärter hinweg führen und brachen so ihr Versprechen dem Lande
gegenüber, denn sie hatten mir freies Reden zugestanden und mir
es doch nicht gewährt [...]

Während meiner Gefangenschaft kam es über mich, unsere
Anschauungen und Grundsätze dem König darzutun, nicht 
hauptsächlich um meiner eigenen Leiden willen, sondern damit er unsere
Anschauungen und unsere Gemeinschaft besser verstehe.

\brief{König}{
  An den König!

  \bigskip 

  Der Ausgangspunkt der Quäker ist der Geist von Christus,
  der für uns gestorben und um unsrer Gerechtigkeit willen auferweckt
  worden ist, durch welchen wir wissen, das wir sein sind. Er
  wohnet in uns mit seinem Geist und der Geist Christi macht uns
  frei von aller Ungerechtigkeit und Gottlosigkeit. Der Geist Christi
  machet, das wir allem gottlosen Wesen absagen, als da sind
  Lügen, Stehlen, Töten, Verbrechen, Hurerei und alle Arten von
  Unreinigkeit, Unzucht, Bosheit, Hass, Betrügerei, Schlemmen und
  alle Werke des Teufels.\index{Sündlosigkeit} Der Geist Christi führt uns dazu, für
  alle Menschen den Frieden und das Gute zu suchen und friedlich
  zu leben. Er machet, das wir uns aller Anschläge und 
  Verschwörungen gegen den König oder irgend sonst Jemand 
  enthalten. Er hält uns zurück von jenem bösen Tun und Treiben,
  gegen welches das Schwert der Obrigkeit sich richtet. Unser Wunsch
  und Bestreben ist, das alle, die Christus bekennen, auch im Geiste
  Christi wandeln, damit sie durch denselben des Fleisches Geschäfte
  töten möchten und mit dem Schwert des Geistes ihre Sünde und
  Bosheit ausrotten. Dann würden die Richter und Beamten nicht
  % \picinclude{./240-249/p_s240.jpg} 
  soviel damit zu tun haben, das Böse im Reich zu bestrafen, und
  die Könige und Fürsten brauchten keinen ihrer Untertanen zu
  fürchten, wenn alle im Geist Christi wandelten; denn die Früchte
  des Geistes sind Liebe, Gerechtigkeit, Gütigkeit und Mäßigkeit.
  Wenn alle, die sich als Anhänger Christi bekennen, auch in feinem
  Geiste wandeln und durch denselben Sünde und Bosheit in sich
  ertöten würden, so wäre dies eine große Erleichterung für die
  Obrigkeit und würde ihr viel Mühe ersparen, denn dann würden
  alle dazu geführt, andern zu tun, wie sie wollten, das man ihnen
  tue, und das Königliche Gesetz der Freiheit würde somit erfüllt [...]
  
  Wir können aus großer Gewissenhaftigkeit gegen die Gebote
  Christi und seiner Apostel nicht schwören, denn es wird uns 
  geboten, in Matth. 5\bibel{Matth. 5} und Jak. 5\bibel{Jak. 5} 
  bei ja und nein zu bleiben und überhaupt nicht zu schwören, 
  weder beim Himmel, noch bei der Erde,
  noch bei irgend sonst etwas, auf das wir nicht übles tun und in
  Verdammnis fallen. Christus sagte: \zitat{Ihr habt gehört, das zu
  den Alten gesagt ist, ihr sollt keinen falschen Eid tun und Gott
  euren Eid halten} (Math. 5). Es waren dies wahre und feierliche 
  Eide, und die, welche sie einst leisteten, hatten sie zu halten,
  aber Christus und die Apostel verbieten sie zur Zeit des 
  Evangeliums so gut wie die falschen und unnützen Eide. Wenn wir irgend
  einen Eid leisten könnten, so wäre es der Huldigungseid, weil
  wir wissen, das König Karl durch Gottes Macht nach England
  zurückkam und zum König von England gemacht und über unsere
  früheren Verfolger gesetzt wurde, und was die Oberherrschaft des
  Papstes anbelangt, so erkennen wir sie in keiner Weise an.\index{Papstum} Aber
  da Christus und seine Apostel uns geboten, nicht zu schwören,
  sondern bei ja und nein zu bleiben, so dürfen wir ihren Geboten
  nicht ungehorsam sein. Darum haben viele uns den Eid vorglegt 
  als Falle, damit wir ihnen zur Beute würden. Unsre
  Weigerung des Eides geschieht nicht aus Eigensinn und 
  Hartnäckigkeit oder Missachtung, sondern nur aus Gehorsam gegen
  die Gebote Christi und der Apostel, und wir sind bereit, wenn wir
  unser ja und nein brechen, die gleiche Strafe zu leiden, wie jemand
  der seinen Eid bricht. Wir bitten darum den König, solches zu
  bedenken, und wie lange wir schon leiden um dieser Sache willen.
  Dies ist von einem, der dem König allezeit Glück und alles
  Wohlergehen wünscht und allen seinen Untertanen, durch Jesus
  Christus.
  \bigskip 
  \begin{flushright}G. F.\end{flushright}

}

% \picinclude{./240-249/p_s240.jpg} 
% \picinclude{./240-249/p_s241.jpg} 
% \picinclude{./240-249/p_s242.jpg} 
% \picinclude{./240-249/p_s243.jpg} 
% \picinclude{./240-249/p_s244.jpg} 
% \picinclude{./240-249/p_s245.jpg} 
% \picinclude{./240-249/p_s246.jpg} 
% \picinclude{./240-249/p_s247.jpg} 
% \picinclude{./240-249/p_s248.jpg} 
% \picinclude{./240-249/p_s249.jpg} 

\picinclude{./250-259/p_s250.jpg} 
\picinclude{./250-259/p_s251.jpg} 
\picinclude{./250-259/p_s252.jpg} 
\picinclude{./250-259/p_s253.jpg} 
\picinclude{./250-259/p_s254.jpg} 
\picinclude{./250-259/p_s255.jpg} 
\picinclude{./250-259/p_s256.jpg} 
\picinclude{./250-259/p_s257.jpg} 
\picinclude{./250-259/p_s258.jpg} 
\picinclude{./250-259/p_s259.jpg} 

% \picinclude{./260-269/p_s260.jpg} 
260 Kapitel 1311.
Jch hörte später, daß seine Hörer ihn wegen dessen, was er
in unsrer Versammlung gesagt, zu Rede stellten, und als er dazu
stand, ihn bei den andern Priestern der Stadt oerklagten, die ihn
darüber zur Rechenschaft zogen, aber vom Ausgang der Sache
konnte ich nichts erfahren .....
Am folgenden Tage gingen wir nach Amsterdam, wo wir
etwas nach Mitternacht ankamen, und da die Tore geschlossen
waren, so blieben wir bis zum folgenden Tage auf dem Schiff
und gingen dann ins Haus oon Gertrud Ditick; hier besuchten
uns viele Freunde, froh, daß wir wohlbehalten wieder zurück
waren. Am folgenden Tage fühlte ich mich im Geist beunruhigt
wegen etlicher verführerischer Geister, die Uneinigkeit tmter die
Freunde brachten, und weil ich merkte, daß sie sich suchten in Grmst
zu bringen, so trieb es mich, einige Zeilen deswegen an die
Freunde zu schreiben: ,,Alle, die sich in die Gunst der Leute ein-
schmeicheln wollen, trachten, sich in Gunst zu bringen statt Christus.
Aber Freunde, euer friedsames Bleiben in der Wahrheit, die ewig
ist und sich nicht verändert, wird alles, was nicht aus der Wahr-
heit ist, überdauern, auch wenn es mit noch so viel Worten aus-
tritt. Lasset denn die, welche so für J. S. und J. W. auftreten
zu ihnen halten und sich von euch trennen, und ihr, die ihr Zeug-
nis abgelegt gegen diesen Geist, beharret bei diesem Zeugnis, bis
sie euch mit Anschuldigungen angreifen. Zanket nicht, lasset euch
nicht ein mit etwas, das nicht in der Wahrheit steht, noch suchet
lebendig zu erhalten, was sollte Gott zum Opfer gebracht werden,
damit ihr nicht des Reiches oerlustig geht«.
Amsterdam, 14. des 7. Monats 1677. G. F.
An einem großen Fasttage wohnte ich einer der Versamm—
lungen der Freunde bei. Jch hatte eigentlich vorgehabt, nach
Haarlem zu gehen, aber ich wurde in meinem Geist gehalten, zu
bleiben. Wir hatten eine sehr große Versammlung, eine große
Menge Leute strömte herbei, worunter viele angesehene Personen.
Die Kraft des Herrn war über der Versammlung, und in den
Offenbarungen, die ich während derselben hatte, trieb es mich,
darzutun, daß niemand, mit allem Studieren und allem Verstand
oder mit dem Lesen der Geschichte, wenn er sie nach seinem eigenen
Willen lese, die Abstammung von Christus wisse, der nicht nach
dem Willen eines Menschen, sondern nach dem Willen Gottes
gezeugetssei. . . . Nachdem ich ihnen das ausführlich erklärt hatte,


% \picinclude{./260-269/p_s261.jpg} 
Reise nach Holland. Einrichtung der kirchlichen Ordnung usnz. 261
erklärte ich ihnen den Unterschied zwischen wahrem und falschem
Fasten. Jch zeigte ihnen, wie alle, ob sie sich nun Christen, Juden
oder Türken nennen, nicht in der rechten Weise fasten, sondern sie
fasten, »daß sie hadern und zanken und mit gottloser Faust schlagen«
(Jes. 58,4), sie erheben nicht reine Hände zu Gott. Und wenn sie
schon vor den Leuten tun, als ob sie fasteten, und ,,deS Tagß den
Kopf hängen wie ein Schilf, so ist es doch nicht das Fasten, das
Gott erwählt« (Jes. 58,5). Darum sind ihre Gebeine oertrocknet,
und wenn sie den Herrn anrufen, so hört er sie nicht und »ihre
Besserung wächst nicht« (Jes. 58,8), weil sie ihr eigeneö Fasten
halten und nicht daß des Herrn. Jch ermahnte sie, daß Fasten
deß Herrn zu halten, welchetz ein Fasten von der Ungerechtigkeit
und der Sünde sei, vom Streiten, Hadern und Unterdrücken, und
auch allen bösen Schein zu meiden. Die Leute, die Fasttage
hielten, wunderten sich sehr über diese Eröffnungen, und die Ver-
sammlung nahm ein schöne-Z friedlicheß Ende.
Am folgenden Tage ging ich nach Haarlem, wo ich zuvor eine
Versammlung angesagt hatte. Peter Hendricke und Gertrud
Dirick Nieson gingen mit mir und wir hatten eine gesegnete Ver-
sammlung. ES waren verschiedene ,,Fromme« dabei, auch ein
Priester der Lutheraner, der mehrere Stunden andächtig zuhörte,
während ich ihnen die Wahrheit verkündete, Gertrud verdol-
metschte. Als die Versammlung zu Ende war, sagte der Priester,
er habe nichts darin gehört, daß nicht nach dem Worte Gottes
gewesen wäre, und er wünschte unß, daß der Segen Gotteß mit
unß und unsren Versammlungen sein möge. Auch andere erklärten,
man habe ihnen noch nie zuoor die Dinge so verständlich auß-
einandergesetzt.
Wir brachten die Nacht im Hause eineö Freundes, Dirirk
Klassen, zu, am folgenden Tage kehrten wir nach Amsterdam zu
Gertrud Dirick zurück; wir waren noch nicht lange da, alß ein
berühmter Priester kam, der früher unter dem deutschen Kaiser
gestanden, mits einem anderen deutschen Priester, um mit mir zu
reden..««Jch ergriff die Gelegenheit, um ihnen den Weg der
Wahrheit zu erklären, indem ich ihnen zeigte, wie sie dazu kommen
können, Gott und Christuß und sein Goangelium und Gesetz zu
kennen; ich zeigte ihnen, daß sie niemals durch Studieren und
durch Philosophie dazu kommen skönnen, sondern durch göttliche
Offenbarung, durch den Geist Gotteö, der ihnen in der Stille sdeß-


% \picinclude{./260-269/p_s262.jpg} 
262 Kapitel 25111.
Herzenß kund werde. Die Beiden waren empfänglich und gingen
befriedigt fort.
Am folgenden Ersten Tage war ich in einer Versammlung
der Freunde in Amsterdam; außer vielen Verschiedenen ,,Frommen«
war auch ein Doktor auß Polen anwesend, der um seiner Religion
willen auS seiner Heimat verbannt war; während der Versamm-
lung wurde er ergriffen vom Zeugniß der Wahrheit und kam
nachher zu mir und wünschte eine Unterredung mit mir, und
nachdem wir eine Zeitlang miteinander geredet, und ich ihm die
Dinge noch mehr erklärt hatte, ging er sehr empfünglich und in
Liebe zur Wahrheit fort.
Während ich in Amsterdam mar, brachte ich viel Zeit mit
Schreiben für die Wahrheit zu. Jch schrieb von hier mehrere Briefe
an die Freunde in England, ebenso: »Eine Warnung an die
Bewohner der Stadt Oldenburg«, die kürzlich abgebrannt war,
ferner: ,,Eine Warnung an die Bewohner der Stadt Hamburg« . . .
Ferner schrieb ich einen Brief an die Gesandten, die zu Nym-
wegen über den Frieden oerhandelten .....
Jch schrieb auch an die Behörden und Priester von Emden,
um ihnen zu zeigen, wie unchristlich etz sei, die Freunde zu ver-
folgen. Mehrere andere Bücher schrieb ich, Antworten an Priester
und andere, in Hamburg, Danzig und anderwärtß, um die Freunde
und die Wahrheit von allen Beschuldigungen und Verleumdungen
zu reinigen ..... Ferner ,,Ein Brief über daß wahre Fasten,
daß wahre Beten, und die wahre Ehre, gegen die Verfolgungen
und für die wahre Freiheit in Christuß Jesu?-, damit ihr in eurem
Halten von Tagen, Monaten, Zeiten und Festen Sorge tragen
möget, daß der Apostel nicht umsonst an euch gearbeitet habe,
und ihr nicht von neuem »den dürftigen Satzungen dienet«
(Gal. 4, 9) und sie anderen auferlegt.«
,,Wo haben je Christus oder seine Apostel den Gläubigen oder
den Christen befohlen, Feste oder Tage zu halten? Zeiget unö,
wo in den Schriften dez Neuen Testamentß, in den vier Evan-
gelien, in den Briefen oder in der Offenbarung geschrieben steht,
daß Christuß oder die Apostel je befahlen, die Zeit, die man
Ch:-istfest nennt, zu feiern, oder den Tag von der Geburt Christi,
oder die Zeit, die man Ostern nennt, oder den weißen Sonntag,
oder Petruö, Pauluß, Lukatz oder Markus, oder irgend eines-
andern Heiligen Tag? . . .


% \picinclude{./260-269/p_s263.jpg} 
Reise nach Holland. Einrichtung der kirchlichen Ordnung usw. 263
GS war des Apostelö Arbeit, sie auö den Banden dieser
Satzungen zu befreien. Und alß sie sich dann dem Halten der
Tage wieder zuwandten, fürchtete er, er habe umsonst an ihnen
gearbeitet; und er ermahnte sie, Gal. 5, 1: ,,So bestehet nun
in der Freiheit, damit euch Christus befreit hat, und fallet
nicht wieder in daß knechtische Joch der Siinde.« Hiemit sagt
er, daß sie einst in dem knechtischen Joch gefangen waren. Aber
ach, wie sehr die sogenannten Christen seit den Tagen der Apostel
wiederum in dieseö Joch gekommen sind, indem sie wieder Fasten
und Tage hielten, daß sieht man an ihrem Tun. Ja, zwingen
nicht sogar sowohl die Papisten wie die Protestanten die Leute,
Tage, Monden und Jahreszeiten zu halten? . . . ET- war und
ist Christus, der die Menschen von diesen dürftigen Satzungen frei
macht, darum sollen die Erlöften fest stehen in der Freiheit, womit
Christus sie befreit hat .... Die so in diesen Satzungen stehen, und
andere dazu zwingen wollen, sind abgewichen von der Grkenntniß
Gottetz und stehen nicht sest in der Freiheit, mit der C-hristuS befreiet.
Was- daß Beten anbelangt, so sehen wir nirgends, daß Christuö
oder die Apostel je jemanden zwangen, mit ihnen zu beten oder
zu fasten. Sondern Christus zeigte, wie man beten solle und sich
von den Heuchlern unterscheiden. Seine Worte sind: ,,Wenn du
beteft, so sollst du nicht sein wie die Heuchler, die da gerne stehen
in den Schulen und an den Ecken aus den Gassen, auf daß sie i
von den Leuten gesehen werden ..... Wenn du betest, so gehe
in dein Kämmerlein und schließe die Türe zu und bete zu deinem
Vater im Verborgenen, und dein Vater, der ins Verborgene sieht,
wird dirö oergelien öfsentlichss (Matth. 6, 5) ..... Und wir
tun nun, wie ez die Apostel und Heiligen getan. Wir beten im
Verborgenen und öffentlich, je nachdem der Geist es uns ein-
gibt, welcher unsrer Schwachheit hilft, wie er den Aposteln und
allen wahren Christen half; so beten wir für unö und für alle
Menschen, hoch und niedrig. .... über daö Fasten sagt
Christus: »Wenn ihr fastet, so sollt ihr nicht sauer sehen wie die
Heuchler, sie verstellen ihr Angesicht, auf daß sie vor den Leuten
scheinen mit ihrem Fasten. Wenn du saftest, so salbe dein Haupt
und wasche dein Angesicht, und dein Vater, der ins Verborgene
sieht, wird dirs vergelten öffentlich-- (Matth. 6). Jn Jesaia 58
heißt es: ,,Rufe laut und schone nicht, erhebe deine Stimme wie
eine Posaune und oerkiindige meinem Volk ihr lildertreten und


% \picinclude{./260-269/p_s264.jpg} 
264 Kapitel 12111.
dem Hause Jakobs ihre Sünde; sie suchen mich täglich und wollen
meine Wege wissen, als- ein Volk, das Gerechtigkeit schon getan,
und das Recht ihres Gottes- nicht verlassen hätte; sie fordern
mich zum Recht und wollen mit ihrem Gott rechten. Warum
fasten wir und du siehest es nicht an? warum tun wir unserm
Leib wehe und du willst es nicht wissen? .... Siehe, ihr fastet,
daß ihr hadert und zanket und schlaget mit der—Faust ungöttlich.
Fastet nicht also, wie ihr jetzt tut, daß ein Geschrei von euch
in der Höhe gehört wird. Sollte das ein Fasten sein, das- ich
erwählen soll, .... wollt ihr das ein Fasten nennen und einen
Tag dem Herrn angenehm? Das ist ein Fasten, das ich erwähle:
»Laß los-, welche du mit Unrecht gebunden hast; laß ledig, welche
du beschwerest; gib frei, welchH du drängest; reiß weg allerlei
Last.« .... Das- Fasten also, das der Herr verlangt, ist nicht,
daß man Lasten auserlege und die Banden der Sünde noch be-
sestige, sondern solche Bande zu lösen und zu sprengen.
Und nun darüber, daß wir den Hut nicht abnehmen vor den
Leuten. Viele, die sich Christen nennen, haben Anstoß an uns-
genommen, weil wir den Hut nicht abnahmen und uns- nicht vor
ihnen Verneigten. Wir finden nirgends, daß Christus das- geboten
hat, sondern eher das Gegenteil. Christus sagt: »nehmet nicht
Ehre von den Menschen;« und ferner sagt Christus: ,,wie könnet
ihr glauben, gdie ihr Ehre von einander nehtnet, und die Ehre
die von Gott kommt, sucht ihr nicht« (Joh. 5). Christus nennt
es ein Kennzeichen der Ungläubigen, Ehre voneinander zu nehmen
und die Ehre, die von Gott kommt, nicht zu suchen, und ist denn
nicht das Mnehmen des- Hutes- und das Vet-neigen eine Ehre, die
sich die Menschen untereinander erzeigen, nach welcher sie trachten
und beleidigt sind, wenn sie ihnen nicht erzeigt wird? Haben sie .
nicht sogar etliche gebüßt, versolgt und gefangen genommen, weil
sie den Hut nicht abnahmen? Ja, verhöhnen nicht die Türken
die Christen in ihrem Sprichwort, welches sagt, die Christen
bringen einen großen Teil ihrer Zeit damit zu, ihre Hüte abzu-
nehmen und einander ihre kahlen Köpfe zu zeigen? Sollten nun
die, welche den edlen Namen Christen tragen diirsen, nicht über
den Türken stehen und über dem Trachten nach Menschenehre
und dem Verfolgen solcher, die ihnen diese Ehre nicht erweisen
wollen, wie überhaupt alle wahren, gläubigen Christen allein die
Ehre suchen sollten, die von Gott kommt? ES heißt: ,,Wer an


% \picinclude{./260-269/p_s265.jpg} 
Reise nach Holland. Einrichtung der kirchlichen Ordnung usw. 265
den Sohn Gottes glaubt, der hat daß ewige Leben; wer aber
nicht glaubt, der wird verdammt werden'' (Joh. 3, 36). Jst
nicht die Redentzart der Türken, daß die Christen so viel Zeit
darauf verwenden, ihre Hüte abzuziehen und einander ihre kahlen
Köpfe zu zeigen, ein Vorwurf für die Christen? Habt ihr nicht
viele gefangen genommen und bestraft, weil sie den Hut nicht vor
euch abnehmen wollten und euch ihre kahlen Köpfe zeigen? Ja,
in vielen eurer Städte und Staaten haben solche, die ihre Hüte
nicht abnehmen und ihre kahlen Köpfe nicht zeigen, weder Freiheit
noch Recht, obgleich sie treue Untertanen sind. Habt ihr nicht
ein Gesetz gegen sie erlassen, daß sie zwei Gulden bezahlen müssen,
wenn sie es nicht tun? Und trachtet ihr nicht, sie dazu zu zwingen
und bestraft sie, wenn sie es- nicht tun, wie in Lan?-meer in
Waterland? Jst denn das nicht trachten nach Menschenehre?
Taten nicht die Pharisäer und Juden also? ....
Jhr habet keinerlei Befehl von Ehristuö oder einem seiner
Apostel, irgend jemanden zu verfolgen, zu bestrafen oder gefangen
zu nehmen um seiner Religion willen.«
Harlingen in Frießland, 11.deS 6.MonatZ 1677. G. F.
Bald daraus kamen William Penn und George Keith von
Deutschland nach Amsterdam zurück und hatten einen Diöput mit
Galenus Abrahamö, einem der bekanntesten Baptisten in Holland.
Viele »Fromme« waren zugegen; da sie nicht Zeit hatten, den
Diöput zu beendigen, kamen sie am folgenden Tage noch einmal
zusammen, und da wurde der Baptist gänzlich geschlagen, und die
Wahrheit gewann Boden ..... Als wir nun unsern Dienst in
Amsterdam getan, gingen wir, zu Wagen, nach Leyden. Wir
kamen dort mit einem Deutschen zusammen, der teilweise bekehrt
wurde. Er sagte uns von einem hervorragenden Mann, der die
Wahrheit suche. Etliche fanden ihn auf und besuchten ihn und
sanden einen ernst gesinnten Mann in ihm. Jch redete auch mit
ihm, und er bekannte sich zur Wahrheit. William Penn und Benjamin
Furly besuchten noch einen andern angesehenen Mann, der ein wenig
außerhalb von Leyden wohnte, von dem es hieß, er sei General
beim König von Dänemark gewesen. Gr und seine Frau waren
sehr liebevoll mit ihnen und nahmen die Wahrheit mit Freuden auf.
Von Leyden gingen wir nach dem Haag, wo der Prinz von
Oranien seinen Hof hielt, und wir besuchten einen von der Regierung
von Holland, mit dem wir eine ziemlich lange Unterredung hatten.


% \picinclude{./260-269/p_s266.jpg} 
266 Kapitel IIUI.
. . .. Von da gingen wir über Delft nach Rotterdam, wo wir
einige Tage blieben und mehrere Versammlungen hatten. Hier
verfaßte ich auch ein Buch an die Juden, mit denen ich gerne,
alö ich in Amsterdam war, mich unterredet hätte, aber sie wollten
nicht. Jch erhielt hier auch einige andere Bücher und Schristen,
die ich früher herauzigegeben und die nun übersetzt waren.
Kapitel Ickclll.
Rückkehr nach England. Kampf der Orduungöpartei gegen die
nnbotmsißigen Quiiker. Briefe über Toleranz an den König von
Polen, den Großmogul und andere.
Da wir spürten, daß wiryunser Werk in Holland getan
hatten, nahmen wir Abschied von den Freunden in Rotterdam ....
Am 21. des 8. Monate reisten wir nach England ab, William
Penn, George Keith und ich und Gertrud Dirick Nieson mit ihren
Kindern. Wir hatten eine lange und gefahroolle Überfahrt ....
aber der Herr, der den Winden gebieten kann und die stürmischen
Wellen des Meeres- stille macht, daß sie auf und nieder gehen,
wie es ihm gefällt, er behütete unß ..... Am Abend des 23.
kamen wir in Harwich an. Am nächsten Morgen gingen William
Penn und George Keith mit mir nach Eolchester .... Dort blieben
wir biz zum Ersten Tag, da es mich oerlangte, der Versammlung
der Freunde beizuwohnen. EZ war eine riberauß zahlreiche und
wirksame, denn ale- die Freunde von meiner Rückkehr hörten,
strömten sie von allen Seiten herbei vom Lande und auch autz
der Stadt, so daß etwa tausend Menschen anwesend waren ....
Am 9. dez 9. Monats kam ich nach London, wo ich mit großer
Freude empfangen wurde.
Als ich einige Zeit in London war, schrieb ich folgenden Brief
an meine Fran: ,
,,LiebeS Herz,
Dir und den Kindern meine Liebe und allen andern Freunden
in der Wahrheit, der Kraft und dem Samen deß Herrn, der über
allem ist. Dem Herrn sei Ehre und sein Name sei immerdar
hochgelobt! Er hat mich durch allerlei Trübsal und Gefahr hin-
durch geführt, in seiner ewigen Kraft; ich bin zweimal in der Ver-
sammlung in Gracechurchstreet gewesen, und obgleich auch feindliche
Geister zugegen waren, mar doch alleö ruhig; der Tau des Himmelß
fiel aus die Anwesenden, und die Herrlichkeit dez Herrn schien


% \picinclude{./260-269/p_s267.jpg} 
Rückkehr nach England. Kampf der Ordiunigöpartei usw. 267
über allen. Jch muß wohl oder übel täglich zu Versammlungen
gehen, in geschäftlichen Angelegenheiten und wegen allerlei Drang-
sal, deren ez viele gibt rings umher, und oiele Freunde haben
gegenwärtig darunter zu leiden, darum in Eile euch alle gr«üßend.«
London, 24. des 9. Monats 1677. G. F.
Um diese Zeit erhielt ich Briefe aus Neu-England, welche
berichteten, wie die Behörden grausam und unchristlich gegen die
dortigen Freunde oerfuhren, indem sie sie abscheulich mifzhandelten
und peitschten; sie peitschten viele Frauen unter den Freunden.
Eine Frau banden sie an einen Karren und fchleppten sie halb-
entblößt durch die Straßen. Sie peitschten einige Schisfökapitäne,
die selber keine Freunde waren, nur weil sie Freunde hergebracht
hatten. Währenddem sie aber in dieser barbarischen Weise die
Freunde verfolgten, schlugen die Jndianer sechzig ihrer Leute,
nahmen einen der Führer gefangen und zogen ihm bei lebendigein
Leib die Haut vom Kopf und trugen sie im Triumph davon.
Manche einsichtige Leute sagten: »GotteS Gericht ist über sie ge-
kommen, weil sie die Quiiker oerfolgten.« Aber die oerblendeteu,
verfinsterten Priester sagten, ez sei, weil sie sie nicht genug ver-
folgt hätten. Jch hatte große Mühe, für die fernen leidenden
Freunde Erleichterung zu schaffen, damit sie nicht unter die Rute
der Bösewichter kämen .....
Ich blieb etwa einen Monat in London; darauf ging ich
nach Buckinghamshire und besuchte die dortigen Freunde und
hatte mehrere Versammlungen. Qfterß machten während derselben ,
solche, die von der wahren Einigkeit der Freunde in der Wahrheit
abgewichen und in Zank, Zwiespalt und Auflehnung geraten
waren, große Störungen, besonder-3 während der Männeroer=
sammlungen bei Thomaß Gllwoodß in Hunger Hill; ihr Anführer
kam von Wickham und versuchte die Freunde zu stören und
an der weiteren Abhaltung der Versammlung zu hindern. A18
ich ihr Vorhaben merkte, ermahnte ich sie, ruhig und vernünftig
zu sein und die Versammlung nicht durch Unterbrechungen zu
stören; sondern, wenn sie mit dem Vorgehen der Freunde nicht
einverstanden seien und etwas dagegen einzuwenden hätten, dafür
eine Versammlung auf einen andern Tag zu veranstalten. Die
Freunde boten ihnen an, an einem folgenden Tag eine Versamm-
lung für sie abzuhalten, und schließlich wurde eine solche für die
darauffolgende Woche bei Thoma?. Ellwood festgesetzt. Die


% \picinclude{./260-269/p_s268.jpg} 
268 Kapitel 12c11l.
Freunde trafen sie dort, und die Versammlung sand in der
Scheune statt, weil so viele gekommen waren, daß das Haus sie
nicht fassen konnte. Nachdem wir eine Zeitlang dagesessen hatten,
fingen sie an mit ihren Zänkereien. Die meisten ihrer Pfeile
waren gegen mich gerichtet; aber der Herr war mit mir und
stärkte mich, daß ich in seiner Kraft die Pfeile der Bosheit und
Falschheit gegen sie selber zurück schleudern konnte. Jhre Ent-
gegnungen wurden widerlegt, und manches wurde den Leuten
geoffenbart, und die Wahrheit wurde gefördert; viele, die zuvor
schwach gewesen, wurden gestärkt und gefestigt; etliche, die ge-
schwankt und gezweifelt, wurden überzeugt und befestigt, und die
gläubigen Freunde wurden erquickt und ermuntert im Wachstum
des Lebens. Denn die Kraft wuchs unter uns, und das Leben
gedieh, und manch lebendiges Zeugnis wurde abgelegt gegen die
bösen, trennenden und spaltenden Geister, von denen jene Gegner
getrieben wurden, und die Versammlung endete zur Zufriedenheit
der Freunde. Jch iibernachtete mit anderen Freunden bei
Thomas Ellwood; in der gleichen Woche hatte ich noch eine Ver-
sammlung mit den Gegnern in Wickhatn, wo sie abermals ihre
Bosheit zeigten und vor den Rechtgefinnten blvßgestellt wurden ....
Hierauf besuchte ich die Freunde in Henley in Oxfordshire,
und dann gings durch Cosham nach Reading, wo ich eine große
Versammlung mit Freunden hatte. Am folgenden Tage in einer
Versammlung zur Besprechung über die Einrichtung einer Frauen-
versammlung gerieten etliche, die dem Geist der Uneinigkeit Raum
gegeben hatten, in Streit und waren eine Zeitlang widerspenstig,
bis die Wucht der Wahrheit sie bezwang. Daraus hatte ich
Versammlungen an verschiedenen Orten, und am 24. des 11.Mo-
nats, gerade zum Jahrmarkt, kam ich nach Bristol.
Jch blieb während der ganzen Zeit des Jahrmarkts da und
noch einige Zeit nachher. Wir hatten viele schöne Versammlungen.
Aus allen Gegenden des Landes waren viele Freunde da, teils
in Geschäften, teils um Sachen der Wahrheit willen. Groß war
die Liebe und Einigkeit unter denjenigen Freunden, die der
Wahrheit treu blieben. Jedoch etliche, die von der heiligen
Einigkeit abgewichen waren und in Streit, Uneinigkeit und Feind-
seligkeit geraten, waren grob und beleidigend und benahmen sich
unchristlich gegen mich. Aber die Krast des Herm war über
allen; weil sie mich in der himmlischen Geduld erhielt, welche


% \picinclude{./260-269/p_s269.jpg} 
Rückkehr nach England. Kampf der Ordnungspartei usw. 269
kann Schmähungen um seines Namens willen ertragen, so fühlte
ich mich Herr über die groben und ungeregelten Widerspenftigen
und überließ sie dem Herrn, der meine Unschuld kannte und sich
meiner Sache annehmen würde. Je eisriger diese waren, um
mich zu schmähen und zu erniedrigen, desto mehr Liebe strömte
mir von den ausrichtigen, wahren, ehrlichen Freunden entgegen,
und etliche, die von den Gegnern verführt worden waren, trennten
sich von ihnen, als sie ihre Schlechtigkeit und Bosheit und ihr
grobes Benehmen sahen; sie haben alle Ursache, Gott für ihre
Errettung zu preisen .....
Am 8. des 3. Monats 1678 kam ich nach London; das Parla-
ment tagte gerade, und Freunde, die eine Klage über ihre Leiden ein-
gereicht hatten, warteten nun auf die Erklärung, daß das Gesetz
gegen päpstliche Rekusanten uns nicht treffe. Man wußte zwar
wohl, daß wir nichts mit diesen zutun hatten; aber dennoch
hatten einige böswillige Behörden davon gegen uns Gebrauch
gemacht, um uns in verschiedenen Gegenden zu verfolgen. Ich
schloß mich nun den Freunden, die sich in dieser Sache bemühten,
an, und es war Aussicht vorhanden, etwas zur Erleichterung der
Freunde aus diesem Wege zu erreichen, weil viele der Parlaments-
mitglieder den Freunden geneigt und wohlgesinnt waren und
einsahen, daß uns unsere Gegner oft falsch darstellten. Als ich
aber eines Morgens mit George Whitehead zum Parlamentsgebäude
kam, war das Parlament vertagt .....
Etwa zwei Wochen nach meiner Ankunft in London fand
die Jahres?-versammlung statt .... worüber ich meiner Frau
bald darauf in einem Brief berichtete: L
»Liebes Herz,
Dir meine Liebe in dem ewigen Samen des Lebens, welcher
alles regieret. Große Versammlungen sind hier gewesen, und die
Kraft des Herrn hat alle gepackt wie noch nie. Der Herr hat
durch seine Kraft die Freunde herrlich untereinander verbunden,
und seine glorreiche Gegenwart erschien unter ihnen. Und jetzt,
da die Versammlungen vorüber sind, lobe man den Herm in
Ruhe und Frieden. Aus Holland oernehme ich, daß dort alles
gut geht. Es sind einige Freunde hingegangen, um der Jahres-
versammlung in Amsterdam beizuwohnen. In Emden sind
Freunde, die verbannt gewesen waren, wieder in die Stadt
zurückgekehrt. Jn Danzig waren Freunde im Gefängnis und die


% \picinclude{./270-279/p_s270.jpg} 
% \picinclude{./270-279/p_s271.jpg} 
% \picinclude{./270-279/p_s272.jpg} 
% \picinclude{./270-279/p_s273.jpg} 
% \picinclude{./270-279/p_s274.jpg} 
% \picinclude{./270-279/p_s275.jpg} 
% \picinclude{./270-279/p_s276.jpg} 
% \picinclude{./270-279/p_s277.jpg} 
% \picinclude{./270-279/p_s278.jpg} 
% \picinclude{./270-279/p_s279.jpg} 

% \picinclude{./280-289/p_s280.jpg} 
Von Kingston ging ich nach London\ort{London} [...] und dann nach
Hertford\ort{Hertford} [...]. Dort traf ich John 
Story\person{Story, John}\footnote{John Story, der Führer einer Partei, 
die gegen die Frauenversammlungen und andere Einrichtungen 
auftrat.}, und etliche seiner
Richtung; aber das Zeugnis der Wahrheit hielt sie nieder, so
das wir eine ruhige Versammlung hatten. Dies war an einem Ersten
Tage, und da am folgenden Tage die geschäftliche Männer- und
Frauen-Versammlung war, so blieb ich auch noch zu dieser, um
so mehr, als etliche eine Geringschätzung derselben hatten aufkommen
lassen. Darum trieb es mich, über den Nutzen dieser Versammlungen
zu reden und über ihren Segen für die Kirche Christi, wie der
Herr es mir eingab, und dies tat den Freunden einen guten
Dienst [...].

Ich blieb den größten Teil dieses Winters 1680—81\index{Jahr!1680-81} in London,
eifrig im Dienst des Herrn, sowohl in Versammlungen als auch
sonst; denn da es eine Zeit schwerer Prüfungen für die Freunde
war, so zog es mich, ihre Versammlungen noch mehr als sonst
zu besuchen, um sie zu ermutigen und zu ermahnen durch Wort
und Beispiel. Das Parlament\index{Parlament} tagte, und die Freunde warteten
gespannt, bis sie ihre Anliegen Vorbringen konnten; denn wir
hörten fast jeden Tag von neuen Leiden, die die Freunde in 
verschiedenen Teilen des Landes zu erdulden hatten. Ich brachte
viel Zeit damit zu, meinen leidenden Brüdern Linderung zu 
verschaffen; mit einigen andern Freunden, die von sich aus sich auch
dieser Sache annahmen, wartete ich manchen Tag im 
Parlamentshaus und benutzte jede Gelegenheit, Parlanientsmitglieder, die
unsre berechtigten Klagen anhören wollten,\index{Politik}\index{Lobbyismus} 
zu sprechen. Einige
waren auch sehr entgegenkommend und Versprachen, für uns zu
tun, was sie könnten. Aber weil das Parlament gerade mit
aller Macht daran war, das päpstliche Komplott zu entdecken,
und die dabei Beteiligten herauszufinden, so benützten unsre Gegner,
die wussten, das wir nicht schwören noch die Waffen gebrauchen
dürfen, die Gelegenheit, die Strafen die über die Päpstlichen 
verhängt wurden, auch uns zuzuwenden, obgleich ihnen ihr Gewissen
sagte, das wir nicht päpstlich waren, und sie aus Erfahrung
wussten, das wir uns nicht an Verschwörungen beteiligten. Um
nun unsre Unschuld darzutun und unsern Gegnern das Maul zu
stopfen, verfasste ich ein kurzes Schreiben und lies es im 
Parlament vorweisen:
% \picinclude{./280-289/p_s281.jpg} 

\brief{Parlament}{

Wir haben den Grundsatz, uns von allen Verschwörungen\index{Verschwörung}
gegen den König oder seine Untertanen fernzuhalten; denn wir
haben den Geist Christi und durch ihn die Gesinnung Christi,
und er \zitat{ist gekommen, die Menschen zu erretten und nicht sie zu
verderben} (Luk. 9, 56\bibel{Luk. 09:56@Luk. 9:56}). Wir wollen die Sicherheit des Königs
und aller seiner Untertanen, darum erklären wir hiermit, das
wir trachten werden, alle Verschwörungen gegen ihn, von denen
wir hören, entdecken zu helfen. Solches versprechen wir euch
aufrichtig; was aber das Schwören\index{Schwören} und Kriegen\index{Krieg} anbelangt, so
können wir das ums unserer Gewissen willen nicht tun, wie ihr
ja wisset, denn wir haben alle diese Jahre hindurch um dieser
Weigerung willen viel gelitten. Wir hoffen nun, da euch der
Herr hier zusammengeführt hat, ihr werdet uns von diesen Leiden
befreien und nicht Dinge von uns verlangen, um deretwillen wir
schon so viel und so lange gelitten haben; dadurch würdet ihr
unsre Bande noch härter machen, als sie schon sind, anstatt sie
uns zu erleichtern.
\bigskip 

\begin{flushright}
G. F.\end{flushright}
}

\chapter[Mahn- und Trostschreiben]{Mahn- und Trostschreiben}

\begin{center}
\textbf{Allerlei Mahn- und Trostschreiben.}
\end{center}


Ungefähr zur gleichen Zeit erhielt ich zwei sehr gehässige
Bücher, die gegen die Wahrheit und die Freunde gerichtet waren;
das eine war von einem sogenannten Doktor aus Bremen in
Deutschland, das andere von einem Priester auö Danzig. Beide
waren Voll arger Falschheit und oerleumderischer Vorwürfe. ES
kam über mich, auf beide zu antworten, und um nicht durch
andere Geschäfte und Besuche gestört zu werden, ging ich nach
Kingston an der Themse, wo ich eine Antwort auf jedeö der
Bücher schrieb, sowie aus einige andere gehässige Schriften, die
geschrieben und verbreitet worden waren, um die Freunde falsch
darzustellen .....
Die Scherife für die Stadt sollten neu gewählt werden und
die, welche für die Wahl vorgeschlagen waren, wünschten die
Stimmen der Freunde zu erhalten; da schrieb ich einige Zeilen,
um zu erfahren, weö Geistes sie wären, und wie sie sich zu der
wahren Freiheit stellten. Ich tat es in Form einer Frage folgen-
dermasen:


% \picinclude{./280-289/p_s282.jpg} 
282 Kapitel Rzclli.
,,Gibt irgend einer, der hier in London möchte zum Sherif
gewählt werden, zu, das Christus, der vor den Toren Jerusalems
gekreuzigt wurde, ,,das Licht der Welt ist, das jeden, der in die
Welt kommt, erleuchtet-- und sagt, ,,glaubet an das Licht, aus das
ihr Kinder des Lichts seid-- (Joh. 12,Z6)? Widersetzt sich einer,
das man die Leute verfolge um der Religion willen, und darum, das
sie Gottes Gebot halten und ihn im Geist und in der Wahrheit
anbeten? Denn Christus sagt: ,,Ich bin nicht vdn dieser Welt--
(Joh.17), noch ,,ist mein Reich von dieser Welt-- (Joh. 18), darum
hält er seine Religion nicht mit weltlichen Waffen aufrecht. Christus
sagte: »Jhr sollt überhaupt nicht schwören--, und sein Apostel
Jakobus sagt dasselbe, und nun wollet ihr uns zwingen, zu
schwören und somit die Gebote Christi und seiner Apostel zu
brechen, indem ihr uns Gide vorlegt? Christus sagt zu seinen
Aposteln: ,,Umsonst habt ihr es empfangen, umsonst gebt es
auch-- (Matth. 10,8). Werdet ihr uns nicht zwingen, Zehnten
und Abgaben zu zahlen an solche Lehrer, von denen wir wissen,
das Gott sie nicht gesandt hat? Werden wir frei sein, Gott
zu dienen und ihn anzubeten und seine und seines Sohnes Gebote
zu halten, wenn wir euch freiwillig unsre Stimmen geben? Denn
wir sind nicht willens, unsre Stimmen solchen zu geben, die uns
gefangen nehmen und uns unsre Habe uehmen« .... .
Ich schrieb auch, während ich in London war, so oft ich
zwischen den Versammlungen und andern öffentlichen Gottes-
diensten Zeit hatte, verschiedene Bücher und Schriften, von denen
einige gedruckt und andere im Manuskript verbreitet wurden.
Eines davon richtete sich an die Bischöfe und andere, welche Ver-
solgungen anzetteln, und bewies ihnen aus der heiligen Schrift,
das sie nicht nach derselben wandelten und nach dem ,,königlichen
Gesetz, das gebietet, seinen Nächsten zu lieben wie sich selbst--
[Jak. 2,8), und andern zu tun, wie man möchte, das andere uns
tun. Eine andere Schrift war: »An die Menge derer, Protestanten
wie Papisten, die sich flir Christen ausgeben, deren Gottesdienst
und Religion aber in äusern Formeln und Zeremonien besteht-.
Ich wies sie mit Nachdruck aus die Worte des Apostels Paulus
hin, Galater 5,2-4: ,,Ich, Paulus, sage euch, wo ihr euch be-
schneiden läst, so ist euch Christus kein nütze. Ich sage aber
jedem, der sich beschneiden läst, das er noch das ganze Gesetz
schuldig ist. Jhr habt Christus verloren, die ihr noch durch das


% \picinclude{./280-289/p_s283.jpg} 
Allerlei Mahn- und Trostschreiben. 283
Gesetz gerecht werden wollt und seid aus der Gnade gesallen«.
Eine andere Schrift war: ,,Jeder wende sich dem Geiste Gottes
zu, um durch denselben ein rechtes Verständnis zu bekommen und
fähig zu werden, zwischen Recht und Unrecht zu unterscheiden,
zwischen Wahrheit und Irrtum und nicht unter dem Vorwand,
llbeltäter zu bestrafen, sich selber Übles zu tun, indem man den
Gerechten oerfolgt« .....
Eine andere Schrift schrieb ich über »Betrachtung, Grgötzen,
übung, Streben und Forschen«; ich zeigte aus der Schrift der
Wahrheit, worüber die wahren Christen nachdenken sollten und
worin ihren Sinn üben, was ihr E-rgötzen sein sollte, und was sie sich
zu tun bestreben sollten. Denn in diesen Dingen sind nicht nur
die Weltlichen und die leichtfertigen Leute in grosem Jrrtum,
sondern auch die grosen ,,Frommen«, sie freuen sich über die irdi-
schen, vergünglichen Dinge, während sie über himmlische Dinge
nachdenken und sich am Gesetz Gottes ergötzen und sich bestreben
sollten, immer ein reines Gewissen gegen Gott und Menschen zu
haben, gleich dem Apoftel« .....
Da die Leiden immer noch schwer und drückend auf den
Freunden lasteten, nicht nur in der Stadt, sondern auch in fast
allen Gegenden des Landes, setzte ich ein Schreiben aus, das dem
König eingereicht werden sollte. Ich brachte darin unsre Ve-
kümmernisse vor und bat für die Fälle, die mir in seiner Macht
zu stehen schienen, um Abhilfe. Da ich aber keine Hilfe von
ihm erhielt, kam es über mich, einen Brief an die Freunde zu
schreiben, um sie in ihren Leiden zu ermutigen, damit sie in Ge-
duld die vielen Prüfungen ertragen möchten, die über sie ge-
bracht wurden, sowohl oon Seiten der Behörden, als auch von
falschen Brüdern und Abtrünnigen, deren böse Bücher und gemeine
Verleumdungen die Rechtschaffenen betrübten .....
Ich blieb den grösten Teil des Winters in London im Dienst
der Wahrheit unter den Freunden, ausgenommen eine kurze Zeit,
die ich in Kingston zubrachte, im 10. Monat des Jahres, wo ich
ein Buch schrieb über: ,,Das Wesen der zeitlichen Geburt und
das der geistigen«, worin ich die Pflicht und Stellung eines Kindes,
Jünglings, Erwachsenen und Greises der Wahrheit gegenüber dar-
le te .....
g An einem Ersten Tage kam es über mich, am Nachmittage
in die Versammlung in Deoonshire House zu gehen, und da ich


% \picinclude{./280-289/p_s284.jpg} 
284 Kapitel KX17.
erfuhr, das die Freunde dort am Morgen nicht eingelassen worden
waren, wie ez-Z an dem Tage bei den meisten Versammlungen in
der Stadt geschehen war, ging ich früher hin und ging in den
Hof, ehe die Soldaten kamen, um die Eingänge zu bewachen,
aber die Konstabler waren vor mir dort und standen im Torweg
mit ihren Stäben. Ich bat sie, mich hinein zu lassen; sie sagten,
sie können und dürfen etz nicht, man habe ihnen dgö Gegenteil
geboten, ez tue ihnen leid. Ich sagte, ich wolle nicht in sie
dringen; so stand ich neben ihnen und sie waren sehr höflich.
Ich stand, biz ich müde war, und dann gab mir einer einen
Stuhl, um mich zu setzen, und nach einer Weile fing die Kraft
dez Herrn an, unter den Freunden lebendig zu werden, und
einer fing an zu reden. Die Konstabler verboten etz gleich und
sagten, er solle nicht sprechen; und da er nicht aufhörte, wurden
sie zornig. Da legte ich sanft meine Hand. auf die dez einen Kon-
stablerö und bat ihn, den Mann ruhig zu lassen, der Konstabler tat
etz und war still, und der Mann redete nicht lang. Nachdem er
geendet, trieb ez mich, aufzustehen und zu reden, und in meiner
Verkündigung sagte ich, sie brauchen nicht gegen unö vorzugehen
mit Stäben und Schwertern, denn wir seien friedliche Leute und
etz sei nichts als Wohlwollen in unsern Herzen gegen den König
und die Behörden und gegen alle Menschen auf der ganzen Welt.
Wir gebrauchen die Religion nicht als Vorwand, um uns zu
Verschwörungen und Bündnissen oder Aufständen zu versammeln,
sondern wir versammeln uns, um Gott im Geist und in der
Wahrheit anzubeten. Wir haben Christus zum Bischof, Priester
und Hirten (1. Petr. 2), er erlabt und leitet unsre Seelen, darum
können wir alle hier stille sitzen und unsres Lehrers- geniesen und
un; seiner Lehre freuen, und ich befahl sie alle Christa?-, ihrem
Bischof und Hirten. Darauf setzte ich mich nieder, und nach einer
Weile trieb ez mich, zu beten, und die Kraft des Herm war über
allen, und die Versammelten, die Soldaten und die Konstabler,
nahmen ihre Hüte ab. Alö die Versammlung zu Ende war, und
die Freunde anfingen hinaus zu gehen, nahm der Konstabler,
seinen Hut ab und bat den Herrn, das er uns segne, denn die
Kraft dee Herrn war über ihm und allen andern und über-
wältigte sie .....
Im ersten Monat des Jahres 1683 ging ich nach Kingston
an der Themse ..... Daraus nach Guildsord in Snrrey, und


% \picinclude{./280-289/p_s285.jpg} 
Zweite Reise nach Holland. Brief an den Herzog von Holstein usw. 285
nachdem ich die Freunde dort besucht hatte, weiter nach Wor-
minghurst in Sussex, wo ich eine sehr gesegnete Versammlung
mit den Freunden hatte, ohne jegliche Störung. Während ich
dort war, wurde James Claypole aus London, der mit seiner
Frau auch dort war, plötzlich krank; es war ein so heftiger Anfall,
das er weder stehen noch liegen konnte und vor heftigen Schmerzen
schrie. Als ich es hörte, wurde ich sehr betrübt im Geist um ihn
und ging zu ihm. Nachdem ich einige Worte mit ihm gesprochen
hatte, um seinen Sinn nach innen zu richten, trieb es mich, ihm
meine Hand auszulegen, und ich bat den Herrn, seine Krankheit
von ihm zu nehmen; während ich ihm meine Hand auslegte, kam
die Krast des Herrn über ihn, rmd durch den Glauben an diese
Kraft wurde es ihm gleich leichter, und er siel in einen Schlaf.
Als er erwachte, war er so wohl, das er am nächsten Tage
25 Meilen mit mir in einem Wagen fuhr, während er früher,
wie er mir sagte, gewöhnlich zwei Wochen, manchmal einen
Monat an einem solchen Anfall darniederlag. Aber der Herr
war siir ihn angerufen worden und schenkte ihm durch seine Kraft
diesmal schnelle Besserung; sein heiliger Name sei dafür gelobt
und gepriesen .....
Kapitel XXV.
Zweite Reise nach Holland. Brief an den Herzog von Holstein
zur Verteidigung des öffentlichen Redens der Frauen.
Ich reiste umher, besuchte Freunde und wohnte den Ver-
sammlungen bei .... und im 6. Monat 1683 war ich wieder
in London. Hier besuchte ich namentlich die Freunde, die im
Gefängnis waren, weil sie für Jesus Zeugnis abgelegt hatten;
ich ermuiigte sie in ihren Leiden und ermahnte sie treu zu bleiben
in ihrem Zeugnis, welches der Herr ihnen auferlegt hatte. Auch
zu solchen ging ich, die krank, schwach und im Gemüt angefochten
waren und hals ihnen, das ihr Geist nicht in ihrer Trübsal ver-
sank. Unsre Versammlungen waren oft ruhig und friedlich, oft
wurden sie aber auch von den Beamten gestört und ausgelöst.
An einem Ersten Tag kam es über mich, zu einer sehr grosen
Versammlung in Savoy zu gehen, es waren viele ,,Fromme« und
Grnstgesinnte anwesend. Der Herr ofsenbarte mir viele wichtige
Dinge, die ich den Leuten verkündete und sie aus den Geist


% \picinclude{./280-289/p_s286.jpg} 
286 Kapitel 1217.
Gottes in ihrem Jnnern hinwies, durch den alle möchten die
Schrift verstehen, die er eingegeben hat, und Christus erkennen,
welchen Gott gesandt hat .... Während ich noch redete in der
Kraft des Herm, und die Leute sehr ergriffen waren, brachen
plötzlich der Pöbel und die Konstabler wie eine Welle herein.
Einer der Konstabler rief mir zu: »Komm herunter!« und legte
Hand an mich. Ich fragte ihn: »Bist du ein Christ? Wir sind
Christen«. Gr hatte mich bei der Hand gepackt und wollte mich
herunterreisen, aber ich blieb stehen und redete ein paar Worte
zu den Leuten, indem ich den Herrn bat, das der Segen Gottes
auf ihnen sein möge. Der Konstabler rief mir immer noch zu,
herimter zu kommen und zerrte mich schlieslich herunter und hies
einen andern mit einem Stab mich ergreifen und gefangen nehmen.
Ich wurde ins Haus eines andern Beamten geführt, der an-
ständiger war; nach einer Weile wurden vier weitere Freunde,
die sie gefangen genommen hatten, hereingebracht. Ich war sehr
ermüdet und erhitzt und viele Freunde kamen zu mir, als sie
hörten, wo ich sei; aber ich hat sie alle, ihrer Wege zu gehen,
damit die Konstabler und Aufseher sie nicht ergreifen. Nach einer
Weile führten uns die Konstabler eine Meile weit zu einem Richter,
einem sehr zornigen, heftigen Mann; der Konftabler hatte ihm mitge-
teilt, das ich in Versammlungen predige; nachdem er mich nach dem
Namen gefragt, und der Schreiber ihn aufgeschrieben hatte, sagte er
mit ärgerlicher Stimme: »Wist ihr nicht, das es gegen des Königs
Gesetz ist, in solchen Konventikeln etwas zu predigen, das im
Widerspruch mit der Liturgie der Kirche von England steht?«
Gs war einer da, Shad, ein böser Aufseher, von dem es hies,
er sei aus dem Gefängnis zu Eooentry ausgebrochen und in
London in die Hand gebrandmarkt worden; als der den Richter
so zu mir sprechen hörte, ging er auf ihn zu und sagte ihm: er
habe uns schon schuldig erklärt, nach Paragraph 22 des Gesetzes
König Karls ll. »Wie! ihr sie schuldig erklären!« rief der
Richter. ,,Ja«, antwortete Shad, ,,ich erklärte sie schuldig
und ihr müst es auch tun nach dem Gesetz«. Hierauf wurde
der Richter ärgerlich über ihn und sagte: ,,Jhr wollt mich lehren?
Wer seid ihr? Ich werde sie wegen Ausruhrs schuldig erklären«.
Als der Ankläger das hörte und den Zorn des Richters sah,
ging er oerdrieslich hinweg, denn er war in seinem Vorhaben
getäuscht. Ich vermutete, das er jemand wollte gegen mich


% \picinclude{./280-289/p_s287.jpg} 
Zweite Reise nach Holland. Brief au den Herzog von Holstein usw. 287
schwören lassen, darum sagte ich: »Last niemand gegen mich
schwören, denn es ist mein Grundsatz, nicht zu schwören; darum
möchte ich nicht, das jemand gegen mich schwört«. Der Richter
fragte mich darauf, ob ich nicht in den Versammlungen predige?
Ich antwortete: »Ich bekenne, was Gott und Christus- fiir meine
Seele getan haben, und preise Gott; ich kann solches auch in den
Strasen und aus allen Plätzen tun, und schäme mich nicht, dietz zu
bekennen, auch widerspricht es der Liturgie der Kirche von England
nicht«. Der Richter sagte, das Gesetz sei gegen die Versamm-
lungen, welche der Liturgie der Kirche von England widersprechen.
Ich sagte: ,,Ich kenne kein Gesetz gegen unsre Versammlungen,
wenn er aber jene Verordnung meine, die sich gegen die richte,
welche zusammen kommen, um Komplotte, Verschwörungen und
Aufstände gegen den König zu machen, so gehörten wir nicht da-
zu, sondern wir verabscheuten alle solche Dinge. Wir hegten auf-
richtige Liebe und Wohlwollen gegen den König und alle Menschen
auf Erden. Der Richter fragte mich darauf, ob ich dem geist-
lichen Stand angehört habe, ich antwortete: ,,nein«. Darauf
nahm er sein Buch und suchte nach Gesetzen gegen uns und hies
den Schreiber unterdessen, die Namen der übrigen notieren; als
er aber kein andereö Gesetz gegen unö finden konnte, lies er den
Konstabler gegen uns schwören. Einige der Freunde hiesen den
Konstabler, sich in acht nehmen, das er nicht einen Meineid tue,
da er sie am Eingang und nicht in der Versammlung festge-
nommen habe. Aber der Konstabler war ein schlechter Kerl und
schwor, sie seien in der Versammlung gewesen.
Dennoch sagte der Richter, weil er nur einen Zeugen habe,
so müsse er die andern freisprechen; aber mich wolle er nach
Newgate schicken; ich könne dann dort predigen. Ich fragte ihn,
ob er es; mit seinem Gewissen vereinigen könne, mich nach New-
gate zu schicken dafür, das ich Gott preise und Ehristus bekenne?
Er rief: ,,Gewissen, Gewissen!«, aber ich sah, das meine Worte
sein Gewissen getroffen hatten. Er hies den Konstabler, mich
wegführen, er werde einen Verhaftbefehl machen, um mich inö
Gesängniö zu schicken, wenn er gegessen habe. Ich sagte ihm,
ich wünsche ihm Frieden und den Seinen alles Gute, und das sie
in der Furcht Gotteö bewahrt bleiben möchten; damit ging ich
fort. Der Konstabler nahm einigen Freunden das Versprechen
ab, das ich am folgenden Morgen um acht zu ihm kommen werde.


% \picinclude{./280-289/p_s288.jpg} 
288 Kapitel XIV.
Dies tat ich denn auch. Der Konstabler teilte mir mit, als er
zum Richter gekommen sei nach dem Mittagessen, um den Ver-
haftbefehl zu holen, habe dieser ihn geheisen, nach dem Abend-
gottesdienst noch einmal zu kommen. Dies habe er getan, und da
habe ihm der Richter gesagt, er solle mich zlausen lassen. ,,Somit«,
sagte der Konstabler, ,,seid ihr sreigesprochen«. Ich machte dem
Konstabler Vorwürfe, das er als Ankläger ausgetreten sei, und
er sagte, er wolle es nicht mehr tun .....
Da es über mich kant, Verschiedenes zu schreiben, ging ich
nach Kingston, damit ich vor Unterbrechung sicher sei .... Hier
schrieb ich ein kleines Buch, das bald darauf gedruckt wurde, be-
titelt: »Der Heiligen himmlische und geistige Anbetung, Einigkeit,
Gemeinschaft usw.«, in welchem ich dartat, was die wahre An-
betung nach dem Evangelium sei, und worin die wahre Einigkeit
und Gemeinschaft der Heiligen bestehe, nebst einer Blosstellung
derer, die von dieser heiligen Ginigtzit und Gemeinschaft abge-
wichen waren und sich gegen die Heiligen, die darin blieben, ge-
wandt hatten .....
Die Jahresoersammlung war im 3. Monat 1684. Sie war ge-
segnet und herrlich, und die Freunde wurden erquickt und gehoben,
denn der Herr war unter uns und teilte uns seine himmlischen
Schätze mit. Und trotzdem es eine Zeit groser Not und Gefahr
war, wegen der vielen Versolgungen durch die Obrigkeiten, so
schiitzte doch der Herr die Seinen.
Ich fühlte einen Zug in meinem Geist, nach Holland zu gehen,
um dort den Samen Gottes zu besuchen, und sobald die Jahres-
versammlung zu Ende war, rüstete ich mich für die Reise.
Alexander Parker, George Watts und Nathaniel Brassey, die
auch einen Zug nach diesem Land oerspürten, gingen mit mir ....
In Harwich schifften wir uns ein ..... Wir hatten eine gute
Überfahrt und landeten am Morgen des darauf folgenden Tages
in Briel in Holland ..... Am Morgen nach unsrer Ankunft
gingen wir nach Rotterdam, wo wir einige Tage blieben. Am
Tag nach unsrer Ankunft in Rotterdam lud mich einer namens
Wilbert Frouzen, ein Bürgermeister, auf sein Landgut ein. Er
war ein Verwandter von Aarent Sunneman, und als er von
meinem Kommen gehört, wünschte er, mit mir eine Angelegenheit,
Sunnemans Töchter betreffend, zu besprechen. Ich nahm George
Watts mit, und ein Bruder von Aarent Sunnemann brachte uns


% \picinclude{./280-289/p_s289.jpg} 
Zweite Reise nach Holland. Brief an den Herzog von Holstein usw. 2 89
hin. Der Bürgermeister empfing uns- sehr freundlich und freute
sich sehr, mich zu sehen; im Lauf des Gesprächs über die Töchter
seines Verwandten merkte ich, das er fürchtete, man werde sie,
da ihr Vater gestorben war und ihnen ein beträchtliches Vermögen
hinterlassen hatte, überoorteilen und ungünstig verheiraten. Ich er-
klärte ihm, es sei unser Prinzip, das niemand bei uns sich oer-
heirate ohne eine Einwilligung oder ein Gutachten der Verwandten
oder des Vormunds; denn es- sei unsre Ehristenpflicht, alle jungen
Leute, die zu uns kommen, zu überwachen und ihnen nachzugehen,
besonders solchen, deren rechtmüsige Angehörige gestorben seien.
Und was die Töchter seines Verwandten anbelange, so würden
wir Sorge tragen, das ihnen nichts vorgeschlagen werde, als was
mit der Wahrheit und Gerechtigkeit übereinstimme, und das sie
in der Furcht Gottes bewahrt bleiben, nach dem Sinn ihres
Vaters. Dies schien ihn sehr zu beruhigen. Während ich dort
war, kamen oiele angesehene Leute zu mir, und ich ermahnte sie
alle, in der Furcht Gottes zu bleiben und aus seinen guten Geist
in ihrem Jnnern zu achten, und ihren Sinn aus Gott zu richten.
Nachdem ich zwei oder drei Stunden dort gewesen war und mit
ihm über verschiedene Dinge geredet hatte, nahm ich Abschied oon
ihm, und er lies mich sehr freundlich in seinem Wagen nach
Rotterdam zurück führen .....
Von da gingen wir nach Amsterdam ..... Jn Amsterdam
findet die Jahresoersammlung für die Freunde von Holland,
Deutschland und andere Länder statt, und sie begann nun gerade,
am 8. des 4. Monats und endete am 12. Da hatten wir nun
eine schöne Gelegenheit, Freunde oon überall her zu sehen und
gemeinsam in der Liebe Gottes erquickt zu werden. Nach den
Versammlungen, beoor die, welche aus den verschiedenen Gegenden
hergekommen waren, wieder sort gingen, hatten wir eine Ver-
sammlung mit einigen Freunden besonders, um zu beraten über
die Gegenden und Orte, in zdie wir im Dienst des- Herrn gehen
sollten, und um zu erfahren, wer unter ihnen sich eigne, als Dol-
metscher mit uns zu gehen. Nachdem man übereingekommen war,
schissten sich William Bingley und Samuel Waldensield mit Jakob
Claus, ihrem Dolmetscher, für Friesland ein.
Alexander Parker und George Watts blieben mit mir
noch einige Tage länger in Amsterdam, wo ich noch zu tun
hatte .... Ehe ich fort ging, besuchte ich Galenus Abrahams,
George Fo:. 19


% \picinclude{./290-299/p_s290.jpg} 
290 Kapitel XIV.
einen hervorragenden Lehrer der Mennoniten oder Baptisten.
Jch war bei ihm gewesen, als ich vor sieben Jahren in Holland
gewesen war, und William Penn und George Keith Dispute
mit ihm gehabt hatten. Gr war damals sehr hochmütig und
mißtrauisch gewesen, sodaß er nicht von mir angeriihrt noch
angesehen werden wollte, sondern mich hieß, meine Augen oon
ihm abwenden, da sie, wie er behauptete, ihn stechen. Jetzt aber
war er sehr empfänglich und geneigt und bekannte sich bis zu
einem gewissen Grad zur Wahrheit; auch seine Frau und Tochter
waren sehr empsänglich und freundlich, und wir trennten uns
voller Liebe .....
Wir hatten eine Versammlung in Amsterdam, bei welcher
viele außer den Freunden anwesend waren, unter anderm der
große Baptistenlehrer Galenus, der dem Zeugnis der Wahrheit
sehr aufmerksam zuhörte und nachher zu mir kam und mir sehr
liebevoll die Hand gab .....
Wir gingen noch nach Alkmar, Sardam, einer großen Stadt
von Schissbauern, .... dann nach Harlem, wo wir im Haus
eines Freundes eine große Versammlung hatten, . .. daraus nach
Rotterdam, wo wir zu zwei Versammlungen blieben, .... und
am 16. des 5. Monats nach Briel, um uns wieder für England
einzuschifsen .....
Den folgenden Winter brachte ich in London zu, nur zwei
oder dreimal ging ich mit meiner Frau, die bei mir in der
Stadt war, zu ihrem Sohn nach Kingston. Und obgleich ich
sehr elend war, war ich beständig an der Arbeit, entweder in
öffentlichen Versammlungen, wenn es mir möglich war, sie zu
ertragen, oder in privaten Angelegenheiten unter den Freunden
und mit Besuchen bei solchen, die um der Wahrheit willen litten,
in der Gefangenschaft oder durch Verlust ihrer Habe. Ich schrieb
auch allerlei in dieser Zeit, einiges für den Druck, anderes zu
privatem Gebrauch, so Briefe an den König von Dänemark und
den Herzog von Holstein wegen der Freunde, die in ihren Ländern
zu leiden hatten. Einer dieser Briefe lautet:
,,An den Herzog von Holstein, den ich in der Liebe Gottes
bitte, dieses durchzulesen, das ihm in Liebe gesandt wird.
Jch hörte, daß, als Elisabeth Hendricks nach Friedrichstadt
kam, um die Leute, die ihr Quäker nennt, zu besuchen, dir durch
einige übelmeinende Personen hinterbracht wurde, es sei ein Ärger-


% \picinclude{./290-299/p_s291.jpg} 
Zweite Reise nach Holland. Brief an den Herzog von Holstein usw. 291
niz für die christliche Religion, wenn einer Frau gestattet werde, in
einer öffentlichen religiösen Versammlung zu reden. Daraufhin hast
du den Behörden von Friedrichstadt den Befehl gegeben, dafür
zu sorgen, daß besagte Leute die Stadt verlassen, oder sie fortzu-
schicken. Da aber jene Behörden Arminianer waren, und alß ein
in Holland versolgtes Volk vor etwa sech-Zzig Jahren hierher ge-
kommen waren, schickten sie die Antwort: »Wir sind nicht willenß,
andere um der Religion willen zu verfolgen, denn alö wir selber
die Verfolgten waren, so sahen wir die Verfolgung alö anti-
christlich an.« Daraufhin hat das- Volk Gottes, zum Spott Quäker
genannt, an dich, o Herzog, von Friedrichstadt aus geschrieben,
und seither haben sie ihre Freiheit nnd friedlichen Versammlungen
gehabt und konnten frei und ohne Hinderniö die letzten beinahe
zwanzig Jahre in Friedrichstadt Gott dienen und ihn anbeten;
diese Freiheit betrachten sie als- eine grosze Gunst von dir.
Und in der Tat, o Herzog, der du dich zum Christentum be-
kennst, daö sich nach dem großen, mächtigen Namen Jesu Christi,
des Könige der Könige, deö Herrn aller Herren, nennt und sich
aus die heiligen Schriften der Wahrheit de-3 alten und neuen
Testamentö gründet, gebrauchst du nicht in deinem Gottesdienst
ost Worte von Frauen aus dem alten und neuen Testament? Der
Apostel sagt: ,,Gure Weiber sollen schweigen in der Gemeinde«
(1. Cor. 14), und er gestattet den Weibern nicht, daß sie lehren,
sondern sie sollen stille sein, und wenn sie etwaö lernen wollen,
,,so sollen sie daheim ihre Männer fragen, eß- stehe den Weibern
übel an, in der Gemeinde zu reden«; und 1. Tim. 2, 11.12: »Ein
Weib lerne in der Stille mit aller Untertänigkeit, und ist ihr nicht
gestattet, daß sie lehre, noch daß sie des Mannes Herr sei, sondern
daß sie sei stille«. Der Herzog kann aber sehen, waß daß für
Frauen waren, die stille sein sollten und untertänig, denen das-
Gesetz befiehlt, stille zu sein: es sind die unordentlichen Frauen!
Denn im gleichen Kapitel gebietet der Apostel denWeibern, »sich nicht
mit Zöpfen oder Gold oder Perlen oder köstlichem Gewand zu
schmücken«. Diese Dinge hat der Apostel verboten, und solche
Weiber, welche sich also schmücken, die sollen ,,in der Stille lernen
und untertänig sein« und nicht über die Männer herrschen, denn
solchen stehet eß übel, in der Gemeinde zu reden. Sind ez aber
nicht gerade solche Weiber, die Gold, Silber, Perlen und köstliche
Kleider tragen und sich mit Haarflechten schmücken, die in euern
19*


% \picinclude{./290-299/p_s292.jpg} 
292 Kapitel XIV.
Kirchen reden, wenn eure Priester sie die Psalmen singen lassen?
oder reden sie etwa nicht, wenn sie Psalmen fingen? Solcheß- be-
denke, oHerzog! Und dennoch sagst du, eure Weiber sollen schweigen
in der Kirche und nicht reden, wenn sie aber Psalmen singen in
der Kirche, sind sie dann still? Und während der Apostel den
oben genannten Weibern verbietet, in der Kirche zu sprechen, so
ermutigt er an anderer Stelle die guten und heiligen Frauen, die
,,Lehrer in guten Dingen zu sein«, wie in Tit. 2,3. Der Apostel
sagt, ,,ich bitte dich, mein treuer Geselle, stehe jenen Frauen, die
mit mir für datz Evangelium gekämpft haben, bei und den andern
meinen Gehilfen, welcher Namen sind im Buche des LebenZ«
(Phil. 4,3). Hier anerkennt er diese heiligen Frauen, die mit ihm
für daß Evangelium gekämpft haben, und ermutigt sie und ver-
bietet ihnen nicht, zu reden. Ebenso anempfiehlt er Phöbe der
Kirche von Rom und nennt sie: i,,eine Dienerin der Kirche zu
Kenchreae« (Röm. 16,1), sendet durch sie seinen Brief von Eorinth
an die Kirche von Rom und bittet, sie aufzunehmen in dem
Herrn, wie ez den Heiligen geziemt, und sie in allem, waß sie
brauche, zu unterstützen, sdenn sie sei vielsen eine Stütze gewesen,
so auch ihm selber. Und ferner sagt er: ,,Grüßet Aquila und
Priöcilla, meine Helfer in Jesus Ehristuß-« (Röm. 16,3). Hier
kann der Herzog sehen, daß dieseö gute, heilige Frauen waren,
denen der Apostel nicht verbot zu reden, sondern eö gebot.
Und Aquila und Prizcilla legten ,,Apollo den Weg Gottez noch
fleißiger au?-« (Act. 18,26), hier war also Priseilla Lehrer so gut
wie Aquila, und solchen heiligen Frauen verbietet der Apostel
daß Reden nicht. Ebenso verbot er den vier Töchtern dez Phi-
lippuö, welche Jungfrauen waren, nicht, zu wei?-sagen. Und die
Weiber dürfen in der Kirche beten und wei?-sagen (1. Cor. 11,5).
Die Apostel zeigten den Juden die Erfüllung der Weißsagungen
des- Propheten Joel: ,,Jn den letzten Tagen will ich meinen
Geist außgießen über alleß Fleisch, und eure Söhne und Töchter,
Knechte und Mägde sollen weiösagen durch- den Geist GotteZ«.
Der Apostel ermahnt also die Töchter und Mägde so gut wie die
Söhne, zu wei?-sagen, und wenn sie weißsagen, ,,so sollen sie zur
Gemeinde oder zum Volk reden« (Joel. 2,28, Act. 2,18). Sangen
nicht Mirjam, die Prophetin, und alle Frauen mit ihr, dem Herrn,
alß er die Kinder Jzraelß von Pharao errettet hatte?s Prieö sie
nicht den Herrn und prophezeite in der Versammlung der Kinder


% \picinclude{./290-299/p_s293.jpg} 
Zweite Reise nach Holland. Vries an den Herzog von Holstein usw. 293
JS-rael?-? und geschah daß nicht vor der Gemeinde? (2.sMos. 15,21).
Moseß und Aaron verboten ihr nicht, zu wei-Jsagen und zu reden,
sondern Moseö sagte: ,,Wollte Gott, daß daß ganze Volk des Herrn
weiß-sagte«! (4. Mos. 11,29). Und ,,daZ Volk des- Herrn« sind
Frauen so gut wie Männer. Deborah war Richterin und Prophetin,
und gebraucht ihr nicht die Worte Deborahß und Mirjamß in euren
Gottezdiensten? Denket an Deborahß lange Rede oder Gesang!
Barak verbot eß ihrnicht, noch irgend einer der jüdischen Priester, und
hielt sie nicht diese Rede in der Versammlung oder Kirche J3-
raelß? (Rich. 5). Jm Buch Ruth sind gute Reden jener treff-
lichen Frauen, die man auch nicht verboten hat. Hanna betete
im Tempel vor Eli, und der Herr erhörte ihr Gebet. Sieh doch,
was; für eine Rede Hanna hielt, und wie sie Gott prieß vor Eli-?
Ohren, und er verbot ez ihr nicht (1. Sam. 2,1-10). Josia
schickte seinen Priester mit mehreren andern zur Prophetin Hulda,
. die in Jerusalem wohnete, um sie um Rat zu fragen (2. Könige
22,14). Hier haben also der König sowie seine Priester
den Rat der Prophetin nicht verachtet, und sie weiß-sagte i
in der Versammlung vor den Jßraeliten, wie man in diesen
Kapiteln sehen kann.
Und sieh bei Lnkaß den göttlichen Gesang der Elisabeth an
Maria und den göttlichen langen Gesang Mariaß, wo Maria sagte:
,,der Herr hat die Niedrigkeit seiner Magd angesehen.« Und braucht
ihr nicht in euern Gotteßdiensten und Gebetea die Worte der
Maria und Elisabeth, auß Lukaö 1, 41—55 und verbietet dennoch
den Frauen in euern Kirchen zu reden? Auch reden alle möglichen
Frauen in euern Kirchen, wenn sie singen und Amen sagen. Jn
Lukas 2 war die Prophetin Hanna, eine Witwe von 84 Jahren,
die nicht aus- dem Tempel ging, sondern Gott diente mit Fasten
und Beten Tag und Nacht. Hatte sie nicht von Ehristuö ge-
zeugt im Tempel und dem Herrn Danksagung gebracht, und zu
allen, die da warteten auf die Erlösung zu Jerusalem, von
Christum geredet? (Luk. 2, 36—38). Solchen heiligen Frauen
war es nicht verboten, in der Kirche zu reden, weder im Gesetz
noch im Evangelium. Waren etz nicht Maria Magdalenazund
andere Frauen, die zuerst die Auferstehung Christi den Aposteln
oerkündeten? Daß Weib war es gewesen (nämlich Eva), Idie
zuerst sündigte, und so waren etz auch Frauen, die zuerst die  
erstehung Christi verkündeten; denn Christus sagte zu Maria, ,,Gehe


% \picinclude{./290-299/p_s294.jpg} 
294 Kapitel HIV.
zu meinen Brüdern und sage ihnen, ich gehe hin zu meinem
Vater und zu euerm Vater, zu meinem Gott und zu euerm Gott''
(Joh. 20, 17). Und Luk. 24, 10 waren es Maria Magdalena und
Johanna und Maria die Mutter des Jakobus und andere Frauen,
die mit ihnen waren, die den Aposteln sagten, daß Christus von
den Toten auferstanden sei: ,,und es deuchten sie ihre Worte eben,
als wären es Märlein, und glaubten ihnen nicht.'' Und Vers 22:
»Auch haben uns erschreckt etliche Weiber der Unsern.'' Hier
kann man sehen, daß die Reden der Frauen von der Auferstehung
Christi die Apostel erstaunten. Christus sandte die Frauen, seine
Auferstehung zu predigen; es ist also keine Schande für Frauen,
Christus zu predigen, und wenn Christus sie schickt, so dürfen sie
nicht schweigen. Der Apostel sagt: ,,Jede Zunge soll bekennen''
(Röm. 14, 11); und Phil. 2, 11: ,,Jede Zunge soll bekennen, daß
Christus der Herr ist, zur Ehre Gottes des Vaters.'' Das zeigt
klar, daß die Frauen so gut wie die Männer Christus bekennen
sollen, wenn jede Zunge ihn bekennen soll. Und der Apostel sagt,
»da ist weder Mann noch Weib, sondern sie sind alle eins in
Christo Jesu'' (Gal. 3, 28). Und wenn es heißt, die Frauen
sollen daheim ihre Männer sragen, so weiß ja der Herzog wohl,
daß Jungfrauen keine Männer haben noch die Witwen. Hanna,
die Prophetin, war eine Witwe. Und da ja Christus der »Eine
Mann ist'' (2. Cor. 11, 2), so iiiüfsen ihn die Männer daheim
sragen so gut wie die Frauen, ehe sie lehren. Und gesetzt der
Fall, das Weib eines Türken sei eine Christin, oder das Weib
eines Papisten eine L-utheranerin oder Calvinistin, müssen sie
dann auch ihre Männer daheim fragen und von ihnen lernen, ehe
sie Christus in der Versammlung des Herrn bekennen? Dann
würden sie den Rat erhalten, Türken oder Papisten zu werden.
Jch bitte den Herzog, diese Dinge zu bedenken. Jch bitte
ihn, aus Gottes Gnade und Wahrheit in seinem Herzen, die aus
Jesus kommen, zu merken, damit er durch diesen Geist der
Gnade und Wahrheit dazu kommen möge, Gott im Geist und in
der Wahrheit anzubeten und ihm zu dienen, ihm, dem lebendigen,
ewigen Gott, der ihn gemacht hat, und Frieden zu haben in
Christus, den Frieden, den die Welt nicht nehmen kann. Jch
wünsche ihm Glück, Frieden und Wohlergehen in dieser Welt und
immerdar Ruhe und Freude im Jenseits. Amen.''
London, 26. des 8. Monats 1684. G. F.


% \picinclude{./290-299/p_s295.jpg} 
Zweite Reise nach Holland. Brief an den Herzog von Holstein usw. 295
Jch verließ London für einige Zeit, und besuchte die Freunde
in South-Street und an andern Orten, und hielt Versammlungen,
.. . . Darauf kehrte ich im 3. Monat nach London zurück.
A13 die Jahreöversammlung nahte, war mir angst, etz möchte
den Freunden, die vom Lande kommen wollten, etwas zustoßen
unterwegß, da das Land in großer Aufregung war, weil es hieß«
der Herzog von Monmouth lande im Westen. Aber es gefiel
dem Herm nach seiner großen Güte, die Freunde zu bewahren, und
wir durften unö in Ruhe und Frieden versammeln, und er war unter
uns in unsern Versammlungen mit seiner lebendigen, erquickenden
Gegenwart; sein heiliger Name sei gepriesen ewiglich. Jn Anbe-
tracht der Unruhen, die daß Land erregten, trieb etz mich, am
Schlusse der Versammlungen, einige Zeilen an die Freunde zu
schreiben, um alle zu warnen, sich vor dem Geist dieser Welt, in
welchem Unruhe ist, zu hüten, und in der friedsamen Wahrheit
zu bleiben:
»Liebe Freunde und Brüder,
die der Herr berufen und erwählt hat in Jesu Christo,
euerm Leben und Heil, in welchem ihr alle Ruhe und Frieden
in Gott habt. Gott der Herr hat euch, durch seine mächtige
Kraft, die über allen ist, biz auf diesen Tag bewahrt, daß ihr
ihm sollt ein aus-erwählteö, heiligeö Volk sein, damit ihr durch
seinen ewigen Geist und seine Kraft alle vor der Welt bewahrt
bleiben möget, denn ,,in der Welt habt ihr Angst« (Joh. 16, 33).
GH ist jetzt der große Tag dez allmächtigen Gotteö; er erschüttert
den Himmel und die Erde der irdischen Religionen; ,,ihre Elemente
werden vor Hitze zerschmelzen« (2. Petr. 3, 12) und »Sonne und
Mond werden den Schein verlieren, und die Sterne werden vom
Himmel sallen« (Matth. 24, 29), wie bei den Juden vor dem
Grscheinen Christi. Darum, liebe Brüder, bleibet im Samen,
in Jesuö Chtistue, im Grund und Felsen, der nicht wanken
kann; stehet im Licht und dem Geist Jesu Christi, damit ihr wie
Fixsterne am Firmament Gotteß seid. Jn seiner Kraft und seinem
Licht werdet ihr über alle jene ,,herumirrenden Sterne« sehen,
über »die Wolken ohne Wasser, über die Bäume ohne Früchte«
(Jud.). Was wanken kann, wird jetzt wanken, also alle die ab-
geirrt sind vom Firmament der Kraft Gotteß.
Liebe Freunde und Brüder, die ihr erlöst seid vom Tod und
Fall Adamß, durch Christo?-, den zweiten Adam, in ihm habt


% \picinclude{./290-299/p_s296.jpg} 
296 Kapitel TRV.
ihr Leben, Ruhe und Frieden! denn Ehristuö sagt: ,,in mir habt
ihr Frieden, aber in der Welt habt ihr Angst«. Und der
Apostel sagt: ,,die glauben, gehen in die Ruhe ein« (Hebr. 4, 3)
nämlich zu Ehristuß, der die Welt überwunden hat und der
Schlange den Kopf zertritt und den Teufel und seine Werke zer-
stört und die Zeichen und Wei?-sagungen dez alten Tesiament;3
und der Propheten erfüllt, welcher der Erste und Letzte, Anfang
und Ende ist, die ewige Ruhe. Darum bleibet in Christus eurer
Ruhe, ein jeder, der ihn ausgenommen hat.
Und nun, liebe Freunde und Brüder, maß auch für Unruhen,
Unordnungen, Gewalttaten, Streitigkeiten und Zänkereien in der
Welt entstehen, haltet euch fern von denselben, und bleibet in
der Krast und der Wahrheit, die über allem ist; dann werdet ihr
den Frieden und daß Beste eineß jeden suchen. Lebet in der Liebe,
die Gott in eure Herzen aus-gegossen hat durch Jesuß Christus-.
Jn ihr kann nichtß euch scheiden von Gott und Ehristuß, weder
Trübsal noch Verfolgung, weder Hohes noch Tiefeö, und nichts
kann eure himmlische Jtingerschast im Licht, dem Geist und dem
Evangelium Christi hindern oder zerstören, noch eure heilige
Gemeinschaft im heiligen Geist, welcher vom Vater und vom
Sohn auzgehet, und euch in alle Wahrheit leitet. In diesem
heiligen Geist habt ihr Gemfinschast mit dem Vater und dem Sohn
und untereinander. Gr ist etz, der die Kirche Christi, den Leib,
zusammenhält und mit Christuö, dem himmlischen und geistlichen
Haupt, verbindet. Er ist dat; Band des Friedenß für alle leben-
digen Glieder seiner ganzen Kirche, darin sie Ruhe und ewigen
Frieden in Christuö und Gott haben. Ihm sei Ehre und Preis?
immerdar.
Liebe Freunde, versäumet nicht, euch untereinander zu ver-
sammeln, die ihr verbunden seid im Namen Jesu eureö Propheten,
den Gott im neuen Testament hat erstehen lassen, damit man
ihn in allen Dingen anhöre. Waß er euch eröffnet, kann niemand
verschließen und was er verschließt, kann niemand öffnen; er ist
euer Priester durch die Kraft, über alle Himmel erhaben in ein
ewigeß Leben; durch ihn seid ihr zum königlichen Priestertum be-
rufen, Gott geistliche Opfer zu bringen. Er ist der Bischof eurer
Seelen (1. Petr. 2), daß er über euch wache, damit ihr nicht von
Gott abweichet; er ist der gute Hirte, der sein Leben gelassen hat


% \picinclude{./290-299/p_s297.jpg} 
Zweite Reise noch Holland. Brief an den Herzog oon,Holsteiu usw. 297
für seine Schafe, und sie hören seine Simme und folgen ihm und
er gibt ihnen ewigeß Leben (Joh. 11).
Und nun, liebe Freunde und Brüder, bleibet in Ehristuß,
dem Weinstock, damit ihr Früchte bringen möget zu Gotteß Ehre.
Und wie ihr nun Ehristuß aufgenommen habt, so wandelt in ihm,
der nicht von dieser sündigen Welt ist, damit ihr bewahret bleibt
vor dem eiteln Tun und Treiben dieser Welt, welcheß deß Fleischeß
Lust und der Augen Lust und hofsürtigeß Wesen befriedigten, die
nicht auß dem Vater sind, sondern von der Welt, die oergehet
(1.Joh.2). Ein jeder, der sich an daß hält, waß nicht vom
Vater ist, oder solcheß begünstigt, wendet den Sinn ab von Gott
dem Vater und dem Herrn Jesuß C-hristuß. Darum lasset
C-hristnß in euern Herzen regieren, damit euch Herz, Sinn,
Seele und Geist bewahrt bleiben mögen vor den Eitelkeiten dieser
Welt, in Worten und Werken, und ihr ein außerwählteß Volk
seid, fleißig zu guten Werken, und dem Herm dieuet durch Jesuß
Ehristnß, zu Gotteß Ehre und Lob ..... «
London, 11. deß 4. Monatß 1685. G. F.
Alß die Jahreßversammlung vorüber war, ging ich ein wenig
vor die Stadt hinauß, da ich sehr ermüdet war von der Hitze,
dem Gedränge in den Versammlungen und der unaußgesetzten
Arbeit. Zuerst ging ich nach South-Street, wo ich einige Tage
blieb. Eß kam so recht daß Gefühl über mich vom Wachßtum
und Zunehmen der Eitelkeit, Hofsart und Außschreitung in der
Kleidung, und daß nicht nur unter den Weltleuten, sondern
auch viel zu viel unter etlichen, die sich zu unß hielten und schienen,
die Wahrheit zu bekennen. Unter dem Eindruck, wie verderblich
dieß sei, schrieb ich eine Schrift, wo eß unter anderm heißt:
»Jch laß von einem weisen Philosophen, der, alß er eine Frau
mit bloßem Halß und Nacken antraf, sie fragte: ,,Frau, wollt ihr
dieseß Fleisch oerkaufen?« und alß sie oerneinte, sagte er; ,,Dann
bitte schließt den Laden'', womit er ihre entblößte Brust meinte.
Also sogar unter den Heiden wurden solche alß schlechte Personen
angesehen, alß nicht ehrbare Leute. Darum sollten die, welche
vorgeben, daß wahre Christentum zu kennen, sich solcher Dinge
schämen. Eß gibt sogar von einem Papisten ein Buch gegen daß
Entblößen von Brust und Nacken, und ein andereß von Richard
Baxter, dem Preßbyterianer. Wenn nun solche, dielnur ein äußer-
licheß Bekenntniß haben, so reden, wievielmehr sollten die, welche


% \picinclude{./290-299/p_s298.jpg} 
298 Kapitel QKV1.
im Besitz der Wahrheit und des wahren Christentums sind, sich
solcher Dinge schämen. Bitte, leset das dritte Kapitel des Jesaia,
wie dieser heilige Prophet betrübt war über die hossärtigen Kleider
der närrischen Weiber, und wie er vom Herrn gesandt war, sie
darum zu tadeln. .... «
South-Street, 24.des 4. Monats i685. G. F.
K
Kapitel 20171.
Kampf sür die Ordnung im Quäkertum. Jakob ll. Amnestie.
Nachdem ich einige Wochen in South=Street gewesen war
und dort manche Versammlungen für die Freunde gehalten hatte,
kehrte ich nach London zurück. Hier half ich unter anderm den
Freunden ein Zeugnis aufsetzen, um sie von dem Verdachte zu
reinigen, sie hätten sich am letzten Aufstand im Westen oder an
irgend andern Ausständen oder Vers chw örungen gegen die Regierung
beteiligt. Und dieses Zeugnis wurde dann dem obersten Richter
eingereicht, der im Begriff war, nach dem Westen zu gehen, um
die Gefangenen zu soerhören.
Jch blieb einige Zeit in London und arbeitete im Dienst der
Wahrheit. Dann ging ich für etwa eine Woche wieder aufs
Land, weil meine Gesundheit unter dem Mangel an frischer Lust
sehr litt .... und kehrte dkgnn wieder in die Stadt zurück, wo
ich während zwei Monaten ie Versammlungen besuchte und mein
Möglichstes tat, um für die Freunde, die in allen Teilen des
Landes viel zu leiden hatten, Erleichterungen zu erwirken. Auch
schrieb ich mehrere Schriften zur Förderung der Wahrheit. Die
eine handelte von der Ordnung in der Kirche Gottes, der sich
etliche unter den Freunden stark widerseizt hatten. Sie lautete:
,,Überall in der Welt besteht fiir Familie, Gesellschaft oder Stadt
irgend eine Ordnung. Jin alten Testament war es die Ordnung
Arens und Melchisedeks (Hebr. 7,1 1) nnd darnach die Ordnung Jesu
Christi, und er verachtete diese Ordnung nicht. Gott ist ein Gott
der Ordnung in seiner ganzen Schöpfung, so auch in seiner Kirche.
Und alle, die an das Licht glauben, an das Leben in Christus,
durch das man vom Tod ins Leben eingeht, sind in der Ordnung
des heiligen Geistes, und im Licht und Leben, der Kraft und dem
Reich Jesu Christi, deren Wachstum kein Ende nimmt. Aber
solches ist verborgen den Geistern der Unordnung, die so viel


% \picinclude{./290-299/p_s299.jpg} 
Kampf für die Ordnung im Qnäkertum. Jakob ll. Amnestie. 299
schreiben und drucken gegen die Ordnung, die der Herr durch
seinen Geist und seine Kraft unter seinem Volke ausgerichtet hat.
Ihr, die ihr so viel gegen die Ordnung schreit, ihr seid ja
in ein ,,Land der Finsterniz und dez Dunkelz geraten; ein Land,
da ez stockfinster ist und da keine Ordnung ist, und wo ez ist
wie Finsternis, wenn ez hell wird« (Hiob 10, 21). Jst nicht
diez euer Zustand, wie alle, die in der Wahrheit und nach dem
Evangelium des Lebenz und dez Heilz wandeln, sehen können?« . . .
G. F.
Ich konnte abermalz nicht lange in London bleiben, da ich
die Gingeschlossenheit in der Stadt nicht lange hintereinander
ertragen konnte .... Ehe ich die Stadt verließ, hörte ich von
einem berühmten Gelehrten auz Polen, der kürzlich hergekommen
war; ich lud ihn in meine Wohnung ein und hatte eine lange
Unterredung mit ihm. Nachdem ich mich über allez, was ich zu
wissen wünschte, erkundigt hatte, schrieb ich einen Brief an den
König von Polen, wegen der Freunde in Danzig, die lange
schwer zu leiden gehabt hatten. Ez folgt hier eine Abschrift
davon:
,,An den König von Polen.
An Johann den Dritten, König von Polen, Großherzog von
Lithauen, Rußland und Preußen, Beschützer der Stadt Danzig, ....
wegen der heimgesuchten und unschuldigen Leute, die man im
Groll Quäker nennt, die jetzt bei Wasser und Brot in oben-
genannter Stadt sind, in strenger Gefangenschaft, wo man ihren
Frauen und Kindern kaum erlaubt, sie zu besuchen.
O König!
Die Behörden der Stadt Danzig sagen, ez sei dein Wille,
· daß diesez unschuldige, heimgesuchte Volk solche Unterdrückung zu
erleiden habe. Nun ist die Strafe nur darum über sie verhängt
worden, weil sie zusammenkommen im Namen Jesu Christi ihrez
Erlöserz und Heilandz, der für ihre Sünden starb und zu ihrer
Rechtfertigung von den Toten auferstanden ist, der ihr Prophet
ift, welchen Gott erweckt hat, wie Mosez.
Und nun, in diesen Tagen dez neuen Goangeliumz und dez
neuen Bundez, sollten alle auf ihn hören, die »gewesen wie die
irrenden Schafe, nun aber sich bekehrt haben zum Hirten und
Bischof ihrer Seelen«. ,,Gr hat sein Leben gegeben für seine


% \picinclude{./300-309/p_s300.jpg} 
% \picinclude{./300-309/p_s301.jpg} 
% \picinclude{./300-309/p_s302.jpg} 
% \picinclude{./300-309/p_s303.jpg} 
% \picinclude{./300-309/p_s304.jpg} 
% \picinclude{./300-309/p_s305.jpg} 
% \picinclude{./300-309/p_s306.jpg} 
% \picinclude{./300-309/p_s307.jpg} 
% \picinclude{./300-309/p_s308.jpg} 
% \picinclude{./300-309/p_s309.jpg} 

% \picinclude{./310-319/p_s310.jpg} 
310 Kapitel Lllcllll. ä
daß sie uns tun''? Und wenn ihr sagt, ihr habet daß Schwert,
die Macht und die Gewalt, so sagen wir: gepriesen sei der Herr,
der euer Schwert, eure Gewalt und eure Macht gekürzt hat, so-
daß es nicht über eure Gericht?-barkeit in Danzig hinaus reicht,
und ihr wisset nicht, wie lange der Herr euch euer Schwert, eure
Gewalt und eure Macht läßt. Wir sind überzeugt, daß ihr nicht
den Geist Christi habt, und der Apostel sagt: ,,Wer Christi Geist
nicht hat, der ist nicht sein« (Röm. 8). Und Chriftuß gebot dem
Petruö, »stecke dein Schwert in die Scheide«. Die, welche um
seinet willen das Schwert zogen, um ihn zu verteidigen, sollten
durchZ Schwert umkommen. Petrusz und die Apostel zogen später
nie mehr das Schwert, sondern sagten, ihre Waffen seien nicht
fleischlich, sondern geistlich und sie kämpfen nicht mit Fleisch und
Blut (Epl). 6). Christus hat den Seinen nie ein Gebot gegeben,
daß sie jemand durch den Henker verbannen sollten, der nicht
ihre Religion habe oder annehmen wolle. Seid ihr nicht ärger
alz die Türken, die dulden, daß allerlei Religionen, sogar Christen,
in ihrem Lande wohnen und sich friedlich versammeln? Ja, die
türkischen Statthalter lassen unsre Freunde, die gefangen waren,
in Algier Versammlungen haben und sagen, sie tuen gut daran.
Jhr seid ärger, als die maurischen Barbaren, die sich gar nicht
zum Christentum bekennen, denn ihr bekennet Christuz in Worten,
aber in euerm Tun verleugnet ihr ihn. Habt ihr je in der Schrift
oder in der Geschichte gesehen, daß E den Verfolgern lange wohl
ging? Jhr seid ärger, alö die im Lande dez Mogulö, welcher,
wie man sagt, sechzig Religionen in seinem Lande zuläßt. Noch
viele andere könnten genannt werden, die ihr alle übertresft mit
eurer Grausamkeit und eurem Verfolgen des Volkeö Gottez, nur
wcil es sich im Namen Jesu versammelt und Gott seinen Schöpfer
anbetet und ihm dient. Sie dürfen nicht einmal frei atmen in
euerm Land, weder leiblich noch geistig. Sagt, wo habt ihr
solches Gebot her? Weder von Christus, noch von seinen Aposteln.
Behauptet ihr nicht, die Schriften des Neuen Testaments seien
euer Gesetz? Aber ich frage euch, was habt ihr für eine Schrift-
stelle für solches Tun? Jhr würdet gut tun, demütig zu sein,
recht zu tun und Barmherzigkeit zu üben; ruft eure Verbannten
zurück und liebt und pflegt sie. Selbst, wenn etz eure Feinde
wären, so solltet ihr sie ja, nach Christi Gebot, lieben. E3 nimmt
mich wunder, wie ihr und die Euren ruhig in euren Betten


% \picinclude{./310-319/p_s311.jpg} 
Ahnung kommender Revolutionen. Christus König usw. 311
schlafen könnt bei diesem grausamen Tun; denkt ihr nicht daran,
daß der Herr euch ein Gleiches tun könnte? Jhr könnt nicht
ganz oerstockt und gefühllos sein, es sei denn, ihr seid der Ver-
wersung anheim gefallen, und habet ein Brandmal in euren
Gewissen (1. Tim. 4, 2). Aber die christliche Liebe hofft, daß ihr
nicht alle in diesem Zustand seid, sondern daß etliche von euch
ihre Handlungen erwägen tmd mildern werden, entweder nach
dein Gesetz Gottes oder dem Evangelium.
Von einem, der euer zeitliches und ewiges Heil und nicht
euer Verderben wünscht.«
Middlesex, 28. des 2. Monats 1689. G. F.
Kurz vor der Jahresversammlung ging ich nach London.
Sie fand in diesem Jahre im dritten Monat statt und war
sehr feierlich und gewichtig; der Herr besuchte, wie er vor Zeiten
getan, sein Volk und würdigte die Versammelten seiner herrlichen
Gegenwart, den Freunden zu Trost und Freude. Nachdem alles
Geschäftliche geordnet war, kam es über mich, dem Brief, der
von der Versammlung an die Freunde gerichtet worden war,
einige Zeilen beizufügen:
,,Liebe Freunde und Brüder!
Die ihr des Herrn ewigen Arm und ewige Kraft erfahren
habt, der euch auf dem ewigen Grund und Fels bewahret und
eure Wohnung darauf erbaut hat; ihr seid in oielen Stürmen
und Wettern gewesen, welche aus den Wassern kommen, daher
das Tier kommt (Off. 13,1) und in vielen Unwettem, die durch
Abtritnnige aller Art entstanden, aber der Same, der der Schlange
den Kopf zertreten hat, und in dem das Volk Gottes gegründet
steht, ftehet fesi.,I Liebe Freunde und Brüder, wenn gleich die Welt
rings um euch erzitteri, so ist die Kraft Gottes über allem und
kann nicht wanken. Darum ihr Kinder Gottes, ihr Kinder des
Lichts und Erben des Reiches, bleibet in den Häusern des
Friedens, und haltet euch fern vom Hader über irdische Dinge;
,,Leget niemand bald die Hände auf« (1. Tim. 5,22), aus daß
ihr nicht aufgeblasen werdet mit dem, was vergänglich ist und so
zu Falle kommt; achtet aber aus die Kraft Gottes, die eure Augen
für das Gegenwärtige und das Zukünftige offen hält. ,,Darinnen
werdet ihr das Wort des Lebens erkennen und betasten« (1. Joh. 1).
G. F .....


% \picinclude{./310-319/p_s312.jpg} 
312 Kapitel 187111.
Bald darauf fand die Jahre?-versammlung in York statt; sie
war während mehrerer Jahre dort abgehalten worden wegen der
Größe der Grafschaft, und weil ez für Viele Freunde bequem war.
Weil durch etliche, die auß der Gemeinschaft der Freunde auß-
getreten waren, Schaden angerichtet worden war, so kam es über
mich, einige Zeilen an die Versammlung zu schreiben, um sie zu
ermahnen, in der reinen, himmlischen Liebe zu bleiben, die zur
wahren Einigkeit sührt und darin bewahrt:
»Liebe Freunde und Brüder in Christa?-,
Welche der Herr durch seinen ewigen Arm und seine Kraft biz
auf den heutigen Tag bewahrt hat, wandelt alle in der Kraft
und dem Geist Gotteö, der über allem ist, in Liebe und Einigkeit;
denn die Liebe einigt und erbauet alle Glieder Christi zu ihm, dem
Haupt. Die Liebe bewahret vor allem Zank und ist auö Gott;
barmherzige Liebe höret nicht auf, sondern erhebt den Sinn über
die äußern Dinge, und über allen Streit über äußere Dinge.
Sie überwindet daß Böse und treibet alle falsche Furcht auß. Sie
ist aus Gott und einigt die Herzen seines Volkes in der himm-
lischen Freude und Einigkeit. Der Gott der Liebe erhalte euch
und gründe euch fest in Christuö, euerm Leben und Heil, in
welchem ihr alle Frieden habt mit Gott. So wandelt nun in
ihm, damit ihr die friedsame Weißheit erlanget, Gott zur Ehre
und euch zum Trost. Amen.«
London, 27. dee 3. Monatö 1689. G. F.
Da ich von den vielen großen Versammlungen sehr ermüdet
und erschöpft war, sowie auch durch die viele Arbeit während der
Jahre-Joersammlung, und meine Gesundheit dadurch sehr ange-
griffen war, verließ ich die Stadt mit meiner Tochter Rouß und
ging nach ihrem Landsitz in die Nähe von Kingston, wo ich den
größten Teil deS folgenden Sommerö zubrachte; ich besuchte zu-
weilen von dort aus- die Freunde in Kingston und schrieb allerlei
zum Nutzen der Wahrheit ..... Viele Freunde besuchten mich,
sowie auch mehrere angesehene Personen, um allerlei Fragen
über Gott mit mir zu verhandeln .... Dann, im 7. Monat, verließ
ich Kingston und ging zu Wasser nach London, unterwegs- besuchte
ich Freunde und ging nach Hammersmith, das mir am Weg lag.
Da ich mich auf dem Lande einigermaßen gekrästigt hatte, ging ich
mm in London von Versammlung zu Versammlung, fleißig im Dienst


% \picinclude{./310-319/p_s313.jpg} 
Ahnung kommender Revolutionen. Christus König usw. 313
dez: Herm und in der Verkündigung göttlicher Geheimnisse, wie
der Geist der- Herrn sie mir ofsenbarte. Aber ich spürte, daß ich
es- nicht lange in der Stadt au?-halten konnte, und ich ging darum,
nachdem ich etwa einen Monat lang die Freunde dort besucht
hatte, nach Tottenham-High-Croß und von da nach Enfteld, wo
ich während etwa 3 Wochen überall die Freunde besuchte und
Versammlungen hielt. Dann, nachdem ich mich wieder ein wenig
erholt hatte, kehrte ich nach London zurück, wo ich etwa biö zum
9. Monat blieb; dann ging ich mit meinem Sohn Mead nach
seiner Wohnung in Essex, wo ich den ganzen Winter blieb ....
Dort schrieb ich Verschiedene?-, unter anderm einen Brief an die
Jahreö- und Vierteljahres-Versammlungen von Pennsylvanien, Neu-
England, Virginia, Maryland, Jersey, Earolina und andern
Niederlassungen in Amerika .... E-3 hieß darin unter anderm:
,,Bleibet in der Liebe Gotteß, die über die Liebe der Welt
erhebt, so daß eure Herzen nicht verderbt und ersüllet werden mit
den äußern Dingen oder mit den Sorgen der Welt, welche ver-
gänglich sind; trachtet nach dem, maß unrergäuglich ist, damit ihr
dessen teilhaftig werdet .,,’ Jhr sollt Alle, die ,,Frommen« wie die Ung-
gläubigen, übertrefsen in Rechtschaffenheit, Menschlichkeit und
Christlichkeit, Bescheidenheit, Mäßigkeit und in gerechtem, göttlichen
Wandel; zeiget ihnen die Früchte deö Geistes- und daß ihr Kinder
seid des lebendigen Gottetz, Kinder dez Lichts und nicht der Finster-
nis. Dienet Gott durch ein neues- Leben, denn ez ist daß Leben und
Wandeln in der Wahrheit, welchez Zeugnitz gibt von dem Gött-
lichen im Menschen, damit ,,sie eure guten Werke sehen und euren
Vater im Himmel preisen« (Matth. 5,16). Darum seid tapfer für
Gottes reine, heilige Wahrheit und verbreitet sie unter den
Gläubigen und den Ungläubigensund unter den Jndianern. Jhr
solltet einmal jährlich von allen euren Jahreöoersammlungen an
die hiesige Jahre?-Versammlung schreiben über euem Eifer sür die
Wahrheit und ihre Verbreitung, und wie die Leute sie aufnehmen,
die Gläubigen und die Ungläubigen und die Jndianer, und über
den Frieden der Kirche Christi unter euch. Denn, gelobt sei der
Herr, die Wahrheit gewinnt dort Boden, und viele werden den
Freunden recht geneigt, und die Kraft und der Same Gotteß sind
über allen; darin erhalte Gott der Allmächtige sein ganzes Volk,
zu seiner Ehre. Amen«.
Gooseß, 28. des 11. Tlltonatz 1689. G. F.


% \picinclude{./310-319/p_s314.jpg} 
314 Kapitel IKVU1.
Während ich in London gewesen war, hatte ich ein Gesicht
gehabt, von einer doppelten Gefahr die etlichen Bekennern der
Wahrheit drohte. Die eine war, daß viele der jungen Leute dem
Treiben der Welt verfielen, die andere, daß auch Alte sich den
weltlichen Dingen zuwandten. Weil mir daz nun wieder schwer
auf der Seele lag, trieb ez mich, etwasz dagegen zu schreiben .....
»An alle, die sich zu Gotteö Wahrheit bekennen,
E-Z ist mein Wunsch, daß ihr alle demütig in ihr wandelt,
denn alö der Herr mich zuerst berief, da zeigte er mir wie »junge
Leute zusammen in Eitelkeit gerieten und die alten in irdische Ge-
sinn1mg«, und diesen beiden mußte ich ,,ein Fremdling werden«.
Und nun, Freunde, sehe ich gar zu viele junge Leute, die sich
zur Wahrheit bekennen, in Weltlichkeit geraten, und zu viele
Eltern, die ez dulden. Und auch unter den Alten sehe ich viele,
die sich dem Jrdischen zuwenden. Hütet euch, daß ihr euch nicht
euer Grab grabet, dieweil ihr äußerlich noch am Leben seid, und
nicht ,,oiel Schlamm auf euch ladet« (Hab. 2,6). Der Geist der
Welt ist ein Geist der Unruhe; der Geist Christi aber ist Frieden,
darin erhalte Gott alle Gläubigen!« G. F.
Ferner schrieb ich etwaö »über daß Zeichen, von dem Jesaia
sagt, daß ,,Gott ez den Heiden geben werde« (Jes. 11), und zeigte,
daß ez Christuö sei. « .
Ende des- 7. Monate- 1690 ging ich nach London und blieb
dort bis zum Anfang dez 9. Monats. Das Parlament tagte, und
da eö gerade daran war, ein Gesetz zu machen über daö Schwören
und ein andereö über daß Heiraten, so wohnten mehrere Freunde
den Verhandlungen bei, um dahin zu wirken, daß diese Gesetze
so abgefaßt werden, daß sie den Freunden nicht schaden können.
Diese Bestreben suchte auch ich mit zu fördern, indem ich den
Verhandlungen im Parlament beiwohnte und die Sache mit
mehreren Mitgliedern besprach ....
Nachdem ich so mehr alß einen Monat in London gewesen
war, ging ich nach Tottenham und darauf nach Ford Green
und besuchte während mehrerer Wochen die Versammlungen
der Freunde in diesen Gegenden. GS trieb mich auch während
dieser Zeit allerlei zu schreiben, .... so einen Brief an die
Freunde, die alß Prediger nach Amerika gegangen waren.
,,Liebe Freunde und Brüder, Lehrer, Prediger, Ermahner
und Warner, die ihr nach Amerika und den dortigen Inseln ge-


% \picinclude{./310-319/p_s315.jpg} 
Ahnung kommender Revolutionen. Christus König usw. 315
gangen seid. Fachet die euch von Gott verliehenen Gaben in
euch an und die reine Gesinnung, und bildet eure Fähigkeiten
auö, damit ihr daß Licht der Welt werdet, ,,eine Stadt auf dem
Berge, die nicht verborgen sein kann« (Matth. 5). Lasset euer
Licht leuchten vor den Indianern, den Schwarzen und den Weißen,
aus daß ihr der Wahrheit, die in ihnen ist, entgegenkommen
möget und sie unter das Panier bringet, das Gott in Christus
ausgerichtet hat. Denn vom Ausgang der Sonne bis zum Nieder-
gang soll Gotteä Name groß sein unter den Heiden und in
jchem Tempel, d. h. vielmehr jedem geheiligten Herzen ,,soll Gott
ein Opfer dargebracht werden« (Maleachi 1). Habet Salz bei euch,
damit ihr daß Salz der Erde sein könnet, daß sie durch euch ge-
salzen werde und bewahret vor Verderben und Fäulniß, so daß
alle Opfer, die dem Herrn dargebracht werden, gewürzt und dem
Herm angenehm sind. Wachset im Glauben und in der
Gnade Christi, daß ihr nicht wie Zwerge seid, denn ein Zwerg
soll nicht herzu treten, daß er seinem Gott opfere (3. Mos. 21, 20),
wenn gleich er von Gottes Brot essen dars, daß er sich daran
nähre. Meine Freunde, seid nicht lässig, haltet eure Negewer-
sammlungen und eure Familienversammlungen, und haltet Ver-
sammlungen mit den Jndianerkönigen und ihren Räten und
Untertanen und mit andern allenthalben. Bringet sie alle zum
Geist der Taufe und der Beschneidung, durch den sie Gott
erkennen und anbeten können. Vor allem hiitei euch alle, daß
ihr euren Sinn nicht aus irdische Dinge richtet, und nicht darum
zanket und geizet; denn ,,fleischlich gesinnt sein, bringt den Tod«
(Röm. 8,6) und »Geiz ist Abgötterei« (Col. 3, 5). GS ist zuviel deß
Zankenß und Hadernö um dieseß Abgotteö willen, so daß all-
zuviele von der Gotteöfurcht abfallen, und etliche ihre Tugend,
Menschenliebe und wahre christliche Liebe ganz verloren haben .....
Alle Glieder Christi brauchen einander. Der Fuß braucht
die Hand und die Hand den Fuß; das Ohr braucht das Auge
und das Auge daß Ohr. Alle Glieder dienen dem Leibe, davon
Christus- das Haupt ist, und daö Haupt kennt ihre Dienste.
Darum soll keiner auch daß geringste Glied verachten (1. C-or. 12).
Sehet zu, daß ihr die irdische, gewinnsüchtige Gesinnung, die
nach den Gittern und Schätzen dieser Welt trachtet, unterdrürket,
damit ihr nicht herunter sinket zu den Heiden und deß Reicheö
Gotteß verlustig geht, das kein Ende hat. Trachtet am ersten


% \picinclude{./310-319/p_s316.jpg} 
316 Kapitel 2(Rillll.
nach diesem Reiche, so wird euch das andre alles zufallen
(Matth. 6, 33). Gott erhält alles im Himmel und auf der Erde.
Jhm sei Lob und Dank für alle seine unaußsprechlichen Gaben
die zeitlichen wie die geistlichen.«
Tottenham, 11. deß 10. Monats: 1690. G. F.
Bald darauf ging ich nach London zurück und wohnte fast
täglich Versammlungen der Freunde bei. Nachdem ich etwa zwei
Wochen dort gewesen war, legten sich die Leiden, die die Freunde
in Jrland zu erdulden hatten, schwer auf meine Seele,1) und es
trieb mich, ihnen einen Trost zu schreiben:
»Liebe Freunde und Brüder in Christo, denen der Herr
durch den starken Arm seiner Kraft durch die vielen Leiden hin-
durch geholfen hat. .... Jch vertraue auf den Herrn, daß er
euch auch ferner hindurch helfen werde, und den, der glaubt, in
seiner Weißheit erhält, damit ihm gerechterweise kein Leid zugefügt
werden kann. Und wenn ihr Unrecht leiden müßt, so möge der
Gott der Gerechtigkeit euch aufrecht erhalten und beistehen und
euch nach eurem Verdienst lohnen ..... Wahrlich, meine Freunde,
wenn ich bedenke, von was für Geistern ihr umgeben seid, so
sehe ich es alß eine große Barmherzigkeit Goiteß an, daß ihr
nicht alle untergegangen seid. Aber der Herr trägt seine—Lämmer
in seinen Armen (Joh. 10) und sie sind ihm teurer alß der—Apsel
de-Z Augeß (Psalm 178) .... . Darum tun alle seine Kinder gut,
sich ihm mit Seele, Herz und Geist ganz hinzugeben, denn er
ist ein treuer Hüter; er schläft noch schlummert nicht, (Psalm 121).
.... Jhm gehört die Macht im Himmel und auf Erden; und
euch, ,,die ihr ihn ausnehmt, gibt er Macht, Gottes Kinder zu
werden« (Joh. 1, 12).
Darum lebet und bleibet in Jesus Christus-, damit nichtß
zwischen euch und Gott sei als- Christuz, in dem ihr Leben, Er-
lösung, Ruhe und Frieden habt mit Gott.
Uber die Sache der Wahrheit, hier und anderwärttz, kann
ich euch mitteilen, daß in Holland und Deutschland überall die
Freunde in Einigkeit und Liebe und im Frieden sind, sowie auch
in Jamaika, Barbadoeö, Neoiö, Antigua, Maryland und Neu-
England. Der Herr bewahre sie alle vor der Welt, in der
Bekümmerniß ist, in Christo, in welchem Frieden und Liebe ist.
1) Die Leiden der Quäker in Jrlaiid in diesen Jahren waren sehr
dtiickend; allein an Habe haben sie in 3 Jahren bei 1110 000 :8 verloren.


% \picinclude{./310-319/p_s317.jpg} 
Ahnung kommender Revolutionen. Christus König usw. 317
Amen. Meine Liebe im Herrn Jesuß Ehriftuz einem jeden Freund
im ganzen Land, wie wenn ich ihn mit Namen genannt hätte.«
London, 10. dez 11. Monats 1690. G. F.
(Nachschrift von William Penn).
Dies, lieber Leser, war nun die Beschreibung dez Lebens-,
der Reisen, der Arbeit und der tnannigfachen Leiden und Prüfungen
dieses heiligen Gotteßmanneö, von seiner Kindheit an biz fast zu
seinem Tode; er hat selber Aufzeichnungen darüber gemacht, auß
denen die vorangehenden Blätter genommen sind. EZ bleibt nun
nur noch übrig, über die Zeit, den Ort und die Art seineß Todeö
und Begräbnissez zu berichten.
Am Tage nachdem er den vorhergehenden Brief nach Jrland
geschrieben hatte, ging er zur Versammlung in Graeechurch=Street,
die sehr zahlreich war, da ez ein Erster Tag war. Der Herr
schenkte ihm die Kraft, die Wahrheit mächtig und eindringlich zu
predigen und viele wichtige und tiefe Dinge mit großer Kraft und
Klarheit darzutun. Nachdem er gebetet hatte, und die Versammlung
zu Gnde war, ging er zu Henry Goldney, in White-Hart-Court,
nahe beim Vetsammlungtzhauß;. Zu einigen Freunden, die mit
ihm gingen, sagte er: ,,Jch fühlte, wie die Kälte mir bi-8 ins
Herz drang, als ich aus der Versammlung kam. Aber ich bin
froh, daß ich gekommen bin, denn nun ist eö getan! völlig getan!«
Sobald die Freunde fort waren, legte er sich auf ein Bett, wie
er oft getan hatte, wenn er nach den Versammlungen müde war.
Bald erhob er sich wieder, um sich aber sogleich wieder nieder
zu legen, immerwährend über Kälte klagend. Seine Kräfte nahmen
zusehendö ab, und er war bald genötigt, ganz zu Bett zu gehen,
wo er ganz ruhig und friedlich und bei vollem Bewußtsein biz
zum Ende lag. Und wie während seines ganzen Lebentz sein
Geist in allumfassender Liebe Gotteö auf die Ausrichtung der
Wahrheit und Gerechtigkeit gerichtet gewesen war, und darauf,
daß der Weg dazu allen Völkern und denen, die noch ferne davon
sind, bekannt werde, so war auch jetzt in seiner Schwachheit fein
Sinn ganz davon in Anspruch genommen, und er ließ einige
nähere Freunde kommen, denen er seinen Wunsch und Willen über
die Verbreitung der Bücher der Freunde und der Wahrheit durch


% \picinclude{./310-319/p_s318.jpg} 
318 Kapitel KHVUI.
dieselben, aussprach. Mehrere Freunde besuchten ihn in seiner
Krankheit. Zu einigen von ihnen sagte er, ,,Es ist alles gut! Der
Same Gottes herrscht über alles, selbst über den Tod. Und obgleich
mein Körper schwach ist, so ist die Kraft Gottes doch über allem,
und der Same herrscht über alle widerspenstigen Geister.«
So lag er da, in einer himmlischen Gemütsoersassung, den
Geist gänzlich aus den Herrn gerichtet, während seine Kräfte mehr
und mehr abnahmen; und am dritten Wochentage, zwischen neun
und zehn Uhr Abends, schied er friedlich aus diesem Leben und
entschlies sanft im Herrn, dessen herrliche Wahrheit er lebendig
und kräftig noch zwei Tage zuvor gepredigt hatte. So endete
er seine Tage in treuem Zeugnis, in völliger Liebe und Einigkeit
mit seinen Brüdern, und in Frieden und Wohlwollen gegen alle
Menschen, am 13. des 11. Monats 1690, in seinem 67. Lebens-
jahr. ....
Am Tage da George Fox begraben wurde, kam Mittags
eine große Menge von Freunden im Versammlungshause in
White-Hart-Court, nahe bei Graeechurch-Street, zusammen, um
seinen Leib zu Grabe zu geleiten. Die Versammlung dauerte
etwa zwei Stunden. Sie war sehr feierlich und sichtlich gesegnet
mit der Gegenwart des Herrn und seiner herrlichen Kraft; und
mancher legte Zeugnis ab von dem Eindruck und dem Andenken,
welches das Wirken des treuen bewährten Diener Gottes hinter-
ließ; von seinem frühen Eintreten in den Dienst beim Anbruch des
Tages des Evangeliums, seinem reinen Leben, seinen mühsamen Reisen
und s einem unermüdlichen Arbeiten in der Liebe für das unoergäng-
liche Evangelium und für die Bekehrung vieler Tausender von der
Finsternis zum LichteJesu Christi, dem Grund des wahren Glaubens;
von den mancherlei Leiden und Heimsuchungen und allem Widerstand,
den er um seines treuen Bezeugens willen zu erdulden hatte, sowohl
von seiten seiner öffentlichen Gegner als auch von falschen Brüdern;
von allen Befreiungen, aller Hilfe und allen Siegen in diesem
allem durch die Kraft Gottes, welcher immerdar die Ehre und
der Ruhm gebührten, sowie auch fürderhin ihr allein gebühren sollen.
Nachdem die Versammlung zu Ende war, wurde sein Leib
von Freunden nach dem Begräbnisplatz der Freunde, in der Nähe
von Bunhill-Field getragen; dort wurde sein Leib der Erde über-
geben, nach andächtigem Warten aus den Herm, und nachdem
mehrere lebendig Zeugnis abgelegt hatten, die Anwesenden der


% \picinclude{./310-319/p_s319.jpg} 
Ein Brief bon George Fox nach seinem Tode vorgesunden usw. 319
Führung und dem Schutz des heiligen Geistes und der Kraft be-
sehlend, durch welche dieser heilige Mann Gotteß erwählt, auß-
gerüstet, gestärkt und beschützt worden war bis anö Ende seineß
Leben?-; sein Andenken aber soll bleiben und immerdar ein Segen
sein bei den Gerechten.
Anhang.
Ein Brief von George Fox nach seinem Tode votgesunden mit der
Aufschrift: Nicht vor der Zeit zu össnen.
Ein Brief von George Fox,
,,An die Jahreöversammlung und die Versammlung des
Zweiten Tages in London und an alle Kinder Gotteö an allen
Orten der Welt. An alle Kinder Gottes- überall, die von seinem
Geiste geleitet werden und in seinem Lichte wandeln, in welchem
sie daß Leben haben und Einigkeit mit dem Vater und dem Sohn
tmd untereinander. Haltet alle eure Versammlungen im Namen
des Herm Jesu, die ihr in seinem Namen versammelt seid, in
seinem Licht, seiner Gnade, Wahrheit und Kraft, und in seinem
Geist, durch welchen ihr seine gesegnete und erfrischende Gegen-
wart unter euch spüren werdet, unter euch und in euch, zu euerm
Trost und zu Gotteö Ehre.
Liebe Freunde, alle eure Versammlungen, die Männer- wie
die Frauen-Versammlungen, die monatlichen, vierteljährlichesn und
jährlichen, sind durch die Kraft, den Geist und die Weißheit
Gotteö eingesetzt worden. Durch sie habt ihr seine Kraft, seinen
Geist und seine Wet?-heit, seine gesegnete erhebende Gegenwart
gespürt, unter euch und in euch zu seiner Ehre und euerm Trost,
so daß ihr ,,eine Stadt, die aus dem Berge liegt und nicht ver-
borgen sein kann,« (Matth. 5) geworden seid.
Und wenn schon zu Zeiten manche losen und unreinen Geister
sich gegen euch erhoben haben, um sich in Schriften und auf
andere Weise euch zu widersetzen, so habt ihr ja gesehen wie sie
zu nichte geworden sind; der Herr hat sie versengt, hat ihre
Taten antz Licht gebracht, und machte, daß man sah, wie sie
Bäume ohne Früchte, Brunnen ohne Wasser waren, irrige Sterne,
abgesallen am Firmament Gottetz, wütende Wellen dez Meeres,
die ihren Schaum und Schmutz auswerfen (Jud.); und viele von
ihnen sind wie der Hund, der frisset, waö er gespeiet hat, und wie


% \picinclude{./320-329/p_s320.jpg} 
320 Anhang.
die Sau, die, nachdem sie gewaschen ist, sich wieder im Kot
wälzt (Spr. 26, 11). Dietz ist der Zustand vieler gewesen, wie
Gott und sein Volk weiß. .
Darum bleibet alle fest in Jesus Christus- euerm Haupt, in
welchem ihr alle einß seid, Männer und Weiber, und seine Herr-
schaft kennet. Seine Herrschaft und sein Friede werden nicht
aufhören zu wachsen; aber die Herrschaft deö Teuselö und mit
ihr alle, die nicht in Christo sind, sich ihm und seiner Herrschaft
widersetzten, werden ein Ende haben, ihr Gericht bleibt nicht aus-
und ihre Verdammniß schlummert nicht. Darum lebet und
wandelt alle in Liebe, Unschuld und Reinheit, in Licht, Leben,
Geist und Kraft auß Gott und Ehristue, die über allem sind.
Bleibet in der Rechtschassenheit und Heiligkeit, in der Kraft und
dem heiligen Geist Gotteß, in welchem daß Reich Gottez steht.
Alle, die ihr Kinder des neuen, himmlischen Jerusalem auö der
Höhe seid, richtet eure Blicke dorthin.
Der Geist der Auflehnung und des Widerstandes, der früher
und auch kürzlich wieder sich erhoben hat, stammt nicht auö dem
Reich Gotteß und ist ferne vom Reich Gottes:-’ und vom himmlischen
Jerusalem und fällt dem Gericht und der Verdammniß anheim
mit allen seinen Büchern, Worten und Werken. Darum sollen
die Freunde in der Kraft und dem Geist Gotteß leben und wandeln,
die über jenem Geist sind und im Samen, der ihn vernichten und
in Stücke schlagen wird (1. Mos. 3). Jn diesem Samen habt ihr
Frieden und Freude in Gott, und die Macht, jenen Geist der
Auflehnung zu richten, und eure Einigkeit ist in der Kraft und
dem Geist Gotteß.
Lasset keinen ihm selber leben, sondern alle sollen Gott
leben, wie sie auch ihm sterben sollen (Röm. 14); und suchet den
Frieden der Kirche Christi und den Frieden aller Menschen in
ihm, denn ,,selig sind die Friedfertigen.« Bleibet in der reinen
friedlichen, himmlischen Weiöheit Gotteß, welche friedsam, gelinde
’ und voll Barmherzigkeit ist. Trachtet alle, einerlei Sinnes und
Herzens- zu sein, eine Seele und eine Meinung in Christus,
und habet seinen Sinn und Geist in euch wohnen, ermuntert
euch untereinander in der Liebe Gotteö, welcher den Leib Christi,
seine Kirche, erbauet, deren heiligeß Haupt er ist. Ghre sei Gott
durch Ehristuß, jetzt und immerdar; er ist der Felß und Grund,
der Emmanuel Gott mit und daß Amen in allem, der Anfang


% \picinclude{./320-329/p_s321.jpg} 
Ein Vries von George Fox nach seinem Tode vorgefunden usw. 321
rmd daß Ende. Jn ihm lebet und wandelt, in welchem ihr
ewigeß Leben habet; in ihm werdet ihr mich spüren und ich euch.
Alle Kinder dez neuen Jerusalem auß der Höhe, der heiligen
Stadt, deren Licht der Herr und daß Lamm sind, und welche der
Tempel ist (Offb. 2 1), in ihr sind sie wiedergeboren auö dem
Geist; also ist daß; Jerusalem aus- der Höhe die Mutter derer,
die auS dem Geist geboren sind. Die, welche inß himmlische
Jerusalem gekommen sind und noch kommen, nehmen Christuin
auf, und er gibt ihnen Macht, Gottes Kinder zu sein, und sie sind
wiedergeboren auö dem Geist, und so ist daß Jerusalem auß der
Höhe ihre Mutter (Gal. 4). Solche kommen zum himmlischen
Berge Zion, zu der Menge vieler tausend Engel, zu den Geistern
der vollkommenen Gerechten, zu der Stadt deö lebendigen Gottetz,
zu der Gemeinde der Erstgeborenen, die im Himmel angeschrieben
sind und den Namen Gottes tragen (Ebr. 12). Hier ist eine
neue Mutter, von der ein neues himmlische?-, geistigeö Geschlecht
abstammen wird. ES gibt keine Spaltung, keinen Streit, keine
Entzweiung im himmlischen Jerusalem, noch im Leib Christi,
· welcher autz lebendigen Steinen erbaut ist (l.Petr. 2), ein geistiges
Hauß. Bei Christuö ist keine Spaltung; denn in ihm ist Friede.
Christutz sagt: ,,Jn mir habt ihr Friede« (Joh. 16). Und er ist
aus- der Höhe und nicht von dieser Erde. Jn dieser Welt und
in ihrem Geist ist Angst; darum bleibet in Christuö und wandelt
in ihm.
Jerusalem war die Piutter aller wahren Christen, vor dem
Abfall. Seitdem die äußerlichen Christen sich in viele Sekten
gespalten haben, haben sie oiele Mütter; alle aber, die durch
Christi Kraft und Geist vom Abfall zurück gekommen sind, haben
Jerusalem auß der Höhe zur Mutter und keine sonst; sie ernährt
alle ihre geistigen Kinder.« G. F.
(Dieser Brief wurde an der Jahreöoersanunlung, in London,
im Jahre 1691 gelesen.)
George Fox. 21


% \picinclude{./320-329/p_s321z01.jpg} 

% \picinclude{./320-329/p_s321z03.jpg} 

% \picinclude{./320-329/p_s321z04.jpg} 
% \picinclude{./000-009/p_s001.jpg}
\section{Kapitel 1}

\begin{center}
\textbf{Erweckung und Krisiz bis zum Durchbruch.}
\end{center}


Auf daß Jedermann wisse, was der Herr an mir getan, und
sehe, wie Er mich durch mancherlei Prüfungen, Versuchungen und
Trübsale führte, um mich für daß Werk, für daß; Er mich bestimmt 
hatte, vorzubereiten und auszurüsten, und dadurch getrieben 
werde, seine unendliche Güte und Weisheit anzubeten und
zu preisen — so will ich kurz berichten, wie es in meiner Jugend
um mich stand, und wie das Werk des Herrn in mir angefangen
und fortgesetzt wurde seit meiner Kindheit.


Ich wurde geboren im Monat den man Juli nennt\footnote{Fox 
verwarf die üblichen Monatsbezeichnungen als heidnisch.} 1624,
zu Drayton in-the-Clay, in Leicestershire. Mein Vater hieß
Christoph Fox; er war Weber von Beruf, ein ehrbarer Mann,
und es war ein "`Same von Gott"` in ihm. Die Nachbarn
nannten ihn: den "'gerechten Chrtster"'. Meine Mutter war eine
rechtschaffene Frau; ihr Mädchenname war Mary Lago, aus der
Familie der Lago und aus dem Geschlecht der Märtyrer.

In meiner frühesten Kindheit war ich so ernsten und gesetzten
Gemütez, wie es bei Kindern selten ist, so daß, wenn ich Erwachsene 
leichtfertig und ausgelassen mit einander tun sah, ich
einen Abscheu davor in meinem Herzen verspürte und zu mir
sagte: "`Wenn ich einmal ein Mann sein werde, sicherlich werde
ich nicht so leichtfertig tun."'

A1s ich elf Jahre alt war, wußte ich schon was rein und
recht ist; denn ich war als Kind gelehrt worden, wie man rein
bleibt. Der Herr lehrte mich, treu zu sein in allen Dingen, sowohl
innerlich gegen Gott als äußerlich gegen die Menschen; und daß
ich mich in allen Dingen an "`ja"` und "`nein"` halten solle; nicht
wie die Kinder der Welt, die ihren Mund voll List und gleißnerischer
Worte haben, sondern meine Worte sollen: wenig sein, "'lieblich


\picinclude{./000-009/p_s002.jpg}


und mit Salz gewürzet"` (Col. 4, 6); und daß ich nicht essen
und trinken solle, um mich wollüstig zu machen, sondern um der
Gesundheit willen, jeder Ding dazu gebrauchend, wozu es be-
stimmt ist, zur Ehre dessen, der alleß geschaffen hat ....
Alß ich dann heranwuchß, wollten meine Angehörigen einen
Priester 1) aus mir machen. Aber andere rieten zu anderm; so
kam ich zu einem, der seineö Zeichenz ein Lederhändler war, aber
mit Wolle handelte und Vieh züchtete und verkaufte; und es ging
mancherlei durch meine Hände. . Während ich bei ihm war,
war er gesegnet; aber nachdem ich ihn Verlassen, ging ez ihm
schlecht und er getiet in Verfall. Während dieser ganzen Zeit
tat ich weder gegen einen Mann noch gegen eine Frau etwas-
Unrechteö; denn die Kraft dez Herrn war mit mir und bewahrte
mich. Während ich in diesem Dienste stand, gebrauchte ich im
Verkehr daö Wort "'wahrlich"', und es war eine übliche Redenöart
bei meinen Bekannten: wenn George sagt "`wahrlich"`, so kann
ihn nichts umstimmen. Wenn die Buben oder rohe Leute über
mich lachten, kümmerte ich mich nicht um sie, sondern ging meiner
Wege; aber gewöhnlich hatten mich die Leute gem wegen meiner
Geradheit und Ehrlichkeit.
Alß ich, noch nicht ganz neunzehnjährig, in Geschäften an
einem Jahrmarkt war, kam mein Vetter, namenö Bradford, ein
"'Frommer"' cpt0keS801·) und mit ihm noch ein anderer "`Frommer"'
und forderten mich auf, mit ihnen einen Krug Bier zu trinken,
und da ich durstig war, ging ich mit ihnen hinein; denn ich
liebte jeden, der Sinn für daß Gute hatte und den Herrn
suchte. A13 jeder ein Glas getrunken hatte, fingen sie an, sich
zuzutrinken und verlangten noch mehr, indem sie aus-machten,
daß der, welcher nicht trinken würde, alletz bezahlen sollte. GS
betrübte mich, daß jemand, der sich für religiöß außgab, solchetz
tat; sie taten mir sehr weh, denn ee; war mir dergleichen noch
nie vorgekommen bei keiner Art von Menschen; darum stand ich
aus um zu gehen, indem ich meine Hand in die Tasche steckte,
einen Groschen vor sie aus den Tisch legte und sagte: "`wenn ez
so ist, will ich euch Verlassen."' So kehrte ich nach Hause zurück,
aber ich ging in jener Nacht nicht zu Bett, denn ich konnte nicht
schlafen; bald ging ich im Zimmer auf und ab, bald betete und
1) Fox bezeichnet mit priest die ordinierteu Geistlichen.


\picinclude{./000-009/p_s003.jpg}


schrie ich zum Herrn, welcher also zu mir redete: "`Tu siehst, wie
junge Leute zusammengehen in Eitelkeit und alte Leute in die
Erde. Du mußt dich von ihnen abwenden und dich von ihnen,
den jungen wie den alten, fern halten und ihnen allen ein
Fremdling werden."'
Darauf, am 9. Tage dee; 7. Monatz 1643, verließ ich nach
Gottes Befehl meine Verwandschaft und brach allen Umgang und
alle Kameradschaft mit jung und alt ab. Ich begab mich nach
Lutterworth, wo ich einige Zeit blieb und von da ging ich nach
Northampton, wo ich mich ebensallß aufhielt; darauf nach New-
port Pagnell, von wo ich nach einiger Zeit weiter nach Barnet
ging, im 4. Monat 1644. Als ich nun so da; Land durchzog,
wurden die "`Frommen"` (prokeesore) auf mich aufmerksam und
wollten mich kennen lernen. Aber ich mied fie; denn ich spürte,
daß sie nicht besaßen, was sie bekannten (proteezeä). Während
der Zeit, da ich in Barnet war, kam eine große Anfechtung zu
verzweifeln über mich. Ich sah, wie Chriftuö versucht worden
war, und war in großer Not; bald ging ich nicht aus meinem
Zimmer, und bald wanderte ich einsam durch die Fluten, um auf
den Herrn zu warten.
Ich fragte mich, warum mir solcheö widersahren müsse? Ich
prüfte mich und sagte zu mir selber: "'War ich je zuvor so ge-
wesen?"` Ich dachte, ich hätte mich vielleicht gegen meine An-
gehörigen verfehlt, weil ich sie verlassen hatte. Ich mußte immer-
während darüber nachdenken, daß ich solches getan hatte, und
mich fragen, ob ich einem von ihnen ein Unrecht getan hätte;
aber die Anfechtung wurde schwerer und schwerer, und ich wurde
bis zur Verzweiflung versucht. Und weil Satan sein Vorhaben
auf diese Weise nicht erreichte, so legte er mir Fallstricke und
Lockungen, damit ich eine Sünde begehen möchte, die er auß-
nützen könnte, um mich zur Verzweifluug zu bringen. Ich war
etwa 20 Iahre alt, als diese Prüfungen über mich kamen, und
die Angst dauerte mehrere Jahre und ich hätte mich gerne davon
frei gemacht. Ich ging zu manchem Priester, um Trost zu suchen,
aber ich fand keinen bei ihnen.
Von Barnet ging ich nach London, wo ich eine Wohnung nahm,
und dort war ich in großem Elend und Iammer; denn ich sah
daß die großen "'Frommen"' der Stadt alle in den Banden der
Finfterniß waren. Ich hatte einen Oheim dort, einen Baptisten,



\picinclude{./000-009/p_s004.jpg}

die waren damals gottselig (temier); dennoch konnte ich ihm meine
Stimmung nicht kundtun, noch mich ihm anschließen, denn ich
dUWhsch11Uke celle, jung und alt und wie ez um sie stand. Etliche
gottselige Leute (tenrler people) hätten mich gern dort behalten,
aber ich getraute mich nicht und wandte mich wieder gegen
Leicestershire; der Gedanke, ich könnte meinen Eltern und Ange-
hörigen weh tun, bedrückte mich; denn sie waren, wie ich merkte,
betrübt über meine Abwesenheit.
A15 ich nach Leieesterfhire kam, wollten meine Leute, daß ich
mich oerheirate; aber ich sagte ihnen, ich sei noch ein Knabe und
müsse weise werden. Andre hätten mich gerne bei der Hilfßtruppe
im Militär 1) gesehen, aber ich weigerte mich; und eß betrübte mich,
daß sie mir solche Dinge vorschlugen, dennsich war ein gottseliger
(temier) Jüngling. Darauf ging ich nach Eoventry, wo ich auf
einige Zeit ein Zimmer im Hause eineö "`Frommen"` hatte, bis
die Leute ansingen mich zu kennen; denn ez waren viele gott-
selige Leute in jener Stadt. Nach einiger Zeit ging ich wieder
in meine Heimat und blieb etwa ein Jahr dort, in großer Trüb-
sal; während mancher Nacht irrte ich einsam umher.
Dornach kam der Priester von Drayton, Nathanael Steoenz,
oft zu mir und ich ging oft zu ihm; und ein anderer Priester
kam oft mit ihm und sie verschmiihten nicht, mich anzuhören; ich
stellte ihnen Fragen und di?-kutierte mit ihnen. Dieser Priester
Stevenö stellte mir folgende Frage: warum Christuß am Kreuz
gerufen habe: "'mein Gott, mein Gott, warum hast du mich oer-
lassen?"` und warum er gesagt habe: "'wenn etz möglich, so gehe
dieser Kelch an mir vorüber, aber nicht wie ich will, sondern wie
du willst."` Jch erwiderte ihm, daß zu der Zeit die Sünde der
ganzen Menschheit auf ihm gelegen habe und er ihre Missetat
und Übertrettmg tragen und für sie geopfert und verwundet
werden mußte, sofern er Mensch war; aber er starb nicht, sofern
er Gott war; und weil er so für alle starb und den Tod schmeckte
für jeden Menschen, wurde er zum Opfer für die Sünden der
ganzen Welt. So sprach ich, weil ich zu jener Zeit gewisser-
maßen die Leiden Christi, und maß er durchgemacht, an mir
nachempfand. Der Priester sagte auch, es sei eine sehr treffende
Antwort, eine, wie er sie noch nie gehört habe. Zu jener Zeit
1) E3 war det Anfang der Bürgerkeiege.


\picinclude{./000-009/p_s005.jpg}

pflegte er mich zu loben und anerkennend von mir zu andern zu
sprechen; und das, war- ich ihm während der Woche im Gespräch
mitteilte, predigte er dann am \textit{Ersten Tage} 1); deöwegen mochte
ich ihn nicht leiden. Später wurde dieser Priester ein großer
Verfolger.
Darauf ging ich zu einem andern Priester in Mancetter in
Warwickshire und diökutierte mit ihm über den Grund der Ver-
suchungen und der Verzweiflung, aber er verstand meinen Zustand
nicht; er riet mir, zu rauchen und Psalmen zu singen; nun mochte
ich aber den Tabak nicht und zum Psalmensingen war ich nicht
aufgelegt; ich konnte nicht singen. Er lud mich ein, wieder zu
kommen; dann wolle er mir manches sagen; aber als ich kam,
war er ärgerlich und verdrießlich, weil meine früheren Worte ihm
mißfallen hatten. Gr redete mit seinen Dienstboten über meine
Leiden und Bekümmernisse, und ich bereute, einem solchen meine
Gesinnung aufgedeckt zu haben. Jch sah, daß sie alle leidige
Tröster (Hiob 16, 2) waren, und sie machten meine Unruhe noch
größer. Darauf hörte ich von einem Priester, der in der Nähe
von Tamworth lebte und für einen erfahrenen Mann galt. Ich
ging sieben Meilen weit zu ihm, aber ich fand, daß er nur ein
leereö, hohletz Gefäß war. Auch von einem Or. Cradock in Eoventry
hörte ich und ging zu ihm. Ich befragte ihn über Versuchung
und Verzweiflung und wie die Ansechtungen wohl über den
Menschen kommen, Gr fragte mich, wer Jesu Mutter und Vater
gewesen seien? Jch entgegnete, Maria sei seine Mutter gewesen
und er gelte als der Sohn Joseph?-, aber er sei der Sohn Gotteö.
Wir gingen gerade auf einem schmalen Weg in seinem Garten
und beim Umdrehen trat ich mit dem Fuß auf den Rand einetz
Veeteß, worüber der Mann in Wut geriet, als- ob sein Haut; in
Flammen stünde, und unsere ganze Unterredung war gestört und
ich ging in Vekümmerniz hinweg, bekümmeiter alß ich gekommen
war. Ich sah, daß sie alle leidige Tröster waren und so viel
wie nichtö für mich, denn sie konnten sich nicht in meinen Zustand
versetzen. Daraufhin ging ich zu einem, namens Macham, einem
Priester von hohem Ansehen. Gr verordnete mir Arznei und ich
mußte zu Ader lassen. Aber man konnte mir keinen Tropfen Blut
entziehen, weder am Arm noch am Kopf, trotz aller Mühe, die
Iii Fox hat den Grundsatz, statt Sonntag, Erster Tag zu sagen, da etz für
. ihn keine heiligen Tage gibt.


\picinclude{./000-009/p_s006.jpg}

man sich gab, weil mein Körper wie ausgetrocknet war
durch Kummer, Unruhe und Jammer, die so schwer auf mir
lagen, daß ich hätte wünschen können, gar nicht oder blind ge-
boren zu sein, damit ich nie die Schlechtigkeit und Eitelkeit der
Welt gesehen hätte, oder taub, daß ich nie eitle und böse Worte
gehört hätte, und wie der Name des Herrn gelästert wurde. Als
die Zeit, die man Weihnacht nennt, kam, ging ich, während andere
sich belustigten und sichs wohl sein ließen, von Haus zu Haus zu
armen Witwen und gab ihnen Geld. Wenn ich zu Hochzeiten einge-
laden war, wie zuweilen geschah, ging ich nie hin, sondern machte
erst am folgenden Tage oder bald darauf einen Besuch, und wenn
die Leute arm waren, gab ich ihnen Geld; ich besaß davon ge-
rade so viel, daß ich niemanden zur Last zu fallen brauchte und
noch dem Dürftigen etwas spenden konnte.
Zu Anfang des Jahres 1646 als-—ich, auf dem Wege nach
Cooentry, mich den Toren der Stadt näherte, stieg die Frage in
mir auf, wie man sagen könne: alle Christen seien Gläubige, so-
wohl Papisten als Protestanten; und der Herr Offenbarte mir,
daß, wenn alle Gläubige wären, so wären sie alle aus Gott ge-
boren und oom Tode zum Leben durchgedrungen (1. Joh. 3,:;
nur solche seien wahre Gläubige; und wenn auch andere sagen,
sie seien auch wahre Gläubige, so seien sie es doch nicht.
Gin andermal, als ich am Morgen eines Ersten Tages über
ein Feld ging, offenbarte mir der Herr, daß in Oxford oder Cam-
bridge erzogen sein noch nicht genüge, um tüchtig nnd fähig zum
Dienst Christi zu machen; ich verwunderte mich darüber, denn
das war die allgemeine Meinung der Leute. Aber ich sah es
vollständig ein, als der Herr es mir offenbarte und war über-
zeugt davon nnd pries die Güte Gottes, die mir solches an diesem
Morgen geoffenbart hatte. EH griff das Amt des Priesters Stevens
an, daß: ,,in Oxford oder Cambridge erzogen zu sein noch nicht
genüge, um tüchtig und fähig zum Dienst Christi zu machen"'; es
wurde mir klar, daß da-3, was mir geoffenbart worden war, das
priesterliche Amt angreife. Meine Angehörigen waren sehr betrübt,
daß ich nicht mit ihnen kommen wollte, um den Priester zu hören.
Jch ging eben lieber allein ins Freie mit einer Bibel. Ich fragte sie,
ob nicht der Apostel zu den Gläubigen sage, "'sie bedürfen nicht, daß
sie jemand lehre, die Salbung lehre sie"` (1. Joh. 2). Aber wie-
wohl sie wußten, daß solches in der Schrift steht, und daß es


\picinclude{./000-009/p_s007.jpg}

wahr ist, waren sie doch betrübt, daß ich mich in diesem Punkte
nicht unterwerfen und mit ihnen den Priester anhören konnte.
Ich sah ein, daß es ein ander Ding ist, ein wahrer Gläubiger
zu sein, als das worauf es diesen ankommt ...... Warum sollte
ich also diesen anhängen? Weder diesen noch irgendwelchen
Dissentern konnte ich mich anschließen, sondern war allein, ein
Fremdling, und hielt mich einzig an den Herrn Jesus Christus.
Ein andermal hatte ich die Offenbarung, daß Gott, der die
Welt gemacht hat, nicht in Tempeln mit Händen gemacht wohne.
Dies schien mir zuerst ein seltsames Wort, denn sowohl die
Priester als auch das Volk pflegten ihre Tempel oder Kirchen
\textit{Stätten der Ghrsurcht}, \textit{heiliger Boden} und \textit{Tempel Gottes}
zu nennen. Aber der Herr zeigte mir deutlich, daß er nicht
in diesen Tempeln wohne, die von Menschen verordnet und
ausgerichtet waren, sondern in den Herzen der Menschen. Denn
sowohl Stephanus als der Apostel Paulus gaben Zeugnis,
daß er nicht in Tempeln mit Händen gemacht wohne (Art. 7, 48),
nicht einmal in demjenigen, den er einst zu bauen befohlen hatte,
sintemal er ihm ein Ende gemacht hatte, sondern sein Volk sei
sein Tempel, und da wohne er. Solches wurde mir geoffenbart,
während ich durchs Feld zu den Meinigen ging. Als ich kam,
sagten sie mir, Priester Stevens sei dagewesen und habe gesagt,
er sei besorgt um mich, weil ich neuen Lichtern nathgehe. Jch
lächelte bei mir selber, im Gedanken was der Herr mir über ihn
und seinesgleichen geoffenbart hatte. Aber ich sagte meinen Ver-
wandten nichts davon. Denn obgleich sie den Priester durch-
schauten, gingen sie doch, ihn zu hören, und waren betrübt, daß
ich nicht auch ging. Aber ich kam ihnen mit Schriststellen und
zeigte ihnen, daß es eine Salbung gibt im Menschen, die ihn
lehrt, und daß der Herr sein Volk selber lehren will. Ich hatte
auch große Ossenbarungen über das, was in der Mokalypse steht;
wenn ich davon redete, so sagten die "'Frommen"' und die Priester,
sie sei ein versiegeltes Buch, und wollten mich davon abbringen;
aber ich sagte ihnen, Christus könne die Siegel öffnen und sie
sei das, was uns am nächsten angehe; denn die Briefe seien an
die Heiligen früherer Zeiten gerichtet, aber die Apokalypse handle
von den künftigen Dingen.
Ich traf mit Leuten zusammen, welche die Ansicht hatten,
die Frauen hätten keine Seelen, "`nicht mehr als eine Gans"',


\picinclude{./000-009/p_s008.jpg}

wahr ist, waren sie doch betrübt, daß ich mich in diesem Punkte
nicht unterwerfen und mit ihnen den Priester anhören konnte.
Ich sah ein, daß es ein ander Ding ist, ein wahrer Gläubiger
zu sein, als das worauf es diesen ankommt ...... Warum sollte
ich also diesen anhängen? Weder diesen noch irgendwelchen
Dissentern konnte ich mich anschließen, sondern war allein, ein
Fremdling, und hielt mich einzig an den Herrn Jesus Christus.
Ein andermal hatte ich die Offenbarung, daß Gott, der die
Welt gemacht hat, nicht in Tempeln mit Händen gemacht wohne.
Dies schien mir zuerst ein seltsames Wort, denn sowohl die
Priester als auch das Volk pflegten ihre Tempel oder Kirchen
\textit{Stätten der Ghrsurcht}, \textit{heiliger Boden} und \textit{Tempel Gottes}
zu nennen. Aber der Herr zeigte mir deutlich, daß er nicht
in diesen Tempeln wohne, die von Menschen verordnet und
ausgerichtet waren, sondern in den Herzen der Menschen. Denn
sowohl Stephanus als der Apostel Paulus gaben Zeugnis,
daß er nicht in Tempeln mit Händen gemacht wohne (Art. 7, 48),
nicht einmal in demjenigen, den er einst zu bauen befohlen hatte,
sintemal er ihm ein Ende gemacht hatte, sondern sein Volk sei
sein Tempel, und da wohne er. Solches wurde mir geoffenbart,
während ich durchs Feld zu den Meinigen ging. Als ich kam,
sagten sie mir, Priester Stevens sei dagewesen und habe gesagt,
er sei besorgt um mich, weil ich neuen Lichtern nathgehe. Jch
lächelte bei mir selber, im Gedanken was der Herr mir über ihn
und seinesgleichen geoffenbart hatte. Aber ich sagte meinen Ver-
wandten nichts davon. Denn obgleich sie den Priester durch-
schauten, gingen sie doch, ihn zu hören, und waren betrübt, daß
ich nicht auch ging. Aber ich kam ihnen mit Schriststellen und
zeigte ihnen, daß es eine Salbung gibt im Menschen, die ihn
lehrt, und daß der Herr sein Volk selber lehren will. Ich hatte
auch große Ossenbarungen über das, was in der Mokalypse steht;
wenn ich davon redete, so sagten die "`Frommen"` und die Priester,
sie sei ein versiegeltes Buch, und wollten mich davon abbringen;
aber ich sagte ihnen, Christus könne die Siegel öffnen und sie
sei das, was uns am nächsten angehe; denn die Briefe seien an
die Heiligen früherer Zeiten gerichtet, aber die Apokalypse handle
von den künftigen Dingen.
Jch traf mit Leuten zusammen, welche die Ansicht hatten,
die Frauen hätten keine Seelen, "'nicht mehr als eine Gans"`,


\picinclude{./000-009/p_s009.jpg} 

ein ßiziann der Schmerzen, in den Zeiten, da der Herr sein Werk
in mir anfing.
Während dieser ganzen Zeit hatte ich mich nie mit irgend
jemand zu irgend einer religiösen Richtung Verbunden, sondern
gab mich ganz dem Herrn hin; von aller schlechten Gesellschaft
hatte ich mich losgemacht, hatte Abschied genommen von Vater
und Mutter und allen andern Angehörigen und zog als ein
Fremdling umher, wohin der Herr mein Herz lenkte; ich mietete
ein Zimmer jeweilen in der Stadt, in die ich kam und weilte oft
etwa einen Monat an einem Orte; denn ich wagte nie lange an
einem Orte zu bleiben, da ich fürchtete, als gottseliger Jüngling
sowohl bei den "'Frommen"' als auch bei den Ungläubigen Schaden
zu nehmen, wenn ich viel mit den einen oder den anderen umging;
darum oerhielt ich mich meist wie ein Fremdling; ich suchte hinun-
lische Weisheit, und Erkenntnis kam mir einzig vom Herrn. Jrh
wurde losgelöst von den äußeren Dingen, um mich allein auf
den Herrn zu verlassen. Meine Prüfungen und Trübsale waren
sehr schwer; aber wenn es mir zwischen hinein etwas leichter
wurde, so geriet ich ost in solch eine himmlische Freude, daß ich
wiihnte, in Abrahams Schoß gewesen zu sein. Wie ich das Elend,
in dem ich war, nicht schildern kann, ebensowenig kann ich die
Barmherzigkeit beschreiben, die Gott in diesem Elend an mir getan
hat ....
Nachdem ich die Offenbarung vom Herm empfangen hatte,
"'daß in Oxford oder Cambridge erzogen zu sein noch nicht zum
Dienst des Herrn besähige"`, achtete ich die Priester weniger und
sah mehr auf die Dissenter; ich sah, daß unter diesen einige
Gottseligkeit sei, und viele von ihnen kamen auch später, zu einer
festen Uberzeugung, weil sie Offenbarungen hatten. Aber wie
ich die Priester aufgegeben hatte, so ließ ich auch die Separa-
ristenprediger und solche, welche als die Erfahrensten angesehen
wurden; denn ich sah, daß keiner unter ihnen allen war, derzu meinem
Zustand sprechen konnte. Als alle meine Hoffnungen auf sie und alle
Menschen dahin waren, so daß ich nichts hatte, das mir von außen
hals, und ich nicht wußte, was tun — da! o da hörte ich eine
Stimme: "'es ist Einer, der zu deinem Zustand sprechen kann,
nämlich Jesus Christus Und als ich das hörte, hüpfte mein
Herz vor Freude. Dann zeigte mir der Herr, warum niemand
auf der Welt mir in meinem damaligen Zustand helfen konnte,


 
\backmatter

\chapter{Anhang/Verzeichnis}
\chapter{Bildnachweis}

\begin{description}
 \item[Seite \pageref{bild:gfox}] George Fox, Quelle: \url{http://commons.wikimedia.org/wiki/File:Fox-George-LOC.jpg?uselang=de}, gemeinfrei.
 \item[Seite \pageref{bild:swarthmoor}] Swarthmoor Hall, Quelle: \url{http://en.wikipedia.org/wiki/File:SwarthmoreHall-1.jpg}, public domain,  

\end{description}


% \appendix
\cleardoublepage

% Index soll Stichwortverzeichnis heissen
%\renewcommand{\indexname}{Stichwortverzeichnis}

% Stichwortverzeichnis soll im Inhaltsverzeichnis auftauchen
% \addcontentsline{toc}{section}{Stichwortverzeichnis}
\cleardoublepage

\printindex

  \printindex[bibel]
\cleardoublepage

%   \addcontentsline{toc}{section}{Briefverzeichnis}
  \printindex[brief]
\cleardoublepage

%   \addcontentsline{toc}{section}{Briefverzeichnis}
  \printindex[buch]
\cleardoublepage

%   \addcontentsline{toc}{section}{Ortsverzeichnis}
  \printindex[ort]
\cleardoublepage

%   \addcontentsline{toc}{section}{Personenverzeichnis}
  \printindex[person]
\cleardoublepage


\end{document}

% Merkzettel: {Fell, Margaret}