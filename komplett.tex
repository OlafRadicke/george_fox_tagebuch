% Für das erstellen von PDFs, benutze das bash-skript make_pdf.sh
% Für das Erstellen von eBooks benutze den Befehl:
% pandoc -s komplett.tex -o komplett.epub
% Zum lesen benutze das Programm fbreader
% HTML erstellung:
% latex2html -iso_language DE -html_version 4.0,latin1,unicode komplett.tex

% \documentclass[a4paper,10pt]{article}
\documentclass[a4paper,12pt,twoside]{book} %article}
\usepackage[utf8]{inputenc}

\usepackage{ngerman}
% \usepackage[a4paper,landscape]{geometry}
%\usepackage[a4paper]{geometry}
\usepackage{multicol}
\usepackage{graphicx}
\usepackage{floatflt}
\usepackage{float}
\usepackage{url}
\usepackage[pdftex,unicode,pagebackref]{hyperref} 

\usepackage{index}
  \newindex{default}{idx}{ind}{Schlagwortverzeichnis}
  \newindex{bibel}{bidx}{bind}{Bibelstellen}
  \newindex{brief}{brdx}{brnd}{Briefverzeichnis}
  \newindex{buch}{budx}{bund}{Bücherverzeichnis}
  \newindex{ort}{odx}{ond}{Ortsverzeichnis}
  \newindex{person}{pdx}{pnd}{Personenverzeichnis}

\pagestyle{plain}

\newcommand{\picinclude}[1]{\includegraphics[height=1.0\textheight]{#1}}
% \newcommand{\zitat}[1]{"`\textit{#1}"'}
\newcommand{\zitat}[1]{„\textit{#1}“}
\newcommand{\grosszitat}[1]{\bigskip \begin{quote} #1\end{quote} \bigskip}

% Index-Relevante Befehle.
\newcommand{\bibel}[1]{\index[bibel]{#1}}
\newcommand{\brief}[2]{\bigskip \begin{quote}\index[brief]{#1} #2\end{quote} \bigskip}
\newcommand{\buch}[1]{\index[buch]{#1}}
\newcommand{\buchtitel}[1]{"`\textit{#1}"'\index[buch]{#1}}
\newcommand{\ort}[1]{\index[ort]{#1}}
\newcommand{\person}[1]{\index[person]{#1}}
\newcommand{\jahr}[1]{\index{Jahr!#1}}

%opening
\title{Auszüge aus den Tagebüchern von George Fox}

\date{Version: \today}

% \makeindex

\begin{document}

\maketitle


\begin{figure}[h!]
 \centering
 \includegraphics[height=30px]{./pics/cc-lizenz-by.png}
 % cc-lizenz-by.png: 32x32 pixel, 72dpi, 1.13x1.13 cm, bb=0 0 32 32
\end{figure}

\newpage 

\tableofcontents

\newpage

\frontmatter 

\chapter{Vorwort}
% \section{Vorwort}
\label{sec:vorwort}

\section{Urheber und Autoren}

Diese Dokument basiert auf ein Scann des Werkes \zitat{George Fox -- 
Aufzeichnungen und Briefe des ersten Quäkers}, einer 
Übersetzung von Margrit Stähelin, erschienen Tübingen 1908, 
im Verlag I.C.B. Mohr (Paul Siebek). 


Trotz intensiver 
Bemühungen war es mir nicht möglich zu ermitteln ob und wer das 
Urheberrecht an dem Werk besitzt. Sollte jemand zur
Klärung beitragen können, bitte ich um Hinweise an mich!

\begin{center}
Olaf Radicke \\
Ludwig-Richter-Str. 28 \\
80687 München \\
briefkasten@olaf-radicke.de \\
\end{center}

Ich glaube das der Text für das Quakertum in Deutschland so 
unverzichtbar ist, das ich mich entschlossen habe trotzdem den Text 
(wieder) zu veröffentlichen. Meine Text-satz und -bearbeitung 
stelle ich unter 
einer \textit{Creative Commons-Lizenz lizenziert}: 

\bigskip

\begin{center}
\textbf{Namensnennung-Weitergabe unter gleichen Bedingungen 3.0 
Deutschland (CC BY-SA 3.0).}
\end{center}

\section{Lizenz}
\subsection*{Sie dürfen}
%\addcontentsline{toc}{section}{Sie dürfen}

\begin{itemize}
 \item das Werk bzw. den Inhalt vervielfältigen, verbreiten und öffentlich zugänglich machen
 \item Abwandlungen und Bearbeitungen des Werkes bzw. Inhaltes anfertigen
 \item das Werk kommerziell nutzen
\end{itemize}

% \subsection{Zu den folgenden Bedingungen}
\subsection*{Zu den folgenden Bedingungen}
%\addcontentsline{toc}{section}{Zu den folgenden Bedingungen}

\begin{description}
 \item[Namensnennung] Sie müssen den Namen des Autors/Rechteinhabers in der von ihm 
festgelegten Weise nennen.

 \item[Weitergabe unter gleichen Bedingungen] Wenn Sie das lizenzierte Werk bzw. den 
lizenzierten Inhalt bearbeiten oder in anderer Weise erkennbar als Grundlage für 
eigenes Schaffen verwenden, dürfen Sie die daraufhin neu entstandenen Werke bzw. 
Inhalte nur unter Verwendung von Lizenzbedingungen weitergeben, die mit denen dieses 
Lizenzvertrages identisch oder vergleichbar sind.
 \end{description}

% \subsection{Wobei gilt}
\subsection*{Wobei gilt}
%\addcontentsline{toc}{section}{Wobei gilt}

\begin{description}
    \item[Verzichtserklärung] Jede der vorgenannten Bedingungen kann aufgehoben werden, 
	  sofern Sie die ausdrückliche Einwilligung des Rechteinhabers dazu erhalten.
    \item[Public Domain (gemeinfreie oder nicht-schützbare Inhalte)] Soweit das Werk, 
	  der Inhalt oder irgendein Teil davon zur Public Domain der jeweiligen Rechtsordnung 
	  gehört, wird dieser Status von der Lizenz in keiner Weise berührt.
    \item[Sonstige Rechte] Die Lizenz hat keinerlei Einfluss auf die folgenden Rechte:
      \begin{itemize}
          \item Die Rechte, die jedermann wegen der Schranken des Urheberrechts oder 
		aufgrund gesetzlicher Erlaubnisse zustehen (in einigen Ländern als 
		grundsätzliche Doktrin des fair use etabliert);
          \item Das Urheberpersönlichkeitsrecht des Rechteinhabers;
          \item Rechte anderer Personen, entweder am Lizenzgegenstand selber oder bezüglich 
		seiner Verwendung, zum Beispiel Persönlichkeitsrechte abgebildeter Personen.
      \end{itemize}
    \item[Hinweis] Im Falle einer Verbreitung müssen Sie anderen alle Lizenzbedingungen 
	  mitteilen, die für dieses Werk gelten. Am einfachsten ist es, an entsprechender 
	  Stelle einen Link auf diese Seite einzubinden.

 \end{description}

Diese "Commons Deed" ist lediglich eine vereinfachte Zusammenfassung des rechtsverbindlichen 
Lizenzvertrages in allgemeinverständlicher Sprache. Deteils, Erleuterungen und vollständigen
Lizenz-Text erhalten Sie unter \url{http://creativecommons.org/licenses/by-sa/3.0/de/}

\section{Anmerkungen zu Änderungen}

Der Text basiert zwar auf der Übersetzung von Stähelin, doch wurden 
zahlreiche Änderungen gemacht. Die wichtigsten 
seien hier genannt.

\begin{itemize}
 \item Die Kapitelnummerierung wurde nicht statisch übernommen. Die 
  Reihenfolge ist aber geblieben.
 \item Die Schreibweise \zitat{Quäker} wurde durch den englische 
  \zitat{Quaker} ersetzt.
 \item Die Schreibweise \zitat{Renter} wurde durch den englische  
  \zitat{Ranter} ersetzt.
 \item Die Kapitel-Überschriften wurden z.T. gekürzt oder geändert.
 \item Es wurden zur besseren Gliederung Unterabschitte mit Überschriften eingefügt.
 \item Es wurde ein Index erstellt mit Orten, Personen und Anderen
  Dingen.
 \item Es wurden Bilder hinzugefügt.
 \item Lange Zitate (z.B. bei Briefen) wurden durch Einrückung besser  
  kenntlich gemacht.
 \item Namen von Bibel-Büchern geändert, so wie sie Heute in der Regel
  verwendet werden.
 \item Die Altdeutscheschrift durch moderne Schrift ersetzt.
 \item An einigen Stellen Kommentare eingefügt.
 \item Absätze anders gesetzt.
\end{itemize}

Die Arbeiten an dem
Text wurden mit dem Versionskontrollsystem 
Git\footnote{\url{http://de.wikipedia.org/wiki/Git}} protokolliert.
Das Git-Versionsarchiv kann unter dieser URL heruntergeladen werden:
\url{http://www.fkbk.de/git/george_fox_tagebuch/}  

\newpage 

\mainmatter 

% % \picinclude{./vorwort/p_v01.jpg}
George Fox.
Aufzeichnungen und Briefe dez ersten Quäkerz.
« Jn Außwahlüberfetztvon
Marg. stähelin.
Mit einer Einführung von
V Professor 1). Paul Wernle.
M
P.-Iizgiöze czezzelizcbczsss (ier l-reuncso
(HDMI-LCS-:)
S S s Z i si N W 7
I’rin:—i.Oui5—’s:c-iscjZcjoncsssir. Z
Tübing en.
Verlag von J. E. B. Mohr (Paul Siebeck).
1908.


% \picinclude{./vorwort/p_v02.jpg}

Alle Rechte vorbehalten.


% \picinclude{./vorwort/p_v03.jpg}

Jnhaltzverzetchntz.
I: Seite
Zur Einführung. Von Professor 1). Paul Wernle ....... 7
Kap. 1. Erweckung und Krisis bis zum Durchbruch ..... 1
Kap. ll. Erste Versammlungen und Proteste ........ 16
Kap. lll. Der. Tumult in Nottingham. Wachsender Widerstand,
bis szum Gefängnis in Derby .......... 26
Kap. 17. Erlebnisse im Gefängnis zu Derby. Ein ,,Wehe« über
die Stadt Lichfield. Erste Missionsgenossen. Antikirch-
liche Agitation und Kampf gegen die Ranter .... 35
Kap. 7. Christus in uns. Erkenntnis der Quäkerischen Welt-
mission. Das Haus Richter Fells in Swarthmore. Der
Pöbel von Ulverstone. Rechtfertigung vor dem Gericht
in Lancaster ................ 56
Kap. 71. Fox der Hexerei verdächtigt. Falsche Osfenbarungen bei
Freunden. Gefangenschaft in Earlisle ....... 71
Kap. 711. Kämpfe mit schwärmerischen Rantern und zehntengierigen
Priestern. Fox in Wetstone verhaftet und vor Cromwell
geschickt .........   ......... 83
Kap. 7111. Brief an den Papst. Die Studenten von Cambrigde.
Die Quäker in der Bibel. Wachsende Entfremdung von
Cromwell ................. 96
Kap. llc. Angriffe der Jndependenten und Presbyterianer. Ahnungen,
Heilungen, Bekehrungen. Dispute über Taufe und Er-
wählung. Gefangennahme auf Grund angeblicher Ver-
schwörung. Wirken während der Gefangenschaft . . . 104
Kap. 1. Warnung an die Kegelspieler. Naylors Fall. Dis-put
mit Paul Gwin. Besuch bei Eromwell. Herumreisen bei
den gefangenen Freunden. Reise in Wales ..... 120
Kap. I1. Reise nach Schottland. Kampf gegen die Prädestinations-
lehre und Widerstand der schottischen Geistlichkeit . . . 128
Kap. Ill. Erste Jahresversammlung. Warnung an Cromwell vor
der Königskrone. Trostbrief an dessen Tochter. Gesichte
vom Tode Cromwells und der kommenden Reaktion . . 134
Kap. 1111. Ein Gottesgericht. Ermahnung zur Barmherzigkeit bei
Schissbrüchen. Quäkersreundlicher Erlaß des General
Monk. Fox als Königsfeind gefangen und schließlich .
auf Befehl Karls ll. befreit .......... 145


% \picinclude{./vorwort/p_v04.jpg}

17 Jnhaltsverzeichnis.
Kap. 117. Beginn neuer Quäkerverfolgungen bei Anlaß der Ver-
schwörung der Fifthmonarchy -Leute. Des Q-uäkers
John Perots Verirrungen. O-uäker mißhandelt in Neu-
England und Malta .... . ........ 154
Kap. 17. Ein Gottesgericht. Verhaftung wegen .angeblicher Ver-
schwörung und schreckliche Gefangenschaft in Lancaster und .
Scarbro. Dispute im Gefängnis mit Baptisten und andern.
Fox sieht den Brand von London voraus ...... 169
Kap. 171. Einrichtung der Monatsversammlungen. Regelung der
Osuäkerehen. Gründung von Knaben- und Mädchenschulen.
Reformation des Quäkertums .......... 188
Kap. 1711. Reise nach Jrland. Rückkehr, und Heirat mit Margaret
Fell. Jhre abermalige Gefangennahme. Schwere innere
Anfechtungen ................ 201
Kap. )c71l1. Reise nach Amerika. Barbadoes. Jamaika ..... 215
Kap. X11. Arbeit in Nordamerika unter Engländern und Jndianern 224
Kap. IX. Ankunft in Bristol. Zusammentreffen mit William Penn
und andern. Verteidigung der Frauenversammlungen.
Vorgeahnte Gefangenschaft in Worcefter. Brief an den
König über die Grundsätze der Quäker. Krankheit. Be-
freiung. Während der Gefangenschaft verfaßte Schriften 235
Kap. 111. Fox sammelt und ordnet die Bücher und Schriften, die er
geschrieben, und tritt für die Frauenversammlungen ein . 243
Kap. JTI11. Reife nach Holland. Einrichtung der kirchlichen Ordnung
für Holland und Deutschland. Briefwechsel mit Prinzessin
Elisabeth. Reise nach Deutschland bis Oldenburg. Briefe
an verschiedene Behörden von Holland und Deutschland 251
Kap. Jllllll. Rückkehr nach England. Kampf der Ordnungspartei gegen
die unbotmiißigen Quäker. Briefe über Toleranz an den
König von Polen, den Großmogul und andere .... 266
Kap. ZR17. Allerlei Mahn- und Troftschreiben ........ 281
Kap. D17. Zweite Reise nach Holland. Brief an den Herzog von *
Holstein zur Verteidigung des öffentlichen Redens der
Frauen ...... . ........... 285
Kap. II71. Kampf für die Ordnung im Quäkertum. Jakobsll.Amneftie 298
Kap. )T?(71l. Wirken in London unter dem Zeichen der Toleranz . . 303
Kap. II7111. Ahnung kommender Revolutionen. Christus König. Letzte
—— Arbeiten. Krankheit und Tod ......... 306
Anhang. Ein Brief von George Fox nach seinem Tode vorgefunden mit
der Aufschrift: Nicht vor der Zeit zu öffnen ..... 319
Zeittafel ......... . ............. 322
Berichtigungen .................... 324



% \picinclude{./vorwort/p_v05.jpg} 

zur Einführung.
Der Mann, der aus den folgenden Auszeichnungen zu uns
redet, ist schon von seinen Zeitgenossen als ein Rätsel und Ge-
heimnis angestaunt worden. Den einen erschien er als ein Mann,
der gewaltig predigte und nicht wie die Schriftgelehrten, den andern
als ein Verrückter, der selber andere verhexen könne. Man er-
zählte von ihm, er schlafe in keinem Bett, er könne fliegen, er
könne nicht ertrinken und man könne ihn nicht bluten machen,
weil er ein Zauberer sei. Seine magische Wirkung auf die
Zuhörer erklärten sich die einen daraus, daß er Flaschen bei sich
trage und den Leuten daraus zu trinken gebe, damit sie ihm
nachfolgten, andere meinten, er lege den Leuten Bänder um den Arm.
Selbst auf die Tiere gehe eine Kraft von ihm aus: die Hunde
mucksen nicht gegen die Quäker. Den Menschen sehe er den
Teufel im Gesicht geschrieben; ,,durchbohre mich nicht so mit
deinen Augen, wende deine Augen ab von mir«, rief ihm ein
streitsüchtiger Täufer zu. Anders wirkte er auf eine Frau in
Beverly durch eine kurze Ansprache in der Kirche daselbst; sie
erzählte nachher, ein Engel oder ein Geist sei in die Kirche ge-
kommen und habe herrliche Dinge von Gott geredet zur Ver-
wunderung aller Anwesenden, und als er geendet habe, sei er
verschwunden, sie wisse nicht, woher er gekommen noch, wohin er
gegangen sei.
Gin Jahrhundert später hat ihn Voltaire mit Jesus ver-
glichen. Der Vergleich war im Sinne einer Herabsetzung Jesu
gemeint. Voltaire glaubte, daß der englische Kanzelredner Tillotson
unendlich geschmackvoller als Jesus gepredigt habe; um nun die
richtige Analogie für das Bildungs-niveau Jesu zu finden, verglich
er ihn mit einem ungebildeten Schwärmer und Narren aus der
neuern Zeit, mit George Fox.



% \picinclude{./vorwort/p_v06.jpg} 


71 Zur Einführung.
Unter allem, was Voltaire von Jesnö zu sagen weiß, ist doch
dieser Vergleich mit Fox fast daö Beste. Allerdingz wird sich bei
jedem genaueren Zusammenschauen beider die handgreisliche
Uberlegenheit Jesu aufdrängen müssen, aber die Analogien sind
zahlreich und überraschend genug. E8 sind beideö Laien, die auf
Grund einer unmittelbaren inneren Berufung und Erleuchtung
sich getrieben fühlen, eine neue Weise, wie man Gott dienen soll,
zu verkünden, im Gegensatz zu allem, maß gerade von den Frommen
ihrer Zeit als göttlich autzgegeben wurde, heißen sie nun Pharisäer
oder Puritaner. Auch die wunderbaren Begleiterscheinungen haben
sie gemein; Heilkräfte gehen von ihnen aus, selbst auf Sterbende,
die von den Arzten ausgegeben sind, Vorahnungen und Gesichte
scheinen sie über den Zeitverlauf zu erheben, manche Antworten
und Weisungen gibt ihnen direkt der Geist, während sie bei an-
deren Gelegenheiten durch die Selbsteoidenz ihreß gesunden
Menschenoerstandetz überraschen. Nächst den Evangelien ist es
besonderß die Erzählung der Apostelgeschichte, an die man durch
Fox erinnert wird. Das beim Gebet erbebende Versammlungs-
hauö in Jerusalem, die Steinigung und Wiederbelebung des
Paulus in Lystra. die Gesängniöszene in Philippi mit dem
Kerkermeister, die Seefahrt nach Rom mit der Angst der Schiffs-
leute und der göttlichen Zuoersicht deö Apostel?-, alleö daß
wiederholt sich im Leben zdes Fox mit wenig veränderten Um-
ständen. Man glaubt, in die Tage des- UrchristentumZ zurück-
versetzt zu sein, nur mit dem Unterschied, daß, maß dort in der
Regel erst nach Jahrzehnten durch sekundäre Berichterstatter
schriftlich aufgezeichnet wurde, hier in einer eigenhändigen
Niederschrift des Manne?-, der all daß erlebt hat, unö entgegen-
tritt. Man darf hoffen, daß die neuteftamentlichen Gxegeten sich
künftig diesen Laienkommentar zu den Erlebnissen Jesu und der
Apostel nicht entgehen lassen, nicht im Jnteresse einer klein-
gläubigen Apologetik, sondern um ihre Einsicht in das, was in
einer enthusiastischen Zeit bei einem »Mann Gottes'' möglich ist,
zu erweitern und mehr Leben und Farbe der Wirklichkeit in ihre
oft so erstaunlich dürftige Auzlegerphantasie zu bekommen.
Aber nicht nur sür den biblischen Auöleger, für jeden Re-
ligion-Jforscher muß diese Quäkerselbstbiographie von höchster
Anziehungßkrast sein. Die noch junge Wissenschaft der Religions-
psychologie findet hier eineö ihrer allerinstruktiosten Dokumente.


% \picinclude{./vorwort/p_v07.jpg} 
Zur Einsührung. 711
Was unsere heutige Religionsforschung vor den früheren Zeiten
voraus hat, das ist ja eben die Wendung zu den ursprünglichen
religiösen Erlebnissen, während die frühere Forschung allzulange
sich bei der nachträglichen Verarbeitung dieser Erlebnisse in Dog-
men und Systemen aufgehalten hatte. Wir Theologen erkennen
heute, daß es für uns nichts Wichtigeres gibt, als aus die Personen
in der Geschichte zu lauschen, die Gott gehört und gesehen haben,
in denen also, wie der technische Ausdruck heißt, Religion aus
erster Hand uns vorliegt. Die schönste, sruchtbarste Religions-
psychologie der Gegenwart, William James ,,Religiöse Gr-
fahrung in ihrer Mannigsaltigkeitch hat ihren Wert darin, daß
sie den Zeugnissen aus erster Hand möglichst unvoreingenommen
nachgegangen ist. Zu ihnen gehört als eines der merkwürdigsten
eben das ,,Journal« des George Fox.
Man kann hier studieren, «wie die Bekehrung bei einem
solchen Mann Gottes vorgegangen ist. Alle ihre Vorbedingungen
läßt er uns erkennen, die Abstammung, das Milieu, den moralischen
Habitus vor der religiösen Krisis; nur eins, nicht das Umvichtigste,
sehlt in seinen Grinnerungen: die enthusiastische Zeit mit ihren
unerhörten weltgeschichtlichen und kirchlichen Umwälzungen, in die
Weingartens ,,Revolutionskirchen Englands-« immer noch die
klassische Einführung sind. Dann verfolge man den ,,Durchbruch«
selbst mit seiner ganze Jahre ausfüllenden Langsamkeit, dem
Wechsel der Seligkeitsgefühle mit den surchtbarsten anhaltendsten
Depressionen, den vielen pathologischen Begleiterscheinungen bis
zu dem Höhepunkt der Krisis, da Fox 14 Tage lang wie tot
daliegt, so verändert in Aussehen und Gestalt, als ob sein Körper
neu gebildet oder verwandelt wäre. Und dann als Folge das
souveräne Bewußtsein göttlicher Grwählung und Sendung, das
ihn keinen Augenblick in der Seligkeit der Gottesliebe ausruhen
läßt, sondern sofort ihn zu den Brüdern treibt, nicht um sie zu
bekehren, sondern um das schlummernde Bewußtsein des Gottes-
aeistes und seiner Kraft auch in ihnen zu wecken. Sein ganzes
Leben lang geht ihm dies Pneumatische nach, das- die Psychis
aker so gern in ihre Domäne ziehen möchten, während es für
ihn selber der Geist Gottes gewesen ist: plötzliche Stimmen, Ge-
stchte und Gefühle, Heilungen und Bewahrungen der mannig-
sachsten Art. Die Berichte darüber sind erstklassig wegen der er-
staunlichen Schlichtheit und Ausrichtigkeit dieses Berichterstatters,


% \picinclude{./vorwort/p_v08.jpg} 
7lll Zur Einführung.
der seine Wunder so natürlich erzählt, daß wir sie zu verstehen
glauben, und seine Gefichte und ihre Deutung, resp. Erfüllung
so auseinanderhält, daß er uns oft die Mittel der Kritik selber
in die Hand gibt. Nur davor darf vielleicht gewarnt werden,
sich zu einseitig auf die Sammlung dieser außerordentlichen
pneumatischen Erlebnisse zu beschränken. Das königliche Gott-
oertrauen in all den rasenden E-xzessen des englischen Pöbels,
in den schauerlichen Kerkern des damaligen Englands, in See-
sturm und Seeräubergefahr, in den Wäldern und Sümpsen
Nordamerikas breitet das Wunder über sein alltägliches Leben
aus. Dieser Mann scheint aus anderem Stoss zu sein und andere
Kräfte in sich zu tragen, als wir andere Menschen, wir verstehen,
daß man ihn für einen Zauberer hielt, wenn nicht so manche
Krankheiten, Hemmungen und Versuchungen rms wieder daran
erinnern würden, daß auch er ein Mensch gewesen ist.
Aber mir ist, als sehe ich ihn schon lange mit merkbarem Grimm
seine Erregung darüber bemeistern, daß er für uns eine historische
Merkwürdigkeit, ein religionspsychologisches Objekt geworden sei.
Soll das der ganze Wert meines göttlichen Auftrags gewesen
sein, euch interessanten Stoss für eure sogenannte Wissenschaft zu
geben? dazu mein Wahrheitszeugnis, meine Kämpfe und namen-
losen Leiden, meine Sammlung der Kinder Gottes in aller
Welt, damit ihr subtile psychologische und psychiatrische Unter-
suchungen an mir anstellen könnt? An das Licht und an den
Samen Gottes in euch appelliere ich: behandelt den lebendigen
Geist Gottes nicht wie einen Toten!
Welches ist der Platz des George Fox und seiner Quäker
in der Geschichte gewesen? Jndem wir das in Kürze feststellen,
wird deutlich, ob der Mami uns noch heute etwas zu sagen hat.
Weingarten hat einmal treffend das Quäkertum die geist-
liche Nachhut des Enthusiasmus der englischen Revolutionszeit
genannt. Ungeheure Bewegungen sind ihm vorangegangen, auf
denen es fußt, deren Gewinn es voraussetzt. Fox hat gut die
unpolitische neutestamentliche Ethik der Wehrlosigkeit und absoluten
Friedlichkeit predigen, nachdem zuvor der alttestamentliche Pari-
tanismus in einer gewaltigen kriegerischen Erhebung England
zur Vormacht des Protestantismus erhob; ohne das Heldentum
des Schwertes kein Raum für sein stilles, friedliches Heldentum.
Und ebenso hat Fox gut am Beispiel der puritanischen Revo-


% \picinclude{./vorwort/p_v09.jpg} 
Zur Einführung. 11
lutionßkirchen seine Kirchenkritik zu Ende denken, nachdem zuvor
der puritanische Kirchensturm daö prunkvolle Gebäude der anglis
kanischen Staatökirche mit ihrem ss- katholischen Apparat hinweg-
gefegt hatte; ohne die gewaltigen kirchlichen Reformationen und
Reduktionen der Puritaner keine Möglichkeit seine-3 antikirchlichen
Radikalißmuö. Auch der Gnthustaömuö, das Lauschen aus die
Stimmen des gegenwärtigen Gottes-geisteß, ist vor ihm in England
aufgetreten und hat seinen eigenen Enthusiaömuß angesteckt. Man
lernt auö seinen Auszeichnungen die Puritaner fast nur nach ihren
schlechten Seiten kennen; und doch ist der ganze Fox und sein
Quäkertum nur denkbar auf der Grundlage des puritanischen
Befreiung?-kampfß.
ES bleibt darum doch denkwürdig, daß es zu einem so
scharfen Gegensatz zwischen Fox und den Puritanerkirchen der
Preöbyterianer, Jndependenten und Täuser gekommen ist. Was
ist der Grund dieseß Kampfeß?
Die Puritanerkirchen erhohen sich alle auf objektiver, historischer
Grundlage. Daß historische Erlösung?-werk Christi war für sie
alle der Grund der Seligkeit und darum stand der Glaube, das
Bekenntniö, an der Spitze ihreß Christentum?-. Jin Ernstmachen
mit der absoluten Autorität der Bibel suchte jede Gemeinschaft
die andere zu überbieten, jede Kirche wollte reiner nach Gotteß
Wort geordnet sein. So wichtig ihnen auch die Reformation dez
Lebens war, der Nachdruck beim Einzelnen wie in der Offent-
lichkeit ruhte auf einem prononcierten Zur-Schau-stellen des Ve-
kenntnisseß, der Bibel, der kirchlichen Ordnung. EZ ist vielleicht
nie in der Geschichte so viel in der Bibel gelesen, so eifrig gebetet,
so lang und viel gepredigt worden, wie unter der Herrschaft des
Puritanertumß. Und da von der Reformation her die Lehre von
der auch im Ehristenstande bleibenden Sündhaftigkeit sich diesen
Frommen eingeprägt hatte, so lag ez allerdingtz nahe, zu meinen,
daß elementare wie feinere sittliche Gebrechen durch den geistlichen
Habituö, das Bekenntniö, genugsam aufgehoben würden; darin
liegt die Verwandtschaft des Puritanißmuö und jedes Pietißmuß
mit dem Pharisäertum.
An diesem Punkt setzt die Kritik, der Protest, der Gotteßzom
unseres- Quäkerß ein. Ich sehe seine Eigenttimlichkeit gar nicht
in seinem Gnthusiaömuß, sondern in seiner moralischen Gesundheit
und gründlichen Ehrlichkeit. Er scheint mir der ausrichtigste,


% \picinclude{./vorwort/p_v10.jpg} 
K Zur Einsühtung.
lauterste Mann seines Zeitalters zu sein und darin allerdings im
Sinn Thomas Carlyles ein ganzer Held. Er hatte einen
einzigen Sinn, den Sinn für Recht und Unrecht, vorausgesetzt,
daß man das Wort ,,Recht« in seinem weiten Sinn nimmt,
da es alle Liebe und Menschlichkeit in sich schließt. Wenn man
alles übersieht was er in seinem ganzen Leben angreift, wofür
er kämpft, es ist — ein paar Außerlichkeiten, die bei ihm sehr
innerlich gemeint waren, abgerechnet — immer die schlichte
natürliche Moral, für die er eintritt, Recht und Liebe und Treue
(Mt. 23): nicht lügen, nicht Unrecht tun, nicht schwören, sluchen,
stehlen, nicht Gottes Namen mißbrauchen, für alle Menschen den
Frieden und das Gute suchen und friedlich leben mit ihnen. Gr
ist jedesmal empört, wenn er sieht, daß Dienstboten am Lohn
verkürzt werden, daß Wirte ihre Gäste betrunken machen, daß
Steuereinnehmer die Armen bedrücken, daß arme Reisende lieblos
behandelt werden, daß bei einem Schifsbruch die benachbarte Ve-
völkerung sich auf den Raubstürzt, statt sich der Schissbrüchigen
anzunehmen. Aus eigener Anschauung lernte er die Sünden des
englischen Rechts und Strafwesens kennen: die lange Ver-
schleppung der Prozesse, die vorschnelle Füllung von Todes-
urteilen wegen unwichtiger Vergehen in Geldsachen oder das
Vieh betreffend, den Mißbrauch des Eides, die schauerlichen
Gefängnisse Tmit den barbarischen Kerkermeistern, die Ansteckung
der Gefangenen durch die schlechte Gesellschaft, die Unterschlagung
der für die Gefangenen bestimmten Speisen, die gänzliche Ver-
wahrlosung der Familien der Gefangenen während ihrer Ge-
fangenschaft. Wahr ist, daß sich ihm dabei zuweilen Unwichtiges
als wichtig saufdrängte und er aus dem Duzen aller Menschen
und dem Aufbehalten des Hutes selbst vor den Richtern mit
einem Gigensinn bestand, den wir bei Jesus und selbst seinen
Jüngern nicht finden. Aber diese Außerlichkeiten der Konvention
hat er eben anders betrachtet; er verstand nicht und konnte nicht
verstehen, wie Menschen, die doch alle Brüder sind, künstlicher
Formen unter einander bedürfen. Daß er dann in seinem Recht-
sinn schlechterdings keinen Unterschied zwischen Frauen und Mün-
nern, zwischen Engländern und Negern oder Jndianern machen
kann, daß er sie alle schlechtweg als Menschen nimmt mit dem
vollen Anspruch aus menschliche Behandlung, braucht kaum hin-
zugefügt zu werden. Gr nimmt das alle-3 auch gar nicht als


% \picinclude{./vorwort/p_v11.jpg} 
Zur Einführung. Il
christlich in Anspruch; das Licht, das einen jeden Menschen er
leuchtet, das Gewissen, wird in allen Menschen dasselbe Recht
und Unrecht erkennen müssen.
Aber da drängte sich ihm nun die entsetzliche Frage auf;
wo ist bei den Christen, bei diesen Puritanern und Frommen
(Bekennern), diese Wahrhaftigkeit, Menschlichkeit und Liebe? Als
er selbst noch Trost bei einzelnen ihrer Führer suchte, erlebte er
nichts als Enttäuschungen; der eine schwatzte seine Leiden und
Bekümmernisse den Dienstboten aus, ein anderer geriet mitten
im Gespräch, als Fox aus Versehen auf den Rand eines Garten-
beetes trat, in solche Wut, als ob sein Haus in Flammen stünde.
Sie besaßen das gar nicht, was sie bekannten. Und später machte
er die Beobachtung, daß ihm und seinen ,,Freunden« aus diesen
Puritanerkirchen die roheste, gemeinste Verfolgung erwuchs, die
sich in Amerika, in den puritanischen Musterländern, bis zur
Hinrichtrmg einzelner Quäker steigerte. Diese Kirchen, die eben
aus jahrzehntelanger Verfolgungszeit zur Freiheit gelangt waren,
zeigten sich genau so tmduldsam, iso heerschsüchtig- so pfäfsisch
wie ihre früheren Verfolger, Katholiken und Anglikaner. Man
kann dies Urteil als einseitig beanstanden, als einen Ausf-luß des
sektenhaften Richtgeistes, der sich der Quäker bemächtigte; so wie
es hier aussieht, haben sich Licht und Finsternis nicht verteilt,
man denke nur an Richard Baxter und an Oliver Cromwell.
Aber daß Fox auch guten Grund zu dieser Beurteilung hatte, wer
wird das leugnen? Eine so ausgesprochen sromme Bewegung
wie der Puritanistnus fordert den strengsten Maßstab heraus.
Von da aus stellte sich ihm das scharfe Entweder — Oder
einer doppelten Frömmigkeit aus: die der frommen Formen und
Worte, und die der Kraft. Auf der einen Seite standen ihm alle
vorhandenen Religionen, Puritaner und Anglikaner und Katholiken
allesamt, deren Unterschiede doch nur in den Formen bestehen,
aus der andern Seite das, was Gott will, wozu Jesus in die
Welt gekommen ist. Fromm sein, das heißt die Kraft Gottes
besitzen, von ihr allein beherrscht werden, dies, und dies allein.
Gr nannte diese Kraft den Geist oder mit Vorliebe den Samen
Gottes, glaubte kühn, daß in jedem Menschen dieser Same oet-
borgen sei, eben als das Zeugnis seines Gewissens, und daß der
ein Christ sei, in dem der Same durch Gottes Wunder lebendig
und mächtig über alles geworden sei. Nen waren diese Gedanken


% \picinclude{./vorwort/p_v12.jpg} 
Xll Zur Einführung.
nicht, wir finden sie in der Resormationszeit besonders bei
Sebastian Franck und später bei Jakob Böhme, der in der
Zeit des Fox auch in England englisch gelesen wurde. E3
kommt aber nicht aus die Priorität an, sondern darauf, daß sie
hier bei Fox mehr als Gedanken waren, daß sie die Lebenskraft
einer ganzen Gemeinschaft wurden. Und sie traten hier nicht wie
in Deutschland zu einem toten Kirchentum und einer starren
Orthvdoxie in Gegensatz, sondern zku den lebendigsten, jugend-
srischesten Puritanergemeinschaften. tzhnen galt der für fromme
Ohren wahrhaft entsetzliche Kriegsrus: nicht Bibel, nicht Ve—
kenntnis, nicht Kirchen, sondern allein der Geist, der Gott, der in
uns selber als Lehrer tmd als Kraft lebendig ist!
Es war eine höchst gefährliche Losung, die Fox damit auf-
nahm, die Losung aller Schwarmgeister und Fanatiker, aus der
von Jahrhundert zu Jahrhundert die unheinilichsten und grau-
sigsten Exzesse der Religionsgeschichte geboren worden sind. Wie
leicht verbergen sich dunkles Triebleben und oerworrene mensch-
liche Einbildung unter dem hohen Titel des Geistes Gottes! Als
Fox auftrat, wimmelte es in England von Enthusiasten aller
Art, entfesselt durch die allgemeine Emanzipation des Revolutions-
zeitalters. Sie traten mit Träumen, Gesichten und Stimmen aus,
gaben sich selbst für Christus aus und erklärten, sündlos zu sein.
Alle Geschichte war ihnen bloßes Symbol ihrer eigenen Erlebnisse,
man hörte geradezu die Leugnung, daß Jesu Tod eine geschichtliche
Tatsache sei; sein Leiden sei ja in uns. Diese Gott- und Christus-
trunkenen Schwärmer wurden von den Kirchlichen Runter,
d. h. Prahler, genannt. Und bevor der Quäkername sich allgemein
verbreitete, sind auch Fox und seine ersten »Freunde« als Ranter
angeschrien und verfolgt worden. Obschon Fox von Anfang an
dagegen protestierte, es ist Tatsache, daß die ersten Quäker und
die Ranter sich wie ein Haar vom andern unterschieden, daß
einige der hervorragendsten Quäker den Rantergeist nicht los
geworden sind. Der messianische Einzug des James Naylor
in Bristol ist ein echtes Ranterstück. Und äußerlich betrachtet, i
wer will die Ossenbarungen des Fox von den Eingebungen der
Runter unterscheiden?
Aber während diese Ranter spurlos und namenlos in der
Geschichte der religiösen Schwtirmerei wieder untergegangen sind,
bilden die »Freunde« bis heut eine blühende religiöse Gemein-


% \picinclude{./vorwort/p_v13.jpg} 
Zur Einsührung. Xlll
schaft von charakteristischer Eigenart. Daß kommt daher, daß sie
die Niichternen, die moralisch Gesunden in dem enthusiastischen
Wirbelsturm waren. Der Gnthusiaßmus ist bei ihnen nur die
Form, die Grstlingöform, in der ihre neue moralische Kraft sich
manifestiert, nicht anders, als eß beim Urchristentum der Fall war.
Man erkennt an diesem Enthusiasmus die absolute Energie, mit
der sie von ihrer Wahrheit erfaßt waren, mochte die ganze Welt
widersprechen. Sobald man aber aus den Jnhalt achtet, stößt man
aus jene schlichte Menschlichkeit, das- Einfachste und Nüchternste, was
jemalß Grweckungßprediger gefordert haben. Fox ist kein Schwärmer
gewesen, obschon ihn die von ihm erkannte Wahrheit beherrschte
wie eine tiefe Schwärmerei. In keinem Augenblick seineß Lebenß
hat er seinen klaren Sinn für Recht und Unrecht, gut und böse
verloren. Daß rettete ihn an schwindelnden Abgrtinden vorbei
und durch alle Verlorkungen des religiösen Wahnsinnß, dem manche
seiner Genossen erlagen. Er war vielleicht nicht immer Herr über
die von ihm entsachte Bewegung, wie ihm überhaupt das Herrscher-
talent abging, aber er war immer Herr über sich selbst. Daß
ist der eine Grund, daß daß Geistprinzip ihm nichts geschadet
hat: seine gründliche moralische Festigkeit und Gesundheit. Dazu
kommt aber, daß der Gegensatz zum historischen Christentum nicht
von ferne so tief war, wie ihn die Kampslosung deß Fox: ,,nicht
die Bibel, sondern der Geist« könnte erscheinen lassen. Kein
Mensch seiner Zeit hat mehr in seiner Bibel und auö seiner
Bibel gelebt alß eben Fox; seine individuelle, kernige Sprache ist
ihm sogar durch den biblischen Dialekt ganz abhanden gekommen.
Selbst in seinen eigenen Liebling;-’gedanken steht er aus dem festen
Grund dez Reformationöeoangeliuniß von der den Menschen
durch Gottes Gnade geschenkten und durch gar kein Eigenwerk
von ferne zu oerdienenden Erlösung. Jn allem, wat; er redet und
tut, beruft er sich aus das- Wort Jesu und der Apostel, und daß
cthische Jdeal, daß er darauß ableitet, berührt sich ausß engste
mit dem täuferischen Ideal der Friedsamkeit und Wehrlosigkeit,
von dem die Täufer selbst sich in den Reoolutionökriegen hatten
abdräugen lassen. So oerleugnet er nirgends die Kontinuität
mit dem Christentum der von ihm so radikal verworsenen Kirchen;
er ist auch schon viel zu bescheiden und ehrlich, um den Anspruch
zu erheben, autz dem Geist Gotteö heraus ein neuer Religionß-
stifter zu sein. Was er eigentlich will, ist nur die radikale Re-


% \picinclude{./vorwort/p_v14.jpg} 
W1 Zur Einführung.
formation in Tat und Leben, das Emftmachen mit Jesu Wort
und Geist in rücksichtslosem Kampf mit allem, was Welt und
Tradition dariiberlegten. Stellt er also Geist und Bibel in so
scharfen Gegensatz, so meint er letztlich zwei Dinge: das Recht
des Laienverständnisses der Bibel im Gegensatz zum theologischen
Privileg — nicht Gelehrsamkeit, sondern allein Frömmigkeit kann
Gott recht verstehen — und die Notwendigkeit, die Kraft der
biblischen Religion im Leben zu beweisen, statt im Besitz und
Lesen des Bibelbuchs. So sbetrachtet, ist er keine Jnstanz, die
sich gegen den Wert der Bibel und des historischen Christen-
tums anführen läßt, sondern ein Zeugnis der Lebenskraft,
welche aus der Berührung der Geschichte — Gottes in der Ge-
schichte — mit einem aufrichtigen, tapferen Menschenherzen quillt.
So ist auch, was wir heute brauchen, keine neue Offenbarung
verborgener Seiten Gottes, sondern ein ganz anderes Grnstmachen
mit der uns in der Geschichte geschenkten Erkenntnis Gottes und
unserer Pflicht, als das Namenchristentum es kennt.
Damit scheint mir das Wesentliche zum Verständnis des Fox
angedeutet zu sein. Höchstens fehlt noch ein ganz auffallender
Punkt: sein massiver Vergeltungsglaube. Jn seiner sittlichen
Forderung iiberschritt er bewußt die alttestamentliche Stufe und
lebte die Ethik der Bergpredigt wie wenig Christen vor ihm.
Aber in seinem Vergeltungsglauben, der ihn mit größtem Eifer
und innerer Zufriedenheit die jeweiligen Strafen der Quäker-
Verfolger, besonders die auffallenden, plötzlichen Gottesgerichte
notieren ließ, kommt er uns seltsam alttestamentlich zurückgeblieben
oor. Für ihn war das die notwendige Ergänzung seines extremen
Spiritualismus. Der Gott, der ihn innerlich bewegte, war der-
selbe, der die Vorfälle des äußeren Lebens in seiner Hand hielt
und durch sichtbare Strafen oder Segnungen kundgab, auf welcher
Seite das Recht war. Genau so haben es Cromwell und die
Jndependenten geglaubt. Deshalb ist es Fox doch keinen Augen-
blick eingefallen, etwa nun auch seine und der ,,Freunde« Leiden
als Strafen Gottes aufzunehmen und nach der sie verdienenden
menschlichen Verschuldung zu fragen. Die entsetzlichsten Miß-
handlungen und die empörendsten Ungerechtigkeiten der Justiz
nahm er als Kind Gottes gelassen und sogar fröhlich auf, ohne
an eine Strafe Gottes dabei zudenken. Das sind Jnkonsequenzen.
Man kann es doch auch wohl verstehen, daß ein Mann mit


% \picinclude{./vorwort/p_v15.jpg} 
Zur Einführung. 17
solchem Rechtsfmn so fest sich an den Glauben an eine moralische
Weltordnung auch im Außern geklammert hat.
Die Geschichte des Fox und der Freunde, die diese Auf-
zeichnungen uns vorführen, zerfällt in zwei deutlich unterschiedene
Perioden. Zuerst die Sturm und Drangzeit, die Periode des
extremen Enthusiasmus und der extremen antikirchlichen Agitation.
Es ist die Zeit der ersten Liebe, des größten Heroismus, aber
auch einer rohen Unreife, die sich später in den langen Leidens-
jahren korrigiert. Fox ist damals ein Kirchenstürmer der wildesten
Art gewesen, er ging darauf aus, die Leute aus den Puritaner-
kirchen und von den Puritanerpfarrern weg zu reißen mit allen
Mitteln der Agitation. Wenn sich dann der ganze Haß der
Pfarrer und ihrer Anhänger in der brutalsten Weise über ihm
entlud, so ist ihm das sicher nicht unnerschuldet begegnet, wenn
er auch mit Recht darin eine wunderliche Manifestation des puri-
tanischen Christentums, dieser Gxtrafrömmigkeit, sah. Es bestand
für ihn selbst vorübergehend die Gefahr, daß das Nein in der
Agitation das Ja, das Recht und die Liebe, übertöne. Es bildete
sich damals ein Richtgeist unter den Freunden aus, der sie in
allen andern Gemeinschaften Babylon und den Antichrist erblicken
ließ, während sie allein Jerusalem vertraten. Mit Cromwells
Protektorat standen sie auf gespanntestem Fuß; es ist nur mensch-
lich, daß dem toleranten Mann ihnen gegenüber wiederholt die
Geduld ausging. Dazu die gräßliche Schwärmerei des James
Naylor, eines der Führer der Bewegung, und auf der andern
Seite die Phantastik der Weltmission, die gleich den Papst in
Rom und den Sultan in Konstantinopel zu bekehren strebte und
Schreiben an alle Potentaten der Welt ergehen ließ, damit sie
dem ,,Kommen des Herrn« nichts in den Weg stellten. Man
kann diese Zeit in ihrer Weise mit der Sichtungszeit der
Brüdergemeinde vergleichen, es ist fast ein Wunder, daß die
Gemeinschaft diese Schwärmerei überstand. Nun, sie haben auch
dafür gelitten, und unter den Blättern aus der Geschichte religiösen
Heldentums ragen zweifellos die Erzählungen des Fox aus dieser
ersten Zeit immer hervor.
Die zweite Periode, welche ungefähr mit der Restauration
des Königtums einsetzt, ist dann die Periode der Ernüchterung
und der Organisation. Erst jetzt beginnt im vollen Sinn ihr
unschuldiges Leiden, nachdem das frühere so vielfach durch eigene


% \picinclude{./vorwort/p_v16.jpg} 
171 Zur Einführung.
Schuld provoziert worden war. Nicht nur hielten sie sich inder
politischen Neuordnung durchaus neutral, ja grundsätzlich un-
politisch, sodaß auch nicht der Schein einer Gefahr für den Staat
bestand, ihre Kirchenstürmerei hatten sie längst in dem Maß auf-
gegeben, als sie sich zu selbständigen Gemeinschaften mit eigenem
,,Gottesdienste« zusammenschloss en; sie taten schlechterdings niemand
etwas zu leide. Dennoch hat die Verfolgung seitens des neu-
befestigten anglikanischen Staatskirchentums gerade sie mit beson-
derer Härte getroffen, z. T. wegen ihrer absoluten Eidverweigerung,
die ihnen schlimm aus-gedeutet werden konnte, und hat dadurch
dazu beigetragen, daß auch der letzte revolutionäre Gedanke sich
verlor, und gar nichts anderes übrig blieb als die gänzliche
Wehrlosigkeit und Leidsamkeit, durch die einst das alte Täufertum
sich ein bleibendes Andenken in der Geschichte erwarb. Der
stürmische Enthusiasmus war verflogen, aber der stille Enthusias-
mus, mit dem das Gotteskind alles Leid, das Menschen ihm
antun können, friedlich, unverbittert, ja selig im Grunde, hinnimmt,
blieb als die Frucht der großen Zeit. Zugleich aber ist diese
Leidenszeit die Zeit des Bauens, der Organisation. Aus der
kirchenstürmerischen Bewegung geht selbst eine neue Kirche — Fox
selbst braucht den Ausdruck Kirche dafür — hervor, und das
Erstaunlichste ist, daß nicht etwa Epigonen im Gegensatz zur
ursprünglichen Tendenz des Stifters diese Verkirchlichung durch-
setzen, sondern daß der Begründer des quäkerischen Enthusiasmus
auch der kirchliche Organisator ist. Zuerst hatten sich die Jahres-
versammlung und die Vierteljahrsversammlungen eingebürgert zu
wichtigeren Beratungen, während man am ,,Ersten Tag'' (dem
Sonntag) zu zwanglosen Aussprachen aus der Eingebung des
Geistes zusammenkam. Das war ein Anfang von Ordnung, aber
dem einzelnen blieb eine ungebundene Freiheit während des
ganzen Vierteljahrs. Da tat Fox im Jahr 1666 den entschei-
denden Schritt zur geschlossenen kirchlichen Organisation mit der
Einrichtung von Monatsversammlungen sowohl fiir die Männer
als für die Frauen, vornehmlich zur Durchführung der Kirchen-
zucht gegen unordentliche Mitglieder, auch zur Regelung der
Quäkerehen 2c. Er hat damals ganz England im Jnteresse
dieser Organisation bereist und alle Quäker im Ausland, auf
dem Kontinent, in Jrland, Schottland und Amerika, zur Nach-
ahmung dieser Organisation aufgefordert. Eine große Reformation


% \picinclude{./vorwort/p_v17.jpg} 
Zur Einführung. 1711
des Quäkertumß leitet er selbst von dieser neuen Verfassung her.
Allein es fehlte nicht an ganz energischem Widerstand aus den
Kreisen der »Freunde«, welche die christliche Freiheit durch eine
neue Menschensatzung bedroht glaubten, in der Kirchenzucht ein
uneoangelischeß Richten aufkommen sahen und speziell von der
den Frauen in dieser Organisation gewährten Stellung nichtö
wissen wollten. EZ ist kein Zufall, daß in diesem Zusammenhang
wieder von den Rantern die Rede ist. Der alte Rantergeist, der
extreme Subjektiviömuß und Jndividualiömuß, sah sich durch diese
kirchliche Organisation in-8 Herz getroffen. GS berührt in der
. Tat seltsam, wenn man den alten Fox jetzt den göttlichen Ursprung
dieser »eoangelischen Ordnungen« verfechten sieht; war das noch
der Prophet und Enthusiast von ehemal-Z? Und doch ist er nicht
von sich abgefallen, als er für seine Gemeinschaft die für ihren
Bestand notwendigen Formen schuf. Jndioidualist im extremen
Sinn war er nie gewesen, sondern von Anfang an Gemeinschaftß-
mann, und darum Mann der Liebe und Ordnung. Er hat einfach
gelernt, was- jeder gereifte Mensch einmal lernen muß, daß der
Geist zerfließt und zerflattert, wenn er nicht durch Organisationen
sich einen dauerhaften Körper geben kann. Waß müßten wir
heute vom Quäkertum ohne diese kirchliche »Reformation«! Zudem
ist die Quäkerorganisation unter allen mir bekannten kirchlichen
Ordnungen die freieste, formloseste geblieben. Gar kein Glaubens-
bekenntnis, und daß Grftaunlichste — keine Sakramente! Der
Gottesdienst so, daß man zusammenkommt, auf den Geist wartet,
und wenn einmal der Geist niemand zum Reden treibt, sich nach
stiller Versammlung die Hand gibt und friedlich autzeinandergeht.
Bekenntnis und Verfassung und Kultuö sind hier nichtß, das
Leben ist allez, und will und soll sein, wa-3 es- von Anfang war:
Recht und Liebe und Treue, schlichte Menschlichkeit.
Man soll eß nie vergessen, daß —— nächst dem einzelnen
Roger Williams- —— die Quäker die ersten waren, die mit einer
unterschiedölosen religiösen Toleranz praktisch ernst machten, nicht
aus Jndifferenz sondern aus Glauben. Andere, wie die Puritaner,
hatten nur, solang sie selbst verfolgt waren, Toleranz begehrt,
und, zur Herrschaft gelangt, sie schmählich verleugnet. In Penn-
stlvanien hat tatsächlich jeder seineö Glaubenö frei gelebt. Hier
bei den Quäkern zuerst ist den Frauen die volle kirchliche Gleich-
stellung mit dem Mann gegeben worden; ez gibt kein doppeltes


% \picinclude{./vorwort/p_v18.jpg} 
Illlll Zur Einführung.
Recht vor Gott. Von Fox erging während seiner amerikanischen
Reise die Mahnung, die Negersklaven mild und freundlich zu
behandeln und sie frei zu lassen, nachdem sie einige Jahre gedient.
Die Quäker sind damit im Kampf gegen die Sklaverei Voran-
gegangen. Und wenn mehr alz ein Jahrhundert später Elisabeth
Fry alz Resomiatorin deö Gesängnißwesenß England und den
Kontinent durchzogen hat, so sehen wir auß den Aufzeichnungen
des Fox, daß sie nur seine Aufgabe zu Ende führte. EZ gibt kein
großes Werk der Menschlichkeit und Barmherzigkeit, an dem nicht
die Quäker beteiligt sind, und daß nicht letztlich in dem wurzelt,
waz Fox ale die Kraft des Samenö Gotteö erkannte.
Daß ist doch mehr als eine historische oder religionöpsycho-
logische Merkwiirdigkeit, ee ist die beste Kraft dez Evangelium-3,
es- ist Jesuß selbst, der hier wieder einmal die Umhiillungeu, in
denen ihn menschliche Schwachheit und Kleinglaube konferoieren,
konseroieren mtiss en, frei herau?-tritt, um die Menschen einen ganzen
großen Schritt auswärtz zu führen in der Richtung auf sein
Gotteßreich. Wir in der Schweiz und in Deutschland sind nicht
Quäker und werden auch nach unserer Eigenart keine Quäker-
gemeinschaften bilden, aber wir sind Jünger dez Evangeliumß
nur dann, wenn wir ganz allein daß wollen, wa-3* die Quäker
wollten, ein Leben in der Kraft Gotteß statt in den Formen
und Worten, und die Richtung der Kraft: Recht und Liebe
und Treue, Menschlichkeit. Viele haben gemeint, daß die Quäker
doch keine rechten Christen seien, weil sie gar keine Sakra-
mente haben. Aber dem steht daß Wort Jesu entgegen: an
den Früchten sollt ihr sie erkennen! Hätte eine unserer Kirchen
solche Früchte wie die Quäker!
Daß nach dem Tode dee George Fox von William Penn
heraußgegebene Journal, auß dem im folgenden eine Auswahl
gegeben wird, ist nicht, wie der Titel vermuten ließe, ein wirk-
liches Tagebuch, sondern eine zusammenhängend geschriebene
Selbstbiographie, die allerdingß tagebuchartige Notizen voraußsetzt.
GZ ist ein Werk deß Alter;-’, geschrieben in der Restaurationözeit,
wie ich vermute, etwa im Jahre 1677, als Fox sich überhaupt
an die Sammlung und Ordnung seiner älteren Dokumente machte,
und dann in den folgenden Jahren noch ergänzt. Für seine
relativ einheitliche Abfassung spricht einmal der durchweg einheit-
liche Stil, der gar keine Wandlungen aufweist, sodann die Hinzu-


% \picinclude{./vorwort/p_v19.jpg} 
Zur Einführung. 111
fügung einer ganzen Reihe späterer Notizen bei viel früheren
Jahrgängen, vor allem der religiöse und kirchliche Geist des
Ganzen. Die Anerkennung Karl Stuarts als des von Gott be-
stimmten rechtmäßigen Königs beherrscht das ganze Buch, von
dem ursprünglichen Enthusiasmus ist wohl die Erinnerung be-
halten, aber aus ihm heraus geschrieben ist keine Seite, ja es
läßt sich eine gewisse apologetische Absicht nicht verkenneu, auch
das Quäkertum der Vergangenheit als politisch harmlos und in
jeder Weise ungefährlich hinzustellen. Die Verirrungen einzelner
Quäker, z. B. des James Naylor sind behutsam angedeutet, aber
man gewinnt keinen Eindruck, wie kritisch sie damals siir das
Quäkertum gewesen sind. Die Vermengung der Quäker mit den
Rantern wird von Anfang an säuberlich abgewehrt, die wirkliche
Entstehung des Quäkernamens aus den krankhasten Konvulsionen
der ,,Freunde« (Zitterer) wird verdeckt. Etwas Unwahres möchte ich
in dieser Darstellung nicht sehen, wohl aber da und dort eine
unwillkiirliche Verschiebung, veranlaßt durch die eigene Ernüchterung
und den Zwang, sich der Anklagen und Verleumdungen zu er-
wehren. Jch glaube, daß das, was erzählt ist, immer historische
Wahrheit ist; aber ob immer alles, zumal aus der Sturm- und
Drangzeit, erzählt ist, was man später noch wußte, muß für uns
dahin gestellt bleiben; über James Naylor wußte Fox jedenfalls
noch mehr. Aus dem Gedächtnis aber kann er diesen unermeßlich
reichen, im einzelnen so detaillierten Stoss nicht niedergeschrieben
haben. Eine eigene Aufzeichnung der Stationen muß ihm vor-
gelegen haben und zugleich wohl kurze Notizen über die merk-
wiirdigsten Erlebnisse an den einzelnen Orten. Leider hatte er
gar kein Jnteresse an der Chronologie; Jahreszahlen finden sich
im ganzen Buch nur in den mitgeteilten Dokumenten, eigenen
Briefen, Hastbefehlen 2c. Dazwischen erwähnt er jedoch eine
Reihe weltgeschichtlicher Begebenheiten, die der Quäkergeschichte
doch ein gewisses chronologisches Gerippe geben und deshalb in
der Übersetzung mit Fleiß gesammelt sind.
Die Beschränkung aus eine Auswahl ergab sich uns statt
einer ganzen Ubersetzung mit Notwendigkeit, weil das Ganze so
gut wie keine Leser gesunden hätte. Nicht nur des Umfangs
wegen. Die Erzählung wiederholt sich unendlich, und die ein-
gelegten Briefe sind von einer ermiidenden Breite und Monotonie.
Fox ist doch sicher ursprünglich ein origineller Laie gewesen mit



% \picinclude{./vorwort/p_v20.jpg}
II Zur Einführung.
realistischem Ausdruck und oft ungewöhnlichen: Mutterwitz.
Allein die Notwendigkeit, 40 Jahre lang unaufhörlich reden zu
müssen, und eigentlich doch immer dasselbe, hat seine Originalität
stark vermindert und ihm in Sprache und Schrift die Monotonie
verliehen, die im allgemeinen den Gemeinschaft-'spredigern nach-
zugehen pflegt. Einen wesentlichen Vorzug wird man ihm gerade
deöhalb doch zugestehen müssen: er erzählt durchaus schlicht, un-
gesucht, und daz gibt der Sache, die er erzählt, eine um so ge-
waltigere Wirkung, eZ steckt ’auch nicht ein Schimmer Eitelkeit
darin. Die Ubersetzerin — es ist die Tochter des Basler Kirchen-
historikertz und Zwinglibiographen Rud. Stähelin — hat sich
bemüht, so schmuckloö schlicht zu erzählen wie er selbst und die
Sache durch sich selbst reden zu lassen. Aufgenommen haben wir
alles, waß unß sür die Charakteristik deö Fox und die Geschichte
des Quäkertumö wesentlich schien, speziell auch möglichst alle
religiösen Merkwürdigkeiten und die zerstreuten welt- und kirchen-
geschichtlichen Notizen, welche die Verbindung mit der allgemeinen
Geschichte ermöglichen. In dieser beschränkten Außwahl, die im
Grunde doch alletz Wesentliche wiedergibt, wird, wie wir nicht —
zweifeln, die Lektüre unsereö Bucheß Vielen Genuß bringen, und,
hoffen wir, etwaö Besseres alß Genuß. Carlyle hat einmal im
Sartor Resartuz das merkwürdigste Greigniö der neuem Geschichte
den George Fox genannt, der sich einen Anzug von Leder machte.
Wer diese Aufzeichnungen lesen wird, der wird seine Paradoxsie
verstehen. P. Wernle.


% \picinclude{./000-009/p_s001.jpg}
%%%%%%%%%%%%%%%%%%% Kapitel 1. %%%%%%%%%%%%%%%%%%%%%%%%%%%%%%
\chapter[Erweckung und Krisis]{Erweckung und Krisis bis zum Durchbruch.}


% \begin{center}
% \textbf{Erweckung und Krisiz bis zum Durchbruch.}
% \end{center}


Auf das Jedermann wisse, was der Herr an mir getan, und
sehe, wie Er mich durch mancherlei Prüfungen, Versuchungen und
Trübsal führte, um mich für das Werk, für das; Er mich bestimmt 
hatte, vorzubereiten und auszurüsten, und dadurch getrieben 
werde, seine unendliche Güte und Weisheit anzubeten und
zu preisen — so will ich kurz berichten, wie es in meiner Jugend
um mich stand, und wie das Werk des Herrn in mir angefangen
und fortgesetzt wurde seit meiner Kindheit.

\begin{floatingfigure}[3]{4cm}
\includegraphics[width=0.20\textwidth]{./pics/Fox-George-LOC.png}
\label{bild:gfox} 
\end{floatingfigure}




Ich wurde geboren im Monat den man Juli nennt\footnote{Fox 
verwarf die üblichen Monatsbezeichnungen als heidnisch.} 
1624\jahr{1624}, zu Drayton in-the-Clay\ort{Drayton in-the-Clay}, 
in Leicestershire\ort{Leicestershire}. Mein Vater hies
Christoph Fox\person{Fox, Christoph}; er war Weber von Beruf, 
ein ehrbarer Mann,
und es war ein \zitat{Same von Gott} in ihm. Die Nachbarn
nannten ihn: den \zitat{gerechten Chrtster}. Meine Mutter war eine
rechtschaffene Frau; ihr Mädchenname war Mary Lago, aus der
Familie der Lago und aus dem Geschlecht der Märtyrer.

In meiner frühesten Kindheit war ich so ernsten und gesetzten
Gemütes, wie es bei Kindern selten ist, so das, wenn ich Erwachsene 
leichtfertig und ausgelassen mit einander tun sah, ich
einen Abscheu davor in meinem Herzen verspürte und zu mir
sagte: \zitat{Wenn ich einmal ein Mann sein werde, sicherlich werde
ich nicht so leichtfertig tun.}

Als ich elf Jahre alt war, wuste ich schon was rein und
recht ist; denn ich war als Kind gelehrt worden, wie man rein
bleibt. Der Herr lehrte mich, treu zu sein in allen Dingen, sowohl
innerlich gegen Gott als äuserlich gegen die Menschen; und das
ich mich in allen Dingen an \zitat{ja} und \zitat{nein} 
halten solle; nicht
wie die Kinder der Welt, die ihren Mund voll List und gleisnerischer
Worte haben, sondern meine Worte sollen: wenig sein, \zitat{lieblich
% \picinclude{./000-009/p_s002.jpg}
und mit Salz gewürzet} (Col. 4:6\bibel{Col. 04:06@Col. 4:6}); 
und das ich nicht essen\index{Essen zu Spas}
und trinken solle, um mich wollüstig zu machen, sondern um der
Gesundheit willen, jeder Ding dazu gebrauchend, wozu es 
bestimmt ist, zur Ehre dessen, der alles geschaffen hat [...]

Als ich dann heran wuchs, wollten meine Angehörigen einen
Priester\footnote{Fox bezeichnet mit priest die ordinierteu Geistlichen.} 
aus mir machen. Aber andere rieten zu andrem; so
kam ich zu einem, der seines Zeichens ein Lederhändler war, aber
mit Wolle handelte und Vieh züchtete und verkaufte; und es ging
mancherlei durch meine Hände. Während ich bei ihm war,
war er gesegnet; aber nachdem ich ihn Verlassen, ging es ihm
schlecht und er geriet in Verfall. Während dieser ganzen Zeit
tat ich weder gegen einen Mann noch gegen eine Frau etwas
Unrechtes; denn die Kraft des Herrn war mit mir und bewahrte
mich. Während ich in diesem Dienste stand, gebrauchte ich im
Verkehr das Wort \zitat{wahrlich}, und es war eine übliche Redensart
bei meinen Bekannten: wenn George sagt \zitat{wahrlich}, so kann
ihn nichts umstimmen. Wenn die Buben oder rohe Leute über
mich lachten, kümmerte ich mich nicht um sie, sondern ging meiner
Wege; aber gewöhnlich hatten mich die Leute gern wegen meiner
Geradheit und Ehrlichkeit.

Als ich, noch nicht ganz neunzehnjährig, in Geschäften an
einem Jahrmarkt war, kam mein Vetter, namens 
Bradford\person{Vetter Bradford}, ein
\zitat{Frommer} (Professor) und mit ihm noch ein anderer
\zitat{Frommer} und forderten mich auf, mit ihnen 
einen Krug Bier zu trinken,
und da ich durstig war, ging ich mit ihnen hinein; denn ich
liebte jeden, der Sinn für das Gute hatte und den Herrn
suchte. Als jeder ein Glas getrunken hatte, fingen sie an, sich
zuzutrinken und verlangten noch mehr, indem sie ausmachten,
das der, welcher nicht trinken würde, alles bezahlen sollte.
\index{Trinkspiele} Es betrübte mich, das jemand, 
der sich für religiös ausgab, solches
tat; sie taten mir sehr weh, denn es war mir dergleichen noch
nie vorgekommen bei keiner Art von Menschen; darum stand ich
auf um zu gehen, indem ich meine Hand in die Tasche steckte,
einen Groschen vor sie aus den Tisch legte und sagte: \zitat{wenn es
so ist, will ich euch Verlassen.} So kehrte ich nach Hause zurück,
aber ich ging in jener Nacht nicht zu Bett, denn ich konnte nicht
schlafen; bald ging ich im Zimmer auf und ab, bald betete und
% \picinclude{./000-009/p_s003.jpg}
schrie ich zum Herrn, welcher also zu mir redete: \zitat{Du siehst, wie
junge Leute zusammengehen in Eitelkeit und alte Leute in die
Erde. Du musst dich von ihnen abwenden und dich von ihnen,
den jungen wie den alten, fern halten und ihnen allen ein
Fremdling werden.}

Darauf, am 9. Tage des 7. Monats 1643,\index{Jahr!1643} 
verließ ich nach
Gottes Befehl meine Verwandtschaft und brach allen Umgang und
alle Kameradschaft mit jung und alt ab. Ich begab mich nach
Lutterworth\ort{Lutterworth}, wo ich einige Zeit blieb 
und von da ging ich nach
Northampton\ort{Northampton}, wo ich mich ebenfalls 
aufhielt; darauf nach Newport\ort{Newport}
Pagnell, von wo ich nach einiger Zeit weiter nach Barnet
ging, im 4. Monat 1644\jahr{1644}. Als ich nun so das Land durchzog,
wurden die \zitat{Frommen} (\textit{professors}) 
auf mich aufmerksam und
wollten mich kennen lernen. Aber ich mied sie; denn ich spürte,
das sie nicht besaßen, was sie bekannten (\textit{professed}). Während
der Zeit, da ich in Barnet war, kam eine 
große Anfechtung\index{Anfechtung} zu
verzweifeln über mich. Ich sah, wie Christus versucht worden
war, und war in großer Not; bald ging ich nicht aus meinem
Zimmer, und bald wanderte ich einsam durch die Fluren, um auf
den Herrn zu warten.

Ich fragte mich, warum mir solches widerfahren müsse? Ich
prüfte mich und sagte zu mir selber: \zitat{War ich je zuvor so gewesen?} 
Ich dachte, ich hätte mich vielleicht gegen meine Angehörigen 
verfehlt, weil ich sie verlassen hatte. Ich musste immerwährend 
darüber nachdenken, dass ich solches getan hatte, und
mich fragen, ob ich einem von ihnen ein Unrecht getan hätte;
aber die Anfechtung wurde schwerer und schwerer, und ich wurde
bis zur Verzweiflung versucht. Und weil Satan sein Vorhaben
auf diese Weise nicht erreichte, so legte er mir Fallstricke und
Lockungen, damit ich eine Sünde begehen möchte, die er ausnützen 
könnte, um mich zur Verzweiflung zu bringen. Ich war
etwa 20 Jahre alt, als diese Prüfungen über mich kamen, und
die Angst dauerte mehrere Jahre und ich hätte mich gerne davon
frei gemacht. Ich ging zu manchem Priester, um Trost zu suchen,
aber ich fand keinen bei ihnen.

Von Barnet ging ich nach London\ort{London}, wo ich eine 
Wohnung nahm,
und dort war ich in grosem Elend und Jammer; denn ich sah
das die großen \zitat{Frommen} der Stadt alle in den Banden der
Finsternis waren. Ich hatte einen Oheim dort, 
einen Baptisten\index{Baptisten},
% \picinclude{./000-009/p_s004.jpg}
die waren damals gottselig (\textit{tender}); dennoch 
konnte ich ihm meine
Stimmung nicht kundtun, noch mich ihm anschließen, denn ich
durchschaute alle, jung und alt und wie es um sie stand. Etliche
gottselige Leute (\textit{tender people}) hätten 
mich gern dort behalten,
aber ich getraute mich nicht und wandte mich wieder gegen
Leicestershire; der Gedanke, ich könnte meinen Eltern und Angehörigen 
weh tun, bedrückte mich; denn sie waren, wie ich merkte,
betrübt über meine Abwesenheit.

Als ich nach Leicestershire kam, wollten meine Leute, das ich
mich verheirate\index{Heirat}; aber ich sagte ihnen, 
ich sei noch ein Knabe und
müsse weise werden. Andre hätten mich gerne bei der Hilfstruppe
im Militär\footnote{Es war der Anfang der Bürgerkriege.}\index{Militär}
gesehen, aber ich weigerte mich; und es betrübte mich,
das sie mir solche Dinge vorschlugen, denn ich war ein Gottseliger
(tender) Jüngling. Darauf ging ich nach Coventry, wo ich auf
einige Zeit ein Zimmer im Hause eines \zitat{Frommen} hatte, bis
die Leute anfingen mich zu kennen; denn es waren viele gottselige 
Leute in jener Stadt. Nach einiger Zeit ging ich wieder
in meine Heimat und blieb etwa ein Jahr dort, in großer Trübsal; 
während mancher Nacht irrte ich einsam umher.

Danach kam der Priester von Drayton, Nathanael 
Stevens\person{Stevens, Nathanael},
oft zu mir und ich ging oft zu ihm; und ein anderer Priester
kam oft mit ihm und sie verschmähten nicht, mich anzuhören; ich
stellte ihnen Fragen und diskutierte mit ihnen. Dieser Priester
Stevens stellte mir folgende Frage: warum Christus am Kreuz
gerufen habe: \zitat{mein Gott, mein Gott, warum hast 
du mich verlassen?} 
und warum er gesagt habe: \zitat{wenn es möglich, so gehe
dieser Kelch an mir vorüber, aber nicht wie ich will, sondern wie
du willst.} Ich erwiderte ihm, das zu der Zeit die Sünde der
ganzen Menschheit auf ihm gelegen habe und er ihre Missetat
und Übertretung tragen und für sie geopfert und verwundet
werden musste, sofern er Mensch war; aber er starb nicht, sofern
er Gott war; und weil er so für alle starb und den Tod schmeckte
für jeden Menschen, wurde er zum Opfer für die Sünden der
ganzen Welt. So sprach ich, weil ich zu jener Zeit gewissermasen 
die Leiden Christi, und was er durchgemacht, an mir
nachempfand. Der Priester sagte auch, es sei eine sehr treffende
Antwort, eine, wie er sie noch nie gehört habe. Zu jener Zeit
% \picinclude{./000-009/p_s005.jpg}
pflegte er mich zu loben und anerkennend von mir zu andern zu
sprechen; und das, was ich ihm während der Woche im Gespräch
mitteilte, predigte er dann am \textit{Ersten Tage}\footnote{Fox 
hat den Grundsatz, statt Sonntag, Erster Tag zu sagen, da es für
ihn keine heiligen Tage gibt.}; deswegen mochte
ich ihn nicht leiden. Später wurde dieser Priester ein groser
Verfolger.

Darauf ging ich zu einem andern Priester in Mancetter in
Warwickshire und diskutierte mit ihm über den Grund der Versuchungen 
und der Verzweiflung, aber er verstand meinen Zustand
nicht; er riet mir, zu rauchen und Psalmen zu singen; nun mochte
ich aber den Tabak nicht und zum Psalmensingen war ich nicht
aufgelegt; ich konnte nicht singen. Er lud mich ein, wieder zu
kommen; dann wolle er mir manches sagen; aber als ich kam,
war er ärgerlich und verdrieslich, weil meine früheren Worte ihm
missfallen hatten. Er redete mit seinen Dienstboten über meine
Leiden und Bekümmernisse, und ich bereute, einem solchen meine
Gesinnung aufgedeckt zu haben. Ich sah, das sie alle leidige
Tröster (Hiob 16, 2) waren, und sie machten meine Unruhe noch
gröser. Darauf hörte ich von einem Priester, der in der Nähe
von Tamworth lebte und für einen erfahrenen Mann galt. Ich
ging sieben Meilen weit zu ihm, aber ich fand, das er nur ein
leeres, hohles Gefäs war. Auch von einem Dr. Cradock in Coventry
hörte ich und ging zu ihm. Ich befragte ihn über Versuchung
und Verzweiflung und wie die Anfechtungen wohl über den
Menschen kommen, Er fragte mich, wer Jesu Mutter und Vater
gewesen seien? Ich entgegnete, Maria sei seine Mutter gewesen
und er gelte als der Sohn Josephs, aber er sei der Sohn Gottes.
Wir gingen gerade auf einem schmalen Weg in seinem Garten
und beim Umdrehen trat ich mit dem Fus auf den Rand eines
Beetes, worüber der Mann in Wut geriet, als ob sein Haut; in
Flammen stünde, und unsere ganze Unterredung war gestört und
ich ging in Bekümmernis hinweg, bekümmerter als ich gekommen
war. Ich sah, das sie alle leidige Tröster waren und so viel
wie nichts für mich, denn sie konnten sich nicht in meinen Zustand
versetzen. Daraufhin ging ich zu einem, namens Macham, einem
Priester von hohem Ansehen. Er verordnete mir Arznei und ich
musste zu Ader lassen. Aber man konnte mir keinen Tropfen Blut
entziehen, weder am Arm noch am Kopf, trotz aller Mühe, die
% \picinclude{./000-009/p_s006.jpg}
man sich gab, weil mein Körper wie ausgetrocknet war
durch Kummer, Unruhe und Jammer, die so schwer auf mir
lagen, das ich hätte wünschen können, gar nicht oder blind geboren 
zu sein, damit ich nie die Schlechtigkeit und Eitelkeit der
Welt gesehen hätte, oder taub, das ich nie eitle und böse Worte
gehört hätte, und wie der Name des Herrn gelästert wurde. Als
die Zeit, die man Weihnacht nennt, kam, ging ich, während andere
sich belustigten und sichs wohl sein liesen, von Haus zu Haus zu
armen Witwen und gab ihnen Geld. Wenn ich zu Hochzeiten eingeladen 
war, wie zuweilen geschah, ging ich nie hin, sondern machte
erst am folgenden Tage oder bald darauf einen Besuch, und wenn
die Leute arm waren, gab ich ihnen Geld; ich besas davon gerade 
so viel, das ich niemanden zur Last zu fallen brauchte und
noch dem Dürftigen etwas spenden konnte.

Zu Anfang des Jahres 1646 als ich, auf dem Wege nach
Coventry, mich den Toren der Stadt näherte, stieg die Frage in
mir auf, wie man sagen könne: alle Christen seien Gläubige, sowohl 
Papisten als Protestanten; und der Herr Offenbarte mir,
das, wenn alle Gläubige wären, so wären sie alle aus Gott 
geboren und vom Tode zum Leben durchgedrungen (1. Joh. 3);
nur solche seien wahre Gläubige; und wenn auch andere sagen,
sie seien auch wahre Gläubige, so seien sie es doch nicht.

Ein andermal, als ich am Morgen eines Ersten Tages über
ein Feld ging, offenbarte mir der Herr, das in Oxford oder 
Cambridge erzogen sein noch nicht genüge, um tüchtig und fähig zum
Dienst Christi zu machen; ich verwunderte mich darüber, denn
das war die allgemeine Meinung der Leute. Aber ich sah es
vollständig ein, als der Herr es mir offenbarte und war überzeugt 
davon und pries die Güte Gottes, die mir solches an diesem
Morgen geoffenbart hatte. Es griff das Amt des Priesters Stevens
an, das: \zitat{in Oxford oder Cambridge erzogen zu sein noch nicht
genüge, um tüchtig und fähig zum Dienst Christi zu machen}; es
wurde mir klar, das das, was mir geoffenbart worden war, das
priesterliche Amt angreife. Meine Angehörigen waren sehr betrübt,
das ich nicht mit ihnen kommen wollte, um den Priester zu hören.
Ich ging eben lieber allein ins Freie mit einer Bibel. Ich fragte sie,
ob nicht der Apostel zu den Gläubigen sage, \zitat{sie bedürfen nicht, das
sie jemand lehre, die Salbung lehre sie} (1. Joh. 2). Aber wie wohl 
sie wussten, das solches in der Schrift steht, und das es
% \picinclude{./000-009/p_s007.jpg}
wahr ist, waren sie doch betrübt, das ich mich in diesem Punkte
nicht unterwerfen und mit ihnen den Priester anhören konnte.
Ich sah ein, das es ein ander Ding ist, ein wahrer Gläubiger
zu sein, als das worauf es diesen ankommt [...] Warum sollte
ich also diesen anhängen? Weder diesen noch irgendwelchen
Dissentern konnte ich mich anschliesen, sondern war allein, ein
Fremdling, und hielt mich einzig an den Herrn Jesus Christus.

Ein andermal hatte ich die Offenbarung, das Gott, der die
Welt gemacht hat, nicht in Tempeln mit Händen gemacht wohne.
Dies schien mir zuerst ein seltsames Wort, denn sowohl die
Priester als auch das Volk pflegten ihre Tempel oder Kirchen
\textit{Stätten der Ehrfurcht}, \textit{heiliger Boden} und \textit{Tempel Gottes}
zu nennen. Aber der Herr zeigte mir deutlich, das er nicht
in diesen Tempeln wohne, die von Menschen verordnet und
ausgerichtet waren, sondern in den Herzen der Menschen. Denn
sowohl Stephanus als der Apostel Paulus gaben Zeugnis,
das er nicht in Tempeln mit Händen gemacht wohne (Art. 7,48),
nicht einmal in demjenigen, den er einst zu bauen befohlen hatte,
sintemal er ihm ein Ende gemacht hatte, sondern sein Volk sei
sein Tempel, und da wohne er. Solches wurde mir geoffenbart,
während ich durchs Feld zu den Meinigen ging. Als ich kam,
sagten sie mir, Priester Stevens sei dagewesen und habe gesagt,
er sei besorgt um mich, weil ich neuen Lichtern nachgehe. Ich
lächelte bei mir selber, im Gedanken was der Herr mir über ihn
und seinesgleichen geoffenbart hatte. Aber ich sagte meinen 
Verwandten nichts davon. Denn obgleich sie den Priester durchschauten, 
gingen sie doch, ihn zu hören, und waren betrübt, das
ich nicht auch ging. Aber ich kam ihnen mit Schriftstellen und
zeigte ihnen, das es eine Salbung gibt im Menschen, die ihn
lehrt, und das der Herr sein Volk selber lehren will. Ich hatte
auch grose Offenbarungen über das, was in der Apokalypse steht;
wenn ich davon redete, so sagten die \zitat{Frommen} und die Priester,
sie sei ein versiegeltes Buch, und wollten mich davon abbringen;
aber ich sagte ihnen, Christus könne die Siegel öffnen und sie
sei das, was uns am nächsten angehe; denn die Briefe seien an
die Heiligen früherer Zeiten gerichtet, aber die Apokalypse handle
von den künftigen Dingen.

Ich traf mit Leuten zusammen, welche die Ansicht hatten,
die Frauen hätten keine Seelen, \zitat{nicht mehr als eine Gans},
% \picinclude{./000-009/p_s008.jpg}
setzten sie leichtfertig hinzu. Aber ich tadelte sie und sagte ihnen,
das sei nicht recht, denn Maria sage: \zitat{Meine Seele erhebet den
Herrn und mein Geist freuet sich Gottes seines Heilandes} (Luc. 1,47).

Ein andermal traf ich solche, die Viel auf Träume gaben.
Ich sagte ihnen, wenn sie nicht unterscheiden könnten zwischen
Träumen und Träumen, so würden sie grose Verwirrung anrichten; 
denn es gäbe dreierlei Arten von Träumen: erstens:
\zitat{viele Sorgen machen oft Träume} (Pred. 5,2.); sodann habe der
Mensch oft bei Nacht Einflüsterungen des Satan, und endlich
spreche Gott zum Menschen im Traum. Schlieslich liesen auch
diese Leute von solchen Dingen ab und wurden Freunde.\footnote{\zitat{Freubde} 
ist bis heute die Selbstbezeichnung der Quäker: Das Wort
\zitat{Freund} ist also hier immer in diesem Sinn zu verstehen.}

Trotz vieler groser Offenbarungen, die ich hatte, kamen doch
oft schwere Anfechtungen über mich, so das ich bei Tag wünschte,
es wäre Nacht, und bei Nacht wünschte ich den Tag. Mit meinen
Offenbarungen war es wie David sagt: \glqq ein Tag sagt es dem
andern, und eine Nacht tut es kund der andern\grqq (Psalm 19,3).
Denn meine Offenbarungen bezogen sich immer eine auf die
andere: sie bezogen sich aber auch auf die Schrift, über die ich
grose Offenbarungen hatte, und die Anfechtungen, die über mich
kamen, bezogen sich auch immer eine auf die andere.

Zu Anfang des Jahres 1647 hies mich der Herr nach
Derbshire gehen, wo ich mit freundlich gesinnten Leuten (\textit{friendly
people}) zusammentraf und ich hatte viele Unterredungen mit
ihnen. Als ich sodann durch die Gegend des Peak kam, traf
ich noch mehr Gleichgesinnte, worunter aber auch solche, die eitle,
hochfahrende Ideen hatten. In der Gegend von Nottinghamshire
und Leicestershire traf ich ebenfalls Gottselige (\textit{tender people});
unter anderen auch eine sehr gottselige (\textit{tender}) Frau, Elisabeth
Hooton\footnote{Elisabeth Hooten, eine Frau aus den angesehensten 
Gesellschaftskreisen schloss sich später den Quäkern als Predigerin an.}; 
ich hatte mehrere Versammlungen mit ihnen. Aber
meine Trübsal dauerte fort, und ich war oft in grosen Versuchungen. 
Ich fastete viel und ging oft manchen Tag drausen
umher an einsamen Orten, und nahm oft eine Bibel und setzte
mich in hohle Bäume und an verlassene Plätze, bis die Nacht
kam; häufig lief ich in der Nacht traurig umher; denn ich war
% \picinclude{./000-009/p_s009.jpg} 
ein Mann der Schmerzen, in den Zeiten, da der Herr sein Werk
in mir anfing.

Während dieser ganzen Zeit hatte ich mich nie mit irgend
jemand zu irgend einer religiösen Richtung Verbunden, sondern
gab mich ganz dem Herrn hin; von aller schlechten Gesellschaft
hatte ich mich losgemacht, hatte Abschied genommen von Vater
und Mutter und allen andern Angehörigen und zog als ein
Fremdling umher, wohin der Herr mein Herz lenkte; ich mietete
ein Zimmer jeweilen in der Stadt, in die ich kam und weilte oft
etwa einen Monat an einem Orte; denn ich wagte nie lange an
einem Orte zu bleiben, da ich fürchtete, als gottseliger Jüngling
sowohl bei den \glqq Frommen\grqq als auch bei den Ungläubigen Schaden
zu nehmen, wenn ich viel mit den einen oder den anderen umging;
darum verhielt ich mich meist wie ein Fremdling; ich suchte himmlische 
Weisheit, und Erkenntnis kam mir einzig vom Herrn. Ich
wurde losgelöst von den äuseren Dingen, um mich allein auf
den Herrn zu verlassen. Meine Prüfungen und Trübsale waren
sehr schwer; aber wenn es mir zwischen hinein etwas leichter
wurde, so geriet ich oft in solch eine himmlische Freude, das ich
wähnte, in Abrahams Schos gewesen zu sein. Wie ich das Elend,
in dem ich war, nicht schildern kann, ebensowenig kann ich die
Barmherzigkeit beschreiben, die Gott in diesem Elend an mir getan
hat [...]

Nachdem ich die Offenbarung vom Herrn empfangen hatte,
\glqq das in Oxford oder Cambridge erzogen zu sein noch nicht zum
Dienst des Herrn befähige\grqq, achtete ich die Priester weniger und
sah mehr auf die Dissenter; ich sah, das unter diesen einige
Gottseligkeit sei, und viele von ihnen kamen auch später, zu einer
festen Überzeugung, weil sie Offenbarungen hatten. Aber wie
ich die Priester aufgegeben hatte, so lies ich auch die 
Separatistenprediger und solche, welche als die Erfahrensten angesehen
wurden; denn ich sah, das keiner unter ihnen allen war, der zu meinem
Zustand sprechen konnte. Als alle meine Hoffnungen auf sie und alle
Menschen dahin waren, so das ich nichts hatte, das mir von ausen
half, und ich nicht wusste, was tun — da! o da hörte ich eine
Stimme: \glqq es ist Einer, der zu deinem Zustand sprechen kann,
nämlich Jesus Christus.\grqq Und als ich das hörte, hüpfte mein
Herz vor Freude. Dann zeigte mir der Herr, warum niemand
auf der Welt mir in meinem damaligen Zustand helfen konnte,
% \picinclude{./010-019/p_s010.jpg}

nämlich —— damit ihm die Ehre allein gebühre. Alle sind mit
Sünde und Unglauben behaftet, damit Christus, der erleuchtet
und Gnade, Glauben und Kraft gibt, den Vorrang habe [...].
Mein Verlangen nach dem Herrn wurde immer stärker und der
Eifer nach der Erkenntnis Christi und Gottes, ohne jegliche Hilfe
von Menschen oder Büchern. Denn obwohl ich die Schrift las,
die von Gott und Christus sprach, so kannte ich ihn doch nur
durch Offenbarung, als den, der den Schlüssel hat und auftut
(Offb. 3, 7), und als den Vater des Lebens, der mich durch seinen
Geist zum Sohne zog. Dann führte mich der Herr freundlich
weiter und lies mich seine ewige, unendliche Liebe sehen, die
alles übertrifft, was die Menschen in ihrem natürlichen Zustand,
oder durch Bücher oder die Geschichte erkennen können;
und diese Liebe zeigte mir auch, wie ich selber war ohne ihn.
Ich zog mich zurück von allen anderen, denn durch die Liebe
Gottes sah ich deutlich, wie es um sie stand. Ich hatte keinen
Umgang mit irgendjemand, Priester oder Frommen«, oder
irgendwelchen Separatiften; sondern nur mit Christus, der der
Schlüssel ist, und der mir die Tür zum Licht und zum Leben ge-
öffnet hatte. Jch fürchtete mich vor allem Reden über irdische Dinge;
denn ich sah nur Verderbliches darin, und wie das Leben von Ver-
derben belastet war. Als ich selber in der Tiefe war und unter dem
Druck, da glaubte ich nicht, das ich je wieder darüber Herr werden
würde; meine Trübsal, Bekümmernis und Versuchung war so
gros, das ich ost glaubte, verzweifeln zu müssen, so sehr ward
ich versnchet; als aber Christus mir offenbarte, wie er vom gleichen
Satan war versuchet worden, und wie er über ihn Herr geworden
war und ihm den Kopf zertreten hatte (l.Mos. 3, 15), und wie durch
ihn, seine Kraft, sein Licht und seine Gnade und seinen Geist ich
auch siegen werde, da vertraute ich ihm. So war er es, der mir
austat als ich eingeschlossen war und weder Hoffnung noch Glauben
hatte. Christus, der mich erleuchtet hatte, schenkte mir sein Licht,
um daran zu glauben, er schenkte mir Hoffnung, die er selber in
mir ausrichtete, und er gab mir seinen Geist und seine Gnade,
die mir geniigten in meiner Schwachheit. Also erhielt mich der
Herr im tiefsten Elend und Jammer, die oft über mich kamen.
Jch sand in mir zweierlei Durst: nach der Kreatur, um dort
Hilfe und Kraft zu suchen, und nach Gott, dem Schöpfer, und
seinem Sohn Jesus Christus. Jch sah, das die ganze Welt mir


% \picinclude{./010-019/p_s011.jpg} 

s Erweckung und Krisis bis zum Durchbruch. 11
nicht helfen konnte; wenn ich die Kost, den Palast und die
Dienerschast eines Königs gehabt hätte, so wäre es mir nichts
uütze gewesen; denn nichts konnte mich trösten, als die Kraft
des. Herm. Jch sah, das die Ptiestet und die ,,Frommen«
und überhaupt die Menschen hohl waren und ganz zufrieden
in dem Zustand, der mich elend machte; und das sie das
liebten, wovon ich gerne los geworden wäre. Aber der Herr,
von welchem meine Hilfe kam, nahm mein Anliegen auf sich, und
ich wars meine Sorgen aus ihn allein. Darum wartet alle ge-
duldig aus den Herrn, in welchem Zustand ihr auch sein möget;
wartet in der Gnade und Wahrheit, die von Christus kommt;
wenn ihr das tut, so habet ihr eine Verheisung, die der Herr t
an euch erfüllen wird. Wahrlich, selig sind alle, die da hungert
und dürstet nach Gerechtigkeit, denn sie sollen satt werden .....
Wiederum hörte ich eine Stimme, welche sagte: »Du, Schlange,
du suchst das Leben umzubringen, aber kannst es nicht; denn das
Schwert, das den Baum des Lebens (1. Mos. 3.) bewacht, wird
dich umbringen.« Christus, das Wort Gottes, das der Schlange,
dem Mörder, den Kopf zertrat, behütete mich, weil mein Jnneres »
empsänglich war für seinen guten Samen, diesen Samen, der der
Schlange, dem Mörder, den Kopf zertrat. Dieses inwendige
Leben sproste in mir empor, also das ich auf alle Einwände der
Priester und der »Frommen« antworten konnte, und brachte mir
Schristworte ins Gedächtnis, um sie zu widerlegen.
Einmal sah ich die grose Liebe Gottes und ich wurde mit
Bewunderung über ihre Unendlichkeit erfüllt; ich sah, wer von
Gott ausgestosen war, und wer ins Reich Gottes einging, und
wie man Einlas bekommt durch Jesum, der mit seinem himm-
lichen Schlüssel die Tür öffnet; und ich sah den Tod, wie er
über alle Menschen hingegangen war und den Samen Gottes in
den Menschen und auch in mir unterdrückt hatte, wie aber nun
dieser Same in mir ausging und was die Verheisung war. Es
war ein Kampf in meinem Innern: Fragen stiegen in mir
aus über Gaben und Weissagungen; und dann wurde ich
versucht bis zur Verzweiflung, als ob ich gegen den heiligen Geist
gesündigt hätte. Jch war in groser Bangigkeit und Trtibsal
tagelang. Dennoch verlies ich mich ganz aus den Herrn. Ein-
mal als ich von einem einsamen Gang zurückkam, wurde ich so
von der Liebe Gottes eingehüllt, das ich unaufhörlich die Gröse


% \picinclude{./010-019/p_s012.jpg} 
leiner Liebe anftaunen muste. Während ich in diesem Zustand
war, eröffnete mir die ewige Klarheit und Kraft, das: ,,alles ge-
schehen mus in und durch Christum; und das er jenen Versucher,
den Teufel besiegt und umbringt tmd alle seine Werke und über
ihm steht; und das alle diese Trübsal gut für mich war, und
die Versuchungen zur Prüfung meines Glaubens .dienten, den
C-hristus mir gegeben. Der Herr schenkte es mir, das ich durch
alle diese Trübsale und Versuchungen hindurch sehen konnte;
mein lebendiger Glaube wurde erweckt, das ich sah, wie alles
durch Christus, das Leben, geschah, und ich glaubte an ihn. Wenn
irgend einmal meine Stimmung getrübt war, so blieb mein innerer
Glaube fest, und meine tiefgegriindete Hoffnung hielt mich wie
ein Anker im Meere?-grund und ankerte meine unsterbliche Seele
in ihren Bischof (1. Petr. 2,25), indem sie ihr half über den Wassern
der Welt, ihren wilden Wogen, Stürmen und Versuchungen zu
schwimmen. Ach, da wurden mir meine Trübsale, Anfechtungen
und Versuchungen klarer, denn je zuoor. Wenn es Licht ward
in mir, da wurde alles, was nicht vom Licht war — Finsternis
Tod, Versuchung, Unrecht und Gottlostgkeit — offenbar und kam
ans Licht. Darnach entstand ein Feuer in mir, und ich sah »ihn
sitzen wie das Feuer eines Goldschmieds und wie die Seife eines
Wäschers«. (Mal. 3, 2). Der Geist der Unterscheidung kam über
mich, durch welchen ich erkannte, mas meine eigenen Gedanken,
mein Seufzen und mein Stöhnen bedeutete, und mas mir die
Erkenntnis trübte, und woher mir die Offenbarungen kamen.
Alles was sich nicht in der Geduld bewähren und das Feuer
nicht erdulden konnte, erkannte ich im Licht als Seufzer des
Fleisches, das sich nicht in Gottes Willen fügen wollte: dieses
hatte mich so verdunkelt, das ich nicht geduldig sein konnte in
Anfechtung, Trtibsal und Verwirrung. Ich konnte mein eigenes
Jch nicht in den Tod ans Kreuz geben, das uns die Kraft
verleiht, Gott zu leben; sie bewirkt, das alles mas uns
die Gegenwart Christi oerhüllt, mas das Schwert des Geistes
niederschlägt und tötet, nicht weiter leben kann. Jch unter-
schied auch das Seuszen des Geistes, der mir Offenbarungen
eingab und der mich bei Gott vertrat (Röm. 8, 20). Ju diesem
Geiste ist das wahre Warten im Herrn aus die Erlösung des
Leibes und der ganzen Kreatur. Durch diesen unsichtbaren Geist,
in dem das wahre Seufzen geschieht, erkannte ich auch das ver-



% \picinclude{./010-019/p_s013.jpg} 

Erweckung und Krisi-3 bis zum Durchbruch. 13
kehrte Seufzen und Flehen. Durch diesen unsichtbaren Geist
Unterschied ich in allem, was-«’ ich hörte, sah und schmeckte, das
Falsche, dach sich über den Geist erhebt und ihn dämpst und
betrübt; und ich sah, wie alle die darin waren, im Jrrtum waren
und Schaden nahmete und im falschen Bitten und Flehen und in
jenem Wandeln und Reden, darinnen man Gottes Namen ver-
geblich anruft; in jenem Geist, der durch das ägyptische Meer
watet und bittet, aber nicht empfängt; denn sie hassen sein
Licht und widerstreisen dem heiligen Geist, sie verwandeln die
Gnade in Wollust und lehnen sich auf wider den heiligen Geist;
und wenden sich ab zoom Glauben, in welchem sie beten sollten,
und vom Geist, in dem sie bitten sollten (Jud.) . . . .
Jch hörte von einer Frau in Laneafhire, die 22 Tage ge-
fastei hatte und ich ging hin, um sie zu sehen; aber als ich zu
ihr kam, sah ich, das sie unter groser Versuchung war. Nach-
dem ich zu ihr gereiet von dem, waö ich vom Herrn empfangen
hatte, verlies ich sie denn ihr Vater war ein Groser unter den
,,Frommen«. Von ia ging ich zu den ,,Frommen« in Duckingsield
und Manchester, wo ich einige Zeit blieb und die Wahrheit unter
ihnen verkündete. G3 wurden etliche von ihr überzeugt und nahmen
die Lehre dez Herrn an und wurden durch dieselbe fest gemacht und
blieben in der Wahrheit. Aber die ,,Frommen« waren wütend;
denn sie eiferten alle für die Lehre von der Sündhaftigkeit und
konnten es nicht ertragen, von Vollkommenheit sprechen zu hören
und von einem heilixen, sitndlosen Leben. Aber de; Herrn Macht
war über allen, wenn sie gleich in Finsternis gebunden waren
und in der Sünde, iiir die sie eiferten und das- Gottselige in sich
erstickten. GS war zu der Zeit eine grose Versammlung der
Baptisten in Brougthon in Leieestershire, mit etlichen, die sich
von ihnen losgetrenmt hatten; eö gingen auch Leute von anderen
Richtungen hin und ich ging auch; es waren nicht viele Baptisten
aber viele andere dort. Der Herr öfsnete mir den Mund und
die ewige Wahrheit wurde unter ihnen verkündet, und die Macht
dez Herrn war über ihnen allen. Ju diesen Tagen fing die Macht
des Herrn an zu treilsen und ich hatte grose Ofsenbarungen über die
Schrift. ES wurdenetliche in dieser Gegend gewonnen und kehrten
sich von der Finstevuis zum Licht, von der Macht des- Satans zu
Gott, rmd manche werden erweckt zu Gottes Preis-. Ob ich mich an
,,Fromme« oder andere wandte, stets- wurden etliche gewonnen.


% \picinclude{./010-019/p_s014.jpg} 
Ich war damals noch in grosen Versuchungen und meine
inneren Leiden waren schwer; aber ich fand keinen, dem ich meinen
inneren Zustand hätte eröffnen können, als allein den Herrn, zu
dem ich Tag und Nacht schrie. Jch ging zurück nach Notting-
hamshire, und dort zeigte mir der Herr, das das Böse, das sich
in den äuseren Dingen zeigt, inwendig in den Herzen und Ge-
danken unserer bösen Menschennatur ist. Ich sah die Natur der
Hunde, Schweine, Schlangen, die Natur von Sodom und Agypten,
von Pharao, Kain, Jsmael, Esau 2c. inwendig in den Menschen,
während andere sie im Äusern suchten. Jch schrie zum Herrn:
,,Warum mus mir solches geschehen, da ich mich doch nie solchen
Lastern ergeben werde?« Und der Herr antwortete mir, ich
müsse einen Begriff bekommen von diesen Zuständen; wie·sollte
ich sonst zu allen den verschiedenen Zuständen sprechen können?
und ich erkannte die unendliche Liebe Gottes darin. Ich erkannte,
das es einen Ozean des Todes und der Finsternis gibt, aber
auch einen unendlichen, unerschöpfllichen Ozean des Lichts und
der Liebe, der über den Ozean der Finsternis fliest. Jch sah K
auch darin die unendliche Liebe Gottes, und ich hatte grose
Offenbarungen.
Als ich beim Turmhaus (eteeplebouze) 1) von Mansfield vor-
bei kam, sagte der Herr zu mir: ,,Das, was die Leute mit Füsen
treten, mus deine Nahrung sein«. Und während der Herr also
zu mir sprach, offenbarte er mir, das das Volk und die »Frommen«
das Leben von Christus . . . das Blut des Sohnes Gottes, welches
mein Leben war, mit Füsen treten und von ihren Ginfällen leben,
wenn sie gleich von ihm schwatzen. Gs schien mir zuerst merk-
würdig, das ich mich nähren sollte mit dem, wa-? die grosen
»Frommen« mit Füsen traten; aber der Herr ossenbarte es mir
deutlich durch seinen ewigen Geist und seine Macht.
Die Leute kamen von nah und fern um mich zu sehen;
aber ich vermied, von ihnen ausgesucht zu werden; doch ich muste
reden und ihnen allerlei eröffnen. Einer, namens Brown, hatte
L grose Weissagungen und Gesichte über mich auf dem Totbett.
Er sprach von nichts anderm, als was ich schaffen werde als
1) Fox gebraucht die Bezeichnung ,,Turmhaus« statt Kitche, weil: ,,die
Fechsctzellnnter Kirche nicht ein Gebäude, sondern die Gemeinde der Gläubigen


% \picinclude{./010-019/p_s015.jpg} 

Erweckung und Ktisis bis zum Durchbruch. 15
Werkzeug des Herrn, und von andern sagte er, das sie in Ver-
derben geraten werden; es erfüllte sich bei einigen, die damals
viel gegolten hatten. Als dieser Mann begraben war, legte sich
die Hand des Herm schwer auf mich, zum Erstaunen vieler, die
glaubten, ich müsse tot gewesen sein; während vierzehn Tagen
kamen viele, nm mich zu sehen. Ich war sehr verändert in Aus-
sehen und Gestalt, als ob mein Körper neu gebildet oder ver-
wandelt worden wäre. Während ich in diesem Zustand war,
schenkte mir der Herr einen Sinn und eine Gabe der Unter-
scheidung, womit ich deutlich erkannte, das bei vielen, wenn sie
von Gott redeten und von Christus, die Schlange aus ihnen redete;
dies war hatt zu ertragen; doch das Werk des Herrn ging all-
mählich vorwärts, und meine Anfechtungen und Trübsale fingen
an abzunehmen, und Tränen der Freude entrannen mir, so das
ich Tag und Nacht dem Herrn hätte Freudentränen weinen mögen,
mit demütigem, zerschlagenem Herzen. Jch tat einen Blick in das,
was ohne Ende ist, in Dinge, die nicht ausgesprochen werden
können, und in die Gröse und Unendlichkeit der Liebe Gottes,
die sich nicht in Worten ausdrücken läst; denn ich war durch H
den Ocean der Finsternis und des Todes und durch die Macht
des Satans gebracht worden vermöge der ewigen, herrlichen Kraft
Christi; und selbst durch jene Finsternis wurde ich gebracht, welche
die ganze Welt bedeckt und alles gebunden hält und alle dem Tode
preis gibt. Es war die gleiche Kraft Gottes, die mich durch
solches alles hindurch brachte, welche nachher das ganze Land,
die Priester wie die ,,Frommen« und das Volk ergriff.
Jch konnte von mir sagen, ich sei im geistigen Babylon, Sodom,
Egypten und im Grabe gewesen; aber durch die ewige Kraft Gottes
war ich Herr geworden über jene Mächte und hindurchgedrungen
in die Kraft Christi. Ich sah die Ernte weis und den Samen
Gottes so dicht im Boden, wie nur je Weizen ausgesäet worden
war, und niemand ihn zu sammeln, darüber trauerte ich mit NT
Tränen.
Es ging das Gerücht über mich, ich sei einer, der den Geist
der Unterscheidung hätte; daraufhin kamen Viele zu mir von nah
und fern, »Fromme«, Priester und Volk. Die Macht des Herm
brach hervor, und ich hatte grose Weissagungen; ich redete zu
ihnen von den göttlichen Dingen; sie hörten aufmerksam und an-
dächtig zu, gingen hinweg und machten es ruchbar.


% \picinclude{./010-019/p_s016.jpg} 
Dann kam der Versucher und setzte mir wieder zu und
klagte mich nn, ich hätte wider den heiligen Geist gesündigt; aber
ich muste nicht, worin. Da kam mir der Zustand Paulus-’ in
den Sinn, wie er in den dritten Himmel verzückt gewesen undf
Dinge gesehen hatte, welche kein Mensch sagen kann, und wie
darauf ein Vote des Satans gesandt worden war, ihn mit
Fäusten zu schlagen. So überwand ich durch die Krast Christi
auch diese Versuchung.


%%%%%%%%%%%%%%%%%%% Kapitel 2. %%%%%%%%%%%%%%%%%%%%%%%%%%%%%%
\chapter[Erste Versammlungen]{Erste Versammlungen}

\begin{center}
\textbf{Erste Versammlungen und Proteste.}
\end{center}

Die Macht dez Herrn hatte nun, im Jahre 1648, schon vielen
die Herzen geöffnet, daß ste daß Wort des Lebenß und der Ver:
söhnung aufnahmen. A15 ich nun einmal im Hause eineß Freunde?-,
in Nottinghamshire, saß, erkannte ich, daß ein großeß Krachen
durch die ganze Erde gehen mußte und ein großer Rauch auf-
steigen, überall wo es krachte, und darnach würde ein großes
Beben entstehen: es war die Erde in der Menschen Herzen, die
erbeben mußte, bevor der Same Gotteß au-es der Erde hervor-
gehen konnte. Und so geschah eö: die Macht dez Herrn fing an,
sie erbeben zu machen, und wir fingen an, große Versammlungen
zu haben, und man spürte die mächtige Kraft und daß Wirken
Gottes unter den Leuten, zu ihrer und der Priester Erstaunen ....
Jch ging nach Manöfield, wo eine große Versammlung von
,,Frommen« und andern Leuten stattfand; da trieb etz mich zu
beten, und die Kraft des- Herrn war so mächtig, daß etz schien,
als ob das- ganze Hauö erbebte. Alö ich geendet, sagten etliche
der Frommen, es sei gerade wie in den Tagen der Apostel, da
sich ,,daT- Hauß bewegte, in dem sie versammelt waren« (Act. 2, 2).
Nachdem ich gebetet, wollte einer der ,,Frommen« beten, aber
dadurch kam eine Trübung und etwas toteß über sie und die
andern ,,Frommen« wurden betrübt über ihn und sagten, ez sei
eine Versuchung über ihn gekommen; darauf kam er zu mir und
bat mich, ich solle wieder beten, aber ich konnte nicht auf eineiz
Menschen Geheiß beten.
Bald darauf war abermalß eine Versammlung von ,,Frommen«


% \picinclude{./010-019/p_s017.jpg} 
Erste Versammlungen und Proteste. 17
und ein Hauptmann namenßt Stoddard wohnte ihr bei. Sie
redeten über das Blut Christi, und während sie darüber sprachen,
sah ich durch die unmittelbare Offenbarung des unsichtbaren
Geistes das Blut Christi. Und ich schrie auf und rief: »Seht
ihr nicht das Blut Christi? Seht in eure Herzen, wie ee eure
Herzen und Gewissen besprengt, daß sie, loß von den toten
Werken, dem lebendigen Gott dienen« (Gbr. 9). Denn ich sah
ez, das Blut dez neuen Testamenteß, wie ez ins Herz kam. Daß
erschreckte die »Frommen«; sie wollten daß Blut nur aus?-wendig,
nicht inwendig haben. Aber Hauptmann Stoddard war ergriffen
und sagte: ,,Laßt den Jüngling reden, hört ihn an«, alß er sah,
wie sie mich mit vielen Worten zu besiegen suchten.
ES waren auch eine Anzahl Priester da, die ster gottselig
galten; einer von ihnen hieß Kellett, und etliche, die empfänglichen
Gemüteö waren, gingen hin, um sie zu hören. GS trieb mich,
ihnen nachzugehen, um sie zu ermahnen, auf die Lehre Gotteß in
ihrem Jnnern zu hören. Damals war der Priester Kellett gegen
das Priesteramt; später jedoch nahm er selbst ein solchetz an und
wurde ein Verfolger.
Nachdem ich etliche Arbeit getan hatte in dieser Gegend,
ging ich durch Derbshire in meine Heimat Leicestershire, und
ez wurden mehrere, die empsänglich waren, gewonnen. A15 ich
von dort wegzog, begegnete ich einer großen Zusammenkunft
von »Frommen«, die im Freien beteten und die Schrift auß-
legten. Sie reichten mir die Bibel und ich öffnete sie beim 5. Kap.
des Jtzzatth., wo Ehristuß daß; Gesetz auölegt; und ich erklärte
ihnen en inneren Zustand und den äußeren Zustand worüber sie
in heftigen Streit gerieten und so auszeinandergingen; aber die
Kraft des Herrn nahm überhand.
Darauf hörte ich von einer großen Versammlung, die in
Leicester stattfinden würde; eß sollte eine Die-putation geben, die die
Preßbhterianer, Jndependenten, Baptisten und Common-Payen
Leute gleicherweise angehen sollte. Die Versammlung war in einem
Turmhause, und der Herr trieb mich, dorthin zu gehen und
zugegen zu sein. Jch hörte ihren Verhandlungen und Beweis-
führungen zu. Einige saßen in Kirchenstühlen und der Priester
war aus der Kanzel; es war eine große Menge versammelt.
Zuletzt tat eine Frau eine Frage über die Stelle bei Petrus:
»Wiedergeboren auß ewiglichem Samen, auö dem lebendigen Wort
George Ft--. 2



% \picinclude{./010-019/p_s018.jpg} 
Gottes-, das-8 ewiglich bleibet (1. Petr. 1). Der Priester sagte ihr:
,,Jch erlaube keiner Frau in der Kirche zu reden,« obgleich er
vorher allen die Freiheit erteilt hatte, zu reden. Da wurde ich-
von der Krast dez Herrn übermannt wie in einer Verziickung,
und ich erhob mich und fragte den Priester: »Nennst du dieß
hier, dieseö Turmhauz, eine ,,Kirche«? oder nennst du diese
bunte Menge eine Kirche?« Denn er hätte der Frau auf ihre
Frage antworten sollen, nachdem er vorher allen die Freiheit
erteilt hatte, zu reden. Anstatt mir zu antworten, sragte er mich:
maß eine Kirche sei. Ich sagte: »Die Kirche ist der Pfeiler und
Grund der Wahrheit, auß lebendigen Steinen gemacht, aus
lebendigen Gliedern (1. Petr. 2), eine geistige Haußgemeinde,
deren Haupt Christus- ift; aber er ist nicht daS Haupt einer bunten
Menge oder eines alten Hauseß auß Kalk, Steinen und Holz.«
Diese Worte brachten alleß auß Rand und Band; der Priester
kam auß seiner Kanzel, andere auö ihren Stühlen, und die Ver-
handlungen waren gestört. Ich ging in eine große Herberge und
dißputierte dort mit Priestern und ,,Frommen« aller Richtungen;
und alle waren furchtbar hitzig. Aber ich bestand auf der wahren
Kirche und ihrem wahren Haupt, trotz ihnen allen, bi-3 sie nach-
gaben und auöeinanderstoben. Einer schien sehr geneigt und kam
eine Zeit lang, in der Absicht, sich mir anzuschließen; aber bald
kehrte er sich ganz gegen mich und schloß sich einem Priester an,
trat für die Kindertanse ein, obgleich er vorher selber ein
Baptist gewesen war, und verließ mich. Aber etz wurden an dem
Tage etliche gewonnen; auch die Frau, welche die Frage getan
hatte, wurde gewonnen, samt den Jhrigen; und deö Herrn Kraft
und Herrlichkeit leuchtete über allen.
Hierauf kehrte ich zurück nach Nottinghamshire und ging
ink- Vale of Beavor. Unterwegß predigte ich den Leuten Buße
und ez wurden viele gewonnen, im Vale of Beavor und in den
Städten; denn ich blieb einige Wochen dort. Eineß Morgen?-,
alß ich am Feuer saß, kam eine große Wolke über mich, und eine
große Versuchung überkam mich; aber ich blieb ganz ruhig. Und
ich hörte eine Stimme zu mir sagen: »Alle Dinge gehen auß der
Natur heroor«; und die Elemente und die Sterne kamen über
mich, so daß ich ganz davon eingehiillt war. Aber die andern
im Hause merkten nichtß von all dem, weil ich ganz still und
ruhig war. Und weil ich still und ruhig war und wartete, so


% \picinclude{./010-019/p_s019.jpg} 

Erste Versammlungen und Proteste. 19
stieg eine lebendige Hoffnung in mir auf, und ich Vernahm deutlich
eine Stimme, welche sagte: ,,EZ gibt einen lebendigen Gott, der
alle Dinge geschaffen hat«; und sogleich verschwand die Wolke
und auch die Versuchung, und Leben breitete sich über alles; mein
Herz ward fröhlich und ich prieö den lebendigen Gott. Einige
Zeit darauf traf ich etliche, die behaupteten, ez gebe keinen Gott,
sondern alle Dinge gehen aus-3 der Natur hervor. Ich hatte einen
langen Di?-put mit ihnen und brachte sie herum, so daß mehrere
zugaben, es gebe einen lebendigen Gott. Da sah ich, daß etz gut
gewesen war, daß ich jene Prüfung durchgemacht hatte. Wir
hatten große Versammlungen in jenen Gegenden, denn die Kraft
deß Herm brach hervor in diesem Teil deß Landeö. A13 ich nach
Nottinghamshire zurück kam, traf ich eine Schar von verworrenen
Baptisten und andem; die Kraft des Herrn wirkte mächtig und
gewann viele unter ihnen. Darauf ging ich in die Umgegend von
Manöfield, wo die Kraft deß Herrn herrlich kund ward, in der
Stadt Manßfield und auch in anderen Städten. Jn Derbshire
wirkte sie in herrlicher Weise. Jn Eton in der Nähe von Derby
war eine Versammlung von Freunden; die Kraft dez Herrn tat sich
darin so mächtig kund, daß viele gewaltig erschüttert wurden, und
vieler Mund wurde aufgetan durch die Kraft dez Herrn. Viele wurden
vom Herrn getrieben in die Turmhäuser zu gehen, zu den Priestern
und zum Volk, um ihnen die ewige Wahrheit zu verkünden.
Einmal als- ich in Man?-field war, fand eine Sitzung der
Richter wegen dez Dingenö von Dienstboten statt. ES trieb
mich hinzugehen und den Richtern zu sagen, sie sollten die
Dienstboten richt am Lohn verkürzen. Jch kam in die Nähe
der Herberge, in der die Sitzung abgehalten wurde; aber
alL ich dort eine Musikantenbande traf, ging ich nicht hinein,
sondern gedachte am folgenden Morgen wieder zu kommen, hofsend,
sie dann in ernster Stimmung zu treffen, um mit ihnen zu ver-
handeln; denn ez schien mir jetzt nicht die geeignete Zeit. Aber
alö ich am Morgen kam, war alleö fort; da wurde mir ganz
schwarz vor den Augen, so daß ich fast nichtß mehr sah; ich fragte
den Wirt, wo die Richter an dem Tage Sitzung haben würden;
er sagte mir, in einer etwa acht Meilen entfernten Stadt. Nun
fing ich wieder an zu sehen und lief dorthin, so schnell ich konnte;
altz ich zu dem Hauö kam, in dem sie und ihre zahlreiche Diener-
schaft waren, mahnte ich die Richter, die Dienstboten nicht am



% \picinclude{./020-029/p_s020.jpg} 
Lohn zu verkürzen, sondern ihnen zu geben, was recht und billig
sei, und die Dienstboten ermahnte ich, ihre Pflicht zu tun und
ehrlich zu dienen; sie nahmen meine Mahnungen freundlich auf,
denn ich wurde vom Herm dazu getrieben.
Ferner trieb etz mich, an verschiedene Gerichtöhöfe und in ver-
schiedene Turmhäuser in Manöfield und an andern Orten zu gehen,
um alle zu ermahnen vom Unterdrücken und vom Schwören abzu-
lassen und sich von der Ungerechtigkeit zum Herrn zu bekehren und
recht zu tun. Jnßbesondere trieb es mich, nach einer Gerichtßver-
handlung in Manöfield zu einem zu gehen, der einer der schlech-
testen Menschen der dortigen Gegend war, und mit ihm zu reden;
er war ein Säufer und berüchtigte: Mädchenhändler; ich warnte ihn
beim allmächtigen Gott wegen s eines schlechten Wandels; als ich auß-
geredet hatte und ihn Verlassen wollte, lies er mir nach und sagte
mir, während ich mit ihm gesprochen habe, sei er so ergriffen worden,
daß ihn seine Kräfte ganz verließen. So wurde dieser Mann be-
kehrt, und er ließ ab von seiner Schlechtigkeit und blieb rechtschaffen
und nüchtern zum Erstaunen aller, die ihn vorher gekannt hatten.
Und das Werk des Herrn nahm zu und viele kamen von der Finster- ,
nie zum Licht, im Laufe dieser drei Jahre 1646, 1647 und 1648.
GS wurden in dieser Zeit mehrere Versammlungen für Freunde ein-
gerichtet, damit Gott sich kund tue durch sein Licht, seinen Geist
und seine Kraft; denn dee Herrn Kraft brach immer herrlicher hervor.
Nun war ich ini Geiste bei Idem stammenden Schwert vorbei
inö Paradieß Gotteö eingedrungen. Alle Dinge waren wie um-
gewandelt ftir mich und die ganze Schöpfung hatte einen andern
Geruch für mich, über alles waß Worte außdrücken können. Ich
wußte nur noch von Reinheit, Unschuld und Rechtschaffenheit, denn
ich war erneuert zum Ebenbild Gotteß (Col. 3, 10) durch Christus,
in den Zustand, in dem Adam vor dem Fall gewesen war. Die
ganze Schöpfung wurde mir offenbar und es- wurde mir gezeigt,
wie alle Dinge mit dem Namen genannt wurden, der ihrem
Wesen und ihren Kräften entsprach. Jch war unschliisstg, ob ich
nicht sollte Heilkunde treiben zum Nutzen der Menschheit, als ich
sah, wie die Natur und die Kräfte aller Dinge mir so geoffenbart
wurden vom Herrn. Aber alsbald wurde ich ergriffen im Geist
und erkannte einen andern, sicherem Zustand als die Sitndlosig-
keit Adams, den Zustand Jesu Christi, der nicht fallen konnte.
Und der Herr zeigte mir, daß die, so ihm treu bleiben im Licht


% \picinclude{./020-029/p_s021.jpg} 
Erste Versammlungen und Proteste. 21
und in der Kraft Christi, erhoben werden in den Zustand, darin
Adam vor dem Fall gewesen war, in welchem die bewundernß-
werten Werke der Schöpfung und ihre Kräfte erkannt werden
können durch die Offenbarung deß göttlichen Worteß der Weiß-
heit und der Kraft, durch welche sie gemacht waren. Der Herr
führte mich in große Dinge ein, und wunderbare Tiefen wurden
mir geoffenbart, die alleß iibertrafen, waß Worte beschreiben
können. Aber wer sich dem Geist Gotteß unierwirst und hinein-
wächst in daß Gbenbild und die Kraft deß Allmächtigen, der wird
daß Wort der Weißheit empfangen, daß alle Dinge offenbar macht,
und wird dazu gelangen, die verborgene Einheit in dem ewigen
Wesen zu erkennen.
So reiste ich umher im Dienste deß Herrn, wie mich der
Herr führte. Alß ich nach Nottingham kam, war Gotteß mächtige
Kraft mit den Freunden. Von da ging ich nach Elawson in
Leieestershire im Tale Veavor, und auch dort wirkte die Kraft
Gotteß in Verschiedenen Städten und Dörfern, in denen Freunde
beisammen waren. Während ich dort war, offenbarte mir der
Herr drei Dinge, die sich auf die drei großen Berufßarten in der
Welt — Heilkunde, sogenannte Gotteßgelehrtheit und Recht?--H
wissenschast bezogen. Er zeigte mir, daß die Ärzte nicht die
Wei?-heit Gotteß haben, durch die alle Kreatur geschaffen ist, und
daß sie darum ihre Kräfte nicht kennen, weil sie nicht im Worte der
Weiß-heit sind, durch daß alleß gemacht ist. Gr zeigte mir, daß
die Priester nicht den wahren Glauben haben, dessen Ursprung
Christus ist; den Glauben, der reinigt und den Sieg gibt und
durch des man Gott gefällt, welcheß Geheimniß deß Glaubenß
in reinem Gewissen ist (1. Tim. 3, 9). Gr zeigte mir ferner, daß
die Rechtßgelehrten nicht die wahre Villigkeit und Gerechtigkeit
besitzen und nicht daß Gesetz Gotteß haben, nach welchem schon
die erste Ubertretung und alle weiteren Sünden gerichtet worden
sind und welcheß dem Geiste Gotteß entspricht, den die Menschen
in sich betrüben und gegen den sie sündigen (Eph. 4, 30).
Und daß diese drei, die Ärzte, die Priester und die Rechtßgelehrten,
die Welt ohne Weißheit regieren, ohne Glauben, ohne Billigkeit,
ohne Recht und ohne daß Gesetz Gotteß; die einen, indem sie
vorgeben, den Leib zu heilen, die andern die Seele und die dritten
daß Eigentum der Leute zu schützen. Aber ich sah, daß sie alle
die Weißheit, den Glauben, die Gerechtigkeit und daß GesetzZGotteS


% \picinclude{./020-029/p_s022.jpg} 
nicht hatten. Und als der Herr mir diese Dinge osfenbarte, fühlte
ich, daß seine Kraft sich über alle ergoß und daß sie durch die-
selbe alle umgewandelt werden könnten, wenn sie sie aufnehmen und
sich ihr beugen würden. Die Priester würden umgewandelt werden
und zum wahren Glauben kommen, welcher eine Gabe Gottes
ist. Die Rechtsgelehrten würden umgewandelt werden und zum
Gesetz Gottes (Jar. 2, 2) kommen, welches dem göttlichen im
Herzen entspricht und es möglich macht, seinen Nächsten wie sich
selbst zu lieben. Dieses Gesetz läßt den Menschen erkennen, daß
wenn er seinem Nächsten schadet, so schadet er sich selber, und
es lehret ihn, andern zu tun, wie er möchte, daß die andern ihm
tun. Die Ärzte können umgewandelt werden und zur Weisheit
Gottes kommen, durch die alle Dinge geschaffen sind, und so
eine rechte Erkenntnis über diese Dinge erlangen und ihre Kräfte
erkennen an den Namen, die die Weisheit, die sie gemacht, ihnen
gab ....
Der Herr offenbaite mir durch seine unsichtbare Kraft, daß
ein jeder erleuchtet werde durch das heilige Licht Christi (Joh. 1, 9).
Und ich erkannte, daß es in allen leuchtet, und daß alle, die
daran glaubten, aus der Verdammnis zum Licht des Lebens
kamen und Kinder des Lichts wurden (Joh. 12, 36). Aber die,
welche es haßten und nicht daran glaubten, die verdammte es, wie-
wohl sie schienen Christum zu bekennen.,« Solches sah ich in der
reinen Offenbarung des Lichts, ohne jegliche menschliche Hilfe;
auch wußte ich damals nicht, wo es in der Schrift zu sinden
war; doch später, als ich in der Schrift forschte, fand ich es.
Damals aber hatte ich jenes Licht und jenen Geist geschaut, welche
gewesen, ehe die Schrift gegeben worden war, und welche die
heiligen Männer Gottes getrieben hatten, die Schrift zu schreiben;
und ich erkannte, daß alle, welche Gott, Christus oder die Schrift
recht kennen wollen, zu diesem Geist gelangen müssen. Aber ich
merkte eine Trägheit und faule Schläfrigkeit in den Leuten, die
mich erstaunte; oftmals, wenn ich einschlafen wollte, schweifte
mein Geist über alles hinaus zu dem, der von Ewigkeit zu Ewig-
keit ist. Jch sah, daß der Tod über diesen schltisrigen und faulen
Zustand kommen mußte, und ich sagte den Leuten, sie müßten
dazu kommen, dieses schläfrige, träge Wesen zu töten und zu
kreuzigen durch die Kraft Gottes, damit ihre Herzen und Sinne
droben seien.


% \picinclude{./020-029/p_s023.jpg} 
Erste Versammlungen und Proteste. 23
Einmal alß ich durchs Feld wanderte, sagte der Herr zu mir:
,,Dein Name ist geschrieben im Lebenßbuche deß Lammeö, welcheö
gewesen vor der Erschaffung der Welt«. Alk- der Herr dietz sagte,
da glaubte ich e3 und erkannte es, kraft der neuen Geburt. Einige
Zeit darauf befahl mir der Herr, in die Welt hinaus zu gehen,
die wie eine dornige Wildniß war; und alß ich in der Kraft
Gottes mit dem Wort des Lebenö in die Welt hinauß kam, lehnte
sich die Welt dagegen auf und tobte wie die großen tobenden
Wogen der See; Priester wie ,,Fromme«, die Obrigkeit wie das
Volk, alle waren wie die See, als ich kam, den Tag deß Herrn
unter ihnen zu verkünden und ihnen Buße zu predigen ......
Als mich Gott und sein Sohn Jesuß Christuß außsandten
in die Welt, um sein ewigeö Evangelium und Reich zu predigen,
freute ich mich, daß ich den Befehl hatte, die Leute jenem innern
Licht, Geist und Gnade zuzuführen, durch die alle ihr Heil und
den Weg zu Gott erkennen können; ja, jenem heiligen Geist,
der in alle Wahrheit führt und von welchem ich bestimmt wußte,
daß er nie jemanden trtigt.
Durch diese göttliche Kraft und den Geist Gottes und daß
Licht Jesu sollte ich nun die Menschen von ihren eigenen Wegen
ab zu Christus?-, dem neuen, lebendigen Weg bringen; ab von
ihren Kirchen von Menschen gemacht, zur Kirche in Gott, zur
Gemeinde derHeiligen, die imHi1nmel angeschrieben ist (Gbr. 12, 23),
deren Haupt Ehristuß ist; ab von den Lehrern dieser Welt, die
von Menschen eingesetzt sind, damit sie von Ehristus:3 lernen, der
der Weg, die Wahrheit und daß Leben ist (Joh. 14, 6), von welchem
der Vater sagt: ,,dieS ist mein lisber Sohn, den höret« (Luc. 9, 35);
ab von allem weltlichen Gottezdienst, damit sie den Geist der Wahr-
heit in ihrem Jnnern erkennen und sich von demselben führen
lassen; daß sie in demselben den Vater der Geister anbeten, dem
solcheß anbeten angenehm ist; die, welche nicht in diesem Geiste
anbeten, wissen nicht, maß sie anbeten. Jch sollte die Menschen
abbringen von all den Gottesdiensten dieser Welt, welche eitel
sind, damit sie zu dem wahren Gotteßdienst kommen, welcher die
Witwen und Waisen in ihrer Trübsal tröstet (Jar. 1, 27) und be-
wahret von der Befleckung der Welt; dann gäbe es:) nicht so viele
Bettler, deren Anblick so ost mein Herz betrübt, weil er von so
viel Hartherzigkeit zeugt unter denen, die vorgeben, C-hristus3 zu
bekennen. Ich sollte sie von allen Gemeinschaften, Singereien


% \picinclude{./020-029/p_s024.jpg} 
und Betereien dieser Welt abbringen, welche Formen ohne Kraft
sind, auf daß ihre Gemeinschaft im heiligen Geist sei, im ewigen
Geist Gotteö, und sie darin anbeten und singen, durch die Gnade,
die von Christus kommt; und so dem Herm in ihren Herzen
singen’und spielen, der seinen geliebten Sohn gesandt hat, um
ihr Retter zu sein; der seine himmlische Sonne über und in allen
scheinen läßt und seinen himmlischen Regen über Gerechte und
Ungerechte außgießt (Matth. 5), wie der äußere Regen über alle
fällt und die äußere Sonne fiir alle scheint; dietz ist Gotteö un-
außsprechliche Liebe zur Welt. Jch sollte die Leute von den
jüdischen Zeremonien abbringen und von den heidnischen Fabeln
und den menschlichen Einrichtungen und weltlichen Lehren, durch
welche die Leute hin und her von einer Sekte zur andern ge-
trieben werden, und von allen ihren bettelhaften Lehranstalten
und ihren Schulen und Hochschulen, in denen sie Prediger Christi
machen wollen, die aber wahrlich Prediger ihrer eigenen Machen-
schaft sind und nicht Christi; von allen ihren Bildern und Kreuzen
und Besprengen von Kindern; allen ihren sogenannten heiligen
Tagen und nichtigen Traditionen, die sie seit den Tagen der
Apostel eingerichtet haben und gegen welche die Kraft Gottes
sich richtet; vermöge dieser Kraft wurde ich getrieben, gegen
alleß daß aufzutreten und gegen alle, die nicht umsonst pre-
digten und doch solche waren, die umsonst vom Herrn empfangen
hatten. s
Ferner verbot mir der Herr, als er mich in die Welt hinauö
sandte, meinen Hut abzunehmen vor irgendjemand, hoch oder
niedrig; und ich hatte den Befehl, zu allen, Männern und Frauen,
,,Du« zu sagen, ohne irgend einen Unterschied zu machen zwischen
reich oder arm, groß oder klein; und ich sollte unterwegs- auf
meinen Reisen den Leuten nicht guten Morgen oder guten Abend
sagen, noch mich vor irgendjemand neigen oder daß Knie beugen.
Solcheö machte die Sekten und Gemeinschaften zornig. Aber die
Kraft des Herrn half mir durch alleß hindurch, zu seiner Ehre,
und viele kehrten sich in kurzer Zeit zu Gott, denn der große
Tag des- Herrn ging auf auß der Höhe und brach eilendö an,
und in seinem Lichte gingen vielen die Augen über ihren Zu-
stand auf.
Aber o, die Wut, in welcher damals- Priester, Obrigkeit,
»Fromme« und andere waren! Aber hauptsächlich die Priester


% \picinclude{./020-029/p_s025.jpg} 
Erste Versammlungen und Proteste. 25
und die ,,Frommen«; denn obgleich das- ,,Du« gegen eine ein-
zelne Person ihrer eigenen Grammatik und Formenlehre, sowie
auch der Bibel entsprach, so konnten sie sich doch nicht drein
finden, es zu hören; und maß die Hut-Ehre anbetraf, daß ich
den Hut nicht vor ihnen abnehmen konnte, das machte sie ganz
wütend ....
In jener Zeit fühlte ich mich, zu meiner schweren Prüfung,
auch berufen, in die Gerichtßhöfe zu gehen, um nach Gerechtigkeit
zu schreien und die Richter und Behörden in Wort und Schrift
zur Gerechtigkeit zu mahnen; ich mußte solche, die öffentliche Gast-
häuser hielten, ermahnen, den Leuten nicht mehr zu trinken zu
geben, als ihnen gut sei; ich mußte auftreten gegen ihre Feste
und Gelage, Spiele, Späße und Belustigungen aller Art, durch
die die Leute zur Eitelkeit und Liederlichkeit verleitet und von
der Gotteßsurcht abgebracht wurden; am häufigsten schändeten
sie Gott (Röm. 2, 23) in dieser Weise an den Tagen, die sie als-
heilige bezeichneten. Auch an Jahrmärkten und Märkten mußte
ich mich gegen ihr trügerischeö Handeln wenden, ihren Schwindel
und Betrug; ich mußte sie mahnen, die Wahrheit zu sagen, ihr
ja—ja und ihr neinsnein sein zu lassen, und andern zu tun, wie
sie wollten, daß man ihnen tue, alleß indem ich sie an den großen
Tag dez Herrn erinnerte, der über sie alle kommen werde. Auch
gegen allerlei Musizieren und gegen die Schwindler, die in den
Vuden ihr Wesen trieben, mußte ich auftreten, denn sie gefähr-
deten die Unschuld und reizten den Sinn der Leute zur Eitelkeit.
Jch mußte auch manchen schweren Gang zu Lehrern und Lehrerinnen
tun, um sie zu erinahnen, die Kinder in der Furcht deö Herm zu
erziehen, damit sie nicht in Eitelkeit, Leichisinn und Schlechtigkeit
aufwachsen. Ebenso mußte ich Lehrer und Lehrerinnen, sowie die
Väter und Mütter ermahnen, darauf zu achten, daß man die
Kinder und die Dienstboten daheim im Hanse zur Gotteßstircht an-
halte, damit sie Vorbilder der Tugend und Mäßigkeit werden.
Die irdische Gesinnung der Priester tat mir weh, und wenn
ich die Glocken läuten hörte, welche die Leute inö Turnthauß
rufen sollten, ging es mir durch Mark gund Bein, denn eS war
gerade wie eine Marktglocke, welche die Leute zusammenruft, daß I
der Priester seine Ware Izum Verkauf außbieteu kann. O, die
großen Geldsummen, die zusammenkamen durch ihr Handeln mit
Bibeln und durch ihr Predigen, vom höchsten Bischof biz zum


% \picinclude{./020-029/p_s026.jpg} 
einfachsten Priester! Wa;-’ für ein Handel in der Welt kommt
diesem gleich! Und doch wurde die Schrift gegeben umsonst! Und
Christus hatte seinen Jüngern befohlen, umsonst zu predigen;
und die Propheten und Apostel verkündeten allen geizigen Miet-
lingen und allen, die für Geld iveiösagten, daß Gericht. Jch
aber wurde au?-gesandt, in diesem freien Geist daß Wort vom
Leben und der Versöhnung umsonst zu predigen, auf daß alle zu
Christus kommen, welcher umsonst gibt und in daß E-benbild
Gotteß erneuert, nach dem Mann und Weib geschaffen waren
vor dem Fall, auf daß sie himmlische Güter in Jesuß Ehristuß
haben möchten.
%%%%%%%%%%%%%%%%%%% Kapitel 3. %%%%%%%%%%%%%%%%%%%%%%%%%%%%%%
\chapter[Tumult in Nottingham]{Tumult in Nottingham}

\begin{center}
\textbf{Der Tumult in Nottingham. Wachsender Widerstand, bis zum
Gefängnis in Derby.}
\end{center}

\section{Tumulte bei dem Gottesdienst in Nottingham}
Als  ich einmal am Morgen eines Ersten Tages in der Nähe
von Nottingham\ort{Nottingham} von einem Hügel aus die 
Stadt überblickte, da
gewahrte ich das riesige Turmhaus, und der Herr sagte zu mir:
\zitat{Du musst hingehen und gegen jene großen Götzen schreien und
gegen die, welche drinnen anbeten}. Ich sagte den \textit{Freunden},
die mit mir waren, nichts davon, sondern ging mit ihnen hin in
die Versammlung, wo die mächtige Kraft des Herrn mit uns
war; hier lies ich sie und ging zum Turmhaus. Die Menge,
die ich hier sah, kam mir vor wie ein Brachfeld und der Priester
wie ein großer Erdklumpen, der oben auf seiner Kanzel stand.
Er hatte zum Text die Worte des Petrus: \zitat{Wir haben ein festes
prophetisches Wort und ihr tut wohl, das ihr darauf achtet, als
auf ein Licht, das da scheinet an einem dunkeln Ort, bis der Tag
anbreche und der Morgenstern aufgehe in eueren Herzen}
(2. Petr. 1:19\bibel{Petr. 2. 01:19@2. Petr. 1:19}). 
Er sagte den Leuten, nach dem, was hier geschrieben 
stehe, sollten sie alle Lehren, Bekenntnisse und Meinungen
prüfen. Da kam die Kraft des Herrn so mächtig über mich und
war so stark in mir, das ich nicht an mich halten konnte, sondern
rufen\index{Gottesdienst!Störung} musste: \zitat{O 
nein, nicht nach dem, was geschrieben stehet!}\index{Exegese}
und ich sagte ihnen, nach was: nämlich nach dem heiligen Geist,
durch den die heiligen Männer Gottes die Schrift geschrieben
haben. Durch diesen, sagte ich, müssen alle Lehren, Bekenntnisse
und Meinungen geprüft werden. Dieser Geist leitet in alle
% \picinclude{./020-029/p_s027.jpg} 
Wahrheit und zur Erkenntnis aller Wahrheit. Die 
Juden\index{Juden} haben
die Schrift gehabt und widerstanden dem heiligen Geist doch und 
verwarfen Christus, den schönen Morgenstern ; sie verfolgten Christus
und seine Apostel und wollten ihre Lehren nach der Schrift prüfen;
aber sie irrten in ihrem Urteil und prüften sie nicht richtig, weil
sie ohne den heiligen Geist prüften. Da ich nun so zu ihnen
redete, kamen die Wachen und führten mich weg und brachten
mich in einen wüsten, stinkenden Kerker; der Geruch stieg mir so
in die Nase und den Hals, das es eine Qual war, aber die
Kraft des Herrn schallte an dem Tage so in ihren Ohren, das
sie ganz von dem Schall betäubt waren, und ihre Ohren wurden
noch eine zeitlang nicht frei davon, so waren sie im Turmhause
von der Kraft des Herrn ergriffen worden. 


\section{Bekehrung des Sheriff John Neckles}

Am Abend brachten
sie mich vor die Behörden der Stadt; als ich vor sie trat, war
der Bürgermeister in verdrieslicher, mürrischer Laune, aber die
Kraft des Herrn beschwichtigte ihn. Sie verhörten mich 
ausführlich und ich berichtete ihnen, wie der Herr mich 
getrieben hatte
zu kommen. Nach einigem Hin- und Herreden schickten sie mich
ins Gefängnis zurück. Aber bald darauf lies mich der Ober-Sheriff,
John Neckles\person{Neckles, John (Sheriff)}, zu sich in 
sein Haus holen. Als ich eintrat, begegnete mir sein Weib 
im Flur und sagte: \zitat{Unserm Hause ist
Heil widerfahren.} Sie reichte mir die Hand und war mächtig
ergriffen von der Kraft Gottes, und ihr Mann und ihre Kinder
und Dienstboten wurden ganz umgewandelt, denn die Kraft des
Herrn war mächtig in ihnen. Ich wohnte bei ihnen und wir
hatten große Versammlungen in ihrem Hause; es kamen auch
etliche angesehene Standespersonen, und des Herrn Kraft tat sich
mächtig kund unter ihnen; John 
Reckles\person{Reckles, John} lies dann einen andern
Sheriff holen und eine Frau, mit der sie in Geschäften zu tun
gehabt hatten, und erklärte in Anwesenheit des andern Sheriff,
das sie beide diese Frau bei einem Handel geschädigt hätten und
sie entschädigen müssten. Er sagte es sehr freundlich, aber der
andere Sheriff leugnete, und die Frau sagte, sie wisse nichts 
davon. Aber der gerechte Sheriff sagte, es sei so, und der andere
wisse das ganz gut; nachdem er die Sache aufgedeckt und das
Unrecht, das sie getan, eingestanden hatte, entschädigte er die
Frau und ermahnte den andern ein gleiches zu tun; die Kraft
Gottes war mit diesem guten Sheriff und wirkte eine große
Wandlung in ihm und er hatte große Offenbarungen. Als er
% \picinclude{./020-029/p_s028.jpg} 
am darauf folgenden Marktag in den Pantoffeln in seinem
Zimmer auf- und abging, sagte er: \zitat{Ich muss auf den Markt
gehen und den Leuten Buße predigen,} und er ging auf den Markt
und in mehrere Straßen und predigte den Leuten Buße; und auch
noch andere aus der Stadt trieb es, zu den Behörden zu gehen
und die Leute zur Buße zu ermahnen. Die Räte wurden sehr
böse über mich und ließen mich aus dem Hause des Sheriff
holen und verurteilten mich zum Gefängnis. Als die 
Gerichtssitzung stattfand, fühlte einer sich 
getrieben, sich statt meiner anzubieten, \zitat{Leib 
um Leib, Leben um Leben}. Als ich vor den
Richter gebracht werden sollte, ging es ziemlich lang, bis mich
der Diener, der mich hinbringen sollte, abholte, und als ich kam,
hatte sich der Richter schon erhoben, woraus ich sah, das er 
erzürnt war; er sagte, er wolle dem Jüngling schon einen Verweis
geben, wenn er vor ihn gebracht werde; ich war damals unter
dem Namen \zitat{Jüngling} eingesperrt. Ich wurde denn wieder
ins Gefängnis gebracht. Die Kraft des Herrn war mächtig
unter den \textit{Freunden}, aber das Volk fing an, tätlich zu
werden, so das der Schloskommandant Soldaten hinaus schickte,
um die Leute auseinander zu treiben, worauf es ruhig wurde;
alle, Priester und Volk, erstaunten ob der herrlichen Kraft, welche
hervorbrach, und etliche der Priester wurden empfänglich gemacht
und einige von ihnen bekannten sich zur Kraft Gottes.

\section{Krankenheilung, Predigt und Misshandlung in Woodhouse}

Nachdem ich aus dem Gefängnis von Nottingham, wo ich
einige Zeit gefangen gewesen war, entlassen worden, zog ich
umher, wie vorher im Dienst des Herrn. Als ich 
nach Mansfield\ort{Mansfield}
Woodhouse\ort{Woodhouse} kam, war dort eine verrückte 
Frau\index{Wahnsinn}; das Haar hing
ihr wirr über die Ohren und der Arzt war gerade bei ihr. Er
war daran, ihr zu Ader zu lassen, nachdem man sie zuvor 
gebunden hatte; viele Leute waren um sie und hielten sie mit 
Gewalt fest, aber man konnte ihr kein Blut entziehen. Ich befahl,
das man sie frei mache und ruhig lasse, denn sie konnten dem
Geiste, der sie plagte, nicht beikommen; sie machten sie frei und
es trieb mich, zu ihr zu reden und sie im Namen des Herrn still
und ruhig sein zu heißen, und sie war es; die Kraft des Herrn
beruhigte ihr Gemüt und sie genas\index{Heilung}, und sie nahm die Wahrheit
auf und blieb darin bis zu ihrem Tod. Des Herrn Name wurde
verherrlichet, ihm gebührt die Ehre aller seiner Werke [...].


Während ich in Mansfield Woodhouse war, trieb es mich,
% \picinclude{./020-029/p_s029.jpg} 
ins Turmhaus zu gehen, um den Leuten die Wahrheit zu 
verkünden, aber das Volk fiel in großem Zorn über mich her, sie
schlugen mich zu Boden und erstickten mich fast; ich war arg 
zerschlagen und zerquetscht von ihren Händen, Bibeln und 
Stöcken.\index{Misshandlung}
Dann schleppten sie mich hinaus, wie wohl ich kaum fähig war
zu stehen, und taten mich in den Stock, wo ich einige Stunden
saß. Sie brachten Hundepeitschen und Pferdepeitschen und drohten
mir damit. Dann musste ich vor die Behörden im Hause eines
Adligen, wo viele angesehene Leute zugegen waren. Als diese
sahen, wie ich misshandelt worden war, gaben sie mir nach
Vielen Drohungen die Freiheit. Aber der Pöbel trieb mich
zur Stadt hinaus zum Dank dafür, das ich ihnen das Wort des
Lebens verkündet hatte. Ich war kaum imstande zu stehen und
zu gehen, so übel hatten sie mich zugerichtet. Mit großer 
Anstrengung ging ich etwa eine Meile weit vor die Stadt, wo ich
Leute traf, die mir etwas zur Erquickung gaben, denn ich war
innerlich ganz auseinander, aber die Kraft des Herrn heilte mich
bald wieder. Es waren aber an dem Tage etliche von der 
Wahrheit des Herrn überzeugt worden, worüber ich mich freute [...].

\section{Gefängnisbesuch in Coventry und erste Begegnung mit Ranters}

An einem Ersten Tage kamen wir nach Bagworth\ort{} und gingen
ins Turmhaus, wohin einige der Freunde gebracht worden waren;
das Volk schloss sie darin ein und sich selbst mitsamt ihrem
Priester. Als der Priester fertig geredet hatte, machten sie die
Türe auf und wir gingen auch hinein und hatten einen 
Gottesdienst mit ihnen, und hernach hatten wir eine Versammlung in
der Stadt, mit manchen angesehenen Leuten. Als ich weiter zog,
hörte ich von solchen, die in Coventry\ort{Coventry} um ihres Glaubens willen
gefangen waren. Aber als ich unterwegs zu ihrem Gefängnis war,
geschah das Wort des Herrn zu mir: \zitat{Meine Liebe war immer
mit dir und du bist in meiner Liebe}. Und ich fühlte mich 
gehoben in der Liebe Gottes und sehr gestärkt an meinem innern
Menschen. Als ich in den Kerker zu den Gefangenen kam, über-
kam mich eine große Finsternis; ich hielt stille, denn mein Geist
ruhte in der Liebe Gottes. Schließlich fingen die Gefangenen
an zu prahlen, und lärmten und lästerten, worüber meine
Seele sehr betrübt wurde. Sie sagten, das sie Gott seien, aber
wir konnten solches nicht ertragen. Als sie ruhig geworden
waren, stand ich auf und fragte sie, ob sie solches aus innerem
Trieb oder auf Grund der Schrift täten? Sie sagten: \zitat{auf
% \picinclude{./030-039/p_s030.jpg} 
Grund der Schrift.} Da eine Bibel zur Hand war, hieß ich sie,
mir die betreffende Stelle zu zeigen, und sie zeigten mir die Stelle,
wo das Tuch vor Petrus herabgelassen wurde und die Stimme
sagte: ,\zitat{Was Gott gereinigt hat, das mache du nicht gemein}
(Act. 10:15\bibel{Act. 10:15}). Als ich ihnen zeigte, das diese Stelle nichts für
sie beweise, brachten sie eine andere vor, die davon handelte, wie
Gott alle mit sich selbst versöhnt im Himmel und auf Erden
(Col. 1:20\bibel{Col. 01:20@Col. 1:20}). Ich sagte ihnen, 
das ich diese Stelle ebenfalls anerkenne, das sie aber 
ebensowenig für sie passe. Als ich nun
vernahm, wie sie sagten, sie seien Gott, fragte ich sie, ob sie
wissen, ob es morgen regnen werde? Sie antworteten, das sie
das nicht sagen könnten. Ich erwiderte ihnen: Gott könne das
sagen. Darauf fragte ich sie, ob sie immer so bleiben würden,
wie sie jetzt seien, oder ob sie sich ändern würden? Sie 
antworteten: sie wüsten es nicht. Ich erwiderte: \zitat{Gott 
kann es sagen und Gott verändert sich nicht. Ihr sagt, ihr seid Gott und
wisst nicht, ob ihr euch verändert oder nicht?} Sie wurden 
verwirrt und für den Augenblick fast überwunden. Nachdem ich sie
wegen ihrer Gotteslästerungen zurecht gewiesen hatte, ging ich
fort, denn ich merkte, das sie 
Ranter\index{Ranter}\footnote{Ranter, eine Sekte von mystischen 
Schwärmern, die sich rühmten, das Christus in ihnen wohne, 
aus ihnen rede und sie selbst Christus seien; daher der 
Spottname \zitat{Ranter} -- Prahler.} waren. Ich war nie
mit solchen zuvor zusammengetroffen und ich pries die Güte des
Herrn, das sie mir erschienen war, ehe ich zu ihnen gekommen
war. Nicht lange nachher schrieb einer dieser Ranters, namens
Joseph Salmon\person{Salmon, Joseph}, ein Buch, in dem er 
widerrief, worauf sie die Freiheit erhielten ]...].

\section{Ermahnung der Steuereinnehmer in Leicestershire}

Bei meinem Herumziehen auf den Jahrmärkten\index{Jahrmarkt} und Märkten
und in den Städten, sah ich Tod und Finsternis in allen, welche
die Kraft des Herrn nicht ergriffen hatte. Als ich 
durch Leicestershire\ort{Leicestershire}
zog, kam ich nach Twycross\ort{Twycross}; daselbst waren Steuereinnehmer.
Der Herr trieb mich zu ihnen zu gehen und sie zu ermahnen,
sich vor Unterdrückung der Armen zu hüten. Das machte den
Leuten einen großen Eindruck. Es war in jener Stadt ein 
angesehener Mann, welcher lange krank gewesen war und von den
Ärzten aufgegeben wurde; und etliche Freunde aus der Stadt
wünschten, das ich zu ihm gehe. Ich ging zu ihm hinauf in sein
Zimmer und sagte ihm das Wort des Lebens, und es trieb mich,
% \picinclude{./030-039/p_s031.jpg} 
mit ihm zu beten. Und der Herr erhörte uns und machte ihn
gesund\index{Krankenheilung}. Als ich aber darauf 
in einem unteren Raum des Hauses
zu der Dienerschaft und einigen andern Anwesenden redete, stürzte
einer aus einem Nebengemach herein mit dem nackten Degen in
der Hand, gerade auf mich los. Ich sah ihn unerschrocken an
und sagte: \zitat{Wehe Dir, arme Kreatur, was willst Du tun mit
Deiner fleischlichen Waffe? mir ist sie nicht mehr als ein 
Strohhalm.} Die Anwesenden waren sehr bestürzt und er entfernte
sich in Zorn und Wut. Als sein Herr davon hörte, entließ er
ihn aus feinem Dienst. Also beschützte mich der Herr und half
diesem Schwachen und er wurde später den \textit{Freunden} sehr
zugetan; und als ich wieder in jene Stadt kam, besuchte er mich
mit seinem Weibe [...].

\section{Gefangennahme in Derby}

Als ich nach Derby\ort{Derby} kam, wohnte ich im Hause eines Arztes.
Eine Frau wurde gewonnen und noch Viele andere. Als ich in
mein Zimmer ging, läutete die Glocke des Turmhauses; nur schon
sie zu hören, ging mir durch Mark und Bein; ich fragte warum
die Glocke läute? man sagte mir, das an dem Tage eine große
gottesdienstliche Versammlung stattfinde, dazu viele aus dem
Heer, sowie Priester und Prediger kommen werden. Da trieb
es mich, auch hin zu gehen; und als sie fertig waren, redete ich
zu ihnen, was. mir der Herr eingab. Sie waren ziemlich ruhig;
aber eine Wache kam, nahm mich bei der Hand und sagte, ich
müsse vor den Rat sowie auch die andern beiden, die mit mir
waren. Um die erste Nachmittagsstunde wurde ich vorgenommen.
Ich wurde gefragt, warum ich hingegangen sei. Ich sagte, Gott
habe mich getrieben, es zu tun, und weiter sagte ich. \zitat{Gott
wohnet nicht in Tempeln mit Händen gemacht.} Ich sagte ihnen
ferner, all ihr Predigen, ihr Taufen\index{Taufe} und ihr Opfern werde sie
nie heiligen, und ermahnte sie, auf Christum in ihnen zu schauen
und nicht aus Menschen; denn Christus sei es, welcher sie heilige.
Darauf ergingen sie sich in Vielen Worten, aber ich sagte ihnen,
sie sollten sich nicht über Gott und Christus streiten, sondern
ihm gehorchen\index{Ökumene}. Die Kraft Gottes donnerte unter ihnen und
sie zerstoben davor wie Spreu. Sie hießen mich mehrmals
aus dem Zimmer gehen und dann wieder hereinkommen und
trieben mich hin und her; von ein Uhr an bis abends neun 
verhörten sie mich. Zuweilen sagten sie mir mit höhnischen Worten,
ich sei nicht bei Sinnen. Zuletzt fragten sie mich, ob ich 
geheiligt\index{Heiligung}
% \picinclude{./030-039/p_s032.jpg}
sei; ich antwortete: \zitat{Ja, denn ich war im Paradies\index{Paradies} Gottes}
(2. Cor. 12:4\bibel{Cor. 2. 12:04@2. Cor. 12:4}). Dann fragten 
sie mich, ob ich keine Sünde\index{Sünde} habe.
Ich antwortete: \zitat{Christus, mein Erlöser, hat die Sünde von mir
genommen und in ihm ist keine Sünde.} Sie fragten, wie ich
wüste, das Christus in uns wohne? Ich sagte: \zitat{Durch seinen
Geist, den er uns gegeben.} Um mich zu versuchen, fragten sie,
ob einer von uns Christus sei? Ich antwortete: \zitat{Nein, wir
sind nichts, Christus ist alles.} Sie sagten: wenn ein Mann
stehle, ob das keine Sünde sei? Ich antwortete: \zitat{Alles Unrecht
ist Sünde.} Als sie es nun müde geworden, mich zu verhören,
verurteilten sie mich zu sechs Monaten im Korrektionshaus in
Derby als Gotteslästerer, wie aus folgendem Verhaftbefehl zu
ersehen ist:

\grosszitat{
    An den Oberaufseher des Korrektionöhauses in Derby.
    \bigskip

    Hiermit senden wir euch die Personen George Fox, vormals
    in Mansfield in der Grafschaft Nottingham, und John Fretwell,
    Landwirt, vormals in Staniesby in der Grafschaft Derby, vor
    uns gebracht am heutigen Tag und beschuldigt eingestandener
    Äußerungen verschiedener gotteslästerlicher Ansichten, die einem
    jüngst verfassten Parlamentsbeschluss\footnote{Partamentsbeschlus 
    vom 2. Mai 1648\jahr{1648} gegen Gotteslästerung und Ketzerei. 
    Ein Beschluss, 
    der von der unglaublichen Härte der damals regierenden
    Presbyterianer\index{Presbyterianer} zeugt.} zuwider sind; 
    sie sollen daher
    sogleich nach Einsicht Dieses aufgenommen werden, besagter
    George Fox und Johann Fretwell\person{Fretwell, Johann}, in 
    euern Gewahrsam und
    darin sicher verwahrt werden, für die Dauer von 6 Monaten,
    ohne Möglichkeit einer Bürgschaft oder Abkürzung, es wäre denn,
    das sie sich hinlänglich durch ein gutes Betragen ausweisen, oder
    durch unsere eigene Verordnung frei würden. Solches zu tun
    möget ihr nicht versäumen.
    \bigskip
    \begin{flushright}
    Mit unsrer Hand und Siegel gegeben am heutigen Tage
    30.~Oktober~1650. Ger. Bennet. Nath. Barton.\end{flushright}
}

Während ich im Gefängnis war, kamen oft \textit{Fromme}, um eine
Unterredung mit mir zu haben; noch ehe sie etwas sagten, merkte
ich immer, das sie kamen, um für die bleibende 
Sündhaftigkeit\index{Sümdhaftigkeit} und 
Unvollkommenheit\index{Unvollkommenheit} einzutreten. Ich 
fragte sie, ob sie gläubig seien und
% \picinclude{./030-039/p_s033.jpg} 
Glauben hätten? Sie sagten: \zitat{Ja.} Ich fragte sie: in wen?
Sie sagten: \zitat{In Christus.} Ich erwiderte: \zitat{Wenn ihr wahre
an Christus Glaubende seid, so seid ihr vom Tode zum Leben
eingegangen, und wenn ihr vom Tode frei seid, dann seid ihr es
auch von der Sünde, die den Tod bringt. Und wenn euer
Glaube wahr ist, so wird er euch den Sieg geben über Sünde
und Teufel\index{Teufel} und eure Herzen und Gewissen 
reinigen -- denn der
wahre Glaube ist in reinen Gewissen 
(1 Tim. 3\bibel{Tim. 1. 03@1 Tim. 3}) und er wird
machen, das ihr Gott gefallet und euch wieder Zugang zu ihm
Verschaffen.} Aber sie wollten nicht von Reinheit und von Sieg
über Sünde und Teufel hören; denn sie sagten, sie können nicht
glauben das jemand könne frei von Sünde sein schon diesseits
des Grabes. Ich hieß sie, das Schwatzen über die Schrift, die
das Wort heiliger Männer sei, aufgeben, wenn sie für Unheiligkeit
eintreten wollten. Einmal kam auch eine Anzahl solcher \textit{Frommer}
zu mir und fingen an, die Sündhaftigkeit zu befürworten. Ich
fragte sie: ob sie Hoffnung hätten? \zitat{Ja, ja! das wäre, wenn
wir keine Hoffnung hätten!} Ich fragte sie: \zitat{Was für eine
Hoffnung ist es, die ihr habt? Ist Christus in euch die Hoffnung
eurer Herrlichkeit? (Col. 1:27\bibel{Col. 01:27@Col. 1:27}) 
Reinigt sie euch, gleich wie er
rein ist?} Aber sie wollten nichts davon hören, das sie selber
hienieden schon rein werden sollten. Darauf gebot ich ihnen,
nicht mehr über die Schrift\index{Bibel} zu reden, welche das Wort heiliger
Männer sei. Denn die heiligen Männer, welche die Schrift 
geschrieben haben, seien für Heiligkeit in Herz, Leben und Wandel
hienieden eingetreten. \zitat{Ihr aber}, sagte ich, \zitat{tretet für Unreinheit
und Sünde ein, die vom Teufel sind, was habt ihr zu schaffen
mit den Worten heiliger Männer?}


\section{Bekehrung des Kerkermeister und Besuch anderer Gottesdienste}

Der Kerkermeister, ein großer \textit{Frommer}, hatte eine 
schreckliche Wut auf mich und redete sehr schlecht von mir. Aber es
gefiel dem Herrn, ihn eines Tages so mächtig zu ergreifen, das
er in großer Angst und innerer Not war. Als ich in meinem
Zimmer umherlief, hörte ich klägliche Laute und hörte, wie er zu
seiner Frau sagte: \zitat{Frau, ich habe den Tag des Gerichts gesehen,
und George Fox war da, und ich hatte Angst vor ihm, weil ich
ihm so viel böses zugefügt hatte und so vieles wider ihn zu den
Vorgesetzten und Frommen gesagt hatte und zu den Richtern
und in den Wirtshäusern.} Hierauf kam er gegen Abend zu mir
ins Zimmer und sagte: \zitat{Ich bin gegen euch gewesen wie ein
% \picinclude{./030-039/p_s034.jpg} 
Löwe; nun aber komme ich wie ein Lamm und wie der 
Kerkermeister, der zitternd zu Paulus und Silas kam.} Und er bat,
das er bei mir bleiben dürfe. Ich sagte, ich sei in seiner Macht
und er könne mit mir machen, was er wolle; aber er sagte:
nein, er wolle meine Erlaubnis haben, und er möchte, das er
immer mit mir sein könnte, aber nicht mich als Gefangenen haben;
er und sein Haus seien meinetwegen geplagt gewesen. Ich erlaubte
ihm denn, bei mir zu sein, und er öffnete mir sein Herz und
sagte, er glaube, das das, was ich vom wahren Glauben und
von der wahren Hoffnung sage, wahr sei, und er wunderte sich,
das der andere, der mit mir gefangen war, nicht dabei bleibe.
Er sagte: \zitat{Jener andere tat unrecht, ihr aber seid ein Gerechter.}
Er gestand mir auch, das oft, wenn ich ihn gebeten hatte, mich
unter das Volk gehen zu lassen, um ihnen das Wort des Herrn
zu verkünden, und er es mir verweigert habe, habe er sich damit
eine große Last auferlegt; denn er sei in große Angst geraten
und einige Zeit ganz verstört und niedergedrückt gewesen, so das
er gar keine Kraft mehr gehabt habe. 

Am Morgen ging er fort
und ging zu den Richtern und sagte ihnen, wie er und sein Haus
meinetwegen geplagt gewesen seien, und einer der Richter erwiderte
ihm, das auch sie geplagt seien, darum das sie mich festhielten.
Es war Richter Bennet\person{Richter Bennet} zu Derby, 
welcher uns zuerst 
Quäkers\index{Quaker!nahmensentstehung}\footnote{Quüker, 
das heist \zitat{Zitterer}, der Spottname, den die Gegner den
Freunden abhängten, wegen der in ihren ersten Versammlungen 
sich einstellenden Konvulsionen. \textbf{Anmerkung Olaf Radicke:} Das 
ist eine original Fußnote, und offensichtlicher Unsinn! Da hier
G. Fox selbst den Grund des Namens nennt, der plausibler scheint.}
genannt hatte, weil ich ihnen gesagt hatte, sie müssten erzittern
vor dem Wort Gottes. Solches geschah im Jahre 1650\jahr{1650}.

Hierauf erlaubten mir die Richter, eine Meile weit zu gehen.
Ich sah, wo sie hinaus wollten und sagte dem Kerkermeister,
wenn sie mir zeigen wollten, wie weit eine Meile sei, so wolle
ich manchmal so weit gehen; denn ich glaube, sie dachten, ich
würde davon laufen. Und der Kerkermeister gestand nachher,
das sie es in dieser Absicht gestattet hätten, damit ich entkomme
und sie von ihrer Angst befreit würden; aber ich sagte ihm, das
ich nicht diesen Geist habe. 

Dieser Kerkermeister hatte eine Schwester, ein kränkliches
junges Weib. Sie kam zu mir, um mich zu besuchen; und nach
dem sie einige Zeit bei mir gewesen war, und ich Worte der
Wahrheit zu ihr geredet hatte, ging sie hinunter und sagte den
% \picinclude{./030-039/p_s035.jpg} 
andern, wir seien unschuldige Leute und täten niemand nichts zu
leide, sondern allen nur Gutes, sogar solchen, die uns hasten,
und bat sie, freundlich gegen mich zu sein. [...].

Während ich im Korrektionshaus war, besuchten mich meine
Verwandten\index{Verwandten}, und da sie über meine Gefangenschaft bekümmert
waren, gingen sie zu den Richtern und baten sie, das ich mit
ihnen heim gehen dürfe. Sie erboten sich, sich mit hundert Pfund
zu verbürgen und einige andere aus Derby, die mit ihnen waren,
je mit fünfzig Pfund, das ich nicht mehr dorthin komme, um
gegen die Priester zu reden. So wurde ich vor die Richter
gebracht, und weil ich nicht einwilligen wollte, das irgendjemand
sich meinetwegen verpflichte, -- denn ich war ja keines Vergehens
schuldig und hatte das Wort des Lebens und der Wahrheit geredet, --  
erhob sich Richter Bennet zornig, und als ich niederkniete, um
Gott zu bitten, ihm zu vergeben, rannte er auf mich los und schlug
mich mit beiden Händen\index{Misshandlung} und schrie: \zitat{Fort mit 
ihm! Kerkermeister, nimm ihn fort!} Hierauf wurde ich wieder in den Kerker gebracht
und musste dort bleiben, bis meine Zeit von sechs Monaten um
war. Aber ich durfte nun eine Meile weit allein gehen, was ich
tat, als ich fühlte, das ich es durfte. Oft ging ich auf den Markt
und in die Straßen und ermahnte die Leute, sich von ihrer
Schlechtigkeit zu bekehren, und ging dann wieder ins Gefängnis.
Und da Leute von allerlei Religionen mit mir im Gefängnis
waren, ging ich hie und da zu ihnen und wohnte ihren 
Versammlungen an den Ersten Tagen bei\index{Ökumene} [...].
%%%%%%%%%%%%%%%%%%% Kapitel 4. %%%%%%%%%%%%%%%%%%%%%%%%%%%%%%
\chapter[Kampf gegen die Ranter]{Kampf gegen die Ranter}

\begin{center}
\textbf{Erlebnisse im Gefängnis zu Derby. Ein \zitat{Wehe} 
über die Stadt
Lichfield\ort{Lichfield}. Erste Missionsgenossen. Antikirchliche 
Agitation und Kampf gegen die Ranter\index{Ranter}.}
\end{center}


Während ich noch im Gefängnis war, kam ein Soldat zu
mir und erzählte hmir, wie er im Turmhause gewesen sei und dem
Priester zugehört habe, und wie dann auf einmal eine grofze Angst
über ihn gekommen sei und die Stimme des Herrn also zu ihm
geschehen sei: ,,Weißt du nicht, daß mein Diener im Gefängnis
ift? zu ihm gehe und frage ihn um Rat«. Jch redete mit ihm
wie es sein gegenwärtiger Zustand erheischte, und sein Verständnis
zbt


% \picinclude{./030-039/p_s036.jpg} 
wurde geöffnet. Jch sagte ihm, daß der, welcher ihm seine Sünden
ausdecke und ihn um ihretwillen ängstige, ihm auch die Rettung
zeigen werde; denn der dem Menschen die Sünden aufdeckt, ist
derselbe, der sie auch hinwegnimmt. Während ich mit ihm redete,
offenbarte sich ihm der Herr, so daß er anfing, die Wahrheit dez
Herrn und Gorteß Gnade zu erkennen; er fing an, unerschrocken
in seinem Regiment unter den Soldaten von der Wahrheit zu
reden; denn die Schrift wurde ihm mehr und mehr offenbar, und
er ging soweit zu sagen: sein Oberst sei blind wie Nebukadnezar,
daß er den Diener deß Herrn inß Gesängnis werfe. Von da an
hegte sein Oberst einen Groll gegen ihn. Alß im darauffolgenden
Jahre in der Schlacht von Worcester die beiden Armeen neben-
einander lagen, kamen zwei auß der Armee des Könige und
forderten, daß zwei anß der Armee des Parlamentß sich mit ihnen
schlagen sollten; da wählte der Oberst ihn und noch einen, um
der Forderung Folge zu leisten. A13 sein Kamerad im Kampfe
gefallen war, trieb er seine beiden Gegner zur Stadt hinauö, ohne
einen Schuß auf sie abzuseuern; dies erzählte er mir nach seiner
Rückkehr mit eigenem Munde. Nach Beendigung der Schlacht
sah er die Betrügerei und Heuchelei der Offiziere ein, und im
Gedanken daran, wie wunderbar der Herr ihn bewahrt hatte und
waß etz eigentlich um den Krieg sei, legte er die Waffen nieder.
Die Zeit meiner Gefangenschaft war nun fast zu Ende und
da viel neue Soldaten aus-gehoben wurden, so wollten mich die
Kommifsäre zu ihrem Hauptmann machen, und die Soldaten
erklärten, sie wollten keinen andern als mich haben. Der Kerken
meister erhielt den Befehl, mich vor die Soldaten und ihre Vor-
gesetzten aus den Marktplatz zu siihren; dort boten sie mir dieseß
Ehrenamt, wie sie eß nannten, an und fragten mich, ob ich nicht
wolle die Waffen ergreifen für den Commonwealth gegen Karl
Stuart.1) Ich erwiderte ihnen, ich wisse wohl, woher aller
Krieg komme: auö der Begierde, wie schon Jakobus- lehre
(Jak. 4); ich aber stehe in jener Kraft und jenem Leben,
die von vornherein allen Krieg ausschließen. Sie wollten mich
überreden, ihr Anerbieten anzunehmen; sie meinten, ich weigere
mich nur aus Bescheidenheit. Aber ich erklärte ihnen, ich sei in
den Bund des- Friedenö eingetreten, welcher bestanden, ehe es
1) 1651 Schlacht von W1-rceöter zwischen Cron1toell(Con1n1onwealth) und
Karl ll.


% \picinclude{./030-039/p_s037.jpg} 
Erlebnisse im Gefängnis zu Derby usw. 37
Krieg und Zank gab. Sie sagten, sie bieten es mir in Liebe und
Zuneigung an wegen meiner Tugend, und ähnliche Schmeicheleien
mehr. Aber ich sagte ihnen, wenn solches ihre Liebe sei, so trete
ich sie mit Füßen. Da wurden sie zornig und sagten: ,,Nimm
ihn hinweg, Kerkermeister, und wirs ihn in den untersten Kerker
zu den Schelmen und Verbrechern.« Jch wurde weggefiihrt und
an einen wiisten, stinkenden Ort 1) gebracht, wo kein Bett war,
mit 30 Verbrechern, wo ich beinahe ein halbes Jahr gefangen
war, außer, wenn sie mich dann und wann ein wenig in den
Garten ließen, weil sie sicher waren, daß ich nicht davon laufe.
Es hatte damals, als man mich in diesen Kerker gebracht hatte,
geheißen, ich werde wohl nicht mehr heraus kommen. Aber ich
glaubte an Gott und daß ich zu seiner Zeit daraus befreit werde.
Denn der Herr hatte es mir vorausgesagt, daß ich nicht bald
von diesem Ort wegkomme, da ich dort eine Ausgabe für ihn zu
erfüllen habe.
Als es bekannt wurde, daß ich im Kerker von Derby sei, kamen
meineAngehörigen, um mich wieder zu besuchen; denn sie betrachteten
es als eine große Schande für sie, daß ich um der Religion
willen gefangen war; und etliche hielten mich für verrückt, weil
ich für die Reinheit, Gerechtigkeit und Vollkommenheit eintrat.
Unter denen, die zu mir kamen, war einer aus Nottingham,
ein Soldat, der früher Baptist gewesen war. Jin Laufe des Ge-
sprächs sagte er zu mir: »Dein Glaube gründet sich auf einen
Mann, der in Jerusalem gestorben sein soll; solches ist aber nie
geschehen«. Gs betrübte mich sehr, ihn so reden zu hören, und
ich sagte: ,,Wie! hat nicht Christus gelitten vor den Toren Jeru-
salems durch die Juden, die ,,Frommen«, die Hohenpriester und
durch Pilatus?« Aber er leugnete, daß Christus je äußerlich
gelitten habe. Jch fragte ihn, ob denn keine Hohenpriester, keine
Juden, kein Pilatus äußerlich dort gewesen sei? und als er das
nicht bestreiten konnte, sagte ich: ,,So gewiß ein Hohepriester,
ein Pilatus und Juden äußerlich dort gewesen sind, so gewiß ist
Christus äußerlich verfolgt worden von ihnen und hat durch sie
1) Die Zustände der Gefängnisse und Korrektionshäuset im 17. Jahrh.
waren überaus traurig. Überall herrschte große Unreinlichkeit; die Verwaltung
war der Willkür des Gefängnisvotstehers anheim gegeben, der nicht besoldet
war, sondern von den Gefangenen bezahlt wurde, die die Kosten ihres Aufent-
haltes selbst tragen mußten. Vgl. Aschrott, Englisches Gesängniswesen.


% \picinclude{./030-039/p_s038.jpg} 
gelitten«.' Die Reden dieses Menschen veranlaßten eine Ver-
leumdung gegen uns, als ob die Quäker bestritten, daß Christus
gelitten habe und in Jerusalem gestorben sei. Gs war dies ganz
falsch; nie war der leiseste Gedanke daoon in unsern Herzen ge
wesen; es war eine bloße Verleumdung, die uns traf, und die
aus dem Gerede dieses Menschen entstanden war. Derselbe
Mässch behauptete auch, niemals habe irgend ein Apostel oder
Prophet, oder Heiliger oder Mann Gottes äußerlich gelitten; alle
ihre Leiden seien innerlich gewesen; aber ich bewies ihm anBei-
spielen, wie viele unter ihnen gelitten und durch wen sie gelitten;
und so widerlegte die Kraft des Herrn seine oerkehrten Ansichten.
Eine andere Sorte kam zu mir, die behaupteten, sie könnten
Geister unterscheiden. Jch fragte sie, welches der erste Schritt
zum Frieden sei? und in was der Mensch seine Rettung suchen
müsse? Sie fuhren auf und sagten in ihrem Hochmut, ich sei
verrückt; und solche wollten Geister unterscheiden können und
kannten nicht einmal ihren eigenen Geist!
Während dieser Zeit meiner Gefangenschaft geriet ich in
große Bekümmernis über das Vorgehen der Richter und Beamten
in ihren Gerichtshösen. Gs trieb mich, an die Richter zu schreiben,
darum daß sie das Todesurteil fällten wegen allerlei unwichtiger
Vergehen, in Geldsachen oder das Vieh betreffend. Jch mußte
ihnen zeigen, wie solches von jeher dem Gesetz Gottes zuwider
war; ich war deswegen in meinem Geiste sehr betrübt bis inden
Tod, aber da ich mich unter den Willen Gottes stellte, so er-
wachte ein himmlisches Sehnen nach dem Herrn in meinem Herzen,
ich sah den Himmel offen und freute mich und gab Gott die
Ehre ....
Jn diesem Zustande trieb es mich, an die Richter zu schreiben,
wie schädlich es für die Gefangenen sei, so lange im Kerker zu
sein, wie sie da schlechtes von einander lernten, wenn sie mit-
einander über ihre bösen Taten reden. Darum sollten die Urteile
rasch gesprochen werden. Denn ich war ein gottseliger Jüngling
rmd wandelte in der Furcht des Herrn; es betrübte mich, ihre
schlechten Reden zu hören, ich mußte ihnen oft Vorstellungen über
ihre bösen Worte machen und über ihr häßliches Betragen unter-
einander. Die Leute wunderten sich, wie ich bewahrt und behiitet
blieb; denn nie konnten sie mir ein Wort oder eine Tat nach-
weisen, die sie hätten zu meinen Ungunsten auslegen können


% \picinclude{./030-039/p_s039.jpg} 
Erlebnisse im Gefängnis zu Derby usw. 39
während der ganzen Zeit, die ich dort war; denn die unendliche
Kraft des Herrn hielt mich aufrecht und bewahrte mich während
der ganzen Zeit; ihm sei Lob und Ehre immerdar.
Es war eine junge Person mit mir im Gefängnis, die ihrem
Herrn Geld gestohlen hatte. Llls sie zum Tode verurteilt werden
sollte, schrieb ich an den Richter und ans Schwurgericht und
stellte ihnen vor, wie es immer gegen das Gesetz Gottes gewesen
sei, die Leute wegen Diebstahls zum Tode zu verurteilen, und
bat um Gnade. Sie wurde aber doch verurteilt, und man grub
ihr ein Grab und führte sie zur Hinrichtung. D,a schrieb ich noch
einmal ein paar Worte und warnte alle, sich vsor Raubgier und
Habsucht zu hüten, da sie von Gott wegführe, und ermahnte alle
den Herrn zu fürchten, allen irdischen Begierden zu entsagen und
die Zeit zu nützen, dieweil sie da ist; solches hieß ich sie unter
dem Galgen vorlesen. Und obgleich sie sie schon auf der Leiter
hatten, bereit gehenkt zu werden, mit einem Tuch über den Augen,
so wurde sie nun nicht hingerichtet, sondern sie führten sie wieder
zurück ins Gefängnis, und im Gefängnis kam sie nachher dazu,
Gottes ewige Wahrheit zu erkennen.
Es war noch ein anderer Gefangener mit mir, ein schlechter,
gottloser Mensch, ein bertichtigter Schwarzkünstler und Zauberer.
Er drohte, was er alles zu mir sagen und mir tun wolle, aber
er hatte keine Macht, den Mund gegen mich aufzutun. Einmal
gerieten der Kerkermeister und er aneinander und er drohte, er
wolle den Teufel rufen und das Haus niederreißen, so daß der
Kerkermeister Angst bekam. Da trieb mich der Herr hinzugehen
und ihm Einhalt zu gebieten und zu sagen: ,,Komm, laß sehen
was du kannst, tue dein Außerstes«. Jch sagte ihm, der Teufel
sei schon in ihm selber bei uns, die Kraft des Herrn binde ihn
aber. Da schlich er sich davon.
Als nun die Zeit der Schlacht von Worcester kam, sandte
der Richter Vennet Konstabler, um mich zu zwingen, Soldat zu
werden, da er gesehen hatte, daß ich kein Kommando übernehmen
würde. Jch sagte ihnen, ich sei ganz gegen allen äußeren Krieg.
Sie kamen wieder, um mir Werbegeld zu geben, aber ich nahm
es nicht. Daraus wurde ich vor den Wachtmeister Holes gebracht,
der mich eine Weile behielt und dann wieder zurückschickte. Nach
einiger Zeit wurde ich wieder heraufgeholt und vor den Kommifsär
gebracht, welcher erklärte, ich müsse als Soldat gehen, aber ich


% \picinclude{./040-049/p_s040.jpg} 
sagte ihnen, ich sei hiestir tot. Sie sagten, ich sei ja am Leben.
Jch sagte ihnen, wo Neid und Zank sei, da sei Verderben (Jak. 3, 16).
Sie boten mir zweimal Geld an, aber ich wollte nichtß an-
nehmen; daraus wurden sie böse und verurteilten mich zum Ge-
fängnis- ....
Jch war tief betrübt und bearbeitet in meinem Geist während
meiner Gefangenschaft wegen der Schlechtigkeit, die in der Stadt
herrschte; denn obgleich etliche gewonnen waren, so war doch die
Mehrzahl sehr oerhärtet. Jch sah, wie sich das- Außgießen der
Liebe Gotteß von ihnen wegwandte. Ich trauerte über sie, und
es kam über mich, folgende Klage über sie zu verbreiten:
,,O Derby! Wie die Wasser abfließen, wenn die Schleusen
sich öffnen, also fließet die Liebe Gotteß von dir ab, o Derby.
Darum siehe zu, wo du stehest und auf welchem Grund du bist,
ehe du gänzlich verlassen wirst. Der Herr hat mich zweimal ge-
rufen, ehe ich zu dir kam, um gegen deine Eitelkeit und Schlech-
tigkeit aufzutreten und alle zu ermahnen, auf den Herm und
nicht auf Menschen zu sehen. ,,Wehe der prächtigen Krone der
Trunkenen! der welken Blume ihrer Herrlichkeit« (Jes. 28, ll.
Wehe denen, die mit Worten ihren Glauben zur Schau tragen und
doch hochmütig und hochfahrend sind und Unterdrückung und Haß
üben. O Derby! Deine Frömmigkeit und dein Predigen stinken
gen Himmel! Jhr feiert einen Sabbat in Worten und versammelt
euch, um euch schön zu kleiden, ihr frönet der Eitelkeit. Die
Weiber gehen mit aufgerichtetem Halse und geschminkten Ge-
sichtern, wie ez die alten Propheten verurteilt haben (Jes. 3, 16).
Eure Versammlungen sind dem Herrn ein Greuel; ihr erhebet
die Eitelkeit und beuget euch davor; das Laster gedeiht und da-Z
Böse wird geehrt; daö Schlechte wird von den Schlechten ge-
duldet und doch bekennen sie alle Christus mit Worten. O über
die Schlechtigkeit unter euch! EH bricht mir fast das Herz, zu
sehen, wie Gott unter euch verachtet ist, o Derby!«
A18 ich gesehen, wie Gottez Liebe sich von diesem Orte ab-
wandte, wußte ich, daß meine Gefangenschaft hier nun nicht mehr
lange andauern werde, aber ich sah, daß, wenn der Herr mich
srei machen werde, so werde eß sein, wie wenn man einen
Löwen auß seiner Höhle auf die wilden Tiere dee Waldes ab-
läßt. Denn alle »Frommen« hatten eine tierische Gesinnung, die
der Sünde huldigte, so lange sie lebten. Sie waren alle dem


% \picinclude{./040-049/p_s041.jpg} 
Erlebnisse im Gefängnis zu Derby usw. 41
Geist und dem Leben seiud, der in der Schrift gegeben ift und
den sie in Worten bekannten. So geschah ez, wie man hernach
sehen wird.
Ez stand ein Gericht über der Stadt, und den Behörden war
ez unbehaglich meinetwegen; aber sie wußten nicht, waz sie mit
mtr machen sollten. Einmal wollten sie mich vorz Parlament
schicken, ein andermal mich nach Jrland oerbannen. Zuerst
nannten sie mich einen Betrüger und Verfiihrer und Gottes-
lästerer; dann, alz Gott seine Strafe über sie schickte, sagten fie,
ich sei ein ehrlicher, tugendhafter Mensch. Aber ob sie eine gute
oder schlechte Meinung von mir hatten, war mir gleichgültig;
denn weder richtete mich daz eine auf, noch warf mich das andere
nieder, dem Herrn sei Lob. Schließlich mußten sie mich frei
lassen, zu Anfang des Winterz 1651, nachdem ich fast etn Jahr
in Derby gefangen gewesen war, sechz Monate im Zuchthauz
und die übrigen im Kerker.
Alz ich nun wieder meine Freiheit hatte, fuhr ich fort wie
zuvor in der Arbeit für den Herrn und zog im Lande umher,
zuerst in der Gegend meiner Heimat, Leicestershire; ich hielt unter-
wegz Versammlungen, und dez Herrn Geist und Kraft war mit
nur ....
Einmal alz ich mit einigen Freunden unterwegz war und
eine Turmhauzspitze erblickte, ging ez mir durch Mark und Bein;
ich fragte, waz daz für eine Ortschaft sei? ez hieß: Lichfield.
Alsobald erging daz Wort dez Herrn an mich, daß ich dorthin
gehen müsse. Alz wir bei dem Hause angelangt waren, in daz
wir gehen wollten, bat ich die Freunde, die mit mir waren, hinein-
zugehen; ich sagte ihnen aber nicht, wohin ich zu gehen hatte.
Sobald sie im Hause waren, entfernte ich mich und lief über
Hecken und Gräben, biz ich eine Meile weit von Lichsield ent-s
fernt war; da waren auf einem weiten Felde Schäfer, die ihre
Schafe hüteten. Hier befahl mir der Herr, meine Schuhe auzzu-
ziehen; ich zögerte, denn ez war Winter; doch daß Wort dez
Herrn war wie Feuer in mir. So zog ich denn meine Schuhe
aus und ließ sie bei den Schäfern, und die armen Schäfer zitterten
und waren ganz bestürzt. Darauf lief ich wieder eine Meile,
und sobald ich wieder in der Stadt war, erging daz Wort dez
Herrn an mich: ,,Rufe: wehe der blutigen Stadt Lichfield!« Ich
ging also die Straße auf und ab und rief: ,,Wehe der blutigen


% \picinclude{./040-049/p_s042.jpg}
Stadt Lichfield!« Da ez Markttag war, ging ich aus den Markt-
platz, lies aus demselben umher und rief von Zeit zu Zeit: ,,W-ehe
der blutigen Stadt Lichfield!« Und niemand tat mir etwaß.
Während ich rufend durch die Straßen ging, schien es mir, 11lS
ob ein Bach von Blut durch die Straße fließe, und der Markt-
platz kam mir vor wie ein Teich von Blut. Alö ich mich der
mir aufgetragenen Verkündigung entledigt hatte, verließ ich im
Frieden die Stadt. Jch kehrte zu den Hirten zurück, gab ihnen
Geld tmd erhielt meine Schuhe von ihnen zurück. Aber das
Feuer dez Herrn war so in meinen Füßen und in meinem ganzen
Körper, daß mir nichts daran lag, meine Schuhe überhaupt wieder
anzuziehen; und ich wußte nicht recht, ob ich ez tun sollte oder
nicht, bis ich die Grlaubniß dazu vom Herrn fühlte; nachdem ich
meine Füße gewaschen, zog ich meine Schuhe wieder an. Darauf
versiel ich in tiefeß Nachstnnen, warum und aus welchem Grunde
ich wohl gesandt worden sei, gegen diese Stadt zu reden und sie
die ,,blutige Stadt« zu nennen; denn obwohl eine Zeitlang daß
Parlament und eine Zeitlang der König die Herrschaft über diesen
Kirchenspengel gehabt hatte und viel Blut in der Stadt vergossen
worden war während des- Krieges zwischen beiden, so war ez
doch nicht schlimmer gewesen alß an vielen anderen Orten auch.
Nach und nach aber fiel es mir ein, wie zur Zeit dez Kaiserß
Diocletian tausend Christen in Lichsield gemartert worden waren;
darum hatte ich ohne Schuhe durch den Bach ihreß Bluteß gehen
müssen, damit die Erinnerung an das Blut jener Märtyrer, daß
vor mehr als tausend Jahren vergossen worden und in ihren
Straßen erkaltet war, wach werde. Die Nachwirkung jenes Bluteß
war über mich gekommen, so daß ich dem Herrn hatte gehorchen
müssen. Man weiß auö alten Uberlieserungen, wie viel christ-
liche Vriten dort gelitten haben. Ich könnte noch viel berichten
über alle:-’, waß sich mir offenbarte über daß hier während der
zehn Verfolgungen und später vergossene Märtyrerblut, aber ich
überlasse es dem Herrn und seinem Buch, au?7 welchem alleß
gerichtet werden wird; denn sein Buch und fein Geist sind sichere
Uberlieserer.
Darauf zog ich im Lande umher und hatte vielerorts Ver-
sammlungen unter den freundlich Gesinnten. Aber meine Ange-
hörigen waren böse über mich. Nach einiger Zeit kehrte ich nach
Nottinghamshire zurück und ging dann nach Derbshire, um dort


% \picinclude{./040-049/p_s043.jpg} 
Erlebnisse im Gefängnis zu Derby usw. 43
die freundlich Gesinnten aufzusuchen. Jn Yorkshire und an einigen
andern Orten predigte ich Buße: darauf kam ich nach Balby,
wo Richard Famöworth 1) und einige andere gewonnen wurden.
So reiste ich im Lande umher, Buße predigend und daö Wort
dez Herm verkündigend, bis-3 ich in die Gegend von Wakefield
kam, wo James Naylor lebte; er und Thomaß Goodyear
kamen zu mir; beide wurden gewonnen und nahmen die Wahrheit
auf. Auch William Dem?-bury und seine Frau und viele andere
kamen zu mir, wurden gewonnen und nahmen die Wahrheit auf.
Von dort begab ich mich nach Hauptmann PurZloe’S Hauß in
die Nähe von Selby, und besuchte John Leek, der inß Gefängniß
zu mir gekommen war, und er wurde gewonnen. Ich besaß ein
Pferd, mußte mich aber leider davon trennen, da ich nicht wußte,
maß damit anfangen, weil mich der Herr trieb in manches- an-
gesehene Hautz zu gehen, um die Leute zu ermahnen, sich zum
Herrn zu bekehren. Unter anderm trieb mich der Herr auch inß
Turmhauß von Beverly zu gehen, daß damalß eine Stätte beson-
derer Frömmigkeit war; da ich vom Regen ganz durchnäßt war,
ging ich zuerst nach der Herberge. Jn der Türe kam ein junge-3
Weib auf mich zu und sagte: ,,Wie! seid ihr ez? Kommt herein«,
wie wenn sie mich schon gekannt hätte; denn die Kraft dez Herrn
hatte ihr Herz vorbereitet. Ich nahm etwaß zu mir und ging
inß Bett. Am Morgen zog ich meine noch nassen Kleider an
und bezahlte meine Zeche und begab mich ins Turmhauö, wo
einer predigte. A15 er geendet, trieb mich die mächtige Kraft Gotteß,
zu ihnen zu reden, und ich wies sie aus Chrisiuz, ihren Lehrer, hin.
Die Kraft dez Herrn war so mächtig, daß alle von großer Furcht
ergriffen wurden. Der Bürgermeister kam und sprach ein paar
Worte mit mir, aber niemand hatte Macht, mir etwaß zu tun.
Jch verließ die Stadt und ging am Nachmittag in ein anderez
Turmhauß, etwa zwei Meilen weit entfernt. A13 der Priester
geendet, trieb ez mich, eingehend zu ihm und den Leuten über
den Weg deö Lebens und der Wahrheit und den Grund der Gr-
wählung und Verdammung zu reden. Der Priester sagte, er sei
1) Richard Farnsworth, William Dewßbury und James Naylor waren
die ersten bedeutenden Missionsprediger der Quätet. (Näheres s. Weingarten,
Revolutionskirchen Englandtz. S. 218ss.) James Naylor ist in der Geschichte
betiichtigt geworden durch seinen Messiatzeinzug in Bristol, dem Höhepunkt der
saft zum Wahnsinn gesteigerten Schwärmerei des älteren Quälertumö.


% \picinclude{./040-049/p_s044.jpg} 
zu kindlich, um mit mir zu dißputieren; ich erklärte ihm, ich sei
nicht gekommen, um zu di?-putieren, sondern um daß Wort dez
Lebenß und der Wahrheit zu verkünden, und damit sie alle den
Samen kennen lernen möchten, den Gott allen verheißeu, den
Männern wie den Frauen. Die Leute waren hier sehr empfänglich
und wünschten, daß ich wiederkäme an einem Wochentag, um
ihnen zu predigen, aber ich wie:3 sie an ihren Lehrer Jesuö
Christus und verließ sie. Am folgenden Tage ging ich nach
Cranstick zu Hauptmann Pnrßloe, der mich zu Richter Hotham
begleitete. Dieser war ein gottseliger Mann, der auch Gottes
Wirken schon in seinem Herzen verspürt hatte. Nachdem wir eine
Zeitlang über göttliche Dinge geredet hatten, nahm er mich mit
in sein Zimmer und bekannte mir, daß ihm diese Ansichten
schon seit zehn Jahren vertraut seien, und wie er sich freue, daß
der Herr sie nun auch verkünden lasse unter den Leuten. Nach-
her kam noch ein Priester zu ihm, mit dem ich auch über die
Wahrheit redete. Aber der war bald zum Schweigen gebracht,
denn er war ein bloßer Phantaft, der sich daß, wovon er redete,
innerlich nicht angeeignet hatte.
Während ich da war, kam eine angesehene Frau auß Beverly,
um Richter Hotham in irgend einer wichtigen Angelegenheit zu
sprechen. Jin Laufe dez Gesprächeß erzählte sie ihm, daß am
vergangenen Sabbat, wie sie diesen Tag nannten, ein Engel oder
ein Geist in die Kirche von Beverly gekommen sei und herrliche
Dinge von Gott geredet habe zur Verwunderung aller Anwesenden,
und alß er geendet habe, sei er verschwunden; sie wisse nicht, woher
er gekommen, noch wohin er gegangen sei, alle haben sich ge-
wundert, die Priester, die »Frommen« und die Behörden der Stadt.
Richter Hotham erzählte mir das- nachher wieder, woraus ich ihm
mitteilte, daß ich e3 gewesen, der an jenem Tage im Turmhauß
gewesen und die Wahrheit verkündet hatte ....
Am Nachmittag ging ich in ein andereß Turmhauö, wo ein
großer, angesehener Priester, ein Doktor, wie sie ihn nannten,
redete, einer von denen, die Richter Hotham wollte kommen lassen.
Jch ging hin und wartete, biS der Priester geendet hatte. Die
Worte, die er alö Text genommen hatte, waren: ,,Wohlan alle,
die ihr dursiig seid, kommet her zum Wasser, und die ihr nicht
Geld habt, kommt her, kauset und esset, kommt her und kauset
ohne Geld, beide:-3 Wein und Milch (Jes. 55, 1).*- Und der Herr


% \picinclude{./040-049/p_s045.jpg} 
Erlebnisse im Gefängnis zu Derby usw. 45
trieb mich zu sagen: ,,Komm herunter, du Verführer; heißest du
die Leute umsonst kommen und umsonst vom Wasser dez Lebenz
nehmen, und nimmst jährlich dreihundert Pfund dafür, daß du die
Schrift oerkündest? Errötest du nicht vor Scham? Tat der
Prophet Jesaiaß und Christuß, die diese Worte umsonst geredet
und mitgeteilt hatten, auch also? Sagte nicht Christus- zu seinen
Jüngern, alö er sie auösandte zu predigen: umsonst habt ihr ez
empfangen, umsonst gebet etz auch?« Der Priester machte sich
ganz bestürzt davon; nachdem er seine Herde verlassen hatte,
hatte ich so oiel Zeit, alß ich wollte, um zu den Leuten zu sprechen;
ich wieö sie von der Finsternis zum Licht und zur Gnade Gotteß,
die sie lehren und ihnen Rettung bringen werde, und zum Geist
Gotteß in ihrem Jnnern, der sie umsonst lehre.
Dann kehrte ich zu Richter Hothamö Hauß zurück; alß ich
eintrat, schloß er mich in seine Arme und sagte, sein Haus sei
mein Hau-3. Denn er freute sich sehr über daß Werk dez Herrn
und daß seine Kraft kund geworden. Dann erzählte er mir,
warum er am Morgen nicht mit mir zum Turmhauö gegangen
war, und was für Gründe er gehabt hatte; er hatte sich gesagt,
wenn er mit mir in-3 Turmhauß gehe, so würden die Wachen
mich ihm übergeben und da werde er so in die Sache verwickelt;
dann wisse er nicht, maß machen. Darum sei er froh gewesen,
alß Hauptmann Pur?-loe gekommen; aber keiner von ihnen war
in Amts-kleidung gewesen oder hatte den Kragen um den Halß I
gehabt. GZ war damalt-3 etwaß ganz Ungewöhnliche?-, daß einer
ohne Kragen inß Turmhauß kam; aber Hauptmann Purßloe
war ohne einen solchen mit mir ins Turmhauö gekommen, so hatte
die Kraft des Herrn ihn übernommen, daß er gar nicht daran
dachte.
Jch zog weiter und kam an einen Abend zu einer Herberge.
Jch bat die Wirtin, mir etwaß Fleisch zu bringen, wenn sie solches-
habe; aber weil ich »du« und ,,dich« zu ihr sagte, sah sie mich
besremdet an; ich fragte sie, ob sie Milch habe. Sie sagte: nein.
Jch merkte, daß sie nicht die Wahrheit sagte, und um sie noch
weiter zu prüfen, fragte ich sie, ob sie Rahm habe; sie verneinte eö
ebenfalls. Nun stand ein Butterfaß im Zimmer und ein kleiner
Knabe, der daneben spielte, steckte seine Hand hinein und stieß eß
um und oerschiittete allen Rahm vor meinen Augen auf den
Boden; da zeigte es sich, daß die Frau eine Lügnerin war. Sie


% \picinclude{./040-049/p_s046.jpg} 
erschrak, stieß eine Verwünschung aus, hob das Kind auf und
schlug es tüchtig; aber ich machte ihr Vorwürfe wegen ihrer
Lüge und ihres Betrügens. Nachdem der Herr solcherweise ihre
Betrügerei und Bosheit aufgedeckt hatte, verließ ich das Haus
und ging weiter, bis ich zu einem Heuschober kam und brachte
nun die Nacht darin zu im Regen und Schnee, denn es war
drei Tage vor dem Tag, den sie Ehristfest nennen.
f Am folgenden Tage kam ich nach York, wo etliche sehr gott-
selige Leute waren. Am Ersten Tage der darauffolgenden
Woche hieß mich der Herr in das große Münster gehen und zum
Priester Bowles und seinen Zuhörern reden tn ihrer großen
Kathedrale. Jch ging hin und als der Ptiestet geendet, sagte ich,
ich habe ihm und der Gemeinde eine Botschaft von Gott dem
Herrn zu bringen. ,,Dann sage sie schnell!« sagte einer der
,,Frotnmen« aus der Versammlung; denn es war gefroren und
schneite und war sehr kaltes Wetter. Jch sagte ihnen, solches
seiüdas Wort des Herrn an sie: ,,Jhr lebet in Worten, aber der
Herr der Allmächtige verlangt Früchte von euch.« Kaum waren
die Worte aus meinem Munde, so stießen sie mich hinaus und
warfen mich die Stufen hinunter; aber ich stand aus, ohne verletzt
zu sein und ging in meine Wohnung. Etliche wurden überzeugt;
denn schon die Seufzer, die ich ausstieß unter dem Druck und
dem Zwang des Geistes Gottes in mir, genügten, um vieler
i Herzen zu öffnen und zu ergreifen, sodaß sie bekannten, die Seufzer,
die ich ausstoße, machen ihnen Eindruck.; mein ganzes Wesen
war bedrückt davon, daß sie bekannten und nicht besaßen, Worte
machten und keine Früchte brachten.
Nachdem ich für den Augenblick meinen Dienst in York getan
hatte und etliche dort gewonnen worden waren und die Wahrheit
Gottes angenommen und sich zu seiner Lehre bekannt hatten,
verließ ich York und wandte mich nach Cleveland und fand dort
Leute, welche die Kraft Gottes geschmeckt hatten. Ich sah, daß
ein Same in jener Gegend war, und daß Gott dort ein demiitiges
Volk hatte. Unterwegs holte mich, gegen Abend, ein Päpstlicher
ein und redete mit mir über seine Religion und über ihre Gottes-
dienste, und ich ließ ihn alles sagen, was er aus dem Herzen
hatte. Ich brachte die Nacht in einer Schänke zu; am folgenden
Morgen trieb mich der Herr, zu diesem Päpstlichen zu reden. Jch
begab mich in seine Wohnung und zeugte gegen seine Religion


% \picinclude{./040-049/p_s047.jpg} 
Erlebnisse im Gefängnis zu Derbi; usw. 47
und alle ihre abergläubischen Gebräuche und sagte ihm, Gott sei
gekommen, sein Volk selbst zu lehren; das brachte den Papisten
dergestalt auf, daß es ihn aus seinem eigenen Hause trieb ....
Obgleich zu der Zeit der Schnee sehr tief war, fuhr ich fort
herutnzureisen und kam zu einem Marktflecken, wo ich viele
,,Fromme« traf, mit denen ich lange Unterredungen hatte. Ich
stellte ihnen viele Fragen, die sie nicht beantworten konnten, weil
sie sagten, man habe sie noch nie in ihrem Leben so schwere
Dinge gefragt. Von da ging ich nach Stath, wo ich ebenfalls
viele ,,Fromme« und einige Ranter traf. Ich hatte große Ver-
sammlungen unter ihnen undsviele Bekehrungen. Viele nahmen:die
Wahrheit aus, worunter einer, der hundert Jahre alt war; ein
anderer war ein Oberkonstabler und einer war ein Priester, namens
Philipp Scafe. Diesen machte der Herr später durch seinen Geist
zu einem freien Verkündiger seines freien Evangeliums.
Der Priester dieses Ortes war sehr hochfahrend und bedrückte
die Leute sehr mit seinen Abgaben. Wenn sie aus den Fischfang
gingen, so machte er sie Abgaben vom Erlös bezahlen, obgleich
sie dieselben so weit her hatten und sie bis nach Yarmouth zum
verkaufen brachten. Es trieb mich, dort ins Turmhaus zu gehen,
um die Wahrheit zu verkünden und den Priester bloß zu stellen.
Als ich mit ihm geredet hatte und ihm die Unterdrückung des
Volkes vorgestellt hatte, lief er davon. Die Ältesten der Gemeinde
waren sehr hochmiitig und leichtfertig; darum verließ ich sie, nach-
dem ich das Wort des Lebens verkündet hatte, weil sie dasselbe
nicht aufnehmen wollten. Aber das Wort des Lebens, das ich
unter ihnen verkündet hatte, blieb bei etlichen von ihnen, so daß
etliche der Ersten aus der Gemeinde des Nachts zu mir kamen,
und die meisten wurden gewonnen und bekannten sich zur Wahrheit;
so begann die Wahrheit sich in dieser Gegend auszubreiten, und
wir hatten große Versammlungen; dadurch wurden die Priester
zornig und die Ranter fingen an, unruhig zu werden und ließen
mir sagen, sie wollten eine Unterredung mit mir haben, die Priester,
welche Unterdrückung übten, und die Runter. Es wurde ein Tag
festgesetzt und die Ranter erichienen; es kam auch noch ein anderer
Priester, ein Schotte, aber der Priester, welcher sich der Unter-
drückung schuldig gemacht hatte, nicht. Philipp Scafe, der be
kehrte Priester, war bei mir und es erschienen viele Leute. Als
wir uns gesetzt hatten, erklärte ein Ranter, namens T. Bushel,


% \picinclude{./040-049/p_s048.jpg}
er habe ein Gesicht von mir gehabt; ich sei an einem großen Pult
gesessen und er habe kommen müssen und seinen Hut vor mir
abnehmen und sich tief vor mir verbeugen, und er habe eß getan;
und noch viele andere Schmeicheleien sagte er mir. Jch sagte
zu ihm, er habe daß nur erfunden und er solle zu sich selber
sagen: ,,Schäme dich, du Hund«. Er sagte, eß sei nur Neid von
mir, so zu sagen. Darauf fragte ich ihn, waß der Neid eigentlich
sei rmd wie er im Menschen entstehe und waß daß Htindische sei
und wie eß im Menschen entstehe. Denn ich sah genau, daß er
etwaß Hündisrheß hatte, und darum wollte ich von ihm wissen,
wie dieseß Hündische in ihm entstanden sei. ,,Denn«, sagte ich
ihm, ,,mir müssen zuerst von dem reden, maß in unserm Leib
geschieht, ehe wir von dem reden können, roaß außer dem Leibe
ist.« Damit stopfte ich ihm daß Maul und allen seinen Runter-
genossen, denn er war ihr Haupt. Dann ries ich den Priester,
welcher die Leute unterdrückte, aber er kam nicht; nur der schottische
Priester erschien, der mit wenig Worten zum Schweigen gebracht
war; denn eß war innerlich kein Leben in ihm von dem, waß er-
bekannte. Nun war die Gelegenheit da, mit den Leuten zu reden.
Jch zeigte klar, wie die Ranter waren und verglich sie mit den
Prahlern in Sodom. Jch zeigte, wie ihre Priester die gleiche
Sorte von Mietlingen seien, wie die falschen Propheten früherer
Zeiten, und wie die Priester damals daß Volk auch in dieser
Weise regierten, indem sie ihren Gewinn im Auge hatten und
um Geld ihr Amt besorgten und um schnöden Gewinnß willen
lehrten. Jch stellte Christuß und die wahren Propheten und die
Apostel den Priestern gegenüber und zeigte, wie Ehristuß, die
Propheten und die Apostel sie schon lange an ihren Früchten
erkannt hätten. Dann wieß ich sie aus den Lehrer in ihrem
Jnnern hin, Jesus Christuß, ihren Heiland. Und ich predigte
Christuß in den Herzen, nachdem ich alle diese Höhen geebnet
hatte. Die Leute waren alle ruhig und die Widersacher zum
Schweigen gebracht. Denn obgleich eß innerlich in ihnen kochte,
so hielt die Kraft sie doch gebunden, so daß sie nicht loßbrechen
konnten .....
Ein anderer Priester ließ mich holen, um mit mir zu reden,
und etliche ,,Freunde« gingen mit mir nach seinem Hauö. Alß
er hörte, daß wir gekommen seien, entwischte er auß dem Hause
und versteckte sich unter einer Hecke. Die Leute gingen, ihn zu


% \picinclude{./040-049/p_s049.jpg} 
Erlebnisse im Gefängnis zu Derby usw. 49
zu suchen und fanden ihn, aber sie brachten ihn nicht dazu, zu
uns zu kommen. Daraus ging ich in ein nahegelegenes Turm-
haus, wo der Priester und das Volk in großer Erregung waren,
denn eben dieser Priester hatte den Freunden mit allem Ntöglichen,
das er tun werde, gedroht; als ich aber kam, machte er sich davon,
denn die Kraft des Herrn kam über ihn und über die andern.
Ja, des Herrn ewige Kraft kam über die Erde und drang zu
den Herzen der Menschen und machte die Priester und die
,,Frommen« zittern. Sie machte die Geister der Erde und der
Lust erbeben, zu welchen sie Vorgaben zu beten, sodaß sie einen
Schreck bekamen, wenn es hieß: »Der Mann in den ledernen
Kleidern kommi!«1) An vielen Orten machten sich die Priester,
wenn sie das hörten, davon, so waren sie von Furcht vor der
ewigen Kraft Gottes ergriffen .....
Von hier gingen wir über Scarbvrough .... nach Malton .....
Am Ersten Tag kam eine Frau, eine der angesehensten ,,Frvmmen«
unter den Jndependenten, welche ein solches Vorurteil gegen mich
hatte, daß sie sagte, ehe sie kam, sie würde sich freuen, mich er-
hängt zu sehen; aber als sie kam, wurde sie gewonnen und ge-
hört seither zu den »Freunden«.
Daraus hatte ich hier große Versammlungen; es hätten noch
mehr Leute daran teil genommen, aber sie wagten es nicht, aus
Furcht vor ihren Angehörigen. Es wurde damals als etwas
Unerhörtes angesehen, daß man in Häusern predigte statt in der
,,Kirche«, wie sie es nannten; darum wurde sehr gewünscht, daß
ich ins Turmhaus gehe und ddrt rede. Einer der Priester schrieb
mir und lud mich ein, im Turmhaus zu predigen, und nannte
mich seinen Bruder. Ein anderer Priester, eine bekannte Persön-
lichkeit, hielt dort eine Stunde. Nun hatte mir der Herr während
meiner Gefangenschaft in Derby kund getan, ich solle in den
Turmhäusern predigen, um die Leute von denselben abzubringen,
und es kamen mir auch zuweilen Bedenken wegen der Kanzeln,
in denen die Priester herumsaulenzten. Die Turmhäuser und
Kanzeln verletzten mein Gefühl, weil sowohl die Priester als auch
das Volk sie Gotteshäuser nannten und im Wahne waren, daß
Gott da in äußern sichtbaren Häusern wohne, statt im Gegenteil
1) Fox trug immer Kleider aus Leder, die et wegen ihrer Einfachheit und
Dauerhaftigkeit allen andern Kleidungsstiicken vorzog. (Vgl. Carlyles, Surtor
Resartus: Ein Ereignis in der neuen Geschichte.)
George Fox. 4


% \picinclude{./050-059/p_s050.jpg} 
Verlangen zu tragen, daß Gott und Christus in ihren Herzen
und Leibern wohne, aus daß sie Tempel Gottes? würden. Denn
der Wostel sagt: ,,Gott wohnet nicht in Tempeln mit Händen
gemacht'' (Act. 7, 48). Weil man aber diese Stätten nun einmal
heilig hielt, so fand man ez- schrecklich, wenn man etwas dagegen
sagte. A18 ich inZ Turmhauß kam, waren nicht mehr altz 11 Zu-
hörer dort, und der Priester hielt ihnen die Predigt. Alß nun
in der Stadt bekannt wurde, ich sei im Turmhause, so füllte sich
daßselbe bald mit Menschen. Alk- der Priester, der an dem Tage
zu predigen hatte, geendet hatte, hieß er den andern Priester, der
mich aufgefordert hatte zu kommen, mich auf die Kanzel führen,
aber ich ließ ihm sagen, ich brauche nicht auf eine Kanzel zu
steigen. Darauf ließ, er mir wieder sagen, er wünsche aber, daß
ich sie befteige, weil dort ein besserer Platz sei, an dem mich die
Leute sehen könnten. Ich ließ ihm darauf sagen, man sehe mich
gut genug, da wo ich sei, ich sei nicht gekommen, solche Stätten
noch aufrecht zu erhalten und ihr Bestehen und den Handel, der
damit getrieben wird. Alk- ich dieö gesagt hatte, fingen sie an,
böse zu werden und sagten: ,,Da8 sind die falschen Propheten
der letzten Zeiten«. Diese Rede Verletzte etliche und sie murrten
darüber; nun stand ich auf und hieß alle ruhig sein; ich stieg
auf einen hohen Stuhl und erklärte ihnen, woran man die falschen
Propheten erkenne, und daß sie schon gekommen seien; und dann
zeigte ich ihnen im Gegensatz dazu die wahren Propheten, Christus-
und die Apostel. Ich wieß sie alle an ihren inneren Lehrer,
Christue, der sie von der Finsterniö zum Lichte führen könne.
Nachdem ich ihnen verschiedene Schriftstellen erklärt hatte, wies
ich sie auf den Geist Gottes in ihren Herzen hin, durch welchen
sie zu ihm kommen könnten und erkennen, wer die falschen
Propheten seien. Nachdem ich so ein reiches Wirken unter ihnen
gehabt hatte, zog ich im Frieden von dannen ....
Hierauf kam ich nach Pickering, wo die Richter im Turm-
hauö ihre Sitzungen hielten; Ftiedenztichtet Robinson war Vor-
sitzender. Ich hatte zur gleichen Zeit eine Versammlung im
Schulhaus und viele »Fromme« und Priester wohnten ihr bei
und stellten allerlei Fragen, die zu ihrer Zufriedenheit beantwortet
wurden. ES war gerade die Zeit der Gerichtösit-zungen, und da
wurden auch vier Oberkonstabler bekehrt. GS kam Richter Robin-
son zu Ohren, daß der Priester, den er allen andern Priestern


% \picinclude{./050-059/p_s051.jpg} 
Erlebnisse im Gefängnis zu Terby usw. 51
oorzog, besiegt und überzeugt worden war. Wir gingen nach
der Versammlung in eine Herberge; Richter Robinson’s Priester
war sehr bescheiden und lieb und wollte sogar durchaus mein
Essen bezahlen, was ich aber nicht zuließ. Dann bot er mir sein
Turmhaus an, um darin zu predigen, aber ich lehnte es ab und
erklärte ihm und den andern, daß ich eben gekommen sei, um die
Leute oon diesen Dingen ab und zu Christus zu bringen.
Am folgenden Morgen ging ich mit den vier Konstablern und
andern, um Richter Robinson zu besuchen, der mir unter der
Türe seines Zimmers entgegenkam. Jch sagte ihm, ich könne ihm
keine menschliche Ehre erweisen; er sagte, er sehe nicht aus das.
Jch ging nun mit ihm ins Zimmer und tat ihm den Unterschied
zwischenswahren und falschen Propheten dar, und wie die wahren
höher stehen als die falschen, und richtete seinen Sinn aus
Christum seinen Lehrer. Jch deutete ihm die Gleichnisse, und
wie es sich mit der Grwählung und Verwersung verhalte, wie
man in der ersten Geburt in der Verwersung sei und in der
zweiten in der Grwählung. Ich zeigte ihm, wer die Verheiß-ungen
Gottes habe und wen sein Gericht verdamme. Gr gab alles zu
und war so offen für die Wahrheit, daß, wenn ein anderer an-«
wesender Richter eine kleine Ginwendung machen wollte, er ihn
belehrte. Beim Fortgehen sagte er, ich tue sehr gut, diese mir
von Gott verliehene Gabe zu gebrauchen. Gr nahm den obersten
Konstabler beiseite und wollte ihm etwas Geld siir mich geben,
weil er nicht wollte, daß ich in ihrer Gegend irgend welche Aus-
gaben habe; aber sie sagten ihm, daß ich nicht dazu zu bringen
sei, etwas anzunehmen. Jch schätzte seine Freundlichkeit, das
Geld jedoch lehnte ich ab.
Jch zog im Lande umher und der Priester, der mich Bruder
genannt hatte, zog mit mir. Als wir in eine Stadt kamen, wo
wir im Sinne hatten etwas zu essen, läuteten die Glocken.
Jch fragte, warum sie läuten; man sagte mtr, sie läuten für mich,
damit ich im Turmhaus predige. Bald daraus trieb es mich
dorthin. Als ich kam, sah ich die Leute auf dem Turmhausplatze
versammelt; der alte Priester wollte, daß ich ins Turmhaus gehe,
  ich sagte, es sei nicht nötig. Ge besremdete die Leute, daß
ich nicht in das gehen wollte, das sie ,,Goiteshaus« nannten. Jch
stellte mich auf den Platz des Turmhauses und erklärte den Leuten,
ich sei nicht. gekommen, ihre göizendienerischen Tempel ausrecht
 


% \picinclude{./050-059/p_s052.jpg}
zu erhalten, noch die Priester mit ihren Zehnten, Zulagen, Ab-
gaben und Pfrtinden, noch ihre jüdischen und heidnischen Zere-
monien und Traditionen; denn die gelten mir alle nicht?-. Jch er-
klärte ihnen, dieses Stück Boden sei nicht heiliger, ale irgend ein
anderes Stück Land. Jch zeigte ihnen, daß die Apostel, wenn
sie in die Synagogen und die Tempel der Juden gegangen seien,
die ja Gott selber sogar vorgeschrieben habe, so sei ez nur ge-
schehen, um die Leute davon Iabzubringen und von den Opfern
und Zehnten und den habsüchtigen Ptiestem jener Zeit. Und
die, welche zur Wahrheit belehrt wurden und an den von den
Aposteln gepredigten C-hristuö Hglaubten, hätten sich nachher in
den Wohnhäusern versammelt. Ich sagte ihnen, daß alle, welche
Christus, daß Wort deö Lebenß, predigen, eö umsonst tun sollen
wie die Apostel, und wie Christus eß geboten habe. So war ich
gesandt worden von Gott dem Herrn Himmels und der Erden
umsonst zu predigen und die Leute von diesen äußeren Tempeln
mit Händen gemacht, worin Gott nicht wohnt, abzubringen, damit
sie erkennen, daß ihre Leiber Tempel Gottes werden sollen. Jch
mußte die Leute abbringen von ihren jüdischen Zeremonien, aber-
gläubischen und heidnis chen Gebräuchen, Traditionen und Menschen-
satzruigen, von der Lehre all der Mietlinge, die Zehnten nehmen
und große Psründen, die um Bestechung predigen und für Geld
weiösagen, die gar nicht von Gott und von Christus gesandt
sind, wie sie ja selber bekennen, wenn sie sagen, sie haben nie
die Stimme Gotteß noch Christi vernommen. So ermahnte ich
denn die—Leute, abzulassen von alle dem, und wieß sie auf den
Geist und die Gnade Gottetz hin, welche inwendig in ihnen sind,
und auf daß Licht Jesu in ihren Herzen, aus daß sie dazu kommen
möchten, C-hristum zu kennen, der sie umsonst lehre und ihnen
Rettung bringe und ihnen die Schrift öffne. Alleö war ruhig
und viele wurden gewonnen, der Herr sei gepriesen.
Jch kam daraus in eine andere Stadt, wo wieder eine große
Versammlung war; der vorhin erwähnte Priester begleitete mich
und allerlei ,,Fromme« kamen dazu herbei. Ich saß mehrere
Stunden auf einem Heuschober und sagte nichte, denn sie sollten
nach Worten hungern. Die ,,Frommen« kamen immer wieder
zu dem alten Priester und fragten ihn, wann ich beginnen werde
zu reden. Er hieß sie warten und sagte ihnen, das Volk habe
immer lange gewartet, bis Christus gesprochen habe. Schließlich


% \picinclude{./050-059/p_s053.jpg} 
Erlebnisse im Gefängnis zu Derby usw. 53
trieb mich der Herr zu reden, und sie wurden von der Kraft deö
Herrn erfaßt; das Wort detz Lebens erreichte sie und es- geschah
eine allgemeine Bekehrung unter ihnen.
Ich zog weiter; der alte Priester und einige andere waren
mit mir. Unterwegß riesen ihn ein paar Leute an: ,,Mr. Bones,
wir sind euch Geld schuldig für Zehnten; kommt doch und nehmt
eZ!« Aber er wehrte mit der Hand ab und sagte, er habe genug,
er wolle nichtß davon, sie sollten ez nur behalten; und er prieß
Gott, daß er solcheö sagen konnte. Schließlich kamen wir zu dem
Turmhauß dieses- alten Priesters im Nioor; alß wir eingetreten
waren, ging er vorausz und öffnete die Kanzeltür, aber ich sagte ihm,
ich würde nicht hineingehen. Das Turmhauß war stark bemalt;
ich sagte ihm und den Leuten, die dabei waren, daß gemalte Tier
(Offb. 17, 3.) habe ein gemalteß Haus. Dann erklärte ich ihnen die
Entstehung aller dieser Häuser und ihre abergläubischen Gebräuche;
ich zeigte ihnen, daß die Apostel nicht in die Tempel gegangen
seien, um diese aufrecht zu erhalten, sondern um die Leute zu
Christuö, dem wahren Gut, zu führen; ich zeigte ihnen den wahren
Gotte?-dienst, den Christus gegründet hat; ich zeigte den Unter-
schied zwischen Ehristuß dem wahren Weg und allen verkehrten
Wegen, indem ich ihnen die Gleichnisse deutete und sie von der
Finsternitz zum wahren Lichte wieö; damit sie durch dasselbe sich
selbst erkennen möchten und ihre Sünden und ihren Erlöser und
durch den Glauben an ihn erlöst würden von ihren Sünden .....
Nun kam ich nach Eranstick, zu Hauptmann Purßloe und
Friedenßrichter Hotham, die mich beide freundlich empsingen, weil
sie sich freuten, daß die Kraft des Herrn erschienen war und daß
die Wahrheit sich auzbreitete und so viele sie aufnahmen, und daß
Frichen?-richter Robinson so freundlich gewesen war. Hotham
sagte, wenn Gott nicht diese Anschauungen von Licht und Leben
hätte kund werden lassen, so wäre daß ganze Land von den
Rantern überschwennnt worden und alle Richter des- Landeß mit
allen ihren Gesetzen hätten ihnen nicht zu wehren vermocht. ,,Denn«,
sagte er, ,,wenn sie auch gesagt und getan hätten, was- wir ihnen
befehlen, so hätten sie doch nicht von ihren Wisichten gelassen.
Aber eure Grundsätze der Wahrheit werfen alle ihre Grundsätze
und daß, worauf sie die ihrigen gründen, über den Hausen«.
Darum war er so froh, daß Gott diese Grundsätze des Leben-3
und der Wahrheit hatte durch mich kund werden lassen ....


% \picinclude{./050-059/p_s054.jpg} 
Als am folgenden Tage die Freunde mich verlassen hatten,
reiste ich allein weiter und verkündete den Tag des Herrn überall,
wohin ich kam, und ermahnte zur Buße. Eines Abends kam ich
in die Stadt Patrington, und während ich durch die Stadt ging,
ermahnte ich sowohl die Priester als das Volk Buße zu tun und
sich zum Herrn zu bekehren. Gs wurde finster, ehe ich ans Ende
der Stadt kam, und eine große Menge hatte sich um mich ver-
sammelt, während ich das Wort des Lebens verkündete. -- Als
ich meine Ausgabe erfüllt hatte, ging ich in eine Herberge und
verlangte Unterkunft für die Nacht, aber sie wurde mir verweigert.
Daraus bat ich um etwas Fleisch und Milch, ich wolle es bezahlen;
aber auch das wollte man mir nicht geben. So verließ ich die
Stadt; einige junge Leute kamen hinter mir drein und fragten
mich, was es neues gebe. Jch hieß sie Buße tun und Gott
fürchten. Als ich eine Strecke weiter gegangen war, kam ich wieder
an ein Haus und bat, man solle mir etwas Fleisch und Milch
geben und Nachtherberge, gegen Bezahlung; aber sie schlugen es
mir ab; dann ging ich zu einem andern Haus und verlangte das-
selbe; aber sie wiesen mich ebenfalls ab. Jnzwischen war es so
dunkel geworden, daß ich die Landstraße nicht mehr sehen konnte;
ich endeckte einen Wassergraben und schöpfte etwas Wasser um
mich zu erfrischen; dann überschritt ich den Graben und da ich
von der Reise müde war, setzte ich mich unter einen Ginsterstrauch
und wartete bis es Tag war. Mit Tagesanbruch erhob ich mich
und ging weiter. Hinter mir drein kam ein Mann mit einer
Heugabel, der schritt neben mir her bis zu einer Stadt, und
noch ehe die Sonne ausgegangen war, hatte er diese Stadt und
die Polizei gegen mich ausgehetzt; ich verkündete Gottes ewige
Wahrheit unter ihnen und warnte sie vor dem Tag des Herrn,
der kommen würde über alle Sünde und Ungerechtigkeit, und er-
mahnte sie, Buße zu tun. Mer sie.griffen mich und brachten
mich nach Patrington zurück, etwa drei Meilen weit, und be-
wachten mich mit Stöcken, Heugabeln und Hellebarden. Als ich
nach Patrington kam, war die ganze Stadt in Aufruhr. Die
Priester und das Volk berieten sich zusammen; so konnte ich
ihnen abermals das Wort des Lebens verkünden und sie zur Buße
ermahnen. Endlich nahm mich einer der »Frommen«, ein guter
Mann, mit in sein Haus, wo ich mich an etwas Brot und Milch
erlabte, denn ich hatte seit mehreren Tagen nicht-3 gegessen. Dann


% \picinclude{./050-059/p_s055.jpg} 
Erlebnisse ini Gefängnis zu Derby usw. 55
schleppten sie mich etwa neun Meilen weit zu einem Richter.
Als wir nahe bei dessen Haus waren, kam einer hinter uns her
geritten und fragte mich, ob ich der sei, der verhaftet worden
war. Ich fragte, warum er es wissen wolle; er sagte, es ge-
schehe in keiner bösen Absicht; da sagte ich ihm, daß ich es
sei; darauf ritt er voraus zum Richter. Meine Begleiter sagten,
hoffentlich sei der Richter nicht betrunken, wenn wir zu ihm
kämen; denn er pflegte schon frtihmorgens betrunken zu sein.
Als ich vor ihn trat und meinen Hut nicht abnahm und ihn mit
Du anredete, fragte er den, welcher uns oorgeritten war, ob ich
verrückt sei, aber er sagte ihm, nein, es sei mein Grundsatz. Jch
ermahnte den Richter, Buße zu tun und sich zum Licht zu be-
kehren, mit dem Christus ihn erleuchtet, damit er durch dasselbe
alle seine bösen Worte und Taten erkennen möge, und zu Christus
zurückzukehren, solange es noch Zeit sei. ,,Ja, ja«, sagte er, ,,das
Licht von dem im dritten Kapitel des Johannes gesprochen wird-«.
Joh bat ihn, er möge doch auf dieses Licht achten und ihm ge-
horchen. Während ich ihn ermahnte, legte ich ihm die Hand auf,
und er ward übernommen von der Kraft des Herm und die
Wächter waren bestürzt. Er führte mich nun in ein kleines Gemach,
um zu untersuchen, was ich von Briefen und Schriften in der
Tasche habe; ich wies ihm meine Kleider und zeigte ihm, daß
ich keine Briefe bei mir hatte; er sagte, man sehe an meiner
Wäsche, daß ich kein Landstreicher sei, und ließ mich frei. Jch
ging mit dem Mann, der vor uns hergeriiten, nach Patrington
zurück, denn er lebte daselbst. Als wir ankamen, wünschte er, ich
solle eine Versammlung auf dem Hauptplatz halten, aber ich sagte
es sei nicht nötig, sein Haus genüge. Gr wollte, daß ich zu Bett
gehe oder mich doch aufs Bett lege; dies wünschte er namentlich,
damit er sagen könne, man habe mich in oder doch wenigstens
auf einem Bett gesehen; denn es ging das Gerücht, ich wolle
in keinem Bett schlafen, weil ich damals oft im Freien über-
nachtete. Als der Erste Tag kam, ging ich ins Turmhans
und verkündete dem Priester und dem Volk die Wahrheit; und
die Leute taten mir nichts, denn die Kraft Gottes war über sie
gekommen. Gleich nachher hatte ich eine große Versammlung in
dem Hause des Mannes, der mich beherbergte, und viele wurden
von Gottes ewiger Wahrheit überzeugt und sind derselben treu
geblieben bis aus den heutigen Tag. Sie bereuten es sehr, daß


% \picinclude{./050-059/p_s056.jpg} 
sie mich nicht aufgenommen und beherbergt hatten, als ich zuerst
bei ihnen gewesen war .......


%%%%%%%%%%%%%%%%%%% Kapitel 5. %%%%%%%%%%%%%%%%%%%%%%%%%%%%%%

\chapter[Quäkerischen Weltmission]{Quäkerischen Weltmission}

\begin{center}
\textbf{Christus in uns. Erkenntnis der Quäkerischen Weltmission. Das
Haus Richter Fells in Swarthmore; der Pöbel von Ulverstone.
Rechtfertigung vor dem Gericht in Lancastre.}
\end{center}



\begin{floatingfigure}[3]{4cm}
\includegraphics[width=0.20\textwidth]{./pics/swarthmore_hall.png}
\label{bild:swarthmoor} 
\end{floatingfigure}



Wir zogen durch Nottinghamshire nach Lineolnshire .....
Hier kam zu einer unserer Versammlungen ein Mann und erhob eine
falsche Anklage gegen mich; er verbreitete überall daö Gerücht, ich
habe gesagt, ich sei Christus-, was gänzlich falsch war. Al-? ich dann
nach Gainßborough kam, wo einer der Freunde auf dem Markt-
platz die Wahrheit verkündet hatte, fand ich die ganze Stadt und
alle Marktleute in Aufruhr. Jch ging inß Haus eineö Freunde?-,
und daß Volk drängte sich hinter mir drein, biz das Haus ganz
voll war von »Frommen«, Giferern und Pöbel; da kam jener
falsche Verleumder herein und klagte mich öffentlich vor allen an,
ich hätte gesagt, ich sei Christuß, und er habe Zeugen, es zu be-
weisen. Das brachte die Leute so in Wut, daß man Miihe hatte,
mich vor ihnen zu schützen. Da trieb mich der Geist dez Herrn
aus einen Tisch zu stehen und in der ewigen Kraft dez Herrn
den Leuten zu verkünden, daß Christuö in ihnen sei, etz sei denn,
daß sie Verdammte seien; und daß eß Christuö, die ewige Kraft
Gottes sei, welche jetzt auß mir zu ihnen rede, nicht ich sei Christus;
die Leute waren im allgemeinen befriedigt außer jenem »Frommen«
und einigen falschen Zeugen. Jch nannte diesen Ankläger Judaö,
und es trieb mich, ihm zu sagen, daß das Ende deö Judaß auch
das seine sein werde; solches- lasse ihm der Herr durch mich sagen.
Dez Herrn Macht kam über alle und beruhigte die Gemüter der
Leute und sie gingen in Frieden fort. Jener Judaß aber machte
sich davon und erhenkte sich und man steckte einen Pfahl in sein
Grab. Daraufhin erhoben die bösen Priester eine Verleumdung
gegen unö und streuten aus, ein Quäker habe sich erhenkt in
Lineolnshire. Diese Lüge ließen sie drucken und verbreiten und
hänften so Sünde auf Sünde. Mer wir und die Wahrheit wurden
nicht davon getroffen; denn jener war so wenig ein Quäker als
der Priester, der solcheß gedruckt hatte; vielmehr war eögeiner


% \picinclude{./050-059/p_s057.jpg} 
Christus in uns. Erkenntniz der Quükerischen Weltmisfion usw. 57
ihrer eigenen Leute. Aber trotz dieser argen Lüge, mit welcher der
Gegner beabsichtigthatte, unß zu verleumden und die Leute von
der von unß verkiindeten Wahrheit abzukehren, nahmen doch viele
in Lineolnshire daß Evangelium an, da sie von der ewigen Wahr-
heit überzeugt waren und sich zu Füßen deß himmlischen Herm
setzten ......
Wir zogen nun wieder .... über Warmßworth . . . Bably,
Doneaster .... nach Tickhill, wo an einem Ersten Tage die
Freunde der Gegend sich versammelten, und eß herrschte durch
Gottes Macht eine tiefe Zerknirschung in der Versammlung.
Jch verließ die Versammlung, da Gott mich trieb inß Turmhauß
zu gehen. Alß ich dorthin kam, fand ich den Priester und saft
alle Gemeindeältesten im Chor beisammen. Jch ging zu ihnen
und hub an zu ihnen zu reden, aber sie sielen sogleich über mich
her, und ein Priester nahm seine Bibel und schlug mich damit
inß Gesicht, so daß ich heftig blutete im Turmhauß; daß Volk
schrie: ,,Hinauß mit ihm auß der Kirche!« Und alß sie mich hinauß
gebracht hatten, prügelten sie mich und warfen mich zu Boden
und über eine Hecke; hernach schleppten sie mich durch ein Hauß
aus die Straße; sie warfen mich mit Steinen und schlugen mich,
während sie mich Vorwärtß tissect, so daß ich über und über mit
Kot beschmiert war. Sie nahmen mir den Hut, den ich nicht
mehr wieder bekam. Alß ich jedoch wieder auf den Füßen war,
verkündete ich ihnen daß Wort deß Lebenß und zeigte ihnen, wo-
hin ihre Lehre sie führe und wie sie daß Christentum entehrten. Nach
einer Weile ging ich wieder in die Versammlung zurück zu den
Freunden. Und alß die Priester und die Leute am Hause vorbei
kamen, ging ich mit einigen Freunden hinauß in den Hof und
redete zum Priester rmd den Leuten. Der Priester verhöhnte
unß und nannte unß ,,Quäker«. Aber die Macht deß Herrn
kam dermaßen über sie und daß Wort deß Lebenß wurde ihnen
so überzeugend und eindringlich verkündet, daß der Priester selber
zu zittern begann und einer sagte: »seht wie der Priester zittert
und bebt, er wird auch ein Quäker«. Alß die Versammlung zu
Ende war, gingen die Freunde sort, und ich ging, ohne Hut,
nach Balby, etwa sieben bis acht Meilen weit. Die Freunde
wurden an dem Tage dergestalt von dem Priester und seinen
Anhängern mißhandelt, daß einige Friedenßrichter, alß sie davon
hörten, kamen und ein Verhör in dieser Stadt anstellten, um


% \picinclude{./050-059/p_s058.jpg} 
die Sache zu untersuchen. Der, welcher mich blutig geschlagen
hatte, fürchtete, man haue ihm die Hand ab; aber ich vergab
ihm und klagte nicht gegen ihn.
Zu Anfang deö Jahres- 1652 regte sich heftiger Widerstand
gegen die Wahrheit und die Freunde, bei Priestern und Volk
und bei etlichen der Behörden in Yorkshire, so daß der Priester
von Warmöworth sich einen Verhaftbefehl gegen mich und Thomaß
Aldam verschaffte, der in allen Teilen im westlichen Bezirk York-
shireö auögeführt werden konnte. Zu dieser Zeit hatte ich ein
Gesicht von einem Bären und zwei großen, riesigen Hunden, und
wie ich bei ihnen vorbei mußte, ohne daß sie mir ein-aß tun
konnten. Und so geschah es; denn der Konstabler ergriff Thomaß
Aldam und brachte ihn nach York; und ich ging ein großeö Stück
Wegß mit ihm. Der Kanstabler hatte auch einen Verhaftbesehl
gegen mich und sagte zu mir: er sehe mich schon, aber er möge
nicht einen der ihm fremd sei, behelligen; Thomaß Aldam sei
eben sein Nachbar. Also hielt ihn die Kraft des Herrn, daß er
mich in Ruhe ließ. Wir kamen in die Wohnung deö Leutnant
Roper, wo wir eine große Versammlung hatten, worunter viele
angesehene Leute waren; die Wahrheit wurde mächtig kund
unter ihnen und die Schrift herrlich erklärt, und die Gleichnisse
und Reden Jesu wurden außgelegt und die Kirche, wie sie in den
Tagen der Apostel war, und der Abfall von derselben. Die
Wahrheit gelangte zur Herrschaft an jenem Tage, so daß jene
angesehenen Leute alle zugestanden: ,,diese Anschauungen werden
sich über die ganze Erde aus-breiten«. Dieser Versammlung
wohnten auch Jametz Naylor, Thomas Goodyear und William
Dewßbury, die das Jahr vorher gewonnen worden waren, sowie
Richard Farne-worth bei. Der Konstabler blieb mit Thomaß
Aldam, bis die Versammlung auö war, darm ging er mit ihm
nach dem Gefängnis in York; mich aber ließ er in Ruhe ....
Darnach kam ich nach Hightown, wo eine Fran wohnte, die
kurz vorher bekehrt worden war. Wir gingen in ihr Haus und.
hielten eine Versammlung, und die Leute versammelten sich, und
wir verkiindeten ihnen die Wahrheit und wirkten für den Herrn
unter ihnen, und sie gingen in Frieden wieder von dannen.
Aber ez war dort eine Witwe, namens Green, von böser Ge-
sinnung; diese ging zu einem sogenannten ,,Herrn« (cieutleman)
und verklagte unö bei ihm, obwohl er kein Beamter war. Am


% \picinclude{./050-059/p_s059.jpg} 
Christus in uns. Erkenntnis der Quäkerischen Weltmission usw. 59
nächsten Morgen sandten wir dem Priester einige Fragen. Alö
wir gerade fort gehen wollten, kamen einige, die sich zu une
hielten, gerannt und sagten, dieser Mörder habe sein Schwert
für unß geschärst und komme mit demselben gegen unß. Da wir
gerade fort gingen, oerfehlten wir ihn. Aber kaum waren wir
sort, so kam er in das Haus, in dem wir gewesen waren, und
es hieß allgemein, wenn wir nicht fort gewesen wären, so wären
wir ermordet worden. Wir brachten die Nacht im Walde zu
und wurden ganz durchnäßt, denn es regnete stark. Am Morgen
trieb es- mich roicher in die Stadt zurück, wo sie unß außfiihrlich
über jenen Bösewicht berichteten.
Von da gingen wir nach Vradsord, wo wir Richard Farnß-
worth trafen, von dem wir unö kurz vorher getrennt hatten.
A18 wir in sein Haus kamen, setzte man unß Fleisch nor, aber alß
. ich anfangen wollte, geschah daß Wort deö Herrn an mich: »Jß nicht
Brot bei einem Neidischen« (Spr. 23, 6). Sogleich stand ich
vom Tische auf und aß nicht;3. Die Frau war eine Baptistin.
Nachdem ich die ganze Familie ermahnt hatte, sich zum Herrn
zu bekehren und auf seine Lehre in ihren Herzen zu merken,
gingen wir von dannen ......
Unterwegs; kamen wir zu einem großen Hügel, genannt Pend-
lehill; und der Herr trieb mich, aus denselben hinauf zu gehen,
maß ich mit großer Anstrengung tat, denn er war sehr steil und
hoch. Alö ich oben ankam, blickte ich auf daß Meer, das Lan-
cashire umspült. Von diesem Hügel aut:-’ zeigte mir der Herr
die Orte, wo ihm ein großetz Volk sollte gesammelt werden.
Beim Hinuntergehen sand ich eine Wasserquelle am Abhang
dez Hiigelß, auß der ich mich ersrischte, denn ich hatte in den
letzten Tagen nur wenig gegessen und getrunken. Am Abend
kamen wir zu einer Herberge .... und hier ließ mich der
Herr ein Gesicht sehen: eine große Schar in weißen Kleidern
am Ufer eines Flusses, die zum Herrn kamen, und der Ort, den
ich sah, war bei Wen?-leydale und Sedbergh. . .
Wir zogen durch die Daleö . . . nach Dent .... Hier ging
ich zu Richard Robinson und redete von der Wahrheit zu ihm:
Ju einer Versammlung bei Frieden?-richter Benson traf ich Leute,
die sich vom öffentlichen Gotteödienst lozgesagt hatten. Dies
war der Ort, den ich gesehen, wo eine Schar in weißen Kleidern
daher kam. GS war eine große Versammlung, und die meisten


% \picinclude{./060-069/p_s060.jpg} 
wurden gewonnen und haben noch jetzt große Versammlungen von
Freunden in der Nähe von Sedbergh, die ich damals zuerst zu-
sammen sammelte im Namen Jesu.
GS fand ein großer Jahrmarkt statt, an welchem man pflegte
Dienstboten zu dingen; ich verkündete den Tag dee Herrn. Nach-
dem ich dietz getan, ging ich auf den Platz des Turmhausetz, und
viele Leute kamen vom Jahrmarkt zu mir und eine Menge Priester
und »Fwmme«. Da verkündete ich die ewige Wahrheit dez
Herrn und das Wort dez Lebenö während mehrerer Stunden
und zeigte, daß der Herr gekommen sei, sein Volk selbst zu lehren
und es abzubringen von den Wegen dieser Welt und ihren Lehrern,
zu Ehristuö dem wahren Lehrer und wahren Weg. Jch machte
ihnen klar, wie ihre Lehrer denen gleich seien, die von jeher
von den Propheten, von Ehristuß und den Aposteln verdammt.
worden sind. Jch ermahnte alle von ihren mit Händen gemachten
Tempeln abzulassen und auf den Empfang dez Geistes zu warten,
damit sie erkennen könnten, daß sie der Tempel Gottes seien.
Nicht ein einziger von den Priestern hatte Macht, seinen Mund
auszutun gegen daß, waß ich verkündete; zuletzt sagte einer von
der Wache: ,,Warum geht ihr nicht in die ,,Kirche«? hier ist
kein geeigneter Platz zum Predigen«. Jch sagte ihm, ich leugne
ihre Kirche. Da erhob sich Francis Howgill,1) Prediger einer
Gemeinschaft. Er hatte mich nie vorher gesehen, aber er unternahm
es, diesem Hauptmann zu antworten und brachte ihn bald zum
Schweigen; und von mir sagte er: ,,dieser predigt gewaltig und
nicht wie die Schriftgelehrten (Matth. 7, 29)«. Jch erklärte da-
rauf den Leuten, daß dieser Boden hier nicht heiliger sei alö an
einem andern Ort und daß nicht dieses Haus die Kirche sei,
sondern die Gemeinde, deren Haupt Christutz ist. Bald nachher
kamen dann einige Priester zu mir und ich ermahnte sie, Buße zu
tun. Einer von ihnen sagte, ichsei verrückt, und wandte sich von
mir ab; aber manche wurden gewonnen an dem Tage und freuten
sich über die Verkündigung der Wahrheit und nahmen sie mit
Freuden auf. Einer unter ihnen, Hauptmann Ward, nahm die
Wahrheit in Liebe auf und lebte darin bis zu seinem Tode.. . .
Von da ging ich nach Unterbarrow, zu einem namenß Miles
Bateman ..... Am Morgen ging ich auß . . und als ich in
1) Franeiö Hotogill, später ein eisriger Qnäkerprediger (s.Wein-
garten a. a. O.)


% \picinclude{./060-069/p_s061.jpg} 
Christns in uns. Erkenntnis der Quäkerischen Weltmission usw. 61
der Nähe auf einem Hügel hin und her ging, sah ich einige
Reisende, welche um Unterstützung baten, und ich sah, daß sie
eß nötig hatten; aber man gab ihnen nichts und sagte ihnen, sie
seien—Strolche. GS betrübte mich solche Hartherzigkeit unter den
,,Frommen« zu sehen, und ale sie alle beim Frühstück saßen, lies
ich den Reisenden etwa eine Viertelmeile nach und gab ihnen
etwaß Geld. A16 nun einige von den andern aus dem Hause
kamen und sahen, daß ich eine Viertelmeile weg war, sagten sie, ich
hätte nicht so weit kommen können, wenn ich nicht Flügel hätte.
Daraufhin war es nahe daran, daß man die Versammlung ab-
sagte; denn man hatte eine so merkwürdige Meinung von mir
bekommen, daß viele nicht eine Versammlung mit mir haben
wollten. Jch sagte ihnen, ich sei jenen armen Reisenden nachge-
laufen, um ihnen etwaß Geld zu geben, weil mich die Hart-
herzigkeit, mit der man sie fortgeschickt, betrübt habe ....
Von da tzging ich nach Ulverstone und Swarthmore zu
Richter Fell; es kam auch einer, Priester Lampitt, der behauptete,
Eingebungen zu haben- Jch redete lange mit ihm, denn er sprach
von wichtigen Eingebungen und von Vollkommenheit und blen-
dete die Leute dadurch. Gr hätte mich gerne gewähren lassen,
aber ich konnte ihn nicht gewähren lassen, weil er so unlauter
war. Gr sagte, er sei mehr als Johanneß, und tat, alz ob er
alle Dinge wüßte. Jch sagte ihm, der Tod habe von Adam biz
Moses regiert (Röm. 5, 14); und weil er tot sei, kenne er Moses
nicht, denn Moseß habe daß Paradies- Gotteß gesehen; er aber
kenne weder Moseß noch die Propheten noch Johannes. Denn
die höckerichte und rauhe Natur war noch in ihm, und der Berg
der Sünde und deö Verderbens, und der Weg für den Herrn
war nicht bereitet in ihm (Jes. 40). Er bekannte, er sei in großer
Trübsal gewesen, beteuerte aber, nun könne er Psalmen fmgen
und alleö machen, maß; man von ihm verlange. Ich sagte ihm,
er gehöre zum Diebögesindel, aber Moses und die Propheten
und Christuö predigen, das- könne er nicht; dazu müßte er den
gleichen Geist haben wie jene. Margaret Fell war den ganzen
Tag nicht zu Hause gewesen; am Abend erzählten ihr ihre Kinder,
daß Priester Lampitt und ich gestritten hätten; die-J betrübte sie,
weil er dem gleichen Bekenntnis angehörte wie sie; aber er
oerbarg sein schmutzigeß Treiben vor ihnen.
Wir sprachen noch lange miteinander am Abend, und ich ver-


% \picinclude{./060-069/p_s062.jpg} 
kiindete ihr und ihrer Familie die Wahrheit. Am folgenden Tage
kam Lampitt wieder, und ich redete lange mit ihm, und Margaret
Fell, die ihn jetzt ganz durchschaute, war dabei. Eine Über-
zeugung der Wahrheit kam über sie und die Jhrigen. Als bald
darauf ein allgemeiner Bußtag abgehalten werden sollte, bat sie
mich, mit ihr ine Turmhaus von Ulverstone zu kommen, denn
sie hatte sich noch nicht gänzlich davon lo?-gemacht. Jch erwiderte
ihr: ,,Jch muß tun, wie mich der Herr heißt.« Jch verließ sie
und ging int? Freie und daß- Wort dee Herrn geschah also zu
mir: ,,Gehe ihnen nach inß Turmhau?-.« Als ich kam, sang
Lampitt gerade mit den Leuten; aber sein Geist war so unlanter,
und waß sie sangen, paßte so wenig für ihr Bedürfnis, daß, als-
sie fertig gesungen hatten, der Herr mich trieb, also zu ihnen zu
reden: ,,Der ist nicht ein Jude, der etz äußerlich ist, sondern der
ist ein Jude, der innerlich einer ist, in seinem Leben, daß er nicht
vor den Menschen, sondern vor Gott führt« (Röm. 2, 28.29).
Dann zeigte ich ihnen nach des Herrn weiterer Offenbarung, daß
Gott gekommen sei, sein Volk zu lehren (1.Joh.2, 27), (Joh.1-i,26).
durch seinen Geist und sie abzubringen von allen ihren früheren
Gebräuchen, ihren Bekenntnissen, Kirchen und Gotte?-diensten;
denn daß alleö seien nur Menschensatzungen; daß Leben und den
Geist, auz dem diese Satzungen entstanden, die hätten sie doch
nicht. Da rief Friedenörichter Sawrey: ,,Fort mit ihm!« Aber
Richter Fell?. Frau sagte zu den Beamten: ,,Laßt ihn gehen;
warum soll er nicht so gut reden wie ein anderer?« Auch Lampitt,
der Betrüger, sagte, man solle mich reden lassen. Aber alö ich eine
Zeitlang geredet hatte, ließ mich Friedenßrichter Sawrey hinaus
bringen durch die Konstabler; da redete ich auf dem Kirchhof
weiter . . . Jch ging nun nach Becliff . . . und andere Orte . . .
Bald darauf, alß Richter Fell nach Hause kam, ließ Margaret
Fell mich holen und ließ mir sagen, ich solle doch zu ihnen
kommen; ich fühlte die Freiheit vom Herrn, etz zu tun und ging
hin. Ich sah, daß die Priester und die ,,Frommen« und der
Frieden?-richter Sawrey, Richter Fell und Hauptmann Sande durch
ihre Lügen gegen die Wahrheit eingenommen hatten, aber alß ich
kam und mit ihnen redete, gelang ez mir, alle ihre Einwände zu
widerlegen, und ich überzeugte Hauptmann Sands an Hand der
Schrift so völlig, daß er ganz befestigt war in seiner Uberzeugung.
Nach einigem Hin- und Herreden war Richter Fell ebenfalltz zu-


% \picinclude{./060-069/p_s063.jpg} 
Chrisiuß in unß. Erkenntnis der Quitkerischen Weltmission usko. 63
frieden gestellt und gelangte dazu, durch daß, waß ihm der Geist
Gotteß eröffnet hatte, etwaß Höhereß zu erkennen, alß waß die
weltlichen Priester und Lehrer lehrten, und ging nicht mehr hin,
sie zu hören alle die Jahre biß zu seinem Tode; denn er wußte
nun, daß daß, waß ich lehrte, die Wahrheit sei, und daß Ehristuß
der Lehrer seineß Volkeß ist und sein Heiland . . . Während ich
in dieser Gegend war, kamen Richard Farnßworth und Jameß
Naylor, mich und die anderen zu sehen, und weil Richter Fell
nun darüber beruhigt war, daß eß die Wahrheit sei, die ich ver-
kündige, so erlaubte er mir, Versammlungen in seinem Hause zu
haben trotz aller Einwände; und eß wurde eine große Versamm-
lung eingerichtet, die fast vierzig Jahre, biß 1690, bestand, so
daß ein neueß Versammlungßhauß in der Nähe gebaut wurde . . .
Jch hörte von einer großen Versammlung, die in Uloerstone
stattfinden sollte, und ging darum dorthin und begab mich inß
Turmhauß, in der Furcht und der Kraft Gottes-. Alß der Priester
geendet hatte, redete ich daß Wort deß Herrn zu ihnen, daß wie ein
Hammer und ein Feuer unter ihnen wirkte (Jer. 23, 29). Lampitt,
der Priester des Orteß, war mit den meisten andern Priestern
uneinß gewesen vorher, nun aber taten sie sich alle zusammen
gegen die Wahrheit. Aber die mächtige Kraft deß Herrn war
über allem und tat sich so herrlich kund, daß Priester Bennett
sagte: »Die Kirche erbebt!« und sich fiirchtete und zitterte. Und
nachdem er einige unverständliche Worte geredet, eilte er hinauß,
auß Furcht, sie möchte über seinem Kopf zusammenstürzen. Viele
Priester versammelten sich, aber sie hatten noch keine Macht, Ver-
folgungen zu veranstalten.
Alß ich nun hier fertig war, ging ich wieder nach Swarth-
more, wohin vier oder siins Priester kamen; im Gespräch mit
ihnen fragte ich, ob einer unter ihnen sei, der sagen könne, daß
Wort deß Herm: ,,Gehe hin und rede zu den oder jenen«, sei je
einmal an ihn ergangen? Keiner wagte, eß von sich zu be-
haupten. Aber einer von ihnen wurde zornig und sagte, er
könne von Erfahrungen so gut berichten wie ich. Ich erwiderte
ihm, Erfahrungen seien allerdingß etwas, aber eine Botschaft
erhalten und damit außziehen, ein Wort vom Herm haben und
verkünden wie die Apostel und Propheten und wie ich, wenn
ich unter ihnen predige, daß sei noch etwaß andereß. Und ich
fragte sie darum noch einmal, ob einer unter ihnen sagen könne,


% \picinclude{./060-069/p_s064.jpg} 
er habe irgend einmal einen Vefehl unmittelbar vom Herrn
empfangen; aber eß konnte eß keiner. Da erklärte ich ihnen, daß
seien falsche Propheten und falsche Apostel und Antichristen,,die
die Worte der wahren Propheten und wahren Apostel und Christi
gebrauchen und die Erfahrungen anderer verwenden und selber
nie eine Stimme Gotteß oder Christi vernommen haben. Solche
wie sie könnten eben bloß die Erfahrungen un.d Worte anderer
vernehmen. Daß verwirrte sie sehr und stellte sie bloß. Ein
andermal im Gespräch mit einigen Priestern im Hause Richter
Fellö und in dessen Anwesenheit stellte ich die gleiche Frage und
siigte hinzu: daß könne eben jeder, der lesen könne, die Erfah-
rungen der Propheten rmd Apostel verkünden, die in der Schrift
aufgezeichnet seien. Hierauf bekannte ein alter Priester, Thomas
Taylor, dem Richter Fell ehrlich: er habe nie die Stimme Gotte-5
oder Christi vernommen, die ihn irgendwohin gesandt habe; er
rede von seinen eigenen Erfahrungen und den Erfahrungen der
Heiligen früherer Zeiten, und dies predige er. Solcheß bestärlte
Richter Fell in der Überzeugung, daß die Priester im Jrrtum
seien. Denn er hatte vorher, wie die meisten Leute damaltz,
geglaubt, sie seien von Gott gesandt.
Zu dieser Zeit wurde Thoma-8 Taylor 1) bekehrt und durch-
reiste init mir Westmorland ..... Die Priester wurden immer
aufgebrachter gegen unß und verfolgten un?-, wo sie nur konnten.
Jameß Naylor und Franciß Howgill wurden inß Gefängnis ge-
worfen . . . Aber dem Herrn sei Lob, die Wahrheit breitete sich
immer mehr aus. Denn um diese Zeit spürten sich John Aud-
land, John Camm, Edward Burroughi), Richard Hubberthorn 9)
1) Thomar Taylor hatte in Oxford studiert und war Puritanerprediget
geworden. Dann, weil er nicht mehr wollte ,,11m Lohn predigen--, schloß er
sich den Quakern an und wirkte eifrig als Prediger und durch Schristen.
2) Edward Burtough, ursprünglich Prediger der Epiekopalkirche, war
aus dieser ausgetreten und hatte sich den Preöbyterianetn angeschlossen; nach
einigen Untertedungen mit Fox bekehrte er sich sodann zum Quäkertnm, dessen
eifriges tätigeö Glied er blieb, bis er 1662 für seinen Glauben im Kerker, wohin
man ihn auz einer Versammlung gebracht hatte, starb.
3) Richard Hubberthorn, eine bescheidene, friedliche Natur, kriinklich und
mit einer schwachen Stimme, arbeitete dennoch Großez als Prediger. 1662 wurde
er ebenfalls aus einer Versammlung in den Kerker geschleppt, wo sein schwacher
Körper bald den Entbehrungen erlag; doch pries er noch auf seinem Toteubeit
die Güte Gottes.


% \picinclude{./060-069/p_s065.jpg} 
Christus in uns. Erkenntnis der Quüterischen Weltmission usw. 65
und andere mit der Kraft von oben ausgerüstet und traten auf
als Prediger und erwiesen sich alß treue Arbeiter, die umher-
zogen und daß Evangelium umsonst predigten, wodurch Tausende
bekehrt wurden und von nun an dem Herrn gehörten ....
Nachdem ich die Freunde in Westmorland besucht hatte, ging
ich wieder nach Uloerstone, wo Priester Lampitt war. Dieser
hatte selber gepredigt, man müsse sich von Gott lehren lassen;
und alle Menschen, Männer und Frauen, können dazu kommen,
daß Evangelium zu predigen. A18 ez sich aber darum handelte,
dieZ mit der Tat zu beweisen, so verfolgte er sowohl diese Lehre
als die Lehrer .... Als nun die Versammlungen ihren Anfang
nahmen und wir in einer Privatwohnung zusammenkamen, wurde
Lampitt sehr aufgebracht und sagte, wir verlassen den Tempel
und gehen in den Götzentempel Jerobeamö, so daß viele der
,,Frommen« sahen, wie wenig er mit dem, maß er gepredigt hatte,
Ernst machte. Nun legte man ihnen die Sache mit dem Götzen-
tempel Jerobeamz C1. Könige 13), auz und zeigte ihnen, daß
eher ihre Häuser, die sie Kirchen nennen, die Götzentempel Jero-
beamß seien und die alten Meßhäuser, die daß finstere Papsttum
eingesetzt und die jene, die sich Protestanten nennen und meinen,
sie seien ausgeklärter als- die Päpstlichen, auch noch festhalten,
obgleich doch Gott sie nie angeordnet. Und der Tempel, den
Gott in Jerusalem eingesetzt, dem habe Ehristuö. eine Ende ge-
macht, und die, welche ihn aufnahmen und an ihn glaubten, deren
Leiber wurden Tempel Gottes, in denen Christus- und der heilige
Geist wohnen (1. Cor. 6, 19T Diese versammelten sich ....
und kamen zusammen in ihren Wohnhäusern, die dann nicht
Tempel genannt wurden oder Kirchen, sondern ihre Leiber waren
i die Tempel und die Gläubigen die Kirche, deren Haupt Ehristutz
ist .... Daraus trieb ee mich ins Turmhauß zu gehen, two Priester
und ,,Fromme« und viel Volk versammelt waren. Jch stand in der
Nähe von Priester Lampitt, der daraus loß wütete in seiner
Predigt. Nachdem der Herr meinen Mund aufgetan, daß ich
reden sollte, kam Friedenßrichter John Sawrey zu mir und sagte,
wenn ich mich an das halten wolle, waz in der Schrift stehe,
so könne ich reden. Jch wunderte mich über diese Rede und sagte,
ich werde mich sicher an die Schrift halten und sie vorweisen,
um daß Gesagte zu begründen; denn ich hätte ihm und Lampitt
etwas zu sagen. Hierauf sagte er wieder, ich solle nicht reden,
George Fox. Z


% \picinclude{./060-069/p_s066.jpg} 
und widersprach sich damit selber, nachdem er ja eben gesagt
hatte, ich solle reden, wenn ich mich an die Schrift halten wolle.
Die Leute waren ruhig und hörten mir gerne zu, bis Friedens-
tichtet Sawrey, der Hauptanstifter der grausamen Verfolgung
im Norden, sie gegen mich aufhetzte, und sie ansingen, mich zu
stoßen, schlagen und quälen. Sie gerieten alsbald in Wut und
fielen über mich her, im Turmhau-8, und schlugen mich vor seinen
Augen zu Boden, stießen mich rmd traten mich mit Füßen. Der
Aufruhr war groß, so daß etliche über ihre Stühle fielen im
Gedränge. Schließlich kam Sawrey und befreite mich aus ihren
Händen und führte mich hinauö und übergab mich den Konstablern
und hieß sie mich peitschen und zur Stadt hinauß führen. Sie
führten mich etwa eine Meile weit, etliche hielten mich am Kragen,
etliche an den Armen und den Schultern und schleppten und
zerrten mich vorwärtö. Von den Freunden, die auf den Markt
und ins Turmhauß gekommen waren, um mich zu hören, wurden
viele auch zu Boden geworfen und dermaßen geschlagen, daß
manche von Blut überströmt waren. Richter Fellö Sohn, der
mir nachrannte, um zu sehen, maß mit Mir geschehe, warfen sie
in einen Wassergraben und einige schrien: ,,schlagt ihm die Zähne
aus dem Kopf.« A16 sie mich nun biz insz Moor hinauzgeschleppt
hatten, gefolgt von einem großen Haufen, gaben mir die Kon-
stabler mit ihren Weidenruten ein paar Schläge über den
Rücken und überließen mich dem Pöbel, der mit Stöcken, Hecken-
pfählen, Stechpalmen und Gichenzweigen versehen über mich
herfiel und mich auf Kopf und Glieder schlug, bis mir die
Sinne vergingen und ich auf den nassen Boden hinfiel. Als-
ich wieder zu mir kam und merkte, daß ich aus der nassen
Erde lag und die Leute mich umstanden, blieb ich einige
Zeit unbeweglich; und die Kraft des Herrn durchzuckte mich und
die ewige Erquickung erquickte mich, so daß ich wieder ausstehen
konnte in der stärkenden Kraft des Herrn; und die Arme auß-
streckend, sagte ich mit lauter Stimme: ,,Schlaget wieder, hier
sind meine Arme, mein Kopf und meine Wangen.« Einer auß
dem Haufen, ein ,,Frommer«, aber ein roher Kerl, schlug mir
mit seinem Stab genade auf die ausgestreckte Hand; meine Hand
wurde von diesem Schlag so zerquetscht und mein Arm so ge-
lähmt, daß ich ihn nicht wieder zurück ziehen konnte, und etliche
riefen: ,,Seine Hand ist für immer verstiimmelt; er wird sie nie


% \picinclude{./060-069/p_s067.jpg} 
Christus in uns. Erkenutniz der Quäkerischen Weltmission usw. 67
mehr gebrauchen können!« Aber ich betrachtete die Hand in der
Liebe zu Gott; denn ich stand zu allen, die mich verfolgt hatten,
in der Liebe Gottes; und nach einer Weile durchzuckte mich die
Liebe Gotte-3 und zuckte durch meinen Arm und meine Hand,
so daß ich augenblicklich die Kraft darin wieder spürte, vor aller
Augen. Daraufhin gerieten sie selber untereinander in Streit und
sagten mir, wenn ich ihnen Geld gebe, so wollten sie mich vor
den andern schützen. Aber der Herr trieb mich, ihnen das Wort
des Lebens zu verkünden, und ich zeigte ihnen, maß sie für ein
oerkehrtetz Christentum haben, und waß für Früchte die Predigten
ihrer Priester brächten; ich sagte ihnen, sie seien eher Juden oder
Heiden al-3 Christen. Darauf trieb mich der Herr, wieder durch
das Volk hindurch aus den Markt von Ulverstone zu gehen. Auf
dem Wege begegnete mir ein Soldat mit dem Schwert an der
Seite: ,,Herr,« sagte er, ,,ich sehe, daß Jhr ein Mann seid, und
es tut mir leid, daß Jhr so mißhandelt werdet;« und er bot mir
an, mir nach Kräften zu helfen. Aber ich sagte ihm: ,,Die Kraft
des Herrn ist über AlleZ,« und ging durch das Volk hindurch auf
den Markt, und keiner hatte Macht, mich anzurühren. Und als
auf dem Markte einige Freunde mißhandelt wurden, und ich jenen
Soldaten mit seinem nackten Schwerte mitten darunter sah, da
sprang ich hinzu, ergriff seinen Arm und befahl ihm, das Schwert
wieder einzustecken, wenn er eß mit mir halten wolle, und mit
mir auß dem Haufen herauö zu kommen; denn ich wolle nicht,
daß durch ihn ein Unheil geschehe. Einige Tage darauf wurde
dieser Soldat von sieben Männern ergriffen und durchgeprligelt,
weil er ez mit mir und den Freunden gehalten habe. ES war
in jenen Tagen die Art der Verfolger in diesen Gegenden,
ihrer 20 oder 40 aus einen einzigen loßzugehen. An Vielen Orten «
wurden die Freunde in der Weise überfallen, daß sie schier nicht
auf die Straße konnten; man warf ihnen Steine an und miß-
handelte sie. Al?. ich nach Swarthmore kam, kam ich gerade
dazu, wie die dortigen Freunde den Freunden die von Lampittö
Zuhörern mißhandelt worden waren die gebrochenen und ver-
letzten Glieder verbunden. Mein ganzer Körper war gelb, schwarz
und blau von den Schlägen, die ich an jenem Tage erhalten
hatte. Und die Priester singen wieder an zu prophezeihen, daß
wir in einem halben Jahr alle vernichtet sein würden.
Etwa zwei Wochen später ging ich auf die Jnsel Walney und
 


% \picinclude{./060-069/p_s068.jpg} 
James Naylor mit mir. An einem Morgen ging ich mit einem
Boot zu James Lancaster. Sowie ich ans Land stieg, stürzten vierzig
Männer mit Stöcken, Fischangeln und Knütteln hervor, überfielen
mich, schlugen und zerrten mich und versuchten, mich ins Wasser zurück
zu stoßen. Tlls sie mich beinahe in die See zurück geworfen hatten
und ich sah, daß sie mich umbringen wollten, lief ich mitten unter sie
zurück; aber sie fielen wieder über mich her und schlugen mich,
bis ich betäubt war. Als ich wieder zu mir kam, sah ich wie
James Lancasters Frau Steine nach meinem Gesicht wars, und
ihr Mann beugte sich über mich, um die Steine von mir abzu-
halten. Die Leute hatten James Lancasters Frau glauben
machen, ich hätte ihren Mann verhext, und hatten ihr ver-
sprochen, wenn sie ihnen melde, wann ich kommen werde, so
wollten sie mich töten. Und als bekannt wurde, daß ich komme,
hatten viele aus der Stadt sich aufgemacht, mit Knütteln und
Stöcken, um mich zu töten. Aber die Kraft des Herrn
schützte mich, daß sie mir nichts antun konnten. Zuletzt gelang
es mir, wieder aufzustehen; aber sie warfen mich sogleich wieder
ins Boot zurück. Als James Lancaster es sah, kam er sogleich
und brachte mich übers Wasser, daß ich vor ihnen sicher war;
aber so lange wir noch erreichbar waren auf dem Wasser, warfen
sie uns Steine nach. Als wir am anderen User ankamen, sahen
wir, wie sie James Naylor schlugen. Während ich noch drüben
gewesen war, hatte er sich abseits gehalten, so daß sie ihn erst sahen,
als ich fort war; da übersielen sie ihn und schrieen: ,,Tötet ihn!-«
Llls ich die Stadt, am anderen User, erreichte, kamen die
Leute mit Drefchflegeln und Stöcken, um mich zu verhindern, in
die Stadt zu kommen und schrieen: ,,Tötet ihn, tötet ihn! Schlagt
ihn auf den Kopf- bringt den Karren und führt ihn aus den
Kirchhoflss Nachdem sie mich mtßhandelt hatten, schleppten sie
mich aus der Stadt und ließen mich liegen. James Lancaster
ging nun zurück, um nach James Naylor zu sehen; als ich nun
allein da war, ging ich zu einem Wassergraben, und nachdem ich
mich gewaschen hatte, ging ich drei Meilen weit zu Thomas
Hutton, wo Lawson, der bekehrte Priester war. Als ich eintrat,
konnte ich fast nicht reden, so war ich zugerichtet; ich sagte ihnen nur,
wie ich James Naylor verlassen; da nahm jeder von ihnen ein
Pferd und holten ihn noch in jener Nacht zu sich. Als Margaret
Fell am nächsten Tag davon hörte, schickte sie ein Pferd und lleß


% \picinclude{./060-069/p_s069.jpg} 
Chrisiutz in unö. Erkenntnis der Quiikerischen Weltmission usw. 69
mich holen; aber ich war so verwundet, daß ich daß Schütteln
dez Pferdes nur mit großen Schmerzen ertragen konnte. Al-3 ich
nach Swarthmore kam, erließen die Friedenörichter Sawrey und
Thompson von Lancaster einen Verhastbefehl gegen mich; aber
als Richter Fell zurückkam, wurde er nicht auzgeführts Richter
Fell war nämlich die ganze Zeit meiner Mißhandlung nicht in
der Stadt gewesen. A13 er zurück kam, schickte er Verhaftbefehle
nach der Jnsel Walney, um alle jene Aufrührer festzunehmen,
worauf viele von ihnen entftohen; James Lanrasterß Frau be-
kehrte sich später zur Wahrheit und bereute, waö sie mir ange-
tan hatte, sowie auch andere jener grausamen Verfolger; aber
viele von ihnen traf daß Gericht Gottes-, und ez sind etliche unter
ihnen seither zu Grunde gegangen. Richter Fell verlangte einen
Bericht meiner Verfolgung; aber ich sagte ihm, sie hätten ja nicht
anderß handeln können in dem Geiste, in dem sie seien; es seien
die Früchte von dem, waö ihre Priester predigten, und beweise,
daß ihre Frömmigkeit und Religion falsch sei; er berichtete seiner
Frau, ich nehme die Sache leicht, wie einer, den sie nichts angehe;
und in der Tat hatte mich deö Herrn Kraft wieder geheilt ....
Jch ging mit Richter Fell zur Gerichtösitzung nach Lancaster;
er gestand mir unterwegs-, daß ihm noch nie eine solche Ange-
legenheit vorgekommen sei, und daß er nicht recht wisse, wie
er sich dabei verhalten solle. Ich sagte ihm, daß Pauluö, als-
er vor die Obersten der Schule trat und die Juden und Priester
viele falsche Anklagen gegen ihn vorbrachten, die ganze Zeit stille
schwieg. Dann, alß Festuö und Agrippa ihn hießen, für sichselber
reden, tat er ek- und reinigte sich von allen jenen falschen An-
schuldigungen; so sollte er ez mit mir machen. Vor dem Gericht
in Lancaster traten etwa vierzig Priester gegen mich aus; zuxihrem
Redner hatten sie einen namens Marshall gewählt und als
Zeugen einen jungen Priester und zwei Priestersöhne, die schon
vorher beschworen, daß ich eine Gotteölästerung ausgesprochen.
Die Richter hörten alleö an, maß die Priester und ihre Zeugen
gegen mich Vorbringen konnten .... aber die Zeugen waren
so verwirrt, daß sie sich bald alö falsche Zeugen verrieten ....
ES waren mehrere Leute anwesend, die auch in jener Ver-
sammlung gewesen waren, in der ich die Gotteßlästerung auöge-
sprochen haben sollte, alles Leute, die geachtet und angesehen waren
in dieser Gegend; sie erklärten, vor offenem Gerichtöhof, daß die


% \picinclude{./070-079/p_s070.jpg} 
Gide der Zeugen gänzlich falsch seien, und daß ich nichts der-
gleichen geäußert habe .... Oberst West, der ale Friedenörichter
der Gegend hier war, schenkte dieser Auzsage Gehör; und
nachdem er vorher lange krank gewesen war, bekannte er nun,
heute habe ihn der Herr geheilt, und fügte bei, er habe noch nie
so viele gute Menschen und liebe Gesichter beisammen gesehen in
seinem ganzen Leben. Und darauf wandte er sich zu mir und
sagte vor allen: ,,George, wenn du irgend etwas zu den Leuten
zu sagen hast, so tue ee ungehindert«. GS trieb mich zu reden,
worauf der Priester, der gegen mich geredet hatte, sich davon
machte. Jch sühlte mich getrieben zu erklären, daß: ,,die heilige
Schrift vom Geist Gotteß eingegeben sei, und daß alle zuerst
den Geist Gottes in ihrem Jnnern erkennen müssen, durch den
sie Gott und Ehristuö, von denen die Propheten und Apostel
lernten, erkennen können; und durch diesen selben Geist werden
sie dann auch die heilige Schrift verstehen. Denn wie der Geist
Gottes in denen war, die die Schrift geschrieben, so muß derselbe
Geist Gottes auch in denen sein, die die Schrift verstehen wollen;
durch diesen Geist haben sie allein Gemeinschaft mit dem Vater
und dem Sohne und mit der Schrift und untereinander; ohne
diesen Geist aber kann man weder Gott noch Christuß, noch die
Schrift kennen, noch Gemeinschaft untereinander haben«. Kaum
hatte ich solches gesagt, so brachen eine Anzahl Priester hinter
mir los und einer, Jackuö, behauptete unter anderm, der
Buchstabe und der Geist seien unzertrennlich. Darauf erwiderte
ich: ,,dann hat also jeder, der den Buchstaben hat, auch den
Geist, und kann also den Geist mit dem Buchstaben der Schrift
kausen«. Richter Fell und Hauptmann West machten den Pnestcttt
Vorstellungen über diesen ofsenkundigen Irrtum und sagten ihnen,
daß sie ja dann ihrer Ansicht nach den Geist in der Tasche herum
tragen könnten, wie den Buchstaben. Als die Priester sich besiegt
sahen, kehrten sie ihre Wut gegen die Friedenörichter, weil sie
ihre Rache gegen mich nicht stillen konnten. A18 die Friedens-
richter sahen, daß die Zeugen nicht mit einander übereinstimmten,
und daß sie eigentlich nur gewonnen worden waren, um der
Bosheit der Priester zu dienen, und daß alle ihre Anklagen nicht
gültig waren vor dem Gesetz, sprachen sie mich frei .... Jch
war also vor offenem Gerichtshof von allen falschen Anschul-
digungen gereinigt, und viele priesen Gott darüber; denn es war


% \picinclude{./070-079/p_s071.jpg} 
Fox der hexerei verdächtigt. Falsche Offenbarungen usw. 71.
ein Tag der Freude für viele. Friedenßrichter Benson von West-
morland 1) und Major Ripan von Lancaster wurden gewonnen.
GS war ein Tag deS Heilö für Hunderte; denn der Herr Jesus
Christus, der ,,Weg zum Vater«, und der ,,Lehrer der umsonst
lehrt--, wurden gepriesen, und sein ewige-8 Evangelium wurde ge-
Hpredigt und das Leben wurde verkündet, trotz allen diesen Priestern
und gewinnsüchtigen Predigern. Der Herr öffnete an dem Tage
vielen den Mund, daß sie den Priestern Vorstellungen machten
in den Herbergen und in den Straßen, so daß sie wie ein altes
morscheö Gebäude zerfielen; und es hieß allgemein, die Quäker
hätten gesiegt, und die Priester seien unterlegen. Unter andern
war auch Thoma-3 Briggs an diesem Tage gewonnen worden. Gr
war ein Gegner der Freunde gewesen, und als er einmal mit
John Lawson, einem Freund, über die Vollkommenheit geredet hatte,
rief er: ,,waZ, du glaubst an V-ollkommenheit?« und gab ihm dabei
eine Ohrfeige. Dieser Thomaß Briggß 2) wurde an diesem Tage
gewonnen und trat gegen seinen eigenenen Priester Jackuz auf; er
wurde nachher ein treuer Diener dez Evangeliums und blieb es
biz anö Ende seiner Tage ......

%%%%%%%%%%%%%%%%%%% Kapitel 6. %%%%%%%%%%%%%%%%%%%%%%%%%%%%%%

\chapter[Falsche Offenbarungen]{Falsche Offenbarungen}

\begin{center}
\textbf{Fox der Hexerei verdächtigt. Falsche Offenbarungen 
bei Freunden. Gefangenschaft in Carlisle.}
\end{center}

.... Von Lancaster ging ich zu Friedenßrichter West; Richard
Hubberthorn begleitete mich. Da wir den Weg und die Gefahr
der Sandbänke nicht kannten, ritten wir über eine Stelle, über
die, wie wir nachher erfuhren, noch nie jemand zuvor geritten
war. Wir ließen unsre Pferde über sehr gefährliche Stellen
schwimmen. Alß wir ankamen, fragte untz Friedenzrichter West,
ob wir nicht zwei Männer hätten über die Sandbänke reiten
sehen. »Jch werde«, fügte er bei, ,,über kurzem ihre Kleider
1) Getoase Benson war früher Oberst in der Armee gewesen und nun
Friedenstichter in Kendal.
2) Thoma-3 Briggs, der bisher ein eisriger Verfolger der Freunde ge-
wesen, wurde nun ihr Anhiiuger und ein bedeutender Prediger. E: hatte eine
große Gabe der Überzeugung. Er begleitete Fox aus vielen Reisen.


% \picinclude{./070-079/p_s072.jpg} 
haben, denn sie sind sicher ertrunken, und ich bin der Leichen-
schauer«. A15 wir ihm nun sagten, daß wir diese Männer seien,
da wunderte er sich sehr und wollte kaum glauben, daß wir nicht
ertrunken seien. Und die Priester und »Fromen« benützten es, um
daß Gerücht über mich zu verbreiten, ich könne nicht ertrinken mid
man könne mich nicht bluten machen, also sei ich ein Zauberer.
EZ war in der Tat oft vorgekommen, daß ich kaum blutete, wenn
sie mich mit ihren Stöcken schlugen und meinen Leib arg miß-
handelten. Alle diese Verleumdungen kitmmerten mich nicht um
meiner selbst willen; nur um die Wahrheit war mir bange,
gegen die sie mit solchen Mitteln die Leute einzunehmen suchten;
denn ich dachte daran, wie ihre verräterischen Vorfahren den
Hau?-herrn Beelzebub genannt hatten (Matth. 10, 25), und so
konnten ja diese von dem Leben und der Kraft Gotteß abge-
fallenen Christen mit seinem Samen nicht anderßz verfahren. Aber
die Kraft dez Herrn erhob mich über ihre Verläumderischen Zungen
und ihre blutige, mörderische Gesinnung; sie waren selber behext
und darum konnten sie nicht zu Gott und Christuö kommen.
Von Frieden?-richter West ging ich nach Swarthmore, wo
die Kraft deß Herrn die Verfolger niederhielt. ES trieb mich,
verschiedene Briefe von hier aus an die Magistrate, Priester und
,,Frommen« der Umgegend, die sich früher an den Verfolgtmgen
beteiligt hatten, zu schreiben. . . und hernach trieb es mich, an
die Leute in Uloerstone im allgemeinen einen Mahnbrief zu
schreiben ....
Unter den eifrigsten Zuhörern und Nachsolgern dez Priester-?
Lampitt vonU10erstone war ein Adam Sands, ein sehr schlechter,
verdorbener Mensch, der gerne die Wahrheit und ihre Anhänger
vernichtet hätte, wenn er gekonnt hätte. GZ trieb mich, an diesen
also zu schreiben:
,,Adam Sands!
Jch wende mich an daß Licht in deinem Gewissen, du Kind
des Teuselö, du Feind der Gerechtigkeit. Der Herr wird dich
darniederwerfen, wenn du schon eine Zeitlang jetzt herrschest. Die
Strafe Gottes muß dich treffen, der du dich in deiner Bosheit gegen
die reine Wahrheit Gotteß verhärtest. Durch die reine Wahrheit
Gotteö, die du verfolgest und der du widerstrebst, wirst du ver-
nichtet werden; sie ist ewig und schließt auch dich ein; du wirst
in dem Lichte, daß du verachteft, gesehen und in demselben ver-


% \picinclude{./070-079/p_s073.jpg} 
Fox der Hexerei verdächtigt. Falsche Osseubarungen nsw. 73
dümmt, du in deinem tierischen Wesen und dein Weib in seiner
Heuchelei; euer Morden der Gerechtigkeit wird erkannt werden;
das Licht in deinem Gewissen wird dir das, was ich dir hier
schreibe, bezeugen und wird dich erkennen lassen, daß du nicht
aus Gott geboren bist, sondern daß du sem von der Wahrheit
noch in einem tierischen Wesen bist. Wenn je einmal deine Augen
die ausgehen werden und du bereust, so wirst du sehen, daß ich
ein Freund deiner Seele bin und dein ewiges Heil will.
G. F.«
Dieser Adam Sands kam später elendiglich um .....
Ich ging nach Swarthmore zurück. Ich hatte große Offen-
banmgen vom Herm, nicht nur uber göttliche Dinge, sondern
auch über äußere, die die Regierung betrafen. Eines Tages,
als ich im Gerichtssaal Richter Fell und Friedensrichter Benson
über die jüngsten Ereignisse sprechen hörte und oom Parlament,
dasidamals tagte, und das man das ,,lange Parlament« nannte,
trieb es mich, ihnen zu sagen, daß, ehe zwei Wochen um seien,
das Parlament aufgelöst und der Redner von seinem Stuhl herunter
gerissen sein werde. Und als nach zwei Wochen Friedensrichter
Benson wieder kam, sagte er zu Richter Fell, jetzt sehe er, daß
George Fox ein wahrer Prophet sei: Oliver Cromwell habe das
Parlament ausgelöst! (20. April 1653.)
Um diese Zeit fastete ich etwa 10 Tage lang, weil mein
Geist um der Wahrheit willen schwer heimgesucht war; denn
James Milner und Richard Näher hatten Einbildungen und viele
machten es ihnen nach. Dieser James Milner und einige seiner
Anhänger hatten zuerst wahre Offenbarungen; aber da sie in
Hochmut und Selbstiiberhebung gerieten, irrten sie von der
Wahrheit ab. Der Herr trieb mich, zu ihnen zu gehen und ihnen
Ihre Verirrungen vorzustellen; und sie kamen dazu, ihre Torheit
emzusehen, und gaben sie auf und kamen aus den Weg der
Wahrheit zurück. Darauf begab ich mich in eine Versammlung
M Arn-Side, der Richard Myer beiwohnte; er hatte lange einen
lshmen Arm gehabt. Der Herr trieb mich, ihm vor allen An-
wesenden zu sagen: ,,Stehe auf!« und er stand aus und streckte
seinen Arm, der so lange lahm gewesen war, aus und sagte:
»Wisset, alle ihr Leute, daß ich heute geheilt worden bin.« Seine
Eltern wollten es kaum glauben, und als die Versammlung
vorbei war, nahmen sie ihn aus die Seite und zogen ihm sein


% \picinclude{./070-079/p_s074.jpg} 
Wamß au?-; da sahen sie, daß ez wahr sei. Er kam bald darauf
in eine Versammlung in Swarthmore und berichtete da, wie
der Herr ihn geheilt habe. Dornach befahl ihm der Herr, nach
York zu gehen in seinem Auftrag; aber er gehorchte dem Herrn
nicht; und der Herr schlug ihn abermals, daß er etwa dreiviertel
Jahr daraus starb ....
Um diese Zeit wurde Anthony Pearson X), der ein Gegner der
Freunde gewesen war, gewonnen. Er kam nach Swarthmore,
und da ich gerade dort bei Oberst West war, holte man mich.
Oberst West sagte: ,,Geht, Fox, denn Jhr könnt dem Mann zu
großem Nutzen gereichen«—. Also ging ich, und die Kraft deß
Herrn ergriff ihn.
Um diese Zeit tat der Herr auch etlichen den Mund auf,
daß sie den Priestern und dem Volk die Wahrheit verkündeten,
und viele wurden de-zwegen inö Gefängniö geworfen. Jch ging
nun nach Cumberland, wo Anthony Pearson, seine Frau und
mehrere Freunde mich nach Bootle begleiteten; Anthony Pearson
verließ uns dann, um zur Gerichtßsitzung nach Carlißle zu gehen;
denn er war Frieden?-richter in drei Grafschasten. An einem
Ersten Tage ging ich ins Turmhauö von Vootle, und als der
Priester fertig war, sing ich an zu reden. Aber die Leute waren
sehr unverschämt und prügelten mich im Hofe. Einer gab mir
einen starken Schlag aus daß Handgelenk, sodaß man allgemein
glaubte, er hätte meine Hand in Stücke geschlagen. Der Kon-
stabler hätte gem den Frieden wieder hergestellt und einige, die
mich geschlagen, eingesteckt; aber ich ließ es nicht zu. Nachdem
ich zu ihnen geredet, ging ich nach der Wohnung dez Joseph
Nicolson und der Konstabler begleitete mich, um mich vor der
Menge zu schützen.
Am Nachmittag hatte der Priester einen andern Priester
kommen lassen, einen sehr angesehenen Mann auß London. Ehe
ich inö Turmhautz eintrat, saß ich eine Weile auf dem Platz davor
und einige Freunde mit mir; aber die Freunde wurden getrieben, ins
Turmhauz zu gehen, und ich ging ihnen nach. Der Londoner
Priester brachte in seiner Predigt alle erdenklichen Schriftftellen
von falschen Propheten rmd Antichristen und wandte sie auf unö
an. Aber alß er geendet, nahm ich alle die Schriftstellen noch
1) Friedenörichtev Pearson wurde ,,bekehrt, als er auf dem Richterstuhl saß«.


% \picinclude{./070-079/p_s075.jpg} 
Fox der Hexerei verdächtigt. Falsche Qssenbarungen usw. 75
einmal durch und kehrte sie gegen ihn. Darauf überfielen mich
die Anwesenden, aber der Konstabler befahl ihnen Ruhe. Nun
wurde der Priester zornig und erklärte, ich dürfe nicht an diesem
Ort reden. Jch erklärte ihm, er habe auch seine Stunde zum
Predigen gehabt, nun sei seine Zeit um, und nun dürfe ich so
gut die meine reden wie er, denn er sei auch nur ein Fremder
hier. Und ich öffnete ihnen die Schrift und zeigte ihnen, daß
diese Stellen, die von falschen Propheten, Betrügern und Anti-
christen reden, sie und ihreögleichen betrefse und alle, die in ihren
Fußstapfen gehen und die gleichen Früchte hervorbringen wie sie;
und nicht uns-, denn man könne uns solche Dinge nicht nach-
sagen. Jch zeigte ihnen, wie sie nicht in den Fußstapfen der
wahren Propheten und Apostel seien und wies ihnen an den
Früchten, die sie hervorbringen, nach, daß sie etz seien, von denen
die Schriftstellen handeln und nicht wir. Und ich verkündete ihnen
die Wahrheit und daß Wort deß Lebenß und wies sie aus Christ-.13,
ihren Lehrer. Alletz war ruhig während ich redete; aber alß ich
geeudet hatte und hinaus kam, waren die Priester in einer solchen
Wut, daß ihr Mund gegen mich schäumte. Der Priester des
Orts redete auf dem Turmplatz zu den Leuten und sagte ihnen:
,,Dieser Mensch hat in Laneashire alle rechtfchassenen Männer
und Frauen für sich zu gewinnen gewußt, und nun will er hier
da?-selbe tun.« Jch erwiderte ihm: ,,WaS bleibt dann für die
Priester übrig, außer solchen wie sie selber sind? Denn wenn es
die Rechtschafsenen sind, die sich zur Wahrheit bekehren und sie
aufnehmen und sich zu Christuß bekehren, so sind es die Schlechten,
die dir und deineßgleichen folgen! Etliche suchten für ihren Priester
einzutreten, und für das Zehntenwesen; aber ich sagte ihnen, sie
täten besser, für Christus einzutreten, der den Zehntenpriestern
und dem Zehntenwesen ein Ende machte und der seine Jünger
aussandte mit der Weisung: ,,untsonst zu geben, wa-3 sie umsonst
empfangen hatten«. Und des-Z Herrn Macht kam über alle und brachte
sie zum Schweigen und hielt die Schreier zurück, daß sie den Unfug,
den sie planten, nicht ausführen konnten. A13 ich zu Joseph
Nieolson zurück kam, entdeckte ich ein großeß Loch in meinem
Rock, daß von einem großen Messerstich herrührte; aber ez war
nicht tiefer alö der Rock gegangen, denn der Herr hatte ihre
Ubeltat vereitelt .....
Darnach ging ich in ein Dorf, und eine große Schar be-


% \picinclude{./070-079/p_s076.jpg} 
gleitete mich. Während ich in einem mit Leuten ganz gefüllten
Hauö das- Wort des Lebenö verkündete, gewahrte ich eine Frau,
die, wie ich gleich merkte, einen unsauberen Geist hatte. Der
Herr trieb mich, ernstlich mit ihr zu reden und ihr zu sagen, sie
sei unter dem Einfluß eineö unsauberen Geisteö; hierauf verließ
sie daß Zimmer. Weil ich fremd war an diesem Orte und die
äußeren Verhältnisse der Frau nicht kannte, wanderten sich die
Leute sehr und sagten mir nachher, ich hätte etwas Merkwiirdigeß
entdeckt; denn diese Frau sei wirklich lange als eine schlechte
Person bekannt gewesen. Der Herr hatte mir die Gabe der
Unterscheidung gegeben, durch welche ich den Zustand und die
Verfassung der Leute ost erkannte und die Geister prüfen konnte
denn nicht lange vorher, alö ich in eine Versammlung ging, sah
ich auf dem Felde einige Frauen, bei denen ich einen unsauberen
Geist erkannte; und etz trieb mich, von meinem Wege ab zu ihnen
zu gehen und ihnen ihren Zustand aufzudecken. Ein andermal
kam eine in die Versammlung in Swarthmore, und etz trieb mich,
ernstlich mit ihr zu reden und zu sagen, sie stehe unter der Macht
eines bösen Geisteö; und die Leute sagten nachher, etz sei daß
allgemein von ihr bekannt. Gin andermal kam eine andere Frau
und stand in einiger Entfernung von mir, und es trieb mich zu
ihr zu gehen und zu sagen: »Du bist eine Hure gewesen«; denn
ich erkannte den Zustand und daß Leben dieser Frau; sie antwortete
mir, es gebe viele, die ihr ihre äußern Sünden nennen können,
aber ihre inwendigen habe ihr noch niemand sagen können; daraus
sagte ich ihr, ihr Herz tue nicht recht vor dem Herrn, rmd
aus dem inwendigen komme das auöwendige; diese Frau wurde
nachher von der Wahrheit dez Herrn überzeugt und schloß sich
den Freunden an .....
Wir gingen nun nach Earliöle ..... An einem Markttage
ging ich aus den Markt. Die Magistrate hatten Drohungen er-
gehen lassen und ihre Leute geschickt; und ihre Frauen hatten
gesagt, wenn ich komme, so reißen sie mir die Haare auö, und
die Schutzleute sollten mich nur sestnehmen. Aber ich ging dennoch
auf den Platz, im Gehorsam gegen den Herrn, und verkündete
ihnen dort, daß der Tag des Herm über all ihr betrügerischeö
Tun und ihre betriigerische Ware komme; und sie sollten sich alle
abwenden von ihrem Beträgen und Überlisten und sich an Ja
und Nein halten und einander die Wahrheit sagen; dann komme


% \picinclude{./070-079/p_s077.jpg} 
Fox der Hexerei verdächtigt. Falsche Offenbarungen usw. 77
die Kraft und die Wahrheit des Herrn zu ihnen. Nachdem ich
ihnen so das Wort des Lebens verkündet hatte, in einem Ge-
dränge, das zu groß gewesen war, als daß die Schutzleute und
die Weiber der Magistrate zu mir hätten gelangen können, zog
ich ruhig weiter. Viele Soldaten und andere kamen zu mir und
einige Baptisten, die heftige Streiter waren; unter diesen war
auch ein Helfer, ein böser Mann, der, als er die Kraft des Herrn
verspürte, aufschrie vor Zorn, worauf ich meine Augen auf ihn
heftete und ernstlich zu ihm redete in der Kraft des Herm; und
er schrie: ,,Durchbohre mich nicht so mit deinen Augen! wende
deine Augen ab von mir«.
Am folgenden Ersten Tage ging ich ins Turmhaus, und
nachdem der Priester geendigt hatte, predigte ich den Leuten
die Wahrheit und das Wort des Lebens. Der Priester entfernte
sich und man wollte mich aus dem Turmhaus jagen. Aber ich
verkündete den Weg des Herrn weiter unter ihnen und sagte:
,,ich komme, euch das Wort des Lebens und der Seligkeit zu ver-
künden«. Die Macht des Herrn tat sich mächtig kund unter
ihnen, so daß sie zitterten und bebten, und meinten, das Turm-
haus schwanke, und einige meinten, es werde auf ihre Köpfe fallen;
die Weiber der Magistrate rasten und suchten mit aller Gewalt,
an mich heran zu kommen; aber die Soldaten und die Freunde
umringten mich. Zuletzt kam der ganze Pöbel der Stadt ins
Turmhaus, mit Stöcken und Steinen und schrie: ,,nieder mit
diesen rundköpfigen Schuften!« und warfen mir Steine an.
Hierauf schickte der Statthalter Soldaten ins Turmhaus, um
Ruhe zu schaffen unter den Leuten; mich nahmen sie freundlich
bei der Hand und hießen mich mit ihnen kommen. Als wir
auf die Straße kamen, war die Stadt in Aufrrchr, und einige
dieser Soldaten kamen ins Gefängnis, weil sie sich meiner ange-
nommen hatten, gegen die Leute aus der Stadt. Gin Leutnant,
der belehrt worden war, nahm mich in sein Haus, wo eine Vap-
tistenversammlung war; auch Freunde kamen dazu, und wir hatten
eine sehr ruhige Versammlung; sie hörten das Wort des Lebens
gerne, und viele nahmen es auf. Am folgenden Tage, als die
Magistrate im Stadthaus versammelt waren, ließen sie mich vor
sie bringen. Jch war eben im Haus eines Baptisten; als ich
von dem Befehl hörte, ging ich nach dem Stadthaus hinauf, wo
viel Pöbel versammelt war, der allerlei falsche Dinge über mich


% \picinclude{./070-079/p_s078.jpg} 
auögesagt hatte. Ich hatte eine lange Unterredung mit den
Magistraten, worin ich auseinandersetzte, waz für Früchte die
Predigten ihrer Priester bringen, und wie wenig Christentum darin
sei; und ich sagte ihnen, daß sie zwar alß große »Fromme«
gelten, — sie waren Preßbhterianer und Jndependenten — aber
eben nicht im Besitz ihrer Frömmigkeit seien. Nach einem langen
Verhör verurteilten sie mich zum Gefängniö, alß Gottes-lästerer,
Ketzer und Verführer, obgleich sie mich gerechter Weise keines
dieser Dinge beschuldigen konnten. EZ waren zwei Kerkermeister
im Kerker von Carliöle, ein oberer und ein unterer, die auß-
sahen wie zwei große Värensührer. A15 ich gebracht wurde,
führte mich der Oberkerkermeister in ein großeß Zimmer und sagte
mir, ich könne hier haben, was ich wolle; aber ich erwiderte
ihm, er solle kein Geld von mir erwarten, denn ich werde weder
in einem seiner Betten schlafen, noch von seinen Speisen essen,
woraus er mich in ein anderes Gemach führte, wo ich nach einiger
Zeit etwas zum drauf liegen erhielt. Hier lag ich gefangen biß-
zur Zeit der Gerichtösitzung, wo ich, wie es- allgemein hieß, er-
henkt werde. Der Oberscherifs Wilfrid Lawson, hetzte sie auf,
mich zu töten, und sagte, er wolle mich selbst biß zu meiner Hin-
richtung bewachen. Sie waren sehr streng und setzten drei Muske-
tiere zu meiner Wache, einen vor meine Türe, einen anderen
unten an die Treppe und einen dritten vor die Haustüre, und
sie ließen niemand zu mir, außer um mir das nötigste zu bringen.
Dez Nachts brachten sie Priester zu mir, oft erst um zehn Uhr,
die schrecklich roh und teuflisch waren. GS gab eine Rotte von
schottischen Priestern, Preßbyterianer, zusammengesetzt auö Neid
tmd Bo?-heit, die nicht »geschickt waren, göttliche Dinge zu reden«
und sehr schmutzige Reden führten. Aber der Herr verlieh mir
durch seine Kraft die Herrschaft über sie alle, so daß sie erkannten,
in welchem Geist sie waren und maß sie für Früchte brachten.
Auch angesehene sogenannte ,,Damen« (lmtjez) kamen, um den
Mann zu sehen, von dem es hieß, er müsse sterben. Während
die Richter und Räte miteinander berieten, auf welche Art ich
sterben solle, vereitelte der Herr in Unerwarteter Weise ihren
Anschlag, indem der Anwalt einen Einwand verbrachte, der
alle ihre Absichten über den Haufen warf, so daß sie keine
Macht mehr hatten, mich vor Gericht zu bringen .....
Nachdem die Richter die Stadt verlassen hatten, erhielt der


% \picinclude{./070-079/p_s079.jpg} 
Fox der Hexerei verdächtigt. Falsche Ossenbarungen usw. 79
Kerkermeister Befehl, mich in den untersten Kerker zu den Straßen-
räubern, Dieben und Mördern zu werfen, obgleich ich schon vor-
her in sehr strengem Gewahrsam gewesen war. Jch kam nun
an einen gräulichen, schmutzigen Ort, wo nicht einmal ein Abtritt
war, Amd Frauen und Männer in unziemlicher Weise zusammen-
gesperrt waren, und die Gefangenen waren voll Läuse, so daß eine
Frau fast davon aufgesressen wurde; aber so schlecht auch der
Ort war, so kamen doch die Gefangenen alle dazu, mir zugetan
und ganz nachgiebig zu werden, und etliche wurden von der Wahr-
heit bekehrt, wie dieß bei Zöllnern und Huren zu allen Zeiten
geschehen, so daß sie jeden Priester, der anß Gitter kam, um mit
ihnen zu diöputieren, zu Schanden machen konnten. Der Kerker-
meister war sehr hart und der Unterkerkermeister roh gegen mich und
gegen die Freunde, die zu mir kamen. Gr schlug oft Freunde,
die nur anß Gitter kamen, um mich zu sehen, mit einem großen
Knüttel. Jch konnte am Gitter hinaus steigen, um zuweilen etwas
Fleisch herein zu langen, was ihn schrecklich böß machte. Einmal
überkam ihn ein solcher Zorn, daß er mich mit einem Kniittel
durchprügelte und dazu schrie: ,,komm vom Fenster weg!« obschon
ich gerade damals nicht dran war. Während er mich schlug,
—kam es in der Kraft des Herrn über mich, zu singen, maß ihn
noch wütender machte. Esr holte einen Geigenspieler und ließ
ihn vor mir spielen, weil er meinte, mich damit zu verdrießen.
Aber während seinem Spiel kam eZ über mich, in Gottes ewiger
Kraft zu singen, und meine Stimme übertäubte den Lärm des
Geigerö, waö ihn so oerwirrte, daß er das Spielen aufgab und
sich daoonmachte.
Richter Vensontz Frau fühlte sich getrieben, mich zu besuchen,
und kein anderes- Fleisch zu essen, al-3 von dem, daö man mir
an die Kerkertür brachte. Später wurde sie selbst in York ins
Gesängniß getan, während sie schwanger war, weil sie einem
Priester widersprochen hatte, und man gestattete ihr nicht, aus
dem Gefängniö zu gehen zur Zeit ihrer Niederkunft; so gebar
sie im Kerker ein Kind. Sie war eine gläubige, gottselige Frau,
und blieb etz biö zu ihrem Tode.
Während meiner Gefangenschaft im Kerker zu Carliöle ver-
breitete sich daß Gerücht von meiner wahrscheinlicher! Hinrichttmg
überall hin. Alß sie im Parlament — ich glaube es wurde das
kleine Parlament genannt — hörten, es sollte in Carliöle ein


% \picinclude{./080-089/p_s080.jpg} 
junger Mann um seines Glaubens willen hingerichtet werden,
schrieben sie deshalb an die Magistrate.
Ungefähr um die gleiche Zeit schrieb ich an die Behörden
von Earlisle, die mich ins Gefängnis geworfen und die die
Freunde auf Anstiften der zehntengierigen Priester verfolgten:
,,Freunde! Thomas Eraston und Cuthbert Stadholm,
Guer Tun ist in London bei den Gutgesinnten bekannt ge-
worden. Was habt ihr alles geleistet an Gesangennehmen,
Güterschändungen, Metzeleien und anderen Scheußlichkeiten in den
letzten paar Jahren! ganz menschenunwiirdig, wie wenn ihr
noch nie die Schrift gelesen und zu Herzen genommen hättet!
Jst das das Ziel der Religion Earlisles und seiner Kirche und
seiner Ehristlichkeit? ihr habt es zu schanden gemacht mit eurer
Blindheit, eurem tollen Treiben und eurem verkehrten Gtfern.
War es nicht immer die Art der blinden Leiter und der falschen
Propheten zu zanken (Jes. 56), mit denen, die ihnen den Mund
nicht füllen wollen? Seid ihr nicht die Lasttiere und Diener der
Priester gewesen? Wenn sie euch anspornen, das Schwert gegen
den Unschuldigen zu gebrauchen, so rennt ihr auf solche, die nach 3
den Befehlen der Schrift die Waffe nicht gebrauchen dürfen, loss
Und doch wollt ihr eure unheiligen Hände und gemeinen Lippen
zu Gott erheben, und gebet oor, zu fasten und seid doch voll
Hader und Zank (Jes. 58, 4). Brannte nie euer Herz in euch?
habt ihr nie über euren Zustand nachgedacht? Seid ihr ganz
der Lust des Teufels, dem Verfolgen, anheimgefallen? Wo ist
eure Feindes-liebe? (Matth. 5). Wo ist euer Beherbergen der
Fremdlinge? (Matth. 25, 35). Wie überwindet ihr Böses mit
Gutem? (Röm. 12, 21). Wo sind eure Lehrer, die ,,durch heil-
same Lehre die Widersprecher strafen?« (Tit. 1, 9) .... Leset die
Schrift und sehet, wie unähnlich ihr den Aposteln und Propheten
seid; und wie ihr denen gleichet, die die Propheten, die Apostel
und Christus verfolgten. Jhr gehet in ihren Fußstapfen und
kämpfet mit Fleisch und Blut, nicht mit den Fürsten der Welt,
die in der Finsternis dieser Welt herrschen, und mit den bösen
Geistern unter dem Himmel« (Gph. 6, 12). Jn keinem anderen
Lande geschehen solche Greuel, daß man den Leuten ihr Gut
raubt, ihnen ihre Ochsen und Rinder nimmt, ihre Schafe, ihr
Getreide und ihr Hausgeräte und gibt es den Priestern, die doch
nichts für sie gearbeitet haben. Jhr seid eher Straßenräuber


% \picinclude{./080-089/p_s081.jpg} 
Fox der Hexerei verdächtigt. Falsche Qsfenbarungen usw. 81
alß Diener Gottes gegen die Freunde; ihr verklagt sie bei euren
Gerichten und legt ihnen Bußen auf, weil sie die Gebote Christi
nicht übertreten, also nicht schwören wollen« ..... G
Anthony Pearson and Gervase Benson dursten mich nicht im
Gesängniö besuchen, obwohl sie Frieden?-richter waren. Sie
schrieben darum an die Magistrate und Priester von Carlißle:
,,Wir bezeugen, daß dieser George Fox, der von den Magi-
straten, von den Friedenörichtern, den Priestern und dem Volt ver-
folgt wird und gegenwärtig alt?. Gotteßlästerer und Verführer
gefangen gesetzt ist, ein Prediger dez Worteß Gottes ist und daß
ewige Evangelium verkündet; durch sein mächtiges Predigen hat
der große Vater der Heiligen den Blinden die Augen geöffnet,
den Tauben die Ohren aufgetan, die Gefangenen erlöst und die
Toten auferweckt (Jes. 35, 5). Christuß wird jetzt gepredigt unter
den Seinen, wie er war und ist; und weil er mm, in der Gestalt
seineö getreuen Dienerß, wieder erscheint, sv verfolgen ihn die
Abgefallenen, Fürsten, Herrscher, Priester und Volk. Nicht alß
ein Übeltäter leidet er von euch, ihr Magistrate, sondern weil er
nicht abgefallen ist und gegen daß Treiben der Welt und daß
Böse auftritt. EZ ist immer so gewesen, daß, wo die oerderbte
Natur den Samen Gottes unterdrückte, die Verderbten suchen die,
in denen dieser Same ausging, gefangen zu nehmen .... Wie
Christuö daö, maß man einem der Geringsten erweist, als ihm
getan ansieht (Matth. 5, 25), also siehet er auch daß, wa?. man
ihnen nicht tut, als ihm nicht getan an. Wenn ihr nun soweit
geht, daß ihr nicht einmal anderen gestatten wollt, einen gefangenen
Bruder in seinen Leiden zu besuchen, so werdet ihr in den feurigen
Pfuhl, der mit Schwefel brennt, geworfen (Offb. 19, 20). Der
Herr ist gekommen, die Berge zu stürzen und zu Staub zu zer-
nialmen (Jes. 41, 15), und er wird rächen die Unterdrückung der
Gewissen seineö Volkes- an allen ungerechten Herrschern, Beamten
und Gesetzen. Er wird seinem Volke sein Gesetz geben nicht nach
dem, wat?. vor Augen ist, sondern nach Recht und Gerechtigkeit.
Man hat nun gesehen, wie eure Herzen voll Haß sind gegen die
Wahrheit Gotteö, die er durch sein von der Welt oerachteteö und
zum Spott ,,Quäker« genannteö Volk verkünden läßt. Jhr seid
ärger als die Heiden, die Pauluö inö Gefängniß warfen; denn
niemand hat damals seinen Freunden verboten, ihn zu besuchen,
Gkotge Fox. 6


% \picinclude{./080-089/p_s082.jpg}
darum treten sie gegen euch als Zeugen auf. M ist offenbar
geworden, daß ihr denen gleich seid, die Christus töteten und die
Apostel gefangen nahmen unter dem gleichen Vorwand, nämlich
daß sie den Jrrtum Wahrheit und die Diener Gottes Gottes-
lästerer nannten. Aber das Gericht, das über euch kommen wird,
ist schrecklich, ihr ungerechten Magistrate und Priester und ihr
alle, die ihr mit Worten die Wahrheit bekennet, und doch die
Kraft der Wahrheit und die, die in der Wahrheit sind und für die
Wahrheit einstehen, verfobget. Gehet in euch, dieweil es Zeit
ist, und bedenket, was Jesaias 17 geschrieben steht!«
Geroase Benson
Anthony Pearson.
Bald darauf kam die Macht des Herrn über die Richter
und sie setzten mich frei. Kurz vorher war Anthony Pearson
mit dem Gouverneur in meinen Kerker gekommen um zu sehen,
wie ich behandelt werde. Sie fanden den Ort so gräulich und
den Geruch so schlecht, daß sie sich über die Magistrate entsetzten,
die solches von dem Kerkermeister geschehen ließen. Sie ließen
die Wärter in den Kerker kommen und sich für ihr Betragen
rechtfertigen. Den Unterkerkermeister, der so grob gewesen war,
sperrten sie darauf zu uns ins Gefängnis unter die Räuber.
Nachdem ich nun frei war, ging ich zu Thomas Bewley . . .
Dann ging ich auss Land und hatte viele große Versammlungen . . .
und tausende bekehrten sich zum Herrn Jesus Christus-.
Dann ging ich nach Westmorland . . . Durham, Hexhain . . .
Gilsland . . . nach Eumberland ..... Hier überall, sowie in
Northumberland, Laneashire und Yorkshire fanden große Be-
kehrungen statt, und was Gott gepflanzt hatte, wuchs und gedieh
unter dem Himmelsregen von oben und Gottes leuchtender Herr-
lichkeit, sodaß sich vieler Mund öffnete zum Lobe Gottes; ja: ,,aus
dem Munde der Unmitndigen und Säuglinge richtete er sich eine
Macht zu« (Psalm 8, Z).


% \picinclude{./080-089/p_s083.jpg} 

% \picinclude{./080-089/p_s083.jpg} 

%%%%%%%%%%%%%%%%%%% Kapitel 7. %%%%%%%%%%%%%%%%%%%%%%%%%%%%%%

\chapter[Begegnung mit Oliver Cromwell]{Begegnung mit Oliver Cromwell}

\begin{center}
\textbf{Kämpfe mit schwärmerischen Ranters und 
zehntengierigen Priestern.
Fox in Wetstone verhaftet und vor Cromwell geschickt.}
\end{center}

\section{Quaker im geschäftlichen Umgang}

Die Priester und \textit{Frommen} traten aufs neue mit ihren
Prophezeiungen gegen uns auf. Schon lange hatten sie 
vorausgesagt, das wir binnen eines Monats vernichtet sein werden;
hernach verlängerten sie die Frist auf ein halbes Jahr; als aber
auch diese Zeit längst um war, und wir im Gegenteil an Zahl
zunahmen, streuten sie aus, wir werden einander gegenseitig 
verzehren. Es kam nämlich oft vor, das nach den Versammlungen
manche, die einen weiten Heimweg hatten, bei Freunden blieben,
es waren oft mehr Leute als Betten vorhanden, so das Viele auf
dem Heu übernachten mussten. Da wurden die \textit{Frommen} von
der Furcht Cains gepackt; sie hatten Angst, das, wenn wir 
einander zu Grunde gerichtet hätten, wir dann der Gemeinde zur
Last fallen und uns von ihr unterhalten lassen werden. Als sie
aber sahen, wie der Herr den Freunden Segen und Gedeihen
gab, wie dem Abraham, \zitat{beim Acker und beim Korb, beim 
Eingehen und beim Ausgehen, beim Aufstehen und beim Niederliegen}
(5. Mose 28\bibel{Mose 5. 28@5. Mose 28}), da erkannten sie 
die Ungerechtigkeit ihrer Prophezeiungen, und das man 
\zitat{umsonst flucht, wo der Herr segnet}
(4. Mose 23\bibel{Mose 4. 23@4. Mose 23}). Als nach den 
ersten Bekehrungen die Freundes
den Hut nicht vor den Leuten abnahmen, einer einzelnen Person
nicht mit \zitat{ihr}, sondern mit \zitat{du}\index{Anrede} und 
\zitat{dich} antworteten, sich nicht
verneigten und nicht bei der Begrüßung schmeichelhafte Worte
gebrauchten und nicht die Art und Weise der Welt mitmachten, da 
verloren viele von ihnen in ihren Geschäften die Kundschaft; man
scheute sich vor ihnen und wollte keine Geschäfte mit ihnen machen,
so das eine Zeit lang die Freunde kaum ihr Brot verdienten. Aber
als die Leute sahen, wie treu und ehrlich die Freunde waren,
und das ihr \zitat{ja — ja} und ihr \zitat{nein — nein} war; 
das sie Wort hielten \index{Zeugnis!Wahrhaftigkeit}
im Verkehr und niemanden hintergingen noch betrogen, und wie
der Herr ihnen Segen und Gedeihen gab; wie ein Kind, das sie
schickten, um einen Einkauf zu machen, gerade so gut bedient
wurde wie sie selbst, da predigte das Leben und der Wandel der
Freunde, und es traf das, was von Gott kam, in ihren Gewissen.
Nun wandelten sich die Dinge dermaßen, das man beständig
fragen hörte: \zitat{Wo ist ein Krämer, ein Tuchhändler, ein 
Schneider,
% \picinclude{./080-089/p_s084.jpg} 
ein Schuster, ein Handwerker, der Quäker ist?}
\index{Geschäftlicher Erfolg} Die Freunde
bekamen mehr Arbeit als manche andere Handwerker und 
beteiligten sich reger am geschäftlichen Verkehr. Nun schlugen die
gehässigen \textit{Frommen} einen anderen Ton an und fingen an zu
murren: \zitat{Wenn wir diese Quäker gewähren lassen, so werden sie
uns den Handel des ganzen Landes an sich reisen.} Also tat
der Herr an seinem Volke, und es ist mein ernstlichster Wunsch
das alle, die seine heilige Wahrheit bekennen, in der Erkenntnits
bewahrt und durch den Geist und die Kraft in der Treue erhalten
bleiben mögen, erstlich gegen Gott, im Gehorsam in allen Dingen,
und dann gegen die Menschen, in Rechtschaffenheit und Gerechtigkeit 
in allem Verkehr; damit Gott der Herr verherrlicht werde
durch einen Wandel in Wahrheit und Heiligkeit, Gerechtigkeit und
Gottseligkeit [...].

Die Priester in Newcastle, Kendal und anderen nördlichen
Gegenden waren sehr aufgebracht gegen uns. Einer, namens
Gilpin\person{Gilpin}, der manchmal zu uns nach Kendal gekommen war, war
bald von der Wahrheit abgefallen und aus allerlei einfältige
Gedanken gekommen, und die Priester gebrauchten nun das gegen
uns, wo sie nur konnten; aber die Kraft des Herrn warf sie
alle darnieder. Der Herr vernichtete zwei der Verfolgungsüchtigen
Richter von Carlisle und der dritte wurde einige Zeit darauf
seines Amts entsetzt und verließ die Stadt.


Um diese Zeit wurde den Soldaten der Eid, den sie Oliver
Cromwell\person{Cromwell, Oliver} schwören sollten, 
vorgelegt, und viele wurden entlassen,
weil sie im Gehorsam gegen Christus nicht schwören konnten.
Einer von diesen war John Stubbs\person{Stubbs, John}, 
der bekehrt worden war
während meiner Gefangenschaft in Carlisle, und ein guter Soldat
im Kampfe des Lammes und ein treuer Jünger Jesu geworden
ist. Er reiste Viel umher im Dienste des Herrn, in Holland,
Schottland\index{Schottland}, Italien\index{Italien}, 
Irland\index{Irland}, Ägypten\index{Ägypten}, 
Amerika\index{Amerika}. Und die Kraft
Gottes bewährte ihn vor den Händen der 
Papisten\index{Papisten}, obgleich er
oft in großer Gefahr vor der Inquisition\index{Inquisition} 
war. Andere unter
den Soldaten\index{Soldat} jedoch, die wohl ihrer 
Überzeugung nach bekehrt
worden waren, aber nicht zum Gehorsam gegen die Wahrheit 
gelangten, schwuren den Eid Cromwells: als diese später in 
Schottland waren, kamen sie in die Nähe einer Garnison; die dortige
Mannschaft glaubte es seien Feinde und töteten sie [...].


Der Herr trieb viele von denen, die er auserlesen hatte, in
% \picinclude{./080-089/p_s085.jpg} 
seinem Weinberg zu arbeiten, nach Süden zu gehen und sich im
Dienste des Evangeliums nach den südlichen und westlichen Teilen
des Landes zu verteilen; so gingen Francis 
Howgill\person{Howgill, Francis} und Edward
Burrough\person{Burrough, Edward} nach London\ort{London}, 
John Camm\person{Camm, John} und John 
Audland\person{Audland, John} nach
Bristol\ort{Bristol}, Richard 
Hubberthorn\person{Hubberthorn, Richard} und George 
Whitehead\person{Whitehead, George}\footnote{George Whitehead 
und Thomas Holmes, zwei eifrige Quäkerprediger.
(Näheres f. Weingarten a. a. D.)} gegen
Norwich\ort{Norwich}, Thomas Holmes\person{Holmes, Thomas} nach 
Wales\ort{Wales} und andere nach anderen
Richtungen; etwa sechzig Diener hatte der Herr ausersehen und
aus dem Norden in die Verschiedenen Teile des Landes gesandt.

\section{Ranter und falsch Propheten}

Um die Zeit fingen Rice Jones\person{Jones, Rice} 
von Nottingham\ort{Nottingham}, ein früherer
Baptist\index{Baptist} und jetzt Ranter\index{Ranter}, und 
seine Anhänger an, gegen mich
zu prophezeien\index{Prophezeiung}; sie sagten, ich hätte 
jetzt meinen Höhepunkt
erreicht und werde nun bald tief fallen [...] Aber seine und
der Seinen Weissagungen\index{Weissagung} erfüllte sich 
an ihnen selber; denn bald
darauf fielen sie ganz auseinander und viele von ihnen wurden
Freunde und blieben es; und durch des Herrn mächtige Macht und
Wahrheit vermehrten sich die Freunde [...] Rice Jones 
dagegen leistete den Eid\index{Eid} und war also dem Gebot Christi 
ungehorsam. Viele falsche Propheten\index{Propheten!falsche} 
haben sich gegen mich erhoben,
aber der Herr hat sie alle vernichtet und wird auch ferner alle
vernichten, die sich gegen seinen gesegneten Samen erheben [...].
In der Nähe von Kidsley-Park\ort{Kidsley-Park} stieß ich auf 
eine Schar Ranter\index{Ranter}; aber die Kraft des Herrn 
hielt sie drunten. Von da
ging ich in die Gegend des Peak zu Thomas 
Hammersley\person{Hammersley, Thomas},
wohin die Ranter dieser Gegend kamen und Viele angesehene
\textit{Fromme}. Die Ranter traten gegen mich auf und fingen an
zu schwören; als ich ihnen deswegen Vorstellungen machte, 
versuchten sie, Schriftstellen zu bringen und sagten, Abraham, Jakob
und Joseph haben geschworen und die Priester und Moses und
die Engel. Ich erwiderte: \zitat{ich gebe zu, das alle diese es taten,
wie die Schrift es berichtet; Christus aber 
sagt: \zitat{schwöret nicht!}
Und Christus ist das Ende der Propheten und des alten 
Priestertums und des Gesetzes Moses und regiert über das Haus Jakobs
und Josephs, und er sagt: \zitat{ihr sollt nicht schwören.} Und als
Gott den Erstgeborenen in die Welt sandte, sagte er: \zitat{alle Engel
sollen ihn anbeten} (Hebr. 1:6\bibel{Hebr. 01:06@Hebr. 1:6}), 
also diesen Christus, der sagte, ihr
sollt nicht schwören. Und was die Begründung anbelangt, welche
% \picinclude{./080-089/p_s086.jpg} 
die Menschen für das Schwören geltend machen, um ihre 
Streitigkeiten zu Ende zu bringen, so hat Christus, der gesagt hat,
ihr sollt nicht schwören, den Teufel und seine Werke, deren eines
eben das Streiten ist, vernichtet. Und Gott sagt: \zitat{dies ist mein
lieber Sohn, an dem ich Wohlgefallen habe, ihn sollt ihr hören.}
(Mark. 9:7\bibel{Mark. 09:07@Mark. 9:7}). Also soll man den 
Sohn hören, der das Schwören
verbietet. Und der Apostel Jakobus, welcher den Sohn hörte,
und ihm folgte und ihn verkündete, verbietet das Schwören,
Jakobus 5:12\bibel{Jakobus 05:12@Jakobus 5:12}.} Die Kraft 
des Herrn erfasste sie und sein
Sohn und seine Lehre beherrschten sie. Das Wort des Lebens
wurde reich und herrlich unter ihnen verkündet an dem Tage,
und viele wurden bekehrt.


Diesem Thomas Hammersley wurde einmal gestattet, an
einem Geschworenengericht als Geschworener zu amtieren ohne
einen Eid\index{Eid} abzulegen; als er dann, als 
Vorsitzender, sein Gutachten abgab, erklärte der Richter, 
er sei nun doch schon seit Vielen
Jahren Richter, aber er habe noch nie ein so redliches Gutachten
gehört, als das von diesem Quäker! ES liese sich noch viel
derartiges berichten, wenn die Zeit reichen würde. Die herrliche
Wahrheit des Herrn gos sich aus; ihr gebühret Preis und Ehre
ewiglich!
Aus der Durchreise durch Derbyshire besuchte ich überall
Freunde, bis ich nach Swannington kam; hier war eine grose
Versammlung, zu der Baptisten, Ranter und viele andere ,,Fromme«
kamen. ES hatte viele Zusammenstöse mit ihnen und den Priestern
der Stadt gegeben. Von überallher kamen Freunde zu dieser
Versammlung, so John Audland, Franciö Howgill, Edward Pyot
von Bristol und Edward Vurrough aus London und es wurden
viele bekehrt. Die Ranter machten Störungen und benahmen sich
sehr unverschämt; aber schlieslich kam die Macht dez Herm über
sie und sie unterlagen. Am darauffolgenden Tage kam Jacob
Bottomley, ein groser Ranter von Leieester; aber die Kraft des
Herrn überwältigte ihn. So auch einen Priester. Wir liesen
den Rantern sagen, sie sollten kommen und ez mit ihrem Gott
versuchen; sie kamen in Haufen und waren sehr wild und sangen
und pfifsen und tanzten; aber die Kraft dez Herrn überwältigte
sie so, das viele von ihnen bekehrt wurden.
Von hier ging ich nach Twycros, wohin auch Ranter kamen
und vor mir sangen und tanzten; aber in der Furcht des Herrn


% \picinclude{./080-089/p_s087.jpg} 
Kämpfe mit schwärmerischen Ranters und zehntengierigen Priestern usw. 87
trieb es mich, sie zu tadeln; und die Kraft des Herrn kam über sie,
so das einige von ihnen bekehrt wurden und den Geist Gottes
aufnahmen. Sie sind tüchtige Leute geworden, die rechtschaffen
in der Wahrheit Christi leben und wandeln. Jch ging zu Anthony
Brickleh in Warwickshire, wo eine grose Versammlung war;
mehrere Baptisten und andere kamen und lärmten; aber die Kraft
des Herrn kam über sie.
Hierauf ging ich nach Drayton in Leicestershire, um meine
Verwandten zu besuchen. Kaum war ich angekommen, so lies
der Priester Nathanael Stephens, der noch einen andern Priester
hatte kommen lassen und die Umgegend von meinem Kommen
benachrichtigt hatte, mich zu sich holen, denn sie konnten nichts
machen, ehe ich kam. Da ich drei Jahre meine Angehörigen
nicht gesehen hatte, so wuste ich nichts von ihren Absichten. Jch
ging nun aus den Platz des Turmhauses, wo die beiden
Priester waren, und wo sich eine Menge Leute versammelt hatten.
Als ich kam, wollten die Leute, das ich ins Turmhaus gehe; ich
fragte sie, was ich dort tun solle; sie erwiderten, Stephens
könne die Kälte nicht ertragen; ich sagte, er könne sie so gut
ertragen wie ich. Zuletzt begaben wir uns in einen grosen Saal;
Richard Farnsworth war auch dabei; wir hatten einen grosen
Disput mit den Priestern über ihren Wandel, und das sie so
sehr das Gegenteil von dem seien, was Christus und die Apostel
gewesen. Die Priester wollten wissen, wo die Zehnten verboten
oder aufgehoben seien; ich wies es ihnen nach im 7. Kap. des
Hebräerbrieses, wo nicht nur die Zehnten, sondern das ganze
Priestertum, das Zehnten annahm, aufgehoben war und das
Gesetz, nach welchem das Priestertum eingesetzt und die Zehnten
erhoben wurden. Hierauf hetzten die Priester das Volk zur Frech-
heit und Roheit gegen uns auf. Jeh hatte Stephens seit seiner
Kindheit gekannt und konnte ihnen darum aufdecken, was für
eine Art von Mensch er sei und was hinter seinen Predigten
stecke, und wie er, wie alle Priester, die Verheisungen aus den
alten Menschen, der sterben mus, bezog; dann zeigte ich ihnen,
das die Verheisungen vielmehr dem Samen galten, nicht den
vielen Samen, sondern dem einen Samen, Christus, der derselbe
ist in Mann und Weib; denn alle müssen wiedergeboren werden,
ehe sie ins Reich Gottes eingehen können. Er erwiderte mir
daraus, ich sollte nicht in der Weise richten; ich entgegnete ihm,


% \picinclude{./080-089/p_s088.jpg} 
,,der Geistliche richtet alle?-« (1. Cor. 2, 15); er gab zu, das dietz
genau der Schrift gemäs sei; dann aber fuhr er fort: ,,ihr Nach-
barn, das ist die Sache: George Fox ist zum Lichte der Sonne
gekommen und nun möchte er mein Sternenlicht au8löschen.« Jch
erwiderte: ,,ich will nicht das- kleinste Mas von dem, was einer
von Gott hat, in jemand unterdrücken, noch viel weniger sein
Sternenlicht auölöschen, wenn eS ein wirkliches Sternenlicht ist,
ein Licht vom Mvrgenstern.« Dann erklärte ich ihm, das, wenn
er etwas von Gott oder Ehristus empfangen habe, er umsonst
predigen müsse und nicht Zehnten nehmen von den Leuten sür
seine Predigten, da er ja gesehen habe, wie Christu;3 seinen
Jüngern befohlen habe, umsonst zu geben, wie sie es- umsonst
empfangen hätten. Jch schärste ihm also ein, nicht mehr für
Zehnten und Lohn zu predigen. Aber er sagte, dem werde er
sich nicht fügen. Die Leute fingen an, unverschämt zu werden,
und wir brachen darum auf. Dennoch waren etliche an dem
Tage der Wahrheit zugetan worden. Ghe ich fort ging, sagte
ich ihnen, das ich im Sinn habe, nächste Woche, so Gott wolle,
wieder in der Stadt zu sein. Jn der Zwischenzeit ging ich in
die Umgegend und hielt Versammlungen, und nach acht Tagen
kam ich wieder zurück. Der Priester hatte für diese Zeit 7 Priester
kommen lassen, um ihm zu helfen; und Stephens hatte in einem
Gottesdienst am Markttage in Adderston angezeigt, das an dem
und dem Tage ein Diöput mit mir stattfinden werde. Jch wuste
nichts davon und hatte nur gesagt, ich werde über acht Tage
wieder in der Stadt sein. Die acht Priester hatten etliche hundert
Leute versammelt, meist aua der Umgegend und wollten, ich sollte
ins Turmhauö gehen; aber ich wollte nicht hingehen, sondem
ich ging aus einen Hügel und redete von dort zum Volk .....
GS kamen einige Unverschämte und nahmen mich aus die
Arme und trugen mich unter die Türe dee Turmhauses, in der
Absicht, mich mit Gewalt ins Turmhaus zu bringen; da aber
die Tür geschlossen war, purzelten sie alle übereinander; und ich
lag zu unterst. So bald ich konnte, kroch ich hervor und ging wieder
auf den Hügel; nun schleppten sie mich biö an die Mauer
dez Turmhauseö und setzten mich auf eine Art Steinbank; alle
Priester waren auch herbeigelaufen und standen mitten unter dem
Volk herum, und alle schrieen: »Beweise, beweise!« Jch sagte,
ich hörte nicht auf ihre Stimmen, denn es seien die Stimmen von


% \picinclude{./080-089/p_s089.jpg} 
Kämpfe mit schwiirmetischen Ramerz und zehntengierigen Priestern usw. 89
Mietlingen und Fremdlingen. Sie schrieen wieder: ,,Beweise,
beweise!« Ich wies aus Johannes, wo sie sehen können, wac-
Christus zu ihresgleichen sage, nämlich: »Jch bin der gute Hirte,
der sein Leben gibt für seine Schafe, der Mietling aber flieht,
wenn der Wolf kommt.« Jch schlug ihnen vor, ihnen zu beweisen,
das sie solche Mietlinge seien; darauf rissen die Priester mich
wieder herunter und stiegen selber alle auf Steinbänke
an der Mauer des Turmhauses. Da fühlte ich, wie Gottes
mächtige Kraft über alle kam und sprach zu ihnen: ,,Wenn ihr
mir Gehör schenken wollt und mich ruhig anhören, so will ich
euch (mus der Schrift zeigen, warum ich die acht Priester oder
Lehrer, die vor mir stehen, nicht anerkenne und überhaupt keine
Mietlingslehrer der Welt.« Priester und Volk erklärten sich bereit
zu hören. Da zeigte, ich ihnen aus den Propheten Jesaja,
Jeremia, Ezechiel, Micha, Maleachi und anderen, das sie in den
Fusstapfen derer wandeln, gegen die Gott seine Propheten ge-
sandt hatte ....
Dann als ich an das neue Testament kam, zeigte ich ihnen,
das sie wie die Hohenpriester und Schriftgelehrten seien, und
wie die Pharisäer, gegen die Christuö wehe! schrie (Matth. 23).
Jsndem ich in dieser Weise ausführlich aus der Schrift bewiesen
hatte, warum sie den Pharisäern gleichen, . . . und sie vor allem
Volk unter die Pharisäer, falschen Propheten und Verführer
gerechnet hatte und gezeigt, wie ihresgleichen von den wahren
Propheten und Christus verdammt werden, wies ich sie aus das
Licht Jesu Christi hin, das einen jeden, der in die Welt kommt,
erleuchtet (Joh. 1, 9), und durch dieses Licht könnten sie erkennen,
, ob das Gesagte wahr sei. Sie mochten nichts davon hören, das
ich sie aus das- Göttliche in ihnen, auf das Licht Jesu Christi
hinwieö. Bis dahin waren sie alle ruhig gewesen, nun aber
rief einer der »From1nen«: ,,Wirst du denn nie fertig, Fox?«
Ich erwiderte, ich sei nun bald fertig; ich fuhr noch eine Weile
fort, bis ich fühlte, das ich an ihnen getan hatte, mas ich muste
in der Kraft dez Herrn ...... A13 ich fertig war, flüsterten
die Priester untereinander, und Priester Stephenö kam zu mir
und verlangte, das mein Vater und mein Bruder und ich mit
ihm beseite kommen, damit er mit uns reden könne; und die
anderen Priester musten das Volk davon abhalten uns nachzu-
kommen. Jch ging sehr ungern mit ihm, aber da daö Volk schrie:


% \picinclude{./090-099/p_s090.jpg} 
,,Geh, George, geh nur,« so fürchtete ich, das, wenn ich nicht
ginge, man sage, ich sei meinen Eltern ungehorsam; so ging ich,
und die übrigen Priester wollten das Volk abhalten, aber es
gelang ihnen nicht, denn da alle uns- hören wollten, wurden wir
ganz umringt. Jch fragte den Priester, mas er zu sagen habe?
er antwortete, wenn er nicht aus dem rechten Wege sei, so sollte
ich für ihn beten; und wenn ich nicht auf dem rechten Wege sei,
so wollte er für mich beten; und er wolle mir oorsagen, was ich
für ihn beten solle. Ich erwiderte ihm: ,,eS scheint, das du nicht
einmal weist, ob du auf dem rechten Wege bist; ich aber weis,
das ich auf dem rechten Wege bin, Jesus Christus, in welchem
du nicht bist, und du wolltest mir vorsagen, wie ich zu beten
habe, und verwirfst doch das Common-Prayerbook so gut wie ich,
und ich verwerfe dein Geplapper ebenfalls. So du willst, das ich
nach etwas Hergesagtem für dich bete, heist das nicht, die Lehre
der Apostel misachten und ihr Beten im Geist, der die Worte
eingibt?« Hier fingen die Leute an zu lachen; mich aber trieb
ez, weiter zu ihm zu reden. Nachdem ich ihm gesagt, was
mir zu sagen oblag, und das ich, so Gott wolle, über acht Tage wieder
in der Stadt sein werde, gingen wir fort. Die Priester machten,
das sie fort kamen und viele wurden gewonnen, denn die Kraft dez
Herrn kam über alle. Wenn sie schon meinten an diesem Tage
der Wahrheit geschadet zu haben, war doch mancher gewonnen
worden, und viele, die schon früher gewonnen worden, wurden durch
das, was an jenem Tage geschehen, bestärkt, und etz gab den
Priestern einen Stos. Mein Vater, obgleich er ein Anhänger der
Priester war, war so befriedigt, das er mit seinem Stock auf die
Erde schlug und sagte: »wahrlich, ich sehe, das wer willens ist,
bei der Wahrheit zu bleiben, dem wird sie durchhelsen« .....
Darauf zog ich wieder umher und hielt Versammlungen und
kam nach Swannington, wohin auch wieder Soldaten kamen;
aber die Versammlung war ruhig, die Macht Gotteö war
über allen, und die Soldaten störten mich nicht. Darauf ging
ich nach Leicester und Whetstone. Dahin kamen siebzehn Soldaten
auö Oberst Hackers Regiment, mit ihrem Anführer, und führten
mich, gerade vor Beginn der Versammlung, hinweg, obgleich die
Freunde, die von allen möglichen Orten hergekommen waren,
schon anfingen sich zu versammeln. Ich sagte dem Vorgesetzten,
er solle wenigstens die Freunde in Ruhe lassen, ich wolle für sie


% \picinclude{./090-099/p_s091.jpg} 
Kämpfe mit schwärmerischen Ranters und zehutengierigen Priestern usw. 91
alle haften; so nahmen sie denn mich und liesen die andern in
Ruhe, ausgenommen Alexander Parken!) der mit mir kam. Am
Abend brachten sie mich vor Oberst Hacker; sein Major, seine
Hauptleute und viele seiner Leute waren zugegen und wir gaben
auöfiihrlich Auskunft über die Priester und über die Versammlungen,
denn ez ging damals gerade das Gerücht von einer Verschwörung
gegen Oliver Eromwell. Jch hatte lange Grörterungen über das
Licht Christi, das einen jeden, der in die Welt kommt, erleuchtet
(Joh. 1, 9). Oberst Hacker fragte, ob e-3 dieses Licht aus Ehristus
gewesen sei, das den Judaö dazu geführt habe, seinen Herrn zu
verraten und sich darnach zu erhängen? Jch sagte ihm: ,,nein,
das war der Geist der Finsternis, der Christuz und sein Licht
haste.« Darauf sagte Hacker, ich solle nach Hause gehen und dort
bleiben, und nicht überall zu den Versammlungen gehen. Jch sagte
ihm, ich sei ein ganz harmloser Mensch und habe nichts mit Ver-
schwörungen zu tun, vielmehr verabscheue ich solches. Sein Sohn
Needham sagte: »Vater, dieser Mensch hat nun schon lange ge-
herrscht, eö ist Zeit, das man ihn unschädlich mache.« q Jch fragte
ihn, ,,warum, was habe ich getan? oder wem habe ich je etwas
zu leide getan? ich bin in dieser Gegend geboren und aufge-
wachsen, wer kann mir irgend etwas Böses nachsagen seit meiner
Kindheit?« Darauf fragte mich Oberst Hacker nochmals-, ob ich
nach Hause gehen wolle und dort bleiben? Ich antwortete ihm,
ich würde mich ja mit einem solchen Versprechen schuldig bekennen,
wenn ich nach Hause ginge und auö meinem Hause ein Gefängnis
machen wollte; und ginge ich dann doch zu den Versammlungen, so
würde ez heisen, ich sei dem Befehl ungehorsam. Jch erklärte
ihnen, ich gehe auf dee- Herrn Geheis zu den Versammlungen,
darum könne ich mich ihren Vorschriften nicht fügen; aber wir
seien ein sriedlichez Volk. ,,Gut denn,« sagte Oberst Hacker, »ich will
euch zum Lord Protektor schicken, durch Hauptmann Drury, einen
aus seiner Leibgarde.« Die Nacht über wurde ich als Gefangener
gehalten und am folgenden Morgen um sechö Uhr dem Haupt-
mann Drury übergeben. Ich wünschte vor dem Fortgehen noch
mit Oberst Hacker zu reden, er lies mich vor sein Bett kommen und
drang sogleich wieder in mich, nach Hause zu gehen und keine
Versammlungen zu halten; ich erklärte ihm, ich könne mich dem
1) Alexander Parker, ein Mann von vornehmer Herkunft, reiste viel im
Dienst des Quäkertnmö und schrieb viele Bücher und Briefe zu seiner Verbreitung. NT


% \picinclude{./090-099/p_s092.jpg} 
nicht fügen, sondern müsse meine Freiheit haben. ,,Dann,« sagte
er, ,,müst ihr vor den Protektorcks Hierauf kniete ich an feinem
Bett nieder und betete zum Herm, ihm zu vergeben, denn er war
ein Pilatus, auch wenn er seine Hände gewaschen hätte; und ich
flehte zum Herrn, das, wenn der Tag seiner Prüfung und Heim-
suchung komme, er sich dessen, was ich ihm gesagt, erinnern möge.
Gr war eben aufgehetzt von Priester Stephens und den andern
Priestern und ,,From1nen«, die darin ihre Bosheit ausliesen, weil
sie mich durch ihr Argument nicht hatten überwinden können
und dem Geiste Gottes in mir nicht hatten widerstehen können;
darum hatten sie nun die Soldaten geschickt, um mich zu greifen.
Als später dieser Oberst Hacker im Gefängnis in London
war, wurde es ihm ein oder zwei Tage vor seiner Hinrichtung
in Erinnerung gebracht, wie er an den Unschuldigen gehandelt
hatte, und er gedachte daran und bekannte es Margaret Fell;
und es bedriickte ihn. Nun konnte sein Sohn, der damals
gesagt hatte, ich habe genug geherrscht, es sei Zeit, mich fort zu
schaffen, zusehen, wie sein Vater sottgeschasft wurde, als man ihn
erhängte in Tyburn.
Jch wurde nun von Hauptmann Drury als Gefangener von
Leieester fortgebracht. Als wir nach Harborough kamen, fragte
er mich, ob ich heimgehen wolle und 14 Tage dort bleiben? Er
versprach mir die Freiheit, wenn ich weder Versammlungen halten
noch zu solchen gehen wolle. Jch erwiderte ihm, ich könne nichts
dergleichen versprechen; er fragte und versuchte mich wiederholt
auf dem Wege in derselben Weise, und immer gab ich ihm die-
selbe Antwort. So brachte er mich nach London und quartierte
mich in Mermaid ein; unterwegs trieb es mich, die Leute zu J
warnen vor dem Tag des Herrn, der über sie kommen werde.,
Nachdem Hauptmann Drury mich untergebracht, verlies er mich
und ging zum Protektor, um Bericht über mich zu erstatten. Als
er zurückkam, sagte er, der Protektor verlange, das ich kein
mörderisches Schwert gegen ihn oder die Regierung gebrauche,
und das ich dies in beliebigen Worten schristlich erklären und mit
meiner Unterschrift versehen solle. Jch antwortete Hauptmann
Drury nur wenig; aber am nächsten Morgen trieb mich der Herr,«
ein Schreiben an den Protektor auszusetzen, in dem ich vor dem«
Angesicht Gottes des Herrn erklärte, das ich das Tragen eines
mörderischen Schwertes oder irgend einer anderen äuseren Waffe


% \picinclude{./090-099/p_s093.jpg} 
Kämpfe mit schwiirmerischen Routers und zehntengierigen Priestern usw. 93
verabscheue, und das ich von Gott gesandt sei, Zeugnis abzu-
legen gegen jegliche Gewaltttitigkeit und gegen die Werke der
Finsternis; und um die Leute von der Finsternis zum Licht zu
bringen und vom Kriegen und Streiten zum Evangelium des
Friedens. Nachdem ich geschrieben, was der Herr mir eingegeben
hatte, setzte ich meinen Namen darunter und übergab es Haupt-
mann Drury, damit er es Oliver Eromwell gebe, was er auch
tat. Rath einiger Zeit brachte mich Hauptmann Drury vor den
Protektor in Whitehall; es war an einem Morgen, ehe er ange-
kleidet war, und einer, namens Harvey, der sich auch eine Zeit
lang zu den Freunden gehalten hatte aber ungehorsam geworden
war, bediente ihn. Als ich eintrat, trieb es mich zu sagen:
,,Friede sei mit diesem Hause,« und ich ermahnte ihn, in der Furcht
Gottes zu bleiben, damit er Weisheit von ihm empfangen möge,
das sie ihn leite; und das er alle Dinge, die in seiner Hand
seien, zu Gottes Ehre regiere. Jch redete lange mit ihm über
die Wahrheit und über die Religion, er zeigte sich sehr verständig;
aber er sagte, wir zankten mit den Priestern, die er Diener Gottes
nannte. Jch entgegnete ihm, ich zanke nicht mit ihnen, sondern
sie mit mir und mit meinen Freunden. ,,Aber«, sagte ich, ,,wenn
wir die Propheten und Apostel anerkennen, so können wir solche
Lehrer, Propheten und Hirten, gegen welche die Propheten und
Christus auftraten, nicht gut heisen, sondem wir müssen auch
gegen sie auftreten, durch denselben Geist und dieselbe Krast.«
Ferner zeigte ich ihm, das die Propheten, Christus und die
Apostel umsonst predigten und gegen die auftraten, welche es
nicht umsonst taten, sondern um schändlichen Gewinnes willen
und die um Geld wahrsagten und um Lohn lehrten (Micha 3, 11),
gierig und geizig waren und nie genug bekamen; und das die,
welche den Geist Christi und der Apostel und Propheten haben,
auch jetzt noch gegen das alles austreten müssen, wie jene damals.
Während ich sprach, sagte er mehrmals, es sei sehr gut, es sei
wahr. Ich sagte ihm, das alle, die sich Christen nennen, die
H Schrift haben, aber nicht alle die Kraft und den Geist, welche
die hatten, die die Schrift geschrieben, und dies sei der Grund,
warum sie nicht in der Gemeinschaft mit dem Vater und dem
Sohne seien, noch mit der Schrift, noch unter einander. Jch
redete noch über vieles andere mit ihm; da aber Leute herein
kamen, zog ich mich ein wenig zurück; als ich mich anschickte fort


% \picinclude{./090-099/p_s094.jpg} 
zu gehen, faste er mich bei der Hand, und sagte mit Tränen in
den Augen: ,,Komm wieder zu mir, denn wenn du und ich nur
eine Stunde im Tage beisammen wären, so würden wir einander
näher kommen«; und er fügte bei, er wünsche mir so wenig etwas
Böseö als seiner eigenen Seele. Jch sagte ihm, wenn er etz tun
würde, so würde er damit seiner eigenen Seele schaden; und ich
bat ihn, aus die Stimme Gottes zu hören, auf das er in seiner
Wei?-heit bleiben möge und ihm gehorchen; wenn er ez tue, so
werde er vor Hartherzigkeit bewahrt bleiben; wenn er aber
nicht auf Gottes Stimme höre, so werde sein Herz verhärtet
werden. E-r sagte, dietz sei wahr; daraus ging ich hinaus, und
Hauptmann Drury kam hinter mir drein und teilte mir mit, sein
Lord Protektor sage, ich sei frei und könne gehen, wohin ich
wolle. Darauf wurde ich in einen grosen Saal geführt, wo
die Kammerherrn des Lord Protektor zu speisen pflegten; ich fragte,
warum ich hierher geführt werde? sie sagten, es geschehe auf
Befehl des Protektor, damit ich mit ihnen speise. Jch hies sie,
dem Protektor sagen, das ich nicht von seinem Brote esse, noch
von seinen Getränken trinke. Al?-’ er dies hörte, sagte er: ,,nun
sehe ich, das ein Volk entstanden und heroorgetreten ist, welches
ich nicht zu gewinnen vermag, weder durch Gaben, noch durch
Ehren, noch Stellen, während mir dies bei allen anderen Sekten
und Menschen gelingt«, worauf man ihm entgegnete, das wir
ja das Eigene hingeben und darum kaum nach dem Seinigen
trachten würden ....
Jch begab mich nach London, wo wir grose und mächtige
Versammlungen hatten. Der Zudrang war so gros, das ich fast
nicht hinein konnte, und die Wahrheit breitete sich ungeheuer aus.
Thomaö Aldam und Robert Craven und viele Freunde kamen
nach London, um nach mir zu sehen; aber Alexander Parker
blieb bei mir.
Nach einiger Zeit ging ich wieder nach Whitehall und ez
trieb mich, den Tag de,8 Herrn unter ihnen zu verkünden und
das der Herr gekommen sei, sein Volk selbst zu lehren, und ich
predigte sowohl den Osfizieren alö denen von der Garde Oliver?-.
Aber ein Priester widersprach, als ich das Wort des Herm
verkündete; denn Oliver hatte verschiedene Priester um sich, und
dieser war ein Neuigkeitökrämer, ein häslicher Priester, ein hinter-
listiger, misgünstiger Mann; ich sagte ihm, er solle Buse tun


% \picinclude{./090-099/p_s095.jpg} 
Kämpfe mit schwärmerischen Ranterö und zehntengierigen Priestern usw. 95
und er setzte in der darauf folgenden Woche in seine Zeitung,
ich sei in Whitehall gewesen und habe dort einem Diener Gottes
gesagt, er solle Buse tun. Als ich wieder dorthin kam, traf ich
ihn wieder, und viele Leute schatten sich um unö. Ich bewiez
dem Priester, das er in verschiedenen Dingen gelogen habe, und
er muste schweigen. Gr schrieb in der Zeitung, ich habe silberne
Knöpfe; was: falsch war, denn sie waren blos aus Blech. Ferner
schrieb er, ich lege den Leuten Bänder um die Arme, damit sie
mir folgen; das war wieder gelogen, denn ich hatte in meinem
ganzen Leben nie Bänder getragen oder gebraucht. Drei Freunde
gingen hin, um den Priester zur Rede zu stellen und ihn zu
fragen, woher er diese Dinge habe; er sagte, eine Frau habe es
ihm gesagt; und wenn sie wieder kommen, so wolle er ihnen ihren
Namen sagen. Alz sie wieder kamen, sagte er, ez sei ein Mami
gewesen, aber er sage den Namen nicht, wenn sie wieder kommen,
wolle er ihn. dann sagen. Als sie das drittemal kamen, sagte
er ihn wieder nicht, behauptete aber, wenn ich erkläre, das alles
nicht wahr sei, so wolle er etz in die Zeitung setzen. Als darauf
die Freunde ihm diese Erklärung brachten, so wollte er sie doch
nicht aufnehmen, sondern wurde zornig. So handelte dieser
infame Lügenschmied, um der Wahrheit zu schaden und um die
Leute gegen die Freunde und die Wahrheit einzunehmen, wovon
ein ausführlicher Bericht in einem Buche, das bald darauf ge-
druckt wurde, kann ersehen werden. Diese liignerischen Priester
waren Jndependenten, wie die zu Leieester; aber des Herrn
Kraft oernichtete alle ihre Lügen, und viele kamen dazu, die
Schlechtigkeit der Priester einzusehen. Der Herr dez Himmelö
brachte mich durch seine Kraft durch alles- hindurch, und seine
herrliche Kraft tat sich kund im Lande, so das in dieser Zeit
viele Freunde getrieben wurden, umher zu ziehen, um das ewige
Evangelium zu verkünden, in allen Teilen dets Landes und auch
in Schottland; und die Herrlichkeit des Herrn erschien allen zu
seiner ewigen Ehre ..... ES fanden grose Bekehrungen in
London statt und auch mehrere im Hause deZ Protektors uf in
seiner Famlie; ich versuchte zu ihm zu gehen, aber ich be am
keinen Zutritt, die Wachen waren so unfreundlich.
Die Presbyterianer, Jndependenten und Baptisten waren sehr
erzürnt, denn viele bekehrten sich zum Herrn Jesus Christus und
hörten seine Lehre. Sie empfingen seine Kraft und fpürten sie


% \picinclude{./090-099/p_s096.jpg} 
in ihren Herzen, und das trieb sie, gegen die übrigen auf-
zutreten.

%%%%%%%%%%%%%%%%%%% Kapitel 8. %%%%%%%%%%%%%%%%%%%%%%%%%%%%%%

\chapter[Brief an den Papst.]{Brief an den Papst.}

\begin{center}
\textbf{Brief an den Papst. Die Studenten von Cambridge. Die Quaker
in der Bibel. Wachsende Entfremdung von Cromwell.}
\end{center}

Es kam über mich vom Herrn, ein kurzes Schreiben 
aufzusetzen und zu verbreiten, als Ermahnung an den 
Papst\person{Papst} und alle
Könige und Herrscher von ganz Europa:

\brief{an alle Könige und Herrscher von ganz Europa}
{
    Freunde,

    \bigskip

    Ihr Häupter und Obersten, ihr Könige und Fürsten alle,
    verfolget nicht in Erbitterung und Eifer die Lämmer Christi; wendet
    euch nicht ab, wenn Gottes Stimme, seine innige Liebe und 
    Barmherzigkeit aus der Höhe euch ruft, auf das nicht sein Arm und
    seine Macht, die jetzt die Welt ergriffen 
    haben\index{Endzeiterwartung}, euch unversehens
    erfassen. Sie kehrt sich gegen die Könige, und die Weisen werden
    weichen müssen, und ihre Krone wird zu Staub werden; und sie
    werden erniedrigt und dem Erdboden gleich gemacht werden. Der
    Herr wird König sein und wird die Krone dem geben, der seinen
    Willen tut. Die Zeit ist gekommen, das Gott der Herr Himmels
    und der Erde die Stolzen entlarven wird und ihren Ruhm stürzen.

    Ihr, die ihr Christus bekennet und liebet doch eure Feinde nicht,
    sondern nehmet im Gegenteil seine Freunde gefangen, ihr zeiget
    damit, das ihr nicht in dem Leben seid, das aus ihm kommt, ihr.
    liebet Christus nicht, wenn ihr nicht seine Gebote haltet. Des
    Herrn Zorn fängt an zu brennen, und sein Feuer verbreitet
    sich, um die Bösewichter zu zerstören, und es wird kein Zweig
    noch Reis übrig lassen. Die so ihren Wandel nicht mehr in
    Gott haben, sind nicht mehr in jenem Geist, der die Schrift 
    eingegeben hat, und nicht mehr im Lichte, damit Christus sie alle
    erleuchtet hat [...]. Darum seid schnell zuhören, schnell zu reden,
    aber langsam zu verfolgen (Jak. 1:19\bibel{Jak. 01:19@Jak. 1:19}); 
    denn der Herr führt nun
    sein Volk aus den Wegen der Welt zu Christus dem wahren
    Weg, und von allen weltlichen Kirchen zu der Kirche, die in ihm,
    dem Vater Jesu Christi, ist, und von allen Lehrern der Welt,
    um selber ihr Lehrer zu sein durch seinen Geist; von den irdischen
    Bildnissen, zum Ebenbilde seiner selbst; und von den irdischen
    % \picinclude{./090-099/p_s097.jpg} 
    Kreuzen aus Holz und Stein, zu der Kraft des Kreuzes Christi.
    Denn alle diese Bilder und Kreuze sind ein Abfall von Gott und
    seiner Kraft und dem Kreuz Christi, welches nun die Welt richten
    wird und alles niederwerfen, was ihm entgegen ist; seine Macht
    hat kein Ende.

    Lasset solches die Könige von Frankreich und von Spanien
    und den Papst wissen, damit sie alles prüfen und das Gute behalten;
    Und sie sollen vor allem prüfen, ob sie nicht den Geist dämpften
    (1. Thess. 5:19\bibel{Thess. 1. 05:19@1. Thess. 5:19}), 
    denn der grose Tag des Herrn ist über die
    Bosheit und Gottlosigkeit und Ungerechtigkeit der Menschen
    gekommen, und der Herr wird \zitat{durchs Feuer richten und durch
    sein Schwert alles Fleisch} (Jes. 66:18\bibel{Jes. 66:18}). 
    Und die Wahrheit
    und die Krone der Ehren und das Zepter der Gerechtigkeit
    werden erhöht werden; und das Göttliche, das in einem jeden
    ist, auch wenn er davon abgefallen ist, wird hiervon Zeugnis
    geben. Christus ist als Licht in die Welt gekommen und erleuchtet
    einen jeden, der in die Welt kommt, damit dadurch alle zum
    Glauben kommen. Und wer das Licht, womit Christus ihn
    erleuchtet, spürt, der spüret Christus in seinem Innern und das
    Kreuz Christi, diese Kraft Gottes; der brauchet kein hölzernes oder
    steinernes Kreuz,\index{Kreuz!Schmuck}\index{Kreuz!Symbol} 
    um an Christus und sein Kreuz gemahnt zu
    werden; denn es ist selber die Kraft Gottes, welche sich ihm
    innerlich kund tut.
    \bigskip
    \begin{flushright}G. F.\end{flushright}
}

Ferner trieb es mich, einen Brief an den Protektor zu
schreiben, um ihn zu ermahnen, aus das große Werk zu achten,
das der Herr unter allen Völkern zu tun im Begriffe ist, und
aus das Beben, das sie alle erzittern macht, damit er auf der
Hut sei, das er nicht mit seinem scharfen Verstand, seiner 
Geschicklichkeit und seiner Klugheit selbstische Nebenzwecke verfolge.
Es wurde zu der Zeit eine Verordnung zur Prüfung der
sogenannten Geistlichen erlassen, ob man sie bestätigen oder ihrer
Ämter und Besoldungen entsetzen solle, und es trieb mich, den
betreffenden Vorgesetzten darum zu schreiben.

\brief{An die vorgesetzten der Geistlichen}{
    Freunde,\index{Klerus}

    \bigskip

    [...] Christus zeigt seinen Jüngern und dem Volk, wie
    man solche wie diese zu prüfen hat Sie werden von den Menschen
    Herr genannt. Sie sitzen auf den ersten Plätzen der 
    Versammlung; sie sind Hörer aber nicht Täter. Er rief 
    siebenmal Wehe!
    über sie und verurteilte sie (Matth. 23\bibel{Matth. 23}) [...]. 
    Es gab in alten Zeiten
    % \picinclude{./090-099/p_s098.jpg} 
    ein Kornhaus, wo die Waisen, die Fremdlinge und die Witwen
    hin kamen und zu essen bekamen, und die, welche ihre Zehnten
    nicht ins Kornhaus brachten, gediehen nicht 
    (Maleachi 3\bibel{Maleachi 3}); hat
    aber Christus nicht allen Zehnten und Priestern und Tempeln
    ein Ende gemacht? [...] Sind je die Priester, die Zehnten nach
    Menschensatzungen nahmen, gediehen? [...] Warfen die Apostel
    je jemanden in den Kerker wegen der Zehnten\index{Kirchensteuer}, 
    wie ihr es jetzt tut?
    Zum Beispiel: Ralph Hollingworth\person{Hollingworth, Ralph}, 
    Priester von Phillingham\ort{Phillingham},
    hat zu Lincoln einen armen Dachdecker namens Thomas 
    Bromby\person{Bromby, Thomas (Dachdecker)}
    wegen einer kleinen Abgabe, nicht mehr als sechs Schilling, ins
    Gefängnis geworfen, wo er nun schon seit achtunddreißig Tagen
    ist; und der Priester ersuchte den Richter, das man dem Mann
    nicht erlaube, etwas zu seinem Unterhalt im Gefängnis in der
    Stadt zu verdienen. Ist dieses eine Empfehlung für euch, die ihr
    die Aufgabe habt, die Priester zu wählen? [...] Christus hieß
    seine Jünger, als er sie aussendete, umsonst zu geben, wie sie
    umsonst empfangen hatten; und in den Städten, durch die sie
    zogen, mussten sie sehen, wer würdig war, und dort bleiben und
    essen, was man ihnen vorsetzte; und als sie zu Christus zurück
    kamen und er sie fragte, ob sie Mangel gelitten hätten, so sagten
    sie: \zitat{nein}. Sie gingen nicht in die Stadt und fragten die
    Leute, wie viel sie im Jahre bekommen, wie dies jetzt geschieht
    von denen, die abgefallen sind. Der Apostel sagt, \zitat{habe ich
    nicht zu essen und zu trinken?} aber er sagt nicht: 
    \zitat{habe ich nicht
    Osterpfründe, Aufbesserungen und Geldsummen} [...] \zitat{Es soll
    dem Ochsen, der da drischt, nicht das Maul verbunden 
    werden} (5. Mos. 25:4\bibel{Mos. 5. 25:04@5. Mos. 25:4}), 
    aber sehet zu, ob ihr auch gedroschen habt und
    ob das Korn in den Scheunen ist! Dies sagt einer, der eure
    Seelen lieb hat und euer ewiges Heil will.

    \bigskip

    \begin{flushright}G. F.\end{flushright}

}

\section{Fox zeigt sich bibeltreu}

Nachdem ich einige Zeit in London\ort{London} gewesen und dort gewirkt
hatte, trieb es mich nach Bedsordshire zu John 
Crook\footnote{John Crook, früher ein angesehener Friedensrichter 
der Grafschaft Bedford, wurde ein in vielen Verfolgungen 
standhafter Quäker.}\person{Crook, John} zu gehen,
wo eine große Versammlung war und viele die Wahrheit annahmen.
John Crook sagte mir, das am folgenden Tage mehrere Herren
der Umgegend mit ihm speisen werden, um mit ihm zu diskutieren.
Sie kamen und ich redete von der ewigen Wahrheit Gottes zu
ihnen. Mehrere Freunde gingen an jenem Tage ins Turmhaus.
% \picinclude{./090-099/p_s099.jpg} 
Und in der Umgegend war auch eine Versammlung und es trieb
mich hin zu gehen, obwohl es mehrere Meilen weit weg war.
John Crook ging mit mir. Es war einer dort, 
Gritton\person{Gritton}, der Baptist\index{Baptisten} gewesen, 
aber jetzt höher hinaus wollte und sich ein Prüfer
der Geister nannte. Er sagte den Leuten, wie viel Vermögen
sie haben, und behauptete, ihnen sagen zu können, wenn ihnen
etwas gestohlen oder verbrannt wurde, wer es getan.\index{Hellsehen}
Dadurch hatte er die Gunst vieler erworben. Dieser Mann redete gerade
laut, als ich kam. Er hieß Alexander Parker\person{Parker, Alexander} 
seine Hoffnung
begründen. Alexander erwiderte: \zitat{Christus ist unsere Hofstiung;}
weil diese Antwort nicht so schnell gegeben wurde, wie er sie
erwartete, so schrie er: \zitat{sein Mund ist gestopft!} Daraus richtete
er sich an mich, denn ich stand schweigend dabei, weil er vieles
sagte, das sich nicht mit der Schrift vertrug.\index{Bibeltreu} 
Ich fragte ihn,
ob er sich aus die Schrift berufen könne? er sagte: \zitat{ja;} ich hieß
die Leute ihre Bibeln nehmen und die Stellen aussuchen, die
er angeben würde, aber er konnte es nicht. So war er beschämt
und ging fort und seine Anhänger wurden meistens gewonnen [...].
John Crook\person{Crook, John} blieb in der Kraft Gottes, aber 
er wurde seines Amtes als Richter entsetzt [...].

\section{}

Ich ging nach Romney\ort{Romney}, wo die Leute von meinem Kommen
gehört, und es war darum eine sehr grose Versammlung. Zu
dieser kam Samuel Fischer\footnote{Samuel Fischer und John 
Stubbs gingen später u. a. nach Rom und
traten dort mutig gegen papistischen Aberglauben auf. 
Fischer starb 1665 im Gefängnis in London an der Pest.}
\person{Fischer, Samuel}\person{Stubbs, John}, ein 
großer Baptistenprediger\index{Baptisten}. Er
hatte eine Pfarrei gehabt, die ihm etwa zweihundert Pfund im
Jahre eingebracht hatte und die er um des Gewissens willen 
aufgegeben hatte. Der Pfarrer der Baptisten war auch dabei und
viele ihrer Leute. Die Kraft des Herrn ward so mächtig kund,
das viele ergriffen wurden [...]. Als die Versammlung vorüber
war, sagte Samuel Fischers Frau: \zitat{so, nun last uns darüber
reden, was geistig und was fleischlich ist, damit wir die Lehre
des Geistes von der Lehre des Fleisches unterscheiden können.}.
Samuel Fischer und manche andere traten für das Wort des
Lebens ein, das an diesem Tage ihnen war erklärt worden. Der
andere Pfarrer und seine Anhänger redeten dagegen [...].
Samuel Fischer nahm die Wahrheit an und wurde ein getreuer
Prediger; er predigte umsonst und arbeitete viel für den Herrn;
% \picinclude{./100-109/p_s100.jpg} 
denn es trieb ihn, das Wort des Lebens in Dunkirk 
und Holland\ort{Holland} zu verkünden und in einigen 
Teilen Italiens\index{Italien}, sogar in Rom\index{Rom}.
Doch der Herr bewahrte ihn und seinen Begleiter John Stubbs
vor der Inquisition [...].


An einem sechsten Wochentage hatte ich eine Versammlung
in Colchester\ort{Colchester}, zu der viel \textit{Fromme} 
und die Lehrer der Independent kamen. Als ich zu reden 
aufgehört hatte und meinen Platz verließ, fing einer 
der Independentenlehrer\index{Independent} an Lärm zu
machen; Amor Stoddart\person{Stoddart, Amor}, der dies 
hörte, sagte zu mir: \zitat{Steh noch einmal aus, George,} 
denn ich hatte eben fort gehen wollen.
Ich stand nun wieder auf, als ich die lärmenden Independenten
hörte; und bald kam die Macht des Herrn über ihn und über
alle und überwältigte sie [...].


Am nächsten Ersten Tage hatten wir eine grose Versammlung
in Colchester .... Von da gingen wir nach Jpswich .... dann
nach Mendlesham, wo wir ein grose Versammlung hatten; dann
gingen wir nach Norfolk, wo wir uns von Amor Stoddart ver-
abschiedeten, der uns später wieder treffen wollte ..... Dann
zogen wir nach Yarmouth .... und Norwich, .... rmd von
dort nach Lynn ..... Von Lynn gingen wir nach Ecmbtidge.
Als ich in diese Stadt kam, waren die Studenten, die von
meinem Kommen gehört hatten, in Aufregung und benahmen
sich sehr ungezogen; ich hielt mich auf meinem Pferde und ritt
mitten durch sie hindurch in der Kraft des Herm; Amor Stoddart
aber warfen sie vom Pferd, ehe er die Herberge erreichte. Als
wkr in der Herberge waren, taten sie so wüst im Hof und inden
Strasen, das Fuhrleute und Kohlengräber nicht witster hätten
tun können. Die Wirtsleute fragten uns, was wir zum Nacht-
essen haben wollten; ich erwiderte: »wenn nicht Gottes Macht
gröser wäre als diese rohen Studenten, so würden sie uns sicher
gerne in Stücke reisen rmd ein Nachtessen aus uns machen.« Sie
wusten, das ich sehr gegen das Gewerbe des Predigens war, das
sie dort als Lehrjungen erlernen sollten; darum tobten sie gegen
-mich, wie nur je die Handwerksleute der Diana gegen den Paulus
(Act. 19). Ju der Nacht kam der Stadtbürgermeister, der es gut
mit mir meinte, und holte mich zu sich heim. Als wir durch
die Strase gingen, war groser Lärm in der Stadt, aber man
erkannte mich nicht, weil es finster war. Man war auch über
den Bürgermeister zornig, so das er sich sehr fürchtete, mit mir


% \picinclude{./100-109/p_s101.jpg} 
Brief an den Papst. Die Studenten von Cambridge usw. 101
über die Strase zu gehen. Nachher liesen wir dann die Freunde
holen und hatten eine schöne Versammlung in der Kraft des
Herrn, und ich blieb die ganze Nacht in der Stadt. Wir hatten
unsere Pferde für den nächsten Morgen um sechs Uhr bestellt
und ritten in Frieden zur Stadt hinaus; die Störenfriede wurden
somit enttäuscht, denn sie hatten geglaubt, ich würde länger da
bleiben, und hatten beabsichtigt, uns etwas anzutun; aber unsre
frühe Abreise oernichtete ihre bösen Anschläge .....
GH trieb mich, ein Schreiben zu senden an die, welche über
I das Zittern und Beben (quake) spotteten:
,,Gin Wort vom Herrn an euch, die ihr über das Zittern
und Beben spottet; und die ihr solche, welche zittern und beben,
uerhöhnt, schlagt, bedroht und Verwünschungen gegen sie aus-
stoszet. Jhr kennet alle die Apostel und Propheten nicht! ....
Moses, der ein Richter über Jsrael war, zitterte und bebte,
als der Herr zu ihm sagte: »ich bin der Gott Abrahams, Jsaaks
und Jakobs-« (2. Mose 3) .... . Der König David zitterte;
und sie verspotteten ihn (Ps. 38). . . . . Hiob zitterte, bebte;
und sie oerlachten ihn (Hiob 21) ..... Der Prophet Jeremia
bebte; es schüttelte ihn, seine Glieder zitierten, und er taumelte
hin und her wie ein trunkener Mann (Jer. 23, 9), als er die
Betrügerei der Priester und Propheten sah, die sich vom Herrn
abgekehrt hatten ..... Jesaia sagte: ,,Höret was der Herr
sagt, ihr, die ihr erzittert bei seinem Wort;« und weiter sagte er:
,,Jch sehe an den Elenden und der zerbrochenen Geistes ist und
der erzittert bei meinem Wort« (Jes. 66, 2) .... Habakuk, der
Prophet des Herrn zitterte ..... Und Joel, der Prophet des
Herrn sagte: ,,Blaset mit der Posaune zu Zion, erzittert alle Ein-
wohner im Lande« (Joel 2, 1) ..... Daniel, ein Diener des
Allerhöchsten, zitterte, und er hatte keine Kraft mehr (Dan. 10, 16);
und er war gefangen, gehast und verfolgt .....
Paulus, ein Apostel Jesu Christi durch den Willen Gottes, ein
auserwähltes Rüstzeug des Herrn, das er seinen Namen trage in
alle Lande, zitterte .... und sagte, als er zu den Eorinthern
kam: ,,ich war bei euch in Schwachheit und Furcht und grosem
Zit—tetn« (1. Eorinth. 2, 3) .....
Hütet euch darum, ihr Grosen der Erde, die zu Verfolgen,
welche man zum Spott Quäker (Zitterer) nennt, die aber in der
Kraft Gottes sind- damit sich die Hand des Herrn nicht gegen


% \picinclude{./100-109/p_s102.jpg} 
euch kehre und euch oerderbe. ES ergeht das Wort dez Herrn
an euch: fürchtet euch und zittert und hütet euch! denn der Herr
siehet den cm, der erzittert bei seinen Wort (Jes. 66, 2); ihr aber,
die ihr von dieser Welt seid, oerspottet, oerlacht, oerhöhnt, ver-
folgt ihn und nehmt ihn gefangen. Daran könnt ihr sehen, das
ihr den Propheten und Aposteln zuwider handelt, wenn ihr die hasfet,
die der Herr ansieht, während wir, die ihr im Spott Quäker
nennt, sie achten. Wir ehren und preisen die Macht, die den
Teufel erzittern macht, die Erde erbeben läst und den Stolz und
Hochmut niederschmettert, die Tiere auf den Feldern erzittern
macht und die Erde wanken (Jes. 2, 11). Diese Kraft ehren
und verkünden wir; aber alle, die spotten und höhnen und
peitschen und plagen, die verabscheuen wir; denn alle, die solche-?
tun und es nicht bereuen, werden das Reich Gotteö nicht ererben,
sondern das Verderben (2. Tess. 1).
Selig aber sind, die um der Gerechtigkeit willen verfolgt
werden; sie werden ihren Lohn im Himmel haben (Matth. 5, 12).« . .
G. F.
JmJahre 1655 wurde der Abschwörungzeid gefordert, wodurch
viele Freunde zu leiden hatten; und viele gingen zum Protektor, um
mit ihm darüber zu sprechen; aber er fing an, härter zu werden.
Durch die Art, in der die gehässigen Beamten den Eid alö
Schlingen gebrauchten, um die Freunde darin zu fangen, weil sie
wusten, das sie nicht schwören durften, nahmen die Leiden der
Freunde immer mehr zu, und es trieb mich, dem Protektor folgendes-
zu schreiben:
,,Die Obrigkeit soll das Schwert, das den Übeltätern ein
Schrecken sein soll, nicht umsonst tragen; wie die Obrigkeit, die
das Schwert umsonst trägt, den Übeltätern kein Schrecken ist, so
ist sie auch kein Zeichen deö Ruhmes für den, der recht tut;
Gott hat nun durch seine Macht ein Volk erwecket, welches
die Priester, die Obrigkeit und das Volk in ihrem Ärger ,,Quäker«
nennen. Dieseö schreit gegen die Trunksucht und das Schwören;
die Trunkenbolde aber, denen das Schwert der Obrigkeit ein
Schrecken sein sollte, gehen, wie wir sehen, frei umher; von denen
jedoch, die gegen dieseö Laster eisern, kommen viele ins Gefängniz,
weil sie Zeugnis ablegen gegen den Stolz, die Unreinheit, gegen
das betrügerische Handeln auf den Märkten, gegen Au?-schweifung
und Leichtfertigkeit, gegen das Spiel mit Kegeln, Würfeln und


% \picinclude{./100-109/p_s103.jpg} 
Brief an den Papst. Die Studenten von Cambridge usw. 103
Karten und andere eitle und stindliche Vergnügen .......
Das Schwert der Obrigkeit wird, wie wir sehen, vergeblich ge-
tragen, während die Ubeltäter frei sind, Böses zu tun; die aber,
welche gegen das Böse eifern, werden dafür bestraft von der
Obrigkeit, die ihr Schwert gegen den Herrn kehrt ..... Gs haben
viele grose Strafen erlitten, darum, das sie nicht schwören konnten
sondern der Lehre von Christus gehorchten, welche sagt: ,,ihr sollt
überhaupt nicht schwören«; sie sind ein Raub geworden (Jes.-12, 22),
weil sie das Gebot Christi hielten. Gs werden viele ins Gefängnis
geworfen, weil sie den Abschwörungseid nicht leisten können, obgleich
sie alles misbilligen, was man darin abschwört; und es werden
viele Diener und Boten des Herrn ins Gefängnis geworfen, weil sie
nicht schwören wollen, noch Christi Gebot tibertreten. Darum bedenke
du dich doch! ich wende mich an das, was von göttlichem Leben
in dir isti, Viele sind auch im Kerker, weil sie den Priestern die
Zehnten nicht bezahlen können; viele hat man ihrer Habe beraubtund
dreifache Mgaben von ihnen gefordert; viele werden gepeitscht und
geschlagen in den Korrektionshäusern, ohne das dadurch ein Gesetz
übertreten würde. Solche Dinge tut man in deinem Namen, damit
man bei solchem Tun geschützt sei. Wenn gottessürchtige Männer
das Schwert trügen, wenn das Unrecht bestraft würde und gottes-
fürchtige Männer angestellt würden, dann würden sie den Übel-
tätern ein Schrecken sein und ein Ruhm denen, die Recht tun, statt
ihnen Leiden zu verursachen. Dann würde Gerechtigkeit in unserm
Lande herrschen und die Rechtschafsenheit sich erheben und aus-
breiten, welche das Unrecht nicht zuläst, sondern es richtet. Jch
rede zu dem, was vom Geiste Gottes in dir ist, das du in dich
gehen und siir Gott regieren mögest, damit du dem Göttlichen,
das in eines jeden Menschen Gewissen ist, folgen mögest, denn
dieses macht, das man alle Menschen achtet in dem Herrn. Siehe
doch zu, für wen du regierest, aus das du Kraft vom Herrn
empfangen mögest, für ihn zu herrschen, und alles, was wider ihn
ist, durch sein Licht verdammt.werde.
s Von einem, der deine Seele lieb hat, und dein ewiges Bestes
wünscht.« G. F. . .




% \picinclude{./100-109/p_s104.jpg} 

%%%%%%%%%%%%%%%%%%% Kapitel 9. %%%%%%%%%%%%%%%%%%%%%%%%%%%%%%

\chapter[Angriffe der Independenten und Presbyterianer.]{Angriffe der Independenten und Presbyterianer.}

\begin{center}
\textbf{Angriffe der Independenten und Presbyterianer. Ahnungen,
Heilungen, Bekehrungen. Dispute über Taufe und Erwählung.
Gefangennahme auf Grund angeblicher Verschwörungen. Wirken
während der Gefangenschaft.}
\end{center}

\section{Zaubereivorwürffe und Besucht in der Heimat}

Nachdem ich meine Arbeit in London\ort{London} getan, ging ich nach
Bedfordshire\ort{Bedfordshire} und
Northamptonshire\ort{Northamptonshire}. In 
Wellingborough\ort{Wellingborough} hatte
ich eine grose Versammlung [...]. Die \textit{Frommen} waren in
groser Aufregung hier, den die bösen Priester der 
Presbyterianer\index{Presbyterianer} und 
Independentens\index{Independenten} hatten fälschlich ausgestreut, 
wir trügen\index{Gerüchte}\index{Zauberei}
Flaschen mit uns herum, aus denen wir den Leuten zu trinken
geben, damit sie uns nachfolgen; aber die Kraft und der Geist
Gottes liesen die Freunde über diesen falschen Gerüchten stehen [...].


Von Wellingborough ging ich nach Leicestershire\ort{Leicestershire}, 
wo Oberst Hacker\person{Hacker, Oberst} drohte, wenn ich hierher 
käme, würde er mich wieder gefangen nehmen lassen; aber als 
ich nach Whetstone\ort{Whetstone} kam, wo er
mich das letzte Mal in der Versammlung hatte festnehmen lassen,
war alles ruhig. Oberst Hackers Frau kam in die Versammlung 
und wurde bekehrt, [...] es waren auch zwei Friedensrichter 
in dieser Versammlung, aus Wales, namens Walter
Jenkin\person{Jenkin, Walter} und Peter Price\person{Price, Peter}, 
die beide später treue Diener des Herrn wurden.

Von da gingen wir nach Sileby\ort{Sileby} [...] und dann nach 
Drayton,\ort{Drayton} meiner Heimat, wo früher so viele 
Priester und \textit{Fromme} gegen
mich aufgetreten waren, jetzt aber rührte sich keiner. Ich fragte
einen meiner Verwandten,\index{Fox!Verwandte} wo alle Priester 
und \textit{Frommen} seien? Man sagte mir, der Priester von 
Nun~Caton\ort{Nun Caton} sei gestorben, und nun bewerben 
sich acht oder neun um seine Stelle.
\zitat{Sie werden dich diesmal in Ruhe lassen}, sagten sie zu mir,
\zitat{denn wie die Krähen sich um ein totes Schaf schaaren, so tun die
Priester, wenn eine Pfründe frei ist}. Das waren von ihren
eigenen Zuhörern, die so redeten! [...].\index{Gier}

\section{Kurze Begegnung mit Naylor}

Als ich nach Derbshire\ort{Derbshire} kam, kam James 
Naylor\person{Naylor, James} zu mir
und sagte mir, sieben oder acht Priester hätten ihn zu einer
Unterredung aufgefordert. Ich war nun seinetwegen sehr 
bekümmert in meinem Geist, und der Herr gebot mir ihm zu sagen,
er solle der Aufforderung folgen; denn der Herr der Allmächtige
wolle bei ihm sein und ihm durch feine Kraft den Sieg geben.
% \picinclude{./100-109/p_s105.jpg} 
Und der Herr tat es, so das die Leute merkten, das die Priester
geschlagen waren; und sie riefen: \zitat{ein Nagler 
(Naylor engl.: Nagler)
hat sie alle zu Grunde gerichtet!} Er kam nach dem Disput zu
mir voll Dank gegen Gott [...].

Nun zogen wir durch Worrestershire\ort{Worrestershire}; ich 
hatte in Birmingham\ort{Birmingham} eine Versammlung [...]. 
Dann kamen wir nach Worcester\ort{Worcester} [...]. Von da 
nach Tewkesbury\ort{Tewkesbury} [...]. Dann nach Warwirk\ort{Warwirk}, 
wo ich im Hause einer Witwe eine Versammlung hatte [...].
Nach derselben, als ich gerade fort gehen wollte, [...] kam ein
Gerichtsdiener herein und fragte: \zitat{Wen hören die Leute zu so
später Stunde?} Er verhaftete John Crook\person{Crook, John}, 
Amor Stoddart\person{Stoddart, Amor},
Gerrard Roberts\person{Roberts, Gerrard} und mich, erlaubte 
uns jedoch in unsere Herberge zu gehen; nur sollten wir am Morgen wiederkommen [...].

Aber am nächsten Morgen hieß es, wir können unsrer Wege
gehen [...]. Nun gingen wir weiter nach Coventry\ort{Coventry}, [...]
dann durch Leicestershire\ort{Leicestershire} nach 
Swannington\ort{Swannington} und Baldock\ort{Baldock}. Hier
fragte ich, ob keinerlei Art von besonderem Bekenntnis vertreten
sei? Es hieß, es gebe einige Baptisten und eine kranke Vap-
tistenfrau. John Rush ging mit mir zu ihr ..... Als wir zu
ihr kamen, waren viele fromme Leute bei ihr. Mau sagte mir,
diese Frau gehöre nicht mehr diesem Leben an; wenn ich ihr aber
etwas über das zukünftige sagen könne, so solle ich es tun. Der
Herr trieb mich, zu ihr zu reden, und sie erholte sich wieder,
zum Erstaunen der Stadt und des ganzen Landes. Diese Bap-
tistensrau und ihr Mann wurden gewonnen, und viele Hunderte
von Leuten haben sich seither in ihrem Hause versammelt .....
Wir gingen nun über Market Street .... und St. Albans
nach London .... Nachdem ich mich einige Zeit in London
aufgehalten hatte und die dortigen Freunde in ihren Versamm-
lungen besucht hatte, verlies ich die Stadt, wo ich James Raylor
zurücklies. Als ich mich von ihm trennte, fiel mein Blick aus
ihn und eine Angst besiel mich seinetwegen, aber ich ging doch
weg und ritt nach Nyegate in Surrey .....
Von da gingen wir nach Dorchester und stiegen in einer
Herberge, die einem Baptisten gehörte, ab; wir baten die in der
Stadt wohnenden Baptisten, uns ihr Versammlungshaus zu
überlassen, damit wir in demselben Versammlungen halten könnten,
aber sie oerweigerten es; wir liesen sie fragen, warum sie es
verweigerten; dadurch ward die Sache in der Stadt ruchbar.


% \picinclude{./100-109/p_s106.jpg} 
Wir liesen ihnen nun sagen, das sie und alle, die Gott fürchteten,
in unsere Herberge kommen könnten, wenn sie wollten. Sie
waren in groser Aufregung, und viele ihrer Lehrer und andere
von ihren Leuten kamen in unsere Herberge und schlugen mit den
Bibeln aus die Tische. Jch fragte sie, worüber sie denn so auf-
gebracht seien, ob sie gegen die Bibel so aufgebracht seien? Da
fingen sie an mit Auseinandersetzungen über ihre Wassertaufe.
Ich fragte sie, ob sie behaupten könnten, von Gott gesandt zu
sein, die Leute zu tausen, wie Johannes (Joh. 1,6) und ob sie
den gleichen Geist haben wie die Apostel? Sie sagten: nein.
Daraus fragte ich sie, wie vielerlei Kräfte es denn gebe! ob es
noch andere gebe als die Krast Gottes und die des Teufels?
Sie sagten, es gebe keine andere auser diesen beiden; darauf
sagte ich: ,,Wenn ihr nicht die Kraft Gottes habt, welche auch
die Apostel hatten, dann handelt ihr in der Macht des Teufels-.«
Viele der Anwesenden, die nüchterne verständige Leute waren,
sagten: ,,die Baptisten treten den Rückzug an!« Viele angesehene
Leute wurden an dem Abend gewonnen, und wir hatten einen köst-
lichen Gottesdienst und des Herrn Kraft war über allen. Am
folgenden Morgen, als wir fortgingen, schüttelten die Baptisten in
ihrer Wut hinter uns her den Staub von ihren Füsen. ,,So«
sagte ich, ,,ihr tut solches in der Macht der Finsternis? dann tun
wir es auch gegen euch, aber in der Kraft Gottes-«.
Wir verliesen Dorchester und gingen nach Weymouth ....
Gs war ein Kaoalleriehauptmann in der Stadt, der mich zu sich
kommen lies und mich gerne länger gehalten hätte; aber ich
durfte nicht länger bleiben. Gr und ein Diener ritten etwa
sieben Meilen mit mir; Gdward Phot war auch dabei. Dieser
Hauptmann war der behäbigste, sröhlichste, leutseligste und lach-
lustigste Mensch, der mir je begegnete, sodas es mich einige Male
trieb, ihm in der gewaltigen Kraft des Herrn zuzusprechen, aber
es war ihm so zur Gewohnheit geworden, das er immer wieder
über alles, was er sah, lachte. Aber ich ermahnte ihn immer
wieder, ernsthaft zu werden und gottessürchtig. Wir brachten die
Nacht in einem Wirtshause zu; am Morgen trieb es mich, noch
einmal mit ihm zu reden, ehe wir uns trennten. Als ich ihn
das nächste Mal sah, teilte er mir mit, das die Kraft des Herrn
ihn so übernommen habe, während ich damals mit ihm redete
beim Mschied, das er ganz ernsthaft geworden sei, ehe er heim


% \picinclude{./100-109/p_s107.jpg} 
Angriffe der Jndependenten und Presbyterianer usw. 107
kam und sein Lachen gelassen habe; er bekehrte sich später und
wurde ernsthaft und gut und starb in der Wahrheit .....
Wir kamen nach Kingszbridge, wo wir in unsrer Herberge
nach den Grnstgesinnten in der Stadt fragten. Sie schickten uns
zu Nieolaz Tripe und seiner Frau und wir gingen dorthin. Sie
liesen den Priester holen, mit dem wir uns längere Zeit unterredeten,
aber da er unterlag, verlies er unö bald. Nieolaö Tripe und
seine Frau wurden gewonnen; und seitdem kommen in jener Gegend
häufig Freunde zusammen. Al?7 wir am Abend in unsere
Herberge kamen und viele dort antrafen, welche tranken, trieb
mich der Herr zu ihnen zu gehen und sie auf das- Licht hinzu-
weisen, welches Christus ihnen allen angezündet habe, durch das-
sie ihr böses Tun erkennen könnten, ihre bösen Reden und auch
Jesus Christus ihren Heiland. Dem Wirt wurde es unbehaglich,
weil er sah, das ich seine Leute vom Trinken abhielt, und sowie
ich die letzten Worte geredt, nahm er ein Licht und sagte: ,,Kommt,
hier ist ein Licht, mit dem ihr in euer Zimmer gehen könnt«. Am
nächsten Morgen, als er abgekühlt war, stellte ich ihm vor, wie
unziemlich er sich benommen hatte und ermahnte ihn beim Abschied,
an den Tag de-:2 Herrn zu denken ..... Wir zogen durch
Penryn nach Helston .... und von da nach Market-Jew, wo
wir in eine Herberge gingen ..... Am nächsten Morgen ver-
sammelten sich die Behörden und schickten ihre Konstabler, um
uns vor sie zu holen. Wir fragten sie nach dem Verhaftbesehl;
sie sagten, sie hätten keinen; .... ez kamen auch mehrere
andere höhere Beamte, und wir stellten ihnen Vor, was das für
ein schmählicheö Betragen sei, Reisende in ihrer Herberge zu be-
helligen ..... Ghe wir die Stadt verliesen, oersaste ich noch
ein Schreiben an die sieben Gemeinden in Lands-End. GZ hies
darin zum Schlus: Nützet eure Zeit, dieweil sie euch gegeben
ist; denn jetzt ist ,,eure angenehme Zeit, jetzt ist euer Tag des
HeilZ« (2. Cor. 6,2). In einem jeden von Euch ist ein Licht?
von Ehristus, das euch zeigt, das ihr nicht lügen, nicht unrecht
tun, nicht schwören, nicht fluchen, nicht stehlen, noch Gottes Namen
misbrauchen sollt. Wenn ihr dieseö Licht lieb habt und ihmysl
folgt, so wird es euch zu Christus führen, welcher der Weg zum
Vater ist, dem Vater dez Licht?-, bei welchem nichts Ungöttliches
ist. Wenn ihr dieseö Licht hasset, so wird e3 euch zum Ver-
derben werden; wenn ihr ez aber liebt, so bringt es euch ab von


% \picinclude{./100-109/p_s108.jpg} 
sden Lehrern der Welt, damit ihr von Christus lernt, und be-
iwahrt euch vor dem Unrecht der Welt und allen ihren Ver-
führern.« G. F.
Dieses Schreiben trug ein Freund, der mich begleitete, bei
sich; als wir nun etwa drei Meilen von Market-Jew gegen
Westen weiter gegangen waren, begegnete er einen Manu, dem
er eine Abschrift davon gab. Gs stellte sich heraus, das dieser
Mann ein Diener vom Gefolge des Peter Eeelh war, des
Obersten der Armee und Friedensrichters jener Gegend. Der Mann
ritt nun voraus und zeigte das Schreiben dem Major Eeely. Als
wirnach St. Jves kamen, verlor Edward Pyots Pferd ein Hufeisen
und wir hielten an, um es wieder beschlagen zu lassen. Während-
dessen ging ich zum Meeresstrand hinunter. Als ich zuriickkam,
fand ich die Stadt in Mfruhr, und sie schleppten eben Edward
Pyot und einen andern vor Major Eeely. Jch solgte ihnen ins
Richthaus, obgleich mich niemand dazu zwang. .... Man
legte uns den Abschwörungseid vor, worauf ich meine Hand in
die Tasche steckte ..... Major Eeely hatte einen albernen
Priester bei sich, der uns viele nichtssagende Fragen stellte; unter
anderm verlangte er, ich solle mein Haar, das damals ziemlich
lang war, schneiden lassen, aber ich mochte es nicht schneiden
lassen, obgleich viele sich oft daran stiesen; ich sagte ihnen, ich
sei ja nicht stolz darauf, und ich lasse es ja nicht selber wachsen.
Zuletzt übergab man uns einer Wache, die so grob gegen uns
war, wie der Richter selber; dessen ungeachtet oerkündeten wir
die Wahrheit unter den Leuten. Am solgenden Morgen sandte
man uns unter Bewachung mehrerer Berittener, die mit Schwertern
und Pistolen bewaffnet waren, nach Redruth ..... und von da
wurden wir nach Launeeston gebracht .....
Gs waren noch 9 Wochen, bis wir vor Gericht erscheinen
musten, wozu dann viel Volk herbeiströmte, um das Verhör der
Quäker zu hören. Hauptmann Bradden war in Launeeston mit
seiner Reiterei, und seine Leute geleiteten uns durch die Volks-
menge, welche die Strasen füllte, und es war kein Geringes, uns
hindurchzubringen; auch an allen Fenstern und Türen standen
Leute, die uns sehen wollten. Jm Gerichtshof angekommen,
warteten wir eine Weile, den Hut auf dem Kopfe; niemand be-
kümmerte sich um uns, zuletzt trieb es mich zu sagen: ,,Friede
sei mit E—uch.« Da sragte Richter Glynne, damals Ober-


% \picinclude{./100-109/p_s109.jpg} 
Angriffe der Jndependeuten und Presbyterianer usw. 109
Richter von England, den Wörter: »was sind das sür Leute, die
ihr in den Gerichtshof gebracht habt?« »Gefangene, Herr,« ant-
wortete dieser. ,,Warum nehmt ihr eure Hüte nicht ab?« fragte
uns der Richter; wir antworteten nichts. ,,Nehmt eure Hüte
ab!« wiederholte der Richter, wir sagten wieder nichts; der
Richter sagte: ,,der Rat befiehlt euch, die Hüte abzunehmen.«
Nun redete ich und sagte: ,,Wann hat je ein Richter, König oder
sonst eine obrigkeitliche Person von Moses bis Daniel, bei den
Juden, dem Volke Gottes, oder bei den Heiden, je befohlen, das
man den Hut abnehme, wenn man vor Gericht erscheint? und
wenn das Gesetz von England irgend etwas derartiges besiehlt,
so zeiget uns dieses Gesetz irgendwo geschrieben oder gedruckt.«
Da wurdesder Richter sehr zornig und sagte: ,,ich trage mein
Gesetzbuch nicht aus dem Rücken!« »So nenne mir irgend ein
Buch, welches Statuten darüber enthält, das ich es lesen kann,«
sagte ich. Da gebot der Richter: ,,führt den Kerl weg! ich will
ihn züchtigen«, und sie führten uns fort, zu den Dieben hinunter.
Doch gleich darauf rief er den Gesangenwärter wieder, und
gebot ihm, uns wieder zu bringen. Dann sagte er: ,,hatten sie
denn etwa Hüte zur Zeit des Moses und Daniels? antwortet
mir! nicht wahr, nun habe ich euch erwischt!« Jch erwiderte:
»Du kannst Daniel 3 lesen, das die drei Männer auf Besehl des
Nebukadnezar »in Rock, Hosen und Hut« in den Feuerofen ge-
worsen wurden«. Dieses einfache Beispiel machte ihn verstummen,
sodas er, weil er nichts mehr zu sagen wuste, ries: »flihret sie
wieder sort!« so wurden wir denn wieder zu den Dieben hin-
untergebracht. Ain Nachmittag wurden wir wieder vor Gericht
gebracht ..... Als wir dort warteten, bis wir an die Reihe
kamen, und ich die Menge derer, die hier schwörten, sah, betrübte
es mich, das so viele, die sich für Christen ausgaben, so offen
dem Gebot Christi ungehorsam waren, und der Herr trieb mich,
ein Blatt auszuteilen gegen das Schwören, welches ich bei mir
trug .....
Dieses Blatt machte die Runde bei den Gerichtspersonen, und
sie gaben es zuletzt dem Richter, und als wir nun vor ihn ge-
rufen wurden, fragte er mich, ob dieses verführerische Blatt mein
sei? Jch antwortete: wenn sie es vor dem ganzen Hofe vorlesen
wollen, so höre ich, ob es mein sei, und dann wolle ich auch da-
zu stehen. Er wollte, das ich es nehme und für mich durchlese.

% \picinclude{./110-119/p_s110.jpg} 
Jch wiederholte, man solle etz vorlesen, damit alle urteilen könnten,
ob etwas Verführerisches darin sei, in dem Falle wolle ich dafür
leiden. Schlieslich laö ez der Angestellte mit lauter Stimme, das
alle es hören konnten; alö er fertig war, sagte ich: »ja, es ist
mein Blatt, ich stehe dazu, und ihr müst auch dazu stehen, wenn
ihr nicht die Schrift verleugnen wollt, denn ist es denn nicht, mas
die Schrift sagt, und Christuö und die Apostel, denen alle wahren
Christen gehorchen müsen?« Nun liesen sie den Gegenstand
fallen, und der Richter kam wieder auf unsere Hüte zurück und
hies den Kerkermeister sie une abnehmen, dieser tat es; aber wir
setzten sie wieder auf ..... Der Richter hielt nun eine lange
Rede über den Lord Protektor, wie er ihn zum obersten Richter
in England gesetzt, und ihn hierhergeschickt, und dergleichen mehr.
Wir baten ihn, er solle uns Gerechtigkeit erzeigen nach unserer
ungerechten Gefangenschaft diese nenn Wochen; statt dessen aber
brachten sie eine Anklage vor, die sie gegen unz zusammengesetzt
hatten, so voll Lügen, das ich meinte, sie richte sich gegen einen
Dieb: wir seien nur mit Wasfengewalt und nach grosem Wider-
stand hierher gebracht worden! und doch waren wir, wie oben
gemeldet, gekommen. Ich sagte ihnen, das sei falsch und wir
wiederholten unser Gesuch um Gerechtigkeit; die Gefangennahme,
sagte ich, sei ungerecht, denn ich sei auf der Reise von Major
Ceely festgenommen worden. Ntm redete Peter Eeely mit dem
Richter und sagte, auf mich zeigend: »Grlaubt mein Herr, dieser
Mann nahm mich bei Seite und sagte, er könne in einer Stunde
vierzigtaufend Mann stellen und da-? Land in Blut stürzen und
König Karl zurückbringen, und ich könne ihm dabei behilflich sein.
Jch wollte ihm auö dem Lande helfen, aber er wollte nicht
gehen; ich habe Zeugen, die?7 zu beschwören«, und er rief den
Zeugen auf. Aber der Richter war nicht gewillt, ihn anzuhören,
und so bat ich, man möchte meine Anklage, auf Grund deren
ich verhaftet sei, vorlesen; der Richter sagte: ,,nein, sie soll nicht
vorgelesen werden«, .... als ich sah, das man sie nicht lesen
wollte, sagte ich zu einem meiner Mitgefangenen, »du hast eine
Abschrift davon, lies die vor.« ,,Kerkermeister«, sagte hieraus der
Richter, »führ ihn sort! wir wollen doch sehen, wer hier Meister
ist, er oder ich!« und so wurde ich hinweg geführt. A16 ich
wieder gerufen wurde, bestand ich wiederum darauf, das mein
Verhastbesehl vorgelesen werde, denn davon hing meine Gefangen-


% \picinclude{./110-119/p_s111.jpg} 
Angriffe der Jndependenien und Ptesbyterianer usw. 111
schaft ab. Jch hies abermalö meinen—Mitgesangenen ihn lesen,
und er tat ez:
,,Peter Ceely, einer der Friedenörichter der Grafschaft, an
den Kerkermeister von Seiner Hoheit Gefängniö zu Launeeston:
»Jch sende Euch hiermit durch den Überbringer dieser- die
Personen Edward Pyot von Bristol und George Fox auz Dray-
ton-in-the-Elay in Leicestershire, und William Salt von London, . .
die alt; Quäker bekannt sind und sich selber als.3 solche bekennen;
sie haben Verschiedene Blätter verbreitet, die den öffentlichen
Frieden gefährden, und können keinen gesetzlichen Grund für ihr
Erscheinen in dieser Gegend angeben, sie sind gänzlich unbekannt
in dieser Gegend, haben keinen Pas, weigern sich, irgendwelche
Beweise ihreö guten Wandels zu geben, die da; Gesetz verlangt,
und weigern den Abschwörungtzeid zu leisten. Wir befehlen euch
darum im Namen seiner Hoheit dez? Lord Protektor, diese Per-
sonen, . . . wenn sie kommen, in Gewahrsam zu bringen und da-
rin zu lassen, bis sie gesetzlich srei gelassen werden. Versäumet
nicht, solcheö zu tun, wo andere'- eö euch gefährlich werden könnte.
Auögegeben mit meiner Unterschrift und Siegel, St. Joes, den
18. Januar 1655. P. Eeely.«
Als dietz vorgelesen worden war, sagte ich zu den Richtern, . . .
mich an Major Ceely wendend: »Wo und wann habe ich dich
beiseite genommen? .... und wenn du mein Ankläger bist, wa-
rum sitzest du auf der Richterbank? du solltest herunter kommen
und mir ins Gesicht sehn. Übrigens möchte ich fragen, ob nicht
Major Ceely sich der- Verrat-3 schuldig machte, dessen er mich an-
klagt, durch sein langeö Schweigen? Kennt er seinen Platz als
Soldat wie alö Friedentzrichter? denn .... wenn ich ihn beiseite
genommen, um ihm zu sagen, ich könne oierzigtaus end Mann stellen,
und so weiter, .... so sehet ihr deutlich, das er ja in dieser
Verschwörung beteiligt gewesen wäre, indem er mich .... aus
der Gegend forthaben wollte .... und den Verrat nicht früher
entdeckte. Aber ich leugne seine Autzsagen und bin unschuldig an
diesem teuflischen Plan.« Die Richter liesen nun die Sache fallen,
denn sie sahen das, anstatt das sie mich in eine Falle gelockt hatten,
ich selber ihnen eine gestellt hatte. Major Eeely behauptete nun,
ich habe ihm ins Gesicht geschlagen ..... Jch fragte ihn, ob er
sich alö Richter und Soldat nicht schäme, solcheö zu sagen ....
Schlieslich, als die Richter sahen, das diese Fallen nichts nutzten,


% \picinclude{./110-119/p_s112.jpg} 
liesen sie uns wieder ins Gefängnis führen und forderten von
jedem zwanzig Goldstücke, weil wir den Hut ausbehalten .....
Als das Urteil so lautete, das keine baldige Freilassung
zu erwarten war, hörten wir aus, dem Wörter wöchentlich
7 Schilling für unsere Pferde und 7 für uns selber zu geben;
daraufhin wurde er böse und ganz teuflisch und brachte uns nach
Doomsdale hinunter, einen greulichen, stinkenden Ort wohin die
Mörder nach der Verurteilung gebracht wurden. Der Ort war
sehr ungesund, so das wenige, die sich hier aushalten musten,
wieder gesund heraus kamen; es war kein Abtritt da, und der
Unrat der Gefangenen war seit Jahren nie hinausgeschasst worden.
Es war ein förmlicher Sumpf darin, stellenweise bis über die
Schuhe, von dem Unrat; und man erlaubte uns nicht, rein zu
machen oder uns Betten oder Stroh zum drauf liegen zu ver-
schaffen. Am Abend brachten uns einige Bekannte aus der Stadt
ein Licht und etwas Stroh, um draus zu liegen; wovon wir einiges
verbrannten, um den Gestank zu vertreiben. Die Diebe schliefen
gerade über uns und der Wörter in einem Zimmer daneben.
Scheints drang der Rauch ins Zimmer des Wärters; er geriet
in einen solchen Zorn, das er die Nachtgeschirre der Diebe nahm,
und sie durch ein Loch gerade auf unsere Köpfe ausleerte; wir
waren so beschmiert davon, das wir weder einander noch uns selber
anrühren konnten; und der Gestank war so arg, das wir fast
darin erstickten. Vorher hatten wir den Gestank zu unsern Füsen
gehabt, jetzt hatten wir ihn auch aus den Köpfen und am Rücken;
und da unser Stroh von dem heruntergeworsenen Dreck beschmutzt
war, so verbreitete es einen greulichen Dunst. Zudem fluchte der
Wörter gräszlich über uns und nannte uns »hackengesichtige Hunde«;
und andere merkwürdige Namen, die wir noch nie gehört. Jn
diesem Zustand gingen wir fast zu Grunde während der Nacht,
denn wir konnten nicht einmal absitzen, alles war so voll Unrat.
Wir musten lange in dem Zustand ausharren, bis uns gestattet
wurde, reinzumachen und uns andere Lebensmittel zu verschaffen als
das, was durchs Gitter kam. Einmal brachte einMädchen uns etwas
Essen; der Wärter arretierte es und führte es vor Gericht, weil
es ins Gefängnis eingedrungen sei, und es geriet in grose Not;
dadurch wurden viele andere entmutigt, so das es uns schwer
wurde, uns Wasser oder Lebensmittel zu verschaffen. Wir liesen
nun eine junge Frau aus Londen kommen, Anna Downer, damit


% \picinclude{./110-119/p_s113.jpg} 
Angriffe der Jndependenten und Presbhterianer usw. 113
sie uns das Essen kaufe und zubereite; sie war dazu bereit, denn
es war über sie gekommen, zu uns zu kommen in der Liebe
Gottes, und sie war sehr dienstfertig gegen uns .....
Die Gefangenen und einige andere verschrobene Leute be-
richteten von Gespenstern, die in Doomsdale umgingen, und von
den vielen, die hier gestorben seien, um einen damit angst zu
machen. Aber ich sagte, das, wenn auch alle Geister und Teufel
der Hölle dort seien, ich darüber stehe, durch die Kraft Gottes,
und nichts dergleichen fürchte; denn Christus unser Priester werde
uns das Haus und die Mauern heiligen, er, der dem Teufel den
Kopf zerbrochen habe .....
Es war gtun bald die Zeit der allgemeinen vierteljährlichen
Getichtssiizuttg, und da der Kerkermeister sich immer noch schlecht
gegen uns benahm, setzten wir einen Bericht über unsere Leiden
auf und schickten ihn zur Gerichtsfitzung nach Bodmin. Als die
Richter ihn gelesen, gaben sie den Befehl, das die Türen von
Doomsdale geöffnet werden sollten und man uns erlaube, rein
zu machen und unsere Nahrung in der Stadt zu kaufen. Wir
sandten eine Abschrift unseres Leidensberichts an den Protektor
und erzählten ihm, wie wir von Major Ceely verhaftet und ver-
urteilt worden waren, und wie uns der Kerkermeister mishandelt
hatte. Der Protektor schickte einen Befehl an Hauptmann Fox,
den Befehlshaber von Schlos Pendennis, das er untersuche, wie
es sich mit den Soldaten, die uns mishandelten, verhalte .....
Solches war der Sache des Herrn sehr förderlich; denn
nachher konnten die Freunde in jedem Turmhaus oder Markt-
platz reden und es tat ihnen niemand etwas. Jch hörte, das
Hugh Peters, einer der Kapläne des Protektor, diesem gesagt habe,
man könne dem George Fox keinen grösern Dienst zur Ausbrei-
’ tung seiner Ansichten in Cornwall tun, als ihn in Cornwall ein-
zusperren. Und wirklich kam meine Gesangennahme in Cornwall
vom Herrn zur Förderung seiner Sache in dieser Gegend; denn
als es nach der Gerichtssitzung hies, wir würden gefangen bleiben,
kamen Freunde aus allen Teilen des Landes, um uns zu besuchen.
Diese westlichen Gegenden waren damals sehr in Finsternis, aber
das Licht und die Kraft und die Wahrheit des Herrn brachen
nun hervor und leuchteten über allen, und viele bekehrten sich
von der Finsternis zum Licht und von der Macht des Satans
zu Gott. Ge trieb viele, in die Turmhäuser zu gehen, und viele
George Fox. 8


% \picinclude{./110-119/p_s114.jpg} 
besuchten unö, denn wir durften nun umher gehen im Schloshof,
und an den Ersten Tagen kamen oiele zu uns, denen wir das
Wort des Lebenö brachten .....
Ju Cornwall, Deoonfhire, Dorsetshire und Somersetshire
fing die Wahrheit an mächtig zu spriesen, und viele bekehrten sich
zuzEhristuS; viele Freunde fühlten sich getrieben, die Wahrheit
in diesen Gegenden zu verkünden, waS die Priester und die
,,Frommen« sehr aufbrachte, so das sie die Behörden anstifteten,
den Freunden Fallen zu stellen. Sie stellten Wachen auf den
Landstraseu, unter dem Vorwand, alle verdächtigen Personen
abzufassen, sie ergriffen nun daraufhin Freunde, die vorbeikamen,
um uns im Gefängnis zu besuchen ..... Aber gerade das,
waö sie taten, um der Wahrheit Einhalt zu tun, diente dazu, sie
auözubreiten; denn dadurch wurden die Freunde oft getrieben, zu
den Konstablern oder den Behörden, vor die sie gebracht wurden,
zu reden, was viel dazu beitrug, das die Wahrheit sich in allen
Distrikten aus-breitete. Oft wenn Freunde in die Hände der
Wachen gerieten, ging es zwei oder drei Wochen, ehe sie wieder
frei wurden.
Als Thomas Rawlinson au-J dem Norden her kam, um unz
zu besuchen, ergriff ihn ein Konstabler in Devonshire und nahm
ihm nachts zwanzig Schilling aus der Tasche, und darauf wurde
er zu Exeter ins Gefängnis geworfen. Henry Pollexfen warfen
sie auch ins Gefängniz, weil er ein Jesuit sei ..... Viele
Freunde wurden von ihnen mishandelt; ja Leute, die an ihrer
Arbeit waren, wurden von ihnen gepeitscht und ergriffen, und
ez waren doch solche darunter, die eine Einnahme von mehr
als achtzig und hundert Pfund im Jahr hatten; und zwar
geschah ihnen solcheö, wenn sie kaum vier oder fünf Meilen von
zu Hause weg waren. Unter dem Eindruck all des Bösen, das «
mit dem Aufstellen der Wachen und dem Gefangennehmen der
Freunde beabsichtigt war, kam ez über mich folgendes zu schreiben:
,,Eine Mahnung und Warnung an die Behörden.
»Jhr Mächte der Erde, Christus ist gekommen um zu regieren,
und er ist unter euch, und ihr kennetihn nicht; er erleuchtet einen
jeden unter euch, damit ihr alle an ihn glauben möchtet, an
das Licht; an den »der die Kelter allein tritt«. Darum prüfet
alle in diesem Lichte, ob ihr reif seid, denn die Kelter ist bereit.
(Offb. 14, 19) ....


% \picinclude{./110-119/p_s115.jpg} 
Angriffe der Jndependenten und Presbhterianet usw. 115
»Jht verkündet Gewissensfreiheit; und doch darf man seinen
Freunden keine Briefe bringen, oder seine Freunde oder die
Gefangenen besuchen, oder ihnen Bücher bringen, ohne das ihr
Wachen ausstellt, um sie anzuhalten und zu greifen; und sogar
bewaffnet müssen diese sein gegen die guten Leute, die kaum
einen Stock mit sich tragen, und die ihr aus Groll Quäker nennt.
Und die, welche diese Wachen ausstellen, die verkünden Gewissens-
freiheit und nehmen solche gefangen, die ihr Gewissen gegen Gott
und gegen die Menschen rein erhalten wollen, die Gott im Geist
und in der Wahrheit anbeten, was die, welche nicht im Licht sind,
Ketzerei nennen! . . . Jst je solch ein Geschlecht gewesen, das
so wahnsinnig schlecht und verfolgungssüchtig war und Bewasfnete
ausstellte gegen die Wahrheit und sie verfolgte, wie Grafschasten
und Städte es jetzt tun? das klingt wie Sodom und Gomorrah!«
G. F.
Gs kam mir eine Abschrift eines von der Sitzung von E-xeter
ausgehenden Verhastbesehls in die Hände, der in starken Aus-
drücken verlangte, ,,alle Quäker zu verhasten«, und der die Wahr-
heit und die Freunde schlecht machte; da trieb es mich, eine
Antwort zu schreiben und zu verbreiten, um die Freunde und
die Wahrheit gegen solche Verleumdungen zu verteidigen, und
die Schlechtigkeit und Bosheit des Verleumdungsgeistes zu zeigen. . .
Wir blieben im Gefängnis bis zur nächsten Sitzung; viele
Freunde, Männer und Frauen, die von der Wache ergriffen
worden waren, waren ins Gefängnis gebracht worden. Viele
von ihnen wurden nach Eröffnung der Sitzung vor die Richter
gebracht und beschuldigt, sie hätten sich gesträubt zu kommen
und waren doch von den Gefängniswärtern gebracht worden.
Der Richter legte ihnen Busen auf, weil sie den Hut nicht ab-
nehmen. Wir hingegen musten nicht mehr oor den Richter.
Während dieser ganzen Zeit und während der Sitzungen war
Unser Wirken für den Herrn reich gesegnet, denn es kamen viele
zu uns, »Fromrne« und andere, mit uns zu reden. Elisabeth
Trelawm) von Plymouth, die Tochter eines Barons, wurde bekehrt,
worüber Priester und ,,Fromme« und viele angesehene Personen
auser sich waren und ihr Briefe deswegen schrieben. Da sie eine
weise und gottselige Frau war und denen, die ihr geschrieben,
nicht wollte etwas in die Hand geben, das sie dann hätten können
gegen sie gebrauchen, so schickte sie mir die Briefe; und ich schrieb
 


% \picinclude{./110-119/p_s116.jpg} 
ihr darüber, rmd sie beantwortete sie dann. Sie nahm zu in der
Kraft und der Weiöheit Gotrtesz, so das sie zuletzt imstande war,
den msichtigsten Priestern und ,,Frommen« zu antworten; sie hatte
die Herrschaft über sie in der Wahrheit durch die Kraft Gotteö,
der sie treu blieb biz in den Tod.
Während ich hier in der Gefangenschaft war, prophezeiten
die Baptisten und Fifthmonarchyleute, in diesem Jahre werde
Christus- kommen und tausend Jahre auf Grden regieren. Sie
erwarteten dieseö Reich als ein äusereö, während er doch in die
Herzen der Menschen gekommen war, um darinnen zu regieren;
aber so wollten ihn diese ,,Frommen« nicht aufnehmen, darum
mislang ihnen ihr Prophezeien. A, Ehristuö ist ja schon ge-
kommen tmd wohnet in den Herzen der Menschen und regieret
darin. Tausende, bei denen er anklopfte, haben ihm aufgetan;
und er ist bei ihnen eingekehrt und hat das Abendmahl mit ihnen
gehalten (Offb. 3, 20). Z Viele dieser Baptisten und Fisthmonarchh-
leute sind die ärgsten Feinde derer, die sich zu Ehristus hielten,
geworden; aber er regieret in den Herzen seiner Heiligen.
Während der Gerichtssitzung kamen mehrere der Richter zu
unö und waren ziemlich höflich und redeten vernünftig über
göttliche Dinge mit uns und bezeugten uns Teilnahme. Haupt-
mann Fox, der Gouverneur von Schlos Pendenniö, trat zu mir
mid sah mir ins Gesicht, sagte aber nichts; aber als er wieder
zu seinen Begleitern zurück kam, sagte er, er habe noch nie in
seinem Leben einen einfältigeren Menschen gesehen. Jch rief ihm
nach: ,,Wir wollen sehen, wer der Ginfältigere ist!« Aber er
ging seines Weges, der hochmütige Tropf.
Thomas Lower 1) besuchte uns ebenfalls .... Er stellte uns
viele Fragen darüber, das wir behaupteten, die Schrift sei nicht
das- Wort Gotteö, und über die Sakramente und andere?-, und
wir konnten über alles Ausschlus geben. Jch redete auch noch
allein mit ihm, und er bekannte nachher, meine Worte hätten
ihn wie ein Blitzstrahl durchzuckt. Er habe noch nie Leute, wie
wir seien, getroffen, die seine innersten Gedanken errieten. Gr
wurde nachher bekehrt und ist ein Freund geblieben bis auf diesen
Tag .... und hat viel um der Wahrheit willen gelitten. R
EZ trieb mich zu dieser Zeit, die folgende Mahnung an die
Freunde, die Prediger waren, zu richten:
1) Thomas Lower, später Schwiegersohn von G. F.


% \picinclude{./110-119/p_s117.jpg} 
Angriffe der Jndependenten und Presbyteriuner usw. 117
,,Freunde!
Bleibet in der Kraft des Lebenö und der Weisheit und in
der Furcht des Herrn Himmels und der Erde, damit ihr in der
Weiöheit Gottes bewahret bleiben möget und seinen Gegnern ein
Schrecken werdet, indem ihr die Wahrheit verbreitet, Zeugen für
sie erwecket, die Betrügerei stürzet, von der Übertretung zum
Leben bringt, in den Bund des Lichts und den Frieden Gotteö.
Lasset alle Welt diese Stimme hören, durch Wort oder Schrift.
Schonet keinen Ort, noch Sprache, noch Feder; tuet das Werk
in Gehorsam gegen Gott; kämpfet tapfer für die Wahrheit auf
Erden und zertretet alles, was ihr entgegen ist. Jhr habt die
Kraft, misbraucht sie nicht ..... Regieret mit Ehristus, dessen
Thron und Szepter nun ausgerichtet sind, und der herrscht bis an
die Enden der Erden ..... GH soll nun das Heil ausgehen
von Zion, zu richten den Berg Esaus, und das Gesetz soll von Jeru-
salem au?-gehen (Obad), damit es redezu dem Göttlichen, dasin einem
Jeden ist, und alle Erfindungen und Erfinder überwältige. Alle
Fürsten der Welt sind Luft vor der Macht Gottes, die ihr ge-
schmecket habt; darum lebet in ihr .....
Ftthret alle zur Anbetung Gotteö; pflüget den brachliegenden
Acker, dreschet das Korn, damit der Same, der Weizen, in die
Scheunen gesammelt werden könne und Alle zum Ursprung,
. zu Ehristus, kommens der war, ehe der Welt Grund gelegt ward.
Die Spreu ist durch die Übertretung unter den Weizen gekommen;
der, welcher ihn auödrescht, hat die llbertretung verlassen und
erkennt sie, und unterscheidet zwischen dem Wertoollen und dem
Unwerten, er kann den Weizen vom Unkraut unterscheiden und
ihn in den Speicher sammeln und bringet so die unsterbliche
Seele zu Gott, von dem sie kamsf. . . Die Prediger des Geistes
müssen dem gefangenen Geist predigen, damit durch den Geist
Christi die Menschen zu Gott, dem Vater alleö Geistes, geführt
werden, ihm zu dienen und eins zu sein mit ihm, mit der Schrift
und untereinander. Dies ist das Wort des Herrn an euch alle.
. .   Seid ein Vorbild und Beispiel in allen Ländern, Ort-
schaften, Jnseln und Völkern, zu denen ihr kommt, damit euer
Wandel allen Menschen predige .... Darin werdet ihr dem
Herrn angenehm sein und ein Segen werden.
Schonet den Betrug nicht; greiset ihn mit dem Schwert an;
bekämpfet ihn; trachtet nicht nach Blut, weder in Wort noch


% \picinclude{./110-119/p_s118.jpg}
Schrist ..... Verktindet allen den lebendigen Gott; denn alle
Lehre, Kirche und Gottesdienst, die durch menschlichen Willen und
Verstand eingesetzt sind, werden von der Kraft Gottes vernichtet.
.... Verkiindet den grosen Tag des Feuers und des Schwertes,
den Tag deö Herrn, der im Geist und in der Wahrheit will
angebetet sein, und bleibt in der Kraft Gottes, damit die Bewohner
der Erde vor euch erzittern; und damit die Kraft und Herrlich-
keit deö Herm unter den Heiden Hund den Heuchlern gepriesen
werde, und ihr in der Weisheit und Furcht, im Leben, im Schrecken
und in der Herrlichkeit bewahrt bleibt zu seiner Ehre. EZ gehet
ein Ruf, das man die Übertretung verlasse, und der Geist ruft:
,,kommet«. EZ ergehet jetzt ein Ruf, die falschen Gotteödienste
zu verlassen und dem wahren Gott zu dienen; ein Ruf zur
Buse .... damit die Gerechtigkeit hervorbreche; und sie wird
über die ganze Erde sich ausbreiten. Darum tut treu in der
Kraft dee- Herrn euer Werk, ihr, die ihr au?-erwählt seid .....
Gehorchet der Kraft, sie wird euch erretten aus der Hand der
Unoernünftigen und von der Welt. Durch sie werdet ihr das
Reich haben, dasz kein Ende hat und in welchem Herrlichkeit und
Leben isi.« .... G. F.
Nach der Sitzung hatten wir manche Unterredungen mit dem
Scheriff und einigen Soldaten, die eine zum Tode verurteilte Frau
bis zur Hintichnmg überwachen musten. Einer von ihnen sagte:
,,Ehristuö war einer der heftigsten Menschen, die je gelebt«; wir
verwiesen ihm dies. Ein andermal fragten wir den Kerkermeister,
was- bei den Gericht?-Verhandlungen vorkomme. Er antwortete:
,,O, nur Kleinigkeiten; nur etwa dreisig, die wegen Bastardschaft
verurteilt sind.« Wir wunderten untz sehr, das solche, die doch
Christen zu sein meinten, derartigeö eine Kleinigkeit fanden.
Aber dieser Kerkermeister war selber ein sehr schlechter Mensch.
Jch ermahnte ihn oft zur Rechtschaffenheit, aber er behandelte
die Leute, die unz besuchen wollten, schlecht. Edward Pyot bekam
einen Käse von seiner Frau geschickt; der Kerkermeifter nahm ihn
ihm weg und brachte ihn dem Major, angeblich um ihn auf
verräterische Briefe hin zu durchsuchen; aber obwohl sie nichts
von Vriefett fanden, so behielten sie ihn doch. EZ hätte diesem
Kerkermeister ganz gut gehen können, wenn er sich anständig
betragen hätte, aber er suchte selber sein Verderben, welcheö auch
bald über ihn kam; denn im darauf folgenden Jahre wurde er


% \picinclude{./110-119/p_s119.jpg} 
Angriffe der Jndependenten und Presbyterianer usw. 119
von seiner Stelle abgesetzt und kam selber ins Gefängnis und
bettelte dort bei den Freunden. Und wegen irgend eines Ver-
gehens brachte ihn sein Kerkermeister nach Doomsdale und legte
ihn in Ketten und schlug ihn, und erinnerte ihn daran, wie er
jene guten Leute mishandelt habe, die er ohne jeden Grund in
diesen greulichen Kerker getan, und das er nun die verdiente
Strafe für seine Bosheit leiden müsse, und ihm nun mit dem
Mase gemessen werde, mit dem er gemessen habe. Es ging ihm
sehr schlecht und er starb in der Gefangenschaft, und sein Weib
und seine Kinder kamen ins Elend.
Während ich zu Launceston gefangen war, ging ein Freund
zu Oliver Cromwell und erbot sich, an meiner Statt in Dooms-
dale gefangen zu sein, wenn er es annehmen und mich dafür
in Freiheit setzen wolle. Dies erstaunte Eromwell dermasen,
das er zu seinen Räten sagte: ,,Welcher unter Euch würde so oiel
für mich tun, wenn ich in dieser Lage wäre?« Und obgleich er
das Anerbieten des Freundes nicht annahm, sondern sagte, er
könne es nicht tun, weil es gegen das Gesetz sei, so ergriff ihn
doch die Wahrheit mächtig. Einige Zeit darauf schickte er den
Generalmajor Desborough in der Absicht, uns frei zu lassen;
dieser kam und bot uns die Freilassung an unter der Bedingung,
das wir oersprechen, heim zu gehen und nicht mehr zu predigen,
, aber wir wollten ihm nichts versprechen; daraufhin schlug er uns
vor zu versprechen, nach Haus zu gehen, ,,wenn der Herr es zu-
lasse«, worauf Edward Phot ihm einen abschlägigen Brief schrieb.
Als einige Zeit verstrichen war, seitdem dieses Schreiben
abgegeben worden war, schrieb ich ebenfalls an ihn, folgender-
masen:
,,Freund,
Wir, die wir in der Kraft Gottes des Herrschers aller Dinge
sind, die wir seine Kraft kennen und in ihr wohnen, müssen ihr
auch gehorchen; und darum müssen wir uns frei halten von
allem, was Menschenwille befiehlt. Wenn es sich darum handelt,
etwas zu kaufen oder zu verkaufen, so mag es etwa angehen zu
sagen: wir wollen, so der Herr es zuläst; aber da wir in der
Kraft Gottes stehen, unter keines Menschen Willen, so können wir
solches nicht mit Wahrhaftigkeit sagen, wo es sich um unsere
Befreiung aus der Gefangenschaft handelt .....
13. des 6. Monats 1656. G. F.

% \picinclude{./120-129/p_s120.jpg} 
Bald darauf kam Major Desborough nach Castle-Green und
spielte Kegel mit einigen Richtern und anderen. Einige Freunde
wurden getrieben zu ihnen zu gehen, und sie zu ermahnen, ihre
Zeit nicht mit solch eitlen Dingen zuzubringen, und sich ihren
unnützen Vergnügen hinzugeben, und ihnen zu bedenken zu geben,
das, wenn sie, die sich doch für Christen ausgeben, dennoch solchen
Vergnügen nachgehen, während sie die Diener Gottes gefangen
halten, Gott sie wegen solchen Treibenö heimsuchen werde. Aber
trotz allem, mas man ihnen sagte oder schrieb, lies er une; im
Gefängnis. Wir hörten später, das er die Sache Hauptmann
Vennet übergeben habe, der unö freigelassen hätte, wenn wir
dem Kerkermeister Geld gegeben hätten. Aber wir erklärten ihm,
das wir das nicht tun könnten, da wir ja unschuldigerweise ge-
fangen waren ..... Schlieslich kam die Kraft dez Herrn so
über ihn, das er uns freiwillig die Freiheit schenkte, am 13. Tage
des 7. Monats 1656. .... Wir waren seit dem Frühling
gefangen gewesen.
%%%%%%%%%%%%%%%%%%% Kapitel 10. %%%%%%%%%%%%%%%%%%%%%%%%%%%%%%

\chapter[Warnung an die Kegelspieler. Naylors Fall.]{Warnung an die Kegelspieler. Naylors Fall.}

\begin{center}
\textbf{Warnung an die Kegelspieler. Naylors Fall. Disput mit Paul
Gwin. Besuch bei Cromwell. Herumreisen bei den gefangenen
Freunden. Reise in Wales.}
\end{center}

\section{Zurück in der Freiheit}

Als ich während meiner Gefangenschaft sah, wie sie sich in
Castle-Green\ort{Castle-Green} mit Kegelschieben\index{Spielen!Kegeln} 
vergnügten, hatte ich eine Schrift geschrieben, worin es hieß:

\brief{An die Kegler im Castle-Green}
{
    Es gehet das Wort des Herrn an euch, ihr eitlen Müsiggänger, 
    die ihr so dem Spiel, dem Vergnügen und solchen 
    einfältigen Übungen zugetan seid, das ihr bedenken möget, was ihr
    tut. Ist das der Zweck eures Daseins? Machte Gott alles zu
    eurem Vergnügen und eurem Gebrauch? Machte nicht Gott alle
    Dinge, damit er darin in Furcht und Anbetung, im Geist und in
    der Wahrheit, in Gerechtigkeit und Heiligkeit geehrt werde? Wie
    könnet ihr Gott dienen, solange ihr euren 
    Vergnügen\index{Vergnügen} nachgeht?
    Ihr könnet nicht Gott dienen und den weltlichen Vergnügen, dem
    Kegeln, Jagen und Trinken und dergleichen; wenn euer Herz bei
    derartigem ist, so will Gott eure Lippen nicht, fraget euch, ob das
    nicht wahr ist [...]. 

    \bigskip
    % \picinclude{./120-129/p_s121.jpg} 
    An die Kegler im Castle-Green\ort{Castle-Green}, geschrieben 
    im Kerker zu Launceston\ort{Launceston}.
}

Als wir nun frei waren, [...] zogen wir [...] über Launceston 
[...]. Okington [...] nach Exeter\ort{Exeter}, wo viele Freunde 
gefangen waren, unter anderem James 
Naylor\person{Naylor, James!Gefangenschaft}. 
Kurz ehe wir frei wurden, hatte James Naylor sich in phantastische 
Ideen verirrt und viele mit ihm, was eine große Verwirrung im Land
anrichtete. Er kam nach Bristol und stiftete dort Unruhe; von
da wollte er nach Launceston gehen, um mich zu besuchen; aber
unterwegs wurde er angehalten und in Exeter gefangen gesetzt,
sowie verschiedene Andere; einer davon, ein ehrlicher, gottseliger
Mensch, starb in der Gefangenschaft; sein Blut kommt auf seine
Verfolger.

\section{Verwirrung und Konflikte mit James Naylor}

Am Abend, als wir nach Exeter kamen, redete ich mit James
Naylor, denn ich sah, das er ganz in Irrtum geraten war, sowie
auch seine Genossen. Am folgenden Tag -- es war der Erste
Tag -- besuchten wir die Gefangenen und hatten im Gefängnis
eine Versammlung mit ihnen; aber James Naylor und einige
von ihnen konnten es in der Versammlung nicht aushalten. Es
kam ein Kavallerie-Korporal in die Versammlung; er wurde 
gewonnen und blieb ein sehr guter Freund.


Am folgenden Tag redete ich wieder mit James Naylor; er
machte herunter, was ich ihm sagte, und war verwirrt und 
verdreht, dennoch wollte er gerne kommen und mich küssen. Aber
ich sagte, \zitat{weil er sich der Kraft Gottes widersetze, 
so könne ich seine Freundlichkeitsbezeugungen nicht annehmen}; 
der Herr trieb mich, ihn zu verweisen und ihn unter die Kraft 
des Herrn zu stellen. So war nun, nachdem ich gegen die Welt 
gekämpft, unter den Freunden ein böser Geist erwacht, gegen 
den man kämpfen musste. Ich ermahnte ihn und seine Genossen. 
Als er nach London\ort{London} kam, wurde ihm sein Widerstand 
gegen Gottes Kraft in
mir und gegen die ihm durch mich verkündete Wahrheit zur
größten Last. Aber er kam dazu, seine Abirrung einzusehen und
zu verdammen,\index{Buße} und nach einiger Zeit kehrte er 
sich der Wahrheit
wieder zu, wie man in den gedruckten Berichten seiner Buße,
Verurteilung\footnote{Im Original steht 
\zitat{Ververurteilung} offebar ein Fehler.} und 
Wiedererhebung ausführlich sieht [...].

Von Exeter gingen wir [...] zu E. Pyot\person{Pyot, E.} 
in Bristol\ort{Bristol}. Am
Morgen des Ersten Tages ging ich zu der Versammlung in
Broadmead\ort{Broadmead}; sie war zahlreich und ruhig. 
Es wurde eine Versammlung angezeigt auf den Nachmittag 
im Garten. Es war
% \picinclude{./120-129/p_s122.jpg} 
ein ungebildeter, unverschämter Baptist in Bristol, namens Paul
Gwin,\person{Gwin, Paul} der schon früher in unseren 
Versammlungen große Störungen\index{Versammlung!Störung}
verursacht hatte, ermutigt und angetrieben durch den 
Bürgermeister, welcher, wie gesagt wurde, ihm sogar manchmal ein
Mittagessen gab, um ihn zu ermutigen. Er war von einer
solchen Pöbelmenge gefolgt, das die Zahl derer, die zu unserer
Versammlung im Freien kamen, oft auf 10.000 geschätzt 
wurde.\index{Versammlung!Große}
Als ich auf dem Wege nach dem Garten war, sagte man mir,
das Paul Gwin, der zänkische Baptist, zur Versammlung kommen
würde. Ich sagte, man solle sich nicht darum kümmern, es sei
mir einerlei, wer käme.\index{Versammlung!für alle offen} 


In Garten angekommen, stieg ich auf
einen Stein, auf den die Freunde zu stehen pflegten, wenn sie
sprachen;\index{Rede(-beitrag)}\index{Versammlung!Rede(-beitrag)}
\index{Versammlung!im Freiehen} 
und der Herr trieb mich, den Hut abzunehmen\index{Hut!abnehmen} und so
geraume Zeit zu stehen und mich von den Leuten ansehen zu
lassen; es waren einige 1000 Leute da. Als ich nun so schweigend
dastand, fing jener Baptist an, mein Haar zu 
tadeln,\person{Fox!lange Haare}\index{Haare!lange} aber ich
sagte nichts zu ihm. Da brach er in einen Wortschwall aus und
rief: \zitat{Ihr Weisen von Bristol, ich staune über euch, das ihr hier
steht, um Einen etwas sagen und behaupten zu hören, der es
nicht beweisen kann}. Da öffnete der Herr meinen Mund, (bis
dahin hatte ich noch nichts geredet,) und ich fragte die Leute, ob
sie mich je hätten reden hören oder je zuvor gesehen hätten? und
ich hieß sie, nicht zu vergessen, was für eine Sorte von Mensch
der sei, der so frech sage, das ich rede und behaupte, was ich
nicht beweisen könne, da doch weder er noch sie mich je zuvor
gesehen hätten. Darum sei es ein lügnerischer, böswilliger und
schlechter Geist, der aus ihm rede; er sei vom 
Teufel\index{Teufel} und nicht
von Gott. Ich gebot ihm, bei der Furcht und Kraft Gottes zu
schweigen, und die mächtige Kraft Gottes kam über ihn und alle
seine Anhänger. Darauf hatten wir eine herrliche, friedliche 


Versammlung, und das Wort des Leben; ward unter ihnen verkündet
und sie kehrten sich von der Finsternis zum Licht, zu Jesus
Christus, dem Heiland. Die Schrift wurde ihnen reichlich 
geöffnet und die menschlichen Überlieferungen, Zutaten, Mittel und
Lehren darin nachgewiesen; sie wurden auf das Licht Christi 
hingewiesen, durch das sie solches alles erkennen können, sowie auch
Christum selbst, damit er sie erlöse. Ich erklärte ihnen auch die
Zeichen und Sinnbilder von Christus in den Zeiten des Gesetzes
und zeigte ihnen, das Christus gekommen war, den Zeichen,
\index{Christologie}\index{Gesetzlichkeit}
% \picinclude{./120-129/p_s123.jpg} 
Zehnten und Eiden ein Ende zu machen und das Schwören 
abzuschaffen, und einsetzte, das man bei \zitat{ja} 
und \zitat{nein} bleibe,
und das man umsonst predige, denn er wolle nun sein Volk
selber lehren und sein herrlicher \zitat{Ausgang aus der Höhe sei nun
erschienen} (Luk. 1:78\bibel{Luk. 01:78@Luk. 1:78}). Mehrere 
Stunden verkündete ich das
Wort des Lebens unter ihnen und die ewige Kraft Gottes, durch
die sie möchten zu dem, der von Anfang war, zurückkehren und
mit ihm oersöhnt werden (2 Cor. 5). Und nachdem ich sie an den
Geist Gottes in ihrem Jnnern gewiesen, der sie in alle Wahrheit
leiten würde (Joh. 16), trieb es mich zu beten in der mächtigen
Kraft Gottes; des Herm Kraft kam über alle. Als ich geendet,
fing jener Kerl aufs Neue an zu schwaizen. John Audland
wurde getrieben, ihn zur Buse und Furcht Gottes zu vermahnen.
Da nun seine eigenen Leute und Anhänger sich seiner schämten,
ging er fort und hat nie mehr eine unserer Versammlungen ge-
stört. Die Versammlung ging ruhig auseinander und des Herm
Kraft und Herrlichkeit leuchtete über allen; es war ein gesegneter
Tag; die Ehre war des Herrn. Einige Zeit darauf ging dieser
Paul Gwin über Meer; Viele Jahre nachher begegnete ich ihm
wieder in Varbadoes .....
Von Kingston ritten wir nach London. Als wir in die Nähe
des HydePark kamen, sahen wir eine grose Volks-menge, und bei
näherem Zusehen erblickten wir den Protektor, der in seinem
Wagen daherkam. Jch ritt an die Seite seines Wagens. Einige
seiner Leibgarde suchten mich wegzutreiben, aber er wehrte es
ihnen. So ritt ich neben ihm her und verkündete, was der Herr
mir eingab über seinen Zustand und über die Not der Freunde
im Lande; ich zeigte ihm, wie sehr diese Verfolgungen Christus
und seinen Aposteln und dem Christentum zuwider seien. Als wir
am Tor des James Park ankamen, verlies ich ihn; ehe wir uns
trennten, sagte er noch, ich solle zu ihm nach Hause kommen.
Am folgenden Tag kam eine Magd seiner Frau zu mir in meine
Wohnung und erzählte mir, ihr Herr sei zu ihr gekommen und
habe gesagt, er wolle ihr eine frohe Nachricht mitteilen. Als sie
ihn fragte, was für eine, sagte er ihr: George Fox sei in die
Stadt gekommen. Sie habe geantwortet, das sei in der Tat eine
gute Nachricht (denn sie hatte die Wahrheit angenommen), aber sie
habe es kaum glauben können, bis er ihr gesagt habe, wie ich ihn
getroffen und mit ihm von Hyde Park bis James Park geritten sei.


% \picinclude{./120-129/p_s124.jpg} 
Nach einiger Zeit gingen Edward Pyot und ich nach White-
hall, und als wir vor den Protektor kamen, war Or. Owen, Vize-
kanzler von Oxford, bei ihm. Es trieb uns, Oliver Eromwell die
Not der Freunde vorzustellen.   wiesen ihn auf das Licht
Jesu Christi, das jeden, der in die Welt kommt, erleuchtet.
Er sagte, es sei ein natürliches Licht, aber wir bewiesen ihm das
Gegenteil und legten dar, wie es göttlich und geistig sei, da es
von Christus ausgehe, dem geistigen und himmlischen Menschen
und das eben das, was das ,,Leben in Ehristus« genannt werde,
das nenne man auch das ,,Licht in uns-.« ,Die Kraft des Herrn
ging aus in mir und trieb mich, ihn zu ermahnen, seine Krone
zu den Füsen Jesu niederzulegenst) Wiederholt redete ich mit
ihm in dieser Absicht. Zuletzt — ich stand neben dem Tisch ——
kam er und setzte sich auf die Tischecke neben mich, sagte, er wolle
so hoch sein wie ich und fuhr fort gegen das Licht Jesu Christi
zu sprechen und ging gleichgültig hinaus. Aber des Herren Macht
kam über ihn, sodas, als er zu seiner Frau und zu andern
Leuten kam, er sagte: ,,nie habe ich sie in der Weise verlassen«;
denn er war in sich selbst gerichtet .....
Als er fort war, trafen wir beim Hinausgehen mit vielen
angesehenen Leuten zusammen, und einer von ihnen sing an, uns
gegen das Licht und die Wahrheit zu reden, und ich verachtete
ihn deswegen. Da sagte mir ein anderer, er sei der General-
Major von Northamptonshire. ,,Was!«, sagte ich, ,,unser früherer
Verfolger, der so viele unsrer Freunde in die Gefangenschaft ge-
schickt hat und eine Schande ist für die Christenheit und die
Religion? Jch bin froh, sdas ich dich getroffen habe.« Und ich
redete nun ernstlich mit ihm über sein unchristliches Benehmen;
mid er schlich hinweg, denn er hatte die Verfolgungen in North-
hamptonshire sehr grausam betrieben gehabt .....
Von London ging ich nach Buekinghamshire, . . dann nach
Huntingdonshire, . . . Boston, . . . Edgehill, . . .Warwick . . .
und wieder zurück nach London. Überall hatte ich mich beflissen,
das, was mir der Herr aufgetragen, zu erfüllen. Denn, nachdem
ich aus der Gefangenschaft in Launceston entlassen war, hatte mich
der Herr getrieben im Lande umher zu reisen, wo die Wahrheit
sich verbreitet und recht befestigt hatte, um noch allerlei Ein-
lt Cromwell lehnte 1656 den Königötitel ab.


% \picinclude{./120-129/p_s125.jpg} 
Warnung an die Kegelspieler. Naylorö Fall usw. 125
wände zu beseitigen, welche die bösen Priester und ,,Frommen«
in den Gemütern gegen uns gepflanzt hatten ..... Und in
dieser Absicht trieb es mich nun auch, allerlei Erklärungen ergehen
zu lassen, .... unter anderm folgende: ,,Es wird den Quäkern
oft vorgeworfen, das sie das sogenannte Sakrament von Brot und
Wein bestreiten, von dem es heist, man müsse ez gebrauchen zum
Gedächtnis Christi (Luc. 22,19) biö an der Welt Ende. Wir
hatten dez-wegen und wegen der verschiedenen Arten des- Sakra-
mentö-Gebrauchö im sogenannten Christentum viel Mühe mit den
Priestern und ,,Frommen«, denn manche nehmen es knieend,
manche sitzend; aber keine von allen, die ich je gesehen, nehmen
etz, wie die Jünger etz nahmen, nämlich in einem Zimmer nach
dem Nachtessen, sondern die meisten nehmen es vor dem Mittag-
essen und manche sagen, wemi der Priester Brot und Wein ge-
segnet hat, »eS ist der Leib Christi«. Christus aber sagte nur, g
»tut etz zu meinem GedächtniZ«. Er sagte ihnen nicht, wie oft
sie es tun müsten oder wie lang; auch gebot er ihnen nicht, es
ihr Lebenlang zu tun, noch das alle, die an ihn glauben, es
tun sollten biz an der Welt Ende. Der Apostel Paulus, der erst
nach Christi Tod bekehrt worden, sagt den Corinthern, er habe vom
Herrn empfangen, was er ihnen in dieser Sache mitteile, und er
führt Christi Worte in bezug aus den Kelch also an: ,,DieseS
tut, so ost ihr trinket, zu meinem Gedächtni-8«; und er selbst fügt
bei: ,,denn so ost ihr das Brot esset und den Kelch trinket, so
verkündet ihr dez Herrn Tod, bis das er kommt« (1. Cor. 11,26).
Nach dem also, was der Apostel hier mitteilt, gebot weder Christuö
noch er, dietz allezeit zu tun, sondern stellten etz jedem frei. . .
Die Juden pflegten einen Kelch zu gebrauchen und Brot zu
brechen und an ihren Festen unter sich zu verteilen, wie man an
den jüdischen Altertümern sieht; datz Brechen des Broteö und das
Trinken des Weineö waren also jüdische Gebräuche, die nicht sitr
immer zu bestehen brauchen. Sie tauften auch mit Wasser; da-
rum besremdete ez sie nicht, als- Johannes der Täufer auftrat
mit seiner Wassertause ..... sWaZ aber Brot und Wein an-
belangt, so hatte Christa?. gesagt, das er das- Brot dez Lebens
sei (Joh. 6,48), das vom Himmel kommt, Amd das er kommen
wolle und in ihnen wohnen. Das betrachteten die Apostel nun
als erfüllt und ermahnten die andern, nach dem zu trachten, das
von oben kommt (Col. 3,2). Jhr nun, die ihr diesen äusern Wein


% \picinclude{./120-129/p_s126.jpg} 
trinket und dieses äusere Brot esset zum Gedächtnis des Todez
Christi, kennet ihr gar nichtö Bessereö, um dem Tode Christi näher
zu kommen? ....
ES mus freilich durch manchen Zustand hindurch gehen, ehe
die Leute dazu gelangen, das, waö von oben kommt, zu sehen
und daran teilzunehmen. Zuerst kommt der Gebrauch dee äuseren
Brotes und Weines zum Gedächtnis Christi; daö war zeitlich und
nicht gezwungen, sondern freiwillig ..... Zweitens kommt da-3
Eingehen in seinen Tod, ein Leiden mit Christus, und das ist
notwendig zum Heil und nicht zeitlich, sondern beständig; etz mus
ein täglichez Sterben sein. Drittens ein Begrabensein mit Ehriftuö;
oiertens ein Auferstehen mit Christus; fünstens nach dem Aufer-
standensein mit Ehristus (Röm. 6) ein Trachten nach dem, wa-3
droben ist, ein Suchen nach dem Brote, das vom Himmel her-
unter kommt, ein Essen davon und eine Gemeinschast durch das-
selbe. .... Die Gemeinschaft, die sich aus den Gebrauch von
Brot, Wein, Wasser, Beschneidung, äusere Tempel und sichtbare
Dinge gründet, wird ein Ende haben; die Gemeinschaft aber, die
sich auf das Evangelium gründet, auf die Kraft Gotteö, die war,
ehe der Teufel gewesen, und die Leben und unoergänglicheö Wesen
anö Licht bringt, und durch welche die Leute über den Teufel
sehen, der sie versinstert, diese Gemeinschaft wird ewig be-
stehen« .... .
Somit waren die Einwände dieser Pticftet und ,,Frommen«
widerlegt .... und die Wahrheit breitete sich in diesem Jahre
(1656) recht auö, und viele Tausende bekehtten sich zum Herm,
sodas selten weniger alö tausend im Gefängnis waren um der
Wahrheit willen, etliche wegen des Zehntenwesens, etliche
weil sie ins Turmhauö gegangen waren, etliche wegen irgend-
welcher sogenannten Misachtung, etliche wegen des Schwörenz
oder weil sie ihre Hüte nicht abgenommen .....
Von London zogen wir wieder weiter im Lande umher ....
nach Farnham . . . Bafmgstoke . . . Exeter . . . Bristol, . . . dann
nach Brecknock (Waleö) und dann . . . wieder nach England zu-
rück, nach Shrewöburh .....
E6 war um die Zeit eine grose Trockenheit im Lande ....
A13 nun Oliver Csromwell ein Fasten proklamierte um Regen,
nahm man wahr, das im Norden, soweit sich die Wahrheit aus-
gebreitet hatte, erquickende Niederschläge waren, während sie im


% \picinclude{./120-129/p_s127.jpg} 
Warnung an die Kegclspieler. Noylote Fall usw. 127
Süden vielerortö schier umkanten auö Mangel an Regen. Da
trieb etz mich, eine E-rwidemng auf die Proklamation dez Pro-
tektorö zu schreiben, worin ich ihm sagte, wenn er sich Gottes
Wahrheit zugewendet hätte, so hätte er Regen gehabt, die Trocken-
heit sei ein Zeichen für ihre Dürre und ihren Mangel an Wasser
des Lebens .....
Wir gingen wieder nach Wales und hatten mehrere Ver-
sammlungen, bie wir nach Tenby kamen. Auf der Strase kam
mir ein Frieden?-richter entgegen und forderte mich auf, zu ihm
in seine Wohnung zu kommen, mas ich denn auch tat. Am
Ersten Tage kam der Stadtmajor und einige der Häupter der
Stadt und blieben während der ganzen Zeit der Versammlung.
John-ap-John aber verlies sie und ging ins Turmhaus, wo ihn
der Gouveneur gefangen nehmen lies. GS war eine herrliche
Versammlung. Am Morgen dez zweiten Tagetz schickte der Gou-
verneur einen seiner Leute inö Haus dez Friedensrichterö, um mich
holen zu lassen, wae dem Major und dem Friedenörichter, die
beide bei mir waren, sehr leid tat. Sie gingen darum gleich
vorauö zum Gouverneur, und nach einiger Zeit kam ich ihnen nach
mit dem Beamten. Alö ich eintrat, sagte ich: ,,Friede sei diesem
Hause.« Und ehe noch der Gouverneur mich etwaz fragen konnte,
fragte ich ihn, warum er meinen Freund John-ap-John ine Ge-
fängnis getan habe? »Weil er seinen Hut in der Kirche aufbe-
hielt«, erwiderte der Gouverneur. Ich sagte darauf: ,,Hatte nicht
der Priester zwei Kappen auf dem Kopf. eine weise und eine
schwarze? Schneide meinem Freund den Rand an seinem Hut
weg, und dann hätte er nur eine solche Kappe, und der Rand
ist nur, um ihn vor dem Regen zu schützen.« ,,DaZ sind dumme
Sachen«, sagte der Gouverneur. ,,Warutn wirfst du dann meinen
Freund ins Gefängnis wegen dummer Sach en ?« sagte ich. E-r fragte
mich nun, ob ich die Grwählimg und Verwerfung annehme?
,,Ja«, sagte ich, ,,und du bist in der Verwerfrmg««. Das machte
ihn bös und er drohte mir, er wolle mich inö Gefängnis werken,
bis ichö ihm beweise; ich sagte, ich könne ihm das gleich beweisen,
wenn er der Wahrheit Gehör schenke.-i Nun fragte ich ihn,. ob denn
Wut und Zorn nicht Zeichen der Verwerfung seien? denn der,
welcher auö dem Fleisch geboren sei, verfolge den, der auö dem
Geist geboren sei. Christuö und seine Jünger haben nie jemand
verfolgt oder gefangen genommen. 7,* Daraufhin bekannte er osfen,


% \picinclude{./120-129/p_s128.jpg} 
das er zu leidenschaftlich und zommütig sei. Ich sagte ihm, er
sei wie Gsau, der Grstgeborene, und nicht wie Jakob .... Die
Kraft dez Herrn kam so mächtig über ihn, das er sich zur Wahr-
heit bekannte, und der Friedenzrichter kam und schüttelte mir
die Hand ..... Als ich weiter zog, trieb es mich, noch einmal
mit dem Gouverneur zu reden und er lud mich zum Essen ein
und gab meinem Freund die Freiheit ..,.. Wir gingen nach
England zurück . . . nach Liverpool . . . Manchester . . . Lan-
caster . . . nach Swarthmore, wo die Freunde sich sehr freuten,
mich wieder zu sehen. Jch blieb während zwei Ersten Tagen
dort und ging in mehrere Versammlungen. Die Freunde freuten
sich mit mir der Güte Gottes, die mir durch so viele Gefahren
hindurch geholfen. Jhm sei Preis ewiglich.

%%%%%%%%%%%%%%%%%%% Kapitel 11. %%%%%%%%%%%%%%%%%%%%%%%%%%%%%%

\chapter[Kampf gegen die Prädestinationslehre]{Kampf gegen die Prädestinationslehre}

\begin{center}
\textbf{Reise nach Schottland. Kampf gegen die 
Prädestinationslehre und Widerstand der schottischen 
Geistlichkeit.}
\end{center}

Ich hatte schon längere Zeit in meinem Innern einen Zug
verspürt, nach Schottland zu gehen, und hatte Oberst William
Osburn\person{Osburn, Oberst William} in Schottland bitten 
lassen, mir entgegen zu kommen; und
so kamen er und einige andere von Schottland\ort{Schottland} 
her zur Versammlung nach Pardsen Crag. Als dieselbe zu 
Ende war, die, wie
er sagte, die aller herrlichste gewesen sei, die er je erlebt habe,
ging ich mit ihm und seinen Begleitern nach Schottland [...].

\section{Kampf gegen die Prädestinationslehre}

An einem ersten Tage hatten wir in Heads\ort{Heads} eine große 
Versammlung, der viele \textit{Fromme} beiwohnten. Die Priester hatten
den Leuten Angst gemacht gehabt mit der Lehre von der Erwählung
\index{Prädestinationslehre}\index{Erwählung}\index{Verwerfung}
und der Verwerfung; sie hatten ihnen gesagt, Gott habe die Meisten
für die Hölle\index{Hölle} bestimmt; sie könnten nun beten, 
predigen und singen,
so viel sie wollten, es sei alles umsonst, wenn sie für die Hölle
bestimmt seien; Gott habe eine gewisse Anzahl für den Himmel
auserlesen; die könnten tun, was sie wollten, wie David der 
Ehebrecher\index{Ehebruch} und Paulus der Verfolger, sie 
seien dennoch für den
Himmel bestimmte Gefäße. Es hänge also nicht vom Tun der
Menschen, sondern von der Bestimmung Gottes 
ab. \index{Werkgerechtigkeit}


Es trieb
mich, diesen Leuten die Verkehrtheit in der Lehre ihrer Priester
aufzudecken, und ich zeigte ihnen, wie sie die Schriftstellen aus
% \picinclude{./120-129/p_s129.jpg} 
dem Judasbrief\bibel{Judasbrief} und andere, auf die sie sich 
beriefen, verdreht hatten. \index{Exegese} Ihre Behauptung, 
das gar nichts von dem Tun des
Menschen abhänge, widerlegte ich ihnen aus dem Judasbrief, wo
die Schuld deutlich zu sehen ist bei Kain, Korah und Bileam, von
denen es heißt, sie seien von Anbeginn zur Verdammung bestimmt
gewesen. Denn hatte nicht Gott Kain und Bileam gewarnt und
an Kain die Frage gerichtet: \zitat{Wenn du recht tust, bist du dann
nicht angenehm vor Gott?} Und hat der Herr Korah und seine
Rotte nicht aus Ägypten geführt und sie haben sich ihm und
seinem Gesetze und Moses trotzdem widersetzt! 
\zitat{Man sieht deutlich,}
sagte ich, 

\grosszitat{
    das Kain und Korah und Bileam schuldig waren, so
    wie es alle sind, die ihre Wege gehen. Oder haben die etwa
    keine Schuld, die sich Christen nennen und doch dem Evangelium
    zuwider handeln, wie Korah wider das Gesetz, und die vom
    Geist abirren, wie Bileam und Übles tun wie Kain? An ihnen
    liegt die Schuld, und nicht an der Verwerfung und nicht an Gott.
    Sagt nicht Christus: \zitat{Gehet hin und predigt das Evangelium
    aller Welt.} Er würde sie nicht in alle Welt geschickt haben, um
    die Lehre vom Heil zu predigen, wenn die meisten Menschen für
    die Hölle bestimmt wären. Und war nicht Christus ein Sühneopfer 
    für der ganzen Welt Sünden? für die Verworfenen wie für
    die Auserwählten? Er starb für alle Menschen, für die Guten
    wie für die Bösen, wie der Apostel bezeugt, 
    2.~Cor.~5:15\bibel{Cor. 2. 05:15@2. Cor. 5:15} und
    Römer 5:6.\bibel{Römer 05:06@Römer 5:6}  Christus gebietet, 
    das alle an das Licht glauben; 
    die, welche aber das Licht hassen, an das Christus zu glauben
    befiehlt, die sind Verworfene. Und wieder heißt es: \zitat{In einem
    Jeglichen zeigen sich die Gaben des Geistes zu gemeinem Nutzen}!
    Die aber den Geist beleidigen, unterdrücken und betrüben, die
    sind Verworfene und Schuldige, sowie auch die, welche das Licht
    hassen. Der Apostel sagt, 
    Tit.~2:11-12\bibel{Tit. 02:11-12@Tit. 2:11-12}: 
    \zitat{Die heilsame Gnade
    Gottes ist allen Menschen erschienen und lehret uns abzusagen
    aller Fleischeslust und hoffartigem Wesen und züchtig, gerecht und
    gottselig zu wandeln auf dieser Erde.} Und darum sind alle,
    Männer wie Frauen, Schuldige, wenn sie gottlos leben und das,
    was sie selig machen würde, verachten. Aber scheints sehen die
    Priester die Schuld nicht bei denen, die nicht an Gott und nicht
    an Christus, der sie erkauft hat, glauben und an sein Licht und
    seine Gnade, die sie selig machen könnte. Alle aber, die Christi Befehl
    gehorchen und an sein Licht glauben, sind Auserwählte und stehen
    % \picinclude{./130-139/p_s130.jpg} 
    in der Zucht der göttlichen Gnade, welche selig macht. Die aber,
    die Gottes Gnade in Mutwillen kehren und sein Licht hassen
    (Jud.\bibel{Jud.}), sind verworfen; darum ermahne ich 
    alle, an das Licht zu
    glauben wie Christus gebietet, und die Gnade, die sie umsonst
    lehrt, anzunehmen; dann werden sie gewislich selig,
    \index{selig} denn sie genüget. 
}

Viele andere Schriftstellen über die Verwerfung wurden
auch noch ausgelegt, und die Augen der Leute wurden geöffnet,
so das eine Quelle des Lebens unter ihnen hervorsprudelte.

\section{Konflikte in Schottland mit den Independenten}

Solches kam den Priestern bald zu Ohren; denn den Leuten,
welche durch ihre schrecklichen Lehren irre geführt worden waren,
gingen allmählich die Augen auf, und sie kamen in den Bund
des Lichts.\index{Bund des Lichts} Die Kunde, das ich 
nach Schottland gekommen sei,
verbreitete sich unter den Priestern. Und sie erhoben ein großes
Geschrei, das jetzt alles aus sei; denn ich hätte schon in England
alle rechten Männer und Frauen abspenstig gemacht, und ihnen
bleibe dann, wie sie selber zugaben, der schlechtere Teil. Sie 
veranstalteten darum große Zusammenkünfte von Priestern und
stellten eine ganze Reihe von 
Verdammungen\index{Verdammung!der Quaker} zusammen, welche
in den Turmhäusern verlesen werden sollten, und die Leute sollten
\zitat{Amen} dazu sagen. Einige davon will ich hier mitteilen. Zuerst
hieß es: \zitat{Verflucht ist, wer sagt, ein jeder habe 
ein Licht in sich, \index{Inneres Licht}
welches genüge, um ihn selig zu machen. Dazu sage ein jeder:
Amen.} [...] Nun sagt aber Christus: \zitat{Glaubet an das Licht,
damit ihr Kinder des Lichtes werdet} 
(Joh. 12:36)\bibel{Joh. 12:36} und weiter:
\zitat{wer da glaubt, der soll selig werden} 
(Mark. 16\bibel{Mark. 16}); und \zitat{wer
da glaubt kommt vom Tode ins Leben} [...] Und der Apostel
sagt: \zitat{Ihr tut wohl, auf das Licht zu achten, das da scheinet
an einem dunklen Ort, bis der Tag anbreche und der 
Morgenstern aufgehe in euren Herzen} 
(2. Petr. 1:19\bibel{Petr. 2. 01:19@2. Petr. 1:19}) [...] Was
den 2. Punkt anbelangt, wo es heißt: \zitat{Verflucht, wer sagt, der
Glaube sei ohne Sünde,} [...] so ist er ja eine Gabe Gottes
und jede Gabe Gottes ist rein [...]. Der Glaube, dessen
Ursprung Christus ist, ist köstlich, göttlich und ohne Sünde. Dies
ist der Glaube, der die Herrschaft über die Sünde gibt und den
Zugang zu Gott [...] Aber sie sind alle von diesem Glauben
abgefallen [...]


Es waren in Schottland zwei Kirchen der 
Independenten\index{Independenten};
in der einen fanden viele Bekehrungen statt; aber der 
Prediger der andern war sehr erbost über die Wahrheit und die
% \picinclude{./130-139/p_s131.jpg} 
Freunde. Sie hatten Älteste die sich oft bestrebten, ihre Gaben
an ihren Gemeindegliedern zu brauchen und sich oft recht empfänglich
zeigten; aber da ihr Prediger so viel gegen uns und gegen das
Licht redete, verdunkelte sich ihr Blick, das; sie ganz blind wurden
und ganz dürr und ihre Empfänglichkeit verloren. Er fuhr fort
gegen die Freunde und gegen das Licht, aus Christus zu
predigen und nannte dasselbe ein natürliches Licht. Eines Tages
beschimpfte er in seiner Predigt das Licht und da fiel er hin wie
tot in seinem Pult. Man trug ihn hinaus und legte ihn auf
einen Grabstein und flößte ihm ein starkes Getränk ein, das ihn
wieder zum Leben brachte; und sie trugen ihn heim, aber er war
schwachsinnig geworden. Er riss sich die Kleider vom Leib, hüllte
sich in einen schottischen Plaid und ging aufs Land zu den
Milchmädchen. Nachdem er etwa zwei Wochen dort gewesen,
kehrte er zurück und stieg wieder auf die Kanzel. Nun erwarteten
die Leute große Eröffnungen von ihm; statt dessen erzählte er,
wie ihm eines der Mädchen abgerahmte Milch, ein anderes
Buttermilch und wieder ein anderes gewöhnliche Milch gegeben
habe; man musste ihn wieder von der Kanzel herunter holen und
heim führen. Der, welcher mir dies alles berichtete, ist Andrew
Robinson,\person{Robinson, Andrew} einer seiner eifrigsten 
Zuhörer, der aber später sich
bekehrte und die Wahrheit annahm. Er sagte mir, das er nie
etwas davon gehört habe, das jener Prediger seinen Verstand
wieder bekommen habe. Daran möge ein jeder sehen, wie es
dem geht, der das Licht beschimpft,. das Licht, welches das Leben
in Christus, dem Wort, ist; und es möge allen zur Warnung
dienen, welche Übles reden gegen das Licht Christi [...].

\section{Ausweisung aus Schottland}

Viele der schottischen Priester waren sehr in Aufregung über
die Verbreitung der Wahrheit, weil sie dadurch ihre Zuhörer
verloren; und viele von ihnen gingen darum nach Edinburg,
um beim Rate Oliver Cromwells eine Klage gegen mich 
vorzubringen. Infolge dieser eingereichten Klage kam, als ich einmal
aus einer Versammlung zurückkam, ein Beamter und brachte mir
folgenden Befehl:

\grosszitat{
Donnerstag, 8. Oktober 1657,\jahr{1657} der Rat Seiner Hoheit in
Schottland.

\bigskip

Es wird befohlen, das George Fox nächsten Dienstag,
13. Oktober, vormittags, vor dem Rat erscheint.

\bigskip

G. Downing, Ratsbeamter\person{Downing, Ratsbeamter G.}
}


% \picinclude{./130-139/p_s132.jpg} 
Als er mir den Befehl übergab, fragte er mich, ob ich kommen
wolle oder nicht. Ich antwortete ihm nicht darauf, sondern
fragte, ob der Befehl auch nicht gefälscht sei? er erwiderte nein,
es sei ein richtiger Befehl vom Rat, und er sei als Bote damit
gesandt. 


Ich erschien also zur vorgeschriebenen Zeit und wurde in
einen großen Saal geführt, wo viele angesehene Leute versammelt
waren, die mich alle aufmerksam betrachteten; schließlich wurde
ich ins Ratszimmer geführt und unter der Türe nahm man mir
den Hut ab;\index{Hut!abnehmen} ich fragte, warum das 
geschehe? wer denn drinnen
sei, das ich den Hut abnehmen müsse? ich habe ihn ja sogar vor
dem Protektor nicht abgenommen. Aber der Hut wurde aufgehängt 
und ich wurde hineingeführt. Als ich schon eine ganze
Weile drinnen war, ohne das jemand etwas zu mir sagte, trieb
mich der Herr zu sagen; \zitat{Friede sei mit euch! wartet in der
Furcht Gottes auf den Empfang seiner Weisheit von oben, durch
die alle Dinge geschaffen sind, das sie euch in allem, was euch
zu tun übergeben ist, leite, damit ihr es tut zur Ehre Gottes.}
Sie fragten mich, weshalb ich nach Schottland gekommen sei? ich
sagte: um den Samen Gottes aufzusuchen, der solange in den
Banden des Bösen gelegen habe, damit alle, welche sich in diesem
Lande zur Schrift, den Worten Christi, der Apostel und der 
Propheten bekennen, zum Licht und Geist und zur Kraft kommen, in
denen jene, die solche Worte geäußert, gewesen sind; und das sie
in diesem Geist die Schrift verstehen und Christus und Gott 
erkennen und mit ihm und unter einander in der rechten 
Gemeinschaft stehen möchten. Sie fragten mich, ob ich irgend etwas 
Geschäftliches hier zu besorgen habe? Ich verneinte; darauf fragten
sie weiter: wie lange ich im Lande bleiben wolle? Ich antwortete,
dies könne ich nicht sagen, wahrscheinlich nicht sehr lange; doch
da meine Freiheit dem Herrn gehöre, so müsse ich den Willen
dessen, der mich gesandt habe, tun. Darauf hieß man mich 
hinausgehen. Bald darauf lies man mich wieder herein kommen
und erklärte mir, ich müsse Schottland verlassen, von jetzt an in
7 Tagen. Ich fragte: warum? was ich getan habe? Sie sagten,
sie wollen nicht mit mir verhandeln. Darauf bat ich sie, zu hören,
was ich ihnen zu sagen habe; aber sie wollten nicht. Ich 
erinnerte sie daran, das Pharao, der doch ein Heide gewesen sei,
Moses und Aaron angehört habe, und Herodes hörte Johannes
den Täufer; sie sollten doch nicht schlechter sein als jene! Aber
% \picinclude{./130-139/p_s133.jpg} 
sie schrien: \zitat{hinaus! hinaus!}, woraus ich wieder hinautsgeführt
wurde. Ich kehrte in meine Wohnung zurück und fuhr fort, in
Edinburg\ort{Edinburg} die Freunde zu besuchen und auszurichten 
im Herrn.


Ich schrieb darauf an den Rat, um ihm sein unchristliches 
Benehmen gegen mich vorzuhalten [...].
Nach einiger Zeit ging ich wieder nach Heads,\ort{Heads} wo die
Freunde in großer Not gewesen waren; denn die 
Pretsbyterianer-Priester\index{Pretsbyterianer} hatten 
sie in den Bann getan und befohlen,
es solle niemand von ihnen kaufen oder ihnen etwas
verkaufen oder mit ihnen essen und trinken. So konnten sie weder
ihre Ware verkaufen, noch sich das ihnen Nötige anschaffen, was
viele in große Bedrängnis brachte. Denn wenn einer von ihren
Nachbarn ihnen Brot oder andere Lebentsmittel verkauft hätte, so
hätte ihn der Priester derart bedroht, das er schleunigst gekommen
wäre, die Sachen wieder zu holen. Aber Oberst 
Ashfield,\person{Oberst Ashfield} welcher
der Friedentsrichter jener Gegend war, machte diesem Vorgehen
der Priester ein Ende. Später wurde er selber gewonnen und hielt
Versammlungen in seinem Hause, verkündete selber die Wahrheit
und lebte und starb in derselben [...].


Die Wahrheit und die Kraft des Herrn breitete sich aus in
Schottland, und durch die Kraft und den Geist Gottes wurden
viele zum Herrn Jesus Christus bekehrt, ihrem Heiland und Lehrer,
der sein Blut für sie vergossen hat; und es ist seither ein großes
Wachstum und wird es immer mehr sein in Schottland. Denn
als zuerst die Hufe meines Pferdes schottischen Boden berührten,
da fühlte ich, wie überall Funken des Samens von Gott um mich
herum aufsprühten, wie unzählige Feuerfunken.\persion{Fox!Vision}
Nicht als ob nicht noch viel hartes, schlechtes Erdreich 
von Falschheit und Heuchelei dort gewesen wäre und ein 
knorriger Boden, der zuerst noch durch Gottes Wort fruchtbar 
gemacht werden muss
und gepflügt mit dem Pflug des Geistes, ehe der Same Gottes
geistliche, himmlische Früchte hervorbringen kann zu Gottes Ehre.
Aber der Landmann muss in Geduld warten 
(Jar. 5:7\bibel{Jar. 05:07@Jar. 5:7}).



% \picinclude{./130-139/p_s134.jpg} 
%%%%%%%%%%%%%%%%%%% Kapitel 12. %%%%%%%%%%%%%%%%%%%%%%%%%%%%%%

\chapter[Erste Jahresversammlung]{Erste Jahresversammlung}

\begin{center}
\textbf{Erste Jahresversammlung. Warnung an Cromwell vor 
der Königskrone. Trostbrief an dessen Tochter. Vorahnung 
vom Tode Cromwells und der kommenden Reaktion.}
\end{center}

\section{Gerüchte über Quaker}

Wir gingen zurück nach England [...] Die Priester
von Newcastel hatten verschiedene Bücher gegen uns geschrieben,
und ein Stadtältester, Ledger\person{Ledger}, war uns und der 
Wahrheit sehr  abgeneigt. Er sowie die Priester hatten 
behauptet, die Quäker könnten
nicht in einer Stadt leben, sondern schwirrten wie die Schmetterlinge
in den Hochtälern. \index{Gerüchte!über Quaker} Zu diesem 
Ledger und einigen andern Stadtältesten ging ich, mit 
Anthony Pearson,\person{Pearson, Anthony} mit der Bitte, eine 
Versammlung in Newcastel\ort{Newcastel} abhalten zu dürfen, 
nachdem sie so viel
gegen uns geschrieben haben, da wir nun ja in ihre grose Stadt
gekommen seien! Aber sie wollten uns keine Versammlung gestatten, 
noch wollten sie mit sich reden lassen.\index{Versammlungsverbot} 
Ich sagte: \zitat{Habt
ihr nicht die Freunde Schmetterlinge genannt und gesagt, wir
könnten nicht in Städten leben; nun sind wir in eure Stadt gekommen,
und ihr wollt uns nicht hören. Wer sind nun die Schmetterlinge?}
Ledger fing an, die Sabbathheiligung \index{Sabbathheiligung} 
\index{Sonntag} zu verteidigen. Aber ich
erwiderte ihm, an dem Tage, welcher der Sabbath sei, dem siebenten
Wochentag, hielten sie ja Märkte und Jahrmärkte; während der Tag,
an dem sich die, welche sich jetzt Christen nennen, versammeln, ja
der erste Wochentag sei. 


Da wir keine öffentliche Versammlung
unter ihnen halten konnten, so veranstaltete 
ich zu Gateshead\ort{Gateshead}
eine kleine im Kreise der Freunde und solcher, die sich zu ihnen
hielten, und dort wird seither eine Versammlung abgehalten im
Namen Jesu. Als ich über den Marktplatz ging, erfasste mich
die Kraft des Herrn, und ich ermahnte das Volk, an den Tag
des Herrn zu denken, der über sie kommen 
werde.\index{Perdigt!spontan} Und nicht
lange darauf wurden alle jene Priester von Newcastel und ihre
Anhänger Vertrieben, bei der Rückkehr des Königs [...] 

\section{Fox kritisiert das Theologiestudium}

Von Heatshead gingen wir nach Durham; es war einer von London
dorthin gekommen, um eine Schule zu errichten, worin \zitat{Prediger
Christi}, wie sie sich ausdrückten, ausgebildet werden sollten. 
\index{Theologie!Studium} Ich
ging zu ihm, um mit ihm zu reden und ihm zu zeigen, das das
Lehren von Griechisch und Latein und der sieben schönen Künste
nur ein Belehren des natürlichen Menschen sei und nicht das
% \picinclude{./130-139/p_s135.jpg} 
Mittel, die Leute zu Predigern Christi zu machen. 
\index{Studium!Latein}\index{Studium!Griechisch}
\index{Studium!schönen Künste} \index{Studium!Hebräisch} 
\index{Sprache!Latein}\index{Sprache!Griechisch}
\index{Sprache!Hebräisch}Die Sprachverschiedenheit 
komme von Babel, und den Griechen, deren Muttersprache 
griechisch war, war das Wort vom Kreuz Torheit, und
den Juden, deren Sprache hebräisch war, war Christus ein
Stein des Anstoßes 
(1.~Cor.~1:23\bibel{Cor. 1. 01:23@1. Cor. 1:23}).
Die Römer, die lateinisch
redeten, verfolgten die Christen; und Pilatus, der römische
Machthaber, schrieb in hebräischer, griechischer und lateinischer
Sprache eine Inschrift über das Kreuz Christi; daran, sagte ich,
könne man sehen, das die Sprachen von Babel kommen, da die
Inschrift über das Kreuz in diesen Sprachen geschrieben war.
Johannes, der das Wort verkündete, welches im Anfang war,
sagt, das das Tier und die Hure Macht haben über die
Zungen und Sprachen, welche dem Wasser gleich seien 
(Offb. 17\bibel{Offb. 17});
man könne also sehen, das das Tier und die Hure diese Macht
haben über die Sprachen, die von der Verwirrung zu Babel 
herrühren. Die Verfolger Christi haben sie dann höher gestellt als
ihn, als sie ihn kreuzigten; aber danach ist er auferstanden, höher
als alles andere, er, der vor allen gewesen ist. \zitat{Gedenkst du
nun,} fragte ich den Mann, \zitat{Prediger Christi zu bilden, 
vermittelst dieser verwirrten äußeren Sprachen, die aus Babel 
kommen und dort gut geheißen und von den Verfolgern Christi gebraucht
wurden!} Er musste das zum Teil zugeben; hierauf zeigten
wir ihm weiter, das Christus seine Prediger selber lehrte, ihnen
Gaben erteilte und sie hieß, den Herrn der Ernte zu bitten, das
er Arbeiter sende. Und Petrus und Johannes, die doch in Sachen
der Schulweisheit\index{Schulweisheit} unwissend und 
ungelehrt waren, verkündeten
Christus, das Wort, das am Anfang war, also auch vor Babel.

\section{Fox bei der Jahresversammlung in Bedforshire}

Auch Paulus hat das Evangelium nicht durch irgend einen
Menschen empfangen, sondern durch Jesus, welcher auch jetzt
derselbe ist, und so ist auch sein Evangelium unverändert. Der
Priester hörte auf uns und errichtete seine Schule nicht.
Wir zogen über Warwickshire\ort{Warwickshire}, 
Northamptonshire\ort{Northamptonshire} und
Leicestershire\ort{Leicestershire}, wo wir überall viele Freunde 
aussuchten, nach Bedforshire\ort{Bedforshire}, zum Hause 
John Crooks,\person{Crooks, John} wo eine allgemeine 
Jahresversammlung\index{Jahresversammlung} für das ganze 
Land abgehalten wurde; sie dauerte
drei Tage, und die Freunde strömten aus dem ganzen Land herzu,
so das alle Herbergen und Wohnungen in der Umgegend überfüllt 
waren. Und trotz einiger Störungen durch einige böse Leute,
die von der Wahrheit abgefallen waren, kam doch die Kraft des
% \picinclude{./130-139/p_s136.jpg} 
Herrn über alle, so das wir eine herrliche Versammlung hatten;
das ewige Evangelium wurde gepredigt, und viele nahmen es
auf, und Leben und unsterbliches Wesen ging auf in allen und
schien über allen [...].

\section{Disput mit einen Jesuiten über die Eucharestie und Glaubenskriege}

Ich suchte noch da und dort etliche Freunde auf und kam
vom Herrn geleitet nach London,\ort{London} (1658\jahr{1658}) [...].
Ich war noch nicht lange dort, als ich hörte, das 
ein Jesuit,\index{Jesuiten}
der mit einem Gesandten von Spanien hergekommen war, alle
Quäker aufgefordert hatte, zu einer Disputation\index{Disput} 
in das Haus des Earl von Newport\person{Earl von Newport} 
zu kommen. Die Freunde antworteten, das
etliche kommen werden. Darauf lies er uns sagen, er wünsche
mit 12 unserer gelehrtesten und weisesten Leuten zu reden; einige
Zeit darauf lies er sagen, es möchten nur 6 kommen, und schließlich
nur 3. Wir beeilten uns, so viel wir konnten, damit es nicht am
Ende nach so vieler Prahlerei heiße, es solle gar niemand kommen.



Als wir hin kamen, hieß ich Nicolas Bond\person{Bond, Nicolas} 
und Edward Burrough\person{Burrough, Edward}
hinausgehen, um die Unterredung zu eröffnen; ich wollte eine
Zeitlang mich unten im Hofe aufhalten und dann nachkommen.
Ich riet ihnen, ihm die Frage zu stellen, ob die römische Kirche,
sowie sie jetzt sei, nicht von der wahren Kirche der ersten Zeiten,
von ihrem Leben und ihrer Lehre, ihrer Kraft und ihrem Geist
abgefallen sei. Diese Frage legten sie denn auch dem Jesuiten sie,
vor. Er erwiderte, die römische Kirche sei jetzt noch in der 
Jungfräulichkeit und Reinheit der ersten Kirche. Da kam ich dazu.
Wir frugen ihn, ob der heilige Geist über sie wie über die Apostel
ausgegossen worden sei? Er antwortete: \zitat{nein!} \zitat{Dann,} sagte
ich, \zitat{wenn nicht derselbe Geist über euch ausgegossen worden ist
und dieselbe Kraft wie über die Apostel, so seid ihr vom Geist
und der Kraft der ersten Kirche abgefallen; weiter braucht dann
nicht viel beigefügt zu werden.} Dann fragte ich ihn, auf was
für Schriftstellen sie sich beriefen bei Errichtung von Nonnen- und
Mönchs-klöstern und Abteien für alle ihre verschiedenen Orden?
und beim Beten mit Rosenkränzen und zu Bildern, und ihrem
Bekreuzen und ihren Verboten wegen allerlei Speisen und beim
Heiraten und bei ihrem Hinrichten um des Glaubens willen?
\zitat{Wenn ihr,} sagte ich, \zitat{die Gebräuche der 
ersten Kirche habt, in
ihrer Reinheit und Jungfräulichkeit, so zeiget uns Schriftworte,
welche beweisen, das sie solches getan.} Wir hatten nämlich
vorher gegenseitig abgemacht, das wir unsere Behauptungen aus-
% \picinclude{./130-139/p_s137.jpg} 
der Schrift beweisen sollten.\index{Absprachen} Er sprach nun 
von einem geschriebenen und einem ungeschriebenen Wort. Ich 
fragte ihn, was er
das ungeschriebene Wort nenne? Er sagte, das geschriebene
Wort sei die Schrift, das ungeschriebene das mündliche Wort der
Apostel, also alle Überlieferungen, nach denen sie wandeln. 
\index{Ungeschriebene Überlieferung} Ich
hieß ihn, das aus der Schrift beweisen. Er kam nun mit der
Stelle, wo der Apostel, 2. Thess. 2:5,
\bibel{Thess. 2. 02:05@2. Thess. 2:5} sagt: \zitat{Gedenket ihr nicht
daran, das ich euch solches sagte, als ich noch bei euch war?}.
\zitat{Damit,} sagte der Jesuit, \zitat{meint der Apostel 
die Klöster und
das Hinrichten um des Glaubens willen und das Beten mit
Rosenkränzen und zu Bildern und was noch mehr der Gebräuche
der römischen Kirche sind. Das ist gemeint mit dem ungeschriebenen 
Wort, das der Apostel damals gesagt und das sich seither
durch Überlieferung bis auf unsere Zeit erhalten hat.} Ich hieß
ihn nun, dieses Wort noch einmal lesen, damit er sehe, wie er es
verdreht hatte. Denn das, wovon er dort den Thessalonichern
sagt, \zitat{das er ihnen zuvor gesagt}, ist nicht ein 
\zitat{ungeschriebenes Wort}, sondern eines, das geschrieben 
ist, nämlich das \zitat{der Mensch
der Sünde, das Kind des Verderbens geoffenbart werde, ehe der Tag
Christi kommen werde} (Thess. 2. 02:03@2. Thess. 2:3). 
Er redete also durchaus nicht von den Gebräuchen der römischen Kirche.
\index{Römischen Kirche} \index{Katholische Kirche} Ähnlich redete
der Apostel im dritten Kapitel dieses Briefes, von \zitat{etlichen argen
Menschen, Vorwitzigen, die nichts treiben}; um deretwillen hatte
der Apostel mündlich gesagt: \zitat{wer nicht arbeiten will, 
soll auch nicht essen}; an das erinnerte er sie nun schriftlich. 
Diese Schriftstelle
bewies also nichts für ihre erfundene Überlieferungen; und eine andere
Stelle konnte er nicht bringen. Ich sagte ihm darum: \zitat{dies ist
eine neue Verirrung eurer Kirche in Überlieferungen und Erfindungen, 
wie die Apostel und die Heiligen der ersten Kirche sie
niemals kannten.}

\index{Abendmahl}\index{Eucharestie}
Er ging nun zum Altarsakrament über; er fing beim Passalamm
und den Schaubroten an und kam zuletzt auf die Worte Jesu:
\zitat{Dies ist mein Leib}, und auf das, was der Apostel darüber an
die Corinther schrieb. Er verstand die Stelle so, das, nachdem der
Priester Brot und Wein geweiht habe, sei dasselbe göttlich und
unvergänglich, und wer es genieße, genieße Christus selber. Ich
folgte ihm bei seiner Aufzählung der Bibelstellen, bis er zu den
Worten Christi und der Apostel kam. Da zeigte ich ihm, wie
derselbe Apostel den Corinthern auch nach ihrem Genus von Brot und
% \picinclude{./130-139/p_s138.jpg} 
Wem gesagt habe, sie seien Verworfene, wenn Christus nicht in
ihnen sei; wenn aber Brot und Wein, die sie genossen, Christus
Leiber Wären, so müsste er ja nun in ihnen sein. übrigens, wenn
Brot und Wein Christi Leib und Blut wären, wie könnte er dann
seinen Leib im Himmel haben? Und zu dem haben die Jünger
Christi Leib essen und sein Blut trinken müssen zu seinem 
Gedächtnis sowohl beim Abendmahl als auch nachher, bis das er
ja klar beweise, das dass Brot und der Wein nicht
sein wirklicher Leib war, denn wenn sie seinen wirklichen Leib 
gegessen hätten, so wäre er ja schon gegenwärtig gewesen, und man
hatte es nicht zu seinem Gedächtnis zu tun brauchen [...].


Was die Worte Christi betreffe: dies ist mein Leib, so nenne
sich Christus auch selbst einen Weinstock 
(Joh. 1:5\bibel{Joh. 01:05@Joh. 1:5}) und eine Tür
(Joh. 1O\bibel{Joh. 1O}); und die Schrift nennt ihn einen Fels; 
ob er darum
äußerlich ein Weinstock, eine Tür, ein Fels sei?  \zitat{Oh} sagte der
Jesuit \zitat{diese Worte mus man auslegen!} \zitat{So mus man auch
die Worte: \emph{die ist mein Leib}, auslegen}, antwortete ich.


Nachdem ich ihm so den Mund gestopft,\person{Fox!Kraftausdruck} 
machte ich ihm folgenden 
Vorschlag: \zitat{Wegen deiner Behauptung, das Brot und
Wein göttlich und unvergänglich und leibhaftig Christus seien und
das jeder, der sie genieße, Christus genieße, last eine 
Zusammenkunft veranstalten zwischen einigen von euch, die der Papst und
die Kardinäle bestimmen sollen, und einigen von uns, und las eine
Flasche Wein und einen Laib Brot bringen und beide in zwei
Teileteilen und den einen dieser Teile weihen. Dann verwahret
sowohl den geweihten als den ungeweihten Teil an einem sichern
Ort, und last sie gut bewachen; und macht den Versuch, ob das
geweihte Brot und der geweihte Wein nicht gerade so schnell
schlecht werden, wie das; ungeweihte Brot und der ungeweihte
Wein [... ] und so wird die Wahrheit über diese Punkte 
offenbar werden. Wenn das geweihte Brot und der geweihte Wein
sich nicht verändern, sondern schmackhaft und gut bleiben, so
werden dadurch viele für eure Kirche gewonnen werden; verändern 
sie sich aber, so müsst ihr nachgeben, euren Irrtum fahren
Essen Und kein Blut mehr darum vergießen. Es ist schon viel
Blut darum vergossen worden, zum Beispiel zur Zeit der Königin
Maria}\person{Königin Maria} Hierauf erwiderte der Jesuit: 
\zitat{Nehmt ein Stück neuen
Stoff und schneidet ihn in zwei Hälften, und machet zwei Röcke
daraus und zieht den einen Rock dem König David an und den
% \picinclude{./130-139/p_s139.jpg} 
andern einem Bettler; beide Röcke werden sich gleichermaßen 
abtragen}. \zitat{Ist das deine Antwort}, fragte ich; \zitat{ja}, 
antwortete er; \zitat{dann}, entgegnete ich, \zitat{werden 
alle Anwesenden überzeugt
sein, das; euer geweihter Wein und euer geweihtes Brot nicht
Christus ist. Habet ihr den Leuten solange 
vorgeschwatzt,\person{Fox!Kraftausdruck} der
geweihte Wein und das geweihte Brot sei Christus, und nun
sagst du, sie verbrauchen sich so gut wie die andern? Ich sage
dir: Christus ist derselbe gestern und heute und verfällt nicht;
sondern er ist der Heiligen himmlische Speise jetzt und zu allen
Zeiten}. Hieraus erwiderte er nichts mehr, sondern hörte gerne
auf, denn die Anwesenden sahen, das er im Irrtum war und sich
nicht verteidigen konnte. 


Ich fragte ihn weiter, warum seine \index{Glaubenskrieg}
Kirche die Leute um des Glaubens willen töte und verfolge?
Er erwiderte: \zitat{nicht die Kirche tue das, sondern die Obrigkeit}.
Ich fragte ihn, ob denn diese Obrigkeit sich nicht auch zu den
Gläubigen und Christen zähle? Er bejahte es. \zitat{Nun also},
sagte ich, \zitat{sind sie denn dann nicht Glieder eurer Kirche?}
\index{Ekklesiologie} Er
antwortete: \zitat{ja}. Daraus überließ ich es den Anwesenden, selber
zu urteilen, ob die römische Kirche nicht die Leute um des Glaubens
willen verfolge und töte. Hiermit trennten wir uns. Seine
Spitzfindigkeiten erklärten sich durch seine Dummheit.

\section{Trostbrief an Lady Elaypole}

Es lag vieles auf mir während der Zeit, da ich in London\ort{London}
war, denn es war eine Zeit großer Not. Es trieb mich, an
Oliver Cromwell\person{Cromwell, Oliver} zu schreiben, 
und ihm die Bedrängnis der Freunde, sowohl in England\ort{England} 
als auch in Irland\ort{Irland} vorzustellen.
Das Gerücht verbreitete sich damals, man wolle Cromwell zum
König machen. Da trieb es mich, zu ihm zu gehen und ihn davor 
zu warnen, wie auch vor mancher anderen Gefahr, die seinen
und seiner Nachkommen Untergang herbeiführen würden, wenn
er sie nicht meide. Er schien alles, was ich ihm sagte, gut 
aufzunehmen, und dankte mir dafür; dennoch trieb es mich, ihm 
nachher noch ausführlicher darüber zu schreiben [...].


Um diese Zeit erkrankte Lady Elaypole
\person{Elaypole, Tochter Cromwells} (die Lieblingstochter 
Oliver Cromwells), und 
war sehr niedergeschlagen, und niemand konnte sie trösten; als 
ich davon hörte, trieb es mich, ihr folgendes
zu schreiben:

\brief{An Lady Elaypole, Tochter Cromwells}{
Freundin!

\bigskip

Sei stille und ruhig in deinem Innern, und frei von eigenem
Denken; dann wirst du das Walten Gottes erfahren, wie es
% \picinclude{./140-149/p_s140.jpg} 
deine Sinne auf den Herrn lenkt, aus welchem das Leben
kommt; und dann wirst du seine Kraft an dir spüren, die dich
stark macht gegen alle Stürme und Unwetter. So allein wirst
du Geduld erlangen, Unschuld, Reinheit, Ruhe, Festigkeit und
Frieden in Gott. Darum lässt dich der Herr ermahnen, dich
seinem Willen zu unterwerfen und Glauben zu haben, damit du
das, was dich bedrückt, überwindest. [...]. 

\index{Inneres Licht}
Ich ermahne alle,
in der Furcht des Herrn zu bleiben, damit ihr die Geheimnisse
Gottes erfahren möget und seine Weisheit und unter dem Schatten
des Allmächtigen sitzet in allen Gefahren und Stürmen. Denn
der Herr ist nahe, und der Höchste regieret die Menschenkinder.
Des Herrn Wort ergehet an alle, das sie in allen Versuchungen
und Verwirrungen, welche das Licht an den Tag bringt, sich nicht
bei diesen Versuchungen und Schlechtigkeiten aufhalten, sondern
auch das Licht sehen, welches sie aufdeckt und an den Tag
bringt; und in diesem Lichte könnet ihr über dies alles steigen
und die Kraft empfangen, ihm entgegen zu treten. Das gleiche
Licht, welches euch die Sünde erkennen lässt, zeigt euch auch den
Bund mit Gott, welcher eure Sünden\index{Sünde} tilgt und euch Sieg gibt
über sie.   Wenn ihr die Versuchung und das Schlechte ansehet,
so wird es euch mitreisen; wenn ihr aber zum Licht ausblickt,
welches die Sünde aufdeckt, so wird es euch überwinden helfen.
Es wird euch den Sieg geben und ihr werdet Gnade und Kraft
von oben erfahren; dies ist der erste Schritt zum Frieden. Ihr
werdet das Heil erlangen und werdet die Herrlichkeit sehen, die
war, ehe der Welt Grund gelegt war; und dadurch werdet ihr
den Samen Gottes erkennen lernen, der das Erbe der 
Verheißungen Gottes ist [...]. 

\bigskip

Also stärke dich der Herr im Namen Jesu Christi.


\begin{flushright}G. F.\end{flushright}
}


Als diese Zeilen Lady Elaypole vorgelesen wurden, sagte sie,
es habe für den Augenblick ihren Geist gestärkt. Später 
verschafften sich viele Freunde in England und Irland Abdrücke
davon und lasen es anderen, die niedergedrückt waren, vor; und
es hat manchem zur Aufrichtung geholfen.

\section{Letzte Begegnung mit Oliver Cromwell}

Um diese Zeit erschien ein Aufruf von Oliver 
Cromwell\person{Cromwell, Oliver} für
eine Kollekte zur Unterstützung der aus Polen\ort{Polen} vertriebenen 
protestantischen Gemeinden \index{Protestanten} und 
für 20 aus Böhmen verbannte
Familien. Schon einige Zeit vorher war ein ähnlicher Ausruf
erlassen worden, der zu einem feierlichen Fast- und Bettag 
% \picinclude{./140-149/p_s141.jpg} 
aufforderte, damit eine Kollekte gemacht würde zum Besten der 
notleidenden Protestanten in den Tälern von 
Lucerne,\ort{Lucerne} Angrona\ort{Angrona} und
an anderen Orten, welche der Herzog von 
Savoyen\person{Herzog von Savoyen} verfolgte.
Es trieb mich, dem Protektor und den obersten Behörden bei
dieser Gelegenheit zu schreiben, um ihnen die Art des wahren
Fastens, das Gott gefällt und von ihm angenommen wird, 
darzulegen und ihnen zum Bewusstsein zu bringen, wie Unrecht sie
tun und sich selbst verdammen, wenn sie die 
Papisten\index{Papisten} tadeln,
das sie die Protestanten in andern Ländern 
verfolgen,\index{Vervolgung} während
sie zu gleicher Zeit ihre protestantischen Nachbarn und die
Freunde im eignen Land verfolgen [...].


Ich begab mich nun nach Hampton Court, um mit dem
Protektor über die Not der Freunde zu reden. Ich traf ihn auf
einem Ritt im Hampton Court Park.\ort{Hampton Court Park} 
Ehe ich ihn an der Spitze
seiner Leibgarde erreichte, spürte ich einen Hauch des Todes
ihm entgegen huschen; und als ich zu ihm kam, sah er aus wie
ein Toter. Nachdem ich ihm die Not der Freunde beschrieben
hatte und nach meinem inneren Trieb ihn gewarnt hatte, hieß
er mich, zu ihm heim kommen. So ging ich denn am folgenden
Tage wieder nach Hampton Court, um nochmals mit ihm zu
reden. Aber als ich kam, hieß es, er sei krank, und Harwey,
einer seiner Bedienten, teilte mir mit, die Ärzte wollten nicht,
das ich mit ihm spreche. So ging ich fort und habe ihn nachher
nie mehr gesehen.

\section{Das Kirchenbekenntnis}


Von hier ging ich zu Isaak Pennington\person{Pennington, Isaak} 
in Buckinghamshire, \ort{Buckinghamshire}
wo ich eine Versammlung angezeigt hatte; und des Herrn 
Wahrheit und Kraft wurden herrlich offenbar unter uns. Nachdem
ich manche Freunde in dieser Gegend besucht hatte, ging ich
nach London\ort{London} und bald darauf nach 
Essex;\ort{Essex} kaum war ich dort,
so hörte ich, der Protektor sei gestorben, und sein Sohn 
Richard\person{Cromwell, Richard}
sei zum Protektor gemacht worden; nun kehrte ich wieder nach
London zurück.


Noch vor dieser Zeit war das sogenannte 
Kirchenbekenntnis\footnote{Die Savoydeklaration der 
Indepedenten von 1658.}
veröffentlicht worden, von dem es hieß, er sei in der Zeit von
11 Tagen gemacht worden. Ich verschaffte mir vor der 
Veröffentlichung eine Abschrift, und schrieb eine Antwort dazu, und
überall wo nun dieses Buch über ihr Bekenntnis verkauft wurde,
% \picinclude{./140-149/p_s142.jpg} 
wurde auch meine Antwort verkauft. \index{Kontroverse} 
\index{Erwiederungsschrift}

Dieses ärgerte etliche der
Parlamentsmitglieder, so das mir einer von ihnen mitteilte, ich
müsste nach Smithfield;\ort{Smithfield} ich antwortete ihm: 
ich stehe über ihrem
Feuer und fürchte sie nicht! Und ich stellte ihm weiter vor, ob
denn alle die vielen Völker seit 1600 Jahren ohne Glauben 
gewesen seien, das die Priester jetzt kommen müssen und ihnen einen
machen? \zitat{Sagte nicht der Apostel, das Jesus der Anfänger und
Vollender des Glaubens war 
(Hebr. 12:2\bibel{Hebr. 12:02@Hebr. 12:2})? Und wenn nun
Christus der Anfänger des Glaubens der Apostel war und des
Glaubens der ersten Kirche\index{Kirche!erste} in den 
ersten Zeiten und des Glaubens
der Märtyrer\index{Märtyrer}, sollten nicht alle Menschen 
zu ihm aussehen als
dem Anfänger und Vollender ihres Glaubents, und nicht zu den
Priestern?} Wir hatten viel Not mit diesem Vekenntnis. [...]

\section{Theatralische Auftritte der Quaker}

Ich ging nach Reading,\ort{Reading} wo ich während etwa zehn Wochen
viel unter schwerer Niedergeschlagenheit und Trübsal zu leiden
hatte. Denn ich sah, wie viel Uneinigkeit und Verworrenheit 
unter den Völkern herrschte und wie die Mächte suchten, sich
gegenseitig aufzufressen. Und ich sah, wie die Unschuld vernichtet
und die Wahrheit verleugnet wurde. Heuchelei, Betrug und
Streit gewannen die Oberhand, so das man überall bereit war,
sich gegenseitig das Schwert durch die Brust zu stoßen. Viele
waren empfänglich gewesen, als sie noch niedrig gewesen waren;
nachdem sie aber empor gekommen waren, und Macht erlangt
hatten und geholfen, andere zu töten, wurden sie bald so
schlecht wie die übrigen, so das wir oft mit ihnen in Streit 
gerieten wegen unsrer Hüte\index{Hut} und wegen des 
\zitat{Du}~--~sagens.index{Anrede} Sie
kehrten ihre zur schau gestellte Geduld und Mäßigkeit in Zorn und
Ungeberdigkeit, und viele von ihnen taten wie Wahnsinnige wegen
dieser Hutehre. Denn sie waren durch die Verfolgung der 
Unschuld verhärtet worden, und kreuzigten nun den Samen,
Christus,\index{Christus!innerlich kreuzigen} in sich und 
in andern; bis sie schließlich anfingen, sich
untereinander zu beißen und auszuzehren, nachdem sie das, was
Gott in ihnen hatte ausgehen lassen, beleidigt und zerstört hatten.
Darum stürzte sie Gott bald und machte die Hohen niedrig, und
stellte den König über die, die so oft behauptet hatten, die Quäker
kommen zusammen, um die Rückkehr König Karls\person{König Karl} 
zu beraten,
während doch die Freunde sich nie um die äußern Mächte und
Regierungen bekümmert hatten.\index{Politik} Zuletzt hat 
Gott ihn dann zurückgebracht, und viele, als sie sahen, 
das er doch kommen werde,
% \picinclude{./140-149/p_s143.jpg} 
stimmten für sein Kommen. So preiset nun Gott mit Herz und
Mund, der die Herrschaft hat über alles [...]. 

Ich ahnte die
Rückkehr des Königs voraus, und so taten manche andere. Ich
schrieb mehrere Male an Oliver\person{Cromwell, Oliver} 
um ihm zu sagen, das, während
er das Volk Gottes verfolge, seine Feinde sich rüsten, ihn zu
stürzen. Als einige Voreilige unter uns 
Somerset House\ort{Somerset House} kaufen
wollten, um Versammlungen drin zu halten, verbot ichs ihnen;
denn ich sah die Rückkehr des Königs voraus. Sodann kam eine
Frau zu mir, welche eine Vorahnung von der Rückkehr des
Königs gehabt hatte, drei Jahre, ehe er wirklich kam; und 
erklärte mir, sie müsse hingehen und es ihm sagen. Ich riet ihr,
das dem Herrn zu überlassen und es für sich zu behalten; denn
wenn es entdeckt würde, in welcher Angelegenheit sie hingehe, so
würde man es als Verrat ansehen; sie beharrte darauf, sie müsse
zu ihm gehen und ihm sagen, das er wieder nach England 
zurückkehren werde. Da erkannte ich, das ihre Vorahnung sich
erfüllen werde; denn es musste ein schwerer Schlag die treffen,
die damals so große Macht hatten und so harte Verfolgungen
ausübten; sie hielten sich für heilig und nahmen doch den Freunden
ihre rechtmäßigen Besitzungen, weil sie nicht schwören wollten.


Oft wenn wir Oliver\person{Cromwell, Oliver} diese 
Dinge berichteten, wollte er sie nicht
glauben. Darum trieb es Thomas Aldam\person{Aldam, Thomas} 
und Anthony Pearson,\person{Pearson, Anthony}
in alle Kerker von ganz England zu gehen und Aufzeichnungen
zu machen darüber, wie die Freunde von den Kerkermeistern
behandelt wurden, damit sie die Größe ihrer Leiden Oliver 
vorbringen könnten. Und als er dennoch keinen Befehl geben wollte,
sie frei zu lassen, trieb es Thomas Aldam seine Mütze vom Kopf
zu nehmen, sie vor Olivers Augen in Stücke zu zerreißen und zu
rufen: \zitat{Also soll auch deine Herrschaft von dir und deinem Hause
gerissen werden}. Eine Frau,\index{Quaker!Frau} die auch zu den Freunden 
gehörte, trieb es, ins Parlament zu gehen, welches den Freunden
übel wollte, mit einem Krug in der Hand, den sie vor ihnen 
zerschlug und rief: \zitat{so sollt ihr in Stücke zerschlagen 
werden!} was auch bald darauf geschah. 

\section{Zwischen den Konfliktparteien}

Während meiner großen 
Niedergeschlagenheit\person{Fox!Niedergeschlagenheit} 
und inneren Prüfung, die ich um meines Landes willen
zu erdulden hatte, weil die große Heuchelei, Falschheit und 
Verräterei mich schwer drückte, sah ich, das Gott die, welche jetzt
unten waren, über die, welche jetzt oben waren, erhöhen werde,
und das alle sich dem, das sie bekehren konnte, verneigen müssen,
% \picinclude{./140-149/p_s144.jpg} 
ehe sie Herr werden würden über den bösen Geist, nach innen
und nach außen. Denn nur der eine unsichtbare Geist kann und
wird die Heuchelei in den Menschen vernichten [...].


Das ganze Land war in Zwiespalt und großer Aufregung;
die verschiedenen Parteien zankten sich beständig untereinander
und rotteten sich gegeneinander zusammen, weil jede ihre eigenen
Interessen durchsetzen wollte. Da ich in großer Sorge war, das
die Jungen und Unerfahrenen unter uns diesen Versuchungen 
erliegen werden, trieb es mich, allen diesen folgendes zu schreiben:

\brief{Brief Gegen Komplotten und Gewallt}{
    Ihr Freunde allenthalben! 

    \bigskip

    Hütet euch vor Komplotten \index{Gewalt}
    und Wühlereien, und vor dem Arm des Fleisches, denn alle
    diese Machthaber sind gefallene Söhne Adams; sie richten der
    Menschen Leben zugrunde, wie Hunde, Schweine und andere
    Tiere sich zugrunde richten, sich beißen und zerreißen. Wie 
    entstand das Streiten und Töten anders als aus der Lust? Und
    dies alles kommt vom gefallenen Adam her, nicht von demjenigen 
    Adam aber, der nicht fiel, in welchem Leben und Frieden ist
    (1. Cor. 15\bibel{Cor. 1. 15@1. Cor. 15}). Ihr seid zum 
    Frieden berufen, darum jaget ihm
    nach, und dieser Frieden ist in Christus und nicht in dem gefallenen
    Adam. Alle, die jetzt vorgeben, für Christus zu kämpfen, betrügen
    sich; denn sein Reich ist nicht von dieser Welt; darum kämpfen
    seine Diener nicht. Die Streitenden gehören nicht zu seinem
    Reich, denn sein Reich ist Frieden und Gerechtigkeit [...]. Ihr,
    die ihr Erben seid des Evangeliums des Friedens, welches 
    gewesen, ehe der Satan\index{Teufel}\index{Satan} war, 
    lebet in diesem Evangelium, suchet den
    Frieden und das Gute für alle, und lebet in Christus, der 
    gekommen ist, die Seele der Menschen vom gefallenen Adam zu
    erlösen; das äußere Schwert der Juden,\index{Juden} mit dem 
    sie die Heiden\index{Heiden} umbrachten, war ein Sinnbild 
    des inwendigen Geistes Gottes, der die inwendige heidnische 
    Natur\index{Natur!heidnische} tötet. So lebet denn im
    friedsamen Reich Jesu Christi,\index{Reich Gottes} im Frieden 
    Gottes und nicht in den Lüften, aus denen der Krieg entsteht 
    [...] und suchet das
    Wohl und Gedeihen für alle Menschen.

    \bigskip
    \begin{flushright}
    G. F.\end{flushright}

}

Bald darauf ergriff George Booth\person{Booth, George} in 
Cheshire\ort{Cheshire} die Waffen
und Lanibert\footnote{Lambert war einer der bedeutendsten 
Generäle aus der Partei Cromwells.} zog gegen ihn; daraufhin 
wollten etliche Hitzköpfe,\index{Quaker!Hitzköpfe}
wie solche zuweilen unter uns waren, auch die Waffen ergreifen,
aber der Herr trieb mich, sie zu ermahnen und sie blieben ruhig.
% \picinclude{./140-149/p_s145.jpg} 


Zur Zeit des sogenannten \index{Wehrdienst}
Sicherheitsausschusses\index{Sicherheitsausschusses} forderte man uns
auf, die Waffen zu nehmen und manchem von uns wurden hohe
Stellen und Kommandos angeboten, aber wir schlugen sie alle aus
und traten mündlich und schriftlich dagegen auf, indem wir 
erklärten, unsere Waffen und Rüstungen seien nicht fleischlich, sondern
geistlich, und damit keiner unter uns in diese Falle gehe, kam es
über mich vom Herrn, bei dieser Gelegenheit einige Zeilen der
Ermahnung an alle zu schreiben [...].


Nachdem ich längere Zeit in London\ort{London} verweilt hatte, zog ich
wieder in den Grafschaften umher, durch Essex\ort{Essex} 
und Suffolk\ort{Suffolk} nach Norwich\ort{Norwich} [...] und von 
da durch Huntingdomshire\ort{Huntingdomshire} und 
Cambridgeshire\ort{Cambridgeshire} wieder nach London, 
gerade als General Monk\footnote{General Monk, 1660 
Genetalleutnant der Republik und nachher eifrig
bemüht für die Rückkehr Karl II.}\person{General Monk} dort
eingezogen war und die Tore und Befestigungen der Stadt
fielen. Lange vorher hatte ich 
ein Gesicht\index{Fox!Vision}\index{Vison} gehabt, in welchem
ich die Stadt in Trümmer und die Tore eingestürzt gesehen
hatte, gerade so, wie ich sie nun mehrere Jahre später nach dem
Brande sehen sollte.
%%%%%%%%%%%%%%%%%%% Kapitel 13. %%%%%%%%%%%%%%%%%%%%%%%%%%%%%%

\chapter[Ein Gottesgericht.]{Ein Gottesgericht.}

\begin{center}
\textbf{Ein Gottesgericht. Ermahnung zur Barmherzigkeit bei 
Schiffbrüchen. Qnakerfreundlicher Erlass des General Monk. 
Fox als Königsfeind
gefangen und schließlich auf Befehl Karls II. befreit.}
\end{center}

\section{Provokationen und Störungen in den Versammlungen}

Als ich nun meine Arbeit in London getan, ging ich nach
Surrey und Sussex,\ort{Sussex} [...] dann nach Hampshire,
\ort{Hampshire} Dorsetshire,\ort{Dorsetshire}
Ringwood\ort{Ringwood} und Poole,\ort{Poole} wo ich überall 
Freunde besuchte und große
Versammlungen unter ihnen hatte.
In Dorchester hatten wir eine große Abendversammlung in
unserer Herberge,\index{Versammlung!in Herberge} bei der 
viele Soldaten\index{Soldaten} zugegen waren, die alle
ziemlich anständig waren. Aber da erschienen die Wachen und
Schutzleute der Stadt unter dem Vorwand, sie müssten einen 
geschorenen Jesuiten suchen und verlangten, das alle ihre Hüte 
abnähmen,\index{Hut!abnehmen} oder sie würden sie abnehmen, 
um die Tonsur des Jesuiten\index{Schikanen}
zu finden. So nahmen sie mir den Hut ab und untersuchten
mich genau, denn mich hatten sie im Verdacht; aber als sie keine
kahle oder geschorene Stelle fanden, gingen sie beschämt fort. Und
% \picinclude{./140-149/p_s146.jpg} 
die Soldaten und andere Leute ärgerten sich sehr über sie. Aber
es förderte die Sache Gottes und alles diente zum Guten, denn
es machte den Leuten Eindruck, und nachdem die Beamten fort
waren, hatten wir eine schöne Versammlung, und viele wurden
zum Herrn Jesus bekehrt, ihrem Lehrer, der sie erkauft hat und
sie versöhnen will mit Gott. 


Von da gingen wir nach Somersetshire, wo die Presbyterianer 
\index{Presbyterianer} und andere \textit{Fromme} sehr böse
waren und oft die Versammlungen der Freunde störten. Einmal
\index{Versammlung!Störung} \index{Versammlung!Provokation}
hatten sie einen sehr schlechten Menschen dazu veranlasst, 
in eine Versammlung der Quäker zu gehen und eine Bärenhaut 
anzuziehen und Unsinn zu treiben. Er setzte sich gerade dem Freund,
der redete, gegenüber mit seiner Bärenhaut über dem Rücken
und streckte die Zunge heraus und machte eine große Unruhe.
Aber ein schweres Gericht kam über ihn, und seine Strafe schlummerte
nicht; als er aus der Versammlung heim ging, kam er an einer
Stierhetze vorüber und blieb stehen, um zuzusehen; als er aber
nahe bei dem Stier war, stieß dieser dem Mann das Horn in
den Hals, so das seine Zunge heraus hing, gerade wie er es 
vorher in der Versammlung gemacht hatte. Und der Stier stieß
sein Horn durch den Kopf des Mannes hindurch, und schwang
ihn schrecklich in der Luft herum. So kam er, der dem Volke 
Gottes hatte Schaden zufügen wollen, selber zu Schaden, und
es wäre gut, wenn solche Beispiele der Rache Gottes andere lehren
würden, sich zu hüten [...]. \index{Gott!Rache} \index{Strafe Gottes}

\section{Appell zur Rettung Schiffbrüchiger}

Wir gingen durch Somersetshire,\ort{Somersetshire} 
Plymouth\ort{Plymouth} und Devonshire\ort{Devonshire}
nach Cornwall\ort{Cornwall} [...]. Während ich hier war, 
geschahen große Schiffbrüche in der Nähe von Landsend. 
Nun war es Brauch \index{Schiffbruch}
in jener Gegend, das bei einem solchen Anlass reich und arm
hinaus ging, um so viel wie möglich von den Überresten an
sich zu bringen, unbekümmert um die Rettung der Menschen. An
einigen Orten nannten sie sogar einen Schiffbruch eine 
Gottesgnade.\index{Gottesgnade} Es betrübte mich, von 
solchem unchristlichen Treiben zu
hören und zu sehen, wie tief diese Leute unter den Heiden von
Melite stehen, die Paulus aufnahmen, ihm ein Feuer machten
und freundlich waren gegen ihn und die anderen Schiffbrüchigen
(Act. 28\bibel{Act. 28}). Darum trieb es mich, ein Schreiben 
an alle Gemeinden,
Priester und Behörden zu senden, um sie wegen ihres habsüchtigen
Treibens zu tadeln und sie zu ermahnen, so oft sie könnten, mit
Eifer behilflich zu sein, wenn es gelte, Menschenleben zu retten
% \picinclude{./140-149/p_s147.jpg} 
und Schiffe und Waren zu schützen; auch sollten sie auch bedenken, 
wie grausam es ihnen vorkommen würde, wenn sie selber
Schiffbruch litten und die Leute suchen würden, ihnen soviel wie
möglich zu rauben, ohne sich um ihre Rettung zu kümmern [...].
Diese Schrift hatte viel Erfolg beim Volk; die Freunde bemühten
sich um die Rettung der Schiffbrüchigen und den Schutz der
Schiffe und der Habe, ja, Freunde haben Schiffbrüchige, die halb
tot und Verhungert waren, bei sich aufgenommen und sie gepflegt
und unterstützt; alle wahren Christen sollten so handeln [...].

\section{Fox wird erneut verhaftet}

Die Soldaten, die unter dem Befehl des General 
Monk\person{General Monk} standen, waren damals oft sehr grob 
und störten die Versammlungen\index{Versammlung!Störung} 
der Freunde an manchen Orten. Als man sich darüber
beim General Monk beklagte, erließ er folgenden Befehl, worauf
es etwas besser wurde:

\grosszitat{
\begin{flushright}St. James, 9. März 1659.\end{flushright}
Ich will, das alle Offiziere und Soldaten sich hüten, die
friedlichen Versammlungen der Quäker zu stören, da sie nichts
tun, das dem Parlament oder dem Commonwealth von England 
zuwider ist. 
\bigskip
George Monk
}


Wir gingen [...] über Oldeston\ort{Oldeston} [...]
Nailsworth\ort{Nailsworth} [...] Drayton\ort{Drayton} [...]
Lancaster\ort{Lancaster} nach Swarthmore.\ort{Swarthmore} 
Ich war noch nicht lange dort, als Henry 
Porter,\person{Porter, Henry} ein Friedensrichter, einen Verhaftbefehl
sandte, um mich zu greifen. Ich hatte dies vorausgefühlt, und
so kam denn auch, während ich mit Richard 
Richardson\person{Richardson, Richard} und
Margaret Fell\person{Fell, Margaret} zusammen im Zimmer saß, 
ihre Dienerschaft und
meldete, es seien einige da, die durchsuchten das Haus, 
angeblich um zu sehen, ob Waffen darin seien. Es kam über mich, zu
ihnen hinaus zu gehen, und als ich an einem von ihnen vorüber
ging, redete ich ihn an, woraus sie mich nach meinem Namen
fragten; ich sagte ihn ohne weiteres, worauf sie mich ergriffen
und sagten, ich sei gerade der Mann, den sie suchten. Und sie
führten mich fort nach Ulverstone\ort{Ulverstone} [...]. 
Von da brachten sie mich nach Lancaster\ort{Lancaster} [...].


Als ich dorthin kam, war das Volk
sehr aufgeregt; ich blieb stehen und sah sie fest an, und sie
schrien: \zitat{seht diese Augen!} Nach einer Weile redete ich mit
ihnen, und da waren sie ziemlich ruhig. Ein junger Mann nahm
mich mit in seine Wohnung, und nach einiger Zeit kam ein 
Beamter und brachte mich zu Major Porter,\person{Major Porter} 
der den Befehl gegen mich erlassen hatte; es waren noch ein 
paar andere bei ihm. Als
% \picinclude{./140-149/p_s148.jpg} 
ich hereinkam, sagte ich: \zitat{Friede sei mit euch}. Porter fragte
mich, warum ich in dieser unruhigen Zeit hierher komme? Ich
erwiderte: \zitat{Um meine Mitmenschen zu besuchen}. \zitat{Ihr habt
überall herum große Versammlungen}, sagte er; ich erwiderte
ihm, diese Versammlungen seien aber im ganzen Lande als 
friedliche und ruhige bekannt, und wir seien ein friedliches Volk. Er
sagte: \zitat{Ihr seht aber den Teufel den Leuten im 
Gesicht geschrieben}.\index{Teufel}\index{Fox!Sieht den 
Teufel in Anderen} Ich erwiderte: 
\zitat{Wenn ich einen Trunkenbold oder
einen Schwörer oder einen Grobian sehe, so kann ich doch nicht
sagen, ich sehe den Geist Gottes in ihm.}\index{Geist Gottes} 
Und ich fragte ihn, ob. \index{Alkohol}
er den Geist Gottes sehen könne? Er sagte, wir treten gegen
ihre Prediger auf. Ich antwortete: \zitat{Als wir noch wie Saulus waren
und unter den Priestern saßen, da hat man uns nicht schädliche
Männer genannt} (Act. 24:5\bibel{Act. 24:05@Act. 24:5}) oder Sektenmacher; aber als wir
anfingen, Gott zu leben, so wurden wir schädliche Leute genannt
wie Paulus. 


Er sagte, wir könnten recht gut reden, er wolle
lieber nicht mit uns disputieren, aber greifen wolle er uns lassen. Ich
fragte ihn, warum und auf wessen Befehl er einen Verhaftbefehl
gegen mich ergehen lasse, und beklagte mich über die Behandlung
der Beamten bei meiner Gefangenschaft und auf dem Wege
hierher. Er hörte nicht auf mich, sondern sagte, er habe einen
Befehl, aber er wolle mich ihn nicht sehen lassen, denn er
wolle die Geheimnisse des Königs nicht preisgeben; und überdies 
brauche ein Gefangener nicht zu wissen, warum er verhaftet 
sei. Ich sagte ihm, das sei unvernünftig, wie der Gefangene 
sich denn dann verteidigen solle? er solle mir eine Abschrift
geben. Er sagte, es sei einmal ein Richter bestraft worden, weil
er einem Gefangenen den Verhaftbefehl gezeigt habe [...] und
er sagte mir, ich sei ein Friedenstörer im Land. Ich sagte, ich
sei im Gegenteil ein Segen für das Land durch die Kraft und
die Wahrheit des Herrn, und der Geist Gottes in den Gewissen
gebe Zeugnis hiervon. Dann beschuldigte er mich, ich sei ein
Feind des Königs und beabsichtige einen neuen Krieg anzustiften
und neues Blutvergießen über das Land zu bringen. Ich erklärte
ihm, ich habe nie die Gebräuche des Krieges gelernt und sei in
diesen Dingen so unwissend wie ein Kind. Da kam der Schreiber
mit dem ausgefertigten Verhaftbefehl und der Kerkermeister wurde
gerufen, und er musste mich ins Loch tun und niemand durfte
mich besuchen. Dort sollte ich nun gefangen bleiben, bis mich der
% \picinclude{./140-149/p_s149.jpg} 
König oder daß Parlament frei sprechen würde [...]. Ich ließ
nun Thomas Eummins und Thomas Green bitten, zum Gefangen-
wärter zu gehen und ihn um eine Abschrist des Verhaftbefehlß zu
bitten, damit ich wisse, warum ich verurteilt sei. Sie gingen hin;
der Wärter sagte, er könne ihnen die Abschrift nicht geben, weil
einmal einer bestraft worden, der dies getan, aber sie könnten sie
durchlesen. Soviel sie sich nachher erinnerten, lautete die An-
klage also: daß sich- im Verdacht stehe, ein Störer dez Land-
friedenö zu sein, ein Feind des-’ Königö und eine Hauptstütze der
Quäker-Sekte, und daß ich, zusammen mit andern dieser Fana-
tiker, kürzlich versucht habe, Aufstände in dieser Gegend anzu-
stiften und daß Land in Blut zu tauchen. Darum müsse der
Kerkermeister mich in sicherem Gewahrsam behalten, biz ich auf
Befehl des Königs; oder Parlamenteß befreit würde.
Alß mir nun meine Anklage in der Hauptsache bekannt war,
schrieb ich eine kurze Erwiderung, um meine Unschuld zu zeigen;
sie lautete:

\bigskip \begin{quote}

  Ich bin Gefangener in Lancaster, durch Friedenßrichter
  Porter verhaftet. Jch kann keine Abschrift der Anklage erhalten;
  doch erfahre ich, daß sie Behauptungen enthält, die durchauö un-
  richtig sind, z. B. daß ich im Verdacht stehe, ein Friedenstörer
  zu sein und ein Feind deö Königß, und daß ich versuche, mit
  andern zusammen Aussiände anzustiften und Blutoergießen überß
  Land zu bringen; das ist gänzlich falsch und ich bestreite etz. GS
  trifft mich keinerlei Verdacht, ein Friedenstörer zu sein; denn
  ich bin über jeden dieser Punkte schon früher oerhört worden, im
  ganzen Land herum. In den Tagen Cromwellß bin ich gefangen
  genommen worden, weil etz hieß, ich habe die Waffen gegen ihn
  ergriffen, waö falsch war, denn ich habe überhaupt nie Waffen
  getragen; dennoch wurde ich alß Gefangener nach London ge-
  bracht und vor ihn geführt; dort bewies ich ihm meine Umschuld
  und daß ich ja überhaupt gegen daß Gebrauchen irgend einer
  fleischlichen Waffe sei, da meine Waffen geistliche seien, solche,
  die die Ursachen deö Kriegeß hinwegnehmen und zum Frieden
  führen. Daraufhin sprach mich Oliver frei. Darnach wurde ich
  gefangen durch Major Ceely in Cornwall mid ins Gefängniß
  gebracht; er behauptete vor Gericht, ich hätte ihn beiseite ge-
  nommen und ihm gesagt, ich könne in Zeit einer Stunde vierzig-
  tausend Mann stellen, um daß Land in Blutoergießen zu stürzen
  % Erster Durchlauf
  % \picinclude{./150-159/p_s150.jpg} 
  und den König Karl\person{König Karl} zurück zu bringen. Das war alles ganz
  falsch und eine Lüge, die er selber erfunden, wie ihm auch bewiesen 
  wurde. Ich harte nie so etwas zu ihm gesagt; ich hatte
  mich nie an einer Verschwörung beteiligt; ich hatte nie einen Eid
  geschworen, nie Kriegsübungen\indexname{Kriegsübungen} gemacht. Wie jenes falsche 
  Anschuldigungen gewesen, so sind es jetzt die, die Major Porter 
  vorgebracht [...] Ich bin kein Störer des Landfriedens, sondern
  ich suche den Frieden aller Menschen [...] Und ebenso ist es
  falsch, wenn Major Porter\person{Major Porter} sagt, ich sei ein Feind des Königs,
  denn ich liebe ihn und alle Menschen, wenn sie schon Feinde
  Gottes sind und ihre eigenen Feinde und meine. Ich weiß, das
  seine Rückkehr vom Herrn kommt, damit er viel begangenes Unrecht 
  wieder gut mache. Ich hatte ein Gesicht davon, drei Jahre
  ehe er zurück kam. Es ist eigentümlich, zu sagen, ich sei ein
  Feind des Königs; ich habe keinerlei Grund es zu sein, trotzdem
  ich allerdings viel verfolgt und eingesperrt gewesen bin während
  der letzten 11 oder 12 Jahre, von den Gegnern sowohl des
  jetzigen Königs als seines Vaters, also eben von der Partei, die
  Porter zum Major gemacht und für die er die Waffen führte,
  aber nicht durch die, die für den König war. Ich war nie ein
  Feind des Königs, noch irgend eines andern auf Erden. Ich
  habe die Liebe, die des Gesetzes Erfüllung ist, die nichts Böses
  denkt, sondern sogar die Feinde liebt, und möchte, das der König
  errettet würde und die Wahrheit erkennete und dazu käme, Gott
  zu fürchten und die Weisheit von oben zu erlangen, durch die
  alle Dinge gemacht sind, damit er in dieser Weisheit regierete zur
  Ehre Gottes [...].

  \medskip 

  Weil ich nun hier gefangen bin, bis ein Befehl vom König
  oder dem Parlament mich frei macht, so habe ich solches geschrieben, 
  damit ihr und der König und das Parlament es
  leset, und alles bedenket, ehe ihr etwas in der Sache tut; und
  in der Weisheit Gottes untersucht, was für Absichten zugrunde 
  liegen, damit ihr nicht etwas tut, womit ihr die Hand
  des Herrn gegen euch wendet, wie viele Machthaber zuvor getan,
  die dann gestürzt wurden von dem Gotte, den wir fürchten und
  dem wir trauen und zu dem wir Tag und Nacht schreien, und
  der uns gehöret hat und noch erhört und uns rächen wird. Viel
  unschuldig Blut ist schon vergossen worden, und viele sind bis in
  den Tod verfolgt worden durch die, die vor euch die Herrschaft
  % \picinclude{./150-159/p_s151.jpg} 
  hatten; und Gott hat sie ausgespien, weil sie sich gegen das
  Recht kehrten. Darum prüfet, wie es um euch steht, solange es
  Tag ist, und nehmet dieses auf als eine Warnung in Liebe
  an euch.\grqq{}

  \medskip 

  Von einem der unschuldig in Lancaster gefangen liegt, genannt
  \begin{center}George Fox [...]\end{center}

\end{quote} \bigskip

Bald darauf gab ich eine Schrift gegen das Verfolgen heraus:

\brief{Verfolger}{
  Die Papisten\index{Papisten}, die 
  Common-Prayerleute\index{Common-Prayerleute}, die 
  Presbyterianer\index{Presbyterianer}, Independenten\index{Independenten} 
  und Baptisten\index{Baptisten} verfolgen einander um ihrer eigenen
  Erfindungen willen, ihren Messen, ihren Common-Prayer Bücherm,
  ihrem \zitat{Directory} und Bekenntnis, dies sie aufgesetzt haben,
  aber nicht zum Nutzen der Wahrheit; denn sie wissen nicht, wes
  Geistes Kind sie sind, wenn sie verfolgen und die Leben der Menschen
  zu zerstören suchen um des Kirchendienstes und der Religion
  willen, während Christus sagte, er sei nicht gekommen, das
  Leben der Menschen zu zerstören, sondern es zu retten (Luc. 9\bibel{Luc. 09@Luc. 9}).
  Wir können uns doch nicht solchen anvertrauen, die nicht wissen,
  wes Geistes Kind sie sind [...] Ihr möchtet gerne ein Gebot
  haben, um zu zerstören, wie einst die Jünger wollten Feuer vom
  Himmel regnen lassen, um die, welche Christus nicht aufnehmen
  wollten, zu zerstören [...] Die, welche das Leben der Menschen
  zerstören, sind nicht Jünger Christi, des Heilands, [...] wenn ihr
  die Leben anderer zerstört und verfolget und nicht Buße tut, werdet
  ihr nicht auferstehen zum Leben mit Gott. Die aber, die wissen
  wes Geistes Kinder sie sind, die haben den untadeligen Eifer und
  geben durch den Geist Gottes dem Herrn Leib, Seele und Geist,
  die sein sind, das er sie bewahre. [...]. 

  \bigskip 

  \begin{flushright}G. F.\end{flushright}

}

Es trieb mich auch, an den König zu schreiben, um ihn zu
ermahnen, Barmherzigkeit zu üben gegen seine Feinde und der
Zügellosigkeit und Gottlosigkeit, die bei seiner Rückkehr im Lande
aufgekommen war, zu steuern.

\brief{König Karl}{
  \begin{center}An den König:\end{center}
  \medskip 

  O König Karl,
  \medskip 

  Du kamst nicht ins Land durch Schwert noch durch Sieg im
  Kriege, sondern durch die Kraft des Herrn; wenn du nun nicht
  in derselben lebest, so wirst du nicht gedeihen. Wenn der Herr
  dir Barmherzigkeit erzeigt hat und dir vergeben hat, und du
  übest nun nicht auch Barmherzigkeit und Vergebung, so wird der
  % \picinclude{./150-159/p_s152.jpg} 
  Herr deine Gebete nicht erhören, noch die Gebete derer, die für
  dich beten. Wenn du nicht den Verfolgungen Einhalt gebietest
  und nicht die Gesetze, welche das Verfolgen um des Glaubens
  willen gestatten, abschaffst [...] so wirst du so blind werden
  wie deine Vorgänger; denn das Verfolgen hat immer die Verfolger 
  blind gemacht. Solche aber stürzet Gott durch seine Macht
  und verfährt streng mit ihnen; aber den Bedrückten schickt er
  Rettung. Wenn du das Schwert umsonst trägst und lässest
  Trunkenheit, Schwören, Spielen und dergleichen eitles Treiben
  ungestraft, wie z.B. das Aufstellen der Maibäume mit dem Bild
  der Krone oben drauf und dergleichen, so wird das Land bald
  sein wie Sodom und Gomorra und so schlecht, wie die alte Welt,
  die den Herrn so betrübte, das er sie untergehen ließ. So wird
  er auch euch tun, wenn ihr solche Dinge nicht abschafft. Es hat
  kaum je solche Freiheit, Unrecht zu tun, geherrscht wie jetzt, als
  ob es nicht Gewalt noch Schwert der Obrigkeit mehr gäbe; und
  solches ist weder der Regierung noch denen, die recht tun, zum
  Nutzen. Wir beten für die, welche die Herrschaft haben, das
  wir ein ruhiges Leben unter ihnen führen können, in Frieden
  und Gottseligkeit, und das wir nicht durch sie in Gottlosigkeit
  fallen. Höre und denke darüber nach und tue Gutes, so lange
  du kannst und Macht hast. Sei barmherzig und vergib; dies
  ist der Weg, aus dem du überwindest und das Reich Christi
  erlangst.

  \medskip 

  \begin{flushright}G. F.\end{flushright}
}

Es ging lange, ehe der Sheriff einwilligte, mich nach London\ort{London}
überzuführen, es sei denn, das ich die Kosten trage, was ich 
verweigerte. Schließlich, als sie sahen, das es nicht anders ging,
gab der Scheriff zu, das ich mit einigen Freunden nach London
gehe, ohne andere Verpflichtungen als mein Versprechen, an
dem und dem Tage vor den Richtern in London zu erscheinen,
so der Herr es zulasse, woraus ich entlassen wurde [...]
Etwa drei Wochen nach meiner Freilassung gelangte ich nach
London. Als wir nach Charing Cros\ort{Charing Cros} kamen, war dort eine
ungeheure Menschenmenge versammelt, um zu sehen, wie die
Überreste einiger der früheren Richter des Königs verbrannt
wurden, die erhängt, ertränkt und gevierteilt worden waren [...].

Als wir den Richtern die gegen uns gerichtete Anklage eingereicht 
hatten, und sie die Worte lasen: meine Freunde und ich
trachteten Blutvergießen im Lande anzurichten, schlugen sie mit
% \picinclude{./150-159/p_s153.jpg} 
der Hand auf den Tisch; ich erklärte, das ich der Mann sei,
gegen den diese Anklage gehe, aber ich sei an allem derartigen
so unschuldig, wie ein neugeborenes Kind, und habe die Anklage
selber hierher gebracht, und meine Freunde seien ohne Wache mit
mir gekommen. Sie hatten bis jetzt meinen Hut noch nicht
beachtet, aber jetzt fiel er ihnen auf und sie fragten: \zitat{Was! ihr
steht hier im Hut?} Ich sagte ihnen, es geschehe keineswegs
aus Mangel an Achtung vor ihnen. Darauf befahlen sie, das
man mir ihn abnehme, und dann riefen sie den Marschall von
Kings-Bench und sagten zu ihm: \zitat{Ihr müsst diesen Mann in
Gewahrsam bringen, gebt ihm aber ein Zimmer und legt ihn nicht
unter die anderen Gefangenen.} Als der Marschall erklärte, er
habe kein Zimmer frei, das er mir geben könnte, so fragten sie
mich: \zitat{Wollt ihr morgen um zehn in Westminsterhall vor 
Kings-Bench erscheinen?} Ich antwortete: \zitat{Ja, wenn der Herr mir die
Kraft gibt.} Hierauf sagte Richter Foster\person{Richter Foster} zum anderen Richter:
\zitat{Wenn er sagt ja und es verspricht, so könnet ihr auf sein Wort
gehen.} Somit war ich entlassen. Am nächsten Tage erschien
ich zur bestimmten Stunde vor Kings-Bench\ort{Kings-Bench} [...] und als ich
eintrat, trieb es mich zu sagen: \zitat{Friede sei mit euch,} und die
Kraft des Herrn kam über alle. Meine Anklage wurde öffentlich
verlesen. Die Leute waren ziemlich still und die Richter ruhig
und freundlich, die Gnade des Herrn war mit ihnen. Aber als
sie an die Stelle kamen, wo es hieß, ich und meine Freunde
wollten Blutvergießen über das Land bringen und einen neuen
Krieg anstiften, und das ich ein Feind der Königs sei, da hoben
sie ihre Hände auf. Da erhob ich meinen Arm und sagte: \zitat{Ich
bin der Mann, gegen den sich diese Anklage richtet, aber ich bin
so unschuldig wie ein neugeborenes Kind in dieser Sache, und
habe nie den Gebrauch der Waffen gelernt. Und meint ihr, wenn
ich oder meine Freunde solche wären, wie es in der Anklage
heißt, so hätten wir unsere Anklage selber hierher gebracht?} [...]
Sie fragten mich, was man mit der Anklage tun solle? ich sagte:
\zitat{Ihr seid die Richter und könnt hoffentlich in dieser Sache
richten; tut also was ihr wollt. [...]}

Sie sagten, sie wollten mich nicht verurteilen, denn sie hätten
nichts gegen mich. Esquire Marsh\footnote{Esquire Marsh, eine 
angesehenen Persönlichkeit am Hofe Karl II, war den
Ouäkern geneigt und bemühte sich oft für sie und schützte sie vor 
Verfolgungen.}\person{Marsh, Esquire} erhob sich und sagte, es sei
% \picinclude{./150-159/p_s154.jpg} 
des Königs Wohlgefallen, das ich freigesprochen werde, wenn sich
kein Kläger gegen mich erhebe. Sie fragten mich, ob ich es dem
König und dem Rat überlassen wolle? Ich sagte: \zitat{Ja, gerne.}
Hierauf schickten sie des Scheriffs Bericht, der die Anklage enthielt,
dem König, damit er sehe, wessen man mich beschuldigte [...]

Der König, nachdem er dies gelesen und von der ganzen
Angelegenheit unterrichtet war, war von meiner Unschuld überzeugt 
und sandte einen Befehl, mich frei zu lassen, welcher lautete:

\brief{König}{
  Es beliebt Seiner Majestät, zu befehlen, das man dem Manne
  George Fox, bislang Gefangener im Kerker von Lancaster, die
  volle Freiheit schenke [...] Und diese Kundgebung von Seiner
  Majestät Belieben soll euch als Befehl genügen.

  \begin{flushright}Whitehall, 24. Oktober,1660\index{Jahr!1660}. 
Edward Nicholat\person{Nicholat, Edward}\end{flushright}
}

Als ich nun mehr als zwanzig Wochen gefangen gewesen
war, war ich auf Befehl des Königs rechtmäßig frei geworden;
die Macht des Herrn hatte meine Unschuld herrlich kund getan.
Porter wagte nicht, die Anklage, die er fälschlich gegen mich
erhoben hatte, öffentlich zu berichtigen [...]




%%%%%%%%%%%%%%%%%%% Kapitel 14. %%%%%%%%%%%%%%%%%%%%%%%%%%%%%%
\chapter[Beginn neuer Quälerverfolgungen]{Beginn neuer Quälerverfolgungen}

\begin{center}
\textbf{Beginn neuer Quälerverfolgungen bei Anlass der Verschwörungen
der Fifthmonarchy-Leute. Des Quäkers John Perots Verirrungen.
Quäker misshandelt in Neu-England und Malta.}
\end{center}

\section{Erfolglose politische Lobby-Arbeit}

Ich sah nun, warum ich durch so schwere Prüfungen hatte
gehen müssen in Reading, denn die ewige Kraft des Herrn war
über alle gekommen, und sein gesegnetes Leben und Licht und
seine Wahrheit war dem Lande ausgegangen; wir hatten herrliche 
Versammlungen, und viele scharten sich um die Wahrheit.
Richard Hubberthorn\person{Richard Hubberthorn} war beim König 
gewesen, und dieser hatte
gesagt, es dürfe uns niemand behelligen, so lange wir friedlich
leben; er versprach uns dies bei seinem königlichen Wort mit der
Ermahnung, sein Versprechen nicht zu missbrauchen. 


Einige Freunde erhielten auch Zutritt im 
\textit{House of Lords}\index{House of Lords}, und man
gestattete ihnen, ihre Gründe gegen das 
Zehntenwesen\index{Kirchensteuer}\footnote{Kirchensteuer}, 
das Schwören, die Turmhäuser, die Gottesdienste und anderes 
auseinander zusetzen, und man hörte ihnen ziemlich lange zu. 
Etwa 700 Freunde,
die unter Richards und Olivers Regierung in die verschiedenen
% \picinclude{./150-159/p_s155.jpg} 
Gefängnisse des Landes gebracht worden waren, wegen Verstößen,
wie sie es nannten, setzte der König, als er kam, in Freiheit.
Man spürte, das die Regierung geneigt war, den Freunden ihre
Freiheit zu sichern, weil sie sah, das wir unter der vorherigen
Herrschaft so gut wie sie gelitten hatten. Sobald aber wirklich
etwas für uns geschehen sollte, so wurde es wieder vereitelt durch
irgend einen Schuft, der dergleichen tat, als ob er uns wohl
wolle. Es hieß, es sei schon ein Befehl ausgesetzt, der unsere
Freiheit bestätige, er müsse nur noch unterzeichnet werden; da
brach gerade jenes hässliche Attentat\index{Attentat} der 
Fisthmonarchy-Leute\index{Fisthmonarchy}
aus und brachte die Hauptstadt\ort{Hauptstadt} 
und das ganze Land in Aufruhr.


Es geschah am Abend eines Ersten Tages, und wir hatten eine
besonders herrliche Versammlung gehabt, in der die Wahrheit
des Herrn allen erschienen war. Da, bald nach Mitternacht wurde
die Trommel geschlagen und erklang der Ruf: \zitat{zu den Waffen,
zu den Waffen!} Ich stand auf und nahm am folgenden Morgen
ein Boot und fuhr nach Whitehall\ort{Whitehall}, und stieg 
dort aus und schritt
durchs Schloss. Sie betrachteten mich erstaunt, aber ich schritt
durch sie hindurch bis nach Pall-Mall\footnote{eine Straße in der 
City of Westminster in London.}\ort{Pall-Mall}, wo sich etliche Freunde
zu mir gesellten, obgleich es jetzt gefährlich war, über die Straßen
zu gehen; denn schon waren die Stadt und die Vorstädte unter
Waffen, und das Volk und die Soldaten waren sehr roh; sie 
misshandelten Henry Fell,\person{Fell, Henry} der zu einem 
Freunde gehen wollte, und
hätten ihn getötet, wenn nicht der Herzog von 
York\person{Herzog von York} dazu gekommen
wäre. Es geschah viel Unheil während dieser Woche, und am
nächsten Ersten Tage wurden viele Freunde auf dem Weg in die
Versammlung gefangen genommen.

\section{Verschleppung von Fox}

Ich blieb in Pall-Mall, weil ich dort der Versammlung beiwohnen 
wollte; doch in der Nacht des Siebenten Tages kamen
Soldaten und klopften an die Tür. Da die Mägde sie einließen,
so stürzten sie herauf und ergriffen mich. Und einer von ihnen,
der beim Parlament gedient, fühlte mir in die Tasche und fragte,
ob ich keine Pistolen bei mir habe. Ich fragte ihn, warum er
auch solche Frage an mich stelle; er wisse ja, das sich ein 
friedlicher Mann sei [...] Diese Soldaten nahmen mich mit und
brachten mich nach Whitehall [...] Dort waren die Soldaten
und das Volk sehr wild; aber ich predigte doch die Wahrheit
unter ihnen. Einige Große aber, als sie das hörten, sagten:
\zitat{Was, ihr lasst ihn noch predigen? Bringt ihn doch an einen
% \picinclude{./150-159/p_s156.jpg} 
Ort, wo er nicht mehr hetzen kann.} Das taten sie denn auch
und bewachten mich. Ich sagte ihnen, wenn sie schon meinen
Leib binden und einsperren können, so können sie doch das Wort
des Lebens nicht aufhalten. Einige kamen und fragten mich,
was ich sei? Ich erwiderte ihnen: \zitat{ein Prediger der 
Gerechtigkeit} (2. Petr. 2, 5\bibel{Petr. 2. 02:05@2. Petr. 2:5}). 
Nachdem ich etwa drei 
Stunden eingesperrt gewesen war, ging Esquire Marsch zum Lord 
Gerrard\person{Lord Gerrard}, und darauf wurde ich frei gelassen [...]


Während dieses Aufstandes\index{Aufstand} der Fifthmonarchy-Leute 
1660\index{Jahr!1660},
fanden arge Metzeleien statt, sowohl auf dem Lande als in der
Stadt, so das es für anständige Leute noch lange gefährlich war,
auszugehen. Man konnte kaum ohne Gefahr Einkäufe machen.
Auf dem Lande schleppten sie die Leute, Männer und Frauen,
aus den Häusern und Kranke rissen sie aus ihren Betten. Ja
einen Fieberkranken rissen sie aus dem Bett und schleppten ihn
ins Gefängnis; als er dort ankam, starb er, er hieß Thomas
Pachyn.\person{Pachyn, Thomas}

Margaret Fell\person{Fell, Margaret} ging zum König 
und berichtete ihm, wie es
zugehe zu Stadt und Land. Sie setzte ihm auseinander, das wir
harmlose, friedliche Leute seien; das wir aber unsere Versammlungen 
auch fernerhin halten würden, was immer wir auch zu
dulden haben würden; aber es sei seine Pflicht für Frieden zu
sorgen, damit nicht noch mehr unschuldig Blut vergossen werde.
Die Gefängnisse waren nachgerade überall angefüllt mit
Freunden und andern aus der Stadt und vom Lande; überall
waren Wachen aufgestellt zur Durchsuchung der Briefe, so das
niemand passieren konnte, ohne untersucht zu werden. Wir hörten
von vielen Tausenden von Freunden, die im Lande herum in den
Gefängnissen waren, und Margaret Fell überbrachte dem König
und dem Rat einen Bericht darüber. Als wir in der darauf 
folgenden Woche von einigen weiteren Tausenden hörten, die 
gefangen genommen worden, ging sie abermals hin, um es dem König
und dem Rat mitzuteilen. Man verwunderte sich, woher wir
diese Nachrichten hätten, da ein strenger Befehl ergangen war,
alle Briefe aufzufangen. Aber der Herr fügte es, das wir Kunde
erhielten trotz allen ihren Hindernissen.

\section{Das Friedenszeugnis wird erstellt}

Wir ließen eine Erklärung gegen das Bekriegen und das
Verschwören drucken und schickten einige Abzüge an den König
und den Rat; andere wurden in den Straßen verkauft [...]
% \picinclude{./150-159/p_s157.jpg} 

Diese Erklärung klärte etwas die Luft, die auf Stadt und
Land lastete; und der König erließ bald darauf einen Befehl,
das die Soldaten keine Haussuchungen ohne einen Konstable
vornehmen sollten; aber noch immer waren die Gefängnisse gefüllt
und viele Freunde gefangen, woran namentlich der Aufstand der
Fifthmonarchy-Leute\index{Fifthmonarchy} Schuld war. 
Als aber die Gefangenen
sollten hingerichtet werden, ließ man ihnen doch Gerechtigkeit 
widerfahren und erklärte uns öffentlich frei von jeglicher Teilnahme an
den Verschwörungen. Und auf wiederholtes Drängen erließ der
König den Befehl, die Freunde frei zu lassen ohne Loskaufung.
Aber es hatte viel Mühe und Arbeit gebraucht, um das zu
erreichen. Thomas Moor und Margaret Fell waren oft deswegen
beim König gewesen.

Es wurde während dieses Jahres viel Blut vergossen, denn
viele von den Räten des früheren Königs wurden gehenkt, ertränkt
oder gevierteilt. Unter diesen war auch Oberst 
Hacker\person{Oberst Hacker}, der mich
unter Oliver Cromwell\person{Cromwell, Oliver} von 
Leicester\ort{Leicester} nach London\ort{London} als Gefangener
schickte, wie oben berichtet worden. Es war ein trauriger Tag
der Vergeltung\index{Vergeltung} von Blut durch Blut [...]

\section{Spektakuläre Auftritte von Willam Sympson}

Es war eine unsichtbare Hand, die diesen Tag über das
heuchlerische Geschlecht gebracht hatte, das, kaum war es zur
Herrschaft gelangt, so hochmütig und über alle Maßen grausam
geworden war und das Volk Gottes verfolgt hatte.
Mehr als einmal waren diese \textit{Frommen} gewarnt worden
durch Worte, Schrift und Zeichen; aber sie wollten es nicht
glauben, bis es zu spät war. \index{Auftritte!spektakuläre} 
\index{Auftritte!Nackt} \index{Nackte!Auftritte}
Willam Sympson\person{Sympson, Willam} war öfter während
drei Jahren getrieben worden, unbekleidet und barfuß unter sie
zu treten in den Städten, Märkten und Ortschaften, vor Priester
und Große, um ihnen zu sagen: \zitat{so nackt wie er würden auch sie
einhergehen.} Und manchmal trieb es ihn, sich mit einem Sack
anzutun und sein Gesicht zu beschmieren und ihnen zu sagen:
\zitat{also werde der Herr ihnen ihre Frömmigkeit besudeln, wie er
besudelt sei.} Der arme Mensch hatte schwere Leiden erduldet,
sich mit Pferdepeitschen auf seinen bloßen Leib peitschen, sich mit
Steinen bewerfen und einsperren lassen während dieser Jahre, vor
der Rückkehr des Königs; aber sie wollten sich nicht warnen
lassen, sie erwiderten seine Liebe mit Grausamkeit. Nur der
Bürgermeister von Cambridge behandelte ihn großmütig, indem
er seinen Rock um ihn legte und ihn in sein Haus nahm.
% \picinclude{./150-159/p_s158.jpg} 

Ein anderer Freund, Robert Huntingdon\person{Huntingdon, Robert}, 
wurde getrieben ins Turmhaus zu Carlisle zu gehen, unter die 
Haupt-Presbtyterianer\index{Presbtyterianer}
und Independeten\index{Independeten} dort, in einem weißen Hemd als
Zeichen, das das Chorhemd wieder aufkommen werde, und mit
einem Halfter, um zu zeigen, daß auch für sie ein Halfter kommen
werde, was sich auch an einigen von den Verfolgern erfüllte.


Zu einem andern, Richard Sale\person{Sale, Richard}, der 
Konstabler in der Nähe
von Chester war, wurde ein Freund mit einem Pass geschickt;
die schändlichen \textit{Frommen} hatten ihn als Vagabunden 
festgenommen, weil er als Prediger reiste. Dieser Konstabler wurde
durch den ihm zugeschickten Freund bekehrt und gab ihm die
Freiheit; später wurde er selber ins Gefängnis geworfen. Danach
an einem Ersten Tage trieb es Richard 
Sale\person{Sale, Richard} ins Turmhaus zu
gehen und den verfolgungssüchtigen Priestern und ihren Genossen
eine Kerze zu bringen als Anspielung auf ihre Finsternis; aber
sie misshandelten ihn, und, so recht wie verstockte 
\textit{Fromme}, warfen 
sie ihn ins Gefängnis von Little-Gase\ort{Little-Gase}, 
und quälten ihn dort dermaßen,
das er bald darauf starb. 

Die Freunde wurden mehrfach getrieben, 
dieses Geschlecht auf allerlei Art zu mahnen, aber nicht
nur hörte man sie nicht, sondern sie wurden noch misshandelt und
\zitat{verschrobene Quäker} genannt! Aber Gott schickte sein Gericht
\index{Gottesstrafe} \index{Gottesgericht} über die 
verfolgungssüchtigen Priester und Behörden; als der
König zurück kam, wurden den meisten von ihnen ihre Stellen und
Einkünfte entzogen; die Räuber wurden beraubt und es war nun
an uns zu sagen: \zitat{wo sind jetzt die Verschrobenen?} Viele gaben
jetzt zu, das wir wahre Propheten\index{Wahre Propheten} seien, 
und behaupteten, wenn
wir nur gegen einzelne Priester geeifert hätten, so hätten sie sich
sogar über uns gefreut; weil wir aber gegen alle geeifert hatten,
so haben sie sich über uns verärgert. 

\section{Die zwei Seiten der Toleranz}

Aber sie sahen es jetzt ein,
das die Priester, die damals für die besten gehalten wurden, so
schlecht waren wie die übrigen. Ja, viele, die als die aller 
hervorragendsten gegolten hatten, hetzten die Behörden am 
allermeisten zu den Verfolgungen auf. Diese wurden aber, als der
König zurück kam, damit bestraft, das ihnen die Gewissensfreiheit
entzogen wurde, die sie vorher, als sie die Oberhand gehabt hatten,
den andern nicht gegönnt hatten. Einer, namens Hewes, von
Plymouth,\person{Priester Hewes von Plymouth} ein angesehener Priester 
zu Olivers Zeits hatte immer,
wenn irgendwo Gewissensfreiheit zugestanden wurde, gebetet, Gott
wolle es den Behörden ins Herz geben, das sie diese verdammte
% \picinclude{./150-159/p_s159.jpg} 
Toleranz abschaffen. \index{Toleranz} \index{Gewissensfreiheit} 
Und andere beteten gegen \zitat{die nicht zu
duldende Duldsamkeit}. Als nun nach des Königs Rückkehr
diesem Priester Hewes seine großen Einkünfte entzogen wurden,
weil er sich nicht dem Common-Prayer Buch unterwerfen wollte,
fragte ihn ein Freund, als er ihm in Plymouth begegnete: ob er
nun die Duldsamkeit noch verdammungswürdig finde und nicht
vielmehr froh darüber wäre? worauf der Priester nicht antwortete
und nur das Gesicht abwandte. Aber so hartnäckig auch diese
Leute früher gegen Duldsamkeit geeifert hatten, — jetzt kamen
viele von ihnen selber beim König um einen Ort, wo sie ihre
Versammlungen halten könnten, ein, und bezahlten sogar, damit
es ihnen bewilligt werde [...]

\section{Die blutige Verfolgung in Neu-England hat ein Ende}

Wir erhielten die Kunde, das ein Freund, der getrieben worden
war, gegen den Götzendienst der Papisten\index{Papisten} 
zu predigen, in Rom
im Gefängnis gestorben war, und man hatte den Verdacht, das
er heimlich im Gefängnis umgebracht worden war. 
John Perrot \person{Perrot, John}
war auch dort gefangen gewesen und kam nach seiner Freilassung
zu uns zurück; später aber wandte er und andere sich ab von
der Gemeinschaft der Freunde und der Wahrheit [...]

Ungefähr um die gleiche Zeit 1661\index{Jahr!1661} erhielten 
wir die Nachricht 
aus Neu-England\ort{Neu-England}, das die dortige Regierung ein Gesetz
erlassen hatte, das die Quäker aus jenen Kolonien verbannte bei
Todesstrafe im Fall der Rückkehr, und das mehrere Freunde, die
nach ihrer Verbannung zurückgekehrt waren, wirklich gehängt
worden waren \footnote{1661 verfolgten die 
Puritaner\index{Puritaner} und 
Independenten\index{Independenten}, die selber nach Amerika geflohen, um 
Religionsfreiheit zu haben, die Quäler aufs Grausamste.
Ein Edikt von 1658\index{Jahr!1658} bestimmte, das jeder 
Quäker, der zum dritten 
mal in den Kolonien gefunden werde, gehängt werde, und 
dieser Befehl wurde an zwei Quäkern und einer Quäkerin 
ausgeführt.}, und andere vom gleichen Schicksal bedroht im
Gefängnis seien. 

Während jene hingerichtet wurden, war ich im
Gefängnis zu Lancaster gewesen und hatte eine deutliche Wahrnehmung 
ihrer Leiden, als ob sie mich selber betroffen hätten,
und der Strick um meinen eigenen Hals gelegt würde, und wir
hatten doch damals noch nichts davon gehört. Sobald wir nun
davon erfuhren, ging Edward Burrough\person{Burrough, Edward} 
zum König und sagte
ihm, es sei in seinem Reich eine Ader offen, aus der unschuldiges
Blut fließe, das, wenn es nicht gestillt werde, alles überschwemmen
werde. Der König erwiderte: \zitat{Ich werde dieses Blut stillen.}
% \picinclude{./160-169/p_s160.jpg} 
Edward Burrough sagte: \zitat{So tue es eilends, denn wir können nicht
wissen, wie viele in Bälde noch hingerichtet werden.} Der König
sagte: \zitat{So bald ihr wollt,} und befahl einem der Anwesenden:
\zitat{holt den Sekretär, so will ich es sogleich tun.} Als der Sekretär
kam, wurde sofort ein Erlass zugesagt. 

Ein paar Tage darauf
ging Edward Burrough wieder zum König, um ihn zu bitten,
den Erlass abzuschicken; der König antwortete, er habe jetzt keine
Gelegenheit ein Schiff dorthin zu schicken; wenn wir es aber tun
wollten, so stehe uns das frei, so bald wir wollten. Darauf fragte
Edward den König, ob er allenfalls auch einen sogenannter:
\zitat{Quäker} mit seiner Sendung betrauen würde? Der König 
antwortete: \zitat{Ja, es kann gehen, wer will.} 


Hierauf nannte 
Edward ihm Samuel Chattok,\person{Chattok, Samuel} der aus 
Neu-England, seiner Heimat
verbannt worden war und nicht zurückkehren durfte, es sei denn
mit dieser Sendung. Dann ließ er Ralph 
Goldsmith\person{Goldsmith, Ralph} kommen,
den Besitzer eines guten Schiffes, und einigte sich mit ihm auf
300 Pfund, in Waren oder bar, und Abfahrt in zehn Tagen. Er
rüstete sich alsbald, unter Segel zu gehen, und, vom Winde 
begünstigt, kam er nach etwa sechs Wochen, am Morgen eines
Ersten Tages, in Boston in Neu-England\ort{Neu-England} an. 
Es reisten viele mit ihm, aus Alt- und Neu-England, 
Freunde, die der Herr
trieb, mitzugehen und aufzutreten gegen die blutigen Verfolger,
welche alle Übrigen an Grausamkeit übertrafen.


Als die Bewohner von Boston ein Schiff mit englischen
Farben in den Hafen von Boston fahren sahen, kamen sie
gleich aufs Schiff und fragten nach dem Kapitän, und Ralph
Goldsmith sagte ihnen, das er es sei. Sie fragten ihn, ob er
Briefe habe? Er sagte: \zitat{ja}. Sie fragten, ob er sie 
ausliefern wolle? er antwortete: \zitat{nein, heute nicht.} 
Darauf begaben sie sich ans Ufer und berichteten, es sei 
ein ganzes Schiff voll Quäker angekommen, und Samuel Shattock 
sei darunter, der nach den Gesetzen hingerichtet werden müsse, 
wenn er aus der Verbannung zurückkomme! denn sie wussten nichts 
von seiner Sendung. Den ganzen Tag wurden alle streng abgesperrt, 
und keiner von der Schiffsmannschaft durfte landen. 

Am folgenden Morgen begaben sich die Gesandten des Königs, 
Samuel Shattock und der Befehlshaber des Schiffs, Ralph 
Goldsmith ans Ufer, und nachdem sie die Männer, die sie ans 
Land geführt hatten, zurückgeschickt hatten,
gingen sie durch die Stadt zum Haus des Gouverneurs, John
% \picinclude{./160-169/p_s161.jpg} 
Endicott,\person{Endicott, Gouverneur John} 
und klopften. Der Gouverneur schickte jemand heraus,
um sie nach ihrem Begehren zu fragen. Sie ließen ihm sagen,
sie kämen vom König von England und werden ihre Botschaft
niemand übergeben, als dem Gouverneur selbst. Darauf wurden
sie vorgelassen. Der Gouverneur erschien und nachdem er ihre
Botschaft vernommen und ihren Auftrag, nahm er seinen Hut ab
und betrachtete sie. Dann verließ er sie und begab sich zum
Untergouverneur, und nach einer kurzen Unterredung mit diesem
kam er zu den Freunden zurück und sagte ihnen: \zitat{Wir
werden Seiner Majestät Befehl gehorchen.} Hierauf erhielten
die Reisenden die Erlaubnis zu landen, und rasch verbreitete sich
die Kunde von dem Vorgefallenen in der Stadt, und die Freunde
aus der Stadt vereinigten sich mit den Reisenden des Schiffes,
um Gott zu loben und zu danken, das er sie so wunderbar aus
den Zähnen derer, die sie umbringen wollten, befreite. Während
sie beisammen waren, kam ein Freund herein, der von ihrem
blutigen Gesetz zum Tode verurteilt worden war und lange Zeit
in Fesseln gelegen und auf seine Hinrichtung gewartet hatte. Da
wurde die Freude noch größer, und alle erhoben ihre Herzen in
inbrünstigem Loben Gottes, welcher würdig ist zu nehmem Preis,
Ruhm und Ehre; denn er allein kann frei machen und erretten
und helfen allen denen, die ihr Vertrauen auf ihn setzen [...]

\section{Wieder der höfischen Anrede}  

Vorher, als ich noch im Gefängnis zu Lancaster war, war
ein Buch von mir (\buchtitel{The Battledore}) veröffentlicht 
worden, das zeigen sollte, wie in allen Sprachen \zitat{du} 
und \zitat{dich} die eigentliche Anrede an eine einzelne 
Person sei und \zitat{ihr} nur an mehrere.\index{Sprache}\index{Anrede}
Ich hatte es an Beispielen aus der Schrift und aus Lehrbüchern
in etwa dreißig Sprachen nachgewiesen. J. 
Stubbs\person{Stubbs, J.n} und Benjamin 
Furly\person{Furly, Benjamin} hatten sich auf meine 
Veranlassung sehr Mühe gegeben,
das Material zu sammeln, und ich fügte dann noch einiges bei.
Als es fertig war, erhielten der König und die Räte, die Bischöfe
von Canterbury und London und die beiden Universitäten eine
Abschrift und es wurde viel gekauft. Der König sagte, es sei
richtig, das diese Völker so sprechen; und als man den Bischof
von Canterbury fragte, was er davon halte, so wusste er nicht,
was er sagen sollte; denn es wirkte so überzeugend auf die Leute,
das viele daraufhin sich kaum mehr ärgerten, wenn wir \zitat{du}
und \zitat{dir} zu ihnen sagten, während man uns das vorher sehr
übel genommen hatte [...]
% \picinclude{./160-169/p_s162.jpg} 

Da die Priester und Bischöfe gerade eifrig am Werk waren,
ihre Gottesdienste einzurichten und alle zu zwingen, daran teil
zu nehmen, trieb es mich, folgendes zu schreiben, um die Art der
wahren Gottesdienste, die Christus eingesetzt hat und die Gott
annimmt, zu zeigen:

\brief{Verfolger}{
  Der wahre Gottesdienst Christ geschieht im Geist und steht
  allen Menschen offen. Die im Geist und in der Wahrheit anbeten, 
  die sind Gott angenehm (Join). Es gibt dem Volk Odem
  und den Geist denen, die auf der Erde sind 
  (Jes. 42:5\bibel{Jes. 42:05@Jes. 42:5}), und er
  gibt ihnen eine unsterbliche Seele; sie sind sie Tempel, in denen
  er wohnen will (1. Kor. 3:16\bibel{Kor. 1. 03:16@1. Kor. 3:16}). 
  Die, welche äußerlich
  Juden\index{Juden} waren, mussten nach Jerusalem\ort{Jerusalem} 
  gehen, um anzubeten, so
  lange sie dort ihren äußeren Tempel\index{Tempel!äußerer} 
  hatten; [...] nun aber
  sollen alle \zitat{Gott im Geist und in der Wahrheit anbeten}. Dies
  ist ein Gottesdienst der Freiheit, denn \zitat{wo der Geist ist, 
  da ist Freiheit} (2. Kor. 3:17\bibel{Kor. 2. 03:17@2. Kor. 3:17}). 
  Die Früchte des Geistes werden
  offenbar werden; und man soll der Geist keine Schranken setzen,
  sondern in ihm wandeln und lebens, damit man seine Früchte
  hervorbringen kann. [...] Denn ,an ihren Früchten sollt ihr
  sie erkennen (Matth. 7:16\bibel{Matth. 07:16@Matth. 7:16}) [...] 
  \medskip 
  \begin{flushright}G. F. \end{flushright}
}


\section{Missionsbemühungen der Katholiken}

Viele Papisten\index{Papisten} und Jesuiten\index{Jesuiten} 
fingen damals an, den Freunden \index{Ökumene}
zu schmeicheln und zu sagen, so oft sie einen von ihnen sahen,
von allen Sekten haben die Quäkeren meisten Selbstverleugnung,
und es sei schade, das sie nicht in die heilige Mutterkirche 
zurückkehrten. In dieser Weise schwatzten sie den Leuten vor und 
behaupteten, sie würden gern mit den Freunden unterhandeln; aber
die Freunde verabscheuten es, sich mit ihnen einzulassen und
hielten es für gefährlich und sogar anstößig, weil es Jesuiten
waren. 

Als ich aber davon hörte, sagte ich: \zitat{lasst uns mit ihnen
unterhandeln, seien sie, wer sie wollen.} Somit wurde die Zeit
festgesetzt, zu der zwei, die wie Höflinge aussahen, kamen; sie
fragten nach unsern Namen, die wir ihnen nannten; wir aber
fragten nicht nach ihren Namen, denn wir wussten ja, das sie
Papisten waren und wir Quäker. Ich fragte sie das selbe, was
ich schon früher einen Jesuiten gefragt hatte, nämlich, ob die
Römische Kirche nicht abgefallen sei von der Kraft, dem Geist
und den Grundsätzen der apostolischen Zeiten? [...] Als sie
sahen, das wir es genau nahmen, wichen sie aus, indem sie sagten,
es sei eine Anmaßung zu behauptet, irgend jemand habe den
% \picinclude{./160-169/p_s163.jpg} 
Geist und die Kraft, den die Apostel hatten. Aber ich sagte, es
sei eine Anmaßung von ihnen, die Worte Christi, der Apostel und
Propheten zu benützen und die Leute glauben zu machen, sie seien
Nachfolger der Apostel und Propheten, da sie doch zugeben müssen,
sie haben nicht den Geist und die Kraft der Apostel. Ich zeigte
ihnen, wie verschieden ihr Tun und ihre Früchte von denen der
Apostel seien. Darauf erwiderte mir einer von ihnen: \glqq Ihr seid
eine Gesellschaft von Träumern.\grqq{} \glqq Nein,\grqq{} erwiderte ich, 
\glqq sondern ihr seid widerwärtige Träumer, die ihr euch als die 
Nachfolger der Apostel träumt, während ihr doch zugebt, das ihr nicht ihren
Geist und ihre Kraft habt. Und ist es nicht Befleckung des Fleischer?
zu sagen, es sei Anmaßung zu behaupten, man habe den Geist
und die Kraft der Apostel? Und wenn ihr nun zugebt, das ihr
nicht den Geist und die Kraft der Apostel habt,\grqq{} sagte ich, \glqq 
so ist es klar, das ihr von einem anderen Geist und einer anderen
Kraft geleitet werdet als die erste Kirche und die Apostel.\grqq{} Ich
erklärte ihnen, das es ein böser Geist sei, der sie leite und sie zu
dem Beten mit Rosenkränzen und zu Bildern geführt habe und
zum errichten von Klöstern \index{Kloster} und zum Töten um des Glaubens
willen. Ich wies sie darauf hin, wie solches Tun gesetzlich und
nicht nach dem Evangelium der Freiheit sei. Sie waren dieser
Reden bald überdrüssig und gingen fort, und wir Vernahmen,
das sie den Papisten rieten, nicht mit uns zu disputieren, noch
von unsern Büchern zu lesen; somit waren wir sie los. Aber
wir setzten uns mit allen andern Sekten auseinander, mit den
Presbyterianern\index{Presbyterianern}, 
den Independenten\index{Independenten}, 
den Seekers\index{Independenten}, 
den Baptisten\index{Baptisten},
den Episkopalen\index{Episkopalen}, 
den Socinianern\index{Socinianern}, 
den Brownisten\index{Brownisten}, 
den Lutheranern\index{Lutheranern},
den Calvinisten\index{Calvinisten}, 
den Arminianern\index{Arminianern}, 
den Fifthmonarchyleuten\index{Fifthmonarchyleuten}, 
den Feministen\index{Feministen}, 
den Rantern\index{Rantern}. 
Von diesen allen behauptete niemand,
den gleichen Geist und die gleiche Kraft wie die Apostel zu haben.
In diesem Geist und dieser Kraft verlieh uns also der Herr den
Sieg über sie alle. Was die Fisthmonarchyleute betrifft, so trieb
es mich, eine Schrift zu schreiben, um ihren Irrtum aufzudecken.
Sie erwarteten Christi persönliche Wiederkunft\index{Christi 
(persönliche) Wiederkunft} in äußerer Form
und Weise und setzten dazu das Jahr 1666 fest, und viele, wenn
es um diese Zeit donnerte und regnete, machten sich bereit, weil
sie meinten, nun komme Christus, um sein Reich aufzurichten,
und bildeten sich ein, sie müssten nun die Hure draußen in der
Welt töten (Offb. 17\bibel{Offb. 17}). Aber ich sagte ihnen, 
die Hure\index{Hure} sei lebendig
% \picinclude{./160-169/p_s164.jpg} 
in ihnen und noch nicht verzehrt vom Feuer Gottes und von
ihnen im Geist und der Kraft des Herrn vernichtet. Und ihre
Erwartungen, daß Christus äußerlich wiederkomme, um sein Reich
aufzurichten, sei wie das \glqq siehe hier, siehe da\grqq{} (Luc. 
17,28\bibel{Luk. 17:28)} der
Pharisäer. Aber Christus sei vor mehr als 1600 Jahren gekommen, 
um sein Reich auszurichten, wie Nebukadnezar\person{Nebukadnezar} 
geträumt und Daniel\person{Daniel} prophezeit, und 
habe die vier Reiche zertrümmert und
das große Volk mit dem goldenen Kopf und den Armen und
Beinen aus Silber, alles habe der Wind Gottes weggeblasen
wie im Sommer die Spreu beim Dreschen (Dan. 2, 32\bibel{Dan. 2:32)}).
Christus habe gesagt, als er auf Erden war: \glqq Mein Reich ist
nicht von dieser Welt.\grqq{} Wenn es von dieser Welt gewesen wäre,
so hätten seine Diener gekämpft (Joh. 18)\bibel{Joh. 18}; aber es war nicht
von dieser Welt, darum kämpften sie nicht. Alle diese 
Fifthmonarchyleute, die mit fleischlichen Waffen kämpfen, sind keine
Diener Christi, sondern Diener des Tieres und der Hure; Christus
sagt: \glqq Mir ist gegeben alle Gewalt im Himmel und auf Erden\grqq{}
(Matth. 28,18)\bibel{Matth. 28:18}, und sein Reich, das vor 
1600 Jahren aufgerichtet  wurde, herrschet noch. Und der Apostel sagt: 
\glqq Wir sehen Christus regieren, und er wird fort regieren, 
bis das alle Dinge ihm untertan sind\grqq{} (1. Kor. 15)\bibel{Kor. 1. 15@1. Kor. 15}.

In diesem Jahre, 1661\index{Jahr!1661}, trieb es viele Freunde übers Meer,
zu gehen, um die Wahrheit in fremden Ländern zu verkünden.
John Stubbs\person{Stubbs, John}, Henry Fell\person{Fell, Henry} 
und Richard Costrob\person{Costrob, Richard} trieb es nach
China\ort{China} und Priester Johannes Gegend zu gehen; aber kein Schiff
wollte sie nehmen. Mit vieler Mühe erhielten sie eine Vollmacht
vom König; aber die Ostindische Gesellschaft fand Mittel und
Wege, sie zu umgehen, und die Schiffsherren wollten sie nicht
nehmen. Sie begaben sich nun nach Holland, in der Hoffnung,
dort überfahren zu können; aber auch dort wollte sie niemand
nehmen. Nun nahm Henry Fell und John Stubbs ein Schiff,
das nach Alexandrien in Ägypten\ort{Ägypten} ging, in der Absicht, von dort
aus sich einer Karawane anzuschließen. Doch da kam Daniel
Baker\person{Baker, Daniel} und veranlasste Richard 
Costrop gegen seine innere 
Freiheit, mit ihm nach Smyrna\footnote{heutiger türkischer Name: 
Izmir}\ort{Izmir} zu gehen. Auf der Überfahrt wurde
Richard krank, da kümmerte sich Daniel Baker gar nicht um ihn
und er starb. Aber der hartherzige Mann verlor später seine Stelle.

John Stubbs und Henry Fell erreichten Alexandrien,\ort{Alexandrien} aber
sie waren kaum dort, als der englische Konsul sie schon verbannte;
% \picinclude{./160-169/p_s165.jpg} 
doch verbreiteten sie, ehe sie fort gingen, viele Bücher und Schriften,
um den Türken und Griechen den Weg der Wahrheit zu zeigen;
das Buch betitelt: \glqq{}Die Gewalt des Papstes gebrochen\grqq{}, gaben
sie einem alten Mönch, damit er es dem Papst bringe oder
schicke; als der Mönch es durchgelesen, legte er die Hand aufs
Herz und sagte: "`Was hier geschrieben steht, ist Wahrheit; wenn
ich es aber öffentlich bekennen würde, so würden sie mich verbrennen."'
John Stubbs und Henry Fell kehrten nach England zurück, weil es 
ihnen nicht erlaubt wurde, weiter zu gehen,
und kamen wieder nach London. Stubbs hatte eine Vision, das
die Engländer und Holländer, die sich verbündet hatten, sie nicht
überzuschiffen, sich untereinander entzweien werden, und so kam
es auch [...]

Wir hatten aber nicht nur Schweres von außen zu erdulden,
sondern auch unter uns durch John Perrot\person{Perrot, John} 
und seine Anhänger. Einem trügerischen Geiste nachgebend, 
suchte er unter den Freunden den schlechten, unziemlichen 
Brauch einzuführen, das man während
des allgemeinen Gebets den Hut aufbehalten solle.\ort{Hutstreit} Viele Freunde
hatten mit ihm und seinen Anhängern darüber gesprochen, und
ich hatte einigen deswegen geschrieben, aber er und andere taten
sich nur noch mehr gegen uns zusammen [...]

Eine der Sorgen, die die Freunde von außen trafen, war,
das man die Art, wie sie sich verheirateten, beanstandete. So
kam zum Beispiel folgender Fall vor das Gericht von Nottingham: \ort{Nottingham}
etwa zwei Jahre vorher hatten sich zwei aus der Gemeinschaft 
der Freunde geehelicht; da starb der Mann und hinterließ
der Frau, die guter Hoffnung war, einen Besitz an Land und Zinslehen. 
Als das Kind geboren war, erklärte es das Gericht als
Erbe\index{Erbschaft} seines Vaters, und es wurde als solcher anerkannt. Später
heiratete ein anderer Freund die Witwe. Daraufhin kam ein
naher Verwandter des ersten Mannes und verklagte den Freund,
der die Witwe geheiratet, und suchte ihm seinen Besitz zu entreißen
und das Kind seines Erbes zu berauben und alles an sich zu
bringen, als nächster Erbe des ersten Mannes. Um dies zu
begründen, suchte er die illegitime Geburt des Kindes zu beweisen
mit der Behauptung, die Ehe\index{Eheschließung} sei nicht nach dem Gesetz gewesen.
Bei den Verhandlungen gebrauchte der Kläger ungebührliche Ausdrücke 
gegen die Freunde und sagte, sie täten sich zusammen wie
das Vieh; und andere scheußliche Dinge. Nachdem die Anwälte
% \picinclude{./160-169/p_s166.jpg} 
beider Parteien gesprochen hatten, nahm der Richter die Sache
in die Hand und sagte, es sei eine Ehe im Paradies geschlossen
worden, als der Adam die Eva und die Eva den Adam genommen
hatte; es sei eben die Zustimmung der beiden Teile, maß eine
Ehe ausmache. Was die Quäker anbelange, so kenne er ihre
Ansichten nicht, aber er glaube nicht, das sie sich zusammentun
wie das unvernünftige Vieh, wie man von ihnen behaupte, sondern
wie Christen, und darum glaube er, die Ehe sei gesetzlich gewesen
und das Kind legitimer Erbe. Um das Gericht zu überzeugen,
brachte er einen anderen Fall: Ein Mann, der schwach und 
bettlägerig war, hatte in diesem Zustande den Wunsch, sich zu 
verehelichen und erklärte vor Zeugen, das er diese Frau zum
Weibe nehme und die Frau erklärte, das sie diesen zum Mann
nehme; diese Ehe wurde später angefochten, aber alle Bischöfe
erklärten damals die Ehe für gültig. Daraufhin entschied das
Gericht auch zu Gunsten des Quäkerkindeß, gegen den Mann,
der es um sein Erbe bringen wollte.

Um diese Zeit wurde der Supremats- und Huldigungseid
von den Freunden gefordert als eine Falle, denn man wusste,
das wir nicht schwören konnten, und es wurden in der Folge
viele gefangen gesetzt. Bei dieser Gelegenheit veröffentlichten
die Freunde die Schrift: "`Die Gründe und Ursachen, warum wir
nicht schwören"' und es trieb mich, derselben einige Linien 
beizufügen, damit man sie dem Magistrate gebe:

\brief{Magistrat}{
  Die Welt sagt: "`Küsse das Buch"'\footnote{Aus der Formel beim 
  Schwören des Eides.}; das Buch aber sagt:
  "`Küsse den Sohn, das er nicht zürne"' (Ps. 2:12)\bibel{Ps. 2:12}. 
  Der Sohn sagt: "`Bleibet bei Ja und Nein in euren Reden, denn maß darüber ist,
  das ist vom Ubel"' (Matth. 5:37)\bibel{Matth. 5:37}. Wiederum 
  sagt die Welt: "`Leget die Hand auf daß Buch"'; aber das Buch sagt: "`Was unsre Hände
  betastet haben vom Worte des Lebens"' (1. Joh. 1:1)\bibel{Joh. 1. 1:1@1. Joh. 1:1} [...] Und
  Gott sagt: "`Dies ist mein lieber Sohn, den sollt ihr hören"';
  er ist das Leben, die Wahrheit, das Licht und der Weg zu Gott."'
  \medskip 
  \begin{flushright}G. F.\end{flushright}
}

Weil so viele Freunde gefangen waren, verfassten Richard
Hubberthorn\person{Hubberthorn, Richard} und ich eine Schrift 
und ließen sie dem König überreichen,
damit er erfahre, wie wir von seinen Beamten behandelt wurden
sie lautete:

% \picinclude{./160-169/p_s167.jpg} 

\brief{König}{
  \begin{center}An den König:\end{center}

  \medskip 

  Freund,

  \medskip 

  Der du der Herrscher dieses Reiches bist! Hier ist eine Aufzählung 
  eines Teiles der Leiden, die das Volk Gottes, das man
  im Ärger Quäker nennt, zu erdulden hat. Unter dem Wechsel
  der Mächte, die deiner Regierung voran gingen, haben sie viel
  gelitten; 3170 wurden gefangen genommen um des Gewissens
  willen, und weil sie Zeugnis ablegten für die Wahrheit, die in
  Christus ist; und noch jetzt sind 73 Personen im Namen des 
  Commonwealth gefangen; 32 Personen starben im Gefängnis während
  der Zeit des Commonwealth und unter Oliver und Richard, in
  harter, grausamer Gefangenschaft, aus schmutzigem Stroh und in
  gräulichen Löchern. Und 3068 Personen sind seit deiner Rückkehr
  gefangen genommen worden durch solche, die sich damit bei dir
  einzuschmeicheln suchten. Zudem werden unsre Versammlungen
  täglich gestört durch Männer mit Waffen und Knütteln, obwohl
  mir friedlich zusammenkommen, nach der Art des Volkes Gottes
  der ersten Zeiten; unsre Freunde werden ins Wasser geworfen
  und werden blutig geschlagen; ja, es können gar nicht alle die
  Gräueltaten aufgezählt werden. Nun möchten wir gerne von dir
  erbitten, das du alle, die im Namen des Commonwealth und im
  Namen der beiden Protektoren und in deinem eigenen Namen
  um des Gewissens und der Wahrheit willen gefangen sind, frei
  gebest; haben sie doch nie die Hand erhoben gegen dich oder
  irgend sonst jemand, und das, wenn sich die Freunde friedlich
  versammeln, um Gott anzubeten, sie nicht mehr durch rohe 
  Bewaffnete gestört werden. Ein Hauptgrund dieser frühern 
  Gefangennahme war der, das wir den Protektoren und den verschiedenen
  Regierungen keine Eide leisten konnten; und nun tut man uns
  ins,Gefängnis, weil wir den Huldigungseid nicht leisten können.

  Wenn nun dir oder irgend einem Menschen gegenüber unser
  ja nicht ja und unser nein nicht nein sein sollte, dann lass uns
  dafür das leiden, was andere leiden müssen, wenn sie einen Eid
  brechen! Wir haben alle diese Jahre viel gelitten an unserm
  eigenen Leib und an unserer Habe, unter mancherlei Regierungen,
  weil wir nicht schwören, sondern Christi Gebot folgen, das sagt,
  "`ihr sollt überhaupt nicht schwören"'; dieses besiegeln wir mit Leib
  und Gut, mit unserm ja und nein, wie Christus es befiehlt.
  Bedenke das in der Weisheit, die aus Gott ist, damit du in
  % \picinclude{./160-169/p_s168.jpg} 
  derselben solchem Tun Einhalt gebietest, du, der du die Herrschaft
  hast und solches vermagst. Wir möchten, das alle, die jetzt im
  Gefängnis sind, frei werden und nicht wieder um der Wahrheit
  und des Gewissens willen gefangen genommen werden. Und wenn
  du untersuchst, ob sie unschuldig leiden, so las ihre Ankläger vor
  dich kommen, und wir wollen, wenn nötig, ausführlich Bericht
  über ihre Leiden erstatten.

  \medskip 

  \begin{flushright}G. F. und R. H.\end{flushright}

}

Zwei Freunde, beides Frauen, waren auf Malta\ort{Malta} bei der
Inquisition gefangen, Katharine Evans\person{Evans, Katharine} 
und Sarah Chevers\person{Chevers, Sarah}; da
es hieß, ein Lord D'Aubeny\person{Lord D'Aubeny}, 
ein römisch-katholischer Priester,
könne ihnen die Freiheit verschaffen, so ging ich zu ihm. Nachdem 
ich ihn über alles, was ihre Gefangennahme betraf, unterrichtet 
hatte, bat ich ihn, an die dortigen Behörden um ihre
Freilassung zu schreiben. Er versprach bereitwilligst, es zu tun
und das, wenn ich in einem Monat wieder komme, man mir
ihre Freisprechung mitteilen wolle. Als ich zur bestimmten Zeit
wieder hin kam, sagte er, sein Brief sei scheins nicht angekommen,
denn er habe keine Antwort erhalten, aber er versprach, nochmals
zu schreiben, und tat es auch, und sie wurden beide frei.
Mit diesem hohen Herrn redete ich viel über Religion, und
er gab zu, das Christus jeden, der in die Welt kommt, erleuchtet
mit seinem geistigen Licht, und das er den Tod für einen jeden
gekostet hat, und das die heilsame Gnade Gottes allen Menschen
erschienen ist und sie lehrt und ihnen das Heil bringt, wenn sie
ihr gehorchen. Ich fragte ihn darauf, wozu denn die Papisten\index{Papisten}
alle ihre Bilder und Reliquien brauchen, wenn sie an dieses Licht
glauben und die Gnade, die sie lehrt und ihnen das Heil bringt,
annehmen? Er antwortete, das seien nur Mittel, um das
Volk in Unterwürfigkeit zu erhalten. Er zeigte sich in dieser
Unterredung sehr weitherzig; ich hörte nie einen Papisten soviel
zugeben wie diesen [...]

Im gleichen Jahre, als ich in Cambridgeshire war, hörte ich,
das Edward Burrough gestorben war; und da ich wuste, wie
schwer und traurig dieser Verlust für die Freunde war, schrieb
ich folgende Zeilen zur Aufrichtung und Beruhigung ihrer Gemüter:


\brief{Quaker-Gemeinde}{
  Freunde,

  \medskip 

  Seid stille und ergeben und gefasst im Samen Gottes, der
  sich nicht ändert, damit ihr den lieben Edward Burrough unter
  % \picinclude{./160-169/p_s169.jpg} 
  euch spüren möget in diesem Samen, durch den er euch bei Gott,
  bei dem er jetzt ist, vertreten wird; durch diesen Samen könnet
  ihr ihn alle sehen und fühlen, denn in diesem ist Einigkeit und
  Leben; freuet euch seiner im unvergänglichen Leben, das 
  unsichtbar ist."' [...] 

  \medskip 
  \begin{flushright}G. F\end{flushright}.

}

\input{./kap_15}
\input{./kap_16}
\input{./kap_17}
\input{./kap_18}
\input{./kap_19}
\input{./kap_20}


%%%%%%%%%%%%%%%%%%% Kapitel 21. %%%%%%%%%%%%%%%%%%%%%%%%%%%%%%

\chapter[Schriften ordnen und für Frauenversammlungen eintreten]{Schriften 
ordnen und für Frauenversammlungen eintreten}

\begin{center}
\textbf{Fox sammelt und ordnet die Bücher und Schriften, die er
geschrieben und tritt für die Frauenversammlungen ein.}
\end{center}


Als ich nun wieder frei war, besuchte ich die Freunde in
London,\ort{London} und da ich mich gar nicht wohl fühlte, ging ich bald
nach Kingston.\ort{Kingston} [...] Ich blieb jedoch nicht lange dort, sondern
kehrte wieder nach London zurück, [...] ging dann nach Coffel,
Preston, [...] Lancaster [...] und am 25. des 4. Monats nach
Swarthmore.\ort{Swarthmore}

Als ich eine Weile dort war, kamen viele Freunde aus verschiedenen 
Gegenden des Landes, um mich zu besuchen. Auch
von Schottland\ort{Schottland} kamen manche; von diesen hörte ich, das vier
junge Studenten in Aberdeen\ort{Aberdeen} in diesem Jahre bekehrt worden
waren, bei einer Disputation\index{Disputation} von Robert Barclay\footnote{Robert 
Barclay, der \zitat{Theo1oge der Quüker}s. Weingarten 
a.a.D.},\person{Barclay, Robert}  und George
Keith\person{Keith, George} mit einigen Schülern der Universität 
[...] Während ich
in dieser Stadt war, lies ich mehrere Bücher drucken. Eines:
\zitat{Über das Schwören}, ein anderes: \zitat{Niemand ist ein Nachfolger
der Apostel und Propheten, als wer ihnen nachfolgt in der gleichen
Kraft und dem gleichen Geist, darin sie waren.} \index[buch]{Niemand 
ist ein Nachfolger der Apostel und Propheten, als wer ihnen nachfolgt in der gleichen
Kraft und dem gleichen Geist, darin sie waren.} Ein anderes:
% \picinclude{./240-249/p_s244.jpg} 
\zitat{Besitzen geht über Bekennen}\index[buch]{Besitzen geht über 
Bekennen}, und das die Bekennenden jetzt
Christus im Geist verfolgen, wie die jüdischen Bekennenden ihn
äußerlich verfolgten in den Tagen, da er im Fleisch wandelte [...] .
Ferner die acht folgenden Bücher: 

\begin{itemize}
 \item \zitat{An die Behörden von Danzig}\index[buch]{An die Behörden von Danzig}
 \item \zitat{Kain gegen Abel}\index[buch]{Kain gegen Abel}, oder eine
 \item \zitat{Antwort auf das Gesetz der Männer von Neu-England}\index[buch]{Antwort 
  auf das Gesetz der Männer von Neu-England}
 \item \zitat{An die Freunde zu Nevis über das Wachsamsein}\index[buch]{An die 
  Freunde zu Nevis über das Wachsamsein}
 \item \zitat{Ein Generalbrief an alle Freunde in Amerika}\index[buch]{Ein 
  Generalbrief an alle Freunde in Amerika}
 \item \zitat{Über das, was des Kaisers und was Gottes ist}\index[buch]{Über 
  das, was des Kaisers und was Gottes ist}
 \item \zitat{Über die Ordnung in den Familien}\index[buch]{Über die Ordnung 
  in den Familien}
 \item \zitat{Der geistliche Mensch richtet alle Dinge}\index[buch]{Der 
  geistliche Mensch richtet alle Dinge}
 \item \zitat{Über die höhere Kraft}\index[buch]{Über die höhere Kraft}
\end{itemize}

\bigskip 

überdies schrieb ich mehrere Briefe, sowohl an die Freunde
in England, als auch jenseits des Meeres; auch Antworten
aus mehrere Flugblätter über die Abtrünnigkeit mehrerer, die
sich der Ordnung des Evangeliums widersetzt hatten, und viel
Unruhe und Zank in Westmorland\ort{Westmorland} angestiftet hatten. Es trieb
mich darum, an die dortigen Freunde ein paar besondere Zeilen
zu schreiben.

\brief{Quaker-Gemeinde}{
  An die Freunde in Westmorland,

  \bigskip 

  Leber alle in der Kraft Gottes, in seinem Licht und Geist,
  die euch zuerst bekehrten, das ihr durch sie in der ersten Einigkeit
  bleibet, in Demut und in der Furcht Gottes, und seiner friedsamen,
  sanften Wahrheit, welche ihr leicht erbitten könnt, damit ihr in
  dieser Kraft, diesem Licht und Geist alle dienstbereit seid in euren
  Männer- und Frauen-Versammlungen, im Besitz der Ordnung
  des Evangeliums, welches Evangelium Leben und unvergängliches
  Wesen ans Licht gebracht hat [...] Darum, ihr Freunde in
  Westmorland, bleibet in der Kraft Gottes; sie muss euch behüten
  und schützen, wenn ihr wollt geschützt sein. Lasset euren Glauben
  in der Kraft Gottes stehen, nicht in der Weisheit menschlicher
  Worte, auf das ihr nicht fallet. In Gottes Kraft habt ihr
  Friede, Leben und Einigkeit; und weil ihr nicht in der Kraft
  Gottes geblieben seid, in seiner Gerechtigkeit und seinem Heiligen
  Geist, ist dieser Zank unter euch gekommen. 
  \bigskip 
  \begin{flushright}
  G. F.\end{flushright}
}

Ich schrieb auch einen Generalbrief\index{Generalbrief} an die Freunde an der
Jahresversammlung in London, [...] darin hieß es unter
anderem:

\brief{Quaker-Gemeinde}{
  Was die wahren Männer- und Frauen-Versammlungen\index{Autorität der Versammlung}
  anbelangt, die nach Gottes Rat eingesetzt waren, so widersetzt
  sich jeder, der sich ihnen widersetzt, zugleich der Kraft Gottes, auf
  % \picinclude{./240-249/p_s245.jpg} 
  deren Befehl sie eingesetzt\index{Von Gott eingesetzt} sind. Die, welche sich dieser Kraft
  widersetzten, sind nicht Diener des Evangeliums oder Christi [...]
  Unsere Männer- und Frauen-Versammlungen, und alle anderen
  Versammlungen, die im Namen Jesu geschehen, sind nach dem
  Evangelium Christi, nach der Kraft Gottes, eingesetzt, also nicht
  von Menschen oder durch Menschen. Darin sollen alle sich 
  versammeln und Gott anbeten; darin sollen auch alle handeln, und
  darinnen haben auch alle Gemeinschaft untereinander, ein frohes,
  friedsames Beisammensein.

  Alle gläubigen Männer und Frauen in jedem Land und
  jeder Stadt, deren Glauben in der Kraft Gottes steht, im Evangelium 
  Christi, und die dieses Evangelium, diese Kraft Gottes, besitzen, 
  haben alle ein Recht an die Kraft in diesen Versammlungen;
  denn sie sind alle Erben der Kraft, nach der die Männer- und
  Frauen-Versammlungen eingesetzt sind. [...]
}


Um diese Zeit sammelte ich so viel wie möglich von den
Briefen, die ich in früheren Jahren an die Freunde geschrieben
hatte. Ich machte eine Sammlung der Briefe, die ich an 
O. Cromwell\person{Cromwell} und seinen Sohn Richard 
geschrieben hatte zur Zeit ihres
Protektorats, und an die damaligen Parlamente und Behörden.
Ebenso sammelte ich die Briefe, die ich an König Karl II nach
seiner Rückkehr geschrieben hatte und an seine Räte und die
Richter und Beamten unter ihm. Ich machte auch eine Sammlung 
der Zeugnisse, die ich von den verschiedenen Statthaltern,
Richtern und Räten, Parlamentsmitgliedern und andern erhalten
hatte, um mich von allerlei Verleumdungen zu reinigen, welche
die bösen Priester und \textit{Frommen}, diesseits und jenseits des
Meeres, auf mich geworfen hatten. Solches tat ich um der
Wahrheit willen, weil ich wusste,\index{Archivierung} das der 
Zweck ihrer Verleumdung\index{Verleumdung}
war, die von mir verkündete Wahrheit zu schmähen, und ihre
Verbreitung unter den Leuten zu hindern. Außerdem machte ich
noch zwei Sammlungen, die eine war eine Liste oder ein 
Verzeichnis der Namen derjenigen Freunde, welche im Norden von
England aufgetreten waren, als die Wahrheit dort zuerst 
hervorgebrochen war, um in jenen Gegenden den Tag des Herrn zu
verkünden. Das andere waren die Namen derjenigen Freunde,
welche zuerst das Evangelium in andern Ländern, Gegenden und
Ortschaften predigten, und in welchem Jahr, und wohin sie gegangen. 
Dann machte ich eine andere Sammlung, in zwei Büchern;
% \picinclude{./240-249/p_s246.jpg} 
in dem einen waren die an mich gerichteten Schreiben und Briefe
von Freunden und andern, bei verschiedenen Gelegenheiten; im
andern waren meine Briefe an Freunde und andere. Ich schrieb
auch ein Buch über die Zeichen und Sinnbilder von Christus
und ihre Bedeutung, und manche andere Dinge, die den Freunden
und der Wahrheit künftig von Nutzen sein werden [...].

Weil ich sah, wie die Wahrheit sich immer mehr ausbreitete
im Lande, und die Zahl der Freunde stets zunahm, so trieb mich
die ewige Kraft des Herrn, auch zum Einrichten von Frauen-Versammlungen 
zu raten, damit alle, sowohl Männer wie Frauen,
welche das Evangelium, das Wort vom ewigen Leben, empfangen,
getrieben von der Kraft Gottes, in die Ordnungen des Evangeliums 
kommen möchten, und in der Kraft für den Herrn wirken
und in derselben dienen möchten, zu seiner und der Kirche
Frommen [...].

Etliche von denen, die Vorgaben, zu den Bekennern der
Wahrheit zu gehören und dies zur Schau getragen hatten,
waren, statt bei dem einfachen Evangelium zu bleiben, in
Zänkereien\index{Zänkereien} und Spaltungen\index{Spaltungen} 
geraten und versuchten, den Freunden,
besonders den Frauen, ihre göttliche Wachsamkeit, die sie in der
Kirche gegenseitig aneinander nach der Wahrheit ausübten, zu
verleiden, indem sie sich ihren Versammlungen widersetzten, die
eben zu diesem Zwecke eingerichtet wurden; darum trieb mich der
Herr, einen Brief zu schreiben, und unter den Freunden zu 
Verbreiten, um den Geist, in dem diese Widersacher handelten, 
aufzudecken und die Freunde davor zu warnen; es hieß darin
unter anderem:

\brief{Quaker-Gemeinde}{
  \index{Innere Konflikte}\index{Geschäftsversammlung}
  Etliche, die in diesem Geiste wandeln, haben mir gesagt,
  sie sehen den Nutzen der Frauenversammlungen nicht ein.\index{Frauenversammlungen}
  Meine Antwort an solche war und ist noch, das wenn sie blind
  sind und nichts sehen, sie sich wenigstens nicht den andern
  widersetzen sollen, denn es widersetzt sich ihnen auch niemand;
  Gott hat die Blindheit nie als Verdienst gerechnet, und sein Volk
  soll es auch nicht. Vielmehr hat Christus alle erleuchtet, und so
  Viele ihn aufnahmen, denen gab er Macht, Gottes Kinder zu
  sein (Joh.1,12)\bibel{Joh. 01:12@Joh. 1:12} Die, welche 
  Erben seiner Kraft sind und seines
  Evangeliums, das \zitat{Leben und unsterbliches Wesen ans Licht
  bringt} (2. Tim. 1,10)\bibel{Tim. 2. 01:10@2. Tim. 1,10}, 
  stehen über der Macht der Finsternis, die
  jene verfinstert; sie halten die Gebote des Evangeliums, und sie
  % \picinclude{./240-249/p_s247.jpg} 
  halten ihre Versammlungen in der Kraft Gottes, die sie im Leben
  und dem unsterblichen Wesen bewahret, sie sehen den Nutzen
  der Männer- und Frauen-Versammlungen nach der Ordnung des
  Evangeliums, dieser Gotteskraft, ein, denn sie haben Teil an dieser
  Kraft, auf welche sich die Versammlungen gründen. Darum
  sage ich euch allen, die ihr gegen die Versammlungen der Frauen
  seid, oder auch der Männer, wenn ihr keinen Nutzen in den
  Versammlungen der Frauen seht und euch denselben widersetzt,
  so seid ihr darin nicht in der Kraft Gottes und lebet nicht in
  seinem Geist. Denn Gott sah einen Nutzen in den Zusammenkünften 
  der Frauen in den Tagen des Gesetzes, für allerlei Geschäfte 
  zu seiner Ehre und seinem Dienst, und für die heiligen
  Verrichtungen in seiner Hütte, und so erkennen sie nun die, welche
  seinen Geist haben, in ihrem Dienst am Evangelium. Vieles in
  diesen Versammlungen eignet sich besser für Frauen als für Männer,
  sie sollen darum in der Kraft und Weisheit Gottes die Männer
  über die Dinge, die diese nicht verstehen, belehren, und die
  Männer sollen die Frauen in den Dingen, die die Frauen nicht
  verstehen, belehren\index{Belehrungen}, als gegenseitige Mithelfer. Denn in den Tagen
  des Gesetzes mussten die Frauen so gut wie die Männer opfern,
  wie Vielmehr also sollten sie in den Tagen des Evangeliums ihre
  geistlichen Gaben darbringen, denn sie sind alle, Männer wie
  Frauen, ein königliches Priestertum 
  (1. Petr. 2,9)\bibel{Petr. 1. 02:09@1. Petr. 2:9} genannt, sie
  gehören zu den Genossen des Glaubens, sie sind die lebendigen
  Steine, welche das geistliche Gebäude bilden, dessen Haupt Christus
  ist, und sie sollen im Dienste des Evangeliums ermutigt werden,
  denn alles was sie tun, Männer wie Frauen, sollen sie tun im
  Geiste und in der Kraft Gottes. Alle, die keinen Nutzen in den
  Versammlungen der Frauen sehen oder der Männer, sondern sich
  denselben widersetzen und Streit unter den Freunden anstiften,
  haben den weltlichen Geist, welcher unsern Versammlungen ent-
  gegen ist und sich ihnen widersetzt; sie haben denselben weltlichen
  Geist, welcher gegen das Reden der Frauen in den Versammlungen
  war und noch ist, den Geist derer die sagen, \zitat{die Weiber sollen
  schweigen in der Gemeinde} 
  (1. Cor. 14, 34)\bibel{Cor. 1. 14:34@1. Cor. 14:34}, obgleich der gleiche
  Apostel befiehlt, das die Männer schweigen sollen, so gut wie
  die Frauen, wenn kein Ausleger da sei (14, 28)\bibel{Cor. 1. 14:28@1. Cor. 14:28}. 
  Ihr seht also,
  das der Geist dieser Welt über diese Gegner gekommen ist, wenn
  sie sich schon einen andern Anschein geben; denn sie wollen, das
  % \picinclude{./240-249/p_s248.jpg} 
  wir überhaupt keine Versammlungen haben. Sie sind gegen die
  Versammlungen der Frauen, und etliche auch gegen diejenigen
  der Männer und sagen, sie sehen keinen Nutzen darin. \textbf{Mögen
  sie das Maul halten}\index{Mögen sie das Maul halten} und 
  sich nicht solchen widersetzen, die ihren
  Gottesdienst in diesen Versammlungen sehen [...].
  \bigskip 
  \begin{flushright}Swarthmore, 5. des 8. Monats 
  1676\index{Jahr!1676}. G. F.\end{flushright}
}

Ich verließ Swarthmore am 28. des 1. Monats 1677\index{Jahr!1677} [...]
Später ging ich nach York\ort{York} [...] Von da schrieb ich an meine
Frau folgendes:

\brief{Fell, Margaret}{
Liebes Herz,

\bigskip 

welcher ich liebevollen Gruß sende, sowie deinen Töchtern und
allen Freunden, die nach mir fragen. Es ist mein Wunsch, das
ihr alle bewahret bleibet im ewigen Samen des Herrn, in welchem
ihr Leben und Friede haben werdet, Herrschaft und Wohnort in
der ewigen Heimat, im Hause, das auf Gott gegründet ist. Durch
Gottes Kraft bin ich nach York gelangt, nachdem ich unterwegs
viele Versammlungen gehalten habe. Die Wege waren oft
schlecht und vom Schnee durchweicht gewesen, unsere Pferde
sanken oft ein, so das wir nicht reiten konnten, und oft hatten
wir starken Sturm und Regen, aber durch Gottes Kraft habe ich
alles überwunden. In Scarehouse\ort{Scarehouse} war eine sehr große 
Versammlung, ebenso in Burrowby, in welcher die Freunde von
Durham und Cleveland herbei kamen, und viele andere Versammlungen 
haben wir gehabt. In York hatten wir gestern eine
überaus zahlreiche Versammlung, da die Freunde von weit herum
dazu gekommen waren, alles war ruhig und sehr befriedigt.
O, die Herrlichkeit des Herrn leuchtete über allen! Heute haben
wir eine große Versammlung für Männer und Frauen gehabt,
da viele Freunde, Männer und Frauen, vom Lande gekommen
sind; alles war ruhig, und heute Abend werden wir die Versammlung 
für die Männer und Frauen der Stadt haben. John
Whitehead\person{Whitehead, John} ist hier mit Robert 
Lodge\person{Lodge, Robert} und andern; die Freunde
sind über die Maßen froh. Ich bin also in meinem heiligen
Element, im heiligen Werk für den Herrn, seinem Name sei Ehre
immerdar. Morgen gedenke ich die Stadt zu verlassen und gegen
Tadcaster zu gehen, ich kann nicht reiten wie früher, doch dem
Herrn sei Dank, das ich immerhin so reisen kann. Ich grüß
dich im Brunnen des Lebens, in welchem ihr, so ihr darin bleibet,
Erquickung zum Leben haben werdet, damit ihr wachsen möget
% \picinclude{./240-249/p_s249.jpg} 
und ewige Kräfte sammeln, um dem Herrn zu dienen und Genüge
zu haben. Ich befehle euch alle dem Gott der Kraft, der
allmächtig ist euch zu bewahren.

\bigskip 

\begin{flushright}York, den 16. des 2. Monats 1677. G. F.\end{flushright}
}

Darauf zog ich weiter. [...] Ich bemerkte während meiner
Reise bei manchen, die Vorgaben die Wahrheit zu bekennen, eine
Schlaffheit und Schläfrigkeit im Auftreten gegen das Zehntenwesen;\index{Kirchenseuer} 
denn wo immer der Geist der Spaltung in der Kirche
Eingang fand, schwächte er solche, die ihm Gehör schenkten
im Zeugnis gegen die Zehnten; darum trieb es mich, einen kurzen
Brief ergehen zu lassen, um die reine Gesinnung anzuspornen,
und in allen das christliche Zeugnis gegen des Antichrists\index{Andichrist} 
Bedrückung und Joch zu stärken und zu ermutigen:

\brief{Quaker-Gemeinde}{
  Meine lieben Freunde!

  \bigskip 

  Seid getreu im Herrn in euerm Zeugnis für Christum,
  welcher das levitische zehntennehmende 
  Priestertum\index{Priestertum} Aarons\person{Aaron}
  aufhob und seine Diener aussandte, umsonst zu predigen, wie sie es
  umsonst empfangen hatten, ohne Stab noch Tasche (Math. 10)\bibel{Math. 10}.
  Christi Jünger können nichts mit denen zu tun haben, die ein
  Geschäft aus dem Predigen machen, und gleich wie ein Zeugnis
  abgelegt werden musste gegen jene Zehnten, welche das Gesetz
  für Aaron und Levi\person{Levi} gebot, also muss Zeugnis abgelegt werden
  gegen diese Zehnten, welche von Menschen eingeführt wurden in
  den dunklen Zeiten des Papsttums\index{Papsttum}, und nicht durch Gott oder
  Christus. Nun ist es ein Widerspruch, mit Worten gegen die
  Priester zu schreien und dennoch sie zu unterstützen und zu
  füttern, damit sie nicht Streit anheben sollen. Darum hütet euch,
  denn wenn der Herr euch segnet mit äußeren Gütern, und ihr
  wendet sie den Baalspriestern\index{Baalspriester} zu, so möchte er füglich die
  äußeren Güter, die er euch gab, wieder von euch zurückfordern; sagte
  er doch, das seine Diener umsonst geben sollten, wie sie auch umsonst
  empfangen hätten; darum muss Zeugnis abgelegt werden in der
  Kraft und in dem Geist des Herrn gegen alle, die um Zehnten
  und Geld predigen, und die Zehnten nehmen oder geben, auf
  das alle zusammen stehen mögen zum Zeugnis für Jesus Christus,
  in seiner Kraft und seinem Geist, gegen die Zehnten-Händler.
  Denket daran, wie viele getreue Diener und Kämpfer des Herrn
  ihr Leben ihretwegen gelassen, in den Tagen des Herrn, und wie
  % \picinclude{./250-259/p_s250.jpg} 
  sie in den Tagen der Märtyrer gegen sie gezeugt haben. Denket
  auch daran, welches Gericht über die gekommen ist, welche den
  Freunden ihre Habe geraubt und sie ins Gefängnis getan, um
  der Zehnten und Unterstützungen willen. Darum führet den 
  Krieg\index{führet den Krieg}
  gegen das Tier weiter in der Kraft des Herrn, und füttert es
  nicht, nur damit es euch \zitat{Friede!} zurufen solle; solchen Frieden
  sollt ihr nicht annehmen, sondern ihr müsset ihn brechen und 
  verwerfen durch den Geist Gottes. Dann werdet ihr in diesem selben
  Geist vom Sohn des Friedens; jenen Frieden empfangen, den
  weder das Tier\index{Das Tier}, noch die Hure, noch die Welt mit allen ihren
  irdischen Lehrern empfangen können und ihn euch auch nicht
  rauben können. Darum bewahret eure Herrschaft und Macht in
  der Kraft, dem Geist und dem Namen Jesu. Ich grüße euch in
  seiner Liebe.

\bigskip 
\begin{flushright}3. Monat des Jahres 1677.\index{Jahr!1677} G. F.\end{flushright}

}
% \picinclude{./250-259/p_s250.jpg} 
Zur Jahresversammlung kamen zahlreiche Freunde aus
allen Teilen des Landes, etliche auch von Schottland\index{Schottland}, Holland\index{Holland}
und andern Ländern, und wir hatten gar herrliche Versammlungen,
in denen die mächtige Gegenwart des Herrn reichlich gespürt
wurde; und die Wahrheit machte gute Fortschritte in der Einigkeit\index{Einigkeit}
des Geistes zur Freude und Stärkung der Aufrichitgen; 
gelobt sei der Herr immerdar! [...]

Nachdem ich, nach der Jahresversammlung, etwa eine Woche
mit Freunden in London zugebracht, ging ich mit William Penn\person{Penn, William}
in sein Haus nach Sussex\ort{Sussex} [...]

Dann blieb ich etwa drei Wochen in Worminghurst, in
welcher Zeit John Burnyeat\person{Burnyeat, John} 
und ich eine Entgegnung auf ein \index{Verteidigungsschrift}
sehr böswilliges Buch schrieben, welches Roger Williams,\person{Williams, Roger} 
ein Priester in Neu-England, gegen die Wahrheit und die Freunde
geschrieben hatte [...]

Dann gingen wir nach Kingston\ort{Kingston} und dann nach London\ort{London}, wo
ich aber nicht lange blieb. Denn es kam über mich vom Herrn,
nach Holland\ort{Holland} zu gehen, um die Freunde dort zu besuchen, und
das Evangelium dort zu predigen, sowie auch in einigen Teilen
in Deutschland\ort{Deutschland}. Darum rüstete ich so schnell wie möglich alles
zur Abreise und nahm Abschied von den Freunden in London,
und ging mit einigen andern Freunden nach Colchester [...] und
von da nach Harwich [...]
\input{./kap_22}
\input{./kap_23}
\input{./kap_24}
\input{./kap_25}

%%%%%%%%%%%%%%%%%%% Kapitel 26. %%%%%%%%%%%%%%%%%%%%%%%%%%%%%%
\chapter[Jakob II. Amnestie]{Jakob II. Amnestie}

\begin{center}
\textbf{Kampf für die Ordnung im Quäkertum. Jakob II. Amnestie.}
\end{center}


Nachdem ich einige Wochen in South-Street\ort{South-Street} gewesen war
und dort manche Versammlungen für die Freunde gehalten hatte,
kehrte ich nach London\ort{London} zurück. Hier half 
ich unter andrem den
Freunden ein Zeugnis\index{Zeugnis} aufsetzen, um sie 
von dem Verdachte zu
reinigen, sie hätten sich am letzten Aufstand im Westen oder an
irgend andern Ausständen oder Verschwörungen gegen die Regierung
beteiligt. Und dieses Zeugnis wurde dann dem obersten Richter
eingereicht, der im Begriff war, nach dem Westen zu gehen, um
die Gefangenen zu verhören.

Ich blieb einige Zeit in London und arbeitete im Dienst der
Wahrheit. Dann ging ich für etwa eine Woche wieder aufs
Land, weil meine Gesundheit unter dem Mangel an frischer Lust
sehr litt [...] und kehrte dann wieder in die Stadt zurück, wo
ich während zwei Monaten die Versammlungen besuchte und mein
Möglichstes tat, um für die Freunde, die in allen Teilen des
Landes viel zu leiden hatten, Erleichterungen zu erwirken. Auch
schrieb ich mehrere Schriften zur Förderung der Wahrheit. Die
eine handelte von der Ordnung in der Kirche Gottes, der sich
etliche unter den Freunden stark 
widersetzt\index{Konflikt!unter Quakern} hatten. Sie lautete:

\grosszitat{
Überall in der Welt besteht für Familie, Gesellschaft oder Stadt
irgend eine Ordnung. Im alten Testament war es die Ordnung\index{Ordnung!der Zusammenlebens}
Arons und Melchisedeks (Hebr. 7:11\bibel{Hebr. 7:11}) und 
danach die Ordnung Jesu
Christi, und er verachtete diese Ordnung nicht. Gott ist ein Gott
der Ordnung in seiner ganzen Schöpfung, so auch in seiner Kirche.
Und alle, die an das Licht glauben, an das Leben in Christus,
durch das man vom Tod ins Leben eingeht, sind in der Ordnung
des heiligen Geistes, und im Licht und Leben, der Kraft und dem
Reich Jesu Christi\index{Reich Gottes}, deren Wachstum 
kein Ende nimmt. Aber
solches ist verborgen den Geistern der Unordnung, die so viel
% \picinclude{./290-299/p_s299.jpg} 
schreiben und drucken gegen die Ordnung, die der Herr durch
seinen Geist und seine Kraft unter seinem Volke aufgerichtet hat.

Ihr, die ihr so viel gegen die Ordnung schreit, ihr seid ja
in ein \zitat{Land der Finsternis\index{Finsternis} und 
des Dunkels geraten; ein Land,
da es stockfinster ist und da keine Ordnung ist, und wo es ist
wie Finsternis, wenn es hell wird} 
(Hiob 10:21\bibel{Hiob 10:21}). Ist nicht
dies euer Zustand, wie alle, die in der Wahrheit und nach dem
Evangelium des Lebens und des Heils wandeln, sehen können? [...]
\bigskip
\begin{flushright}
G. F.\end{flushright}
}


Ich konnte abermals nicht lange in London bleiben, da ich
die Eingeschlossenheit in der Stadt nicht lange hintereinander
ertragen konnte [...]\index{Stadtleben}. Ehe ich die Stadt 
verließ, hörte ich von
einem berühmten Gelehrten aus Polen, der kürzlich hergekommen
war; ich lud ihn in meine Wohnung ein und hatte eine lange
Unterredung mit ihm. Nachdem ich mich über alles, was ich zu
wissen wünschte, erkundigt hatte, schrieb ich einen Brief an den
König von Polen, wegen der Freunde in Danzig, die lange
schwer zu leiden gehabt hatten. Es folgt hier eine Abschrift
davon:

\grosszitat{
    An den König von Polen.
    \bigskip
    An Johann den Dritten\person{Johann III.}, König 
    von Polen\person{König von Polen}, Großherzog von
    Litauen\ort{Litauen}, Russland\ort{Russland} und Preußen\ort{Preußen}, 
    Beschützer der Stadt Danzig, [...]
    wegen der heimgesuchten und unschuldigen Leute, die man im
    Groll Quäker nennt, die jetzt bei Wasser und Brot in oben
    genannter Stadt sind, in strenger Gefangenschaft, wo man ihren
    Frauen und Kindern kaum erlaubt, sie zu besuchen.
    \bigskip
    O König!
    \bigskip
    Die Behörden der Stadt Danzig\ort{Danzig} sagen, es sei dein Wille,
    das dieses unschuldige, heimgesuchte Volk solche Unterdrückung zu
    erleiden habe. Nun ist die Strafe nur darum über sie verhängt
    worden, weil sie zusammenkommen im Namen Jesu Christi ihres
    Erlösers und Heilands, der für ihre Sünden starb und zu ihrer
    Rechtfertigung von den Toten auferstanden ist, der ihr Prophet
    ist, welchen Gott erweckt hat, wie Moses
    Und nun, in diesen Tagen des neuen Evangeliums und des
    neuen Bundes, sollten alle auf ihn hören, die \zitat{gewesen wie die
    irrenden Schafe, nun aber sich bekehrt haben zum Hirten und
    Bischof ihrer Seelen. Er hat sein Leben gegeben für seine
    Schafe, und sie hören seine Stimme und folgen ihm und er führt
    sie auf seine Weide} (Joh. 10:9\bibel{Joh. 10:09@Joh. 10:9}).
    % \picinclude{./300-309/p_s300.jpg} 
    Ich hörte, O König, du bekennest dich öffentlich zum Christen-
    tum und zum mächtigen Namen Jesu Christi, deö Königö der
    Könige, deß Herrn der Herren, dem alle Gewalt im Himmel und
    auf E-rden gegeben ist, der alle Völker mit eisernem Szepter
    regiert. GZ scheint uns darum hart, o König, das; jemand, der
    offen Christuö bekennt, solche Strafen über ein harmloseß und
    unschuldigeß Volk verhängt, nur weil sie zarte Gewissen haben
    und zusammen kommen, um den ewigen Gott, der sie gemacht
    hat, im Geist und in der Wahrheit anzubeten, wie Christuß etz vor
    1600 Jahren eingesetzt hat nach Joh. 4, 23. 24.
    Jch bitte nun den König, darüber nachzudenken, ob Christus
    im neuen Testament je seinen Aposteln ein Gebot gegeben, sie
    sollten jemand ins- Gefängniö werfen bei Wasser und Brot, der
    sich nicht in allen Stücken ihrer Religion, ihrem Glauben und
    ihrer Art der Anbetung anschloß? Wo haben die Apostel nach
    der Himmelfahrt in der wahren Kirche solcheß getan,? Lehren
    nicht Christue und die Apostel, seine Nachfolger sollen die Feinde
    lieben, und bitten für die, welche sie hassen, verfolgen und ver-
    leumden? (Matth. 5.)
    Jst etz nicht eine Schande für das Christentum den Türken
    und andern gegenüber, daß ein Christ den andern verfolgt um
    der Glarfenölehre, der Art der Anbetung und der Religion
    willen? Sie können nicht beweisen, daß Christuz, den sie ihren
    Herrn und Meister nennen, je ein solches Gebot gegeben. Christu-3
    sagt, seine Nachsolger sollen sich unter einander lieben, daran
    werde man erkennen, daß sie seine Jünger seien (Joh. 13). Und
    hat nicht Christuö jene getadelt, die wollten Feuer vom Himmel
    regnen lassen, um alle zu verderben, die ihn nicht aufnehmen
    wollten?« (Luc. 9). Gr sagte zu ihnen: »Wisset ihr nicht, weö
    Geistes Kinder ihr seid? Wissen je die, welche die Menschen
    verfolgen oder töten, weil sie eine Religion nicht annehmen
    wollen, weß Geisteö Kinder sie sind? Wäre ez nicht gut,
    wenn alle durch den Geist Christi wüßten, weß Geistes
    Kinder sie sind? Denn der Apostel sagt, Römer 8, 9: ,,Wer
    Christi Geist nicht hat, der ist nicht sein«; und 2. Cor. 10, 4:
    ,,Die Waffen unsrer Riiterschaft sind nicht sleischlich sondern geist-
    lich«; und Epl). 6, 12: ,,Wir haben nicht mit Fleisch und Blut zu
    % \picinclude{./300-309/p_s301.jpg} 
    Kampf für die Ordnung im Quäkertum. Jakobyll. Amnesiie. 301
    kämpfen, sondern mit den bösen Geistern unter dem Himmel.«
    Daraus- ersehen wir, daß der Kampf der ersten Christen und
    ihre Waffen geistiger Art waren. Würde eß dem König und den
    Behörden von Danzig nicht gegen daß Gewissen gehen, wenn die
    Türken sie zu ihrer Religion zwingen würden? oder wenn die
    Behörden von Danzig zur Religion det-’ Königö von Polen ge-
    zwungen würden? oder würde etz der König von Polen nicht
    grausam und gegen sein Gewissen finden, wenn er zur Religion
    der Behörden von Danzig gezwungen würde? und im Fall sie
    sich derselben nicht unterwerfen wollten, von Weib und Kindern
    getrennt und auz dem Lande verbannt, oder bei Wasser und Brot
    inö Gefängniß geworfen würden ?
    Wir bitten darum den König und die Behörden in aller
    christlichen Demut, daß sie in dieser Angelegenheit nach dem
    königlichen Gesetz Gotteß vorgehen möchten, nämlich ,,andern zu
    tun, waö sie möchten, daß man ihnen tue« (Jak. 2, 8), ,,rmd
    ihren Nächsten zu lieben als- sich selbst« (Matth. 22, 39). Denn
    wir hoffen und glauben denn doch, daß sowohl der König von
    Polen und seine Leute, altz auch die Behörden von Danzig, die
    Schriften deß neuen Testamentß sowie deö alten, kennen; und wir
    bitten darum den König und seine Räte, darauf zu achten, daß
    sie nicht dem königlichen Gesetz Gotteß und dem herrlichen und
    ewigen Evangelium der Wahrheit entgegen, ein unschuldigeß Volk
    gefangen nehmen, nur weil etz zusammen kommt mit zarten Ge-
    wissen, um Gott seinem Schöpfer zu dienen und ihn anzubeten.
    Wir bitten den König in christlicher Liebe, dieseö alles ernst-
    lich und eingehend zu bedenken, und Befehl zu geben, daß die
    unschuldigen Gefangenen, unsere Freunde, die sogenannten Quäker,
    frei gelassen werden autz ihrer harten Gefangenschaft in Danzig,
    daß sie frei sein mögen, den lebendigen Gott im Geist und in der
    Wahrheit anzubeten und ihm zu dienen, und heimzugehen um
    ihr Handwerk weiter zu treiben und ihre Familien zu erhalten.
    Wir glauben, daß der König, wenn er solch ein edles, ruhm-
    oolleß, ja christlicheö Werk tut, nicht unbelohnt bleiben wird
    von dem großen Gott, dem wir dienen, der die Herzen der
    Könige und ihr Leben und die Länge ihrer Tage in seiner
    Hand hat.
    Von Einem, der möchte, daß der König und alle seine
    Räte in der Furcht de?. Herrn bewahrt bleiben mögen und
    % \picinclude{./300-309/p_s302.jpg} 
    sein Wort der Weisheit annehmen, durch welches alle Dinge ge-
    schaffen wurden. [...]
    \bigskip
    \begin{flushright}
    London, 10. des 3. Monats, den man pflegt Mai zu nennen, 1684.
    G. F.\end{flushright}
}


Jch schrieb in dieser Zeit noch manches andere im Dienst
der Wahrheit; etwas »über das Richten,« denn etliche, die von
der Wahrheit abgesallen waren, hatten eine solche Angst, von ihr
gerichtet zu werden, daß sie sich eifrig bemühten, gegen das Richten
zu schreien. Darum erließ ich ein Schreiben, um aus der Schrift
der Wahrheit zu beweisen, daß die Kirche Christi Macht hat, alle,
die vorgeben dazu zu gehören, nicht nur in Dingen dieser Welt,
sondern auch in religiösen Dingen, zu richten. ....
Bei den Verhören in den Gerichtssitzungen, im 2. Monat 1686
in Hicks Hall, wurden viele Freunde vorgenommen; ich war
täglich bei ihnen, um zu raten und zu helfen, damit nichts ver-
säumt und kein Vorteil unbenutzt bleibe; und gewöhnlich hatten
sie guten Erfolg. Bald daraus gefiel es dem König, nachdem
wir ihm immer wieder Klagen über unsere Leiden vorgelegt hatten,
zu befehlen, daß man: »alle die um des Gewissens willen gefangen
waren, sreilasfe, soweit er Macht habe es zu bestimmen.« Die
Türen der Gesängniss e taten sich denn auch aus, und Viele hundert
Frermde, von denen manche lange gefangen gewesen waren, er-
hielten die Freiheit. Viele oon ihnen kamen zur Jahresversamnn
lung, zur großen Freude der Freunde. ....
Jch brachte den größten Teil des Jahres 1686 in London
zu, außer wenn ich nach Bethnal-Green oder Enfield ging, oder
nach Chiswick, wo ein Freund eine Schule errichtet hatte, in der
Kinder von Freunden erzogen wurden.
.... Auch schrieb ich noch allerlei in diesem Jahr, unter anderm
eine Ermahnung » an die Freunde, in der Einigkeit der Wahrheit zu
bleiben, in welcher keine Trennung noch Gntzweiung ist.« ....
Bald daraus, als ich merkte, daß etliche Abtrünnige, welche
der Feind zur Trennung und Spaltung von den Freunden geführt
hatte, fortsuhren in ihrem Schreien und ihrem Widerstand gegen
unsre Monats-, Vierteljahres- und Jahresversammlungen, so trieb
es mich, einen kurzen Brief an die Freunde zu schreiben, um sie zu
erinnern, daß sie durch den Geist des Herrn in ihrem Innern
die Bestätigung und Besiegelung empfangen hatten, daß diese
Versammlungen vom Herrn seien und von ihm angenommen


% \picinclude{./300-309/p_s303.jpg} 
Wirken in London unter dem Zeichen der Toleranz. 303
werden, und daß sie darum nicht von den Gegnern erschüttert
werden können:
,,Meine lieben Freunde im Herrn Jesus Christus,
Jhr alle, die ihr in seinem heiligen Namen versammelt seid,
wisset, daß eure Versammlungen, die oierteljährlichen wie die
andern, durch die Kraft und den Geist Gottes eingesetzt sind.
Sie sind in euern Herzen von dieser Kraft und diesem Geist be-
zeugt, durch die Kraft und den Geist Gottes sind sie in euch gegründet
und ihr in ihnen. Gott der Herr hat es euch durch seinen Geist
besiegelt, daß eure Versammlungen nach seiner Ordnung und
Einberufung geschehen, und er hat sie anerkannt, indem er euch
mit seiner gesegneten Gegenwart in denselben begnadete; ihr
habt es reichlich erfahren, wie er euch mit seinem Leben, seiner
Weisheit und Kraft und mit himmlischen Gütern aus seinen
Schätzen und Quellen ausriistete, und aus denselben sind wiederum
viele Danksagungen und Lobpreisungen in euern Versammlungen
zu seinem heiligen Namen zurückgekehrt. Er hat euch eure Ver-
sammlungen durch seinen Geist besiegelt, und daß euer Zusammen-
kommen im Herrn geschehen ist, in Christus seinem Sohn und in
seinem Namen und nicht durch Menschen. Darum gebühret dem
Herrn, daß er durch sie und in ihnen gepriesen werde, der euch
und sie beschützt hat mit seinem mächtigen Arm gegen alle Gegner
und Feinde und ihre verleumderischen Zungen und Bücher. Denn
des Herrn Macht und Same regieret über alles; er erhält seine
Kinder zu seiner Ehre, als die so »Obmacht und Hoheit haben
am Ehrentage« (Ps. 110, 3) .... aus daß alle dem Herrn in
Jesus Christus dienen von Geschlecht zu Geschlecht.«
London, den 3. des 11. Monats 1686. G. F.

\input{./kap_27}
\input{./kap_28}
% \picinclude{./000-009/p_s001.jpg}
\section{Kapitel 1}

\begin{center}
\textbf{Erweckung und Krisiz bis zum Durchbruch.}
\end{center}


Auf daß Jedermann wisse, was der Herr an mir getan, und
sehe, wie Er mich durch mancherlei Prüfungen, Versuchungen und
Trübsale führte, um mich für daß Werk, für daß; Er mich bestimmt 
hatte, vorzubereiten und auszurüsten, und dadurch getrieben 
werde, seine unendliche Güte und Weisheit anzubeten und
zu preisen — so will ich kurz berichten, wie es in meiner Jugend
um mich stand, und wie das Werk des Herrn in mir angefangen
und fortgesetzt wurde seit meiner Kindheit.


Ich wurde geboren im Monat den man Juli nennt\footnote{Fox 
verwarf die üblichen Monatsbezeichnungen als heidnisch.} 1624,
zu Drayton in-the-Clay, in Leicestershire. Mein Vater hieß
Christoph Fox; er war Weber von Beruf, ein ehrbarer Mann,
und es war ein "`Same von Gott"` in ihm. Die Nachbarn
nannten ihn: den "'gerechten Chrtster"'. Meine Mutter war eine
rechtschaffene Frau; ihr Mädchenname war Mary Lago, aus der
Familie der Lago und aus dem Geschlecht der Märtyrer.

In meiner frühesten Kindheit war ich so ernsten und gesetzten
Gemütez, wie es bei Kindern selten ist, so daß, wenn ich Erwachsene 
leichtfertig und ausgelassen mit einander tun sah, ich
einen Abscheu davor in meinem Herzen verspürte und zu mir
sagte: "`Wenn ich einmal ein Mann sein werde, sicherlich werde
ich nicht so leichtfertig tun."'

A1s ich elf Jahre alt war, wußte ich schon was rein und
recht ist; denn ich war als Kind gelehrt worden, wie man rein
bleibt. Der Herr lehrte mich, treu zu sein in allen Dingen, sowohl
innerlich gegen Gott als äußerlich gegen die Menschen; und daß
ich mich in allen Dingen an "`ja"` und "`nein"` halten solle; nicht
wie die Kinder der Welt, die ihren Mund voll List und gleißnerischer
Worte haben, sondern meine Worte sollen: wenig sein, "'lieblich


\picinclude{./000-009/p_s002.jpg}


und mit Salz gewürzet"` (Col. 4, 6); und daß ich nicht essen
und trinken solle, um mich wollüstig zu machen, sondern um der
Gesundheit willen, jeder Ding dazu gebrauchend, wozu es be-
stimmt ist, zur Ehre dessen, der alleß geschaffen hat ....
Alß ich dann heranwuchß, wollten meine Angehörigen einen
Priester 1) aus mir machen. Aber andere rieten zu anderm; so
kam ich zu einem, der seineö Zeichenz ein Lederhändler war, aber
mit Wolle handelte und Vieh züchtete und verkaufte; und es ging
mancherlei durch meine Hände. . Während ich bei ihm war,
war er gesegnet; aber nachdem ich ihn Verlassen, ging ez ihm
schlecht und er getiet in Verfall. Während dieser ganzen Zeit
tat ich weder gegen einen Mann noch gegen eine Frau etwas-
Unrechteö; denn die Kraft dez Herrn war mit mir und bewahrte
mich. Während ich in diesem Dienste stand, gebrauchte ich im
Verkehr daö Wort "'wahrlich"', und es war eine übliche Redenöart
bei meinen Bekannten: wenn George sagt "`wahrlich"`, so kann
ihn nichts umstimmen. Wenn die Buben oder rohe Leute über
mich lachten, kümmerte ich mich nicht um sie, sondern ging meiner
Wege; aber gewöhnlich hatten mich die Leute gem wegen meiner
Geradheit und Ehrlichkeit.
Alß ich, noch nicht ganz neunzehnjährig, in Geschäften an
einem Jahrmarkt war, kam mein Vetter, namenö Bradford, ein
"'Frommer"' cpt0keS801·) und mit ihm noch ein anderer "`Frommer"'
und forderten mich auf, mit ihnen einen Krug Bier zu trinken,
und da ich durstig war, ging ich mit ihnen hinein; denn ich
liebte jeden, der Sinn für daß Gute hatte und den Herrn
suchte. A13 jeder ein Glas getrunken hatte, fingen sie an, sich
zuzutrinken und verlangten noch mehr, indem sie aus-machten,
daß der, welcher nicht trinken würde, alletz bezahlen sollte. GS
betrübte mich, daß jemand, der sich für religiöß außgab, solchetz
tat; sie taten mir sehr weh, denn ee; war mir dergleichen noch
nie vorgekommen bei keiner Art von Menschen; darum stand ich
aus um zu gehen, indem ich meine Hand in die Tasche steckte,
einen Groschen vor sie aus den Tisch legte und sagte: "`wenn ez
so ist, will ich euch Verlassen."' So kehrte ich nach Hause zurück,
aber ich ging in jener Nacht nicht zu Bett, denn ich konnte nicht
schlafen; bald ging ich im Zimmer auf und ab, bald betete und
1) Fox bezeichnet mit priest die ordinierteu Geistlichen.


\picinclude{./000-009/p_s003.jpg}


schrie ich zum Herrn, welcher also zu mir redete: "`Tu siehst, wie
junge Leute zusammengehen in Eitelkeit und alte Leute in die
Erde. Du mußt dich von ihnen abwenden und dich von ihnen,
den jungen wie den alten, fern halten und ihnen allen ein
Fremdling werden."'
Darauf, am 9. Tage dee; 7. Monatz 1643, verließ ich nach
Gottes Befehl meine Verwandschaft und brach allen Umgang und
alle Kameradschaft mit jung und alt ab. Ich begab mich nach
Lutterworth, wo ich einige Zeit blieb und von da ging ich nach
Northampton, wo ich mich ebensallß aufhielt; darauf nach New-
port Pagnell, von wo ich nach einiger Zeit weiter nach Barnet
ging, im 4. Monat 1644. Als ich nun so da; Land durchzog,
wurden die "`Frommen"` (prokeesore) auf mich aufmerksam und
wollten mich kennen lernen. Aber ich mied fie; denn ich spürte,
daß sie nicht besaßen, was sie bekannten (proteezeä). Während
der Zeit, da ich in Barnet war, kam eine große Anfechtung zu
verzweifeln über mich. Ich sah, wie Chriftuö versucht worden
war, und war in großer Not; bald ging ich nicht aus meinem
Zimmer, und bald wanderte ich einsam durch die Fluten, um auf
den Herrn zu warten.
Ich fragte mich, warum mir solcheö widersahren müsse? Ich
prüfte mich und sagte zu mir selber: "'War ich je zuvor so ge-
wesen?"` Ich dachte, ich hätte mich vielleicht gegen meine An-
gehörigen verfehlt, weil ich sie verlassen hatte. Ich mußte immer-
während darüber nachdenken, daß ich solches getan hatte, und
mich fragen, ob ich einem von ihnen ein Unrecht getan hätte;
aber die Anfechtung wurde schwerer und schwerer, und ich wurde
bis zur Verzweiflung versucht. Und weil Satan sein Vorhaben
auf diese Weise nicht erreichte, so legte er mir Fallstricke und
Lockungen, damit ich eine Sünde begehen möchte, die er auß-
nützen könnte, um mich zur Verzweifluug zu bringen. Ich war
etwa 20 Iahre alt, als diese Prüfungen über mich kamen, und
die Angst dauerte mehrere Jahre und ich hätte mich gerne davon
frei gemacht. Ich ging zu manchem Priester, um Trost zu suchen,
aber ich fand keinen bei ihnen.
Von Barnet ging ich nach London, wo ich eine Wohnung nahm,
und dort war ich in großem Elend und Iammer; denn ich sah
daß die großen "'Frommen"' der Stadt alle in den Banden der
Finfterniß waren. Ich hatte einen Oheim dort, einen Baptisten,



\picinclude{./000-009/p_s004.jpg}

die waren damals gottselig (temier); dennoch konnte ich ihm meine
Stimmung nicht kundtun, noch mich ihm anschließen, denn ich
dUWhsch11Uke celle, jung und alt und wie ez um sie stand. Etliche
gottselige Leute (tenrler people) hätten mich gern dort behalten,
aber ich getraute mich nicht und wandte mich wieder gegen
Leicestershire; der Gedanke, ich könnte meinen Eltern und Ange-
hörigen weh tun, bedrückte mich; denn sie waren, wie ich merkte,
betrübt über meine Abwesenheit.
A15 ich nach Leieesterfhire kam, wollten meine Leute, daß ich
mich oerheirate; aber ich sagte ihnen, ich sei noch ein Knabe und
müsse weise werden. Andre hätten mich gerne bei der Hilfßtruppe
im Militär 1) gesehen, aber ich weigerte mich; und eß betrübte mich,
daß sie mir solche Dinge vorschlugen, dennsich war ein gottseliger
(temier) Jüngling. Darauf ging ich nach Eoventry, wo ich auf
einige Zeit ein Zimmer im Hause eineö "`Frommen"` hatte, bis
die Leute ansingen mich zu kennen; denn ez waren viele gott-
selige Leute in jener Stadt. Nach einiger Zeit ging ich wieder
in meine Heimat und blieb etwa ein Jahr dort, in großer Trüb-
sal; während mancher Nacht irrte ich einsam umher.
Dornach kam der Priester von Drayton, Nathanael Steoenz,
oft zu mir und ich ging oft zu ihm; und ein anderer Priester
kam oft mit ihm und sie verschmiihten nicht, mich anzuhören; ich
stellte ihnen Fragen und di?-kutierte mit ihnen. Dieser Priester
Stevenö stellte mir folgende Frage: warum Christuß am Kreuz
gerufen habe: "'mein Gott, mein Gott, warum hast du mich oer-
lassen?"` und warum er gesagt habe: "'wenn etz möglich, so gehe
dieser Kelch an mir vorüber, aber nicht wie ich will, sondern wie
du willst."` Jch erwiderte ihm, daß zu der Zeit die Sünde der
ganzen Menschheit auf ihm gelegen habe und er ihre Missetat
und Übertrettmg tragen und für sie geopfert und verwundet
werden mußte, sofern er Mensch war; aber er starb nicht, sofern
er Gott war; und weil er so für alle starb und den Tod schmeckte
für jeden Menschen, wurde er zum Opfer für die Sünden der
ganzen Welt. So sprach ich, weil ich zu jener Zeit gewisser-
maßen die Leiden Christi, und maß er durchgemacht, an mir
nachempfand. Der Priester sagte auch, es sei eine sehr treffende
Antwort, eine, wie er sie noch nie gehört habe. Zu jener Zeit
1) E3 war det Anfang der Bürgerkeiege.


\picinclude{./000-009/p_s005.jpg}

pflegte er mich zu loben und anerkennend von mir zu andern zu
sprechen; und das, war- ich ihm während der Woche im Gespräch
mitteilte, predigte er dann am \textit{Ersten Tage} 1); deöwegen mochte
ich ihn nicht leiden. Später wurde dieser Priester ein großer
Verfolger.
Darauf ging ich zu einem andern Priester in Mancetter in
Warwickshire und diökutierte mit ihm über den Grund der Ver-
suchungen und der Verzweiflung, aber er verstand meinen Zustand
nicht; er riet mir, zu rauchen und Psalmen zu singen; nun mochte
ich aber den Tabak nicht und zum Psalmensingen war ich nicht
aufgelegt; ich konnte nicht singen. Er lud mich ein, wieder zu
kommen; dann wolle er mir manches sagen; aber als ich kam,
war er ärgerlich und verdrießlich, weil meine früheren Worte ihm
mißfallen hatten. Gr redete mit seinen Dienstboten über meine
Leiden und Bekümmernisse, und ich bereute, einem solchen meine
Gesinnung aufgedeckt zu haben. Jch sah, daß sie alle leidige
Tröster (Hiob 16, 2) waren, und sie machten meine Unruhe noch
größer. Darauf hörte ich von einem Priester, der in der Nähe
von Tamworth lebte und für einen erfahrenen Mann galt. Ich
ging sieben Meilen weit zu ihm, aber ich fand, daß er nur ein
leereö, hohletz Gefäß war. Auch von einem Or. Cradock in Eoventry
hörte ich und ging zu ihm. Ich befragte ihn über Versuchung
und Verzweiflung und wie die Ansechtungen wohl über den
Menschen kommen, Gr fragte mich, wer Jesu Mutter und Vater
gewesen seien? Jch entgegnete, Maria sei seine Mutter gewesen
und er gelte als der Sohn Joseph?-, aber er sei der Sohn Gotteö.
Wir gingen gerade auf einem schmalen Weg in seinem Garten
und beim Umdrehen trat ich mit dem Fuß auf den Rand einetz
Veeteß, worüber der Mann in Wut geriet, als- ob sein Haut; in
Flammen stünde, und unsere ganze Unterredung war gestört und
ich ging in Vekümmerniz hinweg, bekümmeiter alß ich gekommen
war. Ich sah, daß sie alle leidige Tröster waren und so viel
wie nichtö für mich, denn sie konnten sich nicht in meinen Zustand
versetzen. Daraufhin ging ich zu einem, namens Macham, einem
Priester von hohem Ansehen. Gr verordnete mir Arznei und ich
mußte zu Ader lassen. Aber man konnte mir keinen Tropfen Blut
entziehen, weder am Arm noch am Kopf, trotz aller Mühe, die
Iii Fox hat den Grundsatz, statt Sonntag, Erster Tag zu sagen, da etz für
. ihn keine heiligen Tage gibt.


\picinclude{./000-009/p_s006.jpg}

man sich gab, weil mein Körper wie ausgetrocknet war
durch Kummer, Unruhe und Jammer, die so schwer auf mir
lagen, daß ich hätte wünschen können, gar nicht oder blind ge-
boren zu sein, damit ich nie die Schlechtigkeit und Eitelkeit der
Welt gesehen hätte, oder taub, daß ich nie eitle und böse Worte
gehört hätte, und wie der Name des Herrn gelästert wurde. Als
die Zeit, die man Weihnacht nennt, kam, ging ich, während andere
sich belustigten und sichs wohl sein ließen, von Haus zu Haus zu
armen Witwen und gab ihnen Geld. Wenn ich zu Hochzeiten einge-
laden war, wie zuweilen geschah, ging ich nie hin, sondern machte
erst am folgenden Tage oder bald darauf einen Besuch, und wenn
die Leute arm waren, gab ich ihnen Geld; ich besaß davon ge-
rade so viel, daß ich niemanden zur Last zu fallen brauchte und
noch dem Dürftigen etwas spenden konnte.
Zu Anfang des Jahres 1646 als-—ich, auf dem Wege nach
Cooentry, mich den Toren der Stadt näherte, stieg die Frage in
mir auf, wie man sagen könne: alle Christen seien Gläubige, so-
wohl Papisten als Protestanten; und der Herr Offenbarte mir,
daß, wenn alle Gläubige wären, so wären sie alle aus Gott ge-
boren und oom Tode zum Leben durchgedrungen (1. Joh. 3,:;
nur solche seien wahre Gläubige; und wenn auch andere sagen,
sie seien auch wahre Gläubige, so seien sie es doch nicht.
Gin andermal, als ich am Morgen eines Ersten Tages über
ein Feld ging, offenbarte mir der Herr, daß in Oxford oder Cam-
bridge erzogen sein noch nicht genüge, um tüchtig nnd fähig zum
Dienst Christi zu machen; ich verwunderte mich darüber, denn
das war die allgemeine Meinung der Leute. Aber ich sah es
vollständig ein, als der Herr es mir offenbarte und war über-
zeugt davon nnd pries die Güte Gottes, die mir solches an diesem
Morgen geoffenbart hatte. EH griff das Amt des Priesters Stevens
an, daß: ,,in Oxford oder Cambridge erzogen zu sein noch nicht
genüge, um tüchtig und fähig zum Dienst Christi zu machen"'; es
wurde mir klar, daß da-3, was mir geoffenbart worden war, das
priesterliche Amt angreife. Meine Angehörigen waren sehr betrübt,
daß ich nicht mit ihnen kommen wollte, um den Priester zu hören.
Jch ging eben lieber allein ins Freie mit einer Bibel. Ich fragte sie,
ob nicht der Apostel zu den Gläubigen sage, "'sie bedürfen nicht, daß
sie jemand lehre, die Salbung lehre sie"` (1. Joh. 2). Aber wie-
wohl sie wußten, daß solches in der Schrift steht, und daß es


\picinclude{./000-009/p_s007.jpg}

wahr ist, waren sie doch betrübt, daß ich mich in diesem Punkte
nicht unterwerfen und mit ihnen den Priester anhören konnte.
Ich sah ein, daß es ein ander Ding ist, ein wahrer Gläubiger
zu sein, als das worauf es diesen ankommt ...... Warum sollte
ich also diesen anhängen? Weder diesen noch irgendwelchen
Dissentern konnte ich mich anschließen, sondern war allein, ein
Fremdling, und hielt mich einzig an den Herrn Jesus Christus.
Ein andermal hatte ich die Offenbarung, daß Gott, der die
Welt gemacht hat, nicht in Tempeln mit Händen gemacht wohne.
Dies schien mir zuerst ein seltsames Wort, denn sowohl die
Priester als auch das Volk pflegten ihre Tempel oder Kirchen
\textit{Stätten der Ghrsurcht}, \textit{heiliger Boden} und \textit{Tempel Gottes}
zu nennen. Aber der Herr zeigte mir deutlich, daß er nicht
in diesen Tempeln wohne, die von Menschen verordnet und
ausgerichtet waren, sondern in den Herzen der Menschen. Denn
sowohl Stephanus als der Apostel Paulus gaben Zeugnis,
daß er nicht in Tempeln mit Händen gemacht wohne (Art. 7, 48),
nicht einmal in demjenigen, den er einst zu bauen befohlen hatte,
sintemal er ihm ein Ende gemacht hatte, sondern sein Volk sei
sein Tempel, und da wohne er. Solches wurde mir geoffenbart,
während ich durchs Feld zu den Meinigen ging. Als ich kam,
sagten sie mir, Priester Stevens sei dagewesen und habe gesagt,
er sei besorgt um mich, weil ich neuen Lichtern nathgehe. Jch
lächelte bei mir selber, im Gedanken was der Herr mir über ihn
und seinesgleichen geoffenbart hatte. Aber ich sagte meinen Ver-
wandten nichts davon. Denn obgleich sie den Priester durch-
schauten, gingen sie doch, ihn zu hören, und waren betrübt, daß
ich nicht auch ging. Aber ich kam ihnen mit Schriststellen und
zeigte ihnen, daß es eine Salbung gibt im Menschen, die ihn
lehrt, und daß der Herr sein Volk selber lehren will. Ich hatte
auch große Ossenbarungen über das, was in der Mokalypse steht;
wenn ich davon redete, so sagten die "'Frommen"' und die Priester,
sie sei ein versiegeltes Buch, und wollten mich davon abbringen;
aber ich sagte ihnen, Christus könne die Siegel öffnen und sie
sei das, was uns am nächsten angehe; denn die Briefe seien an
die Heiligen früherer Zeiten gerichtet, aber die Apokalypse handle
von den künftigen Dingen.
Ich traf mit Leuten zusammen, welche die Ansicht hatten,
die Frauen hätten keine Seelen, "`nicht mehr als eine Gans"',


\picinclude{./000-009/p_s008.jpg}

wahr ist, waren sie doch betrübt, daß ich mich in diesem Punkte
nicht unterwerfen und mit ihnen den Priester anhören konnte.
Ich sah ein, daß es ein ander Ding ist, ein wahrer Gläubiger
zu sein, als das worauf es diesen ankommt ...... Warum sollte
ich also diesen anhängen? Weder diesen noch irgendwelchen
Dissentern konnte ich mich anschließen, sondern war allein, ein
Fremdling, und hielt mich einzig an den Herrn Jesus Christus.
Ein andermal hatte ich die Offenbarung, daß Gott, der die
Welt gemacht hat, nicht in Tempeln mit Händen gemacht wohne.
Dies schien mir zuerst ein seltsames Wort, denn sowohl die
Priester als auch das Volk pflegten ihre Tempel oder Kirchen
\textit{Stätten der Ghrsurcht}, \textit{heiliger Boden} und \textit{Tempel Gottes}
zu nennen. Aber der Herr zeigte mir deutlich, daß er nicht
in diesen Tempeln wohne, die von Menschen verordnet und
ausgerichtet waren, sondern in den Herzen der Menschen. Denn
sowohl Stephanus als der Apostel Paulus gaben Zeugnis,
daß er nicht in Tempeln mit Händen gemacht wohne (Art. 7, 48),
nicht einmal in demjenigen, den er einst zu bauen befohlen hatte,
sintemal er ihm ein Ende gemacht hatte, sondern sein Volk sei
sein Tempel, und da wohne er. Solches wurde mir geoffenbart,
während ich durchs Feld zu den Meinigen ging. Als ich kam,
sagten sie mir, Priester Stevens sei dagewesen und habe gesagt,
er sei besorgt um mich, weil ich neuen Lichtern nathgehe. Jch
lächelte bei mir selber, im Gedanken was der Herr mir über ihn
und seinesgleichen geoffenbart hatte. Aber ich sagte meinen Ver-
wandten nichts davon. Denn obgleich sie den Priester durch-
schauten, gingen sie doch, ihn zu hören, und waren betrübt, daß
ich nicht auch ging. Aber ich kam ihnen mit Schriststellen und
zeigte ihnen, daß es eine Salbung gibt im Menschen, die ihn
lehrt, und daß der Herr sein Volk selber lehren will. Ich hatte
auch große Ossenbarungen über das, was in der Mokalypse steht;
wenn ich davon redete, so sagten die "`Frommen"` und die Priester,
sie sei ein versiegeltes Buch, und wollten mich davon abbringen;
aber ich sagte ihnen, Christus könne die Siegel öffnen und sie
sei das, was uns am nächsten angehe; denn die Briefe seien an
die Heiligen früherer Zeiten gerichtet, aber die Apokalypse handle
von den künftigen Dingen.
Jch traf mit Leuten zusammen, welche die Ansicht hatten,
die Frauen hätten keine Seelen, "'nicht mehr als eine Gans"`,


\picinclude{./000-009/p_s009.jpg} 

ein ßiziann der Schmerzen, in den Zeiten, da der Herr sein Werk
in mir anfing.
Während dieser ganzen Zeit hatte ich mich nie mit irgend
jemand zu irgend einer religiösen Richtung Verbunden, sondern
gab mich ganz dem Herrn hin; von aller schlechten Gesellschaft
hatte ich mich losgemacht, hatte Abschied genommen von Vater
und Mutter und allen andern Angehörigen und zog als ein
Fremdling umher, wohin der Herr mein Herz lenkte; ich mietete
ein Zimmer jeweilen in der Stadt, in die ich kam und weilte oft
etwa einen Monat an einem Orte; denn ich wagte nie lange an
einem Orte zu bleiben, da ich fürchtete, als gottseliger Jüngling
sowohl bei den "'Frommen"' als auch bei den Ungläubigen Schaden
zu nehmen, wenn ich viel mit den einen oder den anderen umging;
darum oerhielt ich mich meist wie ein Fremdling; ich suchte hinun-
lische Weisheit, und Erkenntnis kam mir einzig vom Herrn. Jrh
wurde losgelöst von den äußeren Dingen, um mich allein auf
den Herrn zu verlassen. Meine Prüfungen und Trübsale waren
sehr schwer; aber wenn es mir zwischen hinein etwas leichter
wurde, so geriet ich ost in solch eine himmlische Freude, daß ich
wiihnte, in Abrahams Schoß gewesen zu sein. Wie ich das Elend,
in dem ich war, nicht schildern kann, ebensowenig kann ich die
Barmherzigkeit beschreiben, die Gott in diesem Elend an mir getan
hat ....
Nachdem ich die Offenbarung vom Herm empfangen hatte,
"'daß in Oxford oder Cambridge erzogen zu sein noch nicht zum
Dienst des Herrn besähige"`, achtete ich die Priester weniger und
sah mehr auf die Dissenter; ich sah, daß unter diesen einige
Gottseligkeit sei, und viele von ihnen kamen auch später, zu einer
festen Uberzeugung, weil sie Offenbarungen hatten. Aber wie
ich die Priester aufgegeben hatte, so ließ ich auch die Separa-
ristenprediger und solche, welche als die Erfahrensten angesehen
wurden; denn ich sah, daß keiner unter ihnen allen war, derzu meinem
Zustand sprechen konnte. Als alle meine Hoffnungen auf sie und alle
Menschen dahin waren, so daß ich nichts hatte, das mir von außen
hals, und ich nicht wußte, was tun — da! o da hörte ich eine
Stimme: "'es ist Einer, der zu deinem Zustand sprechen kann,
nämlich Jesus Christus Und als ich das hörte, hüpfte mein
Herz vor Freude. Dann zeigte mir der Herr, warum niemand
auf der Welt mir in meinem damaligen Zustand helfen konnte,


 
\backmatter

\chapter{Anhang/Verzeichnis}
\chapter{Bildnachweis}

\begin{description}
 \item[Seite \pageref{bild:gfox}] George Fox, Quelle: \url{http://commons.wikimedia.org/wiki/File:Fox-George-LOC.jpg?uselang=de}, gemeinfrei.
 \item[Seite \pageref{bild:swarthmoor}] Swarthmoor Hall, Quelle: \url{http://en.wikipedia.org/wiki/File:SwarthmoreHall-1.jpg}, public domain,  

\end{description}


% \appendix
\cleardoublepage

% Index soll Stichwortverzeichnis heissen
%\renewcommand{\indexname}{Stichwortverzeichnis}

% Stichwortverzeichnis soll im Inhaltsverzeichnis auftauchen
% \addcontentsline{toc}{section}{Stichwortverzeichnis}
\cleardoublepage

\printindex

  \printindex[bibel]
\cleardoublepage

%   \addcontentsline{toc}{section}{Briefverzeichnis}
  \printindex[brief]
\cleardoublepage

%   \addcontentsline{toc}{section}{Briefverzeichnis}
  \printindex[buch]
\cleardoublepage

%   \addcontentsline{toc}{section}{Ortsverzeichnis}
  \printindex[ort]
\cleardoublepage

%   \addcontentsline{toc}{section}{Personenverzeichnis}
  \printindex[person]
\cleardoublepage


\end{document}

% Notizen:
% Merkzettel: {Fell, Margaret}
% Gbr. heißt Hebr.

